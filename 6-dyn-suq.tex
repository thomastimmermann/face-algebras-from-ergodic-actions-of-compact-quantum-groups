
\section{Dynamical quantum $SU(2)$}

\subsection{Dynamical quantum $SU(2)$ from the Podle\'{s} graph}

Let us now consider the particular case of the Podle\'{s} graph of Example \ref{ExaGraphPod}. We assume in the following that $x\in \lbrack 0,\frac{1}{2}\rbrack$.

Let us denote

\[w_+(k) = w(k,k+1),\]\[w_-(k)  = w(k,k-1) = w_+(k-1)^{-1}.\] 

Let $A_x = A(\Gamma_x)$ be the total $^*$-algebra of the associated partial compact quantum group. Using Theorem \ref{TheoGenRel}, we have the following presentation of $A_x$. Let $B$ be the $^*$-algebra of finite support functions on $\Z\times \Z$, whose Dirac functions we write as $\UnitC{k}{l}$. Let $s_q = \frac{1}{2}(1+\sgn(q))$. Then $A_x$ is generated by a copy of $B$ and elements \[(u_{\epsilon,\nu})_{k,l} = u_{(k,k+\epsilon),(l,l+\nu)}\] for $\epsilon,\nu\in \{-1,1\}=\{-,+\}$ and $k,l\in \Z$ with defining relations \begin{eqnarray*} \sum_{\mu\in \{\pm\}} (u_{\mu,\epsilon})_{m-\mu,k}^* (u_{\mu,\nu})_{m-\mu,l}&=& \delta_{k,l} \delta_{\epsilon,\nu} \UnitC{m}{k+\epsilon},\\ \sum_{\mu\in \{\pm\}} (u_{\epsilon,\mu})_{k,m} (u_{\nu,\mu})_{l,m}^* &=& \delta_{\epsilon,\nu}\delta_{k,l} \UnitC{k}{m}\\ (u_{\epsilon,\nu})_{k,l}^* &=& (\epsilon\nu)^{s_q} \left(\frac{w_{\nu}(l)}{w_{\epsilon}(k)}\right)^{1/2} (u_{-\epsilon,-\nu})_{k+\epsilon,l+\nu}.\end{eqnarray*} The element $(u_{\epsilon,\nu})_{k,l}$ lives inside the component $\Gr{(A_x)}{k}{k+\epsilon}{l}{l+\nu}$.

Consider now $M(A_x)$, the multiplier algebra of $A_x$. We can form in $M(A_x)$ the elements $u_{\epsilon,\nu} = \sum_{k,l} (u_{\epsilon,\nu})_{k,l}$. Then $u=(u_{\epsilon,\nu})$ is a unitary 2$\times$2 matrix. Moreover, \[u_{\epsilon,\nu}^* = (\epsilon\nu)^{s_q} u_{-\epsilon,-\nu}\frac{w_{\nu}^{1/2}(\rho)}{w_{\epsilon}^{1/2}(\lambda)} ,\] where $w_{\pm}^{1/2}(k) = w_{\pm}(k)^{1/2}$ and where for a function $f$ on $\Z$ we write $f(\lambda)(k,l) = f(k)$, $f(\rho)(k,l) = f(l)$. In the following, we then also use the notation $f(\lambda,\rho)$ for a function $f$ on $\Z\times \Z$ interpreted as an element of $M(A)$, and for example $f(\lambda+1,\rho)$ corresponds to the function $(k,l)\mapsto f(k+1,l)$. We then have the following commutation relations between functions on $\Z\times \Z$ and the entries of $u$: \[f(\lambda,\rho)u_{\epsilon,\nu} = u_{\epsilon,\nu}f(\lambda-\epsilon,\rho-\nu).\]

%The relations for the adjoint further imply that \begin{eqnarray*}u_{kl}^{+-} &=&\frac{E(l,l-1)}{E(k,k+1)}(u_{k+1,l-1}^{-+})^*,\\  u_{kl}^{++}&=& \frac{E(l,l+1)}{E(k,k+1)}(u_{k+1,l+1}^{--})^* .\end{eqnarray*}


\begin{eqnarray*} u_{++} &=& \frac{w_+(\rho)^{1/2}}{w_+(\lambda)^{1/2}}u_{--}^*,\\u_{+-}&=& (-1)^{s_q}  \frac{w_-^{1/2}(\rho)}{w_+^{1/2}(\lambda)}u_{-+}^*.  \\ 
\end{eqnarray*}

Let us write \[F(k) = |q|^{-1}w_+(k) =  |q|^{-1}\frac{|q|^{x+k+1}+|q|^{-x-k-1}}{|q|^{x+k}+|q|^{-x-k}},\] and further put\[\alpha = \frac{F^{1/2}(\rho-1)}{F^{1/2}(\lambda-1)}u_{--},\qquad \beta = \frac{1}{F^{1/2}(\lambda-1)}u_{-+}.\] Then the above commutation relations are equivalent to \begin{equation}\label{EqqCom} \alpha \beta = qF(\rho-1)\beta\alpha \qquad \alpha\beta^* = qF(\lambda)\beta^*\alpha\end{equation} \begin{equation}\label{EqDet} \alpha\alpha^* +F(\lambda)\beta^*\beta = 1,\qquad \alpha^*\alpha+q^{-2}F(\rho-1)^{-1}\beta^*\beta = 1,\end{equation}\begin{equation*} F(\rho-1)^{-1}\alpha\alpha^* +\beta\beta^* = F(\lambda-1)^{-1},\qquad  F(\lambda)\alpha^*\alpha +q^{-2}\beta\beta^* = F(\rho),\end{equation*} \begin{equation}\label{EqGrad} f(\lambda)g(\rho)\alpha =
\alpha f(\lambda+1)g(\rho+1),\qquad f(\lambda)g(\rho)\beta = \beta f(\lambda+1)g(\rho-1).\end{equation}%layout might be nicer

These are precisely the commutation relations for the dynamical quantum $SU(2)$-group as in \cite[Definition 2.6]{KoR1}, except that the precise value of $F$ has been changed by a shift in the parameter domain by a complex constant. Clearly, by Theorem \ref{TheoGenRel} the (total) coproduct on $A_x$ also agrees with the one on the dynamical quantum $SU(2)$-group, namely \begin{eqnarray*} \Delta(\alpha) &=& \Delta(1) (\alpha\otimes \alpha - q^{-1}\beta\otimes \beta^*),\\ \Delta(\beta) &=& \Delta(1)(\beta\otimes \alpha^* +\alpha\otimes \beta)\end{eqnarray*} where $\Delta(1) = \sum_{k\in \Z} \rho_k\otimes \lambda_k$.

\subsection{Representation theory of the function algebra of dynamical quantum $SU(2)$}

%Then we study the regular representation. We then show explicitly that the quantum groupoid is amenable in the sense that the universal C$^*$-envelope coincides with the regular representation.

In this section we classify the irreducible $^*$-representations of $A_x$. The parametrisation will hinge on the classification of what we call irreducible $(x,c)$-admissible sets, which we will now discuss. In the following, we fix $0<|q|<1$.

\begin{Def} Let $x\in \lbrack 0,\frac{1}{2}\rbrack$, and let $c\geq 0$. For $\epsilon \in \{\pm\}$, an integer $m\in \Z$ will be called \emph{$(x,c)_{\epsilon}$-adapted} if \begin{equation}\label{EqAd+}c \leq |q|^{2x+m-\epsilon}+|q|^{-2x-m+\epsilon},\end{equation} and \emph{strictly} $(x,c)_{\epsilon}$-adapted if this holds strictly. An integer is called \emph{$(x,c)$-adapted} if it is both $(x,c)_+$ and $(x,c)_-$-adapted. 

A set of integers $Z$ is called an  \emph{$(x,c)$-set} if the following conditions hold: \begin{itemize} 
\item[$\bullet$] $Z$ is not empty.
\item[$\bullet$] $Z$ consists of $(x,c)$-adapted points.
\item[$\bullet$] If $m\in Z$ is strictly $(x,c)_{\epsilon}$-adapted, then $m-2\epsilon$ is in $Z$.
\end{itemize}
An $(x,c)$-set $Z$ is called \emph{even} (resp. \emph{odd}) if $Z\subseteq 2\Z$ (resp. $Z\subseteq 2\Z+1$).

An $(x,c)$-set is called \emph{irreducible} if it can not be written as the union of two disjoint $(x,c)$-sets.
\end{Def}


We aim to classify irreducible $(x,c)$-sets. We start of with the following lemma.

\begin{Lem} Let $c\geq 0$ and $x\in \lbrack 0,\frac{1}{2}\rbrack$.
\begin{enumerate} \item Any irreducible $(x,c)$-set is either even or odd.
\item $Z$ is an irreducible $(x,c)$-set if and only if $-Z-1$ is an irreducible $(\frac{1}{2}-x,c)$-set.
\end{enumerate}
\end{Lem}

\begin{proof} Trivial.
\end{proof}

Hence it suffices to classify even irreducible $(x,c)$-sets. This is achieved in the following proposition.

\begin{Prop}\label{PropClass1D} Let $c\geq 0$ and $x\in \lbrack 0,\frac{1}{2}\rbrack$.  
\begin{itemize}
\item[$\bullet$] If $c< |q|^{2x-1}+|q|^{-2x+1}$, then $2\Z$ is an irreducible $(x,c)$-set.
\item[$\bullet$] If $c\geq |q|^{2x-1}+|q|^{-2x+1}$, write $c=|q|^y+|q|^{-y}$ for some unique $y\geq 1-2x$. 
\begin{itemize}\item[$*$] Assume $x=0$.
\begin{itemize}\item[$\circ$] If $y=1$, then $-2\N_0$, $\{0\}$ and $2\N_0$ are irreducible $(x,c)$-sets.
\item[$\circ$] If $y\in 2\N_0+1$, then $y+1+2\N$ and $-y-1-2\N$ are irreducible $(x,c)$-sets.
\end{itemize}
\item[$*$] Assume $x\in (0,\frac{1}{2})$. Assume $y$ is of the form $y=|2x+M|$, where $M$ is a (uniquely determined) odd integer. 
\begin{itemize}\item[$\circ$] If $M$ is positive, then $2\N+M+1$ is an irreducible $(x,c)$-set. 
\item[$\circ$] If $M$ is negative and $M\neq -1$, then $-2\N+M+1$ is an irreducible $(x,c)$-set. 
\item[$\circ$] If $M=-1$, then $2\N$ and $-2\N_0$ are irreducible $(x,c)$-sets.
\end{itemize}
\item[$*$] Assume $x=\frac{1}{2}$.
\begin{itemize}\item[$\circ$] If $y\in 2\N$, there are two irreducible $(x,c)$-sets $2\N$ and $-2\N_0$.
\end{itemize}
\end{itemize}
\end{itemize}
The above $(x,c)$-sets exhaust all possible even irreducible $(x,c)$-sets.
\end{Prop} 
\begin{proof}
We leave this rather tedious verification task to the reader.
\end{proof} 

Let us now return to the representation theory of the $^*$-algebra $A_x$.

Let $\pi$ be any (necessarily bounded) non-degenerate $^*$-representation of $A_x$ on a Hilbert space $\Hsp_{\pi}$. Then \[\Hsp_{\pi} = \oplus \Hsp^{m}_{n},\qquad \Hsp^{m}_{n} = \pi(\UnitC{m}{n})\Hsp.\] Write \[V_{\pi} =  \textrm{the (non-closed) linear span of all }\Hsp^{m}_{n}.\] Then $\pi(A_x)V_{\pi} = V_{\pi}$. It follows that one can extends $\pi$ to a map \[\pi: M(A) \rightarrow \End(V_{\pi}).\] As the $u_{\epsilon,\nu}$ form a unitary matrix, we can then in fact make sense of the $\pi(u_{\epsilon,\nu})$ as contractions on $\Hsp_{\pi}$. On the other hand, the generators $\alpha,\beta$ and their adjoints give rise to endomorphisms $V_{\pi}\rightarrow V_{\pi}$ which are bounded when restricted to any $\Hsp^{m}_{n}$.

We then have the following easy lemma.

\begin{Lem} There is a one-to-one correspondence, obtained by restriction, between non-degenerate $^*$-representations of $A_x$ on Hilbert spaces, and $\Z^2$-bigraded vector spaces $V$ with finite-dimensional Hilbert spaces as summands and equipped with adjointable maps $\alpha,\beta:V_{\pi}\rightarrow V_{\pi}$ satisfying the commutation relations as in \eqref{EqqCom}, \eqref{EqDet} and \eqref{EqGrad}.\end{Lem}

\begin{Def} The \emph{Casimir} of $A_x$ is defined to be the following element $\Omega\in M(A_x)$, \[\Omega =q^{\lambda-\rho+1}+q^{\rho-\lambda-1} - \sgn(q)^{\lambda-\rho}q^{-1}(|q|^{x+\lambda+1}+|q|^{-x-\lambda-1})(|q|^{x+\rho-1}+|q|^{-x-\rho+1})\beta^*\beta.\] 
\end{Def}

\begin{Lem} The element $\Omega$ is central in $M(A_x)$.
\end{Lem}
\begin{proof}
Straightforward, cf. \cite[Lemma 3.3]{KoR1}.
\end{proof}

\begin{Cor} If $\pi$ is an irreducible $^*$-representation of $A_x$, there exists $c\in \R$ such that $\Omega\xi = c\xi$ for all $\xi \in V_{\pi}$. 
\end{Cor} 
\begin{Cor} If $\pi$ is an irreducible $^*$-representation of $A_x$ on a Hilbert space $\Hsp_{\pi}$, then $\Hsp^m_n$ is at most one-dimensional for each $m,n\in \Z$.
\end{Cor} 
\begin{proof} 
This follows immediately from \eqref{EqqCom} and the fact that for any $\epsilon,\nu$ one has that $u_{\epsilon,\nu}^*u_{\epsilon,\nu}$ can be expressed as an element of the form $g_{\epsilon,\nu}(\lambda,\rho) + h_{\epsilon,\nu}\Omega$ by \eqref{EqDet}.
\end{proof}

Let now $\pi$ be an irreducible $^*$-representation of $A_x$ on a Hilbert space $\Hsp_{\pi}$. From the commutation relations of the generators with the $f(\lambda,\mu)$, it is clear that either all $\Hsp_{m,n}$ with $m-n$ odd are zero, or all $\Hsp_{m,n}$ with $m-n$ even are zero. 

\begin{Def} Let $(\Hsp_{\pi},\pi)$ be an irreducible $^*$-representation of $A_x$. We call $\pi$ even (resp. odd) if all $\Hsp^m_n$ with $m-n$ odd (resp. even) are zero.
\end{Def} 

%We can hence identity $\Hsp$ as a quotient of $l^2(\Z\times \Z)$. Write the images of the standard basis vectors $e_{m,n}$ of $l^2(\Z\times \Z)$ as $f_{m,n}$. Let us write $F = \{(m,n)\mid f_{m,n}\neq 0\}$.

The following lemma follows from a straightforward computation.

\begin{Lem} Inside $M(A_x)$, we have the following identities:
\begin{eqnarray*}
\alpha^*\alpha &=& \frac{|q|^{2x+\lambda+\rho+1}+|q|^{-2x-\lambda-\rho-1}+\sgn(q)^{\lambda-\rho+1}\Omega}{(|q|^{x+\lambda+1}+|q|^{-x-\lambda-1})(|q|^{x+\rho}+|q|^{-x-\rho})}\\
\alpha\alpha^* &=& \frac{|q|^{2x+\lambda+\rho-1}+|q|^{-2x-\lambda-\rho+1}+\sgn(q)^{\lambda-\rho-1}\Omega}{(|q|^{x+\lambda}+|q|^{-x-\lambda})(|q|^{x+\rho-1}+|q|^{-x-\rho+1})}\\
\beta^*\beta &=& |q|\frac{|q|^{\lambda-\rho+1}+|q|^{-\lambda+\rho-1}-\sgn(q)^{\lambda-\rho+1}\Omega}{(|q|^{x+\lambda+1}+|q|^{-x-\lambda-1})(|q|^{x+\rho-1}+|q|^{-x-\rho+1})}\\
\beta\beta^* &=&  |q|\frac{|q|^{\lambda-\rho-1}+|q|^{-\lambda+\rho+1}-\sgn(q)^{\lambda-\rho-1}\Omega}{(|q|^{x+\lambda}+|q|^{-x-\lambda})(|q|^{x+\rho}+|q|^{-x-\rho})}.
\end{eqnarray*}
\end{Lem}

Note that the right hand sides are well-defined because of centrality of $\Omega$.

\begin{Def} Fix $c\in \R$. A couple $(k,l)\in \Z^2$ is called \emph{$(\epsilon,\nu)_c$-adapted} if the following inequality holds: \begin{equation}\label{EqAd}(|q|^{(x+k)+\epsilon\nu(x+l)-\epsilon}+|q|^{-(x+k)-\epsilon\nu(x+l)+\epsilon})+\sgn(q)^{k-l+1}\epsilon\nu c\geq 0.\end{equation} A couple $(k,l)$ is called \emph{strictly} $(\epsilon,\nu)_c$-adapted if this is a strict equality. We call $(k,l)$ \emph{$c$-adapted} if it is $(\epsilon,\nu)_c$-adapted for all $\epsilon,\nu\in \{+,-\}$. Finally, let us call $(k,l)$ \emph{$c$-compatible} if there exists an irreducible representation $\pi$ of $A_x$ with $\pi(\Omega) = c$ and $\Hsp^{k}_{l}\neq \{0\}$.
\end{Def} 

\begin{Def} Fix $c\in \R$. We call a subset $T\subseteq \Z^2$ a \emph{$c$-set} if the following conditions are satisfied: 
\begin{itemize} 
\item[$\bullet$] $T$ is not empty.
\item[$\bullet$] $T$ consists of $c$-adapted points.
\item[$\bullet$] If $(k,l)\in T$ is strictly $(\epsilon,\nu)_c$-adapted, then $(k-\epsilon,l-\nu)$ is in $T$.
\end{itemize}

We say that $T$ \emph{irreducible} if it is not the disjoint union of two $c$-sets.

Writing $\Z^2_{\even} = \{(k,l)\mid k-l \textrm{ even}\}$ and $\Z^2_{\odd} = \Z^2\setminus \Z^2_{\even}$, we call a $c$-set even or odd according to whether it lies in $\Z^2_{\even}$ or $\Z^2_{\odd}$.
\end{Def}

\begin{Prop}\label{PropClassRep} For any even (resp. odd) irreducible representation $\pi$ of $A_x$, the set of $c=\pi(\Omega)$-compatible $(k,l)$ forms an irreducible even (resp. odd) $c$-set. 

Conversely, if $T$ is an irreducible $c$-set, then necessarily $T$ is either an even or odd $c$-set, and there exists a unique irreducible even/odd $^*$-representation $\pi$ of $A_x$ (up to unitary equivalence) having $\pi(\Omega) = c $ and with $T$ as its set of $c$-compatible $(k,l)$.
\end{Prop}

\begin{proof} The first assertion is trivial. For the second assertion, let us put $\Hsp_{\pi} = l^2(T)$ with  \begin{eqnarray*} \alpha e_{k,l} &=&  \left(\frac{|q|^{2x+k+l+1}+|q|^{-2x-k-l-1}+\sgn(q)^{k-l+1}c}{(|q|^{x+k+1}+|q|^{-x-k-1})(|q|^{x+l}+|q|^{-x-l})}\right)^{1/2}e_{k+1,l+1},\\ \beta e_{k,l} &=& \sgn(q)^{k}\left(|q|\frac{|q|^{k-l+1}+|q|^{-k+l-1}-\sgn(q)^{k-l+1}c}{(|q|^{x+k+1}+|q|^{-x-k-1})(|q|^{x+l-1}+|q|^{-x-l+1})}\right)^{1/2} e_{k+1,l-1},\end{eqnarray*} where the right hand side is considered as the zero vector when the scalar factor on the right is zero. 

Then $\pi$ is clearly a well-defined irreducible $^*$-representation with $\pi(\Omega) =c$ and with $T$ as its set of $c$-compatible vectors. As any non-zero $\Hsp^{k}_l$ is cyclic for $\pi$, we deduce that $\pi$ is the unique $^*$-representation up to unitary equivalence which satisfies this condition.
\end{proof}

What remains is to classify irreducible $c$-sets for each $c\in \R$.

\begin{Lem}\label{LemClass2D} A set $T\subseteq \Z^2_{\even}$ is an  irreducible $c$-set if and only if there exists an even irreducible $(x,-\sgn(q)c)$-set $Z_+\subseteq \Z$ and an even irreducible $(0,\sgn(q)c)$-set $Z_-\subseteq \Z$ such that $(k,l)\in T$ if and only if $k-l\in Z_-$ and $k+l\in Z_+$.

A set $T\subseteq \Z^2_{\odd}$ is an irreducible $c$-set if and only if there exists an odd irreducible $(x,-c)$-set $Z_+\subseteq \Z$ and an irreducible $(0,c)$-set $Z_-\subseteq \Z$ such that $(k,l)\in T$ if and only if $k-l\in Z_-$ and $k+l\in Z_+$.
\end{Lem} 

\begin{proof}
Immediate.
\end{proof}

Combining Proposition \ref{PropClassRep} with Proposition \ref{PropClass1D} and Lemma \ref{LemClass2D}, we thus obtain a concrete description of the spectrum of $A_x$. The following pictures illustrate the situation for the cases $x=0$, $x\in (0,\frac{1}{2})$ and $x=\frac{1}{2}$.



%More generally, 

%As a concrete instance of the example of monoidal equivalence, let $\tilde{A}$ be the generalized compact Hopf face algebra obtained from the set $\tilde{I} =I_1\sqcup I_2$ with $I_1= \Z$ and $I_2= \{\bullet\}$ with the $B_{kl} =\emptyset$ and $E(k,l)$ for $k,l\neq \bullet$ as in section ..., with $B_{k,\bullet} = B_{\bullet,k}= \emptyset$, and $B_{\bullet,\bullet} = \{\pm\}$ with $E_{\bullet,\bullet} = \begin{pmatrix} 0 & |q|^{1/2} \\ -\sgn(q)|q|^{-1/2}&0\end{pmatrix}$ (with the basis ordered as $-,+$). Then this will be obtained from the direct sum of the functor from ... and the ordinary forgetful functor from $\Rep(SU_q(2))$ into $\Hilb$. It follows that the components $\tilde{A}(ij)$ can be described by the generators and relations as in ..., but with $F(\lambda)$ and $F(\rho)$ set equal to 1 whenever the corresponding index is $\bullet$.




% Study spectrum fundamental character
% Study dual quantum groupoid
% Make connection with dynamical cocycle
% In case of qgroupoid constructed from identity functor for Rep(SU_q(2)): rep theory of associated Galois object should just be: a single representation (Galois object is type I factor, cutdown of $B(\mathscr{L}^2(SU_q(2)))$). Yes: in general, Galois object is Morita equivalent with algebra of original ergodic action, should also be stressed for Podles spheres





%%% Local Variables: 
%%% mode: latex
%%% TeX-master: "dyn-suq-main"
%%% End: 
