\section{Partial tensor categories}

% Rough version. 

The notion of partial algebra has a nice categorification. % Level of generality could possibly be avoided, but seems nice to incorporate a categorified version which neatly mimicks the local behaviour.  
Recall first that the appropriate (vertical) categorification of a unital $\C$-algebra is a $\C$-linear additive tensor category. From now on, by `category' we will by default mean a $\C$-linear additive category. 

\begin{Def} A \emph{partial tensor category} $\CatCC$ over a set $I_0$ consists of 
\begin{enumerate}[label=(\alph*)]
\item a collection of (small) categories $\mathcal{C}_{ij}$ with $i,j\in I_0$, 
\item $\C$-bilinear functors \[\otimes: \CatC_{ij}\times \CatC_{jk}\rightarrow \CatC_{ik},\] 
\item natural isomorphisms \[ \alpha_{X,Y,Z}: (X\otimes Y)\otimes Z \rightarrow X\otimes (Y\otimes Z),\qquad X \in \CatC_{ij},Y\in \CatC_{jk},Z\in \CatC_{kl},\] 
\item non-zero objects $\Unit_{i} \in \CatC_{ii}$,
\item natural isomorphisms \[\lambda_X^{(i)}:\Unit_i\otimes X \rightarrow X,\qquad \rho_X^{(j)}:X\otimes \Unit_j\rightarrow X, \qquad X\in \CatC_{ij},\]
\end{enumerate}
satisfying the obvious associativity and unit constraints. 
\end{Def}
% This definition should be in a previous section.

\begin{Rem} In true analogy with the partial algebra case, we could let the $\Unit_i$ also be zero objects, but this generalisation will not be needed in the following. 
\end{Rem}

The corresponding total notion is as follows. 

\begin{Def} A \emph{tensor category with local units (indexed by $I_0$)} consists of
\begin{enumerate}[label=(\alph*)]
\item a (small) category $\CatC$, 
\item a $\C$-bilinear functor $\otimes: \CatC\times \CatC \rightarrow \CatC$ with compatible associativity constraint $\alpha$, 
\item\label{FinSup} a collection $\{\Unit_i\}_{i\in I_0}$ of objects such that 
\begin{itemize}\item[$\bullet$] $\Unit_i\otimes \Unit_j \cong 0$ for each $i\neq j$, and
\item[$\bullet$] for each object $X$,  $\Unit_i\otimes X \cong 0 \cong X\otimes \Unit_i$ for all but a finite set of $i$,
\end{itemize}
\item\label{UnCon} natural isomorphisms $\lambda_X:\oplus_i (\Unit_i\otimes X) \rightarrow X$ and $\rho_X:\oplus_i(X\otimes \Unit_i)\rightarrow X$ satisfying the obvious unit conditions. 
\end{enumerate} 
\end{Def}
% Notion of local finiteness for abelian categories: finite-dim hom spaces and each object finite length

% Distinction between monoidal category and tensor category sometimes: latter abelian with irreducible unit. 

Note that the condition \ref{UnCon} makes sense because of the local support condition in \ref{FinSup}. 

\begin{Rem} \begin{enumerate}
\item There is no problem in modifying Maclane's coherence theorem, and we will henceforth assume that our partial tensor categories and tensor categories with local units are strict, just to lighten notation. 
\item One can also see the global tensor category $\CatC$ as an inductive limit of (unital) tensor categories. 
\end{enumerate}
\end{Rem}

\begin{Not} If $(\CatC,\otimes,\{\Unit_i\})$ is a tensor category with local units, and $X\in \CatC$, we define \[X_{ij} = \Unit_i\otimes X \otimes \Unit_j,\] and we denote by \[\eta_{ij}:X_{ij} \rightarrow \oplus_{k,l} \left(\Unit_k \otimes X \otimes \Unit_l\right) \cong X\] the natural inclusion maps. % Necessary to introduce eta-maps?
\end{Not}

%The following lemma is trivial. 

\begin{Lem} Up to the appropriate notion of equivalence, there is a canonical one-to-one correspondence between partial tensor categories and tensor categories with local units. 
\end{Lem}

We will not expand upon the appropriate notion of equivalence, as it can easily be furnished by the reader. % OK?

\begin{proof} Let $(\CatC,\otimes,\{\Unit_i\}_{i\in I_0})$ be a tensor category with local units indexed by $I_0$. Then the $\CatC_{ij} = \{X \in \CatC\mid X_{ij} \underset{\eta_{ij}}{\cong} X\}$, seen as full subcategories of $\CatC$, form a partial tensor category upon restriction of $\otimes$.

Conversely, let $\CatCC$ be a partial tensor category. Then we let $\CatC$ be the category formed by formal finite direct sums $\oplus X_{ij}$ with $X_{ij}\in \CatC_{ij}$, and with \[\Mor(\oplus X_{ij},\oplus Y_{ij}) := \oplus_{ij} \Mor(X_{ij},Y_{ij}).\] The tensor product can be extended to $\CatC$ by putting $X_{ij} \otimes X_{kl} = 0$ when $k\neq j$. The associativity constraints can then be summed to an associativity constraint for $\CatC$. It is evident that the $\Unit_i$ provide local units for $\CatC$. 
\end{proof}

\begin{Rem} Another global viewpoint is to see the collection of $\CatC_{ij}$ as a 2-category with 0-cells indexed by the set $I_0$, the objects of the $C_{ij}$ as 1-cells, and  the morphisms of the $C_{ij}$ as 2-cells. As for partial algebras vs.~ linear categories, we will not emphasize this way of looking at our structures, as this viewpoint is not compatible with the notion of monoidal functor between partial tensor categories.
\end{Rem} 

Continuing the analogy with the algebra case, we define the enveloping \emph{multiplier tensor category} of a tensor category with local units. 
%This notion is important for the formulation of the global notion of morphism between tensor categories with local units. 

\begin{Def} Let $\CatCC$ be a partial tensor category over $I_0$ with total tensor category $\CatC$. The \emph{multiplier tensor category} $M(\CatC)$ of $\CatC$ is defined to be the category consisting of formal sums $\oplus X_{ij}$ which are rcf, and with \[\Mor(\oplus X_{ij},\oplus Y_{ij}) = \left(\prod_j\oplus_i  \Mor(X_{ij},Y_{ij}) \right) \cap \left(\prod_i\oplus_j \Mor(X_{ij},Y_{ij})\right),\] the composition of morphisms being entry-wise (`Hadamard product'). 
\end{Def}

\begin{Rem} Because of the rcf condition on objects, we could in fact have written simply $\Mor(\oplus X_{ij},\oplus Y_{ij}) = \prod_{ij} \Mor(X_{ij},Y_{ij})$. 
\end{Rem} 

The tensor product of $\CatC$ can be extended to $M(\CatC)$ by putting \[\left(\oplus X_{ij}\right)\otimes \left(\oplus Y_{ij}\right) = \oplus_{i,j,k} \left(X_{ij}\otimes Y_{jk}\right),\] and similarly for morphism spaces. This makes sense because of the rcf condition of the objects of $M(\CatC)$. The associativity constraints of the $\CatC_{ij}$ can be summed to an associativity constraint for $M(\CatC)$, while $\Unit := \oplus_{i\in I_0} \Unit_i$ becomes a unit for $M(\CatC)$, rendering $M(\CatC)$ into an ordinary tensor category (with unit object).

\begin{Rem} With some effort, a more intrinsic construction of the multiplier tensor category can be given in terms of couples of endofunctors, in the same vein as the construction of the multiplier algebra of a non-unital algebra.
\end{Rem} 
%Alternatively, one could define $M(\CatC)$ more intrinsically (as in the algebra case) as a collection of couples of functors $L_X,R_X:\CatC\rightarrow \CatC$ together with natural isomorphisms $L_X(Y\otimes Z)\rightarrow L_X(Y)\otimes Z$ and $R_X(Y\otimes Z)\rightarrow Y\otimes R_X(Z)$ satisfying appropriate coherence conditions. % Cf. Leinster, Sketch of proof 1.2.15 in higher operads book

\begin{Exa}\label{ExaVectBiGr} Let $I$ be a set. We can consider the partial tensor category $\CatCC = \{\Vect_{\fin}\}_{i,j\in I}$ where each $\CatC_{ij}$ is a copy of the category of finite-dimensional vector spaces $\Vect_{\fin}$, and with each $\otimes$ the ordinary tensor product. The total category $\CatC$ can then be identified with the category $\Vectif$ of finite-dimensional bi-graded vector spaces with the `balanced' tensor product over $I$. More precisely, the tensor product of $V$ and $W$ is $V\itimes W$ with components \[\Gru{V\itimes W}{k}{m} = \oplus_l \;\Gru{V}{k}{l}\otimes \Gru{W}{l}{m}\subseteq V\otimes W.\] The multiplier category $M(\Vect^{I^2}_{\fin})$ equals $\Vectrcf$, the category of bigraded vector spaces which are rcf.
\end{Exa}

We now formulate the appropriate notion of functor between partial tensor categories. Let us first give an auxiliary definition.

\begin{Def} Let $\CatCC$ be a partial tensor category over $I_0$. If $J_0\subseteq I_0$, we call $\CatDD = \{\CatC_{rs}\}_{r,s\in J_0}$ a \emph{restriction} of $\CatCC$. 
\end{Def} 

\begin{Def} Let $\CatCC$ and $\CatDD$ be partial tensor categories over respective sets $I_0$ and $J_0$, and let \[\phi_0:J_0\twoheadrightarrow I_0,\quad k\mapsto k'\] determine a partition $J_0 = \{J_r\mid r\in I_0\}$ with $k\in J_r \iff \phi_0(k)=r$. 

A \emph{unital morphism} from $\CatCC$ to $\CatDD$ (based on $\phi_0$) consists of $\C$-linear functors \[F_{kl}: \CatC_{k'l'}\rightarrow \CatD_{kl},\] natural monomorphisms \[\iota^{(klm)}_{X,Y}: \GrDA{F(X)}{k}{l} \otimes \GrDA{F(Y)}{l}{m} \hookrightarrow \GrDA{F(X\otimes Y)}{k}{m}, \quad X\in \CatC_{k'l'},Y\in \CatD_{l'm'},\] and isomorphisms \[\mu_{k}:  \Unit_k \cong \GrDA{F(\Unit_{k'})}{k}{k}\] with \begin{enumerate}[label=(\alph*)]
\item (Unitality)  $\GrDA{F(\Unit_{r})}{k}{l}= 0$ if $k\neq l$ in $J_r$,
\item (Local finiteness) For each $r,s\in I_0$ and $X\in \CatC_{rs}$, the application $(k,l)\mapsto \GrDA{F(X)}{k}{l}$ is rcf on $J_k\times J_l$. 
\item (Multiplicativity) For all $X\in \CatC_{k's}$ and $Y\in \CatC_{sm'}$, one has\[\oplus_{l\in J_s} \iota^{(klm)}_{X,Y}: \left(\oplus_{l\in J_s} \GrDA{F(X)}{k}{l} \otimes \GrDA{F(Y)}{l}{m}\right) \cong \GrDA{F(X\otimes Y)}{k}{m}.\]
\item (Coherence) %should this be called coherence? 
The $\iota^{(klm)}$ satisfy the 2-cocycle condition making \[\xymatrix{F(X)\otimes F(Y)\otimes F(Z) \ar[rr]^{\id\otimes \iota^{(lmn)}_{Y,Z}} \ar[d]_{\iota^{(klm)}_{X,Y}\otimes\id}&& F(X)\otimes F(Y\otimes Z)\ar[d]^{\iota^{(kln)}_{X,Y\otimes Z}}\\ F(X\otimes Y)\otimes F(Z) \ar[rr]_{\iota^{(kmn)}_{X\otimes Y,Z}}&& F(X\otimes Y \otimes Z)}\] commute for all $X\in \CatC_{k'l'},Y\in \CatC_{l'm'}, Z\in \CatC_{m'n'}$, and the $\mu_k$ satisfy the commutation relations \[\xymatrix{ \GrDA{F(X)}{k}{l}\otimes \Unit_l \ar[r]^{\!\!\!\!\id\otimes \mu_l} \ar@{=}[d] & \GrDA{F(X)}{k}{l} \otimes \GrDA{F(\Unit_{l'})}{l}{l} \ar[d]^{\iota^{(kll)}_{X, \Unit_{l'}}} \\ \GrDA{F(X)}{k}{l} & \ar@{=}[l] \GrDA{F(X\otimes \Unit_{l'})}{k}{l}} \qquad \xymatrix{  \Unit_k\otimes \GrDA{F(X)}{k}{l}\ar[r]^{\!\!\!\!\mu_k\otimes \id} \ar@{=}[d] & \GrDA{F(\Unit_{k'})}{k}{k} \otimes \GrDA{F(X)}{k}{l} \ar[d]^{\iota^{(kkl)}_{\Unit_{k'},X}} \\ \GrDA{F(X)}{k}{l} & \ar@{=}[l] \GrDA{ F(\Unit_{k'}\otimes X)}{k}{l}} \]
\end{enumerate}

A \emph{morphism} from $\CatCC$ to $\CatDD$ is a unital morphism from $\CatCC$ to a restriction of $\CatDD$. 
\end{Def}

The corresponding global notion (of unital morphism) is as follows.

\begin{Lem} Let $\CatCC$ and $\CatDD$ be partial tensor categories over respective sets $I_0$ and $J_0$. Fix an application \[\phi_0: J_0\twoheadrightarrow I_0\] inducing a partition $\{J_k\mid k\in I_0\}$. Then there is a one-to-one correspondence between unital morphisms $\CatCC\rightarrow \CatDD$ based on $\phi_0$ and functors $F:\CatC \rightarrow M(\CatD)$ with isomorphisms \[\iota_{X,Y}:F(X)\otimes F(Y)\cong F(X\otimes Y),\qquad \mu_r:\oplus_{k\in J_r} \Unit_k \cong F(\Unit_r)\] satisfying the natural coherence conditions. 
\end{Lem} 
%a partition $J_0 = \sqcup_{i\in I_0} J_{0,i}$ and a strong monoidal functor $F: \CatCC\rightarrow M(\CatDD)$ with accompanying coherence isomorphisms $\alpha_{X,Y}: F(X)\otimes F(Y) \rightarrow F(X\otimes Y)$ and $\eta_i: F(\Unit_i) \rightarrow \oplus_{j\in J_{0,i}} \Unit_j$ satisfying the obvious compatibility conditions. % Don't overuse obvious.   

The reader will have no problem in furnishing the definition of \emph{equivalence} of partial tensor categories. There is a closely related but weaker notion of equivalence corresponding to chopping up a partial tensor category into smaller pieces (or, vice versa, gluing certain blocks of a partial tensor category together). Let us formalize this in the following definition.

\begin{Def} Let $\CatCC$ and $\CatDD$ be partial tensor categories. We say $\CatDD$ is a \emph{partitioning} of $\CatCC$ (or $\CatCC$ a \emph{globalisation} of $\CatDD$) if there exists a unital morphism $\CatCC\rightarrow \CatDD$ inducing an equivalence of categories $\CatC\rightarrow \CatD$.
\end{Def}

The partial tensor categories that we will be interested in will be required to have some further structure. 

%We will need partial tensor categories with more properties or structure, such as \emph{semi-simplicity}, \emph{duality} and \emph{C$^*$-structure}.

\begin{Def} A partial tensor category $\CatCC$ is called \emph{semi-simple} if all $\CatC_{ij}$ are semi-simple. 

A partial tensor category is said to have \emph{indecomposable units} if all units $\Unit_i$ are indecomposable. 
\end{Def}

It is easy to see that any semi-simple tensor category can be partitioned into a semi-simple tensor category with indecomposable units.  Hence we will from now on only consider semi-simple partial tensor categories with indecomposable units.


%Let $\CatCC$ be an $I_0$ partial semi-simple tensor category over an index set $I_0$. Then we may decompose the units $\Unit_k$ as direct summands $\Unit_k = \oplus_{r\in J_k} \Unit_r$ for certain finite sets $J_k$, with $\Unit_r$ indecomposable. Since $\CatCC_{kk}$ is a tensor category with unit, we know that $\End(\Unit_k)$ is abelian, hence $\Unit_r\cong \Unit_{r'}$ if and only if $r=r'$ in $J_k$. It is then easy to see that the associated total tensor category $\CatC$ also has a system of local units over the index set $J_0=\sqcup\{J_k\mid k\in I_0\}$ with associated map $\phi_0:J_0 \twoheadrightarrow I_0$. The associated $J_0$-partial tensor category $\CatDD$ is then semi-simple with indecomposable units, and admits a morphism $\CatDD\rightarrow \CatCC$ based on $\phi_0$. It is easy to see from this that there is no loss in generalisation by
 

The following definition introduces the notion of duality for partial tensor categories.

\begin{Def} Let $\CatCC$ be a partial tensor category. 

An object $X\in \CatC_{ij}$ is said to admit a \emph{left dual} if there exists an object $Y={}^*X \in \CatC_{ji}$ and morphisms $\ev_{X}: Y\otimes X \rightarrow \Unit_j$ and $\coev_X: \Unit_i\rightarrow X\otimes Y$ satisfying the obvious snake identities.

We say $\CatCC$ \emph{admits left duality} if each object of each $\CatC_{ij}$ has a left dual.
\end{Def}

Similarly, one defines right duality and (two-sided) duality. As for tensor categories with unit, if $X$ admits a (left or right) dual, it is unique up to isomorphism. 

\begin{Lem}\label{LemMorDua}
\begin{enumerate}
\item Let $\CatCC$ be a partial tensor category. If $X$ has left dual $^*{}X$, then $X$ is a right dual to ${}^*X$. 
\item Let $F$ be a morphism $\CatCC\rightarrow \CatDD$. If $X\in \CatC_{k'l'}$ has a left dual, $F_{lk}({}^*X)$ is a left dual to $F_{kl}(X)$.
 \end{enumerate}
\end{Lem}
\begin{proof}
We can consider the restriction $\CatCC'$ of $\CatCC$ to any two-element set $J_0$ of $I_0$, and apply the  usual arguments to $\CatCC'$ and the global functor $F:\CatC'\rightarrow M(\CatD)$ (using that local units are self-dual and that duality behaves anti-multiplicatively w.r.t.~ tensor products). % OK?
\end{proof}

A final ingredient which will be needed is an analytic structure on our partial tensor categories.

\begin{Def} A \emph{partial semi-simple tensor C$^*$-category} is a partial tensor category $(\CatC_{ij},\otimes)$ such that all $\CatC_{ij}$ are semi-simple C$^*$-categories, such that all functors $\otimes$ are $^*$-functors (in the sense that $(f\otimes g)^* = f^*\otimes g^*$ for morphisms), and such that the associativity and unit constraints are unitary.
\end{Def} 
% Terminology not completely appropriate, i.e. we shouldn't assume semi-simplicity, but we just do this for convenience of language. 



\begin{Rem}
\begin{enumerate}
\item If $\CatCC$ is a partial tensor C$^*$-category, the total category $\CatC$ only has pre-C$^*$-algebras as endomorphism spaces, as the morphisms spaces need not be closed in the C$^*$-norm. On the other hand, $M(\CatC)$ only has $^*$-algebras as endomorphism spaces, since we did not restrict our direct products. 
%However, since we demand the constraints to be unitary, there is no problem to restrict to morphisms in $M(\CatC)$ which have bounded norm in the multiplier tensor category. We will denote the resulting category by $M_b(\CatC)$. % To expand? Necessary to include?
\item The notion of duality for a partial tensor C$^*$-category is the same as in the absence of a C$^*$-structure. However, because of the presence of the $^*$-structure, any left dual is automatically a two-sided dual, and the dual object of $X$ is then simply denoted $\overline{X}$.  
\end{enumerate}
\end{Rem}

\begin{Exa} Let $I$ be a set. Then we can consider the partial semi-simple tensor C$^*$-category $\CatCC = \{\Hilb_{\fin}\}_{I\times I}$ of finite-dimensional Hilbert spaces, with all $\otimes$ the ordinary tensor product. The associated global category is the category $\Hilbif$ of finite-dimensional bi-graded Hilbert spaces. Note that $\CatCC$ has duality, with the dual of a Hilbert space $\Hsp \in \CatC_{ij}$ being the ordinary dual Hilbert space $\Hsp^* \cong \overline{\Hsp}$, but considered in the category $\CatC_{ji}$. 
\end{Exa}

The notion of morphism for partial semi-simple tensor C$^*$-categories has to be adapted in the following way.

\begin{Def} Let $\CatCC$ and $\CatDD$ be partial semi-simple tensor C$^*$-categories over respective sets $I_0$ and $J_0$, and let $\phi_0:J_0\rightarrow I_0$. 
A \emph{morphism} from $\CatCC$ to $\CatDD$ (based on $\phi_0$) is a $\phi_0$-based morphism $(F,\iota,\mu)$ from $\CatCC$ to $\CatDD$ as partial tensor categories, with the added requirement that all $F_{kl}$ are $^*$-functors and the $\iota$- and $\mu$-maps isometric. 
\end{Def} 

