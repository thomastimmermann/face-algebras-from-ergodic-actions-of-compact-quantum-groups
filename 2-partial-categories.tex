\section{Partial tensor categories}

The notion of partial algebra has a nice categorification. Recall first that the appropriate (vertical) categorification of a unital $\C$-algebra is a $\C$-linear additive tensor category. From now on, by `category' we will by default mean a $\C$-linear additive category. 

\begin{Def} A \emph{partial tensor category} $\CatCC$ over a set $\mathscr{I}$ consists of 
\begin{enumerate}[label=(\arabic*)]
\item a collection of (small) categories $\mathcal{C}_{\alpha\beta}$ with $\alpha,\beta\in \mathscr{I}$, 
\item $\C$-bilinear functors \[\otimes: \CatC_{\alpha\beta}\times \CatC_{\beta\gamma}\rightarrow \CatC_{\alpha\gamma},\] 
\item natural isomorphisms \[ a_{X,Y,Z}: (X\otimes Y)\otimes Z \rightarrow X\otimes (Y\otimes Z),\qquad X \in \CatC_{\alpha\beta},Y\in \CatC_{\beta\gamma},Z\in \CatC_{\alpha\beta},\] 
\item non-zero objects $\Unit_{\alpha} \in \CatC_{\alpha\alpha}$,
\item natural isomorphisms \[\lambda_X^{(\alpha)}:\Unit_\alpha\otimes X \rightarrow X,\qquad \rho_X^{(\beta)}:X\otimes \Unit_\beta\rightarrow X, \qquad X\in \CatC_{\alpha\beta},\]
\end{enumerate}
satisfying the obvious associativity and unit constraints. 
\end{Def}

\begin{Rem} In true analogy with the partial algebra case, we could let the $\Unit_\alpha$ also be zero objects, but this generalisation will not be needed in the following. 
\end{Rem}

The corresponding total notion is as follows. 

\begin{Def} A \emph{tensor category with local units (indexed by $\mathscr{I}$)} consists of
\begin{enumerate}[label=(\arabic*)]
\item a (small) category $\CatC$, 
\item a $\C$-bilinear functor $\otimes: \CatC\times \CatC \rightarrow \CatC$ with compatible associativity constraint $a$, 
\item\label{FinSup} a collection $\{\Unit_\alpha\}_{\alpha\in \mathscr{I}}$ of objects such that 
\begin{itemize}\item[$\bullet$] $\Unit_\alpha\otimes \Unit_\beta \cong 0$ for each $\alpha\neq \beta$, and
\item[$\bullet$] for each object $X$,  $\Unit_\alpha\otimes X \cong 0 \cong X\otimes \Unit_\alpha$ for all but a finite set of $\alpha$,
\end{itemize}
\item\label{UnCon} natural isomorphisms $\lambda_X:\oplus_\alpha (\Unit_\alpha\otimes X) \rightarrow X$ and $\rho_X:\oplus_\alpha(X\otimes \Unit_\alpha)\rightarrow X$ satisfying the obvious unit conditions. 
\end{enumerate} 
\end{Def}

Note that the condition \ref{UnCon} makes sense because of the local support condition in \ref{FinSup}. 

\begin{Rem} \begin{enumerate}
\item There is no problem in modifying Maclane's coherence theorem, and we will henceforth assume that our partial tensor categories and tensor categories with local units are strict, just to lighten notation. 
\item One can also see the global tensor category $\CatC$ as an inductive limit of (unital) tensor categories. 
\end{enumerate}
\end{Rem}

\begin{Not} If $(\CatC,\otimes,\{\Unit_\alpha\})$ is a tensor category with local units, and $X\in \CatC$, we define \[X_{\alpha\beta} = \Unit_\alpha\otimes X \otimes \Unit_\beta,\] and we denote by \[\eta_{\alpha\beta}:X_{\alpha\beta} \rightarrow \oplus_{\gamma,\delta} \left(\Unit_\gamma \otimes X \otimes \Unit_\delta\right) \cong X\] the natural inclusion maps. 
\end{Not}

\begin{Lem} Up to equivalence, there is a canonical one-to-one correspondence between partial tensor categories and tensor categories with local units. 
\end{Lem}

The reader can easily cook up the definition of equivalence referred to in this lemma.

\begin{proof} Let $(\CatC,\otimes,\{\Unit_\alpha\}_{\alpha\in \mathscr{I}})$ be a tensor category with local units indexed by $\mathscr{I}$. Then the $\CatC_{\alpha\beta} = \{X \in \CatC\mid X_{\alpha\beta} \underset{\eta_{\alpha\beta}}{\cong} X\}$, seen as full subcategories of $\CatC$, form a partial tensor category upon restriction of $\otimes$.

Conversely, let $\CatCC$ be a partial tensor category. Then we let $\CatC$ be the category formed by formal finite direct sums $\oplus X_{\alpha\beta}$ with $X_{\alpha\beta}\in \CatC_{\alpha\beta}$, and with \[\Mor(\oplus X_{\alpha\beta},\oplus Y_{\alpha\beta}) := \oplus_{\alpha\beta} \Mor(X_{\alpha\beta},Y_{\alpha\beta}).\] The tensor product can be extended to $\CatC$ by putting $X_{\alpha\beta} \otimes X_{\gamma\delta} = 0$ when $\beta\neq \gamma$. The associativity constraints can then be summed to an associativity constraint for $\CatC$. It is evident that the $\Unit_\alpha$ provide local units for $\CatC$. 
\end{proof}

\begin{Rem} Another global viewpoint is to see the collection of $\CatC_{\alpha\beta}$ as a 2-category with 0-cells indexed by the set $\mathscr{I}$, the objects of the $C_{\alpha\beta}$ as 1-cells, and  the morphisms of the $C_{\alpha\beta}$ as 2-cells. As for partial algebras vs.~ linear categories, we will not emphasize this way of looking at our structures, as this viewpoint is not compatible with the notion of monoidal functor between partial tensor categories.
\end{Rem} 

Continuing the analogy with the algebra case, we define the enveloping \emph{multiplier tensor category} of a tensor category with local units. 

\begin{Def} Let $\CatCC$ be a partial tensor category over $\mathscr{I}$ with total tensor category $\CatC$. The \emph{multiplier tensor category} $M(\CatC)$ of $\CatC$ is defined to be the category consisting of formal sums $\oplus_{\alpha,\beta\in \mathscr{I}} X_{\alpha\beta}$ which are rcf, and with \[\Mor(\oplus X_{\alpha\beta},\oplus Y_{\alpha\beta}) = \left(\prod_\beta\oplus_\alpha  \Mor(X_{\alpha\beta},Y_{\alpha\beta}) \right) \cap \left(\prod_\alpha\oplus_\beta \Mor(X_{\alpha\beta},Y_{\alpha\beta})\right),\] the composition of morphisms being entry-wise (`Hadamard product'). 
\end{Def}

\begin{Rem} Because of the rcf condition on objects, we could in fact have written simply $\Mor(\oplus X_{\alpha\beta},\oplus Y_{\alpha\beta}) = \prod_{\alpha\beta} \Mor(X_{\alpha\beta},Y_{\alpha\beta})$. 
\end{Rem} 

The tensor product of $\CatC$ can be extended to $M(\CatC)$ by putting \[\left(\oplus X_{\alpha\beta}\right)\otimes \left(\oplus Y_{\alpha\beta}\right) = \oplus_{\alpha,\beta,\gamma} \left(X_{\alpha\beta}\otimes Y_{\beta\gamma}\right),\] and similarly for morphism spaces. This makes sense because of the rcf condition of the objects of $M(\CatC)$. The associativity constraints of the $\CatC_{\alpha\beta}$ can be summed to an associativity constraint for $M(\CatC)$, while $\Unit := \oplus_{\alpha\in \mathscr{I}} \Unit_\alpha$ becomes a unit for $M(\CatC)$, rendering $M(\CatC)$ into an ordinary tensor category (with unit object).

\begin{Rem} With some effort, a more intrinsic construction of the multiplier tensor category can be given in terms of couples of endofunctors, in the same vein as the construction of the multiplier algebra of a non-unital algebra.
\end{Rem} 

\begin{Exa}\label{ExaVectBiGr} Let $I$ be a set. We can consider the partial tensor category $\CatCC = \{\Vect_{\fin}\}_{i,j\in I}$ where each $\CatC_{ij}$ is a copy of the category of finite-dimensional vector spaces $\Vect_{\fin}$, and with each $\otimes$ the ordinary tensor product. The total category $\CatC$ can then be identified with the category $\Vectif$ of finite-dimensional bi-graded vector spaces with the `balanced' tensor product over $I$. More precisely, the tensor product of $V$ and $W$ is $V\itimes W$ with components \[\Gru{(}{k}{}V\itimes W\Gru{)}{}{m} = \oplus_l \;(\Gru{V}{k}{l}\otimes \Gru{W}{l}{m})\subseteq V\otimes W.\] The multiplier category $M(\Vect^{I^2}_{\fin})$ equals $\Vectrcf$, the category of bigraded vector spaces which are rcfd (i.e.~ finite-dimensional on each row and column).
\end{Exa}

We now formulate the appropriate notion of functor between partial tensor categories. Let us first give an auxiliary definition.

\begin{Def} Let $\CatCC$ be a partial tensor category over $\mathscr{I}$. If $\mathscr{J}\subseteq \mathscr{I}$, we call $\CatDD = \{\CatC_{\alpha\beta}\}_{\alpha,\beta\in \mathscr{J}}$ a \emph{restriction} of $\CatCC$. 
\end{Def} 

\begin{Def} Let $\CatCC$ and $\CatDD$ be partial tensor categories over respective sets $\mathscr{I}$ and $\mathscr{J}$, and let \[\phi:\mathscr{J}\rightarrow \mathscr{I},\quad k \mapsto k'\] determine a decomposition $\mathscr{J} = \{\mathscr{J}_\alpha\mid \alpha\in \mathscr{I}\}$ with $k\in \mathscr{J}_\alpha \iff \phi(k)=\alpha$. 

A \emph{unital morphism} from $\CatCC$ to $\CatDD$ (based on $\phi$) consists of $\C$-linear functors \[F_{kl}: \CatC_{k'l'}\rightarrow \CatD_{kl},\quad X\mapsto F_{kl}(X) = \Gru{F(X)}{k}{l}\] natural monomorphisms \[\iota^{(klm)}_{X,Y}: \GrDA{F(X)}{k}{l} \otimes \GrDA{F(Y)}{l}{m} \hookrightarrow \GrDA{F(X\otimes Y)}{k}{m}, \quad X\in \CatC_{k'l'},Y\in \CatD_{l'm'},\] and isomorphisms \[\mu_{k}:  \Unit_k \cong \GrDA{F(\Unit_{k'})}{k}{k}\] with \begin{enumerate}[label=(\arabic*)]
\item (Unitality)  $\GrDA{F(\Unit_{\alpha})}{k}{l}= 0$ if $k\neq l$ in $\mathscr{J}_\alpha$,
\item (Local finiteness) For each $\alpha,\beta\in \mathscr{I}$ and $X\in \CatC_{\alpha\beta}$, the application $(k,l)\mapsto \GrDA{F(X)}{k}{l}$ is rcf on $\mathscr{J}_{\alpha}\times \mathscr{J}_{\beta}$. 
\item (Multiplicativity) For all $X\in \CatC_{k'\beta}$ and $Y\in \CatC_{\beta m'}$, one has\[\oplus_{l\in \mathscr{J}_\beta} \iota^{(klm)}_{X,Y}: \left(\oplus_{l\in \mathscr{J}_\beta} \GrDA{F(X)}{k}{l} \otimes \GrDA{F(Y)}{l}{m}\right) \cong \GrDA{F(X\otimes Y)}{k}{m}.\]
\item (Coherence) The $\iota^{(klm)}$ satisfy the 2-cocycle condition making \[\xymatrix{F(X)\otimes F(Y)\otimes F(Z) \ar[rr]^{\id\otimes \iota^{(lmn)}_{Y,Z}} \ar[d]_{\iota^{(klm)}_{X,Y}\otimes\id}&& F(X)\otimes F(Y\otimes Z)\ar[d]^{\iota^{(kln)}_{X,Y\otimes Z}}\\ F(X\otimes Y)\otimes F(Z) \ar[rr]_{\iota^{(kmn)}_{X\otimes Y,Z}}&& F(X\otimes Y \otimes Z)}\] commute for all $X\in \CatC_{k'l'},Y\in \CatC_{l'm'}, Z\in \CatC_{m'n'}$, and the $\mu_k$ satisfy the commutation relations \[\xymatrix{ \GrDA{F(X)}{k}{l}\otimes \Unit_l \ar[r]^{\!\!\!\!\id\otimes \mu_l} \ar@{=}[d] & \GrDA{F(X)}{k}{l} \otimes \GrDA{F(\Unit_{l'})}{l}{l} \ar[d]^{\iota^{(kll)}_{X, \Unit_{l'}}} \\ \GrDA{F(X)}{k}{l} & \ar@{=}[l] \GrDA{F(X\otimes \Unit_{l'})}{k}{l}} \qquad \xymatrix{  \Unit_k\otimes \GrDA{F(X)}{k}{l}\ar[r]^{\!\!\!\!\mu_k\otimes \id} \ar@{=}[d] & \GrDA{F(\Unit_{k'})}{k}{k} \otimes \GrDA{F(X)}{k}{l} \ar[d]^{\iota^{(kkl)}_{\Unit_{k'},X}} \\ \GrDA{F(X)}{k}{l} & \ar@{=}[l] \GrDA{ F(\Unit_{k'}\otimes X)}{k}{l}} \]
\end{enumerate}

A \emph{morphism} from $\CatCC$ to $\CatDD$ is a unital morphism from $\CatCC$ to a restriction of $\CatDD$. 
\end{Def}

\begin{Rem} If $J_{\alpha}=\emptyset$, this means that $\Unit_{\alpha}$ is sent to the zero object.
\end{Rem} 

The corresponding global notion (of unital morphism) is as follows.

\begin{Lem} Let $\CatCC$ and $\CatDD$ be partial tensor categories over respective sets $\mathscr{I}$ and $\mathscr{J}$. Fix an application \[\phi: \mathscr{J}\rightarrow \mathscr{I}\] inducing a disjoint decomposition $\{\mathscr{J}_\alpha\mid \alpha\in \mathscr{I}\}$. Then there is a one-to-one correspondence between unital morphisms $\CatCC\rightarrow \CatDD$ based on $\phi$ and functors $F:\CatC \rightarrow M(\CatD)$ with isomorphisms \[\iota_{X,Y}:F(X)\otimes F(Y)\cong F(X\otimes Y),\qquad \mu_\alpha:\oplus_{k\in \mathscr{J}_\alpha} \Unit_k \cong F(\Unit_\alpha)\] satisfying the natural coherence conditions. 
\end{Lem} 

The reader has already furnished for himself the notion of equivalence of partial tensor categories. There is a closely related but weaker notion of equivalence corresponding to chopping up a partial tensor category into smaller pieces (or, vice versa, gluing certain blocks of a partial tensor category together). Let us formalize this in the following definition.

\begin{Def} Let $\CatCC$ and $\CatDD$ be partial tensor categories. We say $\CatDD$ is a \emph{partitioning} of $\CatCC$ (or $\CatCC$ a \emph{globalisation} of $\CatDD$) if there exists a unital morphism $\CatCC\rightarrow \CatDD$ inducing an equivalence of categories $\CatC\rightarrow \CatD$.
\end{Def}

The partial tensor categories that we will be interested in will be required to have some further structure. 

\begin{Def} A partial tensor category $\CatCC$ is called \emph{semi-simple} if all $\CatC_{\alpha\beta}$ are semi-simple. 

A partial tensor category is said to have \emph{indecomposable units} if all units $\Unit_\alpha$ are indecomposable. 
\end{Def}

It is easy to see that any semi-simple tensor category can be partitioned into a semi-simple tensor category with indecomposable units.  Hence we will from now on only consider semi-simple partial tensor categories with indecomposable units.
 

The following definition introduces the notion of duality for partial tensor categories.

\begin{Def} Let $\CatCC$ be a partial tensor category. 

An object $X\in \CatC_{\alpha\beta}$ is said to admit a \emph{left dual} if there exists an object $Y=\hat{X} \in \CatC_{\beta\alpha}$ and morphisms $\ev_{X}: Y\otimes X \rightarrow \Unit_\beta$ and $\coev_X: \Unit_\alpha\rightarrow X\otimes Y$ satisfying the obvious snake identities.

We say $\CatCC$ \emph{admits left duality} if each object of each $\CatC_{\alpha\beta}$ has a left dual.
\end{Def}

Similarly, one defines right duality $X\rightarrow \check{X}$ and (two-sided) duality $X\rightarrow \bar{X}$. As for tensor categories with unit, if $X$ admits a (left or right) dual, it is unique up to isomorphism. 

\begin{Lem}\label{LemMorDua}
\begin{enumerate}
\item Let $\CatCC$ be a partial tensor category. If $X$ has left dual $\hat{X}$, then $X$ is a right dual to $\hat{X}$. 
\item Let $F$ be a morphism $\CatCC\rightarrow \CatDD$ based over $\phi:\mathscr{J}\rightarrow \mathscr{I}$. If $X\in \CatC_{k'l'}$ has a left dual, $F_{lk}(\hat{X})$ is a left dual to $F_{kl}(X)$.
 \end{enumerate}
\end{Lem}
\begin{proof}
We can consider the restriction $\CatCC'$ of $\CatCC$ to any two-element subset $\mathscr{I}'$ of $\mathscr{I}$, and the first property then follows from the usual argument inside the global (unital) tensor category $\CatCC'$. For the second property, consider also the associated restriction $\mathscr{D}'$ to $\phi^{-1}(\mathscr{I})$. We can then again apply the usual arguments to the associated global category $\CatC'$ and global unital morphism $F:\CatC'\rightarrow M(\CatD')$ to see that $F(\hat{X})\cong \widehat{F(X)}$. Using that local units are evidently self-dual and that duality behaves anti-multiplicatively w.r.t.~ tensor products, we can cut down with unit objects on both sides to obtain the statement in the lemma.
\end{proof}

A final ingredient which will be needed is an analytic structure on our partial tensor categories.

\begin{Def} A \emph{partial fusion C$^*$-category} is a partial tensor category $(\CatC_{\alpha\beta},\otimes)$ with duality such that all $\CatC_{\alpha\beta}$ are semi-simple C$^*$-categories, such that all functors $\otimes$ are $^*$-functors (in the sense that $(f\otimes g)^* = f^*\otimes g^*$ for morphisms), and such that the associativity and unit constraints are unitary.
\end{Def} 

\begin{Rem}
\begin{enumerate}
\item If $\CatCC$ is a partial tensor C$^*$-category, the total category $\CatC$ only has pre-C$^*$-algebras as endomorphism spaces, as the morphisms spaces need not be closed in the C$^*$-norm. On the other hand, $M(\CatC)$ only has $^*$-algebras as endomorphism spaces, since we did not restrict our direct products. 
\item The notion of duality for a partial tensor C$^*$-category is the same as in the absence of a C$^*$-structure. However, because of the presence of the $^*$-structure, any left dual is automatically a two-sided dual, and the dual object of $X$ is then simply denoted $\overline{X}$.  
\item We slightly abuse the terminology `fusion', as strictly speaking this would require there to be only a finite set of mutually non-equivalent irreducible objects in each $\CatC_{\alpha\beta}$.
\item In the same vein, the total C$^*$-category with local units associated to a partial fusion C$^*$-category could be called a \emph{multiplier fusion C$^*$-category}.
\end{enumerate}
\end{Rem}

\begin{Exa} Let $I$ be a set. Then we can consider the partial fusion C$^*$-category $\CatCC = \{\Hilb_{\fin}\}_{I\times I}$ of finite-dimensional Hilbert spaces, with all $\otimes$ the ordinary tensor product. The associated global category is the category $\Hilbif$ of finite-dimensional bi-graded Hilbert spaces. The dual of a Hilbert space $\Hsp \in \CatC_{kl}$ is just the ordinary dual Hilbert space $\Hsp^* \cong \overline{\Hsp}$, but considered in the category $\CatC_{lk}$. 
\end{Exa}

The notion of morphism for partial semi-simple tensor C$^*$-categories has to be adapted in the following way.

\begin{Def} Let $\CatCC$ and $\CatDD$ be partial fusion C$^*$-categories over respective sets $\mathscr{I}$ and $\mathscr{J}$, and let $\phi:\mathscr{J}\rightarrow \mathscr{I}$. 
A \emph{morphism} from $\CatCC$ to $\CatDD$ (based on $\phi$) is a $\phi$-based morphism $(F,\iota,\mu)$ from $\CatCC$ to $\CatDD$ as partial tensor categories, with the added requirement that all $F_{kl}$ are $^*$-functors and the $\iota$- and $\mu$-maps isometric. 
\end{Def} 

\begin{Rem} If a morphism of partial fusion C$^*$-categories is based over a \emph{surjective} map $\varphi: \mathscr{J}\rightarrow \mathscr{I}$, then it is automatically faithful. Indeed, by semisimplicity a non-faithful morphism would send some irreducible object to zero. However, by the duality assumption this would mean that some irreducible unit is sent to zero, which is excluded by surjectivity of $\varphi$ and the definition of morphism.
\end{Rem}
