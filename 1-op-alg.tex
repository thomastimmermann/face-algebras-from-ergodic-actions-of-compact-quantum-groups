\section{Partial compact quantum groups on the level of operator algebras}

\subsection{Preliminaries}

We briefly recall some definitions from \cite{DCT1}. 

Let $I$ be a set. An $I^2$-\emph{partial algebra} (with object set $I$), \cite[Definition 1.1]{DCT1}, will be an algebra, that is, a (not necessarily unital) $\C$-algebra together with a family $\UnitC{k}{l}$ of orthogonal idempotents, indexed by elements $(k,l)\in I^2$, for which the natural map \[\oplus_{k,l,m,n} \UnitC{k}{m}A\UnitC{l}{n} \rightarrow A\] is an isomorphism of vector spaces. Note that some of the $\UnitC{k}{l}$ are allowed to be zero. We will in the following write \[\Gr{A}{k}{l}{m}{n} = \UnitC{k}{m}A\UnitC{l}{n}, \quad \lambda_k = \sum_l \UnitC{k}{l},\quad \rho_l = \sum_k \UnitC{k}{l},\]  where the latter two elements are interpreted as elements in the multiplier algebra $M(A)$. 

We call $A$ a \emph{partial bialgebra}, \cite[Definition 1.5]{DCT1} if one is also given a family of linear maps \[\Delta_{rs}: \Gr{A}{k}{l}{m}{n}\rightarrow \Gr{A}{k}{l}{r}{s} \otimes \Gr{A}{r}{s}{m}{n},\] called \emph{partial comultiplications}, and a map \[\epsilon: A \rightarrow \C,\] called the \emph{counit} satisfying the following assumptions.
\begin{itemize}
\item For all $a\in \Gr{A}{k}{l}{m}{n}$ and $p,q,r,q\in I$, one has \[(\Delta_{pq}\otimes \id)\Delta_{rs}(a) = (\id\otimes \Delta_{rs})\Delta_{pq}(a).\]
\item The map $\epsilon$ vanishes on all $\Gr{A}{k}{l}{m}{n}$ with $k\neq m$ or $l\neq n$. 
\item For all $a\in \Gr{A}{k}{l}{m}{n}$, \[(\id\otimes \epsilon)\Delta_{mn}(a) = (\epsilon\otimes \id)\Delta_{kl}(a) = a.\]
\item $\epsilon(\UnitC{k}{k})= 1$ for all $k\in I$. 
\item $\Delta_{ll'}(\UnitC{k}{m}) = \delta_{l,l'} \UnitC{k}{l}\otimes \UnitC{l}{m}$.
%\item For all $a \in \Gr{A}{k}{l}{m}{n}$, one has $\Delta_{rs}(a)^* = \Delta_{sr}(a^*)$ and $\epsilon(a^*) = \overline{\epsilon(a)}$. 
\item $\epsilon(ab) = \epsilon(a)\epsilon(b)$ for all $a\in \Gr{A}{k}{l}{m}{n}$ and $b\in \Gr{A}{l}{p}{n}{q}$,
\item For all $a\in \Gr{A}{k}{l}{m}{n}$ and all $r$, $\Delta_{rs}(a)=0$ for almost all $s$.
\item For all $a\in \Gr{A}{k}{l}{m}{n}$ and all $s$, $\Delta_{rs}(a)=0$ for almost all $r$.
\item For all $a\in \Gr{A}{k}{l}{m}{n}$ and $b\in \Gr{A}{k'}{l'}{m'}{n'}$, one has \[\Delta_{rs}(ab) = \sum_{t} \Delta_{rt}(a)\Delta_{ts}(b).\]
\end{itemize} 

In the last multiplicative identity, we have used the finite support condition on the maps $s\mapsto \Delta_{rs}(a)$ to make sense of the a priori infinite sum as a finite sum. Note that $\epsilon$ is \emph{not} required to be a homomorphism on the whole of $A$.

The expression $\Delta(a) = \sum_{rs}\Delta_{rs}(a)$ can be made sense of inside the multiplier algebra $M(A\otimes A)$, and we obtain in this way a (non-degenerate) homomorphism \[\Delta: A\rightarrow M(A\otimes A),\] called the \emph{(total) comultiplication}, satisfying the ordinary comultiplication formula. There is then a unique extension of $\Delta$ to $M(A)$ such that $\Delta(1) =\sum_k \rho_k\otimes \lambda_k$. We will use the Sweedler-like notations \[\Delta(a) = a_{(1)}\otimes a_{(2)}.\] Note that because of the finiteness condition, expressions such as $\Delta(a)(1\otimes b)$ lie in $A\otimes A$.

%We will in the following use the linear maps $\Pi^L,\Pi^R,\overline{\Pi}^L$ and $\overline{\Pi}^R$ from $A$ to $M%(A)$, determined by the following formulas for $a\in \Gr{A}{k}{l}{m}{n}$: \[\Pi^L(a) = \epsilon(a)\lambda_k,\quad \overline{\Pi}^L(a) = \epsilon(a)\lambda_l,\quad \overline{\Pi}^R(a) = \epsilon(a)\rho_m,\quad \overline{\Pi}^R(a) = \epsilon(a)\rho_n.\]

A \emph{partial Hopf algebra} \cite[Definition 1.7]{DCT1} is a partial bialgebra $(A,\Delta)$ for which there exists an invertible, antimultiplicative map $S: A\rightarrow A$ with \[S(a_{(1)})a_{(2)} = \sum_{k\in I} \epsilon(\lambda_ka)\lambda_k,\qquad a_{(1)}S(a_{(2)}) = \sum_n\epsilon(a\rho_n)\rho_n,\] where the identities have to be interpreted as identities of multipliers. The map $S$ is then referred to as the \emph{antipode}. 

Following \cite[Definition 1.8]{DCT1}, one can show that the relation $k\sim l$ if $\UnitC{k}{l}\neq0$ defines an equivalence relation on $I$. The quotient set $\mathscr{I} = I/\sim$ will be called the \emph{hyperobject set} of $(A,\Delta)$.

Finally, an invariant integral on $(A,\Delta)$ \cite[Definition 1.12]{DCT1} will consist of a functional $\phi: A\rightarrow \C$ such that $\phi(\UnitC{k}{l})=1$ for all $k,l$ with $\UnitC{k}{l}\neq 0$, and, for all $a\in A$, \[(\id\otimes \phi)\Delta(a) = \sum_{k} \phi(\lambda_k a)\lambda_k,\qquad (\phi\otimes \id)\Delta(a) = \sum_m \phi(\rho_m a)\rho_m,\] again as identities in the multiplier algebra of $A$. Such an invariant integral is unique if it exists.

We can now give the definition of an $I$-partial compact quantum group, obtained by putting also a compatible $^*$-structure in the mix.

\begin{Def}[\cite{DCT1}, Definition 1.17] An \emph{$I$-partial compact quantum group} $\mathscr{G}$ consists of a partial Hopf algebra $P(\mathscr{G})$ with invariant integral $\phi$, endowed with an antimultiplicative, antilinear involution $*:A\rightarrow A$ satisfying the following conditions.
\begin{itemize}
\item All $\UnitC{k}{l}$ are self-adjoint.
\item $\Delta$ is a $^*$-homomorphism.
\item $\epsilon(a^*) = \overline{\epsilon(a)}$ for all $a\in A$. 
\item $\phi(a^*) = \overline{\phi(a)}$ and $\phi(a^*a)\geq 0$ for all $a\in A$.
\end{itemize}
\end{Def} 

We refer to $P(\mathscr{G})$ as the associated (function) algebra of $\mathscr{G}$.

Let us also briefly recall the notion of \emph{unitary representation} of a partial compact quantum group. Let $\Hsp$ be an $I^2$-graded Hilbert space, whose components we write $\GrDA{\Hsp}{k}{l}$. We call $\Hsp$ \emph{row- and column finite dimensional} (rcfd) if all $\GrDA{\Hsp}{k}{l}$ are finite dimensional and if, for any fixed $l$ (resp.~ fixed $k$), there are only finitely many $k$ (resp.~ $l$) for which $\GrDA{\Hsp}{k}{l}$ is not zero.  
%We further write $p_{kl}$ for the projection on $\GrDA{\Hsp}{k}{l}$, and $L_k =\sum_{l} p_{kl}$, $R_l = \sum_k p_{kl}$ (as strongly convergent sums).

\begin{Def}[\cite{DCT1}, Definition 2.1 and Definition 2.18]\label{DefUniCorep} Let $\mathscr{G}$ be a partial compact quantum group. A \emph{unitary representation $\mathscr{X}$ of $\mathscr{G}$} (or unitary corepresentation of $P(\mathscr{G})$) $\mathscr{X}$ will consist of an rcfd Hilbert space $\Hsp$ and a collection of elements \[\Gr{X}{k}{l}{m}{n}\in \Gr{A}{k}{l}{m}{n}\otimes B(\GrDA{\Hsp}{m}{n},\GrDA{\Hsp}{k}{l})\] such that \begin{eqnarray*} (\Delta_{rs}\otimes \id)\Gr{X}{k}{l}{m}{n} &=& \left(\Gr{X}{k}{l}{r}{s}\right)_{13}\left(\Gr{X}{r}{s}{m}{n}\right)_{23},\\ (\epsilon\otimes \id)\Gr{X}{k}{l}{k}{l} &=& \id_{\GrDA{\Hsp}{k}{l}}\end{eqnarray*} and \begin{eqnarray*} \sum_{k} \left(\Gr{X}{k}{l'}{m}{n'}\right)^*\Gr{X}{k}{l}{m}{n} &=& \delta_{l,l'}  \delta_{n,n'} \UnitC{l}{n}\otimes \id_{\GrDA{\Hsp}{m}{n}},\\ \sum_{n} \Gr{X}{k}{l}{m}{n}\left(\Gr{X}{k'}{l}{m'}{n}\right)^* &=& \delta_{k,k'}\delta_{m,m'} \UnitC{k}{m}\otimes \id_{\GrDA{\Hsp}{k}{l}}.\end{eqnarray*}
\end{Def} 

Note that the above sums are finite because of the rcfd condition.

It is easy to see that any collection of maps $\Gr{X}{k}{l}{m}{n}$ as above gives rise to a partial isometry \[X = \sum_{k,l,m,n}\Gr{X}{k}{l}{m}{n}\in M(P(\mathscr{G})\otimes B_{00}(\Hsp)),\] where $B_{00}(\Hsp)$ is the $^*$-algebra of finite rank operators on $\Hsp$. We will then also write $\Hsp = \Hsp_{\mathscr{X}}$.

The left (resp.~ right) hyperobject support of a unitary representation will consist of all equivalence classes of $k$ (resp.~ all $l$) such that there exists $l$ (resp.~ $k$) with $\GrDA{\Hsp}{k}{l}\neq 0$. It can be shown that any unitary representation is a direct sum of irreducible representations \cite[Corollary 2.5]{DCT1}, and that the latter have left and right hypersupport consisting of each a single hyperobject. In general, we will say that a unitary representation has \emph{finite hyperobject support} if its left and right hyperobject supports are finite sets. This implies that the unitary representation will be a \emph{finite} direct sum of irreducibles. 

One has the following lemma which will be crucial in what follows. 

\begin{Lem}[\cite{DCT1}, Proposition 2.10 and Proposition 2.20]\label{LemSpan} The linear spaces \[P(\mathscr{G})(\mathscr{X}) = \{(\id\otimes \omega)(X)\mid \omega \in B(\Hsp_{\mathscr{X}})_*\},\] defined for irreducible unitary representations $\mathscr{X}$, are linearly independent when the unitary representations are not equivalent, and moreover their joint linear span over all irreducible unitary representations equals $P(\mathscr{G})$.
\end{Lem}

Further references to \cite{DCT1} will be given whenever needed. 

For the rest of this section, let $\mathscr{G}$ be a fixed $I$-partial compact quantum group. We construct
completions of the underlying $*$-algebra $P(\mathscr{G})$ in the form
of a universal $C^{*}$-algebra $\CuG$, a reduced $C^{*}$-algebra
$\CrG$ and a von Neumann algebra $\LGinf$, the last two acting on a
Hilbert space $\LGtwo$ associated to the invariant integral of
$\mathscr{G}$.  We then lift the comultiplication and the invariant
integral to the level of
operator algebras and show that $\LGinf$ becomes a measured quantum
groupoid in the sense of Lesieur \cite{Les1} and Enock \cite{Eno2}.

\subsection{Universal C$^*$-algebra of a partial compact quantum group}

We first show that the $*$-algebra $P(\mathscr{G})$ has an enveloping $C^{*}$-algebra $\CuG$. 
%Recall the notion of an rcfd (row and column finite-dimensional) Hilbert space $\Hsp$ over $I$, meaning an $I^2$-graded Hilbert space $\Hsp = \underset{k,l}{\oplus} \,\GrDA{\Hsp}{k}{l}$ with each $\underset{k}{\oplus}\, \GrDA{\Hsp}{k}{l}$ and $\underset{l}{\oplus}\, \GrDA{\Hsp}{k}{l}$ finite-dimensional. Then a \emph{unitary representation $\mathscr{X}$ of $\mathscr{G}$ on $\Hsp$} consists of a family of elements \[\Gr{X}{k}{l}{m}{n} \in \Gr{P(\mathscr{G})}{k}{l}{m}{n}\otimes B(\GrDA{\Hsp}{m}{n},\GrDA{\Hsp}{k}{l})\] satisfying natural corepresentation and unitarity identities, see \cite[Section 2.1 and Section 2.4]{DCT1}.

\begin{Lem}\label{LemUniBound}
 Let $H$ be a pre-Hilbert space with completion $\Hsp$. Let $\pi: P(\mathscr{G})\rightarrow \End(H)$ be a $*$-homomorphism into the $*$-algebra of adjointable endomorphisms of $H$. Then $\pi$ extends uniquely to a $*$-homomorphism $\pi: P(\mathscr{G})\rightarrow B(\Hsp)$, and for each $a\in P(\mathscr{G})$ there exists $M_a>0$, independent of $\pi$, such that $\|\pi(a)\|\leq M_a$.
\end{Lem} 

\begin{proof}
By Lemma \ref{LemSpan}, it suffices to prove that $\pi(a)$ is $\pi$-uniformly bounded for $a\in P(\mathscr{G})$ of the form \[a=(\id \otimes \omega_{\xi,\eta})(\Gr{X}{k}{l}{m}{n}),\]
where $\mathscr{X}$ is a unitary representation of
$\mathscr{G}$ on some rcfd $I^{2}$-graded Hilbert space
$\mathcal{K}$ and $\xi\in \Gru{\mathcal{K}}{k}{l}$, $\eta\in
\Gru{\mathcal{K}}{m}{n}$.  By Definition \ref{DefUniCorep}, we deduce
  \begin{equation}\label{EqUnitary}
    \sum_{p}(\Gr{X}{p}{l}{m}{n})^{*} \Gr{X}{p}{l}{m}{n}  = \UnitC{l}{n}
    \otimes \id_{\Gru{\mathcal{K}}{m}{n}}.
  \end{equation}

As $\pi(\UnitC{l}{n})$ is a self-adjoint projection, it extends uniquely to an element in $B(\Hsp)$. From  $\eqref{EqUnitary}$, it then follows that each $(\pi\otimes \id)(\Gr{X}{p}{l}{m}{n})$ can be extended to a contraction in $B(\Hsp\otimes \GrDA{\mathcal{K}}{m}{n},\Hsp\otimes \GrDA{\mathcal{K}}{p}{l})$. We hence conclude that
 \[\|\pi(a)\| \leq \| \xi\| \|\eta\| \|(\pi \otimes \id)(\Gr{X}{k}{l}{m}{n})\| \leq
    \|\xi\|\|\eta\| .\] 

\end{proof} 


\begin{Prop}
Let $\mathscr{G}$ be an $I$-partial compact quantum group with underlying
$*$-algebra $P(\mathscr{G})$. Then
  there exist a $C^{*}$-algebra $C^{u}_{0}(\mathscr{G})$ and a
  $*$-homomorphism \[\pi_{u} \colon P(\mathscr{G}) \to
  C^{u}_{0}(\mathscr{G}) \] such that any $*$-homomorphism of
  $P(\mathscr{G})$ into a $C^{*}$-algebra factorizes through
  $\pi_{u}$.
\end{Prop}
\begin{proof}
It suffices to show that for each $a \in P(\mathscr{G})$,
\begin{align*} 
  |a|_{u}&:= \sup \{ \|\pi(a)\| : \pi \text{ is a $*$-homomorphism from } P(\mathscr{G})
  \text{ into some $C^{*}$-algebra } B\}
\end{align*}
is finite, as then $|\cdot |_{u}$ defines a $C^{*}$-seminorm on
$P(\mathscr{G})$ and we can take $C^{u}_{0}(G)$ to be the associated
separated completion. This follows immediately from Lemma \ref{LemUniBound}.
\end{proof}

%The construction of the reduced $C^{*}$-algebra $C^{r}_{0}(\mathscr{G})$ 
% Changed from maximal to minimal tensor product

We will show later that the $*$-homomorphism $\pi_{u}
\colon P(\mathscr{G}) \to \CuG$ is in fact injective. For now, we show that the comultiplication of $P(\mathscr{G})$ lifts to a $*$-homomorphism  of
$C^{u}_{0}(\mathscr{G})$ into the multiplier C$^*$-algebra of the minimal tensor product
$C^{u}_{0}(\mathscr{G}) \otimes
C^{u}_{0}(\mathscr{G})$.

% First note that for any $k\in I$, the sums $\sum_l \pi_u(\UnitC{k}{l}$ and $\sum_l \pi_u(\UnitC{l}{k})$ converge strictly inside $C^u_0(\mathscr{G})$, as they are sums of mutually orthogonal projections in $C^{u}_{0}(\mathscr{G})$ that converge strictly on a dense subalgebra. We will denote the resulting elements by $\pi_u(\lambda_k)$ and $\pi_u(\rho_k$ 


% The universal property of $\CuG$ implies that the modular automorphism
% group $\sigma$ and the scaling group $\tau$ for real parameters and
% the unitary antipode $R$ of $\mathscr{G}$ introduced after Corollary
% \ref{cor:rep-characters} lift to one-parameter groups
% $\tau^{u},\sigma^{u}$ and a $*$-anti-automorphism $R_{u}$ of $\CuG$,
% that is, $\tau^{u}_{t} \circ \pi_{u}= \pi_{u} \circ \tau_{t}$,
% $\sigma^{u}_{t} \circ \pi_{u} = \pi_{u} \circ \sigma_{t}$, $R_{u}
% \circ \pi_{u} = \pi_{u} \circ R$. Corollary \ref{cor:rep-characters}
% 5.\ and A.1 in \cite{} [Takesaki:2] imply that elements in
% $\pi_{u}(A)$ are analytic for $\tau^{u}$ and $\sigma^{u}$; in
% particular, $\tau^{u}$ and $\sigma^{u}$ are strongly continuous.
\begin{Prop}
Let $\mathscr{G}$ be a partial compact quantum group with underlying
$*$-algebra $P(\mathscr{G})$. Then  there exists a unique $*$-homomorphism $\Delta^{u}\colon
  C^{u}_{0}(\mathscr{G}) \to M(C^{u}_{0}(\mathscr{G}) \otimes
  C^{u}_{0}(\mathscr{G}))$ such that 
  \begin{equation}\label{EqParts}
    (\pi_u(\UnitC{k}{p}) \otimes
    \pi_u(\UnitC{r}{m}))\Delta^{u}(\pi_{u}(a))(\pi_u(\UnitC{l}{q})\otimes \pi_u(\UnitC{s}{n})) =\delta_{p,r}\delta_{s,q}
    (\pi_{u}  \otimes \pi_{u})(\Delta_{pq}(a))
  \end{equation}
  for all $a \in \Gr{P(\mathscr{G})}{k}{l}{m}{n}$ and $p,q,r,s\in I$.
\end{Prop}
\begin{proof}

Let $\Hsp_u^{(2)}$ carry a (non-degenerate) universal $^*$-representation of $C_0^u(\mathscr{G})\otimes C_0^u(\mathscr{G})$, and define $H_u^{(2)}$ as the algebraic sum of all $(\pi_u(\UnitC{k}{m}) \otimes \pi_u(\UnitC{l}{n}))\Hsp_u^{(2)}$. Then $H_u^{(2)}$ is a pre-Hilbert space with $\Hsp_u^{(2)}$ as its closure, and as $P(\mathscr{G})\otimes P(\mathscr{G})$ has local units, it is clear that the representation $\pi_u\otimes \pi_u$ of $P(\mathscr{G})\otimes P(\mathscr{G})$ extends to a $ *$-representation of $M(P(\mathscr{G})\otimes P(\mathscr{G}))$ on $H_u^{(2)}$ by adjointable endomorphisms, which we will denote by the same symbol.

By Lemma \ref{LemUniBound}, we can extend $(\pi_u\otimes \pi_u)\Delta$ to a $^*$-representation $\Delta^u$ of $C_0^u(\mathscr{G})$ on $\Hsp_u^{(2)}$. It clearly satisfies $\eqref{EqParts}$. Moreover, as $(\pi_u\otimes \pi_u)\Delta(a)$ for $a\in P(\mathscr{G})$ defines a multiplier of $(\pi_u\otimes \pi_u)(P(\mathscr{G})\otimes P(\mathscr{G}))$, we conclude that the range of $\Delta^u$ is in $M(C_0^u(\mathscr{G})\otimes C_0^u(\mathscr{G}))$.
\end{proof}

Since $\underset{k,l,m,n}{\sum} \pi_u(\UnitC{k}{m})\otimes \pi_u(\UnitC{l}{n})$ converges strictly to the unit in $M(C_0^u(\mathscr{G})\otimes C_0^u(\mathscr{G}))$, we have in fact for $a\in P(\mathscr{G})$ that, with respect to the strict topology, \[\Delta^u(a) = \sum_{p,q} (\pi_{u} \otimes \pi_{u})(\Delta_{pq}(a)).\]

The existence of the above coproduct can be used to construct tensor products of C$^*$-representations of $\CuG$. For $(\Hsp,\pi)$ a $^*$-representation of $\CuG$, we write $p_{\pi}\in B(\Hsp)$ for the projection onto the closure of $\pi(\CuG)\Hsp$. We write $\GrLA{\Hsp}{k}{l} = \pi(\UnitC{k}{l})\Hsp$, which gives a direct sum decomposition of $p_{\pi}\Hsp$.  When $(\mathcal{K},\pi')$ is another $^*$-representation, we will write $\pi \boxtimes \pi'$ for the $^*$-representation of $\CuG$ on $\mathcal{H}\otimes \mathcal{K}$ obtained as \[(\pi \boxtimes \pi')(a)(\xi\otimes \eta) = (\pi\otimes \pi')(\Delta^u(a))(p_{\pi}\xi\otimes p_{\pi'}\eta).\] When $\pi$ and $\pi'$ are non-degenerate, the restriction of $\pi\boxtimes \pi'$ to \[\Hsp\itimes \mathcal{K} := p_{\pi \boxtimes \pi'}(\Hsp\otimes \mathcal{K}) = (\pi \boxtimes \pi')(\Delta^u(1))(\Hsp\otimes \mathcal{K})= \oplus_l ({}_l\Hsp\otimes {}^l\mathcal{K})\] will be written as $\pi\iboxtimes \pi'$. Note that, by definition, $\pi\iboxtimes \pi'$ is a non-degenerate $^*$-representation, i.e.~ $p_{\pi\iboxtimes \pi'} = 1$. 

\begin{Def}\label{DefTenProd}  Let $\mathscr{G}$ be a partial compact quantum group, and let $(\Hsp,\pi)$ and $(\mathcal{K},\pi')$ be two non-degenerate $^*$-representations of $\CuG$. Then the non-degenerate $^*$-representation $(\Hsp\itimes \mathcal{K},\pi\iboxtimes \pi')$ will be called the \emph{tensor product representation} of $\pi$ and $\pi'$. 
\end{Def} 
%However, it is possible that this is a representation on the zero Hilbert space $\{0\}$!
%We further write $H=H_{\pi}$ for the linear span of all $\GrLA{\Hsp}{k}{l}$.

\subsection{Reduced C$^*$-algebra and von Neumann algebra of a partial compact quantum group}

We now turn towards the reduced $C^{*}$-algebra $\CrG$ and the von Neumann algebra $\LGinf$. As usual, these will be obtained from a GNS-construction with respect to the invariant integral $\phi$ of $\mathscr{G}$. 

Recall that $\phi$ is faithful by the remark following \cite[Corollary 2.16]{DCT1}.  We define an inner product on $P(\mathscr{G})$ by
\begin{align*}
  \langle a|b\rangle :=\phi(a^{*}b).
\end{align*}
We denote by $L^{2}(\mathscr{G})$ the associated completion, and by
$\Lambda \colon P(\mathscr{G}) \to \LGtwo$ the natural embedding.

\begin{Prop} \label{prop:gns} Let $\mathscr{G}$ be a partial compact
  quantum group with underlying total $*$-algebra $P(\mathscr{G})$ and
  associated Hilbert space $\LGtwo$. Then there exists a unique
  $*$-homomorphism \[\pi_{\red}\colon P(\mathscr{G}) \to {\cal B}(\LGtwo)\]
  such that $\pi_{\red}(a)\Lambda(b)=\Lambda(ab)$ for all $a,b\in
  P(\mathscr{G})$, and $\pi_{\red}$ is faithful.
\end{Prop}
\begin{proof} The existence of $\pi_{\red}$ follows immediately from Lemma \ref{LemUniBound}, and the faithfulness of $\pi_{\red}$ follows from the faithfulness of $\phi$. 
  %Let $a,c \in P(\mathscr{G})$. Then the formula $x \mapsto \langle
%\Lambda(c) | x\Lambda(a)\rangle$ defines a bounded linear functional
  %$\omega_{\Lambda(c),\Lambda(a)}$ on ${\cal B}(\LGtwo)$ and a
  %straightforward computation shows that
  %\begin{align} \label{eq:vn-slice}
  %  (\omega_{\Lambda(c),\Lambda(a)}\otimes \id)(V)\Lambda(b) =
%    \Lambda(\varphi(c^*a_{(1)})a_{(2)}b)
  %\end{align}
 % for all $b\in P(\mathscr{G})$. Hence, left multiplication by
 %$\varphi(c^*a_{(1)})a_{(2)}$ extends to a bounded linear operator on
 % $\LGtwo$.  But elements of this form span $P(\mathscr{G})$ because
 % $\phi$ is normalized and $(P(\mathscr{G})\otimes
 % 1)\Delta(P(\mathscr{G})) = (P(\mathscr{G})\otimes
 % P(\mathscr{G}))\Delta(1)$ by \cite[Proposition 1.9]{DCT1} and
 % \cite[Theorem 6.8]{Boh1}.
\end{proof}
Since $\pi_{\red}$ factorizes through $\pi_{u}$, we can conclude:
\begin{Cor}
  Let $\mathscr{G}$ be a partial compact quantum group with underlying
  total $*$-algebra $P(\mathscr{G})$. Then the
universal $*$-homomorphism $\pi_{u} \colon P(\mathscr{G}) \to\CuG$ is injective.
\end{Cor}

We call $(\LGtwo,\Lambda,\pi_{\red})$ the \emph{GNS-construction associated
  to $\mathscr{G}$}. Whenever convenient, we drop the symbol $\pi_{\red}$. Denote by
\begin{align}
  \CrG &\subseteq {\cal B}(\LGtwo) &&\text{and} & \LGinf &\subseteq {\cal B}(\LGtwo)
\end{align}
respectively the $C^{*}$-algebra and the von Neumann algebra generated by the image
of $\pi_{\red}$. We thus obtain a sequence
\begin{align*}
P(\mathscr{G}) \hookrightarrow \CuG \twoheadrightarrow
  \CrG\hookrightarrow
\LGinf \hookrightarrow {\cal B}(\LGtwo)
\end{align*}
of $*$-algebras.

Note that for every non-degenerate $*$-representation of $P(\mathscr{G})$ on a
Hilbert space $\mathcal{H}$, the sums
  \begin{align*}
    \pi(\lambda_{k}) := \sum_{l'} \pi(\UnitC{k}{l'}) \quad \text{and} \quad
\pi(\rho_{l}) := \sum_{k'} \pi(\UnitC{k'}{l})
  \end{align*}
  converge strictly in $\mathcal{B}(\mathcal{H})$. 
  In the case where $\pi=\pi_{\red}$, we slightly abuse notation and
  simply write $\lambda_{k}$ and $\rho_{l}$ for
  $\pi_{\red}(\lambda_{k})$ and $\pi_{\red}(\rho_{l})$,
  respectively. Since the  Hilbert space $\LGtwo$ is the orthogonal direct sum of the
subspaces $\Lambda(\Gr{P(\mathscr{G})}{k}{l}{m}{n})$, 
there also exist selfadjoint projections
$\lambda_{k}^{\op},\rho_{k}^{\op}\in {\cal
  B}(\LGtwo)$ such that  for all $k\in I$ and $a\in P(\mathscr{G})$,
\begin{align*}
  \lambda^{\op}_{k}\Lambda(a) &= \Lambda(a\lambda_{k}), &
  \rho_{k}^{\op}\Lambda(a) &= \Lambda(a\rho_{k}).
\end{align*}



\begin{Lem} \label{lemma:partial-isometry}
  Let  $\pi$ be a non-degenerate $*$-representation of
  $P(\mathscr{G})$ on a Hilbert space $\mathcal{H}$. Then there exists
  a unique partial isometry $V_{\pi}$ on $\LGtwo \otimes \mathcal{H}$
  such that
  \begin{align*}
    V_{\pi}(\Lambda(a) \otimes \pi(b)\xi) = \Lambda(a_{(1)}) \otimes
    \pi(a_{(2)}b) \xi
  \end{align*}
  for all $a,b\in P(\mathscr{G})$ and $\xi \in \mathcal{H}$. Its
 domain and range projection are given by
\begin{align*}
  V_{\pi}^{*}V_{\pi} &= \sum_{l} \rho_{l}^{\op} \otimes \pi(\rho_{l}), &
  V_{\pi}V_{\pi}^{*} &= \sum_{l}  \rho_{l} \otimes \pi(\lambda_{l}),
\end{align*}
  the sums converging in the strong operator topology. The adjoint is determined by the formula
\begin{align*}
V_{\pi}^{*}(\Lambda(a) \otimes \pi(b)\xi) &= \Lambda(a_{(1)}) \otimes
\pi(S(a_{(2)})b)\xi
\end{align*}
for $a,b\in P(\mathscr{G})$.
\end{Lem}
\begin{proof}
  Let $a,b \in P(\mathscr{G})$. Since $\Delta$ is a $*$-homomorphism and $\phi$ is
invariant,
  \begin{align*}
    \langle \Lambda(a_{(1)}) \otimes
    \pi(a_{(2)}b)\xi|\Lambda(a'_{(1)}) \otimes
    \pi(a'_{(2)}b')\xi'\rangle &=
    \phi(a_{(1)}^{*}a'_{(1)})\langle \xi|\pi(b^{*}a_{(2)}^{*}a'_{(2)}b')\xi'\rangle \\
    &= \sum_{p}
    \langle \xi|\pi(b^{*}\rho_{p}\phi(\rho_{p}a^{*}a'\rho_{p})\rho_{p}b')\xi'\rangle \\
    & =\sum_{p} \langle\Lambda(a\rho_{p}) \otimes \pi(\rho_{p}b)\xi |
    \Lambda(a'\rho_{p}) \otimes \pi(\rho_{p}b')\xi'\rangle.
  \end{align*}
  The map $\Lambda(a) \otimes \pi(b)\xi \mapsto \Lambda(a_{(1)})
  \otimes \pi(a_{(2)}b)\xi$ therefore extends to a partial isometry
  $V_{\pi}$ such that $V^{*}_{\pi}V_{\pi}$ has the claimed
  form. Clearly, its image is contained in the image of $\sum_{l}
  \rho_{l} \otimes \pi(\lambda_{l})$. Both images coincide
  because the defining properties of the antipode $S$
  imply 
 \begin{align*}
 V_{\pi}(\Lambda(a_{(1)}) \otimes \pi(S(a_{(2)})b)\xi)   &=
 \Lambda(a_{(1)}) \otimes \pi(a_{(2)}S(a_{(3)})b)\xi \\ &=  \sum_{p}\Lambda(a_{(1)}\epsilon(\lambda_p a_{(2)})) \otimes \pi(\lambda_{p}b)\xi   \\
 &= \sum_{p}\Lambda(\rho_{p}a_{(1)}\epsilon(a_{(2)})) \otimes \pi(\lambda_{p}b) \xi
 = \sum_{p} \Lambda(\rho_{p}a) \otimes \pi(\lambda_{p}b)\xi.
 \end{align*}
The formulas for $V_{\pi}V_{\pi}^{*}$ and $V_{\pi}^{*}$ follow.
\end{proof}
In the case  $\pi=\pi_{\red}$, we obtain a partial isometry
$V:=V_{\pi_{\red}}$ on $\LGtwo \otimes \LGtwo$ which we call the
\emph{right regular multiplicative partial isometry} of
$\mathscr{G}$. Write
\begin{align*}
  \overline{E} &:=VV^{*} = \sum_{k} \rho_{k} \otimes \lambda_{k} \in
  \mathcal{B}(\LGtwo \otimes \LGtwo).
\end{align*}
\begin{Prop} \label{prop:vn-delta} Let $\mathscr{G}$ be a partial
  compact quantum group with underlying total $*$-algebra
  $P(\mathscr{G})$ and associated GNS-construction
  $(\LGtwo,\Lambda,\pi_{\red})$. Then there
  exists a unique normal, faithful $*$-homomorphism $\vnDelta \colon
  \LGinf \to \LGinf \vntimes \LGinf$ such that
  \begin{align*}
    \vnDelta(\pi_{\red}(a)) (\Lambda(b) \otimes \Lambda(c)) =
    \Lambda(a_{(1)}b) \otimes \Lambda(a_{(2)}b) 
  \end{align*}
  for all $a,b,c \in P(\mathscr{G})$. Moreover, $\vnDelta(1)=\vnE$ and
  $\vnDelta(C^{r}_{0}(\mathscr{G})) \subseteq \vnE M(\CrG \otimes
  \CrG)\vnE$.
\end{Prop}
\begin{proof}
  Uniqueness is clear. To prove existence, one easily verifies that
  the map
  \begin{align*}
 \vnDelta \colon \LGinf \to {\cal B}(\LGtwo \otimes \LGtwo), \ x
  \mapsto V(x \otimes 1)V^{*}   
  \end{align*}
  has the desired properties. For example it is faithful because
  $\vnDelta(x)=0$ implies $x\otimes 1=0$ on $V^{*}V(L^{2}(\mathscr{G})
  \otimes L^{2}(\mathscr{G}))$ and hence $x=0$ on $\bigoplus_{k}
  \rho_{k}^{\op}L^{2}(\mathscr{G})=L^{2}(\mathscr{G})$.
\end{proof}

Denote by $\nu$ the weight on $l^{\infty}(I)$ given by
integration with respect to the counting measure on $I$. This weight is
normal, faithful, and semi-finite,  briefly written nsf. For general nsf weights $\psi$ on a von Neumann algebra $M$, we will use the standard notation \[\mathfrak{N}_{\psi} = \{a\in M\mid \psi(a^*a)<\infty\},\]\[\mathfrak{M}_{\psi}^+  = \{a\in M^+\mid\psi(a)<\infty\}\] and \[\mathfrak{M}_{\psi} = \mathfrak{N}_{\psi}^*\mathfrak{N}_{\psi} = \spann \mathfrak{M}_{\psi}^+.\] The same notation will be used for operator valued weights.

Furthermore, denote by
\begin{align*}
  \lambda,\rho \colon l^{\infty}(I) \to
  M(\CrG)\subseteq  {\cal B}(\LGtwo)
\end{align*}
the faithful, normal $*$-homomorphisms that send the delta function at
$k\in I$ to the operators $\lambda_{k}$ or $\rho_{k}$,
respectively.
%Seems necessary for uniqueness also to ask for centrality of the unit projections.
\begin{Prop} \label{prop:vn-phi} Let $\mathscr{G}$ be a partial
  compact quantum group with underlying $*$-algebra $P(\mathscr{G})$,
  invariant integral $\phi$ and GNS-construction
  $(\LGtwo,\Lambda,\pi_{\red})$. Then there exist a unique normal semi-finite weight $\vnphi$ on $\LGinf$ and unique normal
  conditional expectations $T_{\lambda}$ and $T_{\rho}$ onto respectively $\lambda(l^{\infty}(I))$ and $\rho(l^{\infty}(I))$ such that
  \[\pi_{\red}(P(\mathscr{G})) \subseteq
  \mathfrak{M}_{\vnphi} \cap   \mathfrak{M}_{T_{\lambda}} \cap
  \mathfrak{M}_{T_{\rho}},\] such that the $\UnitC{k}{m}$ lie in the centralizer of $\vnphi$, and  such that
  \begin{align*}
\vnphi \circ \pi_{\red} &= \phi,&
\nu \circ \lambda^{-1}\circ T_{\lambda}   &= \vnphi, &  \nu \circ \rho^{-1} \circ
  T_{\rho} &= \vnphi.
  \end{align*}
Furthermore,  $(\id \otimes \vnphi)(\vnDelta(x)) =  T_{\lambda}(x)$ and $
    (\vnphi \otimes \id)(\vnDelta(x)) = T_{\rho}(x)$ 
 for all
  $x\in \LGinf^{+}$.
\end{Prop}
\begin{proof}
  We first prove uniqueness of $\vnphi$.  The
  $\Grt{p}{k}{m}:=\pi_{\red}(\UnitC{k}{m})=\lambda_{k}\rho_{m}$ are pairwise orthogonal
  projections in $\mathfrak{N}_{\vnphi}\cap \mathfrak{N}_{\vnphi}^*$ which sum up to $1$. By the centralizing assumption, $\vnphi$ is the sum of the bounded positive functionals
  $\Grt{\vnphi}{k}{m} \colon x \mapsto
  \vnphi(\Grt{p}{k}{m}x\Grt{p}{k}{m})$ which are determined
  by their restrictions to $\pi_{\red}(P(\mathscr{G}))$. The centralizing condition also implies the existence of the maps $T_{\lambda}$ and $T_{\rho}$. 

Let us next prove existence.  Since $\phi$ is normalized,   each nonzero
$\Lambda(\UnitC{k}{m})$ is a unit vector and defines a state
\begin{align*}
  \vnphic{k}{m} \colon x \mapsto \langle \Lambda(\UnitC{k}{m})|x \Lambda(\UnitC{k}{m})\rangle.
\end{align*}The weight $\vnphi$ and the maps $T_{\lambda}$ and $T_{\rho}$
defined by
\begin{align} \label{eq:vn-weights}
  \vnphi(x) &:= \sum_{k,m} \vnphic{k}{m}(x), &
    T_{\lambda}(x)&:= \sum_{k,m}
\vnphic{k}{m}(x)\lambda_{k}, & 
T_{\rho}(x)&:=
    \sum_{k,m} \vnphic{k}{m}(x)\rho_{m}
\end{align}
for all $x\in \LGinf^{+}$  are normal, (semi-)finite and satisfy $\nu
\circ \lambda^{-1} \circ T_{\lambda} = \vnphi = \nu \circ \rho^{-1}
\circ T_{\rho}$.

We finally check the invariance equations. For $x\in P(\mathscr{G})$, we get by right invariance of $\phi$ that   \begin{align*}
    (\vnphic{k}{m}\otimes \id)(\vnDelta(\pi_{\red}(\rho_lx))) &= 
    \phi(\UnitC{k}{l}x\UnitC{k}{l}) \Grt{p}{m}{l}.
  \end{align*}
By taking strong limits, we obtain that for $x\in \LGinf$, 
  \begin{align*}
    (\vnphic{k}{m}\otimes \id)(\vnDelta(\rho_lx)) &= 
    \vnphic{k}{l}(x) \UnitC{m}{l}.
  \end{align*} Summing over $k$, $l$ and $m$, we  find $(\vnphi\otimes \id)
  \circ \vnDelta = T_{\rho}$. 
  
  A similar argument shows that
  $(\id\otimes \vnphi) \circ \vnDelta=T_{\lambda}$.
\end{proof}

We next use the family of functionals $(f_{z})_{z \in \C}$ on
$P(\mathscr{G})$ introduced in \cite[Theorem 2.25]{DCT1} to show that
$\vnphi$ is faithful and to describe the associated objects of
Tomita-Takesaki theory. This way, we  also obtain a second construction of
$\vnphi$. 

Define $\sigma_{z} \colon P(\mathscr{G}) \to P(\mathscr{G})$
by
\begin{align*}
\sigma_{z}(a) &= (f_{iz} \otimes \id \otimes f_{iz})((\Delta_{kl}
\otimes \id)\Delta_{mn}(a)) \quad \text{for } a\in \Gr{A}{k}{l}{m}{n}.
\end{align*}
Then \cite[Theorem 2.25]{DCT1} implies that
\begin{align} \label{eq:alg-mod-aut}
 \sigma_{z}(\sigma_{z'}(a)) &= \sigma_{z+z'}(a), &
 \sigma_{z}(a^{*}) &= \sigma_{\overline{z}}(a)^{*}, &
\phi(\sigma_{z}(a)) &= \phi(a), & \phi(ab) &= \phi(b\sigma_{-i}(a))
\end{align}
for all $a,b\in P(\mathscr{G})$ and $z,z'\in \C$. 
\begin{Prop} \label{prop:tomita}
  Let $\mathscr{G}$ be a partial compact quantum group with underlying
  $*$-algebra $P(\mathscr{G})$ and associated GNS-construction
  $(\LGtwo,\Lambda,\pi_{\red})$. Then 
  $\Lambda(P(\mathscr{G}))$ is a Tomita algebra with
  respect to the operations and the family of operators given by
  \begin{align*}
    \Lambda(a)\Lambda(b)&=\Lambda(ab), & \Lambda(a)^{*}&=
  \Lambda(a^{*}), &
\nabla_{z}\Lambda(a)&=\Lambda(\sigma_{z}(a))
  \end{align*}
  for all $a,b\in P(\mathscr{G})$ and $z\in \C$.  The associated left
  von Neumann algebra and its n.s.f.\ weight are $\LGinf$ and
  $\vnphi$, respectively, the modular operator $\Delta_{\vnphi}$ is
  the closure of $\nabla_{-i}$, and the   modular automorphism group $\sigma^{\vnphi}$
  and modular conjugation $J_{\vnphi}$ are given by
  \begin{align*}
    \sigma^{\vnphi}_{t} \circ \pi_{\red} &= \pi_{\red} \circ \sigma_{t}
    \text{ for all } t\in \R, & J_{\vnphi}
    \Lambda(a)&=\Lambda(\sigma_{i/2}(a)^{*}) \text{ for all } a\in
    P(\mathscr{G}).
  \end{align*}
\end{Prop}
\begin{proof}
The map $\pi_{\red}(a)\colon \Lambda(b) \to \Lambda(ab)$ is bounded for
  each $a \in P(\mathscr{G})$ by Proposition \ref{prop:gns}, and the
  involution is pre-closed because for all $a,b \in P(\mathscr{G})$,
  \begin{align*}
    \langle \Lambda(a)|\Lambda(b^{*})\rangle = \phi(a^{*}b^{*}) =
    \phi(b^{*}\sigma_{-i}(a^{*})) = \langle
    \Lambda(b)|\Lambda(\sigma_{-i}(a^{*}))\rangle.
  \end{align*}
  Therefore, $\Lambda(P(\mathscr{G}))$ is a left Hilbert algebra. It
  is a Tomita algebra because the
  map $z\mapsto \langle \Lambda(a)|\nabla_{z}\Lambda(b)\rangle =
  \phi(a^{*}\sigma_{z}(b))$ is entire for all $a,b\in P(\mathscr{G})$  by
 \cite[Theorem 2.25]{DCT1}
  and, by \eqref{eq:alg-mod-aut}, 
  \begin{gather*}
    \nabla_{z}(\Lambda(a)^{*}) = (\nabla_{\overline{z}}\Lambda(a))^{*}, \qquad
    \langle \Lambda(a)|\nabla_{z}\Lambda(b)\rangle = \langle
    \nabla_{-\overline{z}}\Lambda(a) |\Lambda(b)\rangle, \\ \langle
    \Lambda(a)^{*}|\Lambda(b)^{*}\rangle = \langle \Lambda(b)|\nabla_{-i}\Lambda(a)\rangle
  \end{gather*}
  for all $a,b\in P(\mathscr{G})$ and $z\in \C$ and   

 The associated left von Neumann algebra is $\pi_{\red}(P(\mathscr{G}))''=\LGinf$. The
 associated n.s.f.\ weight $\tilde\phi$ coincides with $\vnphi$ because
 $\tilde \phi(\pi_{\red}(a^{*}a))=\langle\Lambda(a)|\Lambda(a)\rangle =
 \phi(a^{*}a)$ for all $a\in P(\mathscr{G})$, so we can apply Proposition \ref{prop:vn-phi}. The remaining
 assertions follow  from
  \cite[Theorem VI.2.2 and its proof]{Taksak2}.
\end{proof}


The operator-algebraic structures constructed so far fit into the
theory of measured quantum groupoids of Enock and Lesieur \cite{Eno2,Les1} as follows. We will use notations as in these papers.

The relative tensor product of $\LGtwo$ with itself, relative to the
representations $\rho,\lambda$ of $l^{\infty}(I)$ and the weight
$\nu$, has the simple form
\begin{align*}
\LGtwo \otimesrl \LGtwo \cong
  \bigoplus_{k} (\rho_{k}\LGtwo \otimes \lambda_{k}\LGtwo) =
  \vnE(\LGtwo \otimes \LGtwo),
\end{align*}
the relative tensor product of operators $S\in \rho(l^{\infty}(I))'$
and $T \in \lambda(l^{\infty}(I))'$ gets identified with the
compression
\begin{align*}
S \otimesrl T \equiv
  \vnE(S \otimes
  T) = (S \otimes T)\vnE \subseteq {\cal B}(\vnE(\LGtwo
  \otimes \LGtwo)),
\end{align*}
and the fiber product of  $  \LGinf$ with itself, relative to $\rho$
and $\lambda$,  gets identified with
\begin{align} \label{eq:vn-fiber}
  \begin{aligned}
    \LGinf \astrl \LGinf &= (\LGinf' \otimesrl \LGinf')' \\ &\equiv
    (\vnE(\LGinf' \otimes \LGinf'))' = \vnE(\LGinf \otimes
    \LGinf)\vnE,
  \end{aligned}
\end{align} 
where for notational reasons we indexed the fiber product $*$ with $I$ in stead of $l^{\infty}(I)$.

Since $\vnDelta(1)=\vnE$, we can co-restrict $\vnDelta$ to  a
normal, faithful unital $*$-homomorphism
\begin{align*}
  \tilde\Delta \colon \LGinf \to   \LGinf \astrl \LGinf.
\end{align*}
\begin{Theorem}
  Let $\mathscr{G}$ be an $I$-partial compact quantum group with
  associated von Neumann algebra $\LGinf$.  Then
  $(l^{\infty}(I),\LGinf, \lambda,\rho,\tilde\Delta,
  T_{\lambda},T_{\rho},\nu)$ is a measured quantum groupoid in the
  sense of \cite{Eno2}.
\end{Theorem}
\begin{proof}
We first show that  $(l^{\infty}(I),\LGinf,
\lambda,\rho,\tilde\Delta)$ is  a Hopf-von
  Neumann bimodule in the sense of  \cite{Val1}, that is, 
  \begin{align*}
  \tilde\Delta(\lambda(x)) &= \lambda(x) \otimesrl 1, &
  \tilde\Delta(\rho(x)) &= 1 \otimesrl \rho(x), &
  (\tilde\Delta \astrl \id)\circ \tilde\Delta &= (\id \astrl
    \tilde\Delta) \circ \tilde\Delta,
\end{align*}
where $x\in l^{\infty}(I)$.
The first two equations follow immediately from the relation
$\Delta(\UnitC{k}{m})=\lambda_{k} \otimes \rho_{m}$. To verify
the third equation, we identify
\begin{align*}
 \LGinf \astrl \LGinf \astrl \LGinf \cong \vnE^{(2)}(\LGinf
  \otimes \LGinf \otimes \LGinf)\vnE^{(2)},
\end{align*}
where $\vnE^{(2)}=(\vnE \otimes 1)(1 \otimes \vnE)$. Then the two
compositions  become corestrictions of the maps $(\vnDelta \otimes
\id)\circ \vnDelta$ and $(\id \otimes \vnDelta)\circ \vnDelta$,
respectively, which coincide on $\pi_{\red}(P(\mathscr{G}))$ by
co-associativity of $\Delta$. 

Next, we need to
show that the modular automorphism groups of the weights $\nu \circ
\lambda^{-1} \circ T_{\lambda}$ and $\nu \circ \rho^{-1} \circ
T_{\rho}$ commute, which is trivially true because the two
compositions coincide with $\vnphi$. Finally,
$T_{\lambda}$ is left-invariant in the sense that
  \begin{align*}
   (\id \underset{\nu}{_{\rho}\ast_{\lambda}} \vnphi)(\tilde\Delta(x)) = T_{\lambda}(x) 
  \end{align*}
  for all $x\in \LGinf_{+}$, because the left hand
  side coincides with $(\id \ast \vnphi)(\vnDelta(x))$, whence the
  equation follows from Proposition
  \ref{prop:vn-phi}. Likewise $T_{\rho}$ is right-invariant in
  the appropriate sense. 
\end{proof}

%Added Remark
\begin{Rem} It is not clear at the moment if a measured quantum groupoids of the form $(N,M, \lambda,\rho,\Delta,
  T_{\lambda},T_{\rho},\nu)$ with $N$ atomic abelian and $T_{\lambda}$ and $T_{\rho}$ bounded arises from a partial compact quantum group. 
  \end{Rem}
  
The scaling group and unitary antipode of the measured quantum
groupoid above can be described in terms of the functionals
$(f_{z})_{z\in \C}$ on $P(\mathscr{G})$ introduced in \cite[Theorem
2.25]{DCT1} as follows.  Define $\tau_{z} \colon P(\mathscr{G}) \to
P(\mathscr{G})$ by
\begin{align*}
% Changed the order!
\tau_{z}(a) &= (f_{iz} \otimes \id \otimes f_{-iz})((\Delta_{kl}
\otimes \id)\Delta_{mn}(a)) \quad \text{for } a\in \Gr{A}{k}{l}{m}{n}.
\end{align*}
Then \cite[Theorem 2.25]{DCT1} implies that $\tau_{z}$ is an algebra
homomorphism and
\begin{gather} \label{eq:alg-scale}
  \begin{aligned}
    \tau_{z} \circ \tau_{z'} &= \tau_{z+z'}, & \tau_{z}\circ \ast &=
    \ast\circ \tau_{\overline{z}}, & S^{2} &= \tau_{-i}, & S \circ
    \tau_{z} &= \tau_{z} \circ S,
  \end{aligned}
 \\  \label{eq:alg-mod-aut-delta} (\tau_{z} \otimes \sigma_{z})
    \circ \Delta_{pq} = \Delta_{pq} \circ \sigma_{z}
  \end{gather}
for all $z\in \C$. Moreover, $R:=S\circ
\tau_{i/2}$ is a $*$-anti-automorphism of $P(\mathscr{G})$.  The
one-parameter group $\tau=(\tau_{t})_{t\in \R}$ and $R$ lift to
$C^{u}_{0}(\mathscr{G})$ by the universal property of
$C^{u}_{0}(\mathscr{G})$, and to the reduced level as well, as the next proposition shows.
\begin{Prop}
  Let $\mathscr{G}$ be an $I$-partial compact quantum group with
  associated von Neumann algebra $\LGinf$. Then there exist
a unique strongly continuous one-parameter group $\vntau$ and
 a unique $*$-anti-automorphism $\vnR$ of $\LGinf$ such that
 $\vntau_{t} \circ \pi_{\red} = \pi_{\red} \circ
\tau_{t}$ for  all $t\in \R$ and
$\vnR \circ \pi_{\red} = \pi_{\red} \circ R$.
 These are the scaling group and the unitary antipode of the measured
 quantum groupoid $(l^{\infty}(I),\LGinf, \lambda,\rho,\tilde\Delta,
 T_{\lambda},T_{\rho},\nu)$, respectively.
\end{Prop}
\begin{proof}
  Uniqueness is clear.  Let us prove existence of $\vntau$.
  Invariance of $\phi$ and \cite[Theorem 2.25]{DCT1} imply $\phi \circ
  \tau_{z} = \phi$ for all $z\in \C$. Therefore, the formula
  $P_{t}\Lambda(a) = \Lambda(\tau_{t}(a))$ defines a one-parameter
  group of unitaries $(P_{t})_{t\in \R}$ on $\LGtwo$. By \cite[Theorem
  2.25]{DCT1}, elements of $\Lambda(P(\mathscr{G}))$ are analytic with
  respect to $P$, and hence $P$ is strongly
  continuous.  Therefore, we can define $\vntau$ by  $\vntau_{t}(x)=
  P_{t}xP_{t}^{*}$ for all $x\in \LGinf$ and $t\in \R$.
  

  Next, we prove existence of $\vnR$. We claim that the formula $
  I\Lambda(a) = \Lambda(R(a)^{*})$ defines a conjugate-linear
  isometry $I$. Indeed, $\phi(R(a)R(b)^{*})= (\phi\circ R)(b^{*}a)$
  for all $a,b\in P(\mathscr{G})$ and $\phi \circ R=\phi \circ S \circ
  \tau_{i/2} = \phi \circ \tau_{i/2}=\phi$, where we used \cite[
  Corollary 2.13]{DCT1}. Next, $I^{2} = \id$ because $*\circ R \circ *
  \circ R= R^{2}=\id$ by \eqref{eq:alg-scale} and \cite[Proposition 1.16]{DCT1}. Now, short calculations show that the map $\vnR \colon
  x \mapsto Ix^{*}I$ has the desired properties.

Let us show   that the unitary antipode  $\tilde R$ and the
  scaling group   $\tilde\tau$ of the measured quantum groupoid
  coincide with $\vnR$ and $\vntau$, respectively.
  By \cite[Theorem A.6]{Eno2},
  \begin{align*}
    \tilde R(\id \underset{\nu}{_{\rho} \ast_{\lambda}}
    \omega_{J\Lambda(b),J\Lambda(b)})(\tilde\Delta(\pi_{\red}(a^{*}a))) &= (\id
    \underset{\nu}{_{\rho} \ast_{\lambda}}
    \omega_{J\Lambda(a),J\Lambda(a)})(\tilde\Delta(\pi_{\red}(b^{*}b))),
  \end{align*}
  for all $a,b\in P(\mathscr{G})$.  Let $c=a^{*}a$ and $d=b^{*}b$.  A
  short calculation using \eqref{eq:alg-mod-aut} shows that the right hand
  side is equal to
\begin{align*}
  \pi_{\red} ( d_{(1)}\phi(\sigma_{i/2}(a)d_{(2)}\sigma_{i/2}(a)^{*})) =
  \pi_{\red}(d_{(1)}\phi(\sigma_{i/2}(c)d_{(2)})).
  \end{align*}
By \cite[Lemma 1.13]{DCT1},  \eqref{eq:alg-mod-aut-delta} and
  \eqref{eq:alg-mod-aut},  this is equal to
  \begin{align*}
   \pi_{\red}(S(\sigma_{i/2}(c)_{(1)}\phi(\sigma_{i/2}(c)_{(2)}d))) &=
   \pi_{\red}(S(\tau_{i/2}(c_{(1)})\phi(\sigma_{i/2}(c_{(2)})d))) \\
   &= \pi_{\red}(R(c_{(1)}\phi(\sigma_{i/2}(d)c_{(2)}))) \\
   &=\vnR((\id \underset{\nu}{_{\rho} \ast_{\lambda}}
    \omega_{J\Lambda(b),J\Lambda(b)})(\vnDelta(\pi_{\red}(a^{*}a)))).
  \end{align*}
  Since elements of the form $\pi_{\red}(a^{*}a)$ are dense in
  $\LGinf^{+}$ and $\Lambda(P(\mathscr{G}))$ is dense in $\LGtwo$, we
  can conclude from \cite[Theorem A.7]{Eno2} that elements of the form
  \begin{align*}
  (\id \underset{\nu}{_{\rho}
    \ast_{\lambda}}
  \omega_{J\Lambda(b),J\Lambda(b)})(\tilde\Delta(\pi_{r}(a^{*}a)))  
  \end{align*}
 are
  dense in $\LGinf_{+}$. Consequently, $\tilde R=\vnR$. 

Next,  \eqref{eq:alg-mod-aut-delta} implies $(\vntau_{t} \otimes
  \sigma^{\vnphi}_{t}) \circ \vnDelta = \vnDelta \circ
  \sigma^{\vnphi}_{t}$ for all $t\in \R$.  But by \cite[Theorem
  A.5]{Eno2},
  \begin{align*}
    (\tilde \tau_{t} \astrl \sigma^{\vnphi}_{t}) \circ \tilde \Delta
    &=\tilde \Delta \circ \sigma^{\vnphi}_{t}.
  \end{align*}
Now,  a similar density  argument as above shows that
  $\tilde \tau_{t}=\vntau_{t}$.
\end{proof}

%%% Local Variables: 
%%% mode: latex
%%% TeX-master: "dynamical-SUq-file"
%%% End: 
