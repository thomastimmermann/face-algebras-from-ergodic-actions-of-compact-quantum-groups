\section{Partial compact quantum groups and  $C^{*}$-algebraic ones}


We now clarify the relation between the $C^{*}$-partial compact quantum groups studied above and the partial compact quantum groups introduced in \cite{DCT1}. Similarly as in the case of compact quantum groups, every $C^{*}$-partial compact quantum group contains a unique dense partial compact quantum group which is spanned  by the matrix coefficients   of irreducible unitary corepresentations, and conversely, every partial compact quantum group has a universal and a reduced $C^{*}$-algebraic completion.

\subsection{Partial compact quantum groups}

We first recall the notion of a partial compact quantum group  from \cite{DCT1}. 

Let $I$ be a set. An $I^2$-\emph{partial algebra} (with object set $I$), \cite[Definition 1.1]{DCT1}, will be an algebra, that is, a (not necessarily unital) $\C$-algebra together with a family $\UnitC{k}{l}$ of orthogonal idempotents, indexed by elements $(k,l)\in I^2$ such that
\begin{itemize}
\item[(PA)]  the natural map $\bigoplus_{k,l,m,n} \UnitC{k}{m}A\UnitC{l}{n} \rightarrow A$ is an isomorphism of vector spaces.
\end{itemize}
 Note that some of the $\UnitC{k}{l}$ are allowed to be zero. We will in the following write \[\Gr{A}{k}{l}{m}{n} = \UnitC{k}{m}A\UnitC{l}{n}, \quad \lambda_k = \sum_l \UnitC{k}{l},\quad \rho_l = \sum_k \UnitC{k}{l},\]  where the latter two elements are interpreted as elements in the multiplier algebra $M(A)$. 

We call $A$ a \emph{partial bialgebra}, \cite[Definition 1.5]{DCT1} if one is also given a family of linear maps \[\Delta_{rs}: \Gr{A}{k}{l}{m}{n}\rightarrow \Gr{A}{k}{l}{r}{s} \otimes \Gr{A}{r}{s}{m}{n},\] called \emph{partial comultiplications}, and a map \[\epsilon: A \rightarrow \C,\] called the \emph{counit} satisfying the following assumptions.
\begin{enumerate}\renewcommand{\labelenumi}{(PB\arabic{enumi})}
\item For all $a\in \Gr{A}{k}{l}{m}{n}$ and $p,q,r,q\in I$, one has \[(\Delta_{pq}\otimes \id)\Delta_{rs}(a) = (\id\otimes \Delta_{rs})\Delta_{pq}(a).\]
\item The map $\epsilon$ vanishes on all $\Gr{A}{k}{l}{m}{n}$ with $k\neq m$ or $l\neq n$. 
\item For all $a\in \Gr{A}{k}{l}{m}{n}$, \[(\id\otimes \epsilon)\Delta_{mn}(a) = (\epsilon\otimes \id)\Delta_{kl}(a) = a.\]
\item $\epsilon(\UnitC{k}{k})= 1$ for all $k\in I$. 
\item $\Delta_{ll'}(\UnitC{k}{m}) = \delta_{l,l'} \UnitC{k}{l}\otimes \UnitC{l}{m}$.
%\item For all $a \in \Gr{A}{k}{l}{m}{n}$, one has $\Delta_{rs}(a)^* = \Delta_{sr}(a^*)$ and $\epsilon(a^*) = \overline{\epsilon(a)}$. 
\item $\epsilon(ab) = \epsilon(a)\epsilon(b)$ for all $a\in \Gr{A}{k}{l}{m}{n}$ and $b\in \Gr{A}{l}{p}{n}{q}$,
\item For all $a\in \Gr{A}{k}{l}{m}{n}$ and all $r$, $\Delta_{rs}(a)=0$ for almost all $s$.
\item For all $a\in \Gr{A}{k}{l}{m}{n}$ and all $s$, $\Delta_{rs}(a)=0$ for almost all $r$.
\item For all $a\in \Gr{A}{k}{l}{m}{n}$ and $b\in \Gr{A}{k'}{l'}{m'}{n'}$, one has \[\Delta_{rs}(ab) = \sum_{t} \Delta_{rt}(a)\Delta_{ts}(b).\]
\end{enumerate} 

In the last multiplicative identity, we have used the finite support condition on the maps $s\mapsto \Delta_{rs}(a)$ to make sense of the a priori infinite sum as a finite sum. Note that $\epsilon$ is \emph{not} required to be a homomorphism on the whole of $A$.

The expression $\Delta(a) = \sum_{rs}\Delta_{rs}(a)$ can be made sense of inside the multiplier algebra $M(A\otimes A)$, and we obtain in this way a (non-degenerate) homomorphism \[\Delta: A\rightarrow M(A\otimes A),\] called the \emph{(total) comultiplication}, satisfying the ordinary comultiplication formula. There is then a unique extension of $\Delta$ to $M(A)$ such that $\Delta(1) =\sum_k \rho_k\otimes \lambda_k$. We will use the Sweedler-like notations \[\Delta(a) = a_{(1)}\otimes a_{(2)}.\] Note that because of the finiteness condition, expressions such as $\Delta(a)(1\otimes b)$ lie in $A\otimes A$.

%We will in the following use the linear maps $\Pi^L,\Pi^R,\overline{\Pi}^L$ and $\overline{\Pi}^R$ from $A$ to $M%(A)$, determined by the following formulas for $a\in \Gr{A}{k}{l}{m}{n}$: \[\Pi^L(a) = \epsilon(a)\lambda_k,\quad \overline{\Pi}^L(a) = \epsilon(a)\lambda_l,\quad \overline{\Pi}^R(a) = \epsilon(a)\rho_m,\quad \overline{\Pi}^R(a) = \epsilon(a)\rho_n.\]

A \emph{partial Hopf algebra} \cite[Definition 1.7]{DCT1} is a partial bialgebra $(A,\Delta)$ for which there exists an invertible, antimultiplicative map $S: A\rightarrow A$ with
\begin{align} \label{eq:antipode}
  S(a_{(1)})a_{(2)} = \sum_{k\in I} \epsilon(\lambda_ka)\lambda_k,\qquad a_{(1)}S(a_{(2)}) = \sum_n\epsilon(a\rho_n)\rho_n,
\end{align}
where the identities have to be interpreted as identities of multipliers. The map $S$ is then referred to as the \emph{antipode}. 

Given a partial Hopf algebra $(A,\Delta)$ with  antipode  $S$,  the maps
  $T_{1},T_{2} \colon A \otimes A \to A \otimes A$ given by
  \begin{align*}
  T_{1} (a\otimes b)&= \Delta(a)(1 \otimes b), &
  T_{2} (a\otimes b)&= (a \otimes 1)\Delta(b)
\end{align*}
have partial inverses $R_{1},R_{2}$, which are given by the formulas
\begin{align*}
    R_{1}(a \otimes b) &= a_{(1)}\otimes S(a_{(2)})b, &
    R_{2}(a\otimes b) &= aS(b_{(1)})\otimes b_{(2)}.
  \end{align*}
Indeed, straightforward calculations show that
\begin{align} \label{eq:canonical-maps}
R_{1}T_{1} (a\otimes b) &=
 \sum_{p} a\rho_{p} \otimes \rho_{p}b, &  T_{1}R_{1}(a \otimes b) &=\Delta(1)(a\otimes b), \\ \label{eq:e2g2}
R_{2}T_{2}(a \otimes b) &= \sum_{p} a\lambda_{p} \otimes
    \lambda_{p}b, &
T_{2}R_{2}(a\otimes b) &= (a\otimes b)\Delta(1).
\end{align}

Following \cite[Definition 1.8]{DCT1}, one can show that the relation $k\sim l$ if $\UnitC{k}{l}\neq0$ defines an equivalence relation on $I$. The quotient set $\mathscr{I} = I/\sim$ will be called the \emph{hyperobject set} of $(A,\Delta)$.

Finally, an invariant integral on $(A,\Delta)$ \cite[Definition 1.12]{DCT1} will consist of a functional $\phi: A\rightarrow \C$ such that $\phi(\UnitC{k}{l})=1$ for all $k,l$ with $\UnitC{k}{l}\neq 0$, and, for all $a\in A$,
\begin{align} \label{eq:integral}
  (\id\otimes \phi)\Delta(a) = \sum_{k} \phi(\lambda_k a)\lambda_k,\qquad (\phi\otimes \id)\Delta(a) = \sum_m \phi(\rho_m a)\rho_m,
\end{align}
again as identities in the multiplier algebra of $A$. Such an invariant integral is unique if it exists.

We can now give the definition of an $I$-partial compact quantum group, obtained by putting also a compatible $^*$-structure in the mix.

\begin{Def}[\cite{DCT1}, Definition 1.17] \label{def:pcqg} An \emph{$I$-partial compact quantum group} $\mathscr{G}$ consists of a partial Hopf algebra $P(\mathscr{G})$ with invariant integral $\phi$, endowed with an antimultiplicative, antilinear involution $*:A\rightarrow A$ satisfying the following conditions.
\begin{enumerate}
\item All $\UnitC{k}{l}$ are self-adjoint.
\item $\Delta$ is a $^*$-homomorphism.
\item $\epsilon(a^*) = \overline{\epsilon(a)}$ for all $a\in A$. 
\item $\phi(a^*) = \overline{\phi(a)}$ and $\phi(a^*a)\geq 0$ for all $a\in A$.
\end{enumerate}
\end{Def} 

We refer to $P(\mathscr{G})$ as the associated (function) algebra of $\mathscr{G}$.

\begin{Lem} \label{lem:strong-invariance}
  Let $\mathscr{G}$ be a partial compact quantum group with invariant  integral $\phi$. Then
  for all $a,b\in P(\mathscr{G})$,
  \begin{align*}
a_{(1)}\phi(ba_{(2)}) &= S(b_{(1)})\phi(b_{(2)}a), &
\phi(a_{(1)}b)a_{(2)} &= S(b_{(2)})\phi(ab_{(1)}).    
  \end{align*}
\end{Lem}
\begin{proof}
We only prove the first equation. The relations (PB3),
 \eqref{eq:antipode} and \eqref{eq:integral} imply
  \begin{align*}
    a_{(1)}\phi(ba_{(2)}) &= \sum_{n}
    a_{(1)}\phi(\epsilon(b_{(1)}\rho_{n})b_{(2)}\lambda_{n}a_{(2)}) \\
    &= \sum_{n} \epsilon(b_{(1)}\rho_{n})\rho_{n}a_{(1)}\phi(b_{(2)}a_{(2)})
    \\
    &= S(b_{(1)})b_{(2)}a_{(1)}\phi(b_{(3)}a_{(2)}) =
    S(b_{(1)})\phi(b_{(2)}a). 
  \end{align*}
\end{proof}



Let us also briefly recall the notion of \emph{unitary representation} of a partial compact quantum group. Let $\Hsp$ be an $I^2$-graded Hilbert space, whose components we write $\GrDA{\Hsp}{k}{l}$. We call $\Hsp$ \emph{row- and column finite dimensional} (rcfd) if all $\GrDA{\Hsp}{k}{l}$ are finite dimensional and if, for any fixed $l$ (resp.~ fixed $k$), there are only finitely many $k$ (resp.~ $l$) for which $\GrDA{\Hsp}{k}{l}$ is not zero.  
%We further write $p_{kl}$ for the projection on $\GrDA{\Hsp}{k}{l}$, and $L_k =\sum_{l} p_{kl}$, $R_l = \sum_k p_{kl}$ (as strongly convergent sums).

\begin{Def}[\cite{DCT1}, Definition 2.1 and Definition 2.18]\label{DefUniCorep} Let $\mathscr{G}$ be a partial compact quantum group. A \emph{unitary representation $\mathscr{X}$ of $\mathscr{G}$} (or unitary corepresentation of $P(\mathscr{G})$) $\mathscr{X}$ will consist of an rcfd Hilbert space $\Hsp$ and a collection of elements \[\Gr{X}{k}{l}{m}{n}\in \Gr{A}{k}{l}{m}{n}\otimes B(\GrDA{\Hsp}{m}{n},\GrDA{\Hsp}{k}{l})\] such that \begin{eqnarray*} (\Delta_{rs}\otimes \id)\Gr{X}{k}{l}{m}{n} &=& \left(\Gr{X}{k}{l}{r}{s}\right)_{13}\left(\Gr{X}{r}{s}{m}{n}\right)_{23},\\ (\epsilon\otimes \id)\Gr{X}{k}{l}{k}{l} &=& \id_{\GrDA{\Hsp}{k}{l}}\end{eqnarray*} and \begin{eqnarray*} \sum_{k} \left(\Gr{X}{k}{l'}{m}{n'}\right)^*\Gr{X}{k}{l}{m}{n} &=& \delta_{l,l'}  \delta_{n,n'} \UnitC{l}{n}\otimes \id_{\GrDA{\Hsp}{m}{n}},\\ \sum_{n} \Gr{X}{k}{l}{m}{n}\left(\Gr{X}{k'}{l}{m'}{n}\right)^* &=& \delta_{k,k'}\delta_{m,m'} \UnitC{k}{m}\otimes \id_{\GrDA{\Hsp}{k}{l}}.\end{eqnarray*}
\end{Def} 

Note that the above sums are finite because of the rcfd condition. The defining property of the antipode then gives that \[(S\otimes \id)(\Gr{X}{k}{l}{m}{n}) = \left(\Gr{X}{m}{n}{k}{l}\right)^*.\]

It is easy to see that any collection of maps $\Gr{X}{k}{l}{m}{n}$ as in Definition \ref{DefUniCorep} gives rise to a partial isometry \[X = \sum_{k,l,m,n}\Gr{X}{k}{l}{m}{n}\in M(P(\mathscr{G})\otimes B_{00}(\Hsp)),\] where $B_{00}(\Hsp)$ is the $^*$-algebra of finite rank operators on $\Hsp$. We will then also write $\Hsp = \Hsp_{\mathscr{X}}$.


The left (resp.~ right) hyperobject support of a unitary representation will consist of all equivalence classes of $k$ (resp.~ all $l$) such that there exists $l$ (resp.~ $k$) with $\GrDA{\Hsp}{k}{l}\neq 0$. It can be shown that any unitary representation is a direct sum of irreducible representations \cite[Corollary 2.5]{DCT1}, and that the latter have left and right hypersupport consisting of each a single hyperobject. In general, we will say that a unitary representation has \emph{finite hyperobject support} if its left and right hyperobject supports are finite sets. This implies that the unitary representation will be a \emph{finite} direct sum of irreducibles. 

One has the following result which will be crucial in what follows. 

%% Shouldn't use $BB(\Hsp_{\mathscr{X}})_*$
\begin{Prop}[\cite{DCT1}, Proposition 2.10 and Proposition 2.20]\label{LemSpan} The linear spaces 
\[\mathcal{C}(\mathscr{X}) = \{(\id\otimes \omega)(X)\mid \omega \in B(\Hsp_{\mathscr{X}})_*\} \subseteq P(\mathscr{Ga}),\] 
defined for irreducible unitary representations $\mathscr{X}$, are linearly independent when the unitary representations are not equivalent, and moreover their joint linear span over all irreducible unitary representations equals $P(\mathscr{G})$.
\end{Prop}

Further references to \cite{DCT1} will be given whenever needed. 

\subsection{The partial compact quantum group of a $C^{*}$-algebraic one}

Given a $C^{*}$-partial compact quantum group $(A,\Delta)$, we show that the  matrix coefficients  of irreducible
unitary corepresentations  span a partial
compact quantum group inside $(A,\Delta)$ in the following sense:
\begin{Def} \label{def:pcqg-inside}
  Let $(A,\Delta_{A})$ be a $C^{*}$-pcqg.  A \emph{partial compact
    quantum group underlying $(A,\Delta_{A})$} is a partial compact quantum
  group $\mathscr{G}$ such that
  \begin{itemize}
  \item $\Gr{P(\mathscr{G})}{k}{l}{m}{n}$ is a dense subspace of $\Gr{A}{k}{l}{m}{n}$ for all $k,l,m,n$,
  \item the multiplication, local units and involution on $P(\mathscr{G})$ coincide with those of $A$,
  \item  the comultiplication $\Delta_{pq}$ of $P(\mathscr{G})$ is given by   $\Delta_{pq}(a) = (\rho_{p} \otimes \lambda_{p})\Delta_{A}(a)(\rho_{q}\otimes \lambda_{q})$.
  \end{itemize}
\end{Def}

\begin{Theorem} \label{thm:pcqg-inside} Let $(A,\Delta)$ be a $C^{*}$-partial quantum group. Then  it has a unique underlying partial compact quantum group $\mathscr{G}$, and this is given by
  \begin{align*}
    \Gr{P(\mathscr{G})}{k}{l}{m}{n} = \sum_{[X]}   \Gr{\mathcal{C}(X)}{k}{l}{m}{n},
  \end{align*}
where the sum is taken over a representative family of irreducible unitary corepresentations of $(A,\Delta)$.
  The invariant integral on $P(\mathscr{G})$ is the restriction of the invariant integral on $(A,\Delta)$, and the counit and antipode of $P(\mathscr{G})$ satisfy
  \begin{align} \label{eq:pcqg-counit-antipode}
    \varepsilon((\id \otimes \omega_{\xi,\eta})(X)) &= \langle \xi|\eta\rangle, &
    S((\id \otimes \omega_{\xi,\eta})(X)) &= (\id \otimes \omega_{\xi,\eta})(X^{*})
  \end{align}
  for every unitary rcfd corepresentation $(H,X)$ and vectors $\xi \in
  \Grd{H}{k}{l}$, $\eta \in \Grd{H}{m}{n}$. 
\end{Theorem}
\begin{proof}
We first show that the spaces $\Gr{P(\mathscr{G})}{k}{l}{m}{n}$  above  define a partial compact quantum group underlying $(A,\Delta)$.

 By Corollary \ref{cor:pw}, these spaces are dense inside $\Gr{A}{k}{l}{m}{n}$, and by Corollary  \ref{cor:corep-decompose}, they do not change if we sum over the equivalence class of all unitary rcfd corepresentations instead of only irreducible ones. They contain the local units of $A$ because
\begin{align*}
  \Gr{\mathcal{C}(E)}{k}{l}{m}{n} = \delta_{k,l}\delta_{m,n} \UnitC{k}{m},
\end{align*}
and they are closed with respect to the multiplication and involution because for any two unitary corepresentations $X$ and $Y$,
  \begin{align*}
    \Gr{\mathcal{C}(X)}{k}{l}{m}{n} \cdot \Gr{\mathcal{C}(Y)}{p}{q}{r}{s} &\subseteq
    \delta_{m,p}\delta_{n,r} \Gr{\mathcal{C}(X\Circt Y)}{k}{q}{m}{s}, &
    \left(\Gr{\mathcal{C}(X)}{k}{l}{m}{n}\right)^{*} = \Gr{\mathcal{C}(\overline{X})}{l}{k}{n}{m},
  \end{align*}
as one can easily check. Let us  next show
that for every unitary rcfd corepresentation $(H,X)$, the map $\Delta_{pq}$ defined by the formula in Definition \ref{def:pcqg-inside} sends $\Gr{\mathcal{C}(X)}{k}{l}{m}{n}$ into the algebraic tensor product
$\Gr{\mathcal{C}(X)}{k}{l}{p}{q} \otimes \Gr{\mathcal{C}(X)}{p}{q}{m}{n}$.  Consider an element
  \begin{align*}
    a = (\id\otimes
  \omega_{\xi,\eta})(X), \quad \text{where } \xi \in \Grd{H}{k}{l}, \ \eta\in \Grd{H}{m}{n}.
  \end{align*}
 Lemma \ref{lem:corep-intertwine} implies
  \begin{align*}
    \Delta_{pq}(a) = (\id \otimes \id \otimes \omega_{\xi,\eta})(X_{13}(1\otimes 1 \otimes p^{H}_{pq})X_{23}).
  \end{align*}
  Since $X$ is rcfd,  the projection $p^{H}_{pq}$ is compact. With an orthonormal
  basis $(\Gr{\zeta}{}{(i)}{p}{q})_{i}$ of its image, we obtain
  \begin{align*}
    \Delta_{pq}(a) = \sum_{i} (\id \otimes \omega_{\xi,\Gr{\zeta}{}{(i)}{p}{q}})(X) \otimes (\id \otimes
    \omega_{\Gr{\zeta}{}{(i)}{p}{q},\eta})(X) \in \Gr{\mathcal{C}(X)}{k}{l}{p}{q} \otimes \Gr{\mathcal{C}(X)}{p}{q}{m}{n}.
  \end{align*}

Thus, $P(\mathscr{G})$ is a partial $*$-bialgebra.  
  Corollary  \ref{cor:pw} implies the existence of a functional $\varepsilon \colon P(\mathscr{G})\to
  \C$ and a  linear map $S\colon P(\mathscr{G})\to P(\mathscr{G})$ satisfying \eqref{eq:pcqg-counit-antipode}.  We next show that these maps satisfy the counit and antipode relations.  

  We start with the counit.  By definition, $\varepsilon(\Gr{P(\mathscr{G})}{k}{l}{m}{n})=0$ if $(k,l)\neq
  (m,n)$.  Denote by $(\delta_{k})_{k}$ the canonical basis of $l^{2}(I)$. For the trivial
  corepresentation $E$, \eqref{eq:pcqg-counit-antipode} implies
  \begin{align*}
    \varepsilon(\UnitC{k}{k}) = \varepsilon((\id \otimes \omega_{\delta_{k},\delta_{k}})(E)) =
    \langle \delta_{k}|\delta_{k}\rangle = 1.
  \end{align*}
Next,   for arbitrary $X,\xi,\eta$ and $a$ as above,
\begin{align*}
    (\varepsilon \otimes\id)(\Delta_{kl}(a)) &= \sum_{i}\epsilon((\id \otimes
    \omega_{\xi,\Gr{\zeta}{}{(i)}{p}{q}})(X)) (\id \otimes \omega_{\Gr{\zeta}{}{(i)}{p}{q},\eta})(X)
    \\
    &= \sum_{i}\langle \xi|\Gr{\zeta}{}{(i)}{p}{q}\rangle(\id \otimes
    \omega_{\Gr{\zeta}{}{(i)}{p}{q},\eta})(X) \\ &= (\id \otimes \omega_{\xi,\eta})(X) \\ &= a,
  \end{align*}
  and a similar calculation shows that  $    (\id \otimes \varepsilon)(\Delta_{mn}(a))=a$.
Finally, if also $Y$ is a unitary rcfd corepresentation on a Hilbert space $K$ and
\begin{align*}
  b=(\id \otimes \omega_{\zeta,\theta})(Y), \quad \text{where } \zeta \in \Grd{K}{l}{p}, \theta \in \Grd{K}{n}{q},
\end{align*}
then $ab = (\id \otimes \omega_{\xi\otimes \zeta,\eta\otimes \theta})(X\Circt Y)$, where $\xi\otimes \zeta \in \Grd{(H\itimes K)}{k}{p}$, $\eta\otimes \theta \in \Grd{(H\itimes
  K)}{m}{q}$, whence
\begin{align*}
  \varepsilon(ab) = \langle \xi\otimes \zeta|\eta\otimes \theta\rangle = \langle
  \xi|\zeta\rangle\langle \zeta|\theta\rangle = \varepsilon(a)\varepsilon(b).
\end{align*}

Next, we verify the antipode relations. 
Lemma \ref{lem:corep-intertwine} implies $S(\Gr{P(\mathscr{G})}{k}{l}{m}{n}) \subseteq \Gr{P(\mathscr{G})}{n}{m}{l}{k}$.
Denote by $m \colon P(\mathscr{G})\otimes P(\mathscr{G})\to P(\mathscr{G})$ the multiplication
map and consider $a$ as above. Then by Lemma \ref{lem:corep-intertwine},
\begin{align*}
  \sum_{q} m(\id \otimes S)(\Delta_{pq}(a)) &= 
  \sum_{q,i} (\id \otimes \omega_{\xi,\Gr{\zeta}{}{(i)}{p}{q}})(X)  (\id \otimes
  \omega_{\Gr{\zeta}{}{(i)}{p}{q},\eta})(X^{*}) \\ &=
 (\id \otimes \omega_{\xi,\eta})(X(1\otimes \lambda^{H}_{p})X^{*})  \\
  &= \rho_{p}  (\id \otimes \omega_{\xi,\eta})(XX^{*})  = \UnitC{k}{p}\langle \xi|\eta\rangle =\UnitC{k}{p} \varepsilon(a),
\end{align*}
and a similar calculation shows that $\sum_{p} m(S \otimes \id)(\Delta_{pq}(a)) = \UnitC{q}{n}\varepsilon(a)$.

 It is clear that the invariant integral $h$ on $(A,\Delta)$ restricts to an invariant
integral on $P(\mathscr{G})$. Therefore, $P(\mathscr{G})$ defines a partial compact quantum group $\mathscr{G}$ underlying $(A,\Delta)$. 


Finally, suppose that also $\mathscr{G}'$ is a second partical compact quantum group underlying $(A,\Delta)$.   Given  a unitary rcfd representation $\mathscr{X}$ of $\mathscr{G}'$ with underlying Hilber space $\mathscr{H}$, the sum
\begin{align*}
  X=\sum_{k,l,m,n} \in M(P(\mathscr{G}) \otimes \mathcal{B}_{00}(\mathscr{H})),
\end{align*}
being a partial isometry, extends to a multiplier of $A \otimes \mathcal{K}(\mathscr{X})$ which is a unitary corepresentation of $(A,\Delta)$. Therefore, the matrix coefficients of all  unitary rcfd representations of $\mathscr{G}'$ are contained in $P(\mathscr{G})$, and since $P(\mathscr{G'})$ is spanned by all such matrix coefficients by Proposition \ref{LemSpan},  we get $P(\mathscr{G'}) \subseteq P(\mathscr{G})$. Using the Schur orthogonality relations \ref{cor:schur-2}, we can conclude from the density of $P(\mathscr{G'})$ in $A$ that up to equivalence, every  irreducible unitary corepresentation of $(A,\Delta)$ arises from a representation of  $\mathscr{G'}$ as above, and hence  $P(\mathscr{G'}) = P(\mathscr{G})$.
\end{proof}
\subsection{$C^{*}$-algebraic completions of partial compact quantum groups}


 Let $\mathscr{G}$ be an  $I$-partial compact quantum group. We  first show that the  underlying $*$-algebra $P(\mathscr{G})$  has an enveloping  $C^{*}$-algebra and that the invariant integral on $P(\mathscr{G})$ admits  a GNS-construction, leading to a sequence of  $*$-algebras
\begin{align*}
P(\mathscr{G}) \hookrightarrow \CuG \twoheadrightarrow
  \CrG\hookrightarrow
\LGinf \hookrightarrow {\cal B}(\LGtwo).
\end{align*}
Then, we lift the comultiplication of $P(\mathscr{G})$ to $\CuG$ and to $\CrG$ and show that one obtains $C^{*}$-algebraic partial compact quantum groups.  

\textcolor{red}{Remark: done in other paper}

\begin{Lem}\label{LemUniBound}
 Let $H$ be a pre-Hilbert space with completion $\Hsp$. Let $\pi: P(\mathscr{G})\rightarrow \End(H)$ be a $*$-homomorphism into the $*$-algebra of adjointable endomorphisms of $H$. Then $\pi$ extends uniquely to a $*$-homomorphism $\pi: P(\mathscr{G})\rightarrow B(\Hsp)$, and for each $a\in P(\mathscr{G})$ there exists $M_a>0$, independent of $\pi$, such that $\|\pi(a)\|\leq M_a$.
\end{Lem} 

\begin{proof}
By Lemma \ref{LemSpan}, it suffices to prove that $\pi(a)$ is $\pi$-uniformly bounded for $a\in P(\mathscr{G})$ of the form \[a=(\id \otimes \omega_{\xi,\eta})(\Gr{X}{k}{l}{m}{n}),\]
where $\mathscr{X}$ is a unitary representation of
$\mathscr{G}$ on some rcfd $I^{2}$-graded Hilbert space
$\mathcal{K}$ and $\xi\in \Gru{\mathcal{K}}{k}{l}$, $\eta\in
\Gru{\mathcal{K}}{m}{n}$.  By Definition \ref{DefUniCorep}, we deduce
  \begin{equation}\label{EqUnitary}
    \sum_{p}(\Gr{X}{p}{l}{m}{n})^{*} \Gr{X}{p}{l}{m}{n}  = \UnitC{l}{n}
    \otimes \id_{\Gru{\mathcal{K}}{m}{n}}.
  \end{equation}

As $\pi(\UnitC{l}{n})$ is a self-adjoint projection, it extends uniquely to an element in $B(\Hsp)$. From  $\eqref{EqUnitary}$, it then follows that each $(\pi\otimes \id)(\Gr{X}{p}{l}{m}{n})$ can be extended to a contraction in $B(\Hsp\otimes \GrDA{\mathcal{K}}{m}{n},\Hsp\otimes \GrDA{\mathcal{K}}{p}{l})$. We hence conclude that
 \[\|\pi(a)\| \leq \| \xi\| \|\eta\| \|(\pi \otimes \id)(\Gr{X}{k}{l}{m}{n})\| \leq
    \|\xi\|\|\eta\|. \qedhere \]  
\end{proof} 


\begin{Prop}
Let $\mathscr{G}$ be an $I$-partial compact quantum group with underlying
$*$-algebra $P(\mathscr{G})$. Then
  there exist a $C^{*}$-algebra $C^{u}_{0}(\mathscr{G})$ and a
  $*$-homomorphism \[\pi_{u} \colon P(\mathscr{G}) \to
  C^{u}_{0}(\mathscr{G}) \] such that any $*$-homomorphism of
  $P(\mathscr{G})$ into a $C^{*}$-algebra factorizes uniquely through
  $\pi_{u}$.
\end{Prop}
\begin{proof}
For each $a \in P(\mathscr{G})$, the supremum
\begin{align*} 
  |a|_{u}&:= \sup \{ \|\pi(a)\| : \pi \text{ is a $*$-homomorphism from } P(\mathscr{G})
  \text{ into some $C^{*}$-algebra } B\}
\end{align*}
is finite by from Lemma \ref{LemUniBound}. Therefore, $|\cdot |_{u}$ defines a $C^{*}$-seminorm on
$P(\mathscr{G})$ and we can take $C^{u}_{0}(G)$ to be the associated
separated completion. 
\end{proof}

We now construct the reduced $C^{*}$-algebra $\CrG$ and the von Neumann algebra $\LGinf$, using the invariant integral $\phi$ of  $\mathscr{G}$. This functional is faithful  by the remark following \cite[Corollary 2.16]{DCT1}, so that the formula
\begin{align*}
  \langle a|b\rangle :=\phi(a^{*}b)
\end{align*}
 defines an inner product on $P(\mathscr{G})$.
We denote by $L^{2}(\mathscr{G})$ the associated completion and by
$\Lambda \colon P(\mathscr{G}) \to \LGtwo$ the natural embedding.

\begin{Prop} \label{prop:gns} Let $\mathscr{G}$ be a partial compact
  quantum group with underlying total $*$-algebra $P(\mathscr{G})$ and
  associated Hilbert space $\LGtwo$. Then there exists a unique
  $*$-homomorphism \[\pi_{\red}\colon P(\mathscr{G}) \to {\cal B}(\LGtwo)\]
  such that $\pi_{\red}(a)\Lambda(b)=\Lambda(ab)$ for all $a,b\in
  P(\mathscr{G})$, and $\pi_{\red}$ is faithful.
\end{Prop}
\begin{proof} The existence of $\pi_{\red}$ follows immediately from Lemma \ref{LemUniBound}, and the faithfulness of $\pi_{\red}$ follows from the faithfulness of $\phi$. 
  %Let $a,c \in P(\mathscr{G})$. Then the formula $x \mapsto \langle
%\Lambda(c) | x\Lambda(a)\rangle$ defines a bounded linear functional
  %$\omega_{\Lambda(c),\Lambda(a)}$ on ${\cal B}(\LGtwo)$ and a
  %straightforward computation shows that
  %\begin{align} \label{eq:vn-slice}
  %  (\omega_{\Lambda(c),\Lambda(a)}\otimes \id)(V)\Lambda(b) =
%    \Lambda(\varphi(c^*a_{(1)})a_{(2)}b)
  %\end{align}
 % for all $b\in P(\mathscr{G})$. Hence, left multiplication by
 %$\varphi(c^*a_{(1)})a_{(2)}$ extends to a bounded linear operator on
 % $\LGtwo$.  But elements of this form span $P(\mathscr{G})$ because
 % $\phi$ is normalized and $(P(\mathscr{G})\otimes
 % 1)\Delta(P(\mathscr{G})) = (P(\mathscr{G})\otimes
 % P(\mathscr{G}))\Delta(1)$ by \cite[Proposition 1.9]{DCT1} and
 % \cite[Theorem 6.8]{Boh1}.
\end{proof}
%We call $(\LGtwo,\Lambda,\pi_{\red})$ the \emph{GNS-construction associated
%  to $\mathscr{G}$}. 
% Whenever convenient, we drop the symbol $\pi_{\red}$.
We denote by $\CrG \subseteq {\cal B}(\LGtwo)$ the $C^{*}$-algebra  generated by the image
of $\pi_{\red}$. Then the $*$-homomorphism
\begin{align*}
  \pi_{\red} \colon P(\mathscr{G}) \to \CrG
\end{align*}
 factorizes through $\pi_{u}$, and the latter is injective because the former was:
\begin{Cor}
  Let $\mathscr{G}$ be a partial compact quantum group with underlying
  total $*$-algebra $P(\mathscr{G})$. Then the
universal $*$-homomorphism $\pi_{u} \colon P(\mathscr{G}) \to\CuG$ is injective.
\end{Cor}



Next, we lift  the comultiplication from $P(\mathscr{G})$  to $\CuG$.
% First note that for any $k\in I$, the sums $\sum_l \pi_u(\UnitC{k}{l}$ and $\sum_l \pi_u(\UnitC{l}{k})$ converge strictly inside $C^u_0(\mathscr{G})$, as they are sums of mutually orthogonal projections in $C^{u}_{0}(\mathscr{G})$ that converge strictly on a dense subalgebra. We will denote the resulting elements by $\pi_u(\lambda_k)$ and $\pi_u(\rho_k$ 

\begin{Lem}\label{lemma:delta-u}
% Let $\mathscr{G}$ be a partial compact quantum group with underlying
% $*$-algebra $P(\mathscr{G})$.
%Then 
For each $a \in P(\mathscr{G})$, the sum
\begin{align}\label{eq:delta-u}
   \sum_{p,q} (\pi_{u} \otimes \pi_{u})(\Delta_{pq}(a))
\end{align}
converges strictly in $M(C^{u}_{0}(\mathscr{G}) \otimes C^{u}_{0}(\mathscr{G}))$, and there exists a  unique $*$-homomorphism
\begin{align*}
  \Delta^{u}\colon
  C^{u}_{0}(\mathscr{G}) \to M(C^{u}_{0}(\mathscr{G}) \otimes
  C^{u}_{0}(\mathscr{G}))
\end{align*}
 mapping $\pi_{u}(a)$ to this sum for each $a\in P(\mathscr{G})$.
 \end{Lem}
\begin{proof}
Let $\Hsp_u^{(2)}$ carry a non-degenerate faithful $^*$-representation of $C_0^u(\mathscr{G})\otimes C_0^u(\mathscr{G})$, and define $H_u^{(2)}$ as the algebraic sum of all $(\pi_u(\UnitC{k}{m}) \otimes \pi_u(\UnitC{l}{n}))\Hsp_u^{(2)}$. Then $H_u^{(2)}$ is a pre-Hilbert space with $\Hsp_u^{(2)}$ as its closure, and as $P(\mathscr{G})\otimes P(\mathscr{G})$ has local units, it is clear that the representation $\pi_u\otimes \pi_u$ of $P(\mathscr{G})\otimes P(\mathscr{G})$ extends to a $ *$-representation of $M(P(\mathscr{G})\otimes P(\mathscr{G}))$ on $H_u^{(2)}$ by adjointable endomorphisms, which we will denote by the same symbol.

By Lemma \ref{LemUniBound}, we can extend $(\pi_u\otimes \pi_u)\Delta$ to a $^*$-representation $\Delta^u$ of $C_0^u(\mathscr{G})$ on $\Hsp_u^{(2)}$, and
then 
\begin{equation}\label{EqParts}
    (\pi_u(\UnitC{k}{p}) \otimes
    \pi_u(\UnitC{r}{m}))\Delta^{u}(\pi_{u}(a))(\pi_u(\UnitC{l}{q})\otimes \pi_u(\UnitC{s}{n})) =\delta_{p,r}\delta_{s,q}
    (\pi_{u}  \otimes \pi_{u})(\Delta_{pq}(a))
  \end{equation}
  for all $a \in \Gr{P(\mathscr{G})}{k}{l}{m}{n}$ and $p,q,r,s\in I$.
 As $(\pi_u\otimes \pi_u)\Delta(a)$ for $a\in P(\mathscr{G})$ defines a multiplier of $(\pi_u\otimes \pi_u)(P(\mathscr{G})\otimes P(\mathscr{G}))$, we conclude that the range of $\Delta^u$ is in $M(C_0^u(\mathscr{G})\otimes C_0^u(\mathscr{G}))$.
Since $\underset{k,l,m,n}{\sum} \pi_u(\UnitC{k}{m})\otimes \pi_u(\UnitC{l}{n})$ converges strictly to the unit in $M(C_0^u(\mathscr{G})\otimes C_0^u(\mathscr{G}))$, we can conclude that  for each $a\in P(\mathscr{G})$, the sum \eqref{eq:delta-u} converges strictly to $\Delta^{u}(\pi_{u}(a))$.
\end{proof}

\begin{Prop}\label{prop:universal-cpcqg}
  Let $\mathscr{G}$ be a partial compact quantum group. Then $(C^{u}_{0}(\mathscr{G}),\Delta^{u})$  is a $C^{*}$-partial compact quantum group. If we identify $P(\mathscr{G})$ with its image in $C^{u}_{0}(\mathscr{G})$ under $\pi_{u}$, then $\mathscr{G}$ is the unique dense partial compact quantum group in $(C^{u}_{0}(\mathscr{G}),\Delta^{u})$.
\end{Prop}
\begin{proof}
Let us check that $(C^{u}_{0}(\mathscr{G}),\Delta^{u})$ satisfies the conditions in \ref{DefCpcqg}.  Relations (U1), (U2), (U3) and (C) follow from (PB4), (PA), (PB5) and (PB1), respectively. The relations \eqref{eq:canonical-maps} show that
$\Delta(P(\mathscr{G}))(1\otimes P(\mathscr{G})) = \Delta(1)(P(\mathscr{G}) \otimes P(\mathscr{G}))$ and
$(P(\mathscr{G}) \otimes 1)\Delta(P(\mathscr{G})) = (P(\mathscr{G}) \otimes P(\mathscr{G}))\Delta(1)$, which implies (D1) by density. It remains to check (D2). By \ref{LemSpan}, $P(\mathscr{G})$ is spanned by the matrix entries of irreducible unitary corepresentations. But for the matrix coefficient $(\Gr{X}{k}{l}{m}{n})_{pj}$ of  a irreducible unitary corepresentation, we have \[ (\Gr{X}{k}{l}{m}{n})_{pj} \in \sum_{q,r,s} \phi((\Gr{X}{m}{n}{r}{s})_{jq}A) (\Gr{X}{r}{s}{m}{n})_{qj} =(\phi(\,\cdot\, A)\otimes \id)\Delta((\Gr{X}{m}{n}{m}{n})_{jj})\] by the orthogonality relations \cite[Corollary 2.22]{DCT1}. 
\end{proof}
The counit can be lifted as follows:
\begin{Prop}
  Let $I$ be a set and let $\mathscr{G}$ be an $I$-partial compact quantum group. Then there exists a unique $*$-homomorphism $\epsilon_{u} \colon \CuG \to {\cal B}(l^{2}(U))$ such that 
  \begin{align*}
    \epsilon_{u}(\pi_{u}(a)) = \epsilon(a) e_{kl} \quad \text{ for } a \in \Gr{P(\mathscr{G})}{k}{l}{m}{n},
  \end{align*}
   where the $(e_{kl})_{k,l\in I}$ denote the canonical matrix units in ${\cal B}(l^{2}(I))$.
\end{Prop}
\begin{proof}
  We only need to check that the formula above defines a  $*$-homomorphism from $P(\mathscr{G})$ to ${\cal B}(l^{2}(I))$,   and this follows easily from (PB2), (PB6) and Definition \ref{def:pcqg} (3).
\end{proof}
% The existence of the above coproduct can be used to construct tensor products of C$^*$-representations of $\CuG$. For $(\Hsp,\pi)$ a $^*$-representation of $\CuG$, we write $p_{\pi}\in B(\Hsp)$ for the projection onto the closure of $\pi(\CuG)\Hsp$. We write $\GrLA{\Hsp}{k}{l} = \pi(\UnitC{k}{l})\Hsp$, which gives a direct sum decomposition of $p_{\pi}\Hsp$.  When $(\mathcal{K},\pi')$ is another $^*$-representation, we will write $\pi \boxtimes \pi'$ for the $^*$-representation of $\CuG$ on $\mathcal{H}\otimes \mathcal{K}$ obtained as \[(\pi \boxtimes \pi')(a)(\xi\otimes \eta) = (\pi\otimes \pi')(\Delta^u(a))(p_{\pi}\xi\otimes p_{\pi'}\eta).\] When $\pi$ and $\pi'$ are non-degenerate, the restriction of $\pi\boxtimes \pi'$ to \[\Hsp\itimes \mathcal{K} := p_{\pi \boxtimes \pi'}(\Hsp\otimes \mathcal{K}) = (\pi \boxtimes \pi')(\Delta^u(1))(\Hsp\otimes \mathcal{K})= \oplus_l ({}_l\Hsp\otimes {}^l\mathcal{K})\] will be written as $\pi\iboxtimes \pi'$. Note that, by definition, $\pi\iboxtimes \pi'$ is a non-degenerate $^*$-representation, i.e.~ $p_{\pi\iboxtimes \pi'} = 1$. 

% \begin{Def}\label{DefTenProd}  Let $\mathscr{G}$ be a partial compact quantum group, and let $(\Hsp,\pi)$ and $(\mathcal{K},\pi')$ be two non-degenerate $^*$-representations of $\CuG$. Then the non-degenerate $^*$-representation $(\Hsp\itimes \mathcal{K},\pi\iboxtimes \pi')$ will be called the \emph{tensor product representation} of $\pi$ and $\pi'$. 
% \end{Def} 
%However, it is possible that this is a representation on the zero Hilbert space $\{0\}$!
%We further write $H=H_{\pi}$ for the linear span of all $\GrLA{\Hsp}{k}{l}$.


To lift the comultiplication from $P(\mathscr{G})$ to $\CrG$, we use an analogue of the partial isometry constructed in Lemma \ref{lem:reg-corep-pi}.
\begin{Lem} \label{lemma:partial-isometry}
  Let  $\pi$ be a non-degenerate $*$-representation of  $P(\mathscr{G})$ on a Hilbert space $\mathcal{H}$. Then there exists  a unique partial isometry $V_{\pi}$ on $\LGtwo \otimes \mathcal{H}$  such that
  \begin{align*}
    \tilde{V}_{\pi}(\Lambda(a) \otimes \pi(b)\xi) = \Lambda(a_{(1)}) \otimes
    \pi(a_{(2)}b) \xi
  \end{align*}
  for all $a,b\in P(\mathscr{G})$ and $\xi \in \mathcal{H}$. Its domain and range projection are given by
\begin{align*}
  \tilde{V}_{\pi}^{*}\tilde{V}_{\pi}(\Lambda(a)\otimes \xi) &= \sum_{l} \Lambda(a\rho_{l}) \otimes \pi(\rho_{l})\xi, &
  \tilde{V}_{\pi}\tilde{V}_{\pi}^{*}(\Lambda(a)\otimes \xi) &= \sum_{l}  \Lambda(\rho_{l}a) \otimes \pi(\lambda_{l})\xi.
\end{align*}
\end{Lem}
\begin{proof}
  Let $a,b \in P(\mathscr{G})$. Since $\Delta$ is a $*$-homomorphism and $\phi$ is
invariant,
  \begin{align*}
    \langle \Lambda(a_{(1)}) \otimes
    \pi(a_{(2)}b)\xi|\Lambda(a'_{(1)}) \otimes
    \pi(a'_{(2)}b')\xi'\rangle &=
    \phi(a_{(1)}^{*}a'_{(1)})\langle \xi|\pi(b^{*}a_{(2)}^{*}a'_{(2)}b')\xi'\rangle \\
    &= \sum_{p}
    \langle \xi|\pi(b^{*}\rho_{p}\phi(\rho_{p}a^{*}a'\rho_{p})\rho_{p}b')\xi'\rangle \\
    & =\sum_{p} \langle\Lambda(a\rho_{p}) \otimes \pi(\rho_{p}b)\xi |
    \Lambda(a'\rho_{p}) \otimes \pi(\rho_{p}b')\xi'\rangle.
  \end{align*}
  The map $\Lambda(a) \otimes \pi(b)\xi \mapsto \Lambda(a_{(1)})
  \otimes \pi(a_{(2)}b)\xi$ therefore extends to a partial isometry
  $\tilde{V}_{\pi}$ such that $V^{*}_{\pi}\tilde{V}_{\pi}$ has the claimed
  form. Equation  \eqref{eq:canonical-maps} implies
 \begin{align*}
 \tilde{V}_{\pi}(\Lambda(a_{(1)}) \otimes \pi(S(a_{(2)})b)\xi)   &=
 \sum_{p} \Lambda(\rho_{p}a) \otimes \pi(\lambda_{p}b)\xi,
 \end{align*}
whence the formulas for the range projection $\tilde{V}_{\pi}\tilde{V}_{\pi}^{*}$ follows.
\end{proof}
\begin{Lem} \label{lemma:delta-r}
 There exists a  unique injective $*$-homomorphism
\begin{align*}
  \Delta^{r}\colon
  \CrG \to M(\CrG \otimes
  \CrG)
\end{align*}
such that $\Delta^{r}(\pi_{\red}(a)) =   \sum_{p,q} (\pi_{\red} \otimes \pi_{\red})(\Delta_{pq}(a))$ for all $a\in P(\mathscr{G})$.
\end{Lem}
\begin{proof}
  First, note that the sum above converges strictly because the sum  \eqref{eq:delta-u} does.
  Taking  $\pi=\pi_{\red}$ in the preceding lemma, we obtain a partial isometry
$\tilde V:=\tilde{V}_{\pi_{\red}}$ on $\LGtwo \otimes \LGtwo$, and a straightforward calculation shows that the map
  \begin{align*}
\Delta^{r} \colon \CrG \to {\cal B}(\LGtwo \otimes \LGtwo), \ x
  \mapsto \tilde{V}(x \otimes 1)\tilde{V}^{*}   
  \end{align*}
  has the desired properties. For example it is injective because
  $\Delta^{r}(x)=0$ implies $x\otimes 1=0$ on  $\tilde{V}^{*}\tilde{V}(L^{2}(\mathscr{G})
  \otimes L^{2}(\mathscr{G}))$ and hence $x=0$ on $\bigoplus_{k} \Lambda(P(\mathscr{G})\rho_{k})$ which is dense in $L^{2}(\mathscr{G})$.
\end{proof}
Now, the same arguments as in the proof of Proposition \ref{prop:universal-cpcqg} imply:
\begin{Prop} \label{prop:reduced-cpcqg}
  Let $\mathscr{G}$ be a partial compact quantum group. Then $(\CrG,\Delta^{r})$  is a $C^{*}$-partial compact quantum group. If we identify $P(\mathscr{G})$ with its image in $\CrG$ under $\pi_{\red}$, then $\mathscr{G}$ is the unique dense partial compact quantum group in $(\CrG,\Delta^{r})$.
\end{Prop}

% Taking  $\pi=\pi_{\red}$, we obtain a partial isometry
% $\tilde V:=\tilde{V}_{\pi_{\red}}$ on $\LGtwo \otimes \LGtwo$. Write
% \begin{align*}
%   \overline{E} &:=\tilde{V}\tilde{V}^{*} = \sum_{k} \rho_{k} \otimes \lambda_{k} \in
%   \mathcal{B}(\LGtwo \otimes \LGtwo).
% \end{align*}
% \begin{Prop} \label{prop:vn-delta} Let $\mathscr{G}$ be a partial
%   compact quantum group with underlying total $*$-algebra
%   $P(\mathscr{G})$ and associated GNS-construction
%   $(\LGtwo,\Lambda,\pi_{\red})$. Then there
%   exists a unique normal, faithful $*$-homomorphism $\vnDelta \colon
%   \LGinf \to \LGinf \vntimes \LGinf$ such that
%   \begin{align*}
%     \vnDelta(\pi_{\red}(a)) (\Lambda(b) \otimes \Lambda(c)) =
%     \Lambda(a_{(1)}b) \otimes \Lambda(a_{(2)}b) 
%   \end{align*}
%   for all $a,b,c \in P(\mathscr{G})$. Moreover, $\vnDelta(1)=\vnE$ and
%   $\vnDelta(C^{r}_{0}(\mathscr{G})) \subseteq \vnE M(\CrG \otimes
%   \CrG)\vnE$.
% \end{Prop}
% \begin{proof}
%   Uniqueness is clear. To prove existence, one easily verifies that
%   the map
%   \begin{align*}
%  \vnDelta \colon \LGinf \to {\cal B}(\LGtwo \otimes \LGtwo), \ x
%   \mapsto \tilde{V}(x \otimes 1)\tilde{V}^{*}   
%   \end{align*}
%   has the desired properties. For example it is faithful because
%   $\vnDelta(x)=0$ implies $x\otimes 1=0$ on $\tilde{V}^{*}\tilde{V}(L^{2}(\mathscr{G})
%   \otimes L^{2}(\mathscr{G}))$ and hence $x=0$ on $\bigoplus_{k}
%   \rho_{k}^{\op}L^{2}(\mathscr{G})=L^{2}(\mathscr{G})$.
% \end{proof}



% The universal property of $\CuG$ implies that the modular automorphism
% group $\sigma$ and the scaling group $\tau$ for real parameters and
% the unitary antipode $R$ of $\mathscr{G}$ introduced after Corollary
% \ref{cor:rep-characters} lift to one-parameter groups
% $\tau^{u},\sigma^{u}$ and a $*$-anti-automorphism $R_{u}$ of $\CuG$,
% that is, $\tau^{u}_{t} \circ \pi_{u}= \pi_{u} \circ \tau_{t}$,
% $\sigma^{u}_{t} \circ \pi_{u} = \pi_{u} \circ \sigma_{t}$, $R_{u}
% \circ \pi_{u} = \pi_{u} \circ R$. Corollary \ref{cor:rep-characters}
% 5.\ and A.1 in \cite{} [Takesaki:2] imply that elements in
% $\pi_{u}(A)$ are analytic for $\tau^{u}$ and $\sigma^{u}$; in
% particular, $\tau^{u}$ and $\sigma^{u}$ are strongly continuous.


\subsection{Woronowicz's characters of a partial compact quantum group}


Given a partial compact quantum group $\mathscr{G}$, we construct analogues of Woronowicz's characters for compact quantum groups which provide   measures for the deviation  of the antipode from being involutive and of the invariant integral from being a trace, and   descriptions of several structure maps  of the measured quantum groupoid $L^{\infty}(\mathscr{G})$ in the next subsection.


Suppose $a\in \Gr{A}{k}{l}{m}{n}$ for some partial bialgebra $\mathscr{A}$. Then for $\omega \in \Hom_{\C}(A,\C)$, we can define
\begin{align*}
  \omega \aste{p,q} a
&:= (\id \otimes \omega) (\Delta_{pq}(a)), & a \aste{r,s}
\omega&:=(\omega \otimes \id)(\Delta_{rs}(a)).\end{align*} 
Clearly we can define
\begin{align*} \omega \aste{p,q} a \aste{r,s}
\omega'&:= (\omega \aste{p,q} a)\aste{r,s} \omega' = \omega \aste{p,q}(a \aste{r,s} \omega').\end{align*}
When $\omega$ has support on the $A(K)$ with $K_u=K_d$, we can write, for $a\in \Gr{A}{k}{l}{m}{n}$, \[\omega\ast a := \sum_{p,q} \omega\aste{p,q}a = \omega\aste{m,n}a,\quad  a\ast \omega = \sum_{r,s} a\aste{r,s}\omega = a\aste{k,l}\omega.\] 

We shall say that an entire function $f$ has \emph{exponential growth
  on the right half-plane} if there exist $C,d>0$ such that $|f(x+iy)|\leq
C\mathrm{e}^{dx}$  for all $x,y\in \R$ with $x>0$. 

\begin{Theorem} \label{thm:rep-characters} Let $\mathscr{G}$ be a
  partial compact quantum group with invariant integral $\phi$.  Then there exists a unique
  family of linear functionals $f_{z} \colon P(\mathscr{G})\to \C$ such that
\begin{enumerate}
  \item $f_z$ vanishes on $A(K)$ when $K_u\neq K_d$.
  \item for each $a\in A$, the function $z\mapsto f_{z}(a)$ is entire
    and of exponential growth on the right half-plane.
  \item $f_{0} = \epsilon$ and $(f_{z} \otimes f_{z'}) \circ 
    \Delta= f_{z+z'}$ for all $z,z' \in \C$.
  \item $\phi(ab)=\phi(b(f_{1} \ast a \ast f_{1}))$ for all $a,b\in A$.
  \end{enumerate}
  This family furthermore satisfies
  \begin{enumerate}\setcounter{enumi}{4}
  \item $f_z(ab) = f_z(a)f_z(b)$ for $a\in A(K)$ and $b\in A(L)$ with $K_r = L_l$. 
  \item $S^{2}(a)=f_{-1} \ast a \ast f_{1}$ for all $a\in A$.
  \item $f_{z}(\UnitC{l}{n})=\delta_{l,n}$, $f_{z} \circ S = f_{-z}$   and $\ast \circ f_{z} \circ \ast= f_{-\overline{z}}$.
\end{enumerate}
\end{Theorem}


Note that condition (3) is meaningful by condition (1).

\begin{proof}
Uniqueness and property (5) follow by the same arguments as in the case of compact quantum groups, see \cite[Theorem 5.6]{W1}.

  % We first prove uniqueness.  Assume that $(f_{z})_{z}$ is a family of  functionals satisfying (1)--(4).  Since $\phi$ is faithful, the map  $\sigma\colon a \mapsto f_{1} \ast a \ast f_{1}$ is uniquely  determined by $\phi$, and one easily sees that it is a homomorphism. Using
  % (3), we find that $\epsilon \circ \sigma^n=f_{2n}$, which uniquely determines these functionals. Using (2) and the  fact that every entire function of exponential growth on the right  half-plane is uniquely determined by its values at $\N \subseteq \C$, we can conclude that the family $f_{z}$ is uniquely determined. Moreover, since the property (5) holds for $z = 2n$, we also conclude by the same argument as above that it holds for all $z\in \C$.
  Let us prove existence.  By   \cite[Theorem 2.21]{DCT1}, there exists 
for every irreducible unitary rcfd corepresentation  $(V,\mathscr{X})$
a unique non-zero positive operator $F$ implementing a morphism from $(V,\mathscr{X})$ to   $(V, \dual{\dual{\mathscr{X}}\!{}})$, where
 $\dual{\dual{\mathscr{X}}\!{}}$ is given by  $\Gr{(\dual{\dual{X}{}\!})}{k}{l}{m}{n} = (S^{2}\otimes \id)(\Gr{X}{k}{l}{m}{n})$, that is scaled such that
    \begin{align*}
       \sum_{k} \Tr(\Gr{F}{}{-1}{k}{l}) = \sum_{n}
      \Tr(\Grd{F}{m}{n}) =:d_{F}
    \end{align*}
    for all $l \in \alpha,n\in \beta$, where $(\alpha,\beta)$ is the hyperobject support of $\mathscr{X}$. 
By Proposition \ref{LemSpan}, we can   define for each $z\in \C$ a functional $f_{z} \colon A \to \C$ such  that for every such corepresentation  $(V,\mathscr{X})$ and morphism $F$,
    \begin{align*}
      f_{z}((\id \otimes \omega_{\xi,\eta})(\Gr{X}{k}{l}{m}{n})) &=
      \delta_{k,m}\delta_{l,n} \cdot
      \omega_{\xi,\eta}((\Grd{F}{k}{l})^{z}) \quad \text{for all }
      \xi \in \Gru{V}{k}{l},\eta \in
      \Gru{V}{m}{n},
    \end{align*}
    or, equivalently,
    \begin{align*}
      (f_{z} \otimes \id)(\Gr{X}{k}{l}{m}{n}) =
      \delta_{k,m}\delta_{l,n} (\Grd{F}{k}{l})^{z}.
    \end{align*}
    Then (1) and (2) hold by construction. We show that the
    $(f_{z})_{z}$ satisfy the assertions (3)--(7).
    %We have already argued that (5) is satisfied
    %$f_{z}$ is a character. 
    Throughout the following arguments, let    $(V,\mathscr{X})$ and $F=F_{\mathscr{X}}$ be as above.

    We first prove property (3). By definition of a corepresentation,
    \begin{align*}
      (f_{0}  \otimes \id)(\Gr{X}{k}{l}{m}{n}) &=
      \delta_{k,m}\delta_{l,n} \id_{\Gru{V}{k}{l}} =
      (\epsilon \otimes \id)(\Gr{X}{k}{l}{m}{n})
    \end{align*}
    and
    \begin{align*}
      (((f_{z}\otimes f_{z'})\circ \Delta) \otimes
      \id)(\Gr{X}{k}{l}{m}{n}) &=  \delta_{k,m}\delta_{l,n}(f_{z} \otimes f_{z'} \otimes
      \id)\big((\Gr{X}{k}{l}{k}{l})_{13}
      (\Gr{X}{k}{l}{k}{l})_{23}\big) \\
      &=  \delta_{k,m}\delta_{l,n}(\Grd{F}{k}{l})^{z}  \cdot (\Grd{F}{k}{l})^{z'} \\
      &= (f_{z+z'} \otimes \id)(\Gr{X}{k}{l}{m}{n}).
    \end{align*}
    Applying slice maps of the form $\id
    \otimes \omega_{\xi,\xi'}$ and invoking Proposition \ref{LemSpan}, this proves (3).

% Again? Check if this has already been used before   
    To check (4), write again $ \Delta^{(2)} = (
    \Delta \otimes \id)\circ  \Delta = (\id \otimes 
    \Delta) \circ \Delta$, and put \[\theta_{z,z'}:=(f_{z'} \otimes \id
    \otimes f_{z})\circ  \Delta^{(2)}.\] 
Then
    \begin{align*}
      (\theta_{z,z'} \otimes \id)(\Gr{X}{k}{l}{m}{n}) &= (f_{z'} \otimes
      \id \otimes f_{z} \otimes
      \id)((\Gr{X}{k}{l}{k}{l})_{14}(\Gr{X}{k}{l}{m}{n})_{24}(\Gr{X}{m}{n}{m}{n})_{34})
      \\
      &= (1 \otimes (\Grd{F}{k}{l})^{z'}) \Gr{X}{k}{l}{m}{n} (1
      \otimes (\Grd{F}{m}{n})^{z}).
    \end{align*}
    We take $z=z'=1$, use \cite[Theorem 2.21 (3)]{DCT1}, where
    now $d_{F}= d_{G}$ by our scaling of $F$, and obtain
    \begin{eqnarray*}
     && \hspace{-2cm} (\phi \otimes \id \otimes
      \id)((\Gr{X}{k}{l}{m}{n})_{12}^{*}((\theta_{1,1} \otimes
      \id)(\Gr{X}{k}{l}{m}{n}))_{13})\\ && =d_{F}^{-1}(\id \otimes
      \Grd{F}{k}{l}) (\id \otimes \Gr{F}{}{-1}{k}{l})
      \Sigma_{klmn} (\id \otimes
      \Grd{F}{m}{n}) \\
      &&=d_{F}^{-1}(\Grd{F}{m}{n} \otimes \id) \Sigma_{klmn} \\
      &&= (\phi \otimes \id \otimes
      \id)((\Gr{X}{k}{l}{m}{n})_{13}(\Gr{X}{k}{l}{m}{n})_{12}^{*}),
    \end{eqnarray*}
    where $\Sigma_{klmn}\colon \Grd{V}{k}{l} \otimes \Grd{V}{m}{n} \to
    \Grd{V}{m}{n} \otimes \Grd{V}{k}{l} $ denotes the flip map.  To
    conclude the proof of assertion (4), apply again slice maps of the
    form $\omega_{\xi,\xi'} \otimes \omega_{\eta,\eta'}$.

    As remarked above, property (5) follows similarly as in the case of compact quantum groups, see \cite[Theorem 5.6]{W1}. 

  The calculation above and the fact that $F$ is a morphism from $\mathscr{X}$ to $\dual{\dual{\mathscr{X}}{}\!}$ imply
    \begin{align*}
      (S^{2} \otimes \id)(\Gr{X}{k}{l}{m}{n}) &= (1
      \otimes\Grd{F}{k}{l})
      \Gr{X}{k}{l}{m}{n}(1 \otimes \Gr{F}{}{-1}{m}{n})
      =(\theta_{-1,1}  \otimes \id)(\Gr{X}{k}{l}{m}{n}).
    \end{align*}
     Assertion (6) follows again by applying slice maps.
    
     Properties (1), (2) and (4)   immediately imply the relation  $f_{z}(\UnitC{k}{m})=\delta_{k,m}$. As both $z\mapsto f_{-z}$ and $z \mapsto f_{z} \circ S$ satisfy the conditions (1)-(4) for $P(\mathscr{G})$ with the opposite product and coproduct (using the partial character property (5) and the invariance of $\phi$ with respect to $S$ \cite[Corollary 2.20]{DCT1}), we find $f_{-z} = f_{z} \circ S$.

Let us now write $\bar{f}_z(a) = \overline{f_z(a^*)}$. Using the       the relations $
      (\Gr{X}{k}{l}{k}{l})^{*}= (S \otimes \id)(\Gr{X}{k}{l}{k}{l})$, $f_{z} \circ S = f_{-z}$ and positivity of $\Grd{F}{k}{l}$, which we may assume by \cite[Proposition 2.29]{DCT1},        we conclude
      \begin{align*}
       (\bar{f}_z \otimes
        \id)(\Gr{X}{k}{l}{k}{l})
&=       \left((f_{z} \otimes
       \id)((\Gr{X}{k}{l}{k}{l})^{*})\right)^{*} \\
& = \left((f_{-z} \otimes \id)(\Gr{X}{k}{l}{k}{l})\right)^{*} 
\\ &  =
 ((\Grd{F}{k}{l})^{-z})^{*} 
 =       (\Grd{F}{k}{l})^{-\overline{z}} = (f_{-\overline{z}}
 \otimes \id)(\Gr{X}{k}{l}{k}{l}),
      \end{align*}
whence $\bar{f}_z = f_{-\overline{z}}$ on $\Gr{{\cal C}(\mathscr{X})}{k}{k}{l}{l}$. Since $f_{z}$ and $f_{-\overline{z}}$ vanish on $\Gr{A}{k}{l}{m}{n}$ if $(k,l) \neq (m,n)$, and the matrix coefficients of unitary corepresentations span $A$, we can conclude $\overline{f}_{z}=f_{-\overline{z}}$.
\end{proof}
% \begin{Cor} \label{cor:rep-characters} Let $\mathscr{A}$ be a partial
%   Hopf algebra with integral $\phi$ and define $\theta_{z,z'} \colon A
%   \to A$ by $a \mapsto f_{z} \ast a \ast f_{z'}$ for each $z,z' \in
%   \C$, where the functionals $f_{z}$ are as in Theorem
%   \ref{thm:rep-characters}. Then for all $z,z',w,w'\in \C$, the
%   following conditions hold:
%   \begin{enumerate}
%   \item $\theta_{z,z'}$ is an algebra automorphism and preserves
%     each subspace $A(K)$; in particular,
%     $\theta_{z,z'}(\lambda_{k}\rho_{m}) = \lambda_{k}\rho_{m}$ for all
%     $k,m\in I$;
%   \item $\theta_{z,z'}\circ \theta_{w,w'} = \theta_{z+w,z'+w'}$;
%   \item $ (\theta_{w,z'} \otimes \theta_{z,-w}) \circ \Delta = \Delta
%     \circ \theta_{z,z'}$, $\epsilon \circ \theta_{z,z'} = f_{z+z'}$,
%     $\theta_{z,z'} \circ S = S \circ \theta_{-z',-z}$ and
%     $\phi \circ \theta_{z,z'} = \phi$;
%   \item for every linear map $\omega \colon A \to \C$ and every $a\in
%     A$, the map $(z,z') \mapsto \omega(\theta_{z,z'}(a))$ is entire.
%   \end{enumerate}
% \end{Cor}
% \begin{proof}
%   All of this follows easily from Theorem \ref{thm:rep-characters}.
% \end{proof}
% Using the two-parameter group $\theta$, we define the \emph{modular
%   automorphism group} $\sigma$, the \emph{scaling group} $\tau$   and
% the \emph{unitary antipode} of a partial compact quantum group $A$ by
% \begin{align} \label{eq:rep-groups}
%   \sigma_{z} &:=\theta_{iz,iz}, & \tau_{z} &:=\theta_{iz,-iz}, & R&:=S
%   \circ \tau_{i/2}.
% \end{align}
% Using Corollary \ref{cor:rep-characters}, one verifies that
% $\sigma,\tau,R$ share all the main relations known for locally compact
% quantum groups and measured quantum groupoids, for example, $\sigma$
% and $\tau$ are complex one-parameter groups of algebra automorphisms
% of $A$, the map $R$ is an anti-automorphism,  $\tau_{t}$ and
% $\sigma_{t}$ are automorphisms for all $t\in \R$, the family  $\tau$ commutes with
% $\sigma$ and with $R$ in the obvious sense, 
%   \begin{gather}
%     \begin{aligned} \label{eq:modular}
%       \phi\circ \sigma_{z} &= \phi \circ \tau_{z} = \phi \circ R =
%       \phi, & \phi(ab) &= \phi(b\sigma_{-i}(a)),
%     \end{aligned}
% \\ \label{eq:scaling-modular-delta}
%     \begin{aligned} 
%     \Delta \circ \tau_{z} &= (\tau_{z} \otimes \tau_{z}) \circ \Delta,
%     & (\tau_{z} \otimes \sigma_{z}) \circ \Delta &= \Delta \circ
%     \sigma_{z} = (\sigma_{-z} \otimes \tau_{z}) \circ \Delta,      
%   \end{aligned} \\
%   \begin{aligned} \label{eq:unitary-antipode}
%     R^{2} &= \id_{A}, & \Delta \circ R &= (R \otimes R) \circ
%     \Delta^{\op}.
%   \end{aligned}
%   \end{gather}






\subsection{Partial compact quantum groups as measured quantum groupoids}

Let $\mathscr{G}$ be a partial compact quantum group with positive invariant integral $\phi$. Using the associated GNS-construction $(\LGtwo, \Lambda,\pi_{\red})$, see Proposition \ref{prop:gns}, we define
the associated von Neumann algebra of $\mathscr{G}$ as
\begin{align*}
  \LGinf := \pi_{\red}(P(\mathscr{G}))'' \subseteq \LGtwo.
\end{align*}
We equip this algebra with the structure of a measured quantum groupoid in the sense of \cite{Les1} and \cite{Eno2}. 

The base algebra will be the von Neumann algebra $l^{\infty}(I)$.
Denote by $\nu$ the normal, faithful and semi-finite weight on $l^{\infty}(I)$ given by
integration with respect to the counting measure on $I$, and by
$\lambda,\rho \colon l^{\infty}(I) \to  {\cal B}(\LGtwo)$
the faithful, normal $*$-homomorphisms given by
\begin{align*}
  \lambda(f)&= \sum_{k} f(k)\pi_{\red}(\UnitC{k}{l}), &
 \rho(f) &= \sum_{k,l} f(l)\pi_{\red}(\UnitC{k}{l}).
\end{align*}

We lift the comultiplication $\Delta$ to a $*$-homomorphism
\begin{align*}
  \vnDelta \colon \LGinf \to \LGinf  \astrl  \LGinf \subseteq {\cal B}(\LGtwo \otimesrl \LGtwo)
\end{align*}
such that $(l^{\infty}(I),\LGinf,\lambda,\rho,\vnDelta)$ becomes a Hopf-von Neumann bimodule in the sense of \cite{Val1}. 
Here, the relative tensor product $  \LGtwo \otimesrl \LGtwo$ has the simple form
\begin{align*}
  \LGtwo \otimesrl \LGtwo \cong
\overline{E}(\LGtwo \otimes \LGtwo), \quad \text{ where } 
  \overline{E} =  \sum_{k,l,m} \pi_{\red}(\UnitC{k}{l}) \otimes \pi_{\red}(\UnitC{l}{m}),
\end{align*}
the relative tensor product of operators $S\in \rho(l^{\infty}(I))'$
and $T \in \lambda(l^{\infty}(I))'$  gets identified with the
compression
\begin{align*}
S \otimesrl T \equiv
  \vnE(S \otimes
  T) = (S \otimes T)\vnE \subseteq {\cal B}(\vnE(\LGtwo
  \otimes \LGtwo)),
\end{align*}
and the fiber product $ \LGinf  \astrl  \LGinf$ takes the form
\begin{align} \label{eq:vn-fiber}
  \begin{aligned}
    \LGinf \astrl \LGinf &= (\LGinf' \otimesrl \LGinf')' \\ &\equiv
    (\vnE(\LGinf' \otimes \LGinf'))' = \vnE(\LGinf \otimes
    \LGinf)\vnE.
  \end{aligned}
\end{align} 
\begin{Prop}\label{prop:hopf-vn-bimodule}
  Let  $\mathscr{G}$ be a partial compact quantum group. Then there exists a unique normal, faithful $*$-homomorphism $\vnDelta \colon \LGinf \to \LGinf \astrl \LGinf$ such that with respect to the isomorphism \eqref{eq:vn-fiber}, 
  \begin{align*}
    \vnDelta(\pi_{\red}(a))  = \sum_{p,q} (\pi_{\red} \otimes \pi_{\red})(\Delta_{pq}(a))
  \end{align*}
for all $a\in A$. The tuple  $(l^{\infty}(I),\LGinf,\lambda,\rho,\vnDelta)$ is a Hopf-von Neumann bimodule.
\end{Prop}
\begin{proof}
The lift $\vnDelta$ of $\Delta$ can be constructed similarly like  $\Delta^{\red}$ using the partial isometry $\tilde V_{\pi}$, see Lemma \ref{lemma:partial-isometry} and Proposition \ref{lemma:delta-r}. It remains to show that
   \begin{align*}
     \vnDelta(\lambda(x)) &= \lambda(x) \otimesrl 1, &
  \vnDelta(\rho(x)) &= 1 \otimesrl \rho(x), &
  (\vnDelta \astrl \id)\circ \vnDelta &= (\id \astrl
    \vnDelta) \circ \vnDelta
\end{align*}
for all $x\in l^{\infty}(I)$.
The first two equations follow immediately from the relation
$\Delta(\UnitC{k}{m})=\lambda_{k} \otimes \rho_{m}$. To verify
the third equation, we identify
\begin{align*}
 \LGinf \astrl \LGinf \astrl \LGinf \cong \vnE^{(2)}(\LGinf
  \otimes \LGinf \otimes \LGinf)\vnE^{(2)},
\end{align*}
where $\vnE^{(2)}=(\vnE \otimes 1)(1 \otimes \vnE)$. Then the two
compositions  become corestrictions of the maps $(\vnDelta \otimes
\id)\circ \vnDelta$ and $(\id \otimes \vnDelta)\circ \vnDelta$,
respectively, which coincide on $\pi_{\red}(P(\mathscr{G}))$ by
co-associativity of $\Delta$. 
\end{proof}

To obtain a measured quantum groupoid, we  need a left-invariant conditional expectation from  $\LGinf$ to $\lambda(l^{\infty}(I))$, and a right-invariant conditional expectation to $\rho(l^{\infty}(I))$. We use the vector states on $\LGinf$ given by
\begin{align*}
  \vnphic{k}{m} \colon x \mapsto \langle \Lambda(\UnitC{k}{m})|x \Lambda(\UnitC{k}{m})\rangle,
\end{align*}
and define a weight $\vnphi$ and operator-valued weights  $T_{\lambda}$ and $T_{\rho}$ 
 by
\begin{align} \label{eq:vn-weights}
  \vnphi(x) &:= \sum_{k,m} \vnphic{k}{m}(x), &
    T_{\lambda}(x)&:= \sum_{k,m}
\vnphic{k}{m}(x)\lambda_{k}, & 
T_{\rho}(x)&:=
    \sum_{k,m} \vnphic{k}{m}(x)\rho_{m}
\end{align}
for all $x\in \LGinf^{+}$. Clearly, these   are normal, (semi-)finite and satisfy
\begin{align}\label{eq:wts-equal}
  \nu
\circ \lambda^{-1} \circ T_{\lambda} = \vnphi = \nu \circ \rho^{-1}
\circ T_{\rho}.
\end{align}
We need to show that $\vnphi$ is faithful, and use  the functionals $(f_{z})_{z\in \C}$ constructed in Theorem \ref{thm:rep-characters}, the theory of Hilbert algebras and Tomita algebras \cite{Taksak2}.

\begin{Lem}\label{lem:analytic}
  Let $c,d \in \R$ and consider the family of maps
  \begin{align*}
    \theta_{z} \colon P(\mathscr{G})\to P(\mathscr{G}), \quad a\mapsto f_{cz} \ast a \ast f_{dz},
  \end{align*}
where $z\in \C$.  Then the following conditions hold.
  \begin{enumerate}
  \item for every $a\in P(\mathscr{G})$ and every linear
    functional $\omega$ on $P(\mathscr{G})$, the map $z\mapsto
    \omega(\theta_{z}(a))$ is analytic;
  \item $\theta_{z}\circ \theta_{z'} = \theta_{z+z'}$ and
    $\theta_{z}\circ \ast = \ast \circ \theta_{\overline{z}}$ for all
    $z,z' \in \C$.
  \item There exists a strongly continuous one-parameter group $\theta^{u}$ on $\CuG$ satisfying $\theta^{u}_{t} \circ \pi_{u} = \pi_{u} \circ \theta_{t}$ for all $t\in \R$. 
  \end{enumerate}
\end{Lem}
\begin{proof}
Assertions (1) and (2)   follow immediately from  Theorem \ref{thm:rep-characters}. Now, (2) implies that $\pi_{u}\circ \theta_{t}$ is a $*$-homomorphism for each $t\in \R$ which therefore lifts to a $*$-homomorphism $\theta^{u}_{t}$ on $\CuG$, and (1) that for each $a\in P(\mathscr{G})$, the map $z\mapsto \pi_{u}(\theta_{z}((a)))$ is weakly analytic and therefore also analytic. Therefore, $\theta^{u}$ is strongly continuous.
\end{proof}
Let us call a family of algebra automorphisms $\theta_{z}$ of $P(\mathscr{G})$, defined for all $z\in \C$ and satisfying conditions 1.\ and 2.\ above,  an \emph{analytic one-parameter group} on $P(\mathscr{G})$.

The preceding result and Theorem \ref{thm:rep-characters}  imply:
\begin{Lem} \label{lem:mod-aut}
  The  formula $\sigma_{z}(a)= f_{iz} \ast a \ast f_{iz}$ defines an analytic one-parameter group $(\sigma_{z})_{z}$ on $P(\mathscr{G})$   satisfying $\phi(\sigma_{z}(a)) = \phi(a)$ and $\phi(ab) = \phi(b\sigma_{-i}(a))$ for all $a,b\in P(\mathscr{G})$.  
\end{Lem}
\begin{Prop} \label{prop:tomita}
  Let $\mathscr{G}$ be a partial compact quantum group with underlying
  $*$-algebra $P(\mathscr{G})$ and associated GNS-construction
  $(\LGtwo,\Lambda,\pi_{\red})$. 
  \begin{enumerate}
  \item  $\Lambda(P(\mathscr{G}))$ is a Tomita algebra with
    respect to the operations and operators given by
    \begin{align*}
      \Lambda(a)\Lambda(b)&=\Lambda(ab), & \Lambda(a)^{*}&=
      \Lambda(a^{*}), & \nabla_{z}\Lambda(a)&=\Lambda(\sigma_{z}(a))
    \end{align*}
    for all $a,b\in P(\mathscr{G})$ and $z\in \C$.
  \item Its
    left von Neumann algebra and its n.s.f.\ weight are $\LGinf$ and
    $\vnphi$, respectively.
  \item The modular operator $\Delta_{\vnphi}$ is
    the closure of $\nabla_{-i}$, and the modular automorphism group
    $\sigma^{\vnphi}$ and modular conjugation $J_{\vnphi}$ are given
    by
    \begin{align*}
      \sigma^{\vnphi}_{t} \circ \pi_{\red} &= \pi_{\red} \circ
      \sigma_{t}  & J_{\vnphi}
      \Lambda(a)&=\Lambda(\sigma_{i/2}(a)^{*})
    \end{align*}
for all $t\in \R$ and $a\in
      P(\mathscr{G})$.
  \end{enumerate}
\end{Prop}
\begin{proof}
 The map $\pi_{\red}(a)\colon \Lambda(b) \to \Lambda(ab)$ is bounded for  each $a \in P(\mathscr{G})$ by Proposition \ref{prop:gns}, and the   involution is pre-closed because for all $a,b \in P(\mathscr{G})$,
  \begin{align*}
    \langle \Lambda(a)|\Lambda(b^{*})\rangle = \phi(a^{*}b^{*}) =
    \phi(b^{*}\sigma_{-i}(a^{*})) = \langle
    \Lambda(b)|\Lambda(\sigma_{-i}(a^{*}))\rangle.
  \end{align*}
  Therefore, $\Lambda(P(\mathscr{G}))$ is a left Hilbert algebra. It  is a Tomita algebra because the  map $z\mapsto \langle \Lambda(a)|\nabla_{z}\Lambda(b)\rangle =
  \phi(a^{*}\sigma_{z}(b))$ is entire for all $a,b\in P(\mathscr{G})$  by Theorem \ref{thm:rep-characters}
  and because
  \begin{gather*}
    \nabla_{z}(\Lambda(a)^{*}) = (\nabla_{\overline{z}}\Lambda(a))^{*}, \qquad
    \langle \Lambda(a)|\nabla_{z}\Lambda(b)\rangle = \langle
    \nabla_{-\overline{z}}\Lambda(a) |\Lambda(b)\rangle, \\ \langle
    \Lambda(a)^{*}|\Lambda(b)^{*}\rangle = \langle \Lambda(b)|\nabla_{-i}\Lambda(a)\rangle
  \end{gather*}
  for all $a,b\in P(\mathscr{G})$ and $z\in \C$ by the lemma above.

 The associated left von Neumann algebra  is $\pi_{\red}(P(\mathscr{G}))''=\LGinf$. Denote the associated weight by $\tilde \phi$. Replacing $\vnphi$ by $\tilde \phi$, the assertions in (3)  follows  from  \cite[Theorem VI.2.2 and its proof]{Taksak2}. It remains to show that $\vnphi = \tilde \phi$. For all $a\in P(\mathscr{G})$, we have $\tilde \phi(\pi_{\red}(a^{*}a))=\langle \Lambda(a)|\Lambda(a)\rangle = \phi(a^{*}a)$ by construction. The formula for $\sigma^{\tilde\phi}$ implies that each projection $\pi_{\red}(\UnitC{k}{l})$ is central for $\tilde \phi$, and since these projections sum up to 1, we get that $\tilde \phi$ is the sum of the bounded normal positive functionals $\Grt{\tilde \phi}{k}{l}$ given by $x^{*}x \mapsto \tilde \phi(\pi_{\red}(\UnitC{k}{l})x^{*}x\pi_{\red}(\UnitC{k}{l}))$. These functionals coincide with the $\vnphic{k}{l}$ on  $\pi_{\red}(P(\mathscr{G}))$,  and on all of $\LGinf$ by density. Therefore, $\tilde \phi = \vnphi$.
\end{proof}
We now arrive at the main result of this subsection.
\begin{Theorem}
  Let $\mathscr{G}$ be an $I$-partial compact quantum group with  associated von Neumann algebra $\LGinf$. Then
  $(l^{\infty}(I),\LGinf, \lambda,\rho,\tilde\Delta,
  T_{\lambda},T_{\rho},\nu)$ is a measured quantum groupoid in the  sense of \cite{Eno2}.
\end{Theorem}
\begin{proof}
  Thanks to Proposition \ref{prop:hopf-vn-bimodule} and equality  \eqref{eq:wts-equal}, it only remains to show that $T_{\lambda}$ and  $T_{\rho}$ are left- and right-invariant in the following sense. We only prove left-invariance of  $T_{\lambda}$, which  means that
  \begin{align*}
   (\id \underset{\nu}{_{\rho}\ast_{\lambda}} \vnphi)(\vnDelta(x)) = T_{\lambda}(x)
  \end{align*}
  for all $x\in \LGinf_{+}$. By definition of $\vnphi$ and $T_{\lambda}$, it suffices to check that for all $k,l,m$,
  \begin{align*}
 \pi_{\red}(\UnitC{m}{k})    (\id \astrl \vnphic{k}{l})(\vnDelta(x)) = \pi_{\red}(\UnitC{m}{k})\vnphic{m}{l}(x).
  \end{align*}
But  for $x\in \pi_{\red}(P(\mathscr{G}))$,  this equations holds by \eqref{eq:integral}, and by density and boundedness of  $\vnphic{k}{l}$ and $\vnphic{m}{l}$, it holds for all $x\in \LGinf_{+}$. 
\end{proof}

%Changed remark;  in our construction, T_{\lambda} and T_{\rho} need not be bounded
\begin{Rem} It is not clear at the moment which measured quantum groupoids of the form 
$(N,M, \lambda,\rho,\Delta,  T_{\lambda},T_{\rho},\nu)$  with $N$ atomic abelian arise from a partial compact quantum group. 
  \end{Rem}
   
The scaling group and unitary antipode of the measured quantum
groupoid above can be described in terms of the following analytic one-parameter group on $P(\mathscr{G})$.
\begin{Lem} \label{lem:scaling}
 The formula $\tau_{z}(a):= f_{-iz} \ast a \ast f_{iz}$ defines an analytic one-parameter group on $P(\mathscr{G})$ satisfying
\begin{align*} 
 S^{2} &= \tau_{-i}, & S \circ
    \tau_{z} &= \tau_{z} \circ S, & (\tau_{z} \otimes \sigma_{z})
    \circ \Delta_{pq} &= \Delta_{pq} \circ \sigma_{z}
  \end{align*}
for all $z,z'\in \C$. The composition $R:=S\circ
\tau_{i/2}$ is a $*$-anti-automorphism of $P(\mathscr{G})$.
\end{Lem}
\begin{proof}
  This follows easily from Lemma \ref{lem:analytic} and Theorem \ref{thm:rep-characters} again.
\end{proof}
\begin{Prop}
  Let $\mathscr{G}$ be an $I$-partial compact quantum group with  associated von Neumann algebra $\LGinf$. Then there exist a unique strongly continuous one-parameter group $\vntau$ and
 a unique $*$-anti-automorphism $\vnR$ of $\LGinf$ such that  $\vntau_{t} \circ \pi_{\red} = \pi_{\red} \circ
\tau_{t}$ for  all $t\in \R$ and
$\vnR \circ \pi_{\red} = \pi_{\red} \circ R$.  These are the scaling group and the unitary antipode of the measured
 quantum groupoid $(l^{\infty}(I),\LGinf, \lambda,\rho,\tilde\Delta,
 T_{\lambda},T_{\rho},\nu)$, respectively.
\end{Prop}
\begin{proof}
  Uniqueness is clear.  Let us prove existence of $\vntau$.
  Invariance of $\phi$ and Theorem \ref{thm:rep-characters} imply $\phi \circ
  \tau_{z} = \phi$ for all $z\in \C$. Therefore, the formula
  $P_{t}\Lambda(a) = \Lambda(\tau_{t}(a))$ defines a one-parameter
  group of unitaries $(P_{t})_{t\in \R}$ on $\LGtwo$. Since  elements of $\Lambda(P(\mathscr{G}))$ are analytic with
  respect to $P$, this one-parameter group is strongly
  continuous.  Therefore, we can define $\vntau$ by  $\vntau_{t}(x)=
  P_{t}xP_{t}^{*}$ for all $x\in \LGinf$ and $t\in \R$.
  

  Next, we prove existence of $\vnR$. We claim that the formula $
  I\Lambda(a) = \Lambda(R(a)^{*})$ defines a conjugate-linear   isometry $I$. Indeed, $\phi(R(a)R(b)^{*})= (\phi\circ R)(b^{*}a)$
  for all $a,b\in P(\mathscr{G})$ and $\phi \circ R=\phi \circ S \circ
  \tau_{i/2} = \phi \circ \tau_{i/2}=\phi$, where we used \cite[
  Corollary 2.20]{DCT1}. Next, $I^{2} = \id$ because $*\circ R \circ *
  \circ R= R^{2}=\id$ by Lemma \ref{lem:scaling} and \cite[Proposition 1.17]{DCT1}. Now, short calculations show that the map $\vnR \colon
  x \mapsto Ix^{*}I$ has the desired properties.

Let us show   that the unitary antipode  $\tilde R$ and the   scaling group   $\tilde\tau$ of the measured quantum groupoid   coincide with $\vnR$ and $\vntau$, respectively.   By \cite[Theorem A.6]{Eno2},
  \begin{align*}
    \tilde R(\id \underset{\nu}{_{\rho} \ast_{\lambda}}
    \omega_{J\Lambda(b),J\Lambda(b)})(\tilde\Delta(\pi_{\red}(a^{*}a))) &= (\id
    \underset{\nu}{_{\rho} \ast_{\lambda}}
    \omega_{J\Lambda(a),J\Lambda(a)})(\tilde\Delta(\pi_{\red}(b^{*}b))),
  \end{align*}
  for all $a,b\in P(\mathscr{G})$.  Let $c=a^{*}a$ and $d=b^{*}b$.  A   short calculation using Proposition \ref{prop:tomita}  (3) and Lemma \ref{lem:mod-aut} shows that the right hand  side is equal to
\begin{align*}
  \pi_{\red} ( d_{(1)}\phi(\sigma_{i/2}(a)d_{(2)}\sigma_{i/2}(a)^{*})) =
  \pi_{\red}(d_{(1)}\phi(\sigma_{i/2}(c)d_{(2)})).
  \end{align*}
By Lemma \ref{lem:strong-invariance},  \ref{lem:scaling} and   \ref{lem:mod-aut},  this is equal to
  \begin{align*}
   \pi_{\red}(S(\sigma_{i/2}(c)_{(1)}\phi(\sigma_{i/2}(c)_{(2)}d))) &=
   \pi_{\red}(S(\tau_{i/2}(c_{(1)})\phi(\sigma_{i/2}(c_{(2)})d))) \\
   &= \pi_{\red}(R(c_{(1)}\phi(\sigma_{i/2}(d)c_{(2)}))) \\
   &=\vnR((\id \underset{\nu}{_{\rho} \ast_{\lambda}}
    \omega_{J\Lambda(b),J\Lambda(b)})(\vnDelta(\pi_{\red}(a^{*}a)))).
  \end{align*}
  Since elements of the form $\pi_{\red}(a^{*}a)$ are dense in
  $\LGinf^{+}$ and $\Lambda(P(\mathscr{G}))$ is dense in $\LGtwo$, we
  can conclude from \cite[Theorem A.7]{Eno2} that elements of the form
  \begin{align*}
  (\id \underset{\nu}{_{\rho}
    \ast_{\lambda}}
  \omega_{J\Lambda(b),J\Lambda(b)})(\tilde\Delta(\pi_{r}(a^{*}a)))  
  \end{align*}
 are  dense in $\LGinf_{+}$. Consequently, $\tilde R=\vnR$. 

Next,  Lemma \ref{lem:scaling} implies $(\vntau_{t} \otimes
  \sigma^{\vnphi}_{t}) \circ \vnDelta = \vnDelta \circ
  \sigma^{\vnphi}_{t}$ for all $t\in \R$.  But by \cite[Theorem
  A.5]{Eno2},
  \begin{align*}
    (\tilde \tau_{t} \astrl \sigma^{\vnphi}_{t}) \circ \tilde \Delta
    &=\tilde \Delta \circ \sigma^{\vnphi}_{t}.
  \end{align*}
Now,  a similar density  argument as above shows that
  $\tilde \tau_{t}=\vntau_{t}$.
\end{proof}

%%% Local Variables: 
%%% mode: latex
%%% TeX-master: "dynamical-SUq-file"
%%% End: 
