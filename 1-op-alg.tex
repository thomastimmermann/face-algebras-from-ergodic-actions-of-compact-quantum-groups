\section{Partial compact quantum groups on the level of operator algebras}


Let $\mathscr{G}$ be a partial compact quantum group. We construct
completions of the underlying $*$-algebra $P(\mathscr{G})$ in the form
of a universal $C^{*}$-algebra $\CuG$, a reduced $C^{*}$-algebra
$\CrG$ and a von Neumann algebra $\LGinf$, the last two acting on a
Hilbert space $\LGtwo$ associated to the invariant integral of
$\mathscr{G}$.  We then lift the comultiplication and the invariant
integral to the level of
operator algebras and show that $\LGinf$ becomes a measured quantum
groupoid in the sense of Lesieur \cite{Les1} and Enock \cite{Eno2}.

The $*$-algebra $P(\mathscr{G})$ has an enveloping $C^{*}$-algebra $\CuG$:
\begin{Prop}
Let $\mathscr{G}$ be a partial compact quantum group with underlying
$*$-algebra $P(\mathscr{G})$. Then
  there exist a $C^{*}$-algebra $C^{u}_{0}(\mathscr{G})$ and a
  $*$-homomorphism $\pi_{u} \colon P(\mathscr{G}) \to
  C^{u}_{0}(\mathscr{G})$ such that every $*$-homomorphism of
  $P(\mathscr{G})$ into some $C^{*}$-algebra factorizes through
  $\pi_{u}$.
\end{Prop}
\begin{proof}
It suffices to show that for each $a \in A$,
\begin{align*} 
  |a|_{u}&:= \sup \{ \|\pi(a)\| : \pi \text{ is a $*$-homomorphism from } P(\mathscr{G})
  \text{ into some $C^{*}$-algebra } B\}
\end{align*}
is finite because then $|\cdot |_{u}$ defines $C^{*}$-seminorm on
$P(\mathscr{G})$ and we can take $C^{u}_{0}(G)$ to be the associated
separated completion.


By Corollary \cite[Proposition 3.31]{DCT1}, we can write each $a\in A$
in the form \[a=(\id \otimes \omega_{\xi,\eta})(\Gr{X}{k}{l}{m}{n}),\]
where $\mathscr{X}$ is a unitary representation of
$\mathscr{G}$ on some sfd $I^{2}$-graded Hilbert space
$\mathcal{H}$ and $\xi\in \Gru{H}{k}{l}$, $\eta\in
\Gru{H}{m}{n}$.  By \cite[Lemma 3.12]{DCT1},
  \begin{align*}
    \sum_{p}(\Gr{X}{p}{l}{m}{n})^{*} \Gr{X}{p}{l}{m}{n}  = \lambda_{l}\rho_{n}
    \otimes \id_{\Gru{H}{m}{n}},
  \end{align*}
where  the sum is  finite because $X$ is sfd. As
  $\pi(\lambda_{k}\rho_{m})\in \C$ is a projection, we can conclude
  \begin{align} \label{eq:corep-estimate}
    \sum_{p} \| (\pi \otimes \id)(\Gr{X}{p}{l}{m}{n})\|^{2} \leq 1
  \end{align}
  and deduce $\|\pi(a)\| \leq \| \xi\| \|\eta\| \|(\pi \otimes \id)(\Gr{X}{k}{l}{m}{n})\| \leq
    \|\xi\|\|\eta\| $. 
\end{proof}
The construction of the reduced $C^{*}$-algebra
$C^{r}_{0}(\mathscr{G})$ will show that the $*$-homomorphism $\pi_{u}
\colon P(\mathscr{G}) \to \CuG$ is injective.

To lift the comultiplication from $P(\mathscr{G})$ to
$C^{u}_{0}(\mathscr{G})$, we will use parts of the following result.


% The universal property of $\CuG$ implies that the modular automorphism
% group $\sigma$ and the scaling group $\tau$ for real parameters and
% the unitary antipode $R$ of $\mathscr{G}$ introduced after Corollary
% \ref{cor:rep-characters} lift to one-parameter groups
% $\tau^{u},\sigma^{u}$ and a $*$-anti-automorphism $R_{u}$ of $\CuG$,
% that is, $\tau^{u}_{t} \circ \pi_{u}= \pi_{u} \circ \tau_{t}$,
% $\sigma^{u}_{t} \circ \pi_{u} = \pi_{u} \circ \sigma_{t}$, $R_{u}
% \circ \pi_{u} = \pi_{u} \circ R$. Corollary \ref{cor:rep-characters}
% 5.\ and A.1 in \cite{} [Takesaki:2] imply that elements in
% $\pi_{u}(A)$ are analytic for $\tau^{u}$ and $\sigma^{u}$; in
% particular, $\tau^{u}$ and $\sigma^{u}$ are strongly continuous.
 \begin{Lem} \label{LemBoundDim} Let $\mathscr{X}$ be a unitary
   representation of $\mathscr{G}$ on an sfd $I$-graded Hilbert space $\Hsp$ with finite hyperobject support, and write $d^X_{kl} = \dim(\Gru{H}{k}{l})$. Then the matrix $D^X = (d^X_{kl})_{k,l}$ defines a bounded operator on $l^2(I)$. 
  \end{Lem}
  \begin{proof} We may assume that $\mathscr{X}$ is irreducible, say
    with left hyperobject support $\alpha$ and right hyperobject support
    $\beta$. Then we can find (a priori not necessarily bounded)
    morphisms \[R: \C^{(I)}\rightarrow \underset{k,l,m}{\oplus} \GrDA{H}{k}{m}\otimes
    \GrDA{H}{m}{l},\qquad \bar{R}: \C^{(I)} \rightarrow \underset{k,l,m}{\oplus}
    \GrDA{\bar{H}}{k}{m}\otimes \GrDA{\bar{H}}{m}{l},\] establishing a
    duality between $\underset{k,l}{\oplus} \GrDA{H}{k}{l}$ and
    $\underset{k,l}{\oplus}\GrDA{\bar{H}}{k}{l}$ inside the category of
    rcfd pre-Hilbert spaces. However, as $R^*R$ is a scalar multiple
    of the projection onto $\C^{(I_{\alpha})}$ by irreducibility, and
    similarly for $\bar{R}$, it follows that $R$ and $\bar{R}$ can be
    completed to bounded operators between the respective Hilbert
    space completions. It then follows from \cite[Lemma A.3.2]{DCY1}
    that 
    \begin{align} \label{eq:dim-estimate}
  \sup_r (\sum_s (d_{rs}^X+d_{sr}^X)) < \infty.    
    \end{align}
 By the Schur test,
    this implies that $D^X$ is bounded.
\end{proof} 


\begin{Prop}
Let $\mathscr{G}$ be a partial compact quantum group with underlying
$*$-algebra $P(\mathscr{G})$. Then  there exists a unique $*$-homomorphism $\Delta^{u}\colon
  C^{u}_{0}(\mathscr{G}) \to M(C^{u}_{0}(\mathscr{G}) \otimes
  C^{u}_{0}(\mathscr{G}))$ such that 
  \begin{align*}
    (\rho_{p} \otimes
    \lambda_{p})\Delta^{u}(\pi_{u}(a))(\rho_{q}\otimes \lambda_{q}) =
    (\pi_{u} \otimes \pi_{u})(\Delta_{pq}(a))
  \end{align*}
  for all $a \in P(\mathscr{G})$ and $p,q\in I$.
\end{Prop}
\begin{proof}
  We need to show that for every $a \in P(\mathscr{G})$, the sum
  $\sum_{p,q} (\pi_{u} \otimes \pi_{u})(\Delta_{pq}(a))$ converges in $M(C^{u}_{0}(\mathscr{G})
  \otimes C^{u}_{0}(\mathscr{G}))$ strictly.  It suffices to prove
  that the matrix of operators 
\[\left((\pi_{u} \otimes
    \pi_{u})(\Delta_{pq}(a))\right)_{k,l}\] is bounded, and using the Schur test as above, it
  suffices to show that
  \begin{align} \label{eq:delta-estimate}
    \sup_{p} \sum_{q} (\|(\pi_{u} \otimes \pi_{u})(\Delta_{pq}(a))\| + \|(\pi_{u} \otimes \pi_{u})(\Delta_{qp}(a))\|) < \infty.
  \end{align}
  As in the proof above, we write
  $a=(\id \otimes \omega_{\xi,\eta})(\Gr{X}{k}{l}{m}{n})$, where $\mathscr{X}$ is a unitary
 representation of $\mathscr{G}$  on some sfd $I^{2}$-graded
  Hilbert space $\mathcal{H}$ with finite hyperobject support, and
  $\xi\in \Gru{H}{k}{l}$, $\eta\in \Gru{H}{m}{n}$. Then
  \begin{align*}
    \Delta_{pq}(a) &= (\id \otimes \id \otimes \omega_{\xi,\eta})((\Delta_{pq} \otimes
    \id)(\Gr{X}{k}{l}{m}{n})) =
    (\id \otimes \id \otimes \omega_{\xi,\eta})((\Gr{X}{k}{l}{p}{q})_{13}(\Gr{X}{p}{q}{m}{n})_{23}),
  \end{align*}
and by \eqref{eq:corep-estimate},
\begin{align*}
  \|(\pi_{u} \otimes \pi_{u})(\Delta_{pq}(a))\| \leq \|\xi\|\|\eta\| \|(\pi_{u} \otimes \id)(\Gr{X}{k}{l}{p}{q})\|
  \|(\pi_{u} \otimes \id)(\Gr{X}{p}{q}{m}{n})\|  \leq \|\xi\|\|\eta\|.
\end{align*}
Now \eqref{eq:delta-estimate} follows from \eqref{eq:dim-estimate}.
\end{proof}



The reduced $C^{*}$-algebra $\CrG$ and the von Neumann algebra
$\LGinf$ will be obtained from a GNS-construction for the invariant
integral $\phi$ of $\mathscr{G}$. Recall that $\phi$ is positive by
assumption and faithful by \cite[Lemma
1.39]{DCT1}.  We define an inner product on $P(\mathscr{G})$ by
\begin{align*}
  \langle a|b\rangle :=\phi(a^{*}b),
\end{align*}
denote by $L^{2}(\mathscr{G})$ the associated completion and by
$\Lambda \colon P(\mathscr{G}) \to \LGtwo$ the natural embedding.
Then the Hilbert space $\LGtwo$ is the orthogonal direct sum of the
subspaces $\Lambda(\Gr{P(\mathscr{G})}{k}{l}{m}{n})$. Hence, there exist
operators $\lambda_{k},\lambda_{k}^{\op},\rho_{k},\rho_{k}^{\op}\in
{\cal B}(\LGtwo)$ such that
\begin{align*}
  \lambda_{k}\Lambda(a)&= \Lambda(\lambda_{k} a), &
  \lambda^{\op}_{k}\Lambda(a) &= \Lambda(a\lambda_{k}), &
  \rho_{k}\Lambda(a) &= \Lambda(\rho_{k}a), &
  \rho_{k}^{\op}\Lambda(a) &= \Lambda(a\rho_{k})
\end{align*}
for all $k\in I$ and $a\in P(\mathscr{G})$, and faithful, normal $*$-homomorphisms
\begin{align} \label{eq:vn-lambda-rho}
  \lambda,\rho \colon l^{\infty}(I) \to
  {\cal B}(\LGtwo)
\end{align}
that send the delta function at $k\in I$ to the operators
$\lambda_{k}$ or $\rho_{k}$, respectively. Let
\begin{align*}
  \vnE &:=\sum_{k} \rho_{k} \otimes \lambda_{k} \in \mathcal{B}(\LGtwo
  \otimes \LGtwo), & \overline{G} &:=
  \sum_{k} \rho_{k}^{\op} \otimes \rho_{k} \in \mathcal{B}(\LGtwo
  \otimes \LGtwo),
\end{align*}
where the sums converge with respect to the strong operator
topology.
\begin{Lem} \label{lemma:partial-isometry}
There exists a unique partial isometry $V$ on $\LGtwo \otimes \LGtwo$
such that
\begin{align*}
  V(\Lambda(a) \otimes \Lambda(b)) = \Lambda(a_{(1)}) \otimes \Lambda(a_{(2)}b)
\end{align*}
for all $a,b\in A$. Its range and domain projections are given by $VV^{*} = \vnE$
and $V^{*}V = \overline{G}$.
\end{Lem}
\begin{proof}
  Let $a,b \in A$. Since $\Delta$ is a $*$-homomorphism and $\phi$ is
invariant,
  \begin{align*}
    \langle \Lambda(a_{(1)}) \otimes
    \Lambda(a_{(2)}b)|\Lambda(a'_{(1)}) \otimes
    \Lambda(a'_{(2)}b')\rangle &=
    \phi(a_{(1)}^{*}a'_{(1)})\phi(b^{*}a_{(2)}^{*}a'_{(2)}b') \\
    &= \sum_{p}
    \phi(b^{*}\rho_{p}\phi(\rho_{p}a^{*}a'\rho_{p})\rho_{p}b') \\
    & =\sum_{p} \langle\Lambda(a\rho_{p}) \otimes \Lambda(\rho_{p}b) |
    \Lambda(a'\rho_{p}) \otimes \Lambda(b'\rho_{p})\rangle.
  \end{align*}
  Now, the assertion follows from \cite[Proposition 1.30]{DCT1}.
\end{proof}

\begin{Prop} \label{prop:gns} Let $\mathscr{G}$ be a partial compact
  quantum group with underlying total $*$-algebra $P(\mathscr{G})$ and
  associated Hilbert space $\LGtwo$. Then there exists a unique
  $*$-homomorphism $\pi_{r}\colon P(\mathscr{G}) \to {\cal B}(\LGtwo)$
  such that $\pi_{r}(a)\Lambda(b)=\Lambda(ab)$ for all $a,b\in
  P(\mathscr{G})$, and $\pi_{r}$ is faithful.
\end{Prop}
\begin{proof} 
  Let $a,c \in P(\mathscr{G})$. Then the formula $x \mapsto \langle
\Lambda(c) | x\Lambda(a)\rangle$ defines a bounded linear functional
  $\omega_{\Lambda(c),\Lambda(a)}$ on ${\cal B}(\LGtwo)$ and a
  straightforward computation shows that
  \begin{align} \label{eq:vn-slice}
    (\omega_{\Lambda(c),\Lambda(a)}\otimes \id)(V)\Lambda(b) =
    \Lambda(\varphi(c^*a_{(1)})a_{(2)}b)
  \end{align}
  for all $b\in P(\mathscr{G})$. Therefore, left multiplication by
  $\varphi(c^*a_{(1)})a_{(2)}$ extends to a bounded linear operator on
  $\LGtwo$. 
 Since $(P(\mathscr{G})\otimes 1)\Delta(P(\mathscr{G})) = (P(\mathscr{G})\otimes
  P(\mathscr{G}))\Delta(1)$ by  \cite[Proposition 1.30]{DCT1} and $\phi$ is
  normalized,  elements of the form $\phi(c^{*}a_{(1)})a_{(2)}$ span
  $P(\mathscr{G})$. 
\end{proof}
Since $\pi_{r}$ factorizes through $\pi_{u}$, we can conclude:
\begin{Cor}
  Let $\mathscr{G}$ be a partial compact quantum group with underlying
  total algebra $P(\mathscr{G})$. Then the
universal $*$-homomorphism $\pi_{u} \colon P(\mathscr{G}) \to\CuG$ is injective.
\end{Cor}

We call $(\LGtwo,\Lambda,\pi)$ the \emph{GNS-construction associated
  to $\mathscr{G}$}, denote by
\begin{align}
  \CrG &\subseteq {\cal B}(\LGtwo) &&\text{and} & \LGinf &\subseteq {\cal B}(\LGtwo)
\end{align}
the $C^{*}$-algebra and the von Neumann algebra generated by the image
of $\pi_{r}$, respectively, and identify $M(\CrG)$ with a
$C^{*}$-subalgebra of $\LGtwo$.  We thus obtain a sequence
\begin{align*}
P(\mathscr{G}) \hookrightarrow \CuG \to
  \CrG\hookrightarrow
\LGinf \hookrightarrow {\cal B}(\LGtwo)
\end{align*}
of $*$-algebras.
Note that
 the $*$-homomorphisms $\lambda,\rho$ in
\eqref{eq:vn-lambda-rho} send $l^{\infty}(I)$ to $M(\CrG)$, and that
\begin{align*}
  \vnE \in M(\CrG \otimes \CrG) \subseteq \LGinf \otimes \LGinf.
\end{align*}
\begin{Prop} \label{prop:vn-delta} Let $\mathscr{G}$ be a partial
  compact quantum group with underlying total $*$-algebra
  $P(\mathscr{G})$ and associated GNS-construction
  $(\LGtwo,\Lambda,\pi_{r})$. Then there
  exists a unique normal, faithful $*$-homomorphism $\vnDelta \colon
  \LGinf \to \LGinf \otimes \LGinf$ such that
  \begin{align*}
    \vnDelta(\pi_{r}(a)) (\Lambda(b) \otimes \Lambda(c)) =
    \Lambda(a_{(1)}b) \otimes \Lambda(a_{(2)}b) 
  \end{align*}
  for all $a,b,c \in P(\mathscr{G})$. Moreover, $\vnDelta(1)=\vnE$ and
  $\vnDelta(C^{r}_{0}(\mathscr{G})) \subseteq \vnE M(\CrG \otimes
  \CrG)\vnE$.
\end{Prop}
\begin{proof}
  Uniqueness is clear. To prove existence, one easily verifies that
  the map
  \begin{align*}
 \vnDelta \colon \LGinf \to {\cal B}(\LGtwo \otimes \LGtwo), \ x
  \mapsto V(x \otimes 1)V^{*}   
  \end{align*}
  has the desired properties. For example it is faithful because
  $\vnDelta(x)=0$ implies $x\otimes 1=0$ on $V^{*}V(L^{2}(\mathscr{G})
  \otimes L^{2}(\mathscr{G}))$ and hence $x=0$ on $\bigoplus_{k}
  \rho_{k}^{\op}L^{2}(\mathscr{G})=L^{2}(\mathscr{G})$.
\end{proof}

Denote by $\nu$ the weight on $l^{\infty}(I)$ given by
integration with respect to the counting measure on $I$. This weight is
normal, faithful, and semi-finite,  briefly written n.s.f.\
\begin{Prop} \label{prop:vn-phi}
  Let $\mathscr{G}$ be a partial compact quantum group with underlying
 $*$-algebra $P(\mathscr{G})$, invariant integral $\phi$ and
 GNS-construction $(\LGtwo,\Lambda,\pi_{r})$. Then
 there exist    
 \begin{itemize}
 \item a unique n.s.f.\ weight $\vnphi$ on
  $\LGinf$ such that $\pi_{r}(P(\mathscr{G})) \subseteq
  \mathfrak{M}_{\vnphi}$ and $\vnphi \circ \pi_{r} = \phi$,
\item unique n.s.f.\ conditional expectations
  $T_{\lambda}$ and $T_{\rho}$ from $\LGinf$ to $\lambda(l^{\infty}(I))$ and
  $\rho(l^{\infty}(I))$, respectively, such that $\nu \circ
  \lambda^{-1} \circ T_{\lambda}$ and $ \nu \circ \rho^{-1} \circ
  T_{\rho}$ equal $\vnphi$.
 \end{itemize}
We have  $(\id \otimes \vnphi)(\vnDelta(x)) =  T_{\lambda}(x)$ and $
    (\vnphi \otimes \id)(\vnDelta(x)) = T_{\rho}(x)$ 
 for all
  $x\in \LGinf_{+}$.
\end{Prop}
\begin{proof}
  We first prove uniqueness of $\vnphi$.  The
  $\Grt{p}{k}{m}:=\pi_{r}(\UnitC{k}{m})$ are pairwise orthogonal
  projections in $\mathfrak{M}_{\vnphi}$ which sum up to $1$. Hence, the
  weight $\vnphi$ is the sum of the bounded maps
  $\Grt{\vnphi}{k}{m} \colon x \mapsto
  \vnphi(\Grt{p}{k}{m}x\Grt{p}{k}{m})$ which are determined
  by their restrictions to $\pi_{r}(P(\mathscr{G}))$. Similar
  arguments apply to $T_{\lambda}$ and $T_{\rho}$.

Let us next prove existence.  Since $\phi$ is normalized,   each
$\Lambda(\UnitC{k}{m})$ is a unit vector and
\begin{align*}
  \vnphic{k}{m} \colon x \mapsto \langle \Lambda(\UnitC{k}{m})|x \Lambda(\UnitC{k}{m})\rangle
\end{align*}
defines a state. The weight $\vnphi$ and the maps $T_{\lambda}$ and $T_{\rho}$
defined by
\begin{align} \label{eq:vn-weights}
  \vnphi(x) &:= \sum_{k,m} \vnphic{k}{m}(x), &
    T_{\lambda}(x)&:= \sum_{k,m}
\vnphic{k}{m}(x)\lambda_{k}, & 
T_{\rho}(x)&:=
    \sum_{k,m} \vnphic{k}{m}(x)\rho_{m}
\end{align}
for all $x\in \LGinf_{+}$  are normal, semi-finite and satisfy $\nu
\circ \lambda^{-1} \circ T_{\lambda} = \vnphi = \nu \circ \rho^{-1}
\circ T_{\rho}$.

To  show that $\vnphi$ is faithful, we
use the family of functionals $(f_{z})_{z \in \C}$ on $P(\mathscr{G})$
introduced in \cite[Theorem 3.49]{DCT1}, and define 
$\sigma_{z} \colon P(\mathscr{G}) \to P(\mathscr{G})$
by
\begin{align*}
\sigma_{z}(a) &= (f_{iz} \otimes \id \otimes f_{iz})((\Delta_{kl}
\otimes \id)\Delta_{mn}(a)) \quad \text{for } a\in \Gr{A}{k}{l}{m}{n}.
\end{align*}
Then \cite[Theorem 3.49]{DCT1} implies that
\begin{align} \label{eq:alg-mod-aut}
 \sigma_{z}(\sigma_{z'}(a)) &= \sigma_{z+z'}(a), &
 \sigma_{z}(a^{*}) &= \sigma_{\overline{z}}(a)^{*}, &
\phi(\sigma_{z}(a)) &= \phi(a), & \phi(ab) &= \phi(b\sigma_{-i}(b))
\end{align}
for all $a,b\in P(\mathscr{G})$ and $z,z'\in \C$. Consequently, the conjugate-linear
  map $J$ on $\Lambda(P(\mathscr{G}))$ given by $\Lambda(a) \mapsto
  \Lambda(\sigma_{i/2}(a)^{*})$ extends to an anti-isometry on
  $\LGtwo$.  One easily verifies that $J\pi_{r}(a)J$ commutes with
  $\pi_{r}(b)$ for all $a,b\in P(\mathscr{G})$  and that the vectors
  $J\pi_{r}(a)J \Lambda(\UnitC{k}{l})$, where $a\in P(\mathscr{G})$
  and $k,l\in I$, span $\Lambda(P(\mathscr{G}))$. Therefore, the
  family $(\Lambda(\UnitC{k}{l}))_{k,l}$ is cyclic for $\LGinf'$ and
 separating for $\LGinf$. 


We finally check the invariance equations.  Let $a \in
P(\mathscr{G})$.  Then the relation $\vnphic{k}{m}\circ \pi_{r} = \phic{k}{m}$
and left invariance of $\phi$ imply
  \begin{align*}
    (\id \otimes \vnphic{l}{m})(\vnDelta(\pi_{r}(a))) &= \sum_{k}
    \vnphic{k}{m}(\pi_{r}(a)) \lambda_{k}\rho_{l}.
  \end{align*}
  Since each $\vnphic{k}{m}$ is a vector state and $\pi_{r}(P(\mathscr{G}))$ is
  weakly dense in $\LGinf$,  this equations
  remains true if we replace $\pi_{r}(a)$ by arbitrary $x\in
  \LGinf$. Summing over $l$ and $m$, we  find $(\id \otimes \vnphi)
  \circ \vnDelta = T_{\lambda}$. A similar argument shows that
  $(\vnphi \otimes \id) \circ \vnDelta=T_{\rho}$.
\end{proof}

\begin{Rem}
  \label{remark:vn-hilbert} The subspace $\Lambda(P(\mathscr{G}))
  \subseteq \LGtwo$ is a Hilbert algebra with respect to the
  operations $\Lambda(a)\Lambda(b)=\Lambda(ab)$ and $\Lambda(a)^{*}=
  \Lambda(a^{*})$ for all $a,b\in P(\mathscr{G})$, and a Tomita
  algebra with respect to the family of operators $\nabla_{z}$ given
  by $\nabla_{z}\Lambda(a)=\Lambda(\sigma_{z}(a))$ for all $a\in
  P(\mathscr{G})$, $z\in \C$. Indeed, the map $\pi_{r}(a)\colon
  \Lambda(b) \to \Lambda(ab)$ is bounded for each $a \in
  P(\mathscr{G})$ by Proposition \ref{prop:gns}, the involution is
  pre-closed because for all $a,b \in P(\mathscr{G})$,
  \begin{align*}
    \langle \Lambda(a)|\Lambda(b^{*})\rangle = \phi(a^{*}b^{*}) =
    \phi(b^{*}\sigma_{-i}(a^{*})) = \langle
    \Lambda(b)|\Lambda(\sigma_{-i}(a^{*}))\rangle
  \end{align*}
  the map $z\mapsto \langle \Lambda(a)|\nabla_{z}\Lambda(b)\rangle =
  \phi(a^{*}\sigma_{z}(b))$ is entire for all $a,b\in P(\mathscr{G})$
  and 
  \begin{align*}
    \nabla_{z}\Lambda(a)^{*} &= \nabla_{\overline{z}}\Lambda(a)^{*}, &
    \langle \Lambda(a)|\Lambda(b)\rangle &= \langle
    \nabla_{-\overline{z}}\Lambda(a) |\Lambda(b)\rangle, & \langle
    \Lambda(a)^{*}|\Lambda(b)^{*}\rangle = \langle \Lambda(b)|\nabla_{-i}\Lambda(a)\rangle
  \end{align*}
  for all $a,b\in P(\mathscr{G})$, $z\in \C$ by \eqref{eq:alg-mod-aut} and 
 \cite[Theorem 3.49]{DCT1}.

 The left von Neumann algebra of the Tomita algebra
 $\Lambda(P(\mathscr{G}))$ is $\pi_{r}(P(\mathscr{G}))''=\LGinf$, the
 associated n.s.f.\ weight $\tilde\phi$ coincides with $\vnphi$ because
 $\tilde \phi(\pi_{r}(a^{*}a))=\langle\Lambda(a)|\Lambda(a)\rangle =
 \phi(a^{*}a)$ for all $a\in P(\mathscr{G})$,  the modular
 operator $\Delta_{\vnphi}$ is the closure of $\nabla_{-i}$, the
 modular conjugation $J_{\vnphi}$ is given by
 $J_{\vnphi}\Lambda(a)=\Lambda(\sigma_{i/2}(a)^{*})$ for all $a\in
 P(\mathscr{G})$, and the modular automorphism group $\sigma^{\vnphi}$
 satisfies $\sigma^{\vnphi}_{t} \circ \pi_{r} = \pi_{r} \circ
 \sigma_{t}$ for all $t\in \R$; see
  \cite[Theorem VI.2.2 and its proof]{Taksak2}.
\end{Rem}


The operator-algebraic structures constructed so far fit into the
theory of measured quantum groupoids of Enock and Lesieur \cite{Eno2,Les1} as follows.

The relative tensor product of $\LGtwo$ with itself, relative to the
representations $\rho,\lambda$ of $l^{\infty}(I)$ and the weight
$\nu$, has the simple form
\begin{align*}
\LGinf \otimesrl \LGinf \cong
  \bigoplus_{k} (\rho_{k}\LGtwo \otimes \lambda_{k}\LGtwo) =
  \vnE(\LGtwo \otimes \LGtwo),
\end{align*}
the relative tensor product of operators $S\in \rho(l^{\infty}(I))'$
and $T \in \lambda(l^{\infty}(I))'$ gets identified with the
compression
\begin{align*}
S \otimesrl T \equiv
  \vnE(S \otimes
  T) = (S \otimes T)\vnE \subseteq {\cal B}(\vnE(\LGtwo
  \otimes \LGtwo)),
\end{align*}
and the fiber product of  $  \LGinf$ with itself, relative to $\rho$
and $\lambda$,  gets identified with
\begin{align} \label{eq:vn-fiber}
  \begin{aligned}
    \LGinf \astrl \LGinf &= (\LGinf' \otimesrl \LGinf')' \\ &\equiv
    (\vnE(\LGinf' \otimes \LGinf'))' = \vnE(\LGinf \otimes
    \LGinf)\vnE.
  \end{aligned}
\end{align} 
Since $\vnDelta(1)=\vnE$, we can co-restrict $\vnDelta$ to  a
normal, faithful $*$-homomorphism
\begin{align*}
  \tilde\Delta \colon \LGinf \to   \LGinf \astrl \LGinf.
\end{align*}
\begin{Theorem}
  Let $\mathscr{G}$ be an $I$-partial compact quantum group with
  associated von Neumann algebra $\LGinf$.  Then
  $(l^{\infty}(I),\LGinf, \lambda,\rho,\tilde\Delta)$ is a Hopf-von
  Neumann bimodule in the sense of  \cite{Val1}, and
  $(l^{\infty}(I),\LGinf, \lambda,\rho,\tilde\Delta,
  T_{\lambda},T_{\rho},\nu)$ is a measured quantum groupoid in the
  sense of \cite{Eno2}.
\end{Theorem}
\begin{proof}
  To prove the first assertion, we need to show that
  \begin{align*}
  \tilde\Delta(\lambda(x)) &= \lambda(x) \otimesrl 1 &&\text{and} &
  \tilde\Delta(\rho(x)) &= 1 \otimesrl \rho(x)
\end{align*}
 for all $x\in l^{\infty}(I)$,
and that
\begin{align*}
  (\tilde\Delta \astrl \id)\circ \tilde\Delta = (\id \astrl
    \tilde\Delta) \circ \tilde\Delta.
\end{align*}
The first equations follow immediately from the relation
$\Delta(\lambda_{k}\rho_{m})=\lambda_{k} \otimes \rho_{m}$. To verify
the third equation, we identify
\begin{align*}
 \LGinf \astrl \LGinf \astrl \LGinf \cong \vnE^{(2)}(\LGinf
  \otimes \LGinf \otimes \LGinf)\vnE^{(2)},
\end{align*}
where $\vnE^{(2)}=(\vnE \otimes 1)(1 \otimes \vnE)$, and then the two
compositions  become corestrictions of the maps $(\vnDelta \otimes
\id)\circ \vnDelta$ and $(\id \otimes \vnDelta)\circ \vnDelta$,
respectively, which coincide on $\pi_{r}(P(\mathscr{G}))$ by
co-associativity of $\Delta$. 

To prove that we obtain a measured quantum groupoid, we first need to
show that the modular automorphism groups of the weights $\nu \circ
\lambda^{-1} \circ T_{\lambda}$ and $\nu \circ \rho^{-1} \circ
T_{\rho}$ commute, which is trivially true because the two
compositions coincide with $\vnphi$. Next, we need to show that
$T_{\lambda}$ is left-invariant in the sense that
  \begin{align*}
   (\id \underset{\nu}{_{\rho}\ast_{\lambda}} \vnphi)(\tilde\Delta(x)) = T_{\lambda}(x) 
  \end{align*}
  for all $x\in \LGinf_{+}$. But it is easy to see that the left hand
  side coincides with $(\id \ast \vnphi)(\vnDelta(x))$ so that the
  equation above follows from Proposition
  \ref{prop:vn-phi}. Likewise $T_{\rho}$ is right-invariant in
  the appropriate sense. 
\end{proof}

The scaling group and unitary antipode of the measured quantum
groupoid above can be described in terms of the functionals
$(f_{z})_{z\in \C}$ on $P(\mathscr{G})$ introduced in \cite[Theorem
3.49]{DCT1} as follows.  Define $\tau_{z} \colon P(\mathscr{G}) \to
P(\mathscr{G})$ by
\begin{align*}
\tau_{z}(a) &= (f_{-iz} \otimes \id \otimes f_{iz})((\Delta_{kl}
\otimes \id)\Delta_{mn}(a)) \quad \text{for } a\in \Gr{A}{k}{l}{m}{n}.
\end{align*}
Then \cite[Theorem 3.49]{DCT1} implies that $\tau_{z}$ is an algebra
homomorphism and
\begin{align} \label{eq:alg-scale}
  \tau_{z} \circ \tau_{z'} &= \tau_{z+z'}, &
  \tau_{z}\circ \ast &=  \ast\circ \tau_{\overline{z}}, &
  S^{2} &= \tau_{i}, & S \circ \tau_{z} &= \tau_{z} \circ S
\end{align}
for all $z,z'\in \C$. Moreover, $R:=S\circ
\tau_{i/2}$ is a $*$-anti-automorphism of $P(\mathscr{G})$.  The
one-parameter group $\tau=(\tau_{t})_{t\in \R}$ and $R$ lift to
$C^{u}_{0}(\mathscr{G})$ by the universal property of
$C^{u}_{0}(\mathscr{G})$, and to the reduced level as well:
\begin{Prop}
  Let $\mathscr{G}$ be an $I$-partial compact quantum group with
  associated von Neumann algebra $\LGinf$. Then there exist
  \begin{itemize}
  \item a unique strongly continuous one-parameter group $\vntau$ on
    $\LGinf$ such that $\vntau_{t} \circ \pi_{r} = \pi_{r} \circ
    \theta_{-it,it}$ for all $t\in \R$, and
  \item a unique $*$-anti-automorphism $\vnR$ of $\LGinf$ such that
    $\vnR \circ \pi_{r} = \pi_{r} \circ R$.
  \end{itemize}
These
  are the scaling group and the unitary antipode of the measured
  quantum groupoid    $(l^{\infty}(I),\LGinf, \lambda,\rho,\tilde\Delta,
  T_{\lambda},T_{\rho},\nu)$, respectively.
\end{Prop}
\begin{proof}
  Uniqueness is clear.  Let us prove existence of $\vntau$.
  Invariance of $\phi$ and \cite[Theorem 3.49]{DCT1} imply $\phi \circ
  \tau_{z} = \phi$ for all $z\in \C$. Therefore, the formula
  $P_{t}\Lambda(a) = \Lambda(\tau_{t}(a))$ defines a one-parameter
  group of unitaries $(P_{t})_{t\in \R}$ on $\LGtwo$. By \cite[Theorem
  3.49]{DCT1}, elements of $\Lambda(P(\mathscr{G}))$ are analytic with
  respect to $P$, and by A.1 in \cite{Taksak2}, $P$ is strongly
  continuous.  Therefore, we can define $\vntau$ by  $\vntau_{t}(x)=
  P_{t}xP_{t}^{*}$ for all $x\in \LGinf$ and $t\in \R$.
  

  Next, we prove existence of $\vnR$. We claim that the formula $
  I\Lambda(a) = \Lambda(R(a)^{*})$ defines an conjugate-linear
  isometry $I$. Indeed, $\phi(R(a)R(b)^{*})= (\phi\circ R)(b^{*}a)$
  for all $a,b\in P(\mathscr{G})$ and $\phi \circ R=\phi \circ S \circ
  \tau_{i/2} = \phi \circ \tau_{i/2}=\phi$, where we used \cite[Lemma
  1.30]{DCT1}. Next, $I^{2} = \id$ because $*\circ R \circ * \circ R=
  R^{2}=\id$ by \eqref{eq:alg-scale}. Now,  short calculations show
  that the map $\vnR \colon x \mapsto Ix^{*}I$ has the desired
  properties.

\fixme{Finish this up}

  Denote by $\tilde R$ the unitary antipode and by $\tilde\tau$ the
  scaling group of the measured quantum groupoid.  Let us first prove
  that $\tilde \tau_{t}=\vntau$ for all $t\in \R$.

By
  \cite{} and \eqref{eq:scaling-modular-delta},
  \begin{align*}
    (\tilde \tau_{t} \astrl \sigma^{\vntau}_{t}) \circ \tilde \Delta
    &=\tilde \Delta \circ \sigma^{\vntau}_{t}, & (\vntau_{t} \otimes
    \sigma^{\vntau}_{t}) \circ \vnDelta &= \vnDelta \circ \vntau_{t}.
  \end{align*}
  The second equation implies that the first one remains true if we
  replace $\tilde\tau_{t}$ by $\vntau_{t}$.  Using Theorem A.7 in
  \cite{} [enock:action], we can conclude that $\tilde
  \tau_{t}=\vntau_{t}$.

 To prove that $\tilde R=\vnR$, we use the relations
  \begin{align*}
    \tilde R(\id \underset{\nu}{_{\rho} \ast_{\lambda}}
    \omega_{J\Lambda(b),J\Lambda(b)})(\vnDelta(\pi(a^{*}a))) &= (\id
    \underset{\nu}{_{\rho} \ast_{\lambda}}
    \omega_{J\Lambda(a),J\Lambda(a)})(\vnDelta(\pi(b^{*}b)))
  \end{align*}
from \cite{}.  
Let $c=a^{*}a$ and $d=b^{*}b$. 
A short calculation using \eqref{eq:modular} shows that  the right hand side is equal to
  \begin{align*}
    d_{(1)}\phi(\sigma_{i/2}(a)d_{(2)}\sigma_{i/2}(a)^{*})
    = d_{(1)}\phi(\sigma_{i/2}(c)d_{(2)}).
  \end{align*}
By Lemma \ref{lemma:strong-invariance} and
  \eqref{eq:scaling-modular-delta}, \eqref{eq:modular},  this equals 
  $S(\tau_{i/2}(c_{(1)}))
    \phi(\sigma_{i/2}(c_{(2)})d)$
which is equal to
$\vnR(\id \otimes
  \omega_{J\Lambda(b),J\Lambda(b)})(\vnDelta(\pi(a^{*}a)))$.
\end{proof}

%%% Local Variables: 
%%% mode: latex
%%% TeX-master: "dynamical-SUq-file"
%%% End: 
