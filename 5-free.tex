% In definition of faithful $I^2$-algebra, we can indeed just suppose that the embedding $\Ff(I^2)\rightarrow M(A)$ is faithful.
% Correct use of $\cdot$ versus operator product?

% Podles sphere can be defined as an algebra in $\Rep(SU_q(2))$. Hence, it should make sense as an algebra under the forgetful functor to SU_q(2)-dynamical. By duality for coideals, the same should hold for passage to SU_q(1,1) in fact...

% Also: direct construction should help in making non-compact construction

% Thank Piotr, Makoto and ...(?) for discussions. Link with graph C*-algebras?

\section{Compact quantum groups of face type from reciprocal random walks} 


We recall some notions introduced in [DCY]. We slightly change the terminology for the sake of convenience.

% Should we assume that the graph has a finite number of components? 
\begin{Def} By a graph $\Gamma$ we mean a quadruple $\Gamma=(V,E,s,t)$ where $V$ and $E$ are sets, called respectively the set of \emph{vertices} and the set of \emph{edges}, and $s$ and $t$ are maps \[s,t:E\rightarrow V,\] called respectively the source and target map.

An \emph{involution} of a graph is a map \[i:E \rightarrow E,\quad e\mapsto \overline{e}\] such that $s(\bar{e}) = t(e)$. 
\end{Def}

When $\Gamma$ is a graph, we also write $\Gamma{(0)}$ for the set of vertices and $\Gamma^{(1)}$ for the set of edges.

%An isomorphism of graphs consists of a pair of isomorphisms between the sets of vertices and the sets of edges which commute with the respective source and target maps.

\begin{Def} A \emph{weight} on a graph $\Gamma$ is a function $w:\Gamma^{(1)}\rightarrow \R_0^+$. By \emph{weighted graph} $(\Gamma,w)$ we will mean a graph $\Gamma$ equipped with a weight $w$. 

By a \emph{signed} graph we mean a graph equipped with a map $\sgn:\Gamma^{(1)}\rightarrow \{\pm 1\}$. 

By a \emph{signed weighted graph} we mean a signed graph equipped with a weight.
\end{Def}

%An isomorphism of weighted graphs is an isomorphism of graphs which intertwines the weights.

\begin{Def} A \emph{reciprocal weighted graph} $(\Gamma,w,i)$ consists of a weighted graph $(\Gamma,w)$ and an involution $i$ of $\Gamma$ such that $w(e)w(\bar{e}) = 1$ for all edges $e$. 

A \emph{positive (resp. negative) reciprocal signed graph} $(\Gamma,\sgn,i)$ consists of a signed graph $(\Gamma,\sgn)$ and an involution $i$ of $\Gamma$ such that $\sgn(e)\sgn(\bar{e}) = 1$ (resp. $=-1$) for all edges $e$. 

By a \emph{reciprocal signed weighted graph} we mean a signed and weighted graph which is reciprocal as a weighted and signed graph for the same involution. 

A weighted graph $(\Gamma,p)$ is called a \emph{random walk} if $\sum_{s(e)=v} p(e) = 1$ for all $v\in \Gamma^{(0)}$.
\end{Def}

\begin{Def} Fix $t\in \R_0$. A \emph{$t$-reciprocal random walk} is a $\sgn(t)$-reciprocal signed weighted graph $(\Gamma,w,\sgn,i)$ such that $(\Gamma,p)$ is a random walk for $p(e) = \frac{1}{|t|}w(e)$.  
\end{Def}

%Isomorphisms between $t$-random walks will simply be isomorphisms of weighted graphs (and hence do not remember any structure concerning involutions or signedness). 

% For examples, see ...

Let $0<|q|\leq 1$, and let $\Gamma = (\Gamma,w,\sgn,i)$ be a $-2_q$-reciprocal random walk. Then we can define a couple $\Hsp(\Gamma) = (\Hsp,R)$ where $\Hsp$ is $l^2(\Gamma^{(1)})$ with the $\Gamma^{(0)}$-bigrading $\delta_e \in \Hsp(s(e),t(e))$ for the obvious Dirac functions, and where $R$ is the (bounded) map $l^2(\Gamma^{(0)})\rightarrow \Hsp\underset{\Gamma^{(0)}}{\boxtimes} \Hsp$ defined as \begin{eqnarray*} R\delta_v &=& \sum_{e,s(e) = v} \sgn(e)\sqrt{w(e)}\delta_e \otimes \delta_{\bar{e}}.\end{eqnarray*} Then $R^*R = |q|+|q|^{-1}$ and \[(R^*\boxtimes \id)(\id\boxtimes R) = -\sgn(q)\id.\] By the fundamental property of the Temperley-Lieb tensor C$^*$-category $\mathcal{T}_q = \Rep(SU_q(2))$, this means that we have a unique strongly monoidal $^*$-functor
\[F_{\Gamma}:\mathcal{T}_q \rightarrow {}^{\Gamma^{(0)}}\Hilb_f^{\Gamma^{(0)}}\] such that $\Gamma(\pi_{1/2}) = \Hsp$ and $F(\mathscr{R}) = R$ with $(\pi_{1/2},\mathscr{R},-\sgn(q)\mathscr{R})$ a solution for the conjugate equations for $\pi_{1/2}$. 

Up to equivalence, $F_{\Gamma}$ only depends upon the isomorphism class of $(\Gamma,w)$, and is independent of the chosen involution or sign structure.

It then follows from our main theorem that for each reciprocal random walk on a graph $\Gamma$, one obtains a $\Gamma^{(0)}$-face compact quantum group. Our aim is to give a direct representation of it by generators and relations.

\begin{Theorem}\label{TheoGenRel} Let $0<|q|\leq 1$, and let $\Gamma$ be a $-2_q$-reciprocal random walk. Let $A(\Gamma)$ be the total $^*$-algebra associated to the $\Gamma^{(0)}$-face compact quantum group constructed from the fiber functor $F_{\Gamma}$. Then $A(\Gamma)$ is the universal $^*$-algebra generated by a copy of the $^*$-algebra of finite support functions on $\Gamma^{(0)}\times \Gamma^{(0)}$ and elements $(u_{e,f})_{e,f\in \Gamma^{(1)}}$ where $u_{e,f}\in A(\Gamma)\begin{pmatrix} s(e) & t(e)\\ s(f)&t(f)\end{pmatrix}$ and 
\begin{eqnarray*} 
\sum_{v\in \Gamma^{(0)}}\sum_{g\in \Gamma_{vw}} u_{g,e}^*u_{g,f} = \delta_{e,f}\mathbf{1}\Grru{w}{t(e)}\\
\sum_{w\in \Gamma^{(0)}} \sum_{g\in \Gamma_{vw}} u_{e,g}u_{f,g}^* = \delta_{e,f} \mathbf{1}\Grru{s(e)}{v}\\ 
u_{e,f}^* \;=\; \sgn(e)\sgn(f)\sqrt{\frac{w(f)}{w(e)}} u_{\bar{e},\bar{f}}.
\end{eqnarray*} 
\end{Theorem} 

It is also easy to give formulas for the comultiplication, counit and antipode on these generators. Namely, if $K,L  \in M_2(\Gamma^{(0)})$, we have for $K*L = \begin{pmatrix} s(e) & t(e)\\ s(f)& t(f)\end{pmatrix}$ that \[\Delta\Grru{K}{L}(u_{e,f}) = \underset{s(g) = L_{lu}}{\underset{t(g) = K_{rd}}{\sum_{g\in\Gamma^{(1)}}}} u_{e,g}\otimes u_{g,f}.\] For $e,f\in \Gamma^{(1)}$ with the same source and target, we have
\[\varepsilon(s(e)\;t(e))(u_{e,f}) = \delta_{e,f}.\]
Finally, for the antipode we have \[S(u_{e,f}) = u_{f,e}^*.\]


%%% Local Variables: 
%%% mode: latex
%%% TeX-master: "dyn-suq-main"
%%% End: 
