\section{Generalized compact Hopf face algebras from ergodic actions
  of quantum $SU(2)$}



We apply the construction of the previous section the the specific case of $\CatC = \Rep(SU_q(2))$, the category of finite-dimensional unitary representations of $SU_q(2)$ for some $0<|q|<1$. By [DCY], one can encode strong tensor functors $\CatC\rightarrow \Hilb_{I^2}$ by the following data:
\begin{itemize}
\item[$\bullet$] A collection of finite sets $B_{kl}$ indexed by $I^2$ such that for $l$ fixed, $B_{kl} = \emptyset$ for all but finitely many $k$, and such that for $k$ fixed, $B_{kl}=\emptyset$ for all but finitely many $l$.
\item[$\bullet$] For each $(k,l)\in J^2$ we have an endomorphism $E(kl): \Fun(B_{kl})\rightarrow \Fun(B_{lk})$ such that $\overline{E(kl)}E(lk) = -\sgn(q)\id$ for all $k,l$ and $\sum_l \Tr(E(kl)E(lk)^*) = |q+q^{-1}|$ for all $k$.
\end{itemize}

The associated tensor functor $F_E:\CatC\rightarrow \Hilb_{J^2}$ is then obtained as the unique \emph{strict} one such that

\begin{itemize}
 \item[$\bullet$] The fundamental object $u_{1/2}$, the spin $1/2$-representation, of $\CatC=\Rep(SU_q(2))$ is sent to $\mathscr{H} = \oplus \mathscr{H}_{kl}$ with $\mathscr{H}_{kl} = \Fun(B_{kl})$ with its canonical Hilbert space structure.
 \item[$\bullet$] The normalized self-duality $R: 1\rightarrow u_{1/2}\otimes u_{1/2}$ is sent to the operator $\mathscr{R}:l^2(J)\rightarrow \mathscr{H}\iitimes \mathscr{H}$ which decomposes as $\mathscr{R}_{\mid\C\delta_k} = \sum_l \mathscr{R}_{kl}$ with $\mathscr{R}_{kl}:\C\rightarrow \mathscr{H}_{kl}\otimes \mathscr{H}_{lk}$ such that $E(kl)_{ij} = -\sgn(q)\mathscr{R}_{kl}^*(\delta_j\otimes \delta_i)$.
\end{itemize}

We will in the following write $B=\sqcup B_{kl}$, and $(s(i),t(i))=(k,l)$ for $i\in B_{kl}\subseteq B$. Let $(A,\Delta)$ be the generalized compact Hopf face algebra obtained from the construction of the previous section. Our aim is to find an explicit generators and relations description of $(A,\Delta)$.

\begin{Def} We define $\DA$ to be the $^*$-algebra generated by elements $u_{ij}$ with $i,j\in B$, satisfying the relations \begin{eqnarray*} \sum_{k\in I}\sum_{i\in B_{kl}} u_{ij}^*u_{is} &=& \delta_{js} \lambda_{l}\rho_{t(j)},\qquad \forall l\in I,j,s\in B, \\ \sum_{n\in I}\sum_{j\in B_{mn}} u_{ij}u_{rj}^* &=& \delta_{i,r}\lambda_{s(i)}\rho_m,\qquad\forall m\in I,i,r\in B\end{eqnarray*} and \begin{eqnarray*} u_{ij}^* &=& \underset{p\in B_{t(i),s(i)}}{\underset{r\in B_{t(j),s(j)}}{\sum}} E(t(i)s(i))_{ip}u_{pr}(E(t(j)s(j))^{-1})_{rj},\qquad \forall i,j\in B.\end{eqnarray*}
\end{Def}

Note that because of the finiteness assumption on the $B_{kl}$, the sums in all relations are finite sums.

\begin{Prop} Write $v_{ij}\in \Gr{A}{s(i)}{t(i)}{s(j)}{t(j)}(1/2) = B(F(u_{1/2})_{s(i)t(i)},F(u_{1/2})_{s(j)t(j)})$ for the operator sending the basis vector $\delta_{p}$ to $\delta_{p,i} \delta_j$. Then there is a $^*$-isomorphism $\Phi:\DA\rightarrow A$ such that $u_{ij}$ is sent to $v_{ij}$.
\end{Prop}


\begin{proof} Let us first check that $\Phi$ is well-defined, that is, that the $v_{ij}$ satisfy the same relations as the $u_{ij}$. We will resort to the notations of the previous section, so $\cdot$ denotes the multiplication in $A$ and $\dag$ denotes the $^*$-operation. Then $v_{ij}^{\dag} = S(v_{ji})$ by definition of $\dag,S$ and the $v_{ij}$. Moreover, we have \[\widetilde{\Delta}(v_{ij}) = \sum_{p,r} v_{pr}v_{ij}\otimes v_{rp} = \sum_p  v_{ip}\otimes v_{pj}\] as $v_{pr}v_{ij} = \delta_{j,p}v_{ir}$ under the operator product. Hence \begin{eqnarray*} \sum_{k\in I}\sum_{i\in B_{kl}} v_{ij}^{\dag}\cdot v_{is} &=& S(v_{js(1)})\cdot v_{js(2)}\cdot\lambda_l \\ &=& \sum_p \varepsilon(v_{js}\cdot \lambda_p)\lambda_l\cdot\rho_p \\ &=& \varepsilon(v_{js}) \lambda_l\rho_{t(j)} \\ &=& \delta_{js}\lambda_l\rho_{t(j)}.\end{eqnarray*}

The relation $\sum_{n\in I}\sum_{j\in B_{mn}} v_{ij}\cdot v_{rj}^{\dag} = \delta_{i,r}\lambda_{s(i)}\rho_m$ follows similarly.

Let us turn to the final relation. By definition of the $E(kl)$, we have that $\langle J_a\delta_j,\delta_i\rangle =E(lk)_{ij}$ and $\langle I_a\delta_j,\delta_i\rangle = -\sgn(q)E(kl)_{ij}$. From the definition of $\dag$ and the fact that $R_{1/2} = -\sgn(q)\bar{R}_{1/2}$, we then obtain that for $(s(i),t(i))=(k,l)$ and $(s(j),t(j))=(m,n)$, we have \[v_{ij}^{\dag} = -\sgn(q)\sum_{pr} E(lk)_{ip} v_{pr} \overline{E(mn)_{rj}}.\] Since $\overline{E(mn)} = -\sgn(q)E(nm)^{-1}$, we find that the final relation for the $u_{ij}$ is satisfied.

\end{proof}

\subsection{Generalized compact Hopf face algebras from Podle\'{s} spheres}

As a particular case of the construction in the previous section, consider $I = \Z$ with $B_{kl} =\emptyset$ when $k\neq l\pm 1$, and $B_{kl} = \{(k,l)\}$ when $k = l\pm 1$. Put \[E(k,k\pm1) =c_{\pm}\left(\frac{|q|^{x+k\pm1}+|q|^{-x-k\mp1}}{|q|^{x+k}+|q|^{-x-k}}\right)^{1/2}\] where $c_{+}=1$ and $c_-=-\sgn(q)$. Then this collection satisfies the requirements postulated at the beginning of the previous section. It is obtained from the ergodic action of $SU_q(2)$ on the Podle\'{s} sphere $S_{q,\tau(x)}^2$ with $\tau(x) = ...$ as described in [DCY].

For $\epsilon,\nu\in \{+,-\}$, let us write $u_{kl} = u_{(k,k+\epsilon),(l,l+\nu)}$. Then the unitarity relations ... become \begin{eqnarray*} \sum_{\epsilon} (u_{k-\epsilon,l}^{\epsilon,\nu})^*u_{k-\epsilon,m}^{\epsilon,\mu} &=& \delta_{\nu,\mu}\delta_{l,m} \lambda_k\rho_{l+\nu},\\ \sum_{\nu} u_{kl}^{\epsilon,\nu}(u_{ml}^{\mu,\nu})^* &=& \delta_{\epsilon,\mu}\delta_{k,m}\lambda_k\rho_m.\end{eqnarray*} In turns of the multipliers $u^{\epsilon,\nu} = \sum_{k,l} u^{\epsilon,\nu}_{kl}$, this is equivalent with saying that the matrix \[U = \begin{pmatrix} u^{--}& u^{+-} \\ u^{-+}&u^{++}\end{pmatrix}\] is unitary.

The relations for the adjoint further imply that \begin{eqnarray*}u_{kl}^{+-} &=&\frac{E(l,l-1)}{E(k,k+1)}(u_{k+1,l-1}^{-+})^*,\\  u_{kl}^{++}&=& \frac{E(l,l+1)}{E(k,k+1)}(u_{k+1,l+1}^{--})^* .\end{eqnarray*}

Let us write $F(k)^{1/2} = \frac{1}{|q|^{1/2}E(k,k+1)}$. Further, for any function $f$ on $I$, write $f(\lambda) = \sum_k f(k)\lambda_k$ and $f(\rho) = \sum_m f(m)\rho_m$. Then using that $E(l+1,l) = -\sgn(q)\frac{1}{E(l,l+1)}$, we find that the above adjoint relations are equivalent to

\begin{eqnarray*} u^{+-} &=& -q F(\rho-1)^{1/2}F(\lambda)^{1/2}(u^{-+})^* \\ u^{++} &=&  \frac{F(\lambda)^{1/2}}{F(\rho)^{1/2}}(u^{--})^*.
\end{eqnarray*}

Write now \[\alpha = u^{--},\qquad \beta = qF(\rho)^{1/2}u^{-+}.\] Then the above commutation relations are equivalent to \begin{align*} \alpha \beta = qF(\rho-1)\beta\alpha && \alpha\beta^* = qF(\lambda)\beta^*\alpha\end{align*} \begin{align*} \alpha\alpha^* +F(\lambda)\beta^*\beta = 1,&& \alpha^*\alpha+\frac{q^{-2}}{F(\rho-1)}\beta^*\beta = 1,\\ \frac{1}{F(\rho-1)}\alpha\alpha^* +\beta\beta^* = F(\lambda-1)^{-1},&& F(\lambda)\alpha^*\alpha +q^{-2}\beta\beta^* = F(\rho),\end{align*} \begin{align*} f(\lambda)g(\rho)\alpha =
\alpha f(\lambda+1)g(\rho+1),&& f(\lambda)g(\rho)\beta = \beta f(\lambda+1)g(\rho-1).\end{align*}

These are precisely the commutation relations for the dynamical quantum $SU(2)$-group as in for example [KR], except that the precise value of $F$ has been changed by a shift in the parameter domain by a complex constant. As the coproduct on $A$ is given by $\Delta(u^{\epsilon,\nu}) = \sum_{\mu} u^{\epsilon,\mu}\itimes u^{\mu,\nu}$, we also find that the coproduct agrees with the one on the dynamical quantum $SU(2)$-group, namely \begin{eqnarray*} \Delta(\alpha) &=& \alpha\itimes \alpha - q^{-1}\beta\itimes \beta^*,\\ \Delta(\beta) &=& \beta\itimes \alpha^* +\alpha\itimes \beta.\end{eqnarray*}

More generally, 

As a concrete instance of the example of monoidal equivalence, let $\tilde{A}$ be the generalized compact Hopf face algebra obtained from the set $\tilde{I} =I_1\sqcup I_2$ with $I_1= \Z$ and $I_2= \{\bullet\}$ with the $B_{kl} =\emptyset$ and $E(k,l)$ for $k,l\neq \bullet$ as in section ..., with $B_{k,\bullet} = B_{\bullet,k}= \emptyset$, and $B_{\bullet,\bullet} = \{\pm\}$ with $E_{\bullet,\bullet} = \begin{pmatrix} 0 & |q|^{1/2} \\ -\sgn(q)|q|^{-1/2}&0\end{pmatrix}$ (with the basis ordered as $-,+$). Then this will be obtained from the direct sum of the functor from ... and the ordinary forgetful functor from $\Rep(SU_q(2))$ into $\Hilb$. It follows that the components $\tilde{A}(ij)$ can be described by the generators and relations as in ..., but with $F(\lambda)$ and $F(\rho)$ set equal to 1 whenever the corresponding index is $\bullet$.




% Study spectrum fundamental character
% Study dual quantum groupoid
% Make connection with dynamical cocycle
% In case of qgroupoid constructed from identity functor for Rep(SU_q(2)): rep theory of associated Galois object should just be: a single representation (Galois object is type I factor, cutdown of $B(\mathscr{L}^2(SU_q(2)))$). Yes: in general, Galois object is Morita equivalent with algebra of original ergodic action, should also be stressed for Podles spheres

%%% Local Variables: 
%%% mode: latex
%%% TeX-master: "dyn-suq-main"
%%% End: 
