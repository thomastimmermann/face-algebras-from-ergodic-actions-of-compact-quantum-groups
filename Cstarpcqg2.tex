\documentclass[11pt]{article}

\usepackage{hyperref}
\usepackage{fixme}
\usepackage{mathrsfs}
\usepackage[a4paper]{geometry}
\usepackage{amssymb, amsthm, amsfonts, amsxtra, amsmath}
\usepackage{latexsym}
\usepackage{mathabx}
%\usepackage{enumitem}
%\usepackage[all]{xy}
%\usepackage{graphics}
\usepackage{pdfpages}
\usepackage{tensor}
\usepackage{epic}
\usepackage{fouridx}
\usepackage{parskip} % paragraphs have no indents and vertical spacings inbetween
\makeatletter % need this to avoid the conflict between amsthm and parskip
\def\thm@space@setup{%
  \thm@preskip=\parskip \thm@postskip=0pt
}
\makeatother
\usepackage{enumerate}

%\theoremstyle{change}

\newcommand{\dual}[1]{#1^{\vee}}
\newcommand{\predual}[1]{{^{\vee}\!#1}}
\newcommand{\co}{\mathrm{co}}
\newcommand{\Corep}{\mathrm{Corep}}
\newcommand{\Corepf}{\mathrm{Corep}^{f}}
\newcommand{\sff}{\textrm{s.f.~}}
\newcommand{\sfs}{\mathrm{sfs}}
\newcommand{\sfd}{\mathrm{sfd}}
\DeclareMathOperator{\Hom}{Hom}
\DeclareMathOperator{\img}{img}

\DeclareMathOperator{\id}{id}
\DeclareMathOperator{\ext}{\mathrm{e}}
\DeclareMathOperator{\can}{\mathrm{can}}
\DeclareMathOperator{\ctau}{\tau}
\DeclareMathOperator{\iboxtimes}{\underset{I}{\boxtimes}}
\DeclareMathOperator{\op}{\mathrm{op}}
\DeclareMathOperator{\fin}{\mathrm{f}}
\DeclareMathOperator{\Pol}{\mathrm{P}}
\DeclareMathOperator{\End}{\mathrm{End}}
\DeclareMathOperator{\Par}{\mathrm{Par}}
\DeclareMathOperator{\reg}{\mathrm{reg}}
\DeclareMathOperator{\sgn}{\mathrm{sgn}}
\DeclareMathOperator{\Zz}{\mathrm{Z}}
\DeclareMathOperator{\Ran}{\mathrm{Ran}}
\DeclareMathOperator{\hol}{\mathrm{hol}}
\DeclareMathOperator{\Ind}{\mathrm{Ind}}
\DeclareMathOperator{\Ker}{\mathrm{Ker}}
\DeclareMathOperator{\Char}{\mathrm{Char}}
\DeclareMathOperator{\dyn}{\mathrm{dyn}}
\DeclareMathOperator{\Spec}{\mathrm{Spec}}
\DeclareMathOperator{\adj}{\mathrm{adj}}
\DeclareMathOperator{\rcfd}{\mathrm{rcfd}}
\DeclareMathOperator{\rcf}{\mathrm{rcfd}}
\DeclareMathOperator{\stau}{\tau_{\mathrm{s}}}
\DeclareMathOperator{\tA}{\tilde{A}}
\DeclareMathOperator{\weps}{\tilde{\epsilon}}

\newcommand{\Circt}{{\mathop{\ooalign{$\ovoid$\cr\hidewidth\raise-.05ex\hbox{$\scriptstyle\mathsf T\mkern3.5mu$}\cr}}}} % Woronowicz style tensor product, USUAL SIZE
\newcommand{\Circtv}[1]{\underset{#1}{\mathop{\ooalign{$\ovoid$\cr\hidewidth\raise-.05ex\hbox{$\scriptstyle\mathsf T\mkern3.5mu$}\cr}}}} % Woronowicz style tensor product, USUAL SIZE
\newcommand{\smCirct}{\mathop{\ooalign{$\scriptstyle\ovoid$\cr\hidewidth\raise-.05ex\hbox{$\scriptscriptstyle\mathsf T\mkern2.8mu$}\cr}}}  % Woronowicz style tensor product, SCRIPT SIZE

\newcommand{\nc}{\R}
\newcommand{\g}{\mathfrak{g}}
\newcommand{\h}{\mathfrak{h}}

\newcommand{\kk}{\mathfrak{k}}
\newcommand{\ttt}{\mathfrak{t}}
\newcommand{\p}{\mathfrak{p}}
\newcommand{\n}{\mathfrak{n}}
\newcommand{\llll}{\mathfrak{l}}
\newcommand{\uu}{\mathfrak{u}}
\newcommand{\bb}{\mathfrak{b}}
\newcommand{\q}{\mathfrak{q}}
\newcommand{\su}{\mathfrak{su}}
\newcommand{\ssl}{\mathfrak{sl}}
\newcommand{\SSL}{\mathrm{SL}}
\newcommand{\so}{\mathfrak{so}}
\newcommand{\spp}{\mathfrak{sp}}
\newcommand{\G}{\mathbb{G}}
\newcommand{\e}{\mathfrak{e}}
\newcommand{\s}{\mathfrak{s}}
\newcommand{\C}{\mathbb{C}}
\newcommand{\R}{\mathbb{R}}
\newcommand{\Z}{\mathbb{Z}}
\newcommand{\N}{\mathbb{N}}
\newcommand{\X}{\mathbb{X}}
\newcommand{\Y}{\mathbb{Y}}
\newcommand{\Ss}{\mathbb{S}}
\newcommand{\ZZ}{\mathscr{Z}}
\newcommand{\ad}{\mathrm{ad}}
\newcommand{\Hsp}{\mathcal{H}}
\newcommand{\qn}[2]{\lbrack #1 \rbrack_{#2}}
\newcommand{\fqn}[2]{\lbrack #1 \rbrack_{#2}!}
\newcommand{\bqn}[3]{\left\lbrack \begin{array}{c} \!#1\! \\ \!#2\! \end{array}\right\rbrack_{#3}}
\newcommand{\Tr}{\mathrm{Tr}}
\newcommand{\RR}{\mathcal{R}}
\newcommand{\rd}{\mathrm{d}}
\newcommand{\res}{\mathrm{res}}
\newcommand{\cop}{\mathrm{cop}}
\newcommand{\opp}{\mathrm{op}}
\newcommand{\coop}{\mathrm{coop}}
\newcommand{\Rm}{\mathcal{R}}
\newcommand{\wt}{\mathrm{wt}}
\newcommand{\Ad}{\mathrm{Ad}}
\newcommand{\CatC}{\mathcal{C}}
\newcommand{\CatD}{\mathcal{D}}
\newcommand{\CatCC}{\mathscr{C}}
\newcommand{\CatDD}{\mathscr{D}}
\newcommand{\Corr}{\mathrm{Corr}}

\newcommand{\Vectf}{\mathrm{Vect}^{f}}
\newcommand{\Vecti}{\mathrm{Vect}_{I^{2}}}
\newcommand{\Vectif}{\mathrm{Vect}^{f}_{I^{2}}}
\newcommand{\Hilb}{\mathrm{Hilb}}
\newcommand{\Hilbf}{\mathrm{Hilb}^{\mathrm{f}}}
\newcommand{\Hilbi}{\mathrm{Hilb}_{I^{2}}}
\newcommand{\Hilbif}{\mathrm{Hilb}_{I^{2}}^{\mathrm{f}}}

\newcommand{\Star}[2]{{}_{#1}\!*_{#2}}
\newcommand{\vot}{\bar{\otimes}}
\newcommand{\A}{\mathcal{B}}
\newcommand{\Aa}{\mathscr{B}}
\newcommand{\Mor}{\mathrm{Mor}}
\newcommand{\alg}{\mathrm{alg}}
\newcommand{\Gg}{\mathscr{G}}
\newcommand{\ev}{\mathrm{ev}}
\newcommand{\Rtimes}{\underset{\R}{\times}}
\newcommand{\Rb}{\R^{\bullet}}
\newcommand{\vtimes}{\bar{\otimes}}
\newcommand{\Rr}{\mathscr{R}}
\newcommand{\Tt}{\mathscr{T}}
\newcommand{\Fun}{\mathrm{Fun}}
\newcommand{\Ff}{\Fun_{\fin}}
%\newcommand{\fin}{\mathrm{fin}}
%\newcommand{\iitimes}{\underset{I}{\otimes}}
\newcommand{\itimes}{\underset{I}{\otimes}}
\newcommand{\osum}[1]{\underset{#1}{\sum}^{\oplus}}
\newcommand{\osumc}[1]{\underset{#1}{\sum}^{\bar{\oplus}}}
\newcommand{\oplusc}{\bar{\oplus}}
\newcommand{\wDelta}{\widetilde{\Delta}}
\newcommand{\f}{\mathrm{fin}}
%\newcommand{\Hilb}{\mathrm{Hilb}}
\newcommand{\Rho}{\mathrm{P}}
\newcommand{\Rep}{\mathrm{Rep}}
\newcommand{\DA}{\mathcal{A}}
%\newcommand{\Circt}{\mathop{\ooalign{$\ovoid$\cr\hidewidth\raise-.05ex\hbox{$\scriptstyle\mathsf T\mkern3.5mu$}\cr}}} % Woronowicz style tensor product, USUAL SIZE
\newcommand{\even}{\mathrm{even}}
\newcommand{\odd}{\mathrm{odd}}
\newcommand{\fd}{\mathrm{fd}}
\newcommand{\Forget}{F}

\newcommand{\GrHA}[3]{#1{\begin{pmatrix} #2,  #3\end{pmatrix}}}% Horizontal grading ordinary style, with argument
\newcommand{\Grs}[3]{#1{\begin{pmatrix} #2,  #3\end{pmatrix}}}

\newcommand{\GrDA}[3]{{}_{#2}#1_{#3}} % Horizontal grading bottom style, with argument
%\newcommand{\Grd}[3]{\;{}_{\;#2}#1_{#3}}

\newcommand{\GrVA}[3]{#1{\tiny {\begin{pmatrix} #2\\#3\end{pmatrix}}}} % Vertical grading ordinary style, with argument
\newcommand{\Grt}[3]{#1{\tiny {\begin{pmatrix} #2\\#3\end{pmatrix}}}} 

\newcommand{\GrRA}[3]{#1^{#2}_{#3}} % Vertical grading right style, with argument

\newcommand{\GrLA}[3]{{}^{#2}_{#3}#1} % Vertical grading left style, with argument

\newcommand{\Unit}{\mathbf{1}}
\newcommand{\UnitC}[2]{\Grt{\mathbf{1}}{#1}{#2}} 
\newcommand{\Grru}[2]{{\tiny \begin{pmatrix} #1 \\ #2\end{pmatrix}}}

\newcommand{\eGr}[5]{#1{{\tiny \begin{pmatrix} #2 \quad #3 \\ #4 \quad #5\end{pmatrix}}}}

\newcommand{\pma}[4]{\begin{pmatrix} #1 \quad #2 \\ #3 \quad #4\end{pmatrix}}
\newcommand{\pmat}[4]{{\tiny \begin{pmatrix} #1 \quad #2 \\ #3 \quad #4\end{pmatrix}}}

\newcommand{\UT}[2]{#1{\tiny #2 }}
\newcommand{\Gr}[5]{\tensor*[^{#2}_{#4}]{#1}{^{#3}_{#5}}}%TODO: better typesetting
\newcommand{\Grl}[3]{\Gr{#1}{#2}{}{#3}{}}%TODO: better typesetting
\newcommand{\Gru}[3]{\Gr{#1}{}{}{#2}{#3}}
\newcommand{\Grd}[3]{\Gr{#1}{}{}{#2}{#3}}
% \newcommand{\Gr}[5]{\;{}^{\;#2}_{#4}#1_{#5}^{#3}}%TODO: better typesetting
% %\newcommand{\Gr}[5]{\UT{#1}{\begin{pmatrix} #2\quad #3 \\ #4 \quad #5\end{pmatrix}}}
% %\newcommand{\Gr}[5]{\UT{#1}{\begin{pmatrix} \, #2\;\\ #3 \qquad #4 \\ \,#5\;\end{pmatrix}}}
% \newcommand{\Grl}[3]{\;{}^{\;#2}_{#3}#1}%TODO: better typesetting
% \newcommand{\Gru}[3]{{}^{\;#2}#1^{#3}}
% \newcommand{\Grd}[3]{{}_{\;#2}#1_{#3}}
\newcommand{\gr}[5]{\;{}^{\;#2}_{#4}#1_{#5}^{#3}}%TODO: better typesetting
\newcommand{\eGrr}[3]{#1_{{\tiny \left(#2, #3\right)}}}
\newcommand{\eGrt}[4]{#1{{\tiny \begin{pmatrix} #2 \\ #3 \\ #4 \end{pmatrix}}}}
\newcommand{\Grr}[4]{\begin{pmatrix}#1 \quad #2\\#3&#4\end{pmatrix}}

\newcommand{\Grss}[3]{\UT{#1}{\begin{pmatrix} #2 \; #3\end{pmatrix}}}
\newcommand{\Grb}[7]{\UT{#1}{\begin{pmatrix} #2\quad #3 \\ #4 \quad #5\\ #6 \quad #7\end{pmatrix}}}
\newcommand{\un}[2]{e{{\tiny \begin{pmatrix}#1\\ #2\end{pmatrix}}}}
\newcommand{\unn}[3]{e{{\tiny \begin{pmatrix}#1\\ #2\\#3\end{pmatrix}}}}

\newcommand{\wmult}{\cdot}
\newcommand{\bmult}{*}
\newcommand{\wmate}{\rightarrow}% Change this to source/target notation l(eft) r(ight)
\newcommand{\bmate}{\downarrow}% Change this to source/target notation u(p) d(own)

\newcommand{\aste}[1]{\underset{#1}{\ast}}

\newcommand{\Vv}{\mathcal{V}}

\newcommand{\dT}{\dot T}

\newtheorem{Theorem}{Theorem}[section]
\newtheorem{Lem}[Theorem]{Lemma}
\newtheorem{Prop}[Theorem]{Proposition}
\newtheorem{Cor}[Theorem]{Corollary}

\theoremstyle{definition}
\newtheorem{Def}[Theorem]{Definition}
\newtheorem{Rem}[Theorem]{Remark}
\newtheorem{Exa}[Theorem]{Example}
\newtheorem{Not}[Theorem]{Notation}
\newtheorem{Que}[Theorem]{Question}
\newtheorem{Con}[Theorem]{Conjecture}

%%%%%%%%%%%%%%%%%%%
% Further notation for Section 1
\newcommand{\phic}[2]{\Grt{\phi}{#1}{#2}}

%%%%%%%%%%%%%%%%%%%
% Notation for Section 4
\newcommand{\LGtwo}{L^{2}(\mathscr{G})}
\newcommand{\LGinf}{L^{\infty}(\mathscr{G})}
\newcommand{\CrG}{C^{r}_{0}(\mathscr{G})}
\newcommand{\CuG}{C^{u}_{0}(\mathscr{G})}
\newcommand{\vnDelta}{\overline{\Delta}}
\newcommand{\vnE}{\overline{E}}
\newcommand{\astrl}{\underset{l^{\infty}(I)}{_{\rho}\ast_{\lambda}}}
\newcommand{\otimesrl}{\underset{\nu{}}{_{\rho}\otimes_{\lambda}}}
\newcommand{\vnphi}{\overline{\phi}}
\newcommand{\vnphic}[2]{\Grt{\vnphi}{#1}{#2}}
\newcommand{\vnR}{\overline{R}}
\newcommand{\vntau}{\overline{\tau}}

% q-special functions macros

\newcommand{\qbin}[2]{\left[ \begin{array}{c} #1 \\ #2 \end{array}\right]_{q^2}}
\newcommand{\qortc}[4]{\,\;_1\varphi_1\left(\begin{array}{c} #1  \\#2 \end{array}\mid #3,#4\right)}
\newcommand{\qortPsi}[4]{\Psi\left(\begin{array}{c} #1  \\#2 \end{array}\mid #3,#4\right)}
\newcommand{\qorta}[5]{\,\;_2\varphi_1\left(\begin{array}{cc} #1 & #2 \\ & \!\!\!\!\!\!\!\!\!\!\!#3 \end{array}\mid #4,#5\right)}
\newcommand{\qortd}[6]{\,\;_2\varphi_2\left(\begin{array}{cc} #1 & #2 \\ #3 & #4 \end{array}\mid #5,#6\right)}
\newcommand{\qortb}[7]{\,\;_3\varphi_2\left(\begin{array}{ccc} #1 & #2 & #3 \\ & \!\!\!\!\!\!\!\!#4 & \!\!\!\!\!\!\!\!#5\end{array}\mid #6,#7\right)}

\date{}


\numberwithin{equation}{section}

\begin{document}

\section{C$^*$-pcqg}
For $A$ a C$^*$-algebra, we denote by $M(A)$ the multiplier C$^*$-algebra of $A$. All tensor products of C$^*$-algebras in this paper will be minimal. We denote by $[\,\cdot\,]$ the closed linear span of a subset.

\begin{Def}\label{DefCpcqg} Let $I$ be a set. We call \emph{C$^*$-algebraic $I$-partial compact quantum group}, or \emph{C$^*$-pcqg} (over $I$) for short, a triple consisting of 
\begin{itemize}
\item a (not necessarily unital) C$^*$-algebra $A$,
\item a family of orthogonal self-adjoint projections $\UnitC{k}{l}\in A$ for $k,l\in I$, some of which are possibly zero, and
\item  a (not necessarily unital) $^*$-homomorphism \[\Delta: A\rightarrow M(A\otimes A),\] 
\end{itemize}
satisfying the following conditions:
\begin{enumerate}[(a)]
\item[Ui)] $\UnitC{k}{k}\neq 0$ for all $k\in I$. 
\item[Uii)] $\sum_{k,l} \UnitC{k}{l}$ strictly converges to the unit in $M(A)$.
\item[Uiii)] $\Delta(\UnitC{k}{l}) = \sum_{m}\UnitC{k}{m}\otimes \UnitC{m}{l}$ strictly for all $k,l$. 
\item[Di)] With $\Delta(1) = \sum_{k,l,m} \UnitC{k}{m}\otimes \UnitC{m}{l}$, we have \begin{equation}\label{CondDi}(A\otimes A)\Delta(1) = [(A\otimes 1)\Delta(A)] = [(1\otimes A)\Delta(A)].\end{equation} 
\item[Dii)] With $P=\sum_{k} \UnitC{k}{k}$, and $A_P = PAP$, we have \[[(\omega\otimes \id)\Delta(A_P)\mid \omega \in A^*] = [(\id\otimes \omega)\Delta(A_P)\mid \omega \in A^*] = A.\]
\item[C)] $\Delta$ is coassociative: for all $a,b,c\in A$, we have \[(a\otimes 1\otimes 1)(\Delta\otimes \id)(\Delta(b)(1\otimes c)) = (\id\otimes \Delta)((a\otimes 1)\Delta(b))(1\otimes 1\otimes c).\] 
\end{enumerate}
\end{Def} 

Note that $\Delta(1)$ is a well-defined projection in $M(A\otimes A)$. By condition \eqref{CondDi}, $\Delta$ extends uniquely to a $^*$-homomorphism \[\Delta: M(A)\rightarrow M(A\otimes A)\] with value in the unit precisely $\Delta(1)$. In the same way, $(\id\otimes \Delta)$ and $(\Delta\otimes \id)$ extend to $M(A\otimes A)$, and we can write the coassociativity condition in the usual form \[(\Delta\otimes \id)\Delta = (\id\otimes \Delta)\Delta,\] valid also on $M(A)$.

In the following, we write \[\Gr{A}{k}{l}{m}{n} = \UnitC{k}{m}A\UnitC{l}{n},\] each of which is a Banach subspace of $A$. In particular, each corner $\Gr{A}{k}{k}{m}{m}$ is a unital C$^*$-algebra with unit $\UnitC{k}{m}$. We will write also \[\lambda_k = \sum_{m}\UnitC{k}{m},\qquad \rho_m = \sum_{k}\UnitC{k}{m},\] which give well-defined projections in $M(A)$. Finally, we will write, for $a\in A$, \[\Delta_{rs}(a) = (\rho_r\otimes 1)\Delta(a)(\rho_s\otimes 1) = (1\otimes \lambda_r)\Delta(a)(1\otimes \lambda_s),\] which are well-defined elements in $A\otimes A$ by Definition \ref{DefCpcqg}.Uii). We then have that  \[\Delta(a) = \sum_{r,s} \Delta_{rs}(a)\] in the strict topology. 

\begin{Def} Let $(A,\Delta)$ be a C$^*$-pcqg. An \emph{invariant integral} on $A$ consists of a weight $\phi: A^+ \rightarrow [0,+\infty]$ satisfying the following conditions:
\begin{enumerate}[i)]
\item For all $k,m$ with $\UnitC{k}{m}\neq 0$, \[\phi(\UnitC{k}{m}) = 1.\]
\item For all $a\in A^+$, \[\phi(a) = \sum_{k,m} \phi\left(\UnitC{k}{m}a\UnitC{k}{m}\right).\]
\item For all $a\in A^+$ and all states $\omega\in A^*$, \begin{equation}\label{EqInvL} \phi((\omega \otimes \id)\Delta(a)) = \sum_{k} \omega(\lambda_k)\phi(\lambda_ka\lambda_k),\end{equation} \begin{equation}\label{EqInvR} \phi((\id\otimes \omega)\Delta(a)) = \sum_{m}\omega(\rho_m)\phi(\rho_ma\rho_m).\end{equation}
\end{enumerate} 
\end{Def}


%family of states \[\phi_{km}: \Gr{A}{k}{k}{m}{m}\rightarrow \C\] for each $k,m$ with $\UnitC{k}{m}\neq 0$, so that for each $a\in \Gr{A}{k}{l}{m}{n}$ and each $r$, \[(\id\otimes \phi_{rm})\Delta_{rr}(a) = \phi_{km}(a)\UnitC{k}{r},\qquad (\phi_{kr}\otimes \id)\Delta_{rr}(a) = \phi_{kr}(a)\UnitC{r}{m}.\]
%\end{Def} 

%State zero functional if algebra zero

%Here we interpret $\phi_{rs}(a) =0$ for $a\notin \Gr{A}{r}{r}{s}{s}$. 

Clearly, the formula \[\phi_{km}(a) = \phi(\UnitC{k}{m}a\UnitC{k}{m})\] defines a bounded weight $\phi_{km}$ on $A$, which hence can be seen as a positive functional on $A$. If $\UnitC{k}{m}\neq 0$ it is a state, otherwise it is the zero functional. By abuse of language, we will in the following refer to the complete family of $\phi_{km}$ as `states', so the reader should bear in mind that some of them can be zero functionals.

It is also clear that $\phi$ is completely determined by the family
$\{\phi_{km}\}$.  

In terms of the $\phi_{km}$, the left and right invariance properties
\eqref{EqInvL} and \eqref{EqInvR} take the following form.  For all
$a\in A$,
\begin{align} \label{eq:invariance}
\sum_{k}  \lambda_{k} h_{km}(a) &= \sum_{k}  (\id \otimes
h_{km})(\Delta(a)), &
\sum_{m} \rho_{m} h_{km}(a) &= \sum_{m} (h_{km} \otimes \id)(\Delta(a)),
\end{align}
where the sums converge strictly.

These relations can also  be rewritten in terms of the associative
convolution product on $A^*$  defined by \[(\chi*\omega)(a) =
(\chi\otimes \omega)\Delta(a).\] Let us write \[\Gr{B}{k}{m}{l}{n} = \{\omega \in A^* \mid \forall a\in A, \omega(a) = \omega\left(\UnitC{k}{m}a\UnitC{l}{n}\right)\}.\] Then the convolution product restricts to products \[\Gr{B}{k}{m}{l}{n}\times \Gr{B}{m}{p}{n}{q}\rightarrow \Gr{B}{k}{p}{l}{q},\] all other products being zero. The left and right invariance properties \eqref{EqInvL} and \eqref{EqInvR} can then be written in terms of the $\phi_{km}$ as \begin{equation}\label{EqInvLp}\omega*\phi_{km} = \omega(\UnitC{p}{k})\phi_{pm},\qquad \forall \omega \in \Gr{B}{p}{k}{q}{k},\end{equation}
\begin{equation}\label{EqInvRp}\phi_{km}*\omega = \omega(\UnitC{m}{q})\phi_{kq},\qquad \forall \omega \in \Gr{B}{m}{p}{m}{q}.\end{equation}


We will refer to families of states satisfying \eqref{EqInvLp} as a \emph{left invariant integral}, and to those satisfying \eqref{EqInvRp} as \emph{right invariant integral}.

\begin{Theorem}\label{TheoInvInt} Each $C^*$-pcqg admits a unique invariant integral.
\end{Theorem} 

We will split the proof of the Theorem into several steps, setting the stage so that eventually the arguments of \cite{MVD1} can be applied almost verbatim. 

\begin{Lem} Let $\{\phi_{km}\}$ be a a left invariant integral, and $\{\psi_{km}\}$ a right invariant integral. Then $\phi_{km}= \psi_{km}$ for all $k,m$. 
\end{Lem} 
\begin{proof} By the invariance properties, and the fact that $\UnitC{k}{k}\neq 0$, we have \[\phi_{km}  = \psi_{kk}\left(\UnitC{k}{k}\right)\phi_{km} = \psi_{kk}*\phi_{km}= \phi_{km}\left(\UnitC{k}{m}\right)\psi_{km} = \psi_{km}.\]

\end{proof} 

By the previous Lemma, the unicity in Theorem \ref{TheoInvInt} already follows. It implies as well that it is sufficient to find an invariant left integral for $(A,\Delta)$.

The following lemma will be crucial.

\begin{Lem}\label{LemRefSep} Let $\omega \in \Gr{B}{k}{m}{l}{n}$, and assume $\chi*\omega =  0$ for all $\chi \in \Gr{B}{m}{k}{n}{l}$. Then $\omega =0$.
\end{Lem} 
\begin{proof} By assumption, we have for all $\chi\in A^*$ that \[(\chi\otimes \omega)((\UnitC{m}{k}\otimes \UnitC{k}{m})\Delta(A)(\UnitC{n}{l}\otimes \UnitC{l}{n}))=0.\] But since $\omega = \omega(\UnitC{k}{m}\,\cdot\,\UnitC{l}{n})$, this means, using the notation from Definition \ref{DefCpcqg}.Dii), \[\omega((\chi\otimes \id)(\Delta(A_P))) =(\chi\otimes \omega)((\sum_{k'}\UnitC{m}{k'}\otimes \UnitC{k'}{m})\Delta(A)(\sum_{l'}\UnitC{n}{l'}\otimes \UnitC{l'}{n})) =0.\] By Definition \ref{DefCpcqg}.Dii), we conclude $\omega=0$.
\end{proof} 

\begin{Cor} If $\UnitC{k}{m}=0$, then $\UnitC{m}{k}=0$. 
\end{Cor}
\begin{proof} If $\UnitC{k}{l}=0$, this implies $\Gr{B}{k}{l}{k}{l}=0$. By the previous lemma, this forces also $\Gr{B}{l}{k}{l}{k}=0$, and so $\UnitC{l}{k}=0$. 
\end{proof} 

\begin{Lem} Assume that there exists a family of states $\{\phi_{kk}\}$ in $\Gr{B}{k}{k}{k}{k}$ such that, for any $\omega \in \Gr{B}{k}{k}{k}{k}$, one has \[\omega*\phi_{kk} = \omega\left(\UnitC{k}{k}\right)\phi_{kk}.\] Then $(A,\Delta)$ admits a left invariant state.
\end{Lem}
\begin{proof} Let $\theta_{rm}$ be an arbitary collection of states in $\Gr{B}{r}{m}{r}{m}$ (whenever this algebra is not zero), and write \[\phi_{rm} = \theta_{rm}*\phi_{mm}.\] By assumption, this notation is consistent in the case $r=m$. 

Assume now that $\omega \in \Gr{B}{k}{r}{l}{r}$ and $\chi \in \Gr{B}{m}{k}{m}{l}$. Assume first that $\UnitC{r}{m}\neq 0$ and $\UnitC{k}{m}\neq 0$. Then \begin{eqnarray*} \chi*(\omega*\phi_{rm}) &=& (\chi*\omega*\theta_{rm})*\phi_{mm} \\ &=&  (\chi*\omega*\theta_{rm})(\UnitC{m}{m}) \phi_{mm} \\ &=&  \chi(\UnitC{m}{k})\omega(\UnitC{k}{r})\phi_{mm} \\ &=&  \omega(\UnitC{k}{r}) \; (\chi*\theta_{km})\left(\UnitC{m}{m}\right)\phi_{mm}
\\ &=& \omega(\UnitC{k}{r}) \; (\chi*\theta_{km})*\phi_{mm} \\ &=&  \omega(\UnitC{k}{r}) \; \chi*\phi_{km} .\end{eqnarray*} As $\chi$ was arbitrary, we find by Lemma \ref{LemRefSep} that \begin{equation}\label{EqInvL2} \omega*\phi_{rm} =  \omega(\UnitC{k}{r}) \phi_{km}.\end{equation}

Assume now that $\UnitC{r}{m}=0$. Then also $\Delta_{kk}(\UnitC{r}{m}) = \UnitC{r}{k}\otimes \UnitC{k}{m}=0$, and hence $\UnitC{r}{k}=0$ or $\UnitC{k}{m}=0$. By the previous lemma, we obtain either $\UnitC{k}{r}=0$ or $\UnitC{k}{m}=0$. In either case, both sides of \eqref{EqInvL2} are zero. 

Similarly, if $\UnitC{k}{m}=0$, we conclude that either $\UnitC{k}{r}=0$ or $\UnitC{r}{m}=0$, and again both sides of \eqref{EqInvL2} are zero.

This shows that \eqref{EqInvL2} holds for all indices, and hence $\{\phi_{km}\}$ is a left invariant integral. 
\end{proof} 

Hence Theorem \ref{TheoInvInt} will be proven once we can produce a family of invariant states $\phi_{kk}$ as in the previous lemma. For this, one can follow the proof as in \cite{MVD1} for the existence of an invariant state for a compact quantum group.

\begin{Prop} For each $k\in I$, there exists a state $\phi_{kk}$ in $\Gr{B}{k}{k}{k}{k}$ such that, for any state $\omega \in \Gr{B}{k}{k}{k}{k}$, one has \[\omega*\phi_{kk} =\phi_{kk}.\]
\end{Prop} 
\begin{proof} Let $k\in I$, and $\omega$ a state in $\Gr{B}{k}{k}{k}{k}$. By \cite[Lemma 4.2]{MVD1},
  there exists a state $h_{kk} \in \Gr{B}{k}{k}{k}{k}$ with \[\omega *h_{kk}= h_{kk} =
  h_{kk}*\omega.\]

  Assume now that $\rho\in \Gr{B}{k}{k}{k}{k}$ and $0\leq \rho\leq \omega$. Take $a\in A$. Then the
  beginning of the proof of \cite[Lemma 4.3]{MVD1}, applied with $b= (\id\otimes h_{kk})\Delta(a)$,
  shows that, for all $c\in A$, 
  \begin{equation}\label{EqAuxId} (h_{kk}\otimes
    (\rho*h_{kk}))((c\otimes 1)\Delta(a)) = \rho(1) (h_{kk}\otimes h_{kk})((c\otimes
    1)\Delta(a)).\end{equation} 
Since $(A\otimes A)\Delta(1) = [(A\otimes 1)\Delta(A)]$, we may
  replace $(c\otimes 1)\Delta(a)$ with $\UnitC{k}{k}\otimes a$ for $a\in \Gr{A}{k}{k}{k}{k}$. Then
  \eqref{EqAuxId} becomes $(\rho*h_{kk})(a) = \rho(1)h_{kk}(a)$. Hence \[\rho*h_{kk} =
  \rho(1)h_{kk}.\]

A compactness argument as in \cite[Theorem 4.4]{MVD1} lets us conclude that there exists a state $\phi_{kk}$ as in the statement of the proposition.
\end{proof}


  Assume now that $(A,\Delta)$ is the C$^*$-algebra with comultiplication arising from a $^*$-algebraic pcqg with invariant integral $\phi$. It is easy to see that the conditions in Definition \ref{DefCpcqg} are satisfied. To see that $[(\omega\otimes \id)\Delta(pAp)\mid \omega \in A^*] = A$ for example, for $p = \sum_{k} \UnitC{k}{k}$, note that for the matrix coefficient $(\Gr{X}{k}{l}{m}{n})_{pj}$ of a unitary corepresentation, we have \[ (\Gr{X}{k}{l}{m}{n})_{pj} \in \sum_{q,r,s} \phi((\Gr{X}{m}{n}{r}{s})_{jq}A) (\Gr{X}{r}{s}{m}{n})_{qj} =(\phi(\,\cdot\, A)\otimes \id)\Delta((\Gr{X}{m}{n}{m}{n})_{jj})\] by the orthogonality relations. 


%Also note that for the $(A,\Delta)$ arising from $^*$-pcqg with invariant integral, the above conditions are all satisfied - the fourth condition can be checked using the orthogonality relations for matrix coefficients of unitary corepresentations. 


\section{Corepresentation theory}

\subsection{The $C^{*}$-tensor category of corepresentations}
Let $I$ be a set.

Given an $I^{2}$-graded Hilbert space
$H=\bigoplus_{k,l} \Grd{H}{k}{l}$, we denote by
$p_{kl}^{H} \in \mathcal{B}(H)$ the projections onto the homogeneous components and let
\begin{align*}
  \lambda^{H}_{k} &= \sum_{l} p_{kl}^{H}, &
  \rho^{H}_{l} &= \sum_{k} p_{kl}^{H},
\end{align*}
 the sums converging in the strong operator topology.


 \begin{Def} \label{def:corepresentation} Let $(A,\Delta)$ be a
   $C^{*}$-algebraic $I$-partial compact quantum group. 

   A \emph{unitary corepresentation} of $(A,\Delta)$ on a Hilbert
   space $H$ is a partial isometry $X \in M(A \otimes \mathcal{K}(H))$
   satisfying the following conditions:
   \begin{enumerate}
   \item with $\Delta \otimes \id$ extended to the multiplier algebra, we have
     \begin{align} \label{eq:corep}
     (\Delta \otimes \id)(X) = X_{13}X_{23},  
   \end{align}
 \item there exists an $I^{2}$-grading on $H$ such that
     \begin{align} \label{eq:corep-pi}
       X^{*}X= \sum_{n}\rho_{n} \otimes \rho^{H}_{n} \quad \text{and}
       \quad XX^{*} = \sum_{k} \lambda_{k} \otimes \lambda^{H}_{k}.
     \end{align}
   \end{enumerate}
   We call such a unitary corepresentation \emph{row- and column-finite-dimensional}, briefly
   \emph{rcfd}, if $\lambda^{H}_{k}$ and $\rho^{H}_{m}$ have finite rank for all $k,m\in I$.
\end{Def}
Note that the $I^{2}$-grading on $H$ in  condition 2.\ is uniquely
determined by $X$.

\begin{Exa} \label{exa:corep-trivial}
  Denote by $(e_{km})_{k,m\in I}$ the matrix units in $\mathcal{B}(l^{2}(I))$. Then the sum
  \begin{align*}
    E = \sum_{k,m} \UnitC{k}{m} \otimes e_{km} 
  \end{align*}
  converges strictly in $M(A\otimes \mathcal{K}(l^{2}(I)))$ to a unitary rcfd corepresentation, as
  one can easily check. The associated $I^{2}$-grading on $l^{2}(I)$ is the diagonal grading, that
  is, $p^{l^{2}(I)}_{km} =\delta_{k,m} e_{kk}$. We call $E$ the \emph{trivial corepresentation} of
  $(A,\Delta)$.
\end{Exa}

\begin{Lem} \label{lem:corep-intertwine}
 Let $X$ be a unitary corepresentation of $(A,\Delta)$ on a Hilbert space $H$. Then  
    \begin{align*}
      (\UnitC{k}{m} \otimes 1)X(\UnitC{l}{n} \otimes 1) = (1 \otimes p^{H}_{kl})X(1 \otimes
      p^{H}_{mn})
    \end{align*}
   for all $k,l,m,n \in I$.
\end{Lem}
\begin{proof}
 Conditions 1.\ and 2.\ in the definition above imply
  \begin{align*}
 \sum_{l,n} \rho_{l} \otimes
  \UnitC{l}{n} \otimes \rho_{n}^{H} &=
    \sum_{n} \Delta(\rho_{n}) \otimes \rho_{n}^{H} \\ &= (\Delta \otimes
    \id)(X^{*}X) \\ &= X^{*}_{23}X^{*}_{13}X_{13}X_{23}  =
    \sum_{l}\rho_{l} \otimes X^{*}(1 \otimes \rho_{l}^{H})X.
  \end{align*}
  We multiply on the left by $X$, use  condition 2.\, and deduce
  $X(\lambda_{l}\otimes 1)=(1\otimes \rho_{l}^{H})X$ for all $l$.
  This relation and \eqref{eq:corep} imply
 \begin{align*}
   X_{13}(1\otimes 1 \otimes \rho_{l}^{H})X_{23} &=X_{13}X_{23}(1
   \otimes \lambda_{l} \otimes 1) \\ &= X_{13}X_{23}(\rho_{l} \otimes 1
   \otimes 1) = X_{13} (\rho_{l} \otimes 1
   \otimes 1) X_{23},
 \end{align*}
 and multiplying by $X_{23}^{*}$ on the right we conclude $X(1\otimes
 \rho^{H}_{l})=X(\rho_{l} \otimes 1)$.


  Similarly,  the relation
  \begin{align*}
    \sum_{k,m} \UnitC{k}{m} \otimes \lambda_{m} \otimes
    \lambda_{m}^{H} &= (\Delta \otimes \id)(XX^{*}) = \sum_{m}
    X_{13}(1 \otimes \lambda_{m} \otimes \lambda_{m}^{H})X_{13}^{*}
  \end{align*}
  implies $(\rho_{m} \otimes 1)X=X(1\otimes \lambda^{H}_{m})$ for all
  $m$, and then
  \begin{align*}
    X_{13}(1 \otimes 1 \otimes \lambda^{H}_{m})X_{23} &= (
    \rho_{m} \otimes 1 \otimes 1)X_{13}X_{23} \\ &= (1\otimes
    \lambda_{m}\otimes 1)X_{13}X_{23} = X_{13}(1 \otimes \lambda_{m}
    \otimes 1)X_{23}
  \end{align*}
  implies, after multiplying by $X_{13}^{*}$ on the left, $(1 \otimes
  \lambda_{m}^{H})X=(\lambda_{m} \otimes 1)X$.
\end{proof}

\begin{Def} \label{def:intertwiner}
  Let $(A,\Delta)$ be a $C^{*}$-pcqg with unitary corepresentations $X$ and $Y$ on Hilbert spaces
  $H$ and $K$, respectively.  Then an \emph{intertwiner} from $X$ to $Y$ is an operator $T\in
  \mathcal{B}(H,K)$ satisfying   $Y(1\otimes T)=(1 \otimes T)X$.
\end{Def}
The unitary corepresentations of $(A,\Delta)$ with intertwiners as morphisms form a category.  We
denote by $\Corep(A,\Delta)$ this category, and by $\Corepf(A,\Delta)$ the full subcategory formed
by all unitary rcfd corepresentations. The following result shows that for every intertwiner $T$,
the adjoint $T^{*}$ is an intertwiner as well.
\begin{Lem} \label{lem:def-intertwiner}
  Let $(A,\Delta)$ be a $C^{*}$-pcqg with unitary corepresentations $X$ and $Y$ on Hilbert spaces
  $H$ and $K$, respectively, and let $T\in \mathcal{B}(H,K)$. Then the following conditions are equivalent:
  \begin{enumerate}
  \item $T$ intertwines $X$ and $Y$, that is, $Y(1\otimes T)=(1 \otimes T)X$;
  \item $Y^{*}(1\otimes T)X=\sum_{l} \rho_{l} \otimes \rho^{K}_{l}T$ and $\lambda_{m}^{K}T=T\lambda_{m}^{H}$ for all $l$;
  \item $Y(1\otimes T)X^{*} = \sum_{k} \lambda_{k} \otimes \lambda^{K}_{k}T$ and
    $\rho_{n}^{K}T=\rho_{n}^{H}$ for all $k$.
  \end{enumerate}
  In particular,  these conditions imply that $Tp_{mn}^{H}=p_{mn}^{K}T$ for all $m,n$ and that
  $T^{*}$ intertwines $Y$ and $X$.
\end{Lem}
\begin{proof}
  Assume 1.\ holds. Then $Y^{*}(1\otimes T)X=\sum_{l} \rho_{l} \otimes \rho^{K}_{l}T$ by condition
  2.\ in Definition \ref{def:corepresentation}, and
  \begin{align*}
    Y(1\otimes T\lambda^{H}_{m}) = (\rho_{m} \otimes T)X = (\rho_{m} \otimes 1)Y(1\otimes T) =
    Y(1\otimes \lambda^{K}_{m}T)
  \end{align*}
which implies $T\lambda^{H}_{m} = \lambda^{K}_{m}T$ because of  condition
  2.\ in Definition \ref{def:corepresentation} again. 
Conversely, assume 2.\ holds. Then the first equation implies $YY^{*}(1\otimes T)X = Y(1\otimes T)$,
and the second implies $YY^{*}(1\otimes T)X=(1\otimes T)X$.

 A similar argument shows that conditions 1.\ and 3.\ are equivalent.
\end{proof}
Therefore, the categories $\Corep(A,\Delta)$ and $\Corepf(A,\Delta)$ are
$C^{*}$-categories. Furthermore, they admit the following tensor product.
\begin{Lem}
  Let $X$ and $Y$ be unitary corepresentations of a $C^{*}$-pcqg $(A,\Delta)$ on Hilbert spaces $H$,
  $K$. Then
    \begin{align*}
      X \Circt Y:= X_{12}Y_{13} \in M(A \otimes \mathcal{K}(H \otimes K))
    \end{align*}
    is a unitary corepresentation of $(A,\Delta)$ on the Hilbert space
    \begin{align*}
      H \itimes K := \bigoplus_{k,l,m} \Grd{H}{k}{l} \otimes \Grd{K}{l}{m},
    \end{align*}
    where the  associated $I^{2}$-grading is given by  $k$ and $m$.
\end{Lem}
\begin{proof}
Clearly,
$(\Delta \otimes \id)(X_{12}Y_{13}) = X_{13}X_{23}Y_{14}Y_{24} =
    X_{13}Y_{14}X_{23}Y_{24}$,
and Lemma \ref{lem:corep-intertwine} implies
\begin{align*}
  X_{12}Y_{13}Y_{13}^{*}X_{12}^{*} &= \sum_{l} X_{12}^{*}(\lambda_{l}
  \otimes 1)X_{12} \otimes \lambda_{l}^{K} = \sum_{k,l} \lambda_{k}
  \otimes p_{kl}^{H} \otimes \lambda_{l}^{K}, \\
  Y_{13}^{*}X_{12}^{*}X_{12}Y_{13} &= \sum_{l} Y_{13}^{*}(\rho_{l}
  \otimes \rho_{l}^{H}\otimes 1)Y_{13}  = \sum_{l} \rho_{m} \otimes
  \rho_{l}^{H} \otimes p_{lm}^{K}. 
\end{align*}
The assertions follow. 
\end{proof}
If $S$ and $T$ are intertwiners between unitary corepresentations $X,Y$ and $U,V$, respectively,
then $S\otimes T$ restricts to an intertwiner between $X\Circt U$ and $Y\Circt V$.  
Thus, the category $\Corep(A,\Delta)$ becomes a $C^{*}$-tensor category with the trivial corepresentation $E$ as the unit. 
Indeed, for any unitary corepresentation $X$ on a Hilbert space $H$, the natural isomorphisms
\begin{align*}
  l^{2}(I) \itimes H \cong H  \quad \text{and} \quad H \itimes l^{2}(I)\cong H
\end{align*}
intertwine $E \Circt X$ and $X\Circt E$, respectively, with $X$, as one can easily check.

If $X$ and $Y$ are unitary rcfd corepresentations, then so is $X\Circt Y$, and therefore,  the subcategory
$\Corepf(A,\Delta)$ is a $C^{*}$-tensor category as well.


\subsection{Decomposition into irreducible corepresentations}

We show that the $C^{*}$-tensor category $\Corep(A,\Delta)$ is semi-simple, that is, every unitary
corepresentation decomposes into a direct sum of irreducible unitary corepresentations, following
the approach in \cite{MVD1}. Furthermore, we show that every irreducible unitary corepresentation is rcfd.

\begin{Def}
Let $X$ be a unitary corepresentation of a $C^{*}$-pcqg $(A,\Delta)$ on a
  Hilbert space $H$. We call a closed subspace $K\subseteq H$
  \emph{invariant} if $K=\bigoplus_{k,l} (K \cap \Grd{H}{k}{l})$ and
  the orthogonal projection $P\in \mathcal{B}(H)$ onto $K$ satisfies
  $(1\otimes P)X(1\otimes P)=X(1\otimes P)$.  We call $X$
  \emph{irreducible} if there exists no non-trivial closed invariant
  subspace $K\subseteq H$.
\end{Def}
As in \cite{MVD1}, the  following result will imply that every invariant closed subspace is complemented:
 \begin{Prop}\label{prop:corep-complemented}
   Let $(A,\Delta)$ be a $C^{*}$-pcqg with invariant integral $h$ and
   let $X$ be a unitary corepresentation of $(A,\Delta)$. Then the
   subspace
   \begin{align*}
     B &:= [ \{(h_{km} \otimes
     \id)(X(a\otimes 1)) : a \in A,\, k,m\in I\}] \subseteq \mathcal{B}(H)
   \end{align*}
   is a non-degenerate $C^{*}$-subalgebra and $X\in M(A\otimes B)$.
 \end{Prop}
  \begin{proof}
We first show that $[B^{*}B]= B$. This relation implies $B=B^{*}$ and
$[BB]=B$ so that $B$ is a $C^{*}$-subalgebra of $\mathcal{B}(H)$. 

 Let $a\in \Gr{A}{k}{l}{m}{n}$, $b\in \Gr{A}{p}{q}{r}{s}$ and
 \begin{align*}
   S&:=(h_{ln} \otimes \id)(X(a\otimes 1)), &
   T&:=(h_{qs} \otimes \id)(X(b\otimes 1)).
 \end{align*}
Then  $S \in \mathcal{B}(\Grd{H}{n}{m},\Grd{H}{l}{k})$, $T\in
\mathcal{B}(\Grd{H}{s}{r},\Grd{H}{q}{p})$ and
\begin{align*}
  S^{*}T = \delta_{q,l}\delta_{p,k} (h_{ln} \otimes \id)((a^{*}\otimes
  1)X^{*}(1\otimes T)).
\end{align*}
Assume that $(p,q)=(k,l)$.  Then \eqref{eq:invariance} implies
\begin{align*}
 X^{*}(\rho_{n}\otimes T) &=  X^{*}(\UnitC{l}{n} \otimes (h_{ls} \otimes
  \id)(X(b\otimes 1))) \\
  &=  X^{*}(\id \otimes h_{ns}\otimes
  \id)((\Delta\otimes \id)(X(b\otimes 1))) \\
  &=    (\id \otimes h_{ns}\otimes
  \id)(X^{*}_{13}X_{13}X_{23}(\Delta(b)\otimes 1)) \\
  &= \sum_{t}(\id \otimes h_{ns} \otimes
  \id)((\rho_{t} \otimes 1\otimes \rho^{H}_{t})X_{23}(\Delta(b)\otimes
  1)) \\
&= \sum_{t}(\id \otimes h_{ns} \otimes
  \id)(X_{23}((\rho_{t}\otimes \lambda_{t})\Delta(b)\otimes
  1)), 
\end{align*}
that is,
\begin{align}\label{eq:corep-complemented}
  X^{*}(\rho_{n}\otimes T) &=
 (\id \otimes h_{ns} \otimes
  \id)(X_{23}(\Delta(b)\otimes 1)).
\end{align}
Thus,
\begin{align*}
  S^{*}T &= (h_{ln} \otimes \id)((a^{*}\otimes
  1)X^{*}(1\otimes T))  \\ &= (h_{ln} \otimes h_{ns} \otimes
  \id)((a^{*} \otimes 1\otimes 1)X_{23}(\Delta(b) \otimes 1)) 
  =(h_{ns} \otimes 1)(X(c\otimes 1)), 
\end{align*}
where $c=(h_{ln} \otimes \id)((a^{*}\otimes 1)\Delta(b))$. Therefore,
$S^{*}T \in B$. Since $[(A\otimes 1)\Delta(A)]=[(A\otimes
A)\Delta(A)]$, we can conclude that $[B^{*}B]=B$.


To see that $B$ is non-degenerate, observe that $X(A\otimes
\mathcal{K}(H))$ contains $\lambda_{k}A \otimes
\lambda_{k}^{H}\mathcal{K}(H)$ for all $k$ and hence
$[B\mathcal{K}(H)]=\mathcal{K}(H)$.

We finally show that $X \in M(A\otimes B)$.  Let $a,b$ as above but
allow $(p,q)\neq (k,l)$ again. 
Then
equation \eqref{eq:corep-complemented}  implies 
\begin{align*}
  X^{*}(a\otimes T) = X^{*}(\rho_{m}a\otimes T) = (\id \otimes h_{ms}
  \otimes \id)(X_{23}(\Delta(b)(a\otimes 1))\otimes 1).
\end{align*}
Since $\Delta(b)(a\otimes 1) \subseteq A\otimes A$, the right hand
side belongs to $A\otimes B$. Thus, $X^{*}(A\otimes B) \subseteq
A\otimes B$. On the other hand, 
\begin{align*}
  X(a\otimes T) &= (\id \otimes h_{qs} \otimes
  \id)(X_{13}X_{23}(a\otimes b\otimes 1)) \\ &= (\id \otimes h_{qs}
  \otimes \id)((\Delta \otimes \id)(X)(a\otimes b\otimes 1)) 
\end{align*}
Since $\Delta(1)(A \otimes A) = [\Delta(A)(A \otimes 1)]$, we can
approximate $(\Delta \otimes \id)(X)(a\otimes b\otimes 1)$ in norm by sums of
products of the form $(\Delta \otimes\id)(X(c\otimes 1))(d \otimes
1\otimes 1)$, where $c\in \Gr{A}{k}{t}{m}{n}$, $d \in
\Gr{A}{t}{l}{}{n}$ and $t\in I$. But \eqref{eq:invariance} implies that for such $c,d,t$,
\begin{align*}
(\id \otimes h_{qs}
  \otimes \id)(  (\Delta \otimes\id)(X(c\otimes 1)))(d \otimes
1) &= \UnitC{t}{q}d \otimes (h_{ks} \otimes \id)(X(c\otimes 1)) \in A
\otimes B.
\end{align*}
Therefore, $X(A \otimes B) \subseteq A\otimes B$.
  \end{proof}

Similarly as in the case of compact quantum groups, Proposition \ref{prop:corep-complemented} implies
that a closed subspace is invariant if and only if the orthogonal
projection onto this subspace is an intertwiner, and then a
straightforward application of Zorn's Lemma shows:
\begin{Prop} \label{prop:corep-decompose}
  Every unitary corepresentation of a $C^{*}$-pcqg is a direct sum of
  irreducible unitary corepresentations.
\end{Prop}
We next show that every irreducible unitary corepresentation
is rcfd, using an averaging construction, which
will also be used to prove the Schur orthogonality relations later on.
\begin{Lem} \label{lem:intertwiner-averaged}
  Let $X$ and $Y$ be unitary corepresentations of $(A,\Delta)$ on Hilbert spaces $H$ and $K$,
  respectively, and let $T \in \mathcal{B}(H,K)$. Then for all $m,l \in I$, the sums
  \begin{align*}
    \hat T^{(m)} &:= \sum_{k} (h_{km} \otimes \id)(Y(1 \otimes T)X^{*}) \quad \text{and} \quad
    \check T^{(l)}=\sum_{n} (h_{ln}\otimes \id)(Y^{*}(1 \otimes T)X)
  \end{align*}
  converge in the strong operator topology to intertwiners $\hat
  T^{(m)}$ and $\check T^{(l)}$ from $X$ to $Y$.
\end{Lem}
\begin{proof}
  We only prove the assertion concerning $\hat T^{(m)}$; the proof for $\check T^{(l)}$ is similar.
  
  The sum defining $\hat T^{(m)}$ converges in the strong operator topology because by
  \eqref{eq:corep-pi},
  \begin{align*}
    (h_{km} \otimes \id)(Y(1 \otimes T)X^{*}) = \lambda^{H}_{k}
    (h_{km} \otimes \id)(Y(1 \otimes T)X^{*}) \lambda^{H}_{k}.
  \end{align*}
  To see that $\hat T^{(m)}$ is an intertwiner, we use invariance of $h$ and find
  \begin{align*}
    Y(1 \otimes \hat T^{(m)})X^{*} &=
    \sum_{k} (\id \otimes h_{km} \otimes \id)(Y_{13}Y_{23}(1 \otimes T)X^{*}_{23}X^{*}_{13}) \\
    &= \sum_{k} ((\id \otimes h_{km})\circ \Delta \otimes \id)(Y(1 \otimes T)X^{*}) \\
    &=\sum_{l} \lambda_{l} \otimes (h_{lm} \otimes \id)(Y(1\otimes T)X^{*}) \\
    &= \sum_{l} \lambda_{l}   \otimes \lambda^{H}_{l} \hat T^{(m)}.
  \end{align*}
  With \eqref{eq:corep-pi}, we conclude that $Y(1\otimes \hat T^{(m)})
  = (1 \otimes \hat T^{(m)}X$.
\end{proof} 
\begin{Lem} \label{lem:intertwiner-compact} Let $X$ and $Y$ be unitary corepresentations of
  $(A,\Delta)$ on Hilbert spaces $H$ and $K$, respectively, let $T \in \mathcal{K}(H,K)$ and define
  $\hat T^{(m)}$ and $\check T^{(l)}$ as above. Then $\rho^{H}_{n}\hat T^{(m)}$ and
  $\lambda^{H}_{k}\check T^{(l)}$ are compact for all $k,n\in I$.
\end{Lem}
\begin{proof}
  We only prove the assertion concerning $\hat T^{(m)}$; a similar reasoning applies to $\check
  T^{(l)}$.

By Lemma \ref{lem:corep-intertwine} and
  \begin{align*}
    \rho^{H}_{n}(h_{km} \otimes \id)(Y(1\otimes T)X^{*}) = 
    (h_{km} \otimes \id)(Y(\lambda_{n} \otimes T)X^{*}),
  \end{align*}
and  by \eqref{eq:corep-pi}, 
\begin{align*}
    Y(\lambda_{n} \otimes T)X^{*} =  \sum_{p} Y(\UnitC{n}{p} \otimes \rho^{H}_{p}T)X^{*}.
  \end{align*}
  Since $T$ is compact, the sum converges in norm and each summand $Y(\UnitC{n}{p} \otimes
  \rho^{H}_{p}T)X^{*}$ lies in $A \otimes \mathcal{K}(H,K)$. Thus,
  \begin{align*}
    Y(\lambda_{n} \otimes T)X^{*} \in A \otimes \mathcal{K}(H,K).
  \end{align*}
 We can therefore approximate $Y(\lambda_{n}
  \otimes T)X^{*}$ by finite sums $\sum_{i} a_{i} \otimes S_{i}$,
  where $a_{i} \in A$, $S_{i} \in \mathcal{K}(H,K)$, and by
  \eqref{eq:corep-pi}, we may assume $a_{i}=\lambda_{k_{i}}a_{i}
  \lambda_{l_{i}}$ and $S_{i}=\lambda_{k_{i}}^{K}S\lambda_{l_{i}}^{H}$
  for some $k_{i},l_{i}\in I$. Then
  \begin{align*}
\sum_{i,k}    (h_{km} \otimes \id)(a_{i} \otimes S_{i}) = \sum_{i}
h_{k_{i}m}(a_{i})S_{i} \in \mathcal{K}(H,K)
  \end{align*}
and  $  \rho^{H}_{n}\hat T^{(m)} - \sum_{i}
h_{k_{i}m}(a_{i})S_{i}$ is equal to the sum of the elements
\begin{align*}
  D_{k}:=   (h_{km} \otimes \id)\left(Y(\lambda_{n} \otimes T)X^{*} -
    \sum_{i}a_{i} \otimes S_{i}\right).
\end{align*}
Now, \eqref{eq:corep-pi} and the assumption on the $a_{i}$ and $S_{i}$
implies $D_{k} = \lambda_{k}^{H}D_{k}\lambda_{k}^{H}$, and  since each $h_{km}$ is a state,
\begin{align*}
  \|D_{k}\| \leq \left\|Y(\lambda_{n} \otimes T)X^{*} -
    \sum_{i}a_{i} \otimes S_{i}\right\| \quad\text{for all } k.
\end{align*}
Therefore, 
\begin{align*}
\left\|    \rho^{H}_{n}\hat T^{(m)} - \sum_{i}
h_{k_{i}m}(a_{i})S_{i}\right\| \leq \left\|\sum_{k} D_{k}\right\| \leq \left\|Y(\lambda_{n} \otimes T)X^{*} -
    \sum_{i}a_{i} \otimes S_{i}\right\|.
\end{align*}
Since the right hand side can be made arbitrarily small, we can
conclude that $\rho^{H}_{n}\hat T^{(m)}$ is compact. 
\end{proof}
\begin{Prop} \label{prop:corep-rcfd}
Every irreducible unitary corepresentation of a  $C^{*}$-pcqg is rcfd.
\end{Prop}
\begin{proof}
  Let $(A,\Delta)$ be a $C^{*}$-pcqg with invariant integral $h$ and
  let $X$ be a unitary irreducible corepresentation of $(A,\Delta)$ on
  a Hilbert space $H$.

  We claim that there exists a non-zero intertwiner of $S$ of $X$ such
  that $\rho^{H}_{n}S$ is compact for all $n$.  Indeed, choose
  projections $T_{i} \in \mathcal{K}(H)$ converging strictly to
  $\id_{H}$ and fix $m\in I$. Then products $X(1\otimes T_{i})X^{*}$
  converge strictly to $\sum_{k} \lambda_{k} \otimes \lambda_{k}^{H}$
  and the sums
  \begin{align*}
    \hat T_{i} := \sum_{k}(h_{km} \otimes \id)(X(1\otimes T_{i})X^{*})
  \end{align*}
  converge strictly to $\id_{H}$ in $M(\mathcal{K}(H))$. In
  particular, we find some $i$ such that $\hat T_{i}\neq 0$.  By Lemma 
  \ref{lem:intertwiner-averaged} and  Lemma
  \ref{lem:intertwiner-compact}, we can now take $S:=\hat T_{i}$.

  Since $\rho^{H}_{n}S$ is compact for all $n$ and $S$ is non-zero, we
  find $n \in I$ and a non-zero $\lambda\in \C$ such that the
  intertwiner $S-\lambda\id_{H}$ of $X$ has non-trivial kernel.  But
  this kernel is an invariant subspace and therefore must be $H$.
  Thus, $S=\lambda\id_{H}$ and hence $\rho^{H}_{n}$ is compact for all
  $n$.


  A similar argument shows $\lambda^{H'}_{l}S$ is compact for all
  $l$. Therefore, $X$ is rcfd.
\end{proof}



\subsection{The regular corepresentation}
 Let $(A,\Delta)$ be a $C^{*}$-pcqg with an invariant integral $h$. 
 Then we can construct a left-regular corepresentation of $(A,\Delta)$ as follows.

 
 We denote by $H_{h}$ the associated GNS-space, by $\zeta^k_m \in H_{h}$ the image of $\UnitC{k}{m}
 \in A$, and write the GNS-representation of $A$ on $H_{h}$ as left multiplication.  We equip 
 $H_{h}$ with the $I^{2}$-grading given by
 \begin{align*}
   \lambda^{H_{h}}_{m}  (a\zeta^{k}_{l}) &=  \rho_{m}a \zeta^{k}_{l}, &
   \rho^{H_{h}}_{m} (a\zeta^{k}_{l}) &= a\rho_{m} \zeta^{k}_{l} = \delta_{m,l} a\zeta^{k}_{l}
 \end{align*}
 for all $k,l,m\in I$ and $a\in A$.

 Choose a non-degenerate, faithful representation of $A$ on some Hilbert space $K$ and identify $A$
 with its image.
 \begin{Lem}\label{lem:reg-corep-pi}
   There exists a unique partial isometry $\tilde V$ on $H_{h} \otimes
   K$ such
   that
   \begin{align*}
     \tilde V(a \zeta^k_m \otimes \eta) &=
     \Delta(a) \sum_{l}(\zeta^{k}_{l} \otimes \UnitC{l}{m}\eta)
   \end{align*}
   for all $a\in A$, $k,m\in I$ and $\eta\in K$. Its support and range
   projections are given by
   \begin{align*}
     \tilde V^{*}\tilde V &= \sum_{l} \rho^{H_{h}}_{l} \otimes \rho_{l}, &
     \tilde V \tilde V^{*} &= \sum_{k} \lambda^{H_{h}}_{k} \otimes \lambda_{k}.
   \end{align*}
 \end{Lem}
 \begin{proof}
   Let $p,q\in I$ and $a,k,m,\eta$ as above. Given $\xi \in K$, denote by $\omega_{\xi}\in A^{*}$ the vector
   functional $a \mapsto \langle \xi|a\xi\rangle$.
   Then $\omega_{1\Grru{l}{m}\eta} \in \Gr{B}{l}{m}{l}{m}$ and by \eqref{EqInvRp},
   \begin{align*}
  \left\|\Delta(a)\sum_{l}(\zeta^{k}_{l}\otimes \UnitC{l}{m}\eta)\right\|^{2} &= \sum_{l} (h_{kl}
  \otimes \omega_{1\Grru{l}{m}\eta})(\Delta(a^{*}a)) \\
  &= \sum_{l}\omega_{\eta}(1\Grru{l}{m}) h_{km}(a^{*}a) 
  = \left\|a\zeta^{k}_{m} \otimes  \rho_{m}\eta \right\|^{2}.
\end{align*}
Therefore, the formula above defines a partial isometry $\tilde V$ with support projection $\sum_{m}
\rho^{H_{h}}_{m} \otimes \rho_{m}$. The image of $\tilde V$ is spanned by elements of the form
\begin{align*}
\sum_{l}   \Delta(a)(1 \otimes \UnitC{l}{m}) (\zeta^{k}_{l} \otimes b\eta),
\end{align*}
where $a,b\in A$, $k,l,m\in I$, $\eta\in K$. By  \eqref{CondDi}, this is equal to the span of all
elements of the form
\begin{align*}
  \sum_{l} (\rho_{m}c \zeta^{k}_{l} \otimes \lambda_{m}d\eta),
\end{align*}
where $c,d\in A$, $k,l,m \in I$, $\eta\in K$. Since $\rho_{m}c\zeta^{k}_{l} =
\lambda^{H_{h}}_{m}(c\zeta^{k}_{l})$, we can conclude that $VV^{*}=\sum_{m} \lambda^{H_{h}}_{m}
\otimes \lambda_{m}$ as claimed.
 \end{proof}
 \begin{Lem}\label{lem:reg-corep-mult}
   The products $ \tilde V(\mathcal{K}(H_{h})\otimes A)$ and 
   $(\mathcal{K}(H_{h}) \otimes A)\tilde V$ lie in $A \otimes
   \mathcal{K}(H)$ so that $\tilde V \in M(A\otimes \mathcal{K}(H))$.
 \end{Lem}
 \begin{proof}
   Given $\xi,\eta\in H_{h}$, denote by $|\xi\rangle\langle \eta| \in \mathcal{K}(H_{h})$ ket-bra
   operator $\vartheta \mapsto \langle \eta|\vartheta\rangle \xi$. 

   Let $a\in A$, $k,m,r,s\in I$, $\xi,\vartheta \in H_{h}$ and
   $\eta\in K$. 
   
   We first show that $\tilde V(\mathcal{K}(H_{h})\otimes A) \subseteq
   \mathcal{K}(H_{h}) \otimes A$. By definition,\begin{align*}
     \tilde V(|a\zeta^{k}_{m}\rangle\langle \xi| \otimes \UnitC{r}{s})(\vartheta \otimes \eta) 
     &=     \delta_{s,m} \langle \xi|\vartheta\rangle \Delta(a)(1 \otimes \UnitC{r}{s})(\zeta^{k}_{r}
     \otimes \eta).
   \end{align*} 
   Here, we can approximate $\Delta(a)(1\otimes \UnitC{r}{s}) \in A\otimes A$ by finite sums $\sum_{i} c_{i} \otimes
   d_{i}$ with $c_{i},d_{i} \in A$, and
   \begin{align*}
 \langle \xi|\vartheta\rangle (c_{i} \otimes
d_{i})(\zeta^{k}_{r} \otimes \eta) = 
 (|c_{i}\zeta^{k}_{r}\rangle\langle \xi| \otimes d_{i}) (\vartheta \otimes \eta).
   \end{align*}
But  $\sum_{i} (|c_{i}\zeta^{k}_{r}\rangle\langle \xi| \otimes d_{i}) \in \mathcal{K}(H_{h}) \otimes
A$  and   one can check that
   \begin{align*}
     \left\|\tilde V(|a\zeta^{k}_{m}\rangle\langle \xi| \otimes \UnitC{r}{s}) -      \sum_{i}
       (|c_{i}\zeta^{k}_{r}\rangle\langle \xi| \otimes d_{i})\right\| \leq
     \left\|\Delta(a)(1\otimes \UnitC{r}{s}) -  \sum_{i} c_{i} \otimes
   d_{i}\right\|.
   \end{align*}
   Therefore, $\tilde V(|a\zeta^{k}_{m}\rangle\langle \xi| \otimes \UnitC{r}{s})$ lies in
   $\mathcal{K}(H_{h}) \otimes A$ and hence $\tilde V(\mathcal{K}(H_{h})\otimes A) \subseteq
   \mathcal{K}(H_{h}) \otimes A$.
   

   We next show that $\tilde V^{*}(\mathcal{K}(H_{h}) \otimes A)
   \subseteq \mathcal{K}(H_{h}) \otimes A$.  By Lemma
   \ref{lem:reg-corep-pi},
   \begin{align*}
     \tilde V^{*}(|a\zeta^{k}_{m}\rangle\langle\xi| \otimes
     \UnitC{r}{s})(\vartheta\otimes \eta) &=  \langle
     \xi|\vartheta\rangle \tilde V^{*}(\rho_{r}a\UnitC{k}{m}\otimes
     \UnitC{r}{s})(\zeta^{k}_{m} \otimes \eta).
   \end{align*}
   By assumption \eqref{CondDi}, we can approximate $\rho_{r}a
   \UnitC{k}{m}\otimes
   \UnitC{r}{s}$ by finite sums $\sum_{i} \Delta(c_{i})(1 \otimes d_{i})$ with
   $c_{i}, d_{i} \in A$, such that $c_{i} =c_{i}\rho_{l_{i}}$ and
   $d_{i}=\rho_{l_{i}}d_{i}$ for some $l_{i} \in I$, and then
   \begin{align*}
     \langle \xi|\vartheta\rangle \tilde
     V^{*}\Delta(c_{i})(\zeta^{k}_{m}\otimes d_{i}\eta) = \langle
     \xi|\vartheta\rangle c_{i}\zeta^{k}_{l_{i}} \otimes d_{i}\eta =
     (|c_{i}\zeta^{k}_{l_{i}}\rangle\langle\xi| \otimes
     d_{i})(\vartheta\otimes \eta).
   \end{align*}
But $\sum_{i}     (|c_{i}\zeta^{k}_{l_{i}}\rangle\langle\xi| \otimes
     d_{i}) \in \mathcal{K}(H_{h}) \otimes A$ and again, one can check
     that
     \begin{align*}
       \left\|      \tilde V^{*}(|a\zeta^{k}_{m}\rangle\langle\xi| \otimes
     \UnitC{r}{s}) - \sum_{i}     (|c_{i}\zeta^{k}_{l_{i}}\rangle\langle\xi| \otimes
     d_{i})  \right\| \leq \left\|
\rho_{r}a
   \UnitC{k}{m}\otimes
   \UnitC{r}{s} - \sum_{i} \Delta(c_{i})(1 \otimes d_{i})
   \right\|.
     \end{align*}
     Therefore, $   \tilde V^{*}(|a\zeta^{k}_{m}\rangle\langle\xi| \otimes
     \UnitC{r}{s})$ lies in $\mathcal{K}(H_{h}) \otimes A$ and the
     second inclusion follows.
\end{proof}
\begin{Rem}
  We also have $(1 \otimes A)\tilde V(\mathcal{K}(H_{h})\otimes 1)
  \subseteq \mathcal{K}(H_{h}) \otimes A$. Indeed, for all $r,s\in I$,
   \begin{align*}
     (1 \otimes \UnitC{r}{s})\tilde
     V(|a\zeta^{k}_{m}\rangle\langle\xi|\otimes 1)(\vartheta \otimes
     \eta) = \langle \xi|\vartheta \rangle(1 \otimes
     \UnitC{r}{s})\Delta(a) \sum_{l} (\zeta^{k}_{l} \otimes
     \UnitC{l}{m}\eta),
   \end{align*}
where we can approximate $(1
   \otimes \UnitC{r}{s})\Delta(a) \in A\otimes A$ by finite sums
   $\sum_{i} c_{i} \otimes d_{i}$ with homogeneous $c_{i},d_{i} \in
   A$, and
   \begin{align*}
     \sum_{i,l} \langle \xi|\vartheta\rangle (c_{i} \otimes d_{i})(\zeta^{k}_{l} \otimes
     \UnitC{l}{m} \eta) = \sum_{i,l} (|c_{i}\zeta^{k}_{l}\rangle\langle \xi| \otimes
     d_{i}\UnitC{l}{m}) (\vartheta \otimes \eta),
   \end{align*}
   where the sum becomes finite by homogenity of the $c_{i}$ and $d_{i}$. Now, one concludes
   similarly as above that $(1 \otimes \UnitC{r}{s})\tilde
   V(|a\zeta^{k}_{m}\rangle\langle\xi|\otimes 1) \in \mathcal{K}(H_{h}) \otimes A$ and consequently
   $(1\otimes A)\tilde V(\mathcal{K}(H_{h}) \otimes 1) \subseteq \mathcal{K}(H_{h}) \otimes A$.
   
However,   we do not know whether $(\mathcal{K}(H_{h}) \otimes 1)
   \tilde V(1\otimes A)$ is contained in $A\otimes \mathcal{K}(H)$.
\end{Rem}
Flipping $\tilde V$, we obtain the \emph{regular corepresentation} $V$ of $(A,\Delta)$.
 \begin{Prop}
Let $(A,\Delta)$ be a $C^{*}$-pcqg with invariant integral $h$. Then there exists a unique unitary
corepresentation $V$ of $(A,\Delta)$ on the associated GNS-space
$H_{h}$ such that for every faithful, nondegenerate representation of
$A$ on a Hilbert space $K$,
\begin{align*}
  V(\eta  \otimes a\zeta^{k}_{m}) &= \Delta^{\op}(a) \sum_{l}(\UnitC{l}{m}\eta
  \otimes \zeta^{k}_{l})
\end{align*}
for all $\eta\in K$ and $a\in A$, where $\zeta^{k}_{m}$ denotes the
image of  $\UnitC{k}{m}$ in $H_{h}$ and $\Delta^{\op}$ the composition
of $\Delta$ with the flip map.
 \end{Prop}
 \begin{proof}
   Denote by $\sigma$ the flip $ M(\mathcal{K}(H) \otimes A) \to M(A
   \otimes \mathcal{K}(H))$ and let $V=\sigma(\tilde V)$. Then the
   preceding lemmas show that $V$ lies in $M(A\otimes
   \mathcal{K}(H_{h}))$ and satisfies \eqref{eq:corep-pi}. We show
   that it also satisfies \eqref{eq:corep}. Let $k,n\in I$, $a\in A$ and $\eta,\xi \in K$. Then
 \begin{align*}
\tilde V_{12}\tilde V_{13}(a\zeta^{k}_{n} \otimes \eta\otimes \xi) &=
\tilde V_{12} \Delta(a)_{13}\sum_{m}(\zeta^{k}_{m} \otimes \eta \otimes \UnitC{m}{n}\xi) \\
&= (\Delta \otimes \id)(\Delta(a)) \sum_{l,m} (\zeta^{k}_{l} \otimes \UnitC{l}{m}\eta \otimes
\UnitC{m}{n}\xi)  \\
&= (\id \otimes \Delta)(\Delta(a))  \sum_{l} (\zeta^{k}_{l} \otimes \lambda_{l}\eta \otimes
\rho_{n}\xi),
\end{align*}
and a straightforward calculation shows that this is equal to $(\id \otimes \Delta)(\tilde
V)(a\zeta^{k}_{n} \otimes \eta\otimes \xi)$.
 \end{proof}
\subsection{The dual corepresentation}

Let $X$ be an irreducible unitary corepresentation on a Hilbert space $H$, which is automatically
rcfd by Proposition \ref{prop:corep-rcfd}. For each $k,l \in I$, choose matrix units
$(e^{(kl)}_{i,j})_{i,j}$ of the finite-dimensional space $\mathcal{B}(\Grd{H}{k}{l})$ and write
  \begin{align*}
    \Gr{X}{k}{l}{m}{n} =  \sum_{i,j} x^{(klmn)}_{i,j} \otimes e^{(kl)}_{i,j}.
  \end{align*}
Then the sum
\begin{align*}
X^{\dag}:=  \sum_{k,l,m,n} \sum_{i,j} (x^{(klmn)}_{i,j})^{*} \otimes
  e^{(kl)}_{i,j} 
\end{align*}
converges strictly in $M(A \otimes \mathcal{K}(H))$ and depends on the
choice of the matrix units $(e^{(kl)}_{i,j})_{i,j}$ up to conjugation
by $1\otimes U$ for some $I^{2}$-graded unitary $U\in \mathcal{B}(H)$.
  A straightforward calculation shows that
  \begin{align*}
    (\Delta \otimes \id)(X^{\dag}) = X^{\dag}_{13}X^{\dag}_{23}.
  \end{align*}
We want to show that $X^{\dag}$ is, in a sense, equivalent to an irreducible unitary  corepresentation. The idea is to embed
$X^{\dag}$ into the regular corepresentation.
\begin{Lem} \label{lem:construct-intertwiner}
  Let $K$ be an $I^{2}$-graded Hilbert space, assume that $Y\in M(A
  \otimes \mathcal{K}(K))$ satisfies $(\Delta \otimes
  \id)(Y)=Y_{13}Y_{23}$, and let $\eta\in \Grd{K}{k}{l}$. Then the formula
  \begin{align*}
    T^{(\eta)} \xi := (\id \otimes \omega_{\eta,\xi})(Y) \zeta^{l}_{n}
    \quad \text{for } \xi \in \Grd{K}{m}{n}
  \end{align*}
  defines an $I^{2}$-graded  operator $T^{(\eta)} \in \mathcal{B}(K,H_{h})$ satisfying
  \begin{align*}
    V(1\otimes T^{(\eta)}) = (1 \otimes T^{(\eta)})Y.
  \end{align*}
\end{Lem}
\begin{proof}
The formula above defines an $I^{2}$-graded  operator $T^{(\eta)}$
because  for $\xi \in \Grd{K}{m}{n}$, 
  \begin{align*}
    (\id \otimes \omega_{\eta,\xi})(Y) &=  
    (\id \otimes \omega_{\eta,\xi})((\UnitC{k}{m}\otimes 1)Y(\UnitC{l}{n}\otimes
   1) \in \Gr{A}{k}{l}{m}{n}
  \end{align*}
 by Lemma \ref{lem:corep-intertwine}
  and
  \begin{align*}
   \|    (\id \otimes \omega_{\eta,\xi})(Y)\| \leq
  \|Y\|\|\eta\|\|\xi\|  \| \zeta^{l}_{n}\| =     \|Y\|\|\eta\|\|\xi\|.
  \end{align*}
  To check that $V(1\otimes T)=(1\otimes T)Y$, consider $a \in \Gr{A}{p}{q}{r}{s}$ and
  $\xi \in \Grd{K}{m}{n}$. Then
  \begin{align*}
    V(1\otimes T)(a\otimes \xi)   &= V(a \otimes (\id \otimes
    \omega_{\eta,\xi})(Y)\zeta^{l}_{n}) \\ &=
    (\Delta^{\op} \otimes
    \omega_{\eta,\xi})(Y)(a\otimes \zeta^{l}_{p}) \\
    &=  (\id \otimes \id \otimes
    \omega_{\eta,\xi})(Y_{23}(1 \otimes \rho^{K}_{p})Y_{13})(a\otimes
    \zeta^{l}_{p}) \\
    &= (\id \otimes T)Y(a\otimes \xi)
  \end{align*}
and the claim follows.
\end{proof}


\begin{Prop}
There exists an irreducible unitary corepresentation $\overline{X}$ on a Hilbert
space $K$ and an injective operator $T\in \mathcal{B}(H,K)$  with
dense image such that $(1\otimes
T)X^{\dag} =\overline{X}(1\otimes T)$, and  this unitary corepresentation $\overline{X}$ is unique up to equivalence.
\end{Prop}
\begin{proof}
  Choose $k,l$ and $\eta\in \Grd{H}{k}{l}$ such that the operator $T^{(\eta)} \in
  \mathcal{B}(H,H_{h})$ constructed in Lemma \ref{lem:construct-intertwiner} is non-zero.  We claim
  this $T^{(\eta)}$ is injective.  Indeed, denote by $P$ the projection onto the kernel of
  $T^{(\eta)}$. Then the relation
\begin{align*}
  V(1\otimes T^{(\eta)}) = (1\otimes T^{(\eta)})X^\dag
\end{align*}
implies
$(1 \otimes T^{(\eta)})  X^\dag(1 \otimes P) = 0$
and hence
$X^\dag(1 \otimes P)  = (1 \otimes P)  X^\dag(1 \otimes
P)$, and therefore,
\begin{align*}
  (1 \otimes P)X(1\otimes P) = X(1\otimes P).
\end{align*}
By irreducibility of $X$, we can conclude that $P=0$ or
$P=\id_{H}$. Since $T^{(\eta)}\neq 0$, we must have $P=0$ and hence
$T^{(\eta)}$ is injective. 

We can now take for $K$ the closure of the image of $T^{(\eta)}$, for
$Y$ the restriction of $V$ to $K$,  and
for $T$ the corestriction of $T^{(\eta)}$ to $K$.
\end{proof}

\subsection{Matrix coefficients, Schur orthogonality relations and Peter-Weyl}


To every unitary corepresentation $X$ of $(A,\Delta)$ on a Hilbert space $H$, we associate a space
of \emph{matrix coefficients}
\begin{align*}
\mathcal{C}(X) = \sum_{k,l,m,n}  \Gr{\mathcal{C}(X)}{k}{l}{m}{n},
\text{ where } \Gr{\mathcal{C}(X)}{k}{l}{m}{n} = \span \{ (\id \otimes
  \omega_{\eta,\xi})(\Gr{X}{k}{l}{m}{n}) : \eta,\xi \in H\} \subseteq
  \Gr{A}{k}{l}{m}{n}.
\end{align*}
We call the matrix coefficients contained in the spaces $\Gr{\mathcal{C}(X)}{k}{l}{m}{n}$
\emph{homogeneous}.

These matrix coefficients satisfy the following Schur orthogonality relations.
\begin{Lem}
  Let $X$ and $Y$ be inequivalent unitary irreducible corepresentations of $(A,\Delta)$. Then
  \begin{align*}
    h_{km}(a^{*}b) = 0 \quad \text{for all } a\in \mathcal{C}(X), \ b\in \mathcal{C}(Y).
  \end{align*}
\end{Lem}
\begin{proof}
It suffices to  consider elements of the form
  \begin{align*}
    a = (\id\otimes \omega_{\xi,\eta})(X) \quad\text{and}  \quad b=(\id \otimes \omega_{\zeta,\theta})(Y),
  \end{align*}
  where $\xi \in \Grd{H}{k}{l}$, $\eta\in \Grd{H}{m}{n}$, $\zeta \in \Grd{K}{k}{l}$, $\theta \in
  \Grd{K}{m}{n}$. Define $T\in \mathcal{B}(K,H)$ by $T\omega = \langle \zeta|\omega\rangle \xi$. 
  Then $a^{*}=(\id \otimes \omega_{\eta,\xi})(X^{*})$ and, in the notation of Lemma \ref{lem:intertwiner-averaged},
  \begin{align*}
    h_{km}(a^{*}b) &= (h_{km} \otimes \omega_{\eta,\theta})(X^{*}(1\otimes T)Y) 
 = \langle \eta|
     \check{T}^{(k)} \theta\rangle = 0,
  \end{align*}
  because $\check{T}^{(k)}$ intertwines $X$  and $Y$  and hence is $0$.
\end{proof}

\begin{Prop}
  Let $X$ be a unitary irreducible corepresentation of $(A,\Delta)$. Then
\end{Prop}
 \subsection{The  partial compact quantum group of  matrix coefficients}
\begin{Prop}
  The matrix coefficients of irreducible unitary rcfd corepresentations
  are linearly dense in $A$.
\end{Prop}
\begin{proof}
  Consider the regular corepresentation $V$. Let $a\in \Gr{A}{k}{l}{m}{n}$, $b\in
  \Gr{A}{p}{q}{r}{s}$ and $\eta=a\zeta^{k}_{m}$, $\eta=b\zeta^{q}_{s}$. Then by definition,
  \begin{align*}
    (\id \otimes \omega_{\eta,\xi})(V) = (h_{qs}\otimes \id)((b^{*} \otimes
    1)\Delta(a))
  \end{align*}
  and since $[(A\otimes 1)\Delta(A)]=[(A\otimes A)\Delta(1)]$, we can conclude that $\mathcal{C}(V)$
  is dense in $A$. But by Proposition \ref{prop:corep-decompose} and Proposition \ref{prop:corep-rcfd}, $V$ decomposes as a direct sum of
  irreducible unitary rcfd corepresentations, whence the assertion follows.
\end{proof}
\begin{Prop}
  Let $(A,\Delta)$ be a $C^{*}$-partial quantum group. Then the homogeneous matrix coefficients of
  irreducible unitary rcfd corepresentations form a partial $*$-algebra with respect to the
  multiplication and involution inherited from $A$.
\end{Prop}
\begin{proof}
  Let $X$ and $Y$ be unitary rcfd corepresentations. Then clearly
  \begin{align*}
    \Gr{\mathcal{C}(X)}{k}{l}{m}{n} \cdot \Gr{\mathcal{C}(Y)}{p}{q}{r}{s} &\subseteq
    \delta_{m,p}\delta_{n,r} \Gr{\mathcal{C}(X\Circt Y)}{k}{q}{m}{s}, &
    \left(\Gr{\mathcal{C}(X)}{k}{l}{m}{n}\right)^{*} = \Gr{\mathcal{C}(\overline{X})}{l}{k}{n}{m},
  \end{align*}
and the assertion follows.
\end{proof}


\begin{Theorem} Let $(A,\Delta)$ be a $C^{*}$-partial quantum group.
  Then the partial $*$-algebra of homogeneous matrix coefficients of irreducible unitary rcfd
  corepresentations is a partial compact quantum group with respect to the comultiplication given by
  \begin{align} \label{eq:pcqg-comultiplication}
    \Delta_{pq}(a) &= (\rho_{p} \otimes \lambda_{p})\Delta(a)(\rho_{q}\otimes \lambda_{q}).
  \end{align}
  The invariant integral is the restriction of $h$, and the counit and antipode are given by
  \begin{align} \label{eq:pcqg-counit-antipode}
    \varepsilon((\id \otimes \omega_{\xi,\eta})(X)) &= \langle \xi|\eta\rangle, &
    S((\id \otimes \omega_{\xi,\eta})(X)) &= (\id \otimes \omega_{\xi,\eta})(X^{*})
  \end{align}
  for every unitary rcfd corepresentation $X$ on a Hilbert space $H$ and vectors $\xi \in
  \Grd{H}{k}{l}$, $\eta \in \Grd{H}{m}{n}$. 
\end{Theorem}
\begin{proof}
  Let
  \begin{align*}
    B = \sum_{X} \mathcal{C}(X) \subseteq A \quad \text{and} \quad \Gr{B}{k}{l}{m}{n} = \sum_{X}
    \Gr{\mathcal{C}(X)}{k}{l}{m}{n} \subseteq \Gr{A}{k}{l}{m}{n},
  \end{align*}
  where the sums are taken over all unitary rcfd corepresentations of $(A,\Delta)$.

  We first claim that $\Delta_{pq}$ maps $\Gr{B}{k}{l}{m}{n}$ into the algebraic tensor product
  $\Gr{B}{k}{l}{p}{q} \otimes \Gr{B}{p}{q}{m}{n}$.  Let $X$ be a unitary rcfd corepresentation on a
  Hilbert space $H$ and
  \begin{align*}
    a = (\id\otimes
  \omega_{\xi,\eta})(X), \quad \text{where } \xi \in \Grd{H}{k}{l}, \ \eta\in \Grd{H}{m}{n}.
  \end{align*}
 Then Lemma \ref{lem:corep-intertwine} implies
  \begin{align*}
    \Delta_{pq}(a) = (\id \otimes \id \otimes \omega_{\xi,\eta})(X_{13}(1\otimes 1 \otimes p^{H}_{pq})X_{23}).
  \end{align*}
  Since $X$ is rcfd,  the projection $p^{H}_{pq}$ is compact. Choose an orthonormal
  basis $(\Gr{\zeta}{}{(i)}{p}{q})_{i}$ of its image, we obtain
  \begin{align*}
    \Delta_{pq}(a) = \sum_{i} (\id \otimes \omega_{\xi,\Gr{\zeta}{}{(i)}{p}{q}})(X) \otimes (\id \otimes
    \omega_{\Gr{\zeta}{}{(i)}{p}{q},\eta})(X) \in \Gr{\mathcal{C}(X)}{k}{l}{p}{q} \otimes \Gr{\mathcal{C}(X)}{p}{q}{m}{n}.
  \end{align*}
  The claim follows.
  
  Thanks to the Peter-Weyl result, there exist a well-defined functional $\varepsilon \colon B\to
  \C$ and a well-defined linear map $S\colon B\to B$ satisfying \eqref{eq:pcqg-counit-antipode}, and
  we only need to verify the counit and antipode relations.  

  We start with the counit.  By definition, $\varepsilon(\Gr{B}{k}{l}{m}{n})=0$ if $(k,l)\neq
  (m,n)$.  Denote by $(\delta_{k})_{k}$ the canonical basis of $l^{2}(I)$. For the trivial
  corepresentation $E$, \eqref{eq:pcqg-counit-antipode} implies
  \begin{align*}
    \varepsilon(\UnitC{k}{k}) = \varepsilon((\id \otimes \omega_{\delta_{k},\delta_{k}})(E)) =
    \langle \delta_{k}|\delta_{k}\rangle = 1.
  \end{align*}
Next,   for arbitrary $X,\xi,\eta$ and $a$ as above,
\begin{align*}
    (\varepsilon \otimes\id)(\Delta_{kl}(a)) &= \sum_{i}\epsilon((\id \otimes
    \omega_{\xi,\Gr{\zeta}{}{(i)}{p}{q}})(X)) (\id \otimes \omega_{\Gr{\zeta}{}{(i)}{p}{q},\eta})(X)
    \\
    &= \sum_{i}\langle \xi|\Gr{\zeta}{}{(i)}{p}{q}\rangle(\id \otimes
    \omega_{\Gr{\zeta}{}{(i)}{p}{q},\eta})(X) \\ &= (\id \otimes \omega_{\xi,\eta})(X) \\ &= a,
  \end{align*}
  and a similar calculation shows that  $    (\id \otimes \varepsilon\id)(\Delta_{mn}(a))=a$.
Finally, if also $Y$ is a unitary rcfd corepresentation on a Hilbert space $K$ and
\begin{align*}
  b=(\id \otimes \omega_{\zeta,\theta})(Y), \quad \text{where } \zeta \in \Grd{K}{l}{p}, \theta \in \Grd{K}{n}{q},
\end{align*}
then $ab = (\id \otimes \omega_{\xi\otimes \zeta,\eta\otimes \theta})(X\Circt Y)$, where $\xi\otimes \zeta \in \Grd{(H\itimes K)}{k}{p}$, $\eta\otimes \theta \in \Grd{(H\itimes
  K)}{m}{q}$, whence
\begin{align*}
  \varepsilon(ab) = \langle \xi\otimes \zeta|\eta\otimes \theta\rangle = \langle
  \xi|\zeta\rangle\langle \zeta|\theta\rangle = \varepsilon(a)\varepsilon(b).
\end{align*}

Next, we verify the antipode relations. 
Lemma \ref{lem:corep-intertwine} implies $S(\Gr{B}{k}{l}{m}) \subseteq \Gr{B}{n}{m}{l}{k}$.
Denote by $m \colon B\otimes B\to B$ the multiplication
map and consider $a$ as above. Then by Lemma \ref{lem:corep-intertwine},
\begin{align*}
  \sum_{q} m(\id \otimes S)(\Delta_{pq}(a)) &= 
  \sum_{q,i} (\id \otimes \omega_{\xi,\Gr{\zeta}{}{(i)}{p}{q}})(X)  (\id \otimes
  \omega_{\Gr{\zeta}{}{(i)}{p}{q},\eta})(X^{*}) \\ &=
 (\id \otimes \omega_{\xi,\eta})(X(1\otimes \lambda^{H}_{p})X^{*})  \\
  &= \rho_{p}  (\id \otimes \omega_{\xi,\eta})(XX^{*})  = \UnitC{k}{p}\langle \xi|\eta\rangle =\UnitC{k}{p} \varepsilon(a),
\end{align*}
and a similar calculation shows that $\sum_{p} m(S \otimes \id)(\Delta_{pq}(a)) = \UnitC{q}{n}\varepsilon(a)$.

Finally, it is clear that the invariant integral $h$ on $(A,\Delta)$ restricts to an invariant
integral on $B$.
\end{proof}




\begin{thebibliography}{00}

\bibitem{MVD1} A. Maes, A. Van Daele, Notes on compact quantum groups, \emph{Nieuw Arch. Wisk.} \textbf{16} (4) (1998), no. 1--2, 73--112. 

\end{thebibliography}


\end{document}
