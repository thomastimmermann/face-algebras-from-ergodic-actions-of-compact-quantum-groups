\documentclass[11pt]{article}

\usepackage{hyperref}
\usepackage{fixme}
\usepackage{mathrsfs}
\usepackage[a4paper]{geometry}
\usepackage{amssymb, amsthm, amsfonts, amsxtra, amsmath}
\usepackage{latexsym}
\usepackage{mathabx}
%\usepackage{enumitem}
%\usepackage[all]{xy}
%\usepackage{graphics}
\usepackage{pdfpages}
\usepackage{epic}
\usepackage{fouridx}
\usepackage{parskip} % paragraphs have no indents and vertical spacings inbetween
\makeatletter % need this to avoid the conflict between amsthm and parskip
\def\thm@space@setup{%
  \thm@preskip=\parskip \thm@postskip=0pt
}
\makeatother
\usepackage{enumerate}

%\theoremstyle{change}

\newcommand{\dual}[1]{#1^{\vee}}
\newcommand{\predual}[1]{{^{\vee}\!#1}}
\newcommand{\co}{\mathrm{co}}
\newcommand{\Corep}{\mathrm{Corep}}
\newcommand{\Corepf}{\mathrm{Corep}^{f}}
\newcommand{\sff}{\textrm{s.f.~}}
\newcommand{\sfs}{\mathrm{sfs}}
\newcommand{\sfd}{\mathrm{sfd}}
\DeclareMathOperator{\Hom}{Hom}
\DeclareMathOperator{\img}{img}

\DeclareMathOperator{\id}{id}
\DeclareMathOperator{\ext}{\mathrm{e}}
\DeclareMathOperator{\can}{\mathrm{can}}
\DeclareMathOperator{\ctau}{\tau}
\DeclareMathOperator{\iboxtimes}{\underset{I}{\boxtimes}}
\DeclareMathOperator{\op}{\mathrm{op}}
\DeclareMathOperator{\fin}{\mathrm{f}}
\DeclareMathOperator{\Pol}{\mathrm{P}}
\DeclareMathOperator{\End}{\mathrm{End}}
\DeclareMathOperator{\Par}{\mathrm{Par}}
\DeclareMathOperator{\reg}{\mathrm{reg}}
\DeclareMathOperator{\sgn}{\mathrm{sgn}}
\DeclareMathOperator{\Zz}{\mathrm{Z}}
\DeclareMathOperator{\Ran}{\mathrm{Ran}}
\DeclareMathOperator{\hol}{\mathrm{hol}}
\DeclareMathOperator{\Ind}{\mathrm{Ind}}
\DeclareMathOperator{\Ker}{\mathrm{Ker}}
\DeclareMathOperator{\Char}{\mathrm{Char}}
\DeclareMathOperator{\dyn}{\mathrm{dyn}}
\DeclareMathOperator{\Spec}{\mathrm{Spec}}
\DeclareMathOperator{\adj}{\mathrm{adj}}
\DeclareMathOperator{\rcfd}{\mathrm{rcfd}}
\DeclareMathOperator{\rcf}{\mathrm{rcfd}}
\DeclareMathOperator{\stau}{\tau_{\mathrm{s}}}
\DeclareMathOperator{\tA}{\tilde{A}}
\DeclareMathOperator{\weps}{\tilde{\epsilon}}

\newcommand{\Circt}{{\mathop{\ooalign{$\ovoid$\cr\hidewidth\raise-.05ex\hbox{$\scriptstyle\mathsf T\mkern3.5mu$}\cr}}}} % Woronowicz style tensor product, USUAL SIZE
\newcommand{\Circtv}[1]{\underset{#1}{\mathop{\ooalign{$\ovoid$\cr\hidewidth\raise-.05ex\hbox{$\scriptstyle\mathsf T\mkern3.5mu$}\cr}}}} % Woronowicz style tensor product, USUAL SIZE
\newcommand{\smCirct}{\mathop{\ooalign{$\scriptstyle\ovoid$\cr\hidewidth\raise-.05ex\hbox{$\scriptscriptstyle\mathsf T\mkern2.8mu$}\cr}}}  % Woronowicz style tensor product, SCRIPT SIZE

\newcommand{\nc}{\R}
\newcommand{\g}{\mathfrak{g}}
\newcommand{\h}{\mathfrak{h}}

\newcommand{\kk}{\mathfrak{k}}
\newcommand{\ttt}{\mathfrak{t}}
\newcommand{\p}{\mathfrak{p}}
\newcommand{\n}{\mathfrak{n}}
\newcommand{\llll}{\mathfrak{l}}
\newcommand{\uu}{\mathfrak{u}}
\newcommand{\bb}{\mathfrak{b}}
\newcommand{\q}{\mathfrak{q}}
\newcommand{\su}{\mathfrak{su}}
\newcommand{\ssl}{\mathfrak{sl}}
\newcommand{\SSL}{\mathrm{SL}}
\newcommand{\so}{\mathfrak{so}}
\newcommand{\spp}{\mathfrak{sp}}
\newcommand{\G}{\mathbb{G}}
\newcommand{\e}{\mathfrak{e}}
\newcommand{\s}{\mathfrak{s}}
\newcommand{\C}{\mathbb{C}}
\newcommand{\R}{\mathbb{R}}
\newcommand{\Z}{\mathbb{Z}}
\newcommand{\N}{\mathbb{N}}
\newcommand{\X}{\mathbb{X}}
\newcommand{\Y}{\mathbb{Y}}
\newcommand{\Ss}{\mathbb{S}}
\newcommand{\ZZ}{\mathscr{Z}}
\newcommand{\ad}{\mathrm{ad}}
\newcommand{\Hsp}{\mathcal{H}}
\newcommand{\qn}[2]{\lbrack #1 \rbrack_{#2}}
\newcommand{\fqn}[2]{\lbrack #1 \rbrack_{#2}!}
\newcommand{\bqn}[3]{\left\lbrack \begin{array}{c} \!#1\! \\ \!#2\! \end{array}\right\rbrack_{#3}}
\newcommand{\Tr}{\mathrm{Tr}}
\newcommand{\RR}{\mathcal{R}}
\newcommand{\rd}{\mathrm{d}}
\newcommand{\res}{\mathrm{res}}
\newcommand{\cop}{\mathrm{cop}}
\newcommand{\opp}{\mathrm{op}}
\newcommand{\coop}{\mathrm{coop}}
\newcommand{\Rm}{\mathcal{R}}
\newcommand{\wt}{\mathrm{wt}}
\newcommand{\Ad}{\mathrm{Ad}}
\newcommand{\CatC}{\mathcal{C}}
\newcommand{\CatD}{\mathcal{D}}
\newcommand{\CatCC}{\mathscr{C}}
\newcommand{\CatDD}{\mathscr{D}}
\newcommand{\Corr}{\mathrm{Corr}}

\newcommand{\Vectf}{\mathrm{Vect}^{f}}
\newcommand{\Vecti}{\mathrm{Vect}_{I^{2}}}
\newcommand{\Vectif}{\mathrm{Vect}^{f}_{I^{2}}}
\newcommand{\Hilb}{\mathrm{Hilb}}
\newcommand{\Hilbf}{\mathrm{Hilb}^{\mathrm{f}}}
\newcommand{\Hilbi}{\mathrm{Hilb}_{I^{2}}}
\newcommand{\Hilbif}{\mathrm{Hilb}_{I^{2}}^{\mathrm{f}}}

\newcommand{\Star}[2]{{}_{#1}\!*_{#2}}
\newcommand{\vot}{\bar{\otimes}}
\newcommand{\A}{\mathcal{B}}
\newcommand{\Aa}{\mathscr{B}}
\newcommand{\Mor}{\mathrm{Mor}}
\newcommand{\alg}{\mathrm{alg}}
\newcommand{\Gg}{\mathscr{G}}
\newcommand{\ev}{\mathrm{ev}}
\newcommand{\Rtimes}{\underset{\R}{\times}}
\newcommand{\Rb}{\R^{\bullet}}
\newcommand{\vtimes}{\bar{\otimes}}
\newcommand{\Rr}{\mathscr{R}}
\newcommand{\Tt}{\mathscr{T}}
\newcommand{\Fun}{\mathrm{Fun}}
\newcommand{\Ff}{\Fun_{\fin}}
%\newcommand{\fin}{\mathrm{fin}}
%\newcommand{\iitimes}{\underset{I}{\otimes}}
\newcommand{\itimes}{\underset{I}{\otimes}}
\newcommand{\osum}[1]{\underset{#1}{\sum}^{\oplus}}
\newcommand{\osumc}[1]{\underset{#1}{\sum}^{\bar{\oplus}}}
\newcommand{\oplusc}{\bar{\oplus}}
\newcommand{\wDelta}{\widetilde{\Delta}}
\newcommand{\f}{\mathrm{fin}}
%\newcommand{\Hilb}{\mathrm{Hilb}}
\newcommand{\Rho}{\mathrm{P}}
\newcommand{\Rep}{\mathrm{Rep}}
\newcommand{\DA}{\mathcal{A}}
%\newcommand{\Circt}{\mathop{\ooalign{$\ovoid$\cr\hidewidth\raise-.05ex\hbox{$\scriptstyle\mathsf T\mkern3.5mu$}\cr}}} % Woronowicz style tensor product, USUAL SIZE
\newcommand{\even}{\mathrm{even}}
\newcommand{\odd}{\mathrm{odd}}
\newcommand{\fd}{\mathrm{fd}}
\newcommand{\Forget}{F}

\newcommand{\GrHA}[3]{#1{\begin{pmatrix} #2,  #3\end{pmatrix}}}% Horizontal grading ordinary style, with argument
\newcommand{\Grs}[3]{#1{\begin{pmatrix} #2,  #3\end{pmatrix}}}

\newcommand{\GrDA}[3]{{}_{#2}#1_{#3}} % Horizontal grading bottom style, with argument
%\newcommand{\Grd}[3]{\;{}_{\;#2}#1_{#3}}

\newcommand{\GrVA}[3]{#1{\tiny {\begin{pmatrix} #2\\#3\end{pmatrix}}}} % Vertical grading ordinary style, with argument
\newcommand{\Grt}[3]{#1{\tiny {\begin{pmatrix} #2\\#3\end{pmatrix}}}} 

\newcommand{\GrRA}[3]{#1^{#2}_{#3}} % Vertical grading right style, with argument

\newcommand{\GrLA}[3]{{}^{#2}_{#3}#1} % Vertical grading left style, with argument

\newcommand{\Unit}{\mathbf{1}}
\newcommand{\UnitC}[2]{\Grt{\mathbf{1}}{#1}{#2}} 
\newcommand{\Grru}[2]{{\tiny \begin{pmatrix} #1 \\ #2\end{pmatrix}}}

\newcommand{\eGr}[5]{#1{{\tiny \begin{pmatrix} #2 \quad #3 \\ #4 \quad #5\end{pmatrix}}}}

\newcommand{\pma}[4]{\begin{pmatrix} #1 \quad #2 \\ #3 \quad #4\end{pmatrix}}
\newcommand{\pmat}[4]{{\tiny \begin{pmatrix} #1 \quad #2 \\ #3 \quad #4\end{pmatrix}}}

\newcommand{\UT}[2]{#1{\tiny #2 }}
\newcommand{\Gr}[5]{\fourIdx{#2}{#4}{#3}{#5}{#1}}%TODO: better typesetting
\newcommand{\Grl}[3]{\Gr{#1}{#2}{}{#3}{}}%TODO: better typesetting
\newcommand{\Gru}[3]{\Gr{#1}{}{}{#2}{#3}}
\newcommand{\Grd}[3]{\Gr{#1}{}{}{#2}{#3}}
% \newcommand{\Gr}[5]{\;{}^{\;#2}_{#4}#1_{#5}^{#3}}%TODO: better typesetting
% %\newcommand{\Gr}[5]{\UT{#1}{\begin{pmatrix} #2\quad #3 \\ #4 \quad #5\end{pmatrix}}}
% %\newcommand{\Gr}[5]{\UT{#1}{\begin{pmatrix} \, #2\;\\ #3 \qquad #4 \\ \,#5\;\end{pmatrix}}}
% \newcommand{\Grl}[3]{\;{}^{\;#2}_{#3}#1}%TODO: better typesetting
% \newcommand{\Gru}[3]{{}^{\;#2}#1^{#3}}
% \newcommand{\Grd}[3]{{}_{\;#2}#1_{#3}}
\newcommand{\gr}[5]{\;{}^{\;#2}_{#4}#1_{#5}^{#3}}%TODO: better typesetting
\newcommand{\eGrr}[3]{#1_{{\tiny \left(#2, #3\right)}}}
\newcommand{\eGrt}[4]{#1{{\tiny \begin{pmatrix} #2 \\ #3 \\ #4 \end{pmatrix}}}}
\newcommand{\Grr}[4]{\begin{pmatrix}#1 \quad #2\\#3&#4\end{pmatrix}}

\newcommand{\Grss}[3]{\UT{#1}{\begin{pmatrix} #2 \; #3\end{pmatrix}}}
\newcommand{\Grb}[7]{\UT{#1}{\begin{pmatrix} #2\quad #3 \\ #4 \quad #5\\ #6 \quad #7\end{pmatrix}}}
\newcommand{\un}[2]{e{{\tiny \begin{pmatrix}#1\\ #2\end{pmatrix}}}}
\newcommand{\unn}[3]{e{{\tiny \begin{pmatrix}#1\\ #2\\#3\end{pmatrix}}}}

\newcommand{\wmult}{\cdot}
\newcommand{\bmult}{*}
\newcommand{\wmate}{\rightarrow}% Change this to source/target notation l(eft) r(ight)
\newcommand{\bmate}{\downarrow}% Change this to source/target notation u(p) d(own)

\newcommand{\aste}[1]{\underset{#1}{\ast}}

\newcommand{\Vv}{\mathcal{V}}

\newcommand{\dT}{\dot T}

\newtheorem{Theorem}{Theorem}[section]
\newtheorem{Lem}[Theorem]{Lemma}
\newtheorem{Prop}[Theorem]{Proposition}
\newtheorem{Cor}[Theorem]{Corollary}

\theoremstyle{definition}
\newtheorem{Def}[Theorem]{Definition}
\newtheorem{Rem}[Theorem]{Remark}
\newtheorem{Exa}[Theorem]{Example}
\newtheorem{Not}[Theorem]{Notation}
\newtheorem{Que}[Theorem]{Question}
\newtheorem{Con}[Theorem]{Conjecture}

%%%%%%%%%%%%%%%%%%%
% Further notation for Section 1
\newcommand{\phic}[2]{\Grt{\phi}{#1}{#2}}

%%%%%%%%%%%%%%%%%%%
% Notation for Section 4
\newcommand{\LGtwo}{L^{2}(\mathscr{G})}
\newcommand{\LGinf}{L^{\infty}(\mathscr{G})}
\newcommand{\CrG}{C^{r}_{0}(\mathscr{G})}
\newcommand{\CuG}{C^{u}_{0}(\mathscr{G})}
\newcommand{\vnDelta}{\overline{\Delta}}
\newcommand{\vnE}{\overline{E}}
\newcommand{\astrl}{\underset{l^{\infty}(I)}{_{\rho}\ast_{\lambda}}}
\newcommand{\otimesrl}{\underset{\nu{}}{_{\rho}\otimes_{\lambda}}}
\newcommand{\vnphi}{\overline{\phi}}
\newcommand{\vnphic}[2]{\Grt{\vnphi}{#1}{#2}}
\newcommand{\vnR}{\overline{R}}
\newcommand{\vntau}{\overline{\tau}}

% q-special functions macros

\newcommand{\qbin}[2]{\left[ \begin{array}{c} #1 \\ #2 \end{array}\right]_{q^2}}
\newcommand{\qortc}[4]{\,\;_1\varphi_1\left(\begin{array}{c} #1  \\#2 \end{array}\mid #3,#4\right)}
\newcommand{\qortPsi}[4]{\Psi\left(\begin{array}{c} #1  \\#2 \end{array}\mid #3,#4\right)}
\newcommand{\qorta}[5]{\,\;_2\varphi_1\left(\begin{array}{cc} #1 & #2 \\ & \!\!\!\!\!\!\!\!\!\!\!#3 \end{array}\mid #4,#5\right)}
\newcommand{\qortd}[6]{\,\;_2\varphi_2\left(\begin{array}{cc} #1 & #2 \\ #3 & #4 \end{array}\mid #5,#6\right)}
\newcommand{\qortb}[7]{\,\;_3\varphi_2\left(\begin{array}{ccc} #1 & #2 & #3 \\ & \!\!\!\!\!\!\!\!#4 & \!\!\!\!\!\!\!\!#5\end{array}\mid #6,#7\right)}

\date{}


\numberwithin{equation}{section}

\begin{document}

\section{C$^*$-pcqg}
For $A$ a C$^*$-algebra, we denote by $M(A)$ the multiplier C$^*$-algebra of $A$. All tensor products of C$^*$-algebras in this paper will be minimal. We denote by $[\,\cdot\,]$ the closed linear span of a subset.

\begin{Def}\label{DefCpcqg} Let $I$ be a set. We call \emph{C$^*$-algebraic $I$-partial compact quantum group}, or \emph{C$^*$-pcqg} (over $I$) for short, a triple consisting of 
\begin{itemize}
\item a (not necessarily unital) C$^*$-algebra $A$,
\item a family of orthogonal self-adjoint projections $\UnitC{k}{l}\in A$ for $k,l\in I$, some of which are possibly zero, and
\item  a (not necessarily unital) $^*$-homomorphism \[\Delta: A\rightarrow M(A\otimes A),\] 
\end{itemize}
satisfying the following conditions:
\begin{enumerate}[(a)]
\item[Ui)] $\UnitC{k}{k}\neq 0$ for all $k\in I$. 
\item[Uii)] $\sum_{k,l} \UnitC{k}{l}$ strictly converges to the unit in $M(A)$.
\item[Uiii)] $\Delta(\UnitC{k}{l}) = \sum_{m}\UnitC{k}{m}\otimes \UnitC{m}{l}$ strictly for all $k,l$. 
\item[Di)] With $\Delta(1) = \sum_{k,l,m} \UnitC{k}{m}\otimes \UnitC{m}{l}$, we have \begin{equation}\label{CondDi}(A\otimes A)\Delta(1) = [(A\otimes 1)\Delta(A)] = [(1\otimes A)\Delta(A)].\end{equation} 
\item[Dii)] With $P=\sum_{k} \UnitC{k}{k}$, and $A_P = PAP$, we have \[[(\omega\otimes \id)\Delta(A_P)\mid \omega \in A^*] = [(\id\otimes \omega)\Delta(A_P)\mid \omega \in A^*] = A.\]
\item[C)] $\Delta$ is coassociative: for all $a,b,c\in A$, we have \[(a\otimes 1\otimes 1)(\Delta\otimes \id)(\Delta(b)(1\otimes c)) = (\id\otimes \Delta)((a\otimes 1)\Delta(b))(1\otimes 1\otimes c).\] 
\end{enumerate}
\end{Def} 

Note that $\Delta(1)$ is a well-defined projection in $M(A\otimes A)$. By condition \eqref{CondDi}, $\Delta$ extends uniquely to a $^*$-homomorphism \[\Delta: M(A)\rightarrow M(A\otimes A)\] with value in the unit precisely $\Delta(1)$. In the same way, $(\id\otimes \Delta)$ and $(\Delta\otimes \id)$ extend to $M(A\otimes A)$, and we can write the coassociativity condition in the usual form \[(\Delta\otimes \id)\Delta = (\id\otimes \Delta)\Delta,\] valid also on $M(A)$.

In the following, we write \[\Gr{A}{k}{l}{m}{n} = \UnitC{k}{m}A\UnitC{m}{n},\] each of which is a Banach subspace of $A$. In particular, each corner $\Gr{A}{k}{k}{m}{m}$ is a unital C$^*$-algebra with unit $\UnitC{k}{m}$. We will write also \[\lambda_k = \sum_{m}\UnitC{k}{m},\qquad \rho_m = \sum_{k}\UnitC{k}{m},\] which give well-defined projections in $M(A)$. Finally, we will write, for $a\in A$, \[\Delta_{rs}(a) = (\rho_r\otimes 1)\Delta(a)(\rho_s\otimes 1) = (1\otimes \lambda_r)\Delta(a)(1\otimes \lambda_s),\] which are well-defined elements in $A\otimes A$ by Definition \ref{DefCpcqg}.Uii). We then have that  \[\Delta(a) = \sum_{r,s} \Delta_{rs}(a)\] in the strict topology. 

\begin{Def} Let $(A,\Delta)$ be a C$^*$-pcqg. An \emph{invariant integral} on $A$ consists of a weight $\phi: A^+ \rightarrow [0,+\infty]$ satisfying the following conditions:
\begin{enumerate}[i)]
\item For all $k,m$ with $\UnitC{k}{m}\neq 0$, \[\phi(\UnitC{k}{m}) = 1.\]
\item For all $a\in A^+$, \[\phi(a) = \sum_{k,m} \phi\left(\UnitC{k}{m}a\UnitC{k}{m}\right).\]
\item For all $a\in A^+$ and all states $\omega\in A^*$, \begin{equation}\label{EqInvL} \phi((\omega \otimes \id)\Delta(a)) = \sum_{k} \omega(\lambda_k)\phi(\lambda_ka\lambda_k),\end{equation} \begin{equation}\label{EqInvR} \phi((\id\otimes \omega)\Delta(a)) = \sum_{m}\omega(\rho_m)\phi(\rho_ma\rho_m).\end{equation}
\end{enumerate} 
\end{Def}


%family of states \[\phi_{km}: \Gr{A}{k}{k}{m}{m}\rightarrow \C\] for each $k,m$ with $\UnitC{k}{m}\neq 0$, so that for each $a\in \Gr{A}{k}{l}{m}{n}$ and each $r$, \[(\id\otimes \phi_{rm})\Delta_{rr}(a) = \phi_{km}(a)\UnitC{k}{r},\qquad (\phi_{kr}\otimes \id)\Delta_{rr}(a) = \phi_{kr}(a)\UnitC{r}{m}.\]
%\end{Def} 

%State zero functional if algebra zero

%Here we interpret $\phi_{rs}(a) =0$ for $a\notin \Gr{A}{r}{r}{s}{s}$. 

Clearly, the formula \[\phi_{km}(a) = \phi(\UnitC{k}{m}a\UnitC{k}{m})\] defines a bounded weight $\phi_{km}$ on $A$, which hence can be seen as a positive functional on $A$. If $\UnitC{k}{m}\neq 0$ it is a state, otherwise it is the zero functional. By abuse of language, we will in the following refer to the complete family of $\phi_{km}$ as `states', so the reader should bear in mind that some of them can be zero functionals.

It is also clear that $\phi$ is completely determined by the family $\{\phi_{km}\}$. To interpret the left and right invariance properties \eqref{EqInvL} and \eqref{EqInvR} in terms of the $\phi_{km}$, let us note first that $A^*$ can still be endowed with an associative convolution product by means of the formula \[(\chi*\omega)(a) = (\chi\otimes \omega)\Delta(a).\] Let us write \[\Gr{B}{k}{m}{l}{n} = \{\omega \in A^* \mid \forall a\in A, \omega(a) = \omega\left(\UnitC{k}{m}a\UnitC{l}{n}\right)\}.\] Then the convolution product restricts to products \[\Gr{B}{k}{m}{l}{n}\times \Gr{B}{m}{p}{n}{q}\rightarrow \Gr{B}{k}{p}{l}{q},\] all other products being zero. The left and right invariance properties \eqref{EqInvL} and \eqref{EqInvR} can then be written in terms of the $\phi_{km}$ as \begin{equation}\label{EqInvLp}\omega*\phi_{km} = \omega(\UnitC{p}{k})\phi_{pm},\qquad \forall \omega \in \Gr{B}{p}{k}{q}{k},\end{equation}
\begin{equation}\label{EqInvRp}\phi_{km}*\omega = \omega(\UnitC{m}{q})\phi_{kq},\qquad \forall \omega \in \Gr{B}{m}{p}{m}{q}.\end{equation}

We will refer to families of states satisfying \eqref{EqInvLp} as a \emph{left invariant integral}, and to those satisfying \eqref{EqInvRp} as \emph{right invariant integral}.

\begin{Theorem}\label{TheoInvInt} Each $C^*$-pcqg admits a unique invariant integral.
\end{Theorem} 

We will split the proof of the Theorem into several steps, setting the stage so that eventually the arguments of \cite{MVD1} can be applied almost verbatim. 

\begin{Lem} Let $\{\phi_{km}\}$ be a a left invariant integral, and $\{\psi_{km}\}$ a right invariant integral. Then $\phi_{km}= \psi_{km}$ for all $k,m$. 
\end{Lem} 
\begin{proof} By the invariance properties, and the fact that $\UnitC{k}{k}\neq 0$, we have \[\phi_{km}  = \psi_{kk}\left(\UnitC{k}{k}\right)\phi_{km} = \psi_{kk}*\phi_{km}= \phi_{km}\left(\UnitC{k}{m}\right)\psi_{km} = \psi_{km}.\]

\end{proof} 

By the previous Lemma, the unicity in Theorem \ref{TheoInvInt} already follows. It implies as well that it is sufficient to find an invariant left integral for $(A,\Delta)$.

The following lemma will be crucial.

\begin{Lem}\label{LemRefSep} Let $\omega \in \Gr{B}{k}{m}{l}{n}$, and assume $\chi*\omega =  0$ for all $\chi \in \Gr{B}{m}{k}{n}{l}$. Then $\omega =0$.
\end{Lem} 
\begin{proof} By assumption, we have for all $\chi\in A^*$ that \[(\chi\otimes \omega)((\UnitC{m}{k}\otimes \UnitC{k}{m})\Delta(A)(\UnitC{n}{l}\otimes \UnitC{l}{n}))=0.\] But since $\omega = \omega(\UnitC{k}{m}\,\cdot\,\UnitC{l}{n})$, this means, using the notation from Definition \ref{DefCpcqg}.Dii), \[\omega((\chi\otimes \id)(\Delta(A_P))) =(\chi\otimes \omega)((\sum_{k'}\UnitC{m}{k'}\otimes \UnitC{k'}{m})\Delta(A)(\sum_{l'}\UnitC{n}{l'}\otimes \UnitC{l'}{n})) =0.\] By Definition \ref{DefCpcqg}.Dii), we conclude $\omega=0$.
\end{proof} 

\begin{Cor} If $\UnitC{k}{m}=0$, then $\UnitC{m}{k}=0$. 
\end{Cor}
\begin{proof} If $\UnitC{k}{l}=0$, this implies $\Gr{B}{k}{l}{k}{l}=0$. By the previous lemma, this forces also $\Gr{B}{l}{k}{l}{k}=0$, and so $\UnitC{l}{k}=0$. 
\end{proof} 

\begin{Lem} Assume that there exists a family of states $\{\phi_{kk}\}$ in $\Gr{B}{k}{k}{k}{k}$ such that, for any $\omega \in \Gr{B}{k}{k}{k}{k}$, one has \[\omega*\phi_{kk} = \omega\left(\UnitC{k}{k}\right)\phi_{kk}.\] Then $(A,\Delta)$ admits a left invariant state.
\end{Lem}
\begin{proof} Let $\theta_{rm}$ be an arbitary collection of states in $\Gr{B}{r}{m}{r}{m}$ (whenever this algebra is not zero), and write \[\phi_{rm} = \theta_{rm}*\phi_{mm}.\] By assumption, this notation is consistent in the case $r=m$. 

Assume now that $\omega \in \Gr{B}{k}{r}{l}{r}$ and $\chi \in \Gr{B}{m}{k}{m}{l}$. Assume first that $\UnitC{r}{m}\neq 0$ and $\UnitC{k}{m}\neq 0$. Then \begin{eqnarray*} \chi*(\omega*\phi_{rm}) &=& (\chi*\omega*\theta_{rm})*\phi_{mm} \\ &=&  (\chi*\omega*\theta_{rm})(\UnitC{m}{m}) \phi_{mm} \\ &=&  \chi(\UnitC{m}{k})\omega(\UnitC{k}{r})\phi_{mm} \\ &=&  \omega(\UnitC{k}{r}) \; (\chi*\theta_{km})\left(\UnitC{m}{m}\right)\phi_{mm}
\\ &=& \omega(\UnitC{k}{r}) \; (\chi*\theta_{km})*\phi_{mm} \\ &=&  \omega(\UnitC{k}{r}) \; \chi*\phi_{km} .\end{eqnarray*} As $\chi$ was arbitrary, we find by Lemma \ref{LemRefSep} that \begin{equation}\label{EqInvL2} \omega*\phi_{rm} =  \omega(\UnitC{k}{r}) \phi_{km}.\end{equation}

Assume now that $\UnitC{r}{m}=0$. Then also $\Delta_{kk}(\UnitC{r}{m}) = \UnitC{r}{k}\otimes \UnitC{k}{m}=0$, and hence $\UnitC{r}{k}=0$ or $\UnitC{k}{m}=0$. By the previous lemma, we obtain either $\UnitC{k}{r}=0$ or $\UnitC{k}{m}=0$. In either case, both sides of \eqref{EqInvL2} are zero. 

Similarly, if $\UnitC{k}{m}=0$, we conclude that either $\UnitC{k}{r}=0$ or $\UnitC{r}{m}=0$, and again both sides of \eqref{EqInvL2} are zero.

This shows that \eqref{EqInvL2} holds for all indices, and hence $\{\phi_{km}\}$ is a left invariant integral. 
\end{proof} 

Hence Theorem \ref{TheoInvInt} will be proven once we can produce a family of invariant states $\phi_{kk}$ as in the previous lemma. For this, one can follow the proof as in \cite{MVD1} for the existence of an invariant state for a compact quantum group.

\begin{Prop} For each $k\in I$, there exists a state $\phi_{kk}$ in $\Gr{B}{k}{k}{k}{k}$ such that, for any state $\omega \in \Gr{B}{k}{k}{k}{k}$, one has \[\omega*\phi_{kk} =\phi_{kk}.\]
\end{Prop} 
\begin{proof} Let $k\in I$, and $\omega$ a state in $\Gr{B}{k}{k}{k}{k}$. By \cite[Lemma 4.2]{MVD1}, there exists a state $h_{kk} \in \Gr{B}{k}{k}{k}{k}$ with \[\omega *h_{kk}= h_{kk} = h_{kk}*\omega.\]

Assume now that $\rho\in \Gr{B}{k}{k}{k}{k}$ and $0\leq \rho\leq \omega$. Take $a\in A$. Then the beginning of the proof of \cite[Lemma 4.3]{MVD1}, applied with $b= (\id\otimes h_{kk})\Delta(a)$, shows that, for all $c\in A$, \begin{equation}\label{EqAuxId} (h_{kk}\otimes (\rho*h_{kk}))((c\otimes 1)\Delta(a)) = \rho(1) (h_{kk}\otimes h_{kk})((c\otimes 1)\Delta(a)).\end{equation} Since $(A\otimes A)\Delta(1) = [(A\otimes 1)\Delta(A)]$, we may replace $(c\otimes 1)\Delta(a)$ with $\UnitC{k}{k}\otimes a$ for $a\in \Gr{A}{k}{k}{k}{k}$. Then \eqref{EqAuxId} becomes $(\rho*h_{kk})(a) = \rho(1)h_{kk}(a)$. Hence \[\rho*h_{kk} = \rho(1)h_{kk}.\] 

A compactness argument as in \cite[Theorem 4.4]{MVD1} lets us conclude that there exists a state $\phi_{kk}$ as in the statement of the proposition.
\end{proof}


  Assume now that $(A,\Delta)$ is the C$^*$-algebra with comultiplication arising from a $^*$-algebraic pcqg with invariant integral $\phi$. It is easy to see that the conditions in Definition \ref{DefCpcqg} are satisfied. To see that $[(\omega\otimes \id)\Delta(pAp)\mid \omega \in A^*] = A$ for example, for $p = \sum_{k} \UnitC{k}{k}$, note that for the matrix coefficient $(\Gr{X}{k}{l}{m}{n})_{pj}$ of a unitary corepresentation, we have \[ (\Gr{X}{k}{l}{m}{n})_{pj} \in \sum_{q,r,s} \phi((\Gr{X}{m}{n}{r}{s})_{jq}A) (\Gr{X}{r}{s}{m}{n})_{qj} =(\phi(\,\cdot\, A)\otimes \id)\Delta((\Gr{X}{m}{n}{m}{n})_{jj})\] by the orthogonality relations. 


%Also note that for the $(A,\Delta)$ arising from $^*$-pcqg with invariant integral, the above conditions are all satisfied - the fourth condition can be checked using the orthogonality relations for matrix coefficients of unitary corepresentations. 





\begin{thebibliography}{00}

\bibitem{MVD1} A. Maes, A. Van Daele, Notes on compact quantum groups, \emph{Nieuw Arch. Wisk.} \textbf{16} (4) (1998), no. 1--2, 73--112. 

\end{thebibliography}


\end{document}
