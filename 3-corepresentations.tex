% Formulate this already as one arrow in the TK-reconstruction process.
% Ref to Gaby

\section{Representation theory of partial compact quantum groups}

In this section, the representation theory of partial compact quantum groups is investigated. In what follows, the homogeneous component $A(K) = \eGr{A}{k}{l}{m}{n}$ of a partial bialgebra will now be mainly written as $A(K) = \Gr{A}{k}{l}{m}{n}$. 

%As the situation is quite similar to the case already studied by Hayashi \cite{Hay1}, we do not always provide fully written out proofs, but only draw attention to those parts of the theory which need modification.


\subsection{Corepresentations of partial Hopf algebras}


Let $\mathscr{A}$ be an $I$-partial bialgebra. We write
$\Hom_\C(V,W)$ for the vector space of linear maps between two vector
spaces $V$ and $W$.

Recall that $\Vectrcf$ denotes the category whose objects are $I^{2}$-graded
vector spaces $V=\bigoplus_{k,l\in I} \Gru{V}{k}{l}$  which are row-and column-finite dimensional. Morphisms are linear maps $T$ that preserve the grading and therefore
can be written $T=\prod_{k,l\in I} \Gru{T}{k}{l}$. 
%The subcategory ofis written $\Vectrcf$ (this is a slight redefinition of the term rcf!). %Parenthetical remark formulated differently? 


%Clearly, this
%category is abelian and $\C$-linear.  We call an $I^{2}$-graded vector
%space $V=\bigoplus_{k,l\in I} \Gru{V}{k}{l}$ \emph{row- and
  %column-finite} if $\oplus_l \Gru{V}{k}{l}$ (resp. $\oplus_l
%\Gru{V}{k}{l}$) is finite-dimensional for $k$ (resp. $l$) fixed.
%We henceforth abbreviate ``row- and column-finite'' by rcf.

\begin{Def} \label{definition:corep} Let $\mathscr{A}$ be an
  $I$-partial bialgebra and let $V=\bigoplus_{k,l} \Gru{V}{k}{l}$ % lower the indices?
   be
an rcfd $I^{2}$-graded vector space.  A \emph{corepresentation}
  $\mathscr{X}=(\Gr{X}{k}{l}{m}{n})_{k,l,m,n}$ of $\mathscr{A}$ on $V$
  is a family of elements
 \begin{align} \label{eq:rep-blocks}
   \Gr{X}{k}{l}{m}{n} \in \Gr{A}{k}{l}{m}{n} \otimes
  \Hom_\C(\Gru{V}{m}{n},\Gru{V}{k}{l})
 \end{align}
 satisfying 
 \begin{align}
   \label{eq:rep-comultiplication}
    (\Delta_{pq} \otimes
    \id)(\Gr{X}{k}{l}{m}{n}) &=
    \Big{(}\Gr{X}{k}{l}{p}{q}\Big{)}_{13}\Big{(}\Gr{X}{p}{q}{m}{n}\Big{)}_{23},
    \\ \label{eq:rep-counit}
(\epsilon \otimes
  \id)(\Gr{X}{k}{l}{m}{n})&=\delta_{k,m}\delta_{l,n}\id_{\Gru{V}{k}{l}}
 \end{align}
  for all possible indices. We also call $(V,\mathscr{X})$ an
  \emph{rcfd corepresentation}.
\end{Def}
Here, we use here the standard leg numbering notation, e.g.~ $a_{23}=1\otimes a$.
\begin{Exa} \label{example:rep-triv} Equip the vector space
  $\C^{(I)}=\bigoplus_{k\in I} \C$ with the diagonal
  $I^{2}$-grading. Then the family $\mathscr{U}$ given by
  \begin{align} \label{eq:rep-triv}
    \Gr{U}{k}{l}{m}{n} = \delta_{k,l}\delta_{m,n} \UnitC{k}{m} \in
    \Gr{A}{k}{l}{m}{n}
  \end{align}
is a corepresentation of $\mathscr{A}$ on $\C^{(I)}$. We call it the
\emph{trivial corepresentation}.
\end{Exa}

\begin{Exa} \label{example:rep-regular}
  Assume  given an rcfd family of subspaces
  \begin{align*}
    \Gru{V}{m}{n} \subseteq \bigoplus_{k,l} \Gr{A}{k}{l}{m}{n}
  \end{align*}
  satisfying
  \begin{align} \label{eq:rep-regular-inclusion}
    \Delta_{pq}(\Gru{V}{m}{n}) &\subseteq \Gru{V}{p}{q} \otimes
    \Gr{A}{p}{q}{m}{n}.
  \end{align}
Then the  elements $\Gr{X}{k}{l}{m}{n} \in \Gr{A}{k}{l}{m}{n} \otimes
  \Hom_{\C}(\Gru{V}{m}{n},\Gru{V}{k}{l})$ defined by 
  \begin{align*}
    \Gr{X}{k}{l}{m}{n}(1 \otimes b) &= \Delta^{\op}_{kl}(b) \in
    \Gr{A}{k}{l}{m}{n} \otimes \Gru{V}{k}{l} \quad
    \text{for all } b\in \Gru{V}{m}{n}
  \end{align*}
  form a corepresentation $\mathscr{X}$ of $\mathscr{A}$ on
  $V$. Indeed, 
  \begin{align*}
    (\Delta_{pq} \otimes \id)(\Gr{X}{k}{l}{m}{n})(1 \otimes 1 \otimes
    b) &=(\Delta_{pq}\otimes \id)(\Delta^{\op}_{kl}(b)) =
    \Big{(}\Gr{X}{k}{l}{p}{q}\Big{)}_{13}\Big{(}\Gr{X}{p}{q}{m}{n}\Big{)}_{23}(1
    \otimes 1 \otimes b), \\
    (\epsilon \otimes \id)(\Gr{X}{k}{l}{m}{n})b &= (\epsilon \otimes
    \id)(\Delta^{\op}_{kl}(b)) = \delta_{k,m}\delta_{l,n}b
  \end{align*}
  for all $b\in \Gru{V}{m}{n}$.  We call $\mathscr{X}$ the
  \emph{regular corepresentation on $V$}. 
\end{Exa}

We next consider the total form of a corepresentation.

Let $\mathscr{A}$ be a partial bialgebra with total algebra $A$, and
let $V$ be an rcfd $I^{2}$-graded vector space.
Denote by $\lambda^{V}_{k},\rho^{V}_{l} \in \Hom_{\C}(V)$ the
projections onto the summands $\Gru{V}{k}{} = \bigoplus_{q}
\Gru{V}{k}{q}$, write $\Gru{V}{}{l}=\bigoplus_{p}\Gru{V}{p}{l}$,
respectively, and identify $\Hom_{\C}(\Gru{V}{m}{n},\Gru{V}{k}{l})$ with
$\lambda^{V}_{k}\rho^{V}_{l}\Hom_{\C}(V)\lambda^{V}_{m}\rho^{V}_{n}$. Denote by $\Hom_{\C}^{0}(V) \subseteq \Hom_{\C}(V)$ the algebraic sum of all
these subspaces. Then we can define a homomorphism
\begin{align*}
  \Delta \otimes \id \colon M(A \otimes \Hom_{\C}^{0}(V)) \to M(A
  \otimes A \otimes \Hom_{\C}^{0}(V))
\end{align*}
similarly as we defined $ \Delta \colon A \to M(A\otimes A)$.
\begin{Lem} \label{lemma:rep-multiplier}
  Let  $\mathscr{A}$ be an $I$-partial bialgebra and $V$  an rcfd $I^{2}$-graded vector space.  If $\mathscr{X}$ is a
  corepresentation of  $\mathscr{A}$ on $V$, then the sum
  \begin{align}
    \label{eq:rep-multiplier}
  X:=\sum_{k,l,m,n} \Gr{X}{k}{l}{m}{n} \in  M(A
  \otimes \Hom_{\C}^{0}(V))
  \end{align}
 converges strictly and satisfies the following conditions:
  \begin{enumerate}[label=(\arabic*)] %\setcounter{enumi}{-1}
  \item\label{repma} $(\lambda_{k}\rho_{m} \otimes \id){X}(\lambda_{l}\rho_{n}
    \otimes \id) = (1 \otimes \lambda^{V}_{k}\rho^{V}_{l}){X}(1 \otimes
    \lambda^{V}_{m}\rho^{V}_{n}) = \Gr{X}{k}{l}{m}{n}$,
  \item\label{repmb} $(A \otimes 1){X}$, $ {X}(A \otimes 1)$ and $(1 \otimes
    \Hom^{0}_{\C}(V))X(1 \otimes \Hom^{0}_{\C}(V))$ lie in $A \otimes \Hom_{\C}^{0}(V)$,%Last inlcusion superfluous, necessary to include?
  \item\label{repmc} $(\Delta\otimes \id)(X)=X_{13}X_{23}$, 
  \item\label{repmd} the sum $(\epsilon \otimes \id)({X}) :=\sum (\epsilon \otimes
    \id)(\Gr{X}{k}{l}{m}{n})$ converges in $M(\Hom^{0}_{\C}(V))$ strictly
    to $\id_{V}$.
  \end{enumerate}
  Conversely, if $ X \in M(A \otimes \Hom_{\C}^{0}(V))$ satisfies
  \ref{repma}--\ref{repmd} with $\Gr{X}{k}{l}{m}{n}$ defined by \ref{repma}, then
  $\mathscr{X}=(\Gr{X}{k}{l}{m}{n})_{k,l,m,n}$ is a corepresentation
  of $\mathscr{A}$ on $V$.
\end{Lem}
\begin{proof}
 Straightforward.
\end{proof}

\begin{Def} If $\mathscr{X}$ and $X$ are as in Lemma \ref{lemma:rep-multiplier}, we will call $X$ the \emph{corepresentation multiplier} of $\mathscr{X}$. 
\end{Def}


If $\mathscr{A}$ is a partial Hopf algebra,  then every
corepresentation multiplier has a generalized inverse.
\begin{Lem} \label{lemma:rep-invertible}
  Let $(V,\mathscr{X})$ be an rcfd corepresentation of an $I$-partial Hopf
  algebra $\mathscr{A}$. Then with $\Gr{Z}{k}{l}{m}{n} = (S\otimes \id)(\Gr{X}{n}{m}{l}{k})$, we have $\Gr{Z}{k}{l}{m}{n}\in \Gr{A}{k}{l}{m}{n}\otimes \Hom_{\C}(\Gru{V}{l}{k},\Gru{V}{n}{m})$ and
  \begin{align*}
    \Gr{X}{k}{l}{m}{n}  \cdot \Gr{Z}{l}{k'}{n}{m'} &=0 \text{ if } m'\neq m &
      \sum_{n} \Gr{X}{k}{l}{m}{n} \cdot \Gr{Z}{l}{k'}{n}{m} &= \delta_{k,k'}\UnitC{k}{m} \otimes
      \id_{\Gru{V}{k}{l}}, \\
      \Gr{Z}{n}{m}{l}{k}\cdot \Gr{X}{m}{n'}{k}{l'} &= 0
      \text{ if } n\neq n' & 
      \sum_{m} \Gr{Z}{n}{m}{l}{k}\cdot \Gr{X}{m}{n}{k}{l'} &=
      \delta_{l,l'} \UnitC{n}{l} \otimes \id_{\Gru{V}{k}{l}}.
  \end{align*}
  In particular, the multiplier $Z:=     (S \otimes
  \id)(X) \in M(A \otimes \Hom_{\C}^{0}(V))$
  satisfies
  \begin{align} \label{eq:rep-generalized-inverse}
    XZ &= \sum_{k} \lambda_{k} \otimes \lambda^{V}_{k}, &
    ZX &= \sum_{l} \rho_{l} \otimes \rho^{V}_{l},
  \end{align}
  and is a generalized inverse of $X$ in the sense that $XZX=X$ and $ZXZ=Z$.
\end{Lem}
\begin{proof}
  The grading property of $\Gr{Z}{k}{l}{m}{n}$ follows from  $S(\Gr{A}{p}{q}{r}{s})\subseteq \Gr{A}{s}{r}{q}{p}$, and then the upper left hand identity is immediate.  To
  verify the upper right hand one, we use identities \eqref{eq:rep-comultiplication}, \eqref{eq:rep-counit} and \eqref{eq:antipode-pi-l}. Namely, with $M_{A}$ denoting the multiplication of $A$, we find
  \begin{align*}
      \sum_{n} \Gr{X}{k}{l}{m}{n} \cdot (S \otimes
      \id)(\Gr{X}{m}{n}{k'}{l}) &= \sum_{n} (M_{A}  (\id \otimes S)
      \otimes \id)((\Gr{X}{k}{l}{m}{n})_{13}(\Gr{X}{m}{n}{k'}{l})_{23})
 \\ &= \sum_{n} (M_{A} (\id \otimes S)  \Delta_{m,n} \otimes
      \id)(\Gr{X}{k}{l}{k'}{l}) \\
      &= \delta_{k,k'} \UnitC{k}{l} \otimes (\epsilon \otimes
      \id)(\Gr{X}{k}{l}{k'}{l})
      \\ &=
\delta_{k,k'}\UnitC{k}{m} \otimes
      \id_{\Gru{V}{k}{l}}. \end{align*} The other
equations follow similarly, and the assertions concerning $Z$ are
direct consequences.
\end{proof}
\begin{Def}
  Let $\mathscr{X}$ be an rcfd corepresentation of a  partial Hopf
  algebra.  We  denote the generalized inverse $(S \otimes \id)(X)$
  of $X$  by $X^{-1}$ and let
  \begin{align*}
   \Gr{(X^{-1})}{k}{l}{m}{n}=(S \otimes \id)(\Gr{X}{l}{k}{n}{m}) \in
   \Gr{A}{k}{l}{m}{n} \otimes \Hom_{\C}(\Gru{V}{l}{k},\Gru{V}{n}{m})
  \end{align*}% Prefer to have grading in this way...
\end{Def}
For completeness, we mention the following following converse to Lemma \ref{lemma:rep-invertible}
\begin{Lem}
  Let $\mathscr{A}$ be an $I$-partial bialgebra, $V$ an rcfd $I^{2}$-graded vector space and $X,Z \in M(A \otimes
  \Hom_{\C}^{0}(V))$. If conditions \ref{repma}--\ref{repmc} in Lemma
  \ref{lemma:rep-multiplier} and
  \eqref{eq:rep-generalized-inverse} hold, then the corresponding
  family $\mathscr{X}=(\Gr{X}{k}{l}{m}{n})_{k,l,m,n}$ is a
  corepresentation of $\mathscr{A}$ on $V$.
\end{Lem}
\begin{proof}
  We have to verify condition \ref{repmd} in Lemma
  \ref{lemma:rep-multiplier}.  If $(k,l) \neq (p,q)$, then
  $\epsilon(\Gr{A}{k}{l}{p}{q})=0$ and hence $(\epsilon
  \otimes \id)(\Gr{X}{k}{l}{p}{q}) =0$. The counit property and condition
  \ref{repmc} in Lemma \ref{lemma:rep-multiplier} imply 
\begin{align*}
  \Gr{X}{k}{l}{m}{n} &= ((\epsilon\otimes \id)  \Delta \otimes
  \id)(  \Gr{X}{k}{l}{m}{n}) 
\\ &  = \sum_{p,q} (\epsilon\otimes \id \otimes
  \id)\left((\Gr{X}{k}{l}{p}{q})_{13}(\Gr{X}{p}{q}{m}{n})_{23}\right)
  =  (1 \otimes \Gru{T}{k}{l})\Gr{X}{k}{l}{m}{n},
\end{align*}
where $\Gru{T}{k}{l}=(\epsilon \otimes \id)(\Gr{X}{k}{l}{k}{l}) \in
\Hom_{\C}(\Gru{V}{k}{l})$.  Therefore,  $T=\prod_{k,l} T_{k,l}$  satisfies $(1 \otimes T)X =
X$. Multiplying on the right by $Z$, we find
$T\lambda^{V}_{k}=\lambda^{V}_{k}$ for all $k$. Thus, $T=\id_{V}$.
\end{proof}

Morphisms of corepresentations are defined as follows.
\begin{Def}
  Let $\mathscr{A}$ be an $I$-partial bialgebra.  A \emph{morphism}
  $T$ between rcfd corepresentations
  $(V,\mathscr{X})$ and $(W,\mathscr{Y})$ of $\mathscr{A}$ is a family
  of linear maps
  \[\Gru{T}{k}{l} \in
  \Hom_\C(\Gru{V}{k}{l},\Gru{W}{k}{l})\] satisfying \[(1 \otimes
  \Gru{T}{k}{l})\Gr{X}{k}{l}{m}{n} = \Gr{Y}{k}{l}{m}{n}(1 \otimes
  \Gru{T}{m}{n})\]
\end{Def}
We denote the category of all corepresentations of $\mathscr{A}$ on rcfd $I^2$-graded vector spaces by
$\Corep_{\rcf}(\mathscr{A})$.
\begin{Rem} \label{remark:rep-total-morphism}
 Equivalently, a morphism between
    $(V,\mathscr{X})$ and $(W,\mathscr{Y})$ is just a morphism of
    $I^{2}$-graded vector spaces $T\colon V\to W$ satisfying
    $(1\otimes T) X= Y(1 \otimes T)$. If $\mathscr{A}$ is  a
    partial Hopf algebra, this condition is equivalent to each of the relations
    \begin{align*}
      Y^{-1}(1 \otimes T)X&=\sum_{m,n} \rho_{n} \otimes \Gru{T}{m}{n},
      &
    Y(1\otimes T)X^{-1} &=\sum_{k,l} \lambda_{k} \otimes \Gru{T}{k}{l}.
    \end{align*}
\end{Rem}


Given an rcfd $I^{2}$-graded vector space $V=\bigoplus_{k,l} \Gru{V}{k}{l}$
and a family of subspaces $\Gru{W}{k}{l} \subseteq \Gru{V}{k}{l}$, we
denote by $\iota_{W}\colon W\to V$ and $\pi_{W} \colon V \to
V/W=\bigoplus_{k,l} \Gru{V}{k}{l}/\Gru{W}{k}{l}$ the embedding and the
quotient map.
\begin{Def} Let $(V,\mathscr{X})$ be an rcfd
  corepresentation of a partial bialgebra $\mathscr{A}$.  We call a
  family of subspaces $\Gru{W}{k}{l} \subseteq \Gru{V}{k}{l}$
  \emph{invariant (w.r.t.\ $\mathscr{X}$)} if
 \begin{align} \label{eq:rep-invariant} (1\otimes
   \Gr{\pi}{k}{l}{}{W})\Gr{X}{k}{l}{m}{n}(1 \otimes
   \Gr{\iota}{m}{n}{}{W}) =0.
  \end{align}
We call $(V,\mathscr{X})$ 
 \emph{irreducible} if the only invariant families of subspaces are
 $(0)_{k,l}$ and $(\Gru{V}{k}{l})_{k,l}$.
\end{Def}

The next lemmas deal with restriction, factorisation and Schur's lemma. We skip their proofs which are straightforward.

%In the next lemma, we show that if a corepresentation has an invariant subspace, it restricts to it and factorizes to the quotient.
\begin{Lem}
  Let $(V,\mathscr{X})$ be an rcfd corepresentation
  of a partial bialgebra and let $\Gru{W}{k}{l}
  \subseteq \Gru{V}{k}{l}$ be an invariant family of subspaces. Then
  there exist unique rcfd corepresentations
  $(W,\iota_{W}^{*}\mathscr{X})$ and $(V/W,(\pi_{W})_{*}\mathscr{X})$ 
  such that $\iota_{W}$  and  $\pi_{W}$  are  morphisms  $(W,\iota_{W}^{*}\mathscr{X}) \to (V,\mathscr{X}) \to (V/W,(\pi_{W})_{*}\mathscr{X})$.
\end{Lem}
%\begin{proof}
 % Straightforward.
%\end{proof}
%The following analogue of Schur's Lemma holds.
\begin{Lem} Let $T$ be a morphism of rcfd
  corepresentations $(V,\mathscr{X})$ and $(W,\mathscr{Y})$ of a
  partial bialgebra. Then the families of subspaces $\ker
  \Gru{T}{k}{l} \subseteq \Gru{V}{k}{l}$ and $\img\Gru{T}{k}{l}
  \subseteq \Gru{W}{k}{l}$ are invariant.  In particular, if
  $(V,\mathscr{X})$ and $(W,\mathscr{Y})$ are irreducible, then either
  all $\Gru{T}{k}{l}$ are zero or all $\Gru{T}{k}{l}$ are
  isomorphisms.
\end{Lem} 
%\begin{proof}
  %Straightforward again.
%\end{proof}


Given corepresentations $\mathscr{X}$ and $\mathscr{Y}$ of
a partial bialgebra $\mathscr{A}$ on respective rcfd $I^{2}$-graded vector spaces $V$ and $W$,
we  obtain an $I^{2}$-graded vector space $V\oplus W$ by taking
component-wise direct sums, and use the canonical embedding 
\begin{align*}
  \Hom(\Gru{V}{m}{n},\Gru{V}{k}{l}) \oplus
  \Hom(\Gru{W}{m}{n},\Gru{W}{k}{l}) \hookrightarrow
  \Hom(\Gru{V}{m}{n} \oplus \Gru{W}{m}{n},\Gru{V}{k}{l} \oplus
  \Gru{W}{k}{l})
\end{align*}
to define the \emph{direct sum} $\mathscr{X} \oplus \mathscr{Y}$,
which is a corepresentation of $\mathscr{A}$ on $V\oplus W$. Then the
natural embeddings from $V$ and $W$ into $V\oplus W$ and the
projections onto $V$ and $W$ are evidently morphisms of
corepresentations.  More generally, given a family of rcfd corepresentations
$((V_{\alpha},\mathscr{X}_{\alpha}))_{\alpha}$ such that the sum
$\bigoplus_{\alpha} V_{\alpha}$ is rcfd again, one
can form the direct sum $\bigoplus_{\alpha} \mathscr{X}_{\alpha}$,
which is a corepresentation on $\bigoplus_{\alpha} V_{\alpha}$.
\begin{Prop}
  Let $\mathscr{A}$ be an $I$-partial bialgebra. Then $\Corep_{\rcf}(\mathscr{A})$
  is a $\C$-linear abelian category, and the forgetful functor
  $\Corep_{\rcf}(\mathscr{A}) \to \Vectrcf$ lifts kernels, cokernels and biproducts.
\end{Prop}
\begin{proof}
  The preceding considerations show that the forgetful functor lifts
  kernels, cokernels and biproducts. Moreover, in
  $\Corep_{\rcf}(\mathscr{A})$, every monic is a kernel
  and every epic is a cokernel because the same is true in $\Vecti$
  and because kernels and cokernels lift.
\end{proof}


\subsection{Tensor product and duality}

Recall from Example \ref{ExaVectBiGr} that the category $\Vectrcf$ is a tensor category. The tensor product of morphisms is the
restriction of the ordinary tensor product.  We will interpret this product as being strictly associative.  The unit for this product is the vector
space $\C^{(I)}=\bigoplus_{k\in I} \C$. 
%Indeed, for every
%$I^{2}$-graded vector space $V$, there exist obvious natural
%isomorphisms $\C^{(I)} \itimes V \cong V \cong V \itimes \C^{(I)}$.

%Note that $V\itimes W$ is  rcf if $V$ and $W$ are.

Given $V$ and $W$ in $\Vectrcf$, we identify $\Hom_\C(\Gru{V}{m}{n},\Gru{V}{k}{l})\otimes
   \Hom_\C(\Gru{W}{n}{q},\Gru{W}{l}{p})$ with a subspace of
\begin{align*}
   \Hom_\C(\Gru{V}{m}{n}\otimes
   \Gru{W}{n}{q},\Gru{V}{k}{l}\otimes \Gru{W}{l}{p})\subseteq
   \Hom_\C(\Gru{(}{m}{}V\itimes
     W\Gru{)}{}{q},\Gru{(}{k}{}V\itimes W\Gru{)}{}{p}).
\end{align*}


We can now construct a product of corepresentations as follows.
\begin{Lem} Let $\mathscr{X}$ and $\mathscr{Y}$ be copresentations of
  $\mathscr{A}$ on respective  rcfd $I^{2}$-graded vector spaces $V$ and
  $W$. Then the sum
  \begin{align} \label{eq:rep-product-blocks}
     \Gr{(X\Circt Y)}{k}{p}{m}{q} := \sum_{l,n}
    \left(\Gr{X}{k}{l}{m}{n}\right)_{12}\left(\Gr{Y}{l}{p}{n}{q}\right)_{13}
  \end{align}
  has only finitely many non-zero terms, and the elements
 \[\Gr{(X\Circt
    Y)}{k}{p}{m}{q}\in \Gr{A}{k}{p}{m}{q} \otimes
  \Hom_\C(\Gru{(}{m}{}V\itimes W\Gru{)}{}{q},\Gru{(}{k}{}V\itimes W\Gru{)}{}{p})
\]
define an rcfd corepresentation $\mathscr{X} \Circt \mathscr{Y}$ of
$\mathscr{A}$ on $V\itimes W$. 
\end{Lem} 
\begin{proof}
  The sum \eqref{eq:rep-product-blocks} is finite because $V$ and
  $W$ are  rcfd. Using the identification above, we
  see that
 \[
  \left(\Gr{X}{k}{l}{m}{n}\right)_{12}\left(\Gr{Y}{l}{p}{n}{q}\right)_{13}\in \Gr{A}{k}{p}{m}{q} \otimes \Hom_\C(\Gru{(}{m}{}V\itimes
    W\Gru{)}{}{q},\Gru{(}{k}{}V\itimes W\Gru{)}{}{p}).\] Now,   the fact that $\Gr{(X\Circt
    Y)}{k}{p}{m}{q}$ is a corepresentation follows easily
  from the multiplicativity of $\Delta$ and the weak multiplicativity
  of $\epsilon$.
\end{proof}
\begin{Rem} \label{remark:rep-tensor-multiplier}
  The `total' multiplier corepresentation associated to $\mathscr{X}\Circt
  \mathscr{Y}$   is  just $X_{12}Y_{13}$.
\end{Rem}

\begin{Prop} \label{prop:rep-tensor} Let $\mathscr{A}$ be an
  $I$-partial bialgebra. Then  $\Corep_{\rcf}(\mathscr{A})$ carries the
  structure of strict tensor category such that the product of rcfd corepresentations $(V,\mathscr{X})$ and
  $(W,\mathscr{Y})$ is the corepresentation $(V\itimes
  W,\mathscr{X}\Circt \mathscr{Y})$, the unit is the trivial
  corepresentation $(\C^{(I)},\mathscr{U})$, and the forgetful functor
  $\Corep_{\rcf}(\mathscr{A}) \to \Vecti$ is a strict tensor functor.
\end{Prop}
\begin{proof}
This is clear.
\end{proof}


% Change notation from right duals to left duals !
Given an rcfd corepresentation of a partial Hopf algebra, one can use the
antipode to define a contragredient corepresentation on a dual space.
Denote the dual of vector spaces $V$ and  linear maps $T$ by
$\dual{V}$ and $\dualop{T}$, respectively, and define the dual of an
$I^{2}$-graded vector space $V=\bigoplus_{k,l} \Gru{V}{k}{l}$ to be
the space
\begin{align*}
  \dual{V}=\bigoplus_{k,l} \Gru{(\dual{V})}{k}{l}, \quad \text{where }
\Gru{(\dual{V})}{k}{l} = \dual{(\Gru{V}{l}{k})}.
\end{align*}


%Recall that an object $X$ in a strict tensor category is called a
%\emph{right dual} of an object $Y$ and $Y$ is called a \emph{left
 % dual} of $X$, if there are morphisms $X \otimes Y \to 1$ and $1 \to
%Y \otimes X$, where $1$ denotes the tensor unit, such
%that the obvious compositions
 % \begin{gather*}
 %X \otimes 1 \to X\otimes Y\otimes X
 % \to 1 \otimes X  \quad\text{and} \quad
  %   1 \otimes Y \to Y \otimes X \otimes Y \to Y
   % \otimes 1 
  %\end{gather*}
 % are the identity of $X$ and $Y$,
 % respectively.  


\begin{Prop}
  Let $\mathscr{A}$ be an $I$-partial Hopf algebra with antipode $S$
  and  let $(V,\mathscr{X})$ be an rcfd
  corepresentation of $\mathscr{A}$. Then $\dual{V}$ and the family
  $\dualco{\mathscr{X}}$ given by
   \begin{align} \label{eq:rep-left-dual}
\Gr{\dualco{X}}{k}{l}{m}{n}   :=  (S \otimes \dualop{-})(\Gr{X}{n}{m}{l}{k}) 
   \end{align} 
   form an rcfd corepresentation of $\mathscr{A}$  which is a left dual of $(V,\mathscr{X})$. If the antipode
   $S$ of $\mathscr{A}$ is bijective, then $\dual{V}$ and the family
   $\dualcor{\mathscr{X}}$ given by 
   \begin{align} \label{eq:rep-right-dual}
 \Gr{\dualcor{X}}{k}{l}{m}{n} :=(S^{-1}
   \otimes \dualop{-})(\Gr{X}{n}{m}{l}{k})    
   \end{align}
 form an rcfd corepresentation
 of $\mathscr{A}$ which is a
   right dual of $(V,\mathscr{X})$.
  \end{Prop}
  \begin{proof}
    We only prove the assertion concerning
    $(\dual{V},\dualco{\mathscr{X}})$. To see that this is a corepresentation, note that the element
    \eqref{eq:rep-left-dual} belongs to $\Gr{A}{k}{l}{m}{n} \otimes
    \Hom_{\C}(\Gru{(\dual{V})}{m}{n},\Gru{(\dual{V})}{k}{l})$ and use
    the relations $\Delta \circ S = (S \otimes S)\Delta^{\op}$ and
    $\epsilon \circ S = \epsilon$ from Corollary
    \ref{corollary:antipode} and Lemma \ref{LemCoAnt}.  
    Let us show that $(\dual{V},\dualco{\mathscr{X}})$ is a left dual
    of $(V,\mathscr{X})$.

    Given a finite-dimensional vector space $W$, denote by $\ev_{W}
    \colon \dual{W} \otimes W \to \C$ the evaluation map and by $\coev_{W}
    \colon \C \to W \otimes \dual{W}$ the coevaluation map, given by
    $1\mapsto \sum_{i} w_{i} \otimes \dual{w_{i}}$ if $(w_{i})_{i}$
    and $(\dual{w_{i}})_{i}$ are dual bases of $W$ and
    $\dual{W}$. With respect to these maps, $\dual{W}$ is a left dual
    of $W$. If $F\colon W_{1}\to W_{2}$ is a linear map between
    finite-dimensional spaces, then
\begin{align} \label{eq:coev-vee} (\id_{W_{2}} \otimes F^{\tr}) \circ \coev_{W_{2}} &= (F \otimes \id_{W_{1}^{*}})\circ
  \coev_{W_{1}}, &
\ev_{W_{1}}(F^{\tr}
  \otimes \id_{W_{2}})&=  \ev_{W_{2}}(\id_{W_{2}^{*}} \otimes F).
\end{align}

Now, define morphisms $\coev \colon \C^{(I)} \to V\itimes \dual{V}$ and
$\ev \colon \dual{V} \itimes V \to \C^{(I)}$ by
\begin{align*}
  \Gru{\coev}{k}{l} &= \delta_{k,l} \sum_{p} \coev_{\footnotesize\Gru{V}{k}{p}} \colon
  \C \to 
    \Gru{(}{k}{}V\itimes \dual{V}\Gru{)}{}{l}, &
  \Gru{\ev}{k}{l} &= \delta_{k,l} \sum_{p} \ev_{\Gru{V}{p}{k}} \colon
    \Gru{(}{k}{}V\itimes \dual{V}\Gru{)}{}{l} \to \C.
\end{align*}
One easily checks that with respect to these maps, $\dual{V}$ is a
left dual of $V$ in $\Vectrcf$. 

We therefore only need to show that $\ev$ is a morphism from
$\dualco{\mathscr{X}}\Circt\mathscr{X}$ to $\mathscr{U}$ and that $\coev$ is
a morphism from $\mathscr{U}$ to
$\mathscr{X}\Circt\dualco{\mathscr{X}}$.  But \eqref{eq:coev-vee} and
Lemma \ref{lemma:rep-invertible} imply
  \begin{align*}
    (1\otimes \Gru{\ev}{k}{k})
 \sum_{l,n}  \big(
\Gr{\dualco{X}}{k}{l}{m}{n}\big)_{12}
\big(\Gr{X}{l}{k}{n}{q}\big)_{13} &=
    (1\otimes \Gru{\ev}{k}{k})
 \sum_{l,n} 
(S \otimes \dualop{-})(\Gr{X}{n}{m}{l}{k})_{12}
    (\Gr{X}{l}{k}{n}{q})_{13} \\ &=
(1\otimes \Gru{\ev}{m}{m})  \sum_{l,n}
      (S \otimes \id)(\Gr{X}{n}{m}{l}{k})_{13}(\Gr{X}{l}{k}{n}{q})_{13} \\
    &= \delta_{m,q}\UnitC{k}{q}\otimes \Gru{\ev}{m}{m} \\
    &= \Gr{U}{k}{k}{m}{q}(1 \otimes \Gru{\ev}{m}{m}).
  \end{align*}
A similar  calculation shows that also $\coev$ is a morphism, whence the claim follows.

%\begin{align*}
 %\sum_{l,n}  \big(
%\Gr{X}{k}{l}{m}{n}\big)_{12}
%\big(\Gr{(\dual{X})}{l}{p}{n}{m}\big)_{13} 
 %   (1\otimes \Gru{\coev}{m}{m})
%&= \delta_{k,p} \UnitC{k}{m} \otimes \Gru{\coev}{k}{k} = (1 \otimes
%\Gru{\coev}{k}{k}) \Gr{U}{k}{p}{m}{m},
%\end{align*}
%whence the claim follows.
\end{proof}
\begin{Cor} \label{cor:rep-tensor-duality}
  Let $\mathscr{A}$ be a partial Hopf algebra. Then
  $\Corep_{\rcf}(\mathscr{A})$ is a tensor category with left
  duals and, if the antipode of $\mathscr{A}$ is invertible, with right duals.
\end{Cor}

Let $\mathscr{A}$ be an $I$-partial Hopf algebra.  Then the tensor
unit in $\Corep_{\rcf}(\mathscr{A})$, which is the trivial corepresentation
$\mathscr{U}$ on $\C^{(I)}$, need not be irreducible. Instead, it decomposes
into irreducible corepresentations indexed by the hyperobject set $\mathscr{I}$ of equivalence
classes for the relation $\sim$ on $I$ given by  $k \sim l \iff
  \UnitC{k}{l}\neq 0$ (see Remark
\ref{remark:index-equivalence}).
\begin{Lem}
  Let $\mathscr{A}$ be an $I$-partial Hopf algebra and let
  $(I_{\alpha})_{\alpha\in \mathscr{I}}$ be a labelled partition of $I$ into
  equivalence classes for the relation $\sim$.  Then for each $\alpha\in \mathscr{I}$, the subspace
  $\C^{(I_{\alpha})} \subseteq \C^{(I)}$ is invariant, and the restriction
  $\mathscr{U_{\alpha}}$ of $\mathscr{U}$ to $\C^{(I_{\alpha})}$ is
  irreducible. In particular, $\mathscr{U}=\bigoplus_{\alpha\in\mathscr{I}}
  \mathscr{U_{\alpha}}$ is a decomposition into irreducible corepresentations.
\end{Lem}
\begin{proof}
Immediate from the fact that $\Gr{U}{k}{k}{m}{m} = 
  \UnitC{k}{m}$  is $1$  if $k\sim m$  and $0$ if $k\not\sim m$. 
\end{proof}

\begin{Def} We denote by $\Corep(\mathscr{A})$ the category of rcfd corepresentations $(V,\mathscr{X})$ for which there exists a finite subset of the hyperobject set $\mathscr{I}$ such that $\Gru{V}{k}{l}=0$ for the equivalence classes of $k,l$ outside this subset.
\end{Def}

It is easily seen that $\Corep(\mathscr{A})$ is then a tensor category with local units indexed by $\mathscr{I}$. We will use the same notation for the associated partial tensor category. 

%The decomposition of the tensor unit leads to a decomposition of the
%whole tensor category into full subcategories, where the tensor
%product acts like the multiplication in a partial algebra.


\subsection{Decomposition into irreducible corepresentations}



When there is an invariant integral around, one can average morphisms of vector spaces to obtain morphisms of corepresentations. 
\begin{Lem} \label{lem:rep-average}  Let $(V,\mathscr{X})$ and
  $(W,\mathscr{Y})$ be rcfd corepresentations of  a partial
  Hopf algebra $\mathscr{A}$ with an invariant integral $\phi$, and let
  $\Gru{T}{k}{l} \in \Hom_{\C}(\Gru{V}{k}{l},\Gru{W}{k}{l})$ for all $k,l\in I$. Then for each $m,n$ fixed, the families
  \begin{align*}
    \Gr{\check T}{m}{n}{k}{l} &:= (\phi \otimes
    \id)(\Gr{(Y^{-1})}{n}{m}{l}{k}(1\otimes
    \Gru{T}{m}{n})\Gr{X}{m}{n}{k}{l}), \\
    \Gr{\hat T}{m}{n}{k}{l} &:=(\phi \otimes
    \id)(\Gr{Y}{k}{l}{m}{n}(1\otimes
    \Gru{T}{m}{n})\Gr{(X^{-1})}{l}{k}{n}{m})
  \end{align*}
  % think it's ok to keep this notation as we did not apply the previous hat and check to operators...
form  morphisms $\Grd{\check{T}}{m}{n}$ and $\Grd{\hat{T}}{m}{n}$ from $(V,\mathscr{X})$ to $(W,\mathscr{Y})$. % Slightly changed the averaging so that I do not need a restriction of finite support on $T$.
\end{Lem} 
\begin{proof} Clearly, we may suppose that $T$ is supported only on the component at index $(m,n)$, and we may then drop the upper indices and simply write $\Gru{\check{T}}{k}{l}$ and $\Gru{\hat{T}}{k}{l}$. Then 
 in total form, $\check{T}=(\phi \otimes \id)(Y^{-1}(1 \otimes T)X)$
  and $\hat{T}=(\phi \otimes \id)(Y(1 \otimes T)X^{-1})$.  Now, Lemma
  \ref{lemma:rep-multiplier} and Lemma \ref{lemma:total-integral} 
  imply
  \begin{align*}
    Y^{-1}(1 \otimes \check{T})X &= (\phi \otimes \id \otimes
    \id)((Y^{-1})_{23}(Y^{-1})_{13}(1 \otimes 1
    \otimes T)X_{13}X_{23})  \\
    &= ((\phi \otimes\id)  \Delta  \otimes \id)(Y^{-1}(1 \otimes T)X) \\
    &= \sum_{l} \rho_{l} \otimes (\phi \otimes \id)((\rho_{l} \otimes
    1)Y^{-1}(1 \otimes T)X)  \\
    &= \sum_{k,l} \rho_{l} \otimes \Gru{\check T}{k}{l},
  \end{align*}
  whence $\check{T}$ is a morphism from $\mathscr{X}$ to $\mathscr{Y}$
  by Remark \ref{remark:rep-total-morphism}. The assertion for $\hat
  T$ follows similarly.
\end{proof}

% Choose a representative family of unitary irreducible locally finite
% corepresentations
% $(\Grd{\mathcal{H}}{(\alpha)}{},{_{(\alpha)}X})_{\alpha}$ and a basis
% $(\Gr{\zeta}{k}{l}{(\alpha)}{i})_{i}$ for each
% $\Gr{\mathcal{H}}{k}{l}{(\alpha)}{}$, and let
% \begin{align*}
%   (\Gr{(u_{\alpha})}{k}{l}{m}{n}){i,j} &:= (\id \otimes
%   \Gr{\omega}{k}{l}{(\alpha)}{i,j})( )
% \end{align*}

\begin{Lem}
  Let $\mathscr{A}$ be an $I$-partial Hopf algebra with an invariant integral $\phi$.
  Let $(V,\mathscr{X})$ be an rcfd corepresentation
  and $\Gru{W}{k}{l} \subseteq \Gru{V}{k}{l}$ an invariant family of
  subspaces. Then there exists an idempotent endomorphism $T$ of
  $(V,\mathscr{X})$ such that $\Gru{W}{k}{l}=\img\Gru{T}{k}{l}$ for
  all $k,l$.
\end{Lem}
\begin{proof}
By a direct sum decomposition, we may assume that $V$ is in a fixed component $\Corep(\mathscr{A})_{\alpha\beta}$. For all $k\in I_{\alpha},l\in I_{\beta}$, choose idempotent endomorphisms $\Gru{T}{k}{l}$ of $\Gru{V}{k}{l}$
  with image $\Gru{W}{k}{l}$. By Lemma \ref{lem:rep-average}, we obtain
  endomorphisms $\Grd{\check{T}}{m}{n}$ of $(V,\mathscr{X})$. We want to show
  that linear combinations of these provide the sought-after morphism. %$\img \Gru{\check{T}}{k}{l}= \Gru{W}{k}{l}$. 
  
    In
  total form, invariance of $W$ implies  \[(1 \otimes T)X(1
  \otimes T)=X(1\otimes T).\] Applying
 $(S \otimes \id)$, we get   \[(1 \otimes T)X^{-1}(1
  \otimes T)=X^{-1}(1\otimes T).\]
Now choose $n\in I_{\beta}$ and write $\Grd{\check{T}}{}{n} = \sum_m \Grd{\check{T}}{m}{n}$, which makes sense because of column-finiteness of $V$. We combine  Lemma
  \ref{lemma:rep-multiplier}, Lemma \ref{lemma:rep-invertible} and
  normalisation of $\phi$, and find
  \begin{align*}
    \Grd{\check{T}}{}{n} T &= (\phi \otimes \id)(X^{-1}(1 \otimes
    \rho_{n}^{V}T)X(1 \otimes T)) \\  &= 
     (\phi \otimes \id)(X^{-1}(1 \otimes
    \rho_{n}^{V})X(1 \otimes T)) \\
    &=
  \sum_l \phi(\UnitC{n}{l}) \rho^{V}_{l}T \\& =T,
  \end{align*}
 since we only have to sum over $l\in I_{\beta}$ and $n\in \mathscr{I}_{\beta}$ by assumption. 
 
 Now as $W$ is invariant and $T$ sends $V$ into $W$, we have that $\Gr{\check{T}}{}{n}{k}{l}$ sends $\Gru{V}{k}{l}$ into $\Gru{W}{k}{l}$. Hence it follows that $\img{\check{T}^{n}}=\img T$, and $\check{T}^{n}$ is the sought-after intertwiner.

  %  Similarly, $T\check{T}_{n} =T$. Therefore,
  
\end{proof}
% Make the notion of cosemisimplicity formal, show equivalence existence invariant functionals and cosemisimplicity.
\begin{Cor}  \label{cor:rep-cosemisimple}% Changed! See if later refs are still ok.
  Let $\mathscr{A}$ be a partial Hopf algebra with an invariant integral.  Then
  every rcfd corepresentation of $\mathscr{A}$ decomposes into a (possibly infinite) direct
  sum of irreducible rcfd corepresentations.
\end{Cor} 
\begin{proof} 
The preceding lemma shows that  every non-zero corepresentation is either
irreducible or the direct sum of two non-zero corepresentations, and we can apply Zorn's lemma.
\end{proof}

% Trivial rep should maybe be introduced in a more conspicuous place

% Semi-simplicity requires that each element is a finite direct sum! But possibly this is convention

We can now prove that the category $\Corep(\mathscr{A})$ of a partial Hopf algebra with invariant integral is semisimple, that is, any object is a finite direct sum of irreducible objects. If one allows a more relaxed definition of semisimplicity allowing infinite direct sums, this will be true also for the a potentially bigger category $\Corep_{\rcf}(\mathscr{A})$.

We will first state a lemma which will also be convenient at other occasions.

\begin{Lem}\label{LemInjMor}  Let $\mathscr{A}$ be a partial Hopf algebra with an invariant integral, and fix $\alpha,\beta$ in the hyperobject set.  Then if $T$ is a morphism in $\Corep(\mathscr{A})_{\alpha\beta}$ and $\sum_{k\in I_\alpha} \Gru{T}{k}{l}=0$ for some $l \in I_\beta$, then $T=0$.
\end{Lem} 

\begin{proof} This follows from the equations in Remark \ref{remark:rep-total-morphism}
\end{proof}

\begin{Prop}\label{prop:rep-cosemisimple} Let $\mathscr{A}$ be a partial Hopf algebra with an invariant integral.   Then the components of the partial tensor category $\Corep(\mathscr{A})$ are semisimple.
\end{Prop}
\begin{proof} 

%\texttt{Put that into the first or second subsection}
%Let us now first show that the trivial representation decomposes into irreducibles. Let $I$ be the object set of $\mathscr{A}$, and say $k\sim l$ if $\UnitC{k}{l}\neq 0$. Then $\sim$ is an equivalence relation: as \[\Delta_{ll}(\UnitC{k}{m}) = \UnitC{k}{l}\otimes \UnitC{l}{m},\] the relation $\sim$ is transitive. As $S(\UnitC{k}{l}) = \UnitC{l}{k}$, we have that $\sim$ is symmetric. And as $\varepsilon(\UnitC{k}{k})=1$, we also have that $\sim$ is reflexive. 

%Let then $I = \sqcup_{\alpha\in \mathscr{I}} I_{\alpha}$ be a labeled partition associated to $\sim$. Define $\C_{I_{\alpha}}\subseteq \C_I$ as the linear span of the homogeneous components with index in $\alpha$. It is clear then that the $\C_{I_{\alpha}}$ are invariant and irreducible.

%Consider now a general corepresentation $(X,\Hsp)$. Let $\Grd{\Hsp}{\alpha}{\beta}$ be the closed linear span of the homogeneous components with index in $\alpha\times \beta$. As we can identify \[\Grd{\Hsp}{\alpha}{\beta} \cong \C_{I_{\alpha}}\,\Circt\, \Hsp\,\Circt\, \C_{I_{\beta}},\] we see that $\Grd{\Hsp}{\alpha}{\beta}$ is an invariant subspace of $\Hsp$. Hence we may as well suppose that $\Hsp = \Grd{\Hsp}{\alpha}{\beta}$. 

Let $V$ be in any object of $\Corep(\mathscr{A})_{\alpha\beta}$ for $\alpha,\beta\in \mathscr{I}$.  From the previous lemma, we see that for $T$ a morphism in $\Corep(\mathscr{A})_{\alpha\beta}$, the map $T\mapsto \sum_{k\in I_\alpha} \Gru{T}{k}{l}$ is injective for any choice of $l\in I_\beta$. It follows by column-finiteness of $V$ that the algebra of self-intertwiners of $V$ is finite-dimensional. We then immediately conclude from the previous corollary that $V$ is a finite direct sum of irreducible invariant subspaces.
%But let then $T$ be a bounded self-intertwiner of $\Hsp$.
\end{proof} 


%Another corollary is the following. % Needed? Or can this better be proven directly in our case later on?

% Reference to be added!
%\begin{Cor} \label{cor:rep-irreducible-bidual}
 % Let $\mathscr{A}$ be a partial Hopf algebra with an invariant integral. Then
 % every irreducible rcf corepresentation $(V,\mathscr{X})$ of
 % $\mathscr{A}$ is equivalent to its right bidual
 % $(V,\dual{\dual{\mathscr{X}}{}\!})$.
%\end{Cor}
%\begin{proof} In any tensor category, a right dual of an object $X$ has $X$ as its left dual. The corollary then follows as in every semi-simple tensor
%  category, left and right duals are isomorphic \cite{}. 
%\end{proof}

\subsection{Matrix coefficients of irreducible corepresentations}

Our next goal is to obtain the analogue of Schur's orthogonality
relations for matrix coefficients of corepresentations.

Given finite-dimensional vector spaces $V$ and $W$, the dual space of
$\Hom_{\C}(V,W)$ is linearly spanned by functionals of the form
\begin{align*}
  \omega_{f,v} \colon \Hom_{\C}(V,W) \to \C, \quad T \mapsto  (f|Tv),
\end{align*}
where $v\in V$, $f\in \dual{W}$, and $(-|-)$ denotes the natural
pairing of $\dual{W}$ with $W$.
\begin{Def} Let $\mathscr{A}$ be a partial bialgebra. The space of
  \emph{matrix coefficients} $\mathcal{C}(\mathscr{X})$ of an rcfd
  corepresentation $(V,\mathscr{X})$ is the sum of the subspaces
\begin{align*}
  \Gr{\mathcal{C}(\mathscr{X})}{k}{l}{m}{n} &= \Span \left\{ (\id \otimes
    \omega_{f,v})(\Gr{X}{k}{l}{m}{n}) \mid v\in \Gru{V}{m}{n}, f \in
    \dual{(\Gru{V}{k}{l})} \right\} \subseteq \Gr{A}{k}{l}{m}{n}.
\end{align*}
\end{Def}
Let $(V,\mathscr{X})$ be  an rcfd corepresentation of a partial bialgebra
$\mathscr{A}$.  Condition \eqref{eq:rep-comultiplication} in Definition \ref{definition:corep}
implies
\begin{align} \label{eq:rep-matrix-delta}
  \Delta_{pq}(\Gr{\mathcal{C}(\mathscr{X})}{k}{l}{m}{n}) \subseteq
  \Gr{\mathcal{C}(\mathscr{X})}{k}{l}{p}{q} \otimes
  \Gr{\mathcal{C}(\mathscr{X})}{p}{q}{m}{n}.
\end{align}
Thus, the $\Gr{\mathcal{C}(\mathscr{X})}{k}{l}{m}{n}$ form a partial
coalgebra with respect to $\Delta$ and $\epsilon$.  Moreover, for each
$k,l$, the $I^{2}$-graded vector  space
\begin{align*}
  \Grd{\mathcal{C}(\mathscr{X})}{k}{l}:=\bigoplus_{m,n }
  \Gr{\mathcal{C}(\mathscr{X})}{k}{l}{m}{n}
\end{align*}
is rcfd, and the inclusion above shows that one can
form the regular corepresentation on this space.
\begin{Lem} \label{lemma:rep-regular-embedding}
  Let $(V,\mathscr{X})$ be an rcfd corepresentation
  of a partial bialgebra and let $f\in
  \dual{(\Gru{V}{k}{l})}$. Then the family of maps
  \begin{align*}
    \Gr{T}{}{}{m}{n(f)} \colon \Gru{V}{m}{n} \to
    \Gr{\mathcal{C}(\mathscr{X})}{k}{l}{m}{n}, \ w \mapsto (\id
    \otimes \omega_{f,w})(\Gr{X}{k}{l}{m}{n})=(\id \otimes
    f)(\Gr{X}{k}{l}{m}{n}(1 \otimes w)),
  \end{align*}
  is a morphism from $\mathscr{X}$ to the regular corepresentation on
  $\Grd{\mathcal{C}(\mathscr{X})}{k}{l}$.
\end{Lem}
\begin{proof}
  Denote by $\mathscr{Y}$ the regular corepresentation on
  $\bigoplus_{m,n } \Gr{\mathcal{C}(\mathscr{X})}{k}{l}{m}{n}$. Then
  \begin{align*}
    \label{eq:1}
 \Gr{Y}{p}{q}{m}{n}    (1\otimes \Gr{T}{}{}{m}{n(f)}(v)) &= 
(\Delta^{\op}_{pq} \otimes \omega_{f,v})( \Gr{X}{k}{l}{m}{n}) 
\\ & = (\id \otimes \id \otimes
f)((\Gr{X}{k}{l}{p}{q})_{23}(\Gr{X}{p}{q}{m}{n})_{13}(1 \otimes 1
 \otimes v)) \\ &=(1 \otimes \Gr{T}{}{}{p}{q(f)})\Gr{X}{p}{q}{m}{n}(1 \otimes v)
  \end{align*}
for all $v \in \Gru{V}{m}{n}$.
\end{proof}
\begin{Prop} \label{prop:rep-weak-pw} Let $\mathscr{A}$ be a partial
  Hopf algebra with an invariant integral. Then the total algebra $A$ is the sum
  of the matrix coefficients of irreducible rcfd corepresentations.
\end{Prop}
\begin{proof} 
  Let $a \in \Gr{A}{k}{l}{m}{n}$. Write
  \begin{align*}
    \Delta_{pq}(a)=\sum_{i} b_{pq}^{i} \otimes c^{i}_{pq}
  \end{align*}
 with linearly independent
  $(c_{pq}^{i})_{i}$. Then the family of subspaces
  \begin{align*}
    \Gru{V}{p}{q} = \mathrm{span}\{b_{pq}^{i} : i \}\subseteq \bigoplus_{k,l}
  \Gr{A}{k}{l}{p}{q}
  \end{align*}
is rcfd, and the relation
  \begin{align*}
 \sum_{i}
    \Delta_{rs}(b^{i}_{pq}) \otimes c^{i}_{pq} =
    (\Delta_{rs} \otimes \id)\Delta_{pq}(a) = (\id \otimes
    \Delta_{pq}) \Delta_{rs}(a) = \sum_{j} b^{j}_{rs} \otimes
    \Delta_{pq}(c^{j}_{rs})
  \end{align*}
  implies $\Delta_{rs}(\Gru{V}{p}{q}) \subseteq \Gru{V}{r}{s} \otimes
  \Gr{A}{r}{s}{p}{q}$.  We can therefore form the regular
  corepresentation $\mathscr{X}$ on $V$ as in Example \ref{example:rep-regular}, and
  \begin{align*}
    a = (\id \otimes \epsilon)(\Delta^{\op}_{kl}(a)) =
    (\id \otimes \epsilon)(\Gr{X}{k}{l}{m}{n}(1 \otimes a)) \in
    \Gr{\mathcal{C}(\mathscr{X})}{k}{l}{m}{n}.
  \end{align*}
  Decomposing $(V,\mathscr{X})$, we find that
  $a$ is contained in the sum of matrix coefficients of irreducible
rcfd  corepresentations.
\end{proof}


The first part of the orthogonality relations concerns matrix
coefficients of inequivalent irreducible corepresentations. 
\begin{Prop} \label{prop:rep-orthogonality-1} Let $\mathcal{A}$ be a
  partial Hopf algebra with an invariant integral $\phi$ and inequivalent
  irreducible rcfd corepresentations $(V,\mathscr{X})$ and
  $(W,\mathscr{Y})$.  Then  for all
  $a\in \mathcal{C}(X), b \in \mathcal{C}(Y)$,
  \[\phi(S(b)a) = \phi(bS(a))=0.\]
\end{Prop}
\begin{proof}
Since $\phi$ vanishes on $S(\Gr{A}{k}{l}{m}{n})\Gr{A}{p}{q}{r}{s}$ and
on $\Gr{A}{p}{q}{r}{s}S(\Gr{A}{k}{l}{m}{n})$ unless
$(p,q,r,s) = (m,n,k,l)$, it suffices to prove the assertion for  elements of the form
\begin{align*}
  a&=(\id \otimes \omega_{f,v})(\Gr{X}{k}{l}{m}{n})  && \text{and} &
  b&=(\id \otimes \omega_{g,w})(\Gr{Y}{m}{n}{k}{l})
\end{align*}
where $f\in \dual{(\Gru{V}{k}{l})}, v \in \Gru{V}{m}{n}$ and $g \in
\dual{(\Gru{W}{m}{n})}, w \in \Gru{W}{k}{l}$.  Lemma
\ref{lem:rep-average}, applied to the family
  \begin{align*}
    \Gru{T}{p}{q} \colon \Gru{V}{p}{q} \to \Gru{W}{p}{q}, \quad u
    \mapsto  \delta_{p,k}\delta_{q,l}  f(u)w,
  \end{align*}
  yields morphisms $\Grd{\check{T}}{k}{l},\Grd{\hat{T}}{k}{l}$ from $(V,\mathscr{X})$ to
  $(W,\mathscr{Y})$ which necessarily are $0$. Inserting the
  definition of $\Grd{\check{T}}{k}{l}$, we find
  \begin{align*}
    \phi(S(b)a) &= \phi\big((S \otimes
    \omega_{g,w})(\Gr{Y}{m}{n}{k}{l}) \cdot (\id \otimes
    \omega_{f,v})(\Gr{X}{k}{l}{m}{n})\big) \\ &= (\phi \otimes \omega_{g,v})\left(\Gr{(Y^{-1})}{l}{k}{n}{m}(1 \otimes
      \Gru{T}{k}{l} )     \Gr{X}{k}{l}{m}{n}\right) 
    = \omega_{g,v}( \Gr{\check{T}}{k}{l}{m}{n}) = 0.
  \end{align*}% Resort notation on using leg notation, physics bra-ket, etc.
  
  A similar calculation involving $\hat{T}$ shows that
  $\phi(bS(a))=0$.  
\end{proof}

From now on, we will assume that our partial Hopf algebra is regular (i.e.~ has bijective antipode). %Remark that this is in fact automatic in presence of invariant functional.

\begin{Theorem} \label{thm:rep-orthogonality} Let $\mathcal{A}$ be a
  regular partial Hopf algebra with an invariant integral $\phi$. Let $\alpha,\beta\in \mathscr{I}$, and let $(V,\mathscr{X})$
  be an irreducible rcfd corepresentation of $\mathscr{A}$ inside $\Corep(\mathscr{A})_{\alpha\beta}$. Suppose
  $F_{\mathscr{X}}=F$ is an isomorphism from $(V,\mathscr{X})$ to
  $(V,\hat{\hat{\mathscr{X}}})$ with inverse
  $G_{\mathscr{X}}= G$. Then the following hold.
  \begin{enumerate}[label=(\arabic*)]
  \item The numbers $d_G:=\sum_{k} \Tr (\Gru{G}{k}{l})$ and $d_F:=\sum_{n} \Tr (\Gru{F}{m}{n})$ do not depend on the choice of $l \in I_\beta$ or $m\in I_\alpha$.
    \item  For all $k,m \in I_\alpha$ and $l,n\in I_\beta$,
    \begin{align*}
      (\phi \otimes \id)(\Gr{(X^{-1})}{l}{k}{n}{m}\Gr{X}{k}{l}{m}{n})
      &=d_G^{-1}\Tr(\Gru{G}{k}{l})
      \id_{\Gru{V}{m}{n}}, \\
      (\phi \otimes \id)(\Gr{X}{k}{l}{m}{n}\Gr{(X^{-1})}{l}{k}{n}{m})
      &=d_F^{-1}\Tr(\Gru{F}{m}{n})
      \id_{\Gru{V}{k}{l}}.
    \end{align*}
  \item Denote by $\Sigma_{klmn}$ the flip map $\Gru{V}{k}{l}
    \otimes \Gru{V}{m}{n} \to \Gru{V}{m}{n}
    \otimes \Gru{V}{k}{l}$. Then
 \begin{align*}
   (\phi \otimes \id \otimes
   \id)((\Gr{(X^{-1})}{l}{k}{n}{m})_{12}(\Gr{X}{k}{l}{m}{n})_{13}) &=
   d_G^{-1}
   (\id_{\Gru{V}{m}{n}} \otimes \Gru{G}{k}{l})
   \circ \Sigma_{klmn}, \\
   (\phi \otimes \id \otimes
   \id)((\Gr{X}{k}{l}{m}{n})_{13}(\Gr{(X^{-1})}{l}{k}{n}{m})_{12}) &= d_F^{-1} (\Gru{F}{m}{n}
   \otimes \id_{\Gru{V}{k}{l}}) \circ \Sigma_{klmn}.
 \end{align*}
\end{enumerate}
  \end{Theorem}
\begin{proof}
  We prove the assertions and equations involving $d_G$ in (1), (2)
  and (3)  simultaneously; the assertions involving $d_F$  follow similarly.

  %As above, we denote by $\Sigma_{p,q,r,s}$ the flip
  %$\Gru{\mathcal{H}}{p}{q} \otimes \Gru{\mathcal{H}}{r}{s} \to
  %\Gru{\mathcal{H}}{r}{s} \otimes \Gru{\mathcal{H}}{p}{q}$.  
  Consider
  the following endomorphism $F_{m,n,k,l}$ of $\Gru{V}{m}{n}\otimes \Gru{V}{k}{l}$, 
  \begin{align*}
    F_{m,n,k,l}
    &:=(\phi \otimes \id \otimes \id)\left((\Gr{(X^{-1})}{l}{k}{n}{m})_{12}(\Gr{X}{k}{l}{m}{n})_{13}\right)
    \circ \Sigma_{mnkl} \\ &= (\phi \otimes \id \otimes
    \id)\left((\Gr{(X^{-1})}{m}{n}{k}{l})_{12}
      \Sigma_{klkl,23}(\Gr{X}{k}{l}{m}{n})_{12}\right).
  \end{align*}
  By applying Lemma \ref{lem:rep-average} with respect to the flip map $\Sigma_{klkl}$, we see that the family $(F_{m,n,k,l})_{m,n}$ is
  an endomorphism of $(V \otimes \Gru{V}{k}{l}, X\otimes \id)$ and hence
  \begin{align}
    F_{m,n,k,l} &= \id_{\Gru{V}{m}{n}} \otimes \Gru{R}{k}{l} \label{eq:rep-orthogonal-1}
  \end{align}
  with some $\Gru{R}{k}{l} \in \Hom_{\C}(\Gru{V}{k}{l})$ not
  depending on $m,n$. % In using irreducibility, do we not miss any subtlety in allowing non-bounded morphisms?
  On the other hand, since $\phi = \phi S$,
  \begin{align*}
    F_{m,n,k,l} &= (\phi \otimes \id \otimes \id)((S \otimes
    \id)(\Gr{X}{m}{n}{k}{l})_{12}(\Gr{X}{k}{l}{m}{n})_{13})
    \circ \Sigma_{mnkl} \\
    &= (\phi \otimes \id \otimes \id)\left(((S \otimes
      \id)(\Gr{X}{k}{l}{m}{n}))_{13}
      ((S^{2} \otimes \id)(\Gr{X}{m}{n}{k}{l}))_{12}\right)     \circ \Sigma_{mnkl}\\
    &= (\phi \otimes \id \otimes
    \id)\left((\Gr{(X^{-1})}{k}{l}{m}{n})_{13} (\Sigma_{mnmn})_{23}
      (\Gr{(\dual{\dual{X}{}\!})}{m}{n}{k}{l})_{13}\right).
  \end{align*}
  Hence we can again apply Lemma \ref{lem:rep-average} and
  find that the family $(F_{m,n,k,l})_{k,l}$ is a morphism \[(F_{m,n,k,l})_{k,l}:
  (\Gru{V}{m}{n} \otimes V, \hat{\hat{X}}_{13})\rightarrow (\Gru{V}{m}{n} \otimes V,
 X_{13}).\] Therefore,
  \begin{align}
    F_{m,n,k,l} &= \Gru{T}{m}{n} \otimes \Gr{G}{k}{l}{}{\mathscr{X}} \label{eq:rep-orthogonal-2}
  \end{align}
  with some $\Gru{T}{m}{n} \in \mathcal{\Hom_{\C}}(\Gru{V}{m}{n})$
  not depending on $k,l$. Combining \eqref{eq:rep-orthogonal-1} and
  \eqref{eq:rep-orthogonal-2}, we conclude that, for some $\lambda\in \C$, \[F_{m,n,k,l} = \lambda
  (\id_{\Gru{V}{m}{n}} \otimes \Gr{G}{k}{l}{}{\mathscr{X}})\]
  
  Choose dual  bases
  $(v_{i})_{i}$ for $\Gru{V}{k}{l}$ and $(f_{i})_{i}$ for  $\dual{(\Gru{V}{k}{l})}$. Then
  \begin{align*}
    \lambda   \Tr( \Gr{G}{k}{l}{}{\mathscr{X}}) \id_{\Gru{V}{m}{n}}
 &= \sum_{i} (\id \otimes
    \omega_{f_{i},v_{i}})(F_{m,n,k,l}) = (\phi \otimes
    \id)((\Gr{(X^{-1})}{l}{k}{n}{m}) \Gr{X}{k}{l}{m}{n}).
  \end{align*}
  Take now $n=l$.  By Lemma \ref{LemInjMor}, we can choose $m\in I_{\alpha}$ with $\Gru{V}{m}{n}\neq 0$.   Then summing the previous relation over $k$, the relations $\sum_{k}
  (\Gr{(X^{-1})}{l}{k}{n}{m}) \Gr{X}{k}{l}{m}{n} = \UnitC{l}{n}
  \otimes \id_{\Gru{V}{m}{n}}$ and
  $\phi(\UnitC{l}{l})=1$ give
\begin{align*}
\lambda \cdot  \sum_{k} \Tr(\Gr{G}{k}{l}{}{\mathscr{X}}) = 1.  %Problem if $\Gru{V}{m}{n}=0$!
\end{align*}
Now all assertions in (1)--(3) concerning $d_G$ follow.
 % the second formula, we use the first formula for the
 %    opposite of $(A,\Delta)$. For this opposite, $\phi$ still is a
 %    faithful, positive, normalized invariant functional and
 %    $(\mathcal{H},X)$ still is a unitary irreducible locally finite
 %    corepresentation, but the antipode $S$ gets replaced by $S^{-1}$
 %    and therefore $F_{X}$ gets replaced by $F_{X}^{-1}$.
\end{proof}

\begin{Rem} For semi-simple tensor categories with duals, it is known that any object is isomorphic to its left bidual, hence there always exists an $F_{\mathscr{X}}$ as in the previous Theorem. In fact, from the faithfulness of $\phi$ and Proposition \ref{prop:rep-orthogonality-1}, it follows that not all $F_{m,n,k,l}$ in the previous proof are zero. Hence $G_{\mathscr{X}}$ is a non-zero morphism and thus an isomorphism from the left bidual of $\mathscr{X}$ to $\mathscr{X}$.  
\end{Rem}

\begin{Cor}\label{CorOrth}
  Let $\mathscr{A}$ be a regular partial Hopf algebra with an invariant integral $\phi$, let
  $(V,\mathscr{X})$ be an irreducible rcfd corepresentation of
  $\mathscr{A}$, let $F_{\mathscr{X}}$ be an isomorphism from
  $(V,\mathscr{X})$ to $(V,\dualco{\dualco{\mathscr{X}}})$ and
  $G_{\mathscr{X}}=F^{-1}_{{\mathscr{X}}}$, and let $a=(\id \otimes
  \omega_{f,v})(\Gr{X}{k}{l}{m}{n})$ and $b=(\id \otimes
  \omega_{g,w})(\Gr{X}{m}{n}{k}{l})$, where 
  $f \in   \dual{(\Gru{V}{k}{l})}$, $v \in\Gru{V}{m}{n}$, $g \in
  \dual{(\Gru{V}{m}{n})}$, $w \in  \Gru{V}{k}{l}$.  Then
\begin{align*}
  \phi(S(b)a) &= \frac{(g|v)(f|G_{\mathscr{X}}w)}{\sum_{r}
    \Tr(\Gr{G}{r}{n}{}{\mathscr{X}})}, & \phi(aS(b)) = \frac{(g|F_{\mathscr{X}}v)(f|w)}{\sum_{s}
    \Tr(\Gr{F}{m}{s}{}{\mathscr{X}})}.
\end{align*}
\end{Cor}
\begin{proof}
Apply $\omega_{g,w} \otimes
    \omega_{f,v}$ to the formulas in  Theorem
    \ref{thm:rep-orthogonality}.(c).
\end{proof}
\begin{Cor} \label{cor:rep-pw}
  Let $\mathscr{A}$ be a partial Hopf algebra with an invariant integral and let
  $((V_{i},\mathscr{X}_{i}))_{i \in \mathcal{I}}$ be a maximal family of mutually non-isomorphic irreducible rcfd corepresentations of
  $\mathscr{A}$. Then the map
  \begin{align*}
    \bigoplus_{i \in \mathcal{I}} \bigoplus_{k,l,m,n}
    (\dual{(\Gr{V}{k}{l}{}{i})} \otimes
    \Gr{V}{m}{n}{}{i}) \to A
  \end{align*}
  that sends $f \otimes w \in
  \dual{(\Gr{V}{k}{l}{}{\alpha})} \otimes
  \Gr{V}{m}{n}{}{i}$ to $ (\id \otimes
  \omega_{f,w})(\Gr{(X_{i})}{k}{l}{m}{n})$,
  is a linear isomorphism. 
\end{Cor}
\begin{proof} This follows from Proposition \ref{prop:rep-weak-pw}, Proposition \ref{prop:rep-orthogonality-1} and Corollary \ref{CorOrth}.
\end{proof}
\begin{Cor} \label{cor:rep-pw-morphisms}
  Let $\mathscr{A}$ be a regular partial Hopf algebra with an invariant integral, let
  $((V_{i},\mathscr{X}_{i}))_{i\in \mathcal{I}}$ be a maximal
  family of mutually non-isomorphic irreducible rcfd corepresentations of $\mathscr{A}$,
  fix $i \in \mathcal{I}$ and $k,l\in I$, and denote by $\Gr{\mathscr{Y}}{k}{l}{}{i}$
  the regular corepresentation on
  $\Grd{\mathcal{C}(\mathscr{X}_i)}{k}{l}$. Then there exists a
  linear isomorphism
  \begin{align*}
    \dual{( \Gru{V}{k}{l})} \to
    \Mor((V_{i},\mathscr{X}_{i}),
    (\Grd{\mathcal{C}(\mathscr{X}_i)}{k}{l},\Gr{\mathscr{Y}}{k}{l}{}{i}))
  \end{align*}
  assigning to each $f\in     \dual{( \Gru{V}{k}{l})}$ the morphism
  $T_{(f)}$ of Lemma \ref{lemma:rep-regular-embedding}.
\end{Cor}



\subsection{Unitary corepresentations of partial compact quantum groups}


Let us now enhance our partial Hopf algebras to partial compact
quantum groups. We write $B(\Hsp,\mathcal{G})$ for the linear space of
bounded morphisms between Hilbert spaces $\Hsp$ and $\mathcal{G}$. 
%To be consistent in terminology, we mean by representation of a partial compact quantum group a corepresentation of its associated partial Hopf $^*$-algebra.


%One then considers corepresentations on 

%rcf bigraded
%\emph{Hilbert spaces} such that the inverse of the corepresentation
%coincides with its adjoint. More precisely, we have the following
%definition. 
%Unitary corepresentations act on $I^{2}$-graded Hilbert spaces $\Hsp =
%\bigoplus_{k,l} \Gru{\Hsp}{k}{l}$ which are row- and column-finite,
%where the sum is a direct sum of Hilbert spaces. Associated to each
%such $I^{2}$-graded Hilbert space is the $I^{2}$-graded vector space
%which is obtained by taking the algebraic direct sum of the
%components. 

\begin{Def} Let $\mathscr{A}$ define a partial compact quantum
  group. We call an rcfd corepresentation $\mathscr{X}$ of $\mathscr{A}$ on a collection of Hilbert spaces $\Gru{\Hsp}{k}{l}$ 
   \emph{unitary}
  if \[\Gr{(X^{-1})}{n}{m}{l}{k}=(\Gr{X}{l}{k}{n}{m})^{*}\quad
  \textrm{in }\Gr{A}{k}{l}{m}{n}\otimes
  B(\Gru{\Hsp}{l}{k},\Gru{\Hsp}{n}{m}).\]
\end{Def} 


\begin{Rem} %\begin{enumerate} \item 
The total object $\Hsp$ will then only be a pre-Hilbert space, but as the local components are finite-dimensional, this will not be an issue.
%In the Hilbert space setting, it is more natural to let the total object $\Hsp$ be the \emph{closed} (instead of the purely algebraic) direct sum of all (finite-dimensional) $\Gru{\Hsp}{k}{l}$. This does not change the notion of corepresentation, which had a local definition.
%\item Concerning morphisms, we will say a collection of $\Gru{T}{k}{l}$ defines a \emph{bounded} intertwiner or morphism if the total operator $T= \oplus \Gru{T}{k}{l}$ is bounded. We will denote by $\Corep_{u}(\mathscr{A})$ the category of unitary rcf corepresentations with arbitrary morphisms, and $\Corep_{u}^{\infty}(\mathscr{A})$ for the category with bounded morphisms.
% Give a more prominent place, or check later if it is actually worthwhile to make this distinction.
%\end{enumerate}
\end{Rem}

\begin{Exa}\label{example:rep-trivial-unitary}
  Regard $\C^{(I)}$ as a direct sum of the trivial Hilbert spaces $\C$. Then the
  trivial corepresentation $\mathscr{U}$ on $\C^{(I)}$ is unitary.
\end{Exa}

The tensor product of rcfd corepresentations lifts to a tensor product
of unitary corepresentations as follows.  We define the tensor product
of rcfd $I^{2}$-graded Hilbert spaces similarly as for rcfd
$I^{2}$-graded vector spaces and pretend it to be strict again.
\begin{Lem}\label{lemma:rep-unitary-tensor}
  Let $(\Hsp,\mathscr{X})$ and $(\mathcal{G},\mathscr{Y})$ be unitary
  rcfd representations associated to a partial compact quantum group. Then the
  tensor product $(\Hsp \itimes \mathcal{G},\mathscr{X} \Circt
  \mathscr{Y})$ is unitary again.
\end{Lem}
\begin{proof}
In total form,  $(X\Circt Y)^{-1} = Y_{13}^{-1}X_{12}^{-1}
  =Y_{13}^{*}X_{12}^{*} = (X \Circt Y)^{*}$ by Remark \ref{remark:rep-tensor-multiplier}.
\end{proof}

We hence obtain a tensor C$^*$-category $\Corep_{u,\rcf}(\mathscr{A})$ of unitary rcfd corepresentations. We denote again by $\Corep_u(\mathscr{A})$ the subcategory of all corepresentations with finite support on the hyperobject set. It is the total tensor C$^*$-category with local units of a semisimple partial tensor C$^*$-category.

Our aim now is to show that every (irreducible) rcfd corepresentation is
equivalent to a unitary one. We show this by embedding the
corepresentation into a restriction of the regular corepresentation.
\begin{Lem} \label{lemma:rep-regular-unitary}
  Let $\mathscr{A}$ define a partial compact quantum group with
positive invariant  integral $\phi$, and let $\Gru{V}{m}{n} \subseteq
\bigoplus_{k,l} \Gr{A}{k}{l}{m}{n}$ be subspaces such that
$\Delta_{pq}(\Gru{V}{m}{n}) \subseteq \Gru{V}{p}{q} \otimes
    \Gr{A}{p}{q}{m}{n}$ and $V=\bigoplus_{k,l} \Gru{V}{k}{l}$ is rcfd. Then each $\Gru{V}{k}{l}$ is a Hilbert space with
    respect to the inner product given by $\langle
    a|b\rangle:=\phi(a^{*}b)$, and the regular corepresentation
    $\mathscr{X}$ on $V$ is unitary.
\end{Lem}
\begin{proof} 
By Lemma \ref{lemma:rep-invertible},  it suffices to show that
  \begin{equation}\label{EqUnit} \sum_{k}
    (\Gr{X}{k}{l}{m}{n'})^* \Gr{X}{k}{l}{m}{n} =
    \delta_{n,n'}\UnitC{l}{n}\otimes
    \id_{\Gru{\Hsp}{m}{n}}.
  \end{equation} 
Let  $a\in \Gru{\Hsp}{m}{n}$, $b\in \Gru{\Hsp}{m}{n'}$ and define $\omega_{b,a} \colon
\Hom_{\C}(\Gru{\Hsp}{m}{n},\Gru{\Hsp}{m}{n}) \to \C$ by $T
\mapsto \langle b|Ta\rangle$. Then
\begin{eqnarray*}
\sum_{k }(\id \otimes \omega_{b,a})
((\Gr{X}{k}{l}{m}{n'})^* \Gr{X}{k}{l}{m}{n}))  &=& \sum_k
(\id\otimes \phi)(\Delta_{kl}^{\op}(b)^*\Delta_{kl}^{\op}(a))\\
  &=& \sum_k (\phi\otimes
  \id)(\Delta_{lk}(b^*)\Delta_{kl}(a)) \\ &=& (\phi\otimes
  \id)(\Delta_{ll}(b^*a)) \\ &=& \phi(b^*a)\UnitC{l}{n} \\&=&
  \delta_{n',n} \UnitC{l}{n} \otimes \langle b|a\rangle.
\end{eqnarray*} 
This proves \eqref{EqUnit}.
\end{proof} 

\begin{Prop} \label{prop:rep-unitarisable} Every  rcfd
  corepresentation of a partial compact quantum group $\mathscr{A}$ is
  isomorphic to a unitary rcfd corepresentation.
\end{Prop}
\begin{proof}
  By Proposition \ref{prop:rep-cosemisimple}, it suffices to prove the
  assertion for every rcfd corepresentation $(V,\mathscr{X})$ that is
  irreducible.  For some $k,l$ and $f \in
  \dual{(\Gru{V}{k}{l})}$, the operator $T_{(f)}$ defined in
  Lemma \ref{lemma:rep-regular-embedding} has to be non-zero and
  hence, by Schur's Lemma, injective. Thus, it forms an equivalence
  between $(V,\mathscr{X})$ and a restriction of the regular
  corepresentation on $\Grd{\mathcal{C}(\mathscr{X})}{k}{l}$, which is
  unitary by Lemma \ref{lemma:rep-regular-unitary}.
\end{proof}
%This result and Proposition \ref{prop:rep-cosemisimple}
%imply that the category $\Corep_{u}(\mathscr{A})$ is semisimple:
\begin{Cor} The partial C$^*$-tensor category $\Corep_u(\mathscr{A})$ is a partial fusion category.
\end{Cor}

\begin{Rem}If $\mathscr{A}$ defines a partial compact quantum group $\mathscr{G}$, we will also write $\Corep_u(\mathscr{A})= \Rep_u(\mathscr{G})$, and talk of (unitary) rcfd representations of $\mathscr{G}$.
 \end{Rem}  
   

%With the evident definition on morphisms, we obtain a tensor product
%on $\Corep_{u}(\mathscr{A})$.  This tensor
%category is \emph{rigid} in the sense that every object has a
%left and a right dual:
%\begin{Cor}
 % Let $\mathscr{A}$ define a partial compact quantum group. Then the
 % category $\Corep_{u}(\mathscr{A})$ is a rigid tensor category.
%\end{Cor}
%\begin{proof}
  %Since the antipode of $\mathscr{A}$ is invertible (Corollary
  %\ref{cor:involutive}), every unitary rcf corepresentation
  %$(\Hsp,\mathscr{X})$ has a left and a right dual in
  %$\Corep(\mathscr{A})$ (Corollary \ref{cor:rep-tensor-duality}),
  %and these are equivalent isomorphic to unitary rcf corepresentations
  %(Proposition \ref{prop:rep-unitarisable}).
%\end{proof}
%Note that in $\Corep(\mathscr{A})$, the left dual of a unitary rcfd
%corepresentation $(\Hsp,\mathscr{X})$ is given by the $I^{2}$-graded
%vector space
%$\dual{\Hsp}$  and the corepresentation multiplier
%\begin{align} \label{eq:rep-unitary-right-dual}
 % (S\otimes -^{\vee})(X) = (\id \otimes -^{\vee})(X^{-1}) )=
  %(-^{*}\otimes -^{*\vee})(X).
%\end{align}
%By Corollary \ref{cor:rep-irreducible-bidual},
%$(\mathcal{H},\mathscr{X})$ is isomorphic to the right bidual, which
%is given by the $\Hsp$ and $(S^{2} \otimes \id)(X)$. This isomorphism can be
%chosen to be positive, as the next proposition shows.

Let now $\mathscr{X}$ be a unitary corepresentation of $\mathscr{A}$. Then there exists an isomorphism from $\mathscr{X}$ to $\dualco{\dualco{\mathscr{X}}} = (S^2\otimes \id)\mathscr{X}$. The following proposition shows that it can be implemented by positive operators.

\begin{Prop} \label{prop:rep-unitary-bidual}
  Let $\mathscr{A}$ define a partial compact quantum group and let
  $(\Hsp,\mathscr{X})$ be an irreducible unitary rcfd corepresentation of
  $\mathscr{A}$.  Then there exists an isomorphism $F=F_{\mathscr{X}}$
  from $(\Hsp,\mathscr{X})$ to 
  $(\Hsp,(S^{2} \otimes \id)(\mathscr{X}))$ in $\Corep(\mathscr{A})$ such
  that each $\Gru{F}{k}{l}$ is positive.
\end{Prop}
\begin{proof}
 By Proposition \ref{prop:rep-unitarisable}, there exists an
  isomorphism $T \colon \dualco{\mathscr{X}} \to \mathscr{Y}$ for some
  unitary rcfd corepresentation $\mathscr{Y}$ on $\dual{\Hsp}$, so that in total form,
  $(1\otimes T)\dualco{X} = Y(1 \otimes T)$.
We  apply   $S \otimes -^{\tr}$ and $-^{*} \otimes -^{*\tr}$,
respectively to find 
\begin{align*}
 \dualco{\dualco{X}}(1 \otimes \dualop{T}) &= (1 \otimes
  \dualop{T})\dualco{Y}, & (1 \otimes T^{*\tr})X=\dualco{Y}(1\otimes T^{*\tr}).
\end{align*}
%Here, we identify the the dual of a Hilbert space with its conjugate
%Hilbert space to make sense of $T^{*\tr}$.  
Combining both equations, we
find $\dualco{\dualco{X}}(1 \otimes \dualop{T}T^{*\tr})=(1 \otimes
\dualop{T}T^{*\tr})X$. Thus, we can take
$F_{\mathscr{X}}:=\dualop{T}T^{*\tr}$.
\end{proof}

The Schur orthogonality relations in Corollary \ref{CorOrth} can be
rewritten using the involution instead of the antipode as follows.
Let $(\Hsp,\mathscr{X})$ be a unitary rcfd corepresentation of
$\mathscr{A}$. Since $(S\otimes \id)(X)=X^{-1}=X^{*}$, the space of
matrix coefficients $\mathcal{C}(\mathscr{X})$ satisfies
\begin{align} \label{eq:rep-unitary-matrix-coefficients}
  S(\Gr{\mathcal{C}(\mathscr{X})}{k}{l}{m}{n}) &=
  (\Gr{\mathcal{C}(\mathscr{X})}{m}{n}{k}{l})^{*} \subseteq \Gr{A}{n}{m}{l}{k}.
\end{align}
More precisely, let $v \in \Gru{\Hsp}{k}{l}$, $v' \in \Gru{\Hsp}{m}{n}$
and denote by $\omega_{v,v'}$ the functional
given by $T \mapsto \langle v|Tv'\rangle$. Then
\begin{align*}
  S((\id \otimes \omega_{v,v'})(\Gr{X}{k}{l}{m}{n})) &=
  (\id \otimes \omega_{v,v'}) (\Gr{(X^{-1})}{n}{m}{l}{k})) \\ & =
  (\id \otimes \omega_{v,v'})( (\Gr{X}{m}{n}{k}{l})^{*}) =
  (\id \otimes \omega_{v',v})(\Gr{X}{m}{n}{k}{l})^{*}.
\end{align*}
This equation and Corollary \ref{CorOrth} imply the following corollary.
\begin{Cor}\label{cor:rep-unitary-schur-orthogonality}
  Let $\mathscr{A}$ define a partial compact quantum group with
  invertible integral $\phi$, let $(\Hsp,\mathscr{X})$ be an irreducible
  unitary rcfd corepresentation of $\mathscr{A}$, let $F_{\mathscr{X}}$ be a positive
  isomorphism from $(\Hsp,\mathscr{X})$ to
  $(\Hsp,\dualco{\dualco{\mathscr{X}}})$ and
  $G_{\mathscr{X}}=F^{-1}_{{\mathscr{X}}}$, and let $a=(\id \otimes
  \omega_{v,v'})(\Gr{X}{k}{l}{m}{n})$ and $b=(\id \otimes
  \omega_{w,w'})(\Gr{X}{k}{l}{m}{n})$, where $v,w \in
  \Gru{\Hsp}{k}{l}$ and $v',w' \in \Gru{\Hsp}{m}{n}$.  Then
\begin{align*}
  \phi(b^{*}a) &= \frac{\langle w|v'\rangle\langle v|G_{\mathscr{X}}w'\rangle}{\sum_{r}
    \Tr(\Gr{G}{r}{n}{}{\mathscr{X}})}, & \phi(ab^{*}) = \frac{\langle
    w|F_{\mathscr{X}}v'\rangle \langle v|w'\rangle}{\sum_{s}
    \Tr(\Gr{F}{m}{s}{}{\mathscr{X}})}.
\end{align*}
\end{Cor}
As a consequence of Proposition \ref{prop:rep-weak-pw} and Proposition
\ref{prop:rep-unitarisable} or Lemma \ref{lemma:rep-regular-unitary},
the matrix coefficients of irreducible unitary rcfd corepresentations
span $\mathscr{A}$, and in the Corollary \ref{cor:rep-pw}, we may
assume the irreducible rcfd corepresentations
$(V_{i},\mathscr{X}_{i})$ to be unitary if $\mathscr{A}$
defines a partial compact quantum group.

\subsection{Analogues of Woronowicz's  characters}

% We already need positivity here to define $F^z$!

Let $\mathscr{A}$ be a partial bialgebra, and $a\in \Gr{A}{k}{l}{m}{n}$. Then for $\omega \in \Hom_{\C}(A,\C)$, we can define
\begin{align*}
  \omega \aste{p,q} a
&:= (\id \otimes \omega) (\Delta_{pq}(a)), & a \aste{r,s}
\omega&:=(\omega \otimes \id)(\Delta_{rs}(a)).\end{align*} Clearly we can define
\begin{align*} \omega \aste{p,q} a \aste{r,s}
\omega'&:= (\omega \aste{p,q} a)\aste{r,s} \omega' = \omega \aste{p,q}(a \aste{r,s} \omega').\end{align*}
When $\omega$ has support on the $A(K)$ with $K_u=K_d$, we can write, for $a\in \Gr{A}{k}{l}{m}{n}$, \[\omega\ast a := \sum_{p,q} \omega\aste{p,q}a = \omega\aste{m,n}a,\quad  a\ast \omega = \sum_{r,s} a\aste{r,s}\omega = a\aste{k,l}\omega.\] 

We shall say that an entire function $f$ has \emph{exponential growth
  on the right half-plane} if there exist $C,d>0$ such that $|f(x+iy)|\leq
C\mathrm{e}^{dx}$  for all $x,y\in \R$ with $x>0$. 

\begin{Theorem} \label{thm:rep-characters} Let $\mathscr{A}$ be a partial Hopf $^*$-algebra with positive invariant integral $\phi$.  Then there exists a unique
  family of linear functionals $f_{z} \colon A\to \C$ such that
\begin{enumerate}[label={(\arabic*)}]
  \item $f_z$ vanishes on $A(K)$ when $K_u\neq K_d$.
  \item for each $a\in A$, the function $z\mapsto f_{z}(a)$ is entire
    and of exponential growth on the right half-plane.
  \item $f_{0} = \epsilon$ and $(f_{z} \otimes f_{z'}) \circ 
    \Delta= f_{z+z'}$ for all $z,z' \in \C$.
  \item $\phi(ab)=\phi(b(f_{1} \ast a \ast f_{1}))$ for all $a,b\in A$.
  \end{enumerate}
  This family furthermore satisfies
  \begin{enumerate}[label={(\arabic*)}]\setcounter{enumi}{4}
  \item $f_z(ab) = f_z(a)f_z(b)$ for $a\in A(K)$ and $b\in A(L)$ with $K_r = L_l$. 
  \item $S^{2}(a)=f_{-1} \ast a \ast f_{1}$ for all $a\in A$.
  \item $f_{z}(\UnitC{l}{n})=\delta_{l,n}$ and $f_{z} \circ S = f_{-z}$ for all $a\in A$.
\end{enumerate}
\end{Theorem}


Note that conditions (3), (4) and (6) are meaningful by condition (1).

\begin{proof}
  We first prove uniqueness.  Assume that $(f_{z})_{z}$ is a family of
  functionals satisfying (1)--(4).  Since $\phi$ is faithful, the map
  $\sigma\colon a \mapsto f_{1} \ast a \ast f_{1}$ is uniquely
  determined by $\phi$, and one easily sees that it is a homomorphism. Using
  (3), we find that $\epsilon \circ \sigma^n=f_{2n}$, which uniquely determines these functionals. Using (2) and the
  fact that every entire function of exponential growth on the right
  half-plane is uniquely determined by its values at $\N \subseteq \C$, we can conclude that the family $f_{z}$ is uniquely determined. Moreover, since the property (5) holds for $z = 2n$, we also conclude by the same argument as above that it holds for all $z\in \C$.

  Let us now prove existence.  By Theorem \ref{thm:rep-orthogonality}, Corollary \ref{cor:rep-pw} and Proposition \ref{prop:rep-unitary-bidual}, we can
  define for each $z\in \C$ a functional $f_{z} \colon A \to \C$ such
  that for every 
  %$\alpha,\beta$ in the hyperobject set $\mathscr{I}$ and each 
  irreducible rcfd corepresentation
  $(V,\mathscr{X})$ in $\Corep_{u}(\mathscr{A})$,
    \begin{align*}
      f_{z}((\id \otimes \omega_{\xi,\eta})(\Gr{X}{k}{l}{m}{n})) &=
      \delta_{k,m}\delta_{l,n} \cdot
      \omega_{\xi,\eta}((\Gr{F}{k}{l}{}{\mathscr{X}})^{z}) \quad \text{for all }
      \xi \in \Gru{V}{k}{l},\eta \in
      \Gru{V}{m}{n},
    \end{align*}
    or, equivalently,
    \begin{align*}
      (f_{z} \otimes \id)(\Gr{X}{k}{l}{m}{n}) =
      \delta_{k,m}\delta_{l,n} \cdot (\Gru{(F_{\mathscr{X}})}{k}{l})^{z},
    \end{align*}
    where $F_{\mathscr{X}}$ is a non-zero positive operator implementing a morphism from $(V,\mathscr{X})$ to
    $(V, \dualco{\dualco{\mathscr{X}}})$, scaled such that
    \begin{align*}
      d_{\mathscr{X}}:= \sum_{r} \Tr(\Gru{(F_{\mathscr{X}}^{-1})}{r}{l}) = \sum_{s}
      \Tr(\Gru{(F_{\mathscr{X}})}{m}{s})
    \end{align*}
    for all $l$ in the right and all $m$ in the left hyperobject support of $\mathscr{X}$. By
    construction, (1) and (2) hold. We show that the $(f_{z})_{z}$ satisfy the
    assertions (3)--(7). 
    %We have already argued that (5) is satisfied
    %$f_{z}$ is a character. 
    Throughout the following arguments, let 
    $(V,\mathscr{X})$ be a unitary irreducible corepresentation
    $(V,\mathscr{X})$ and let $F=F_{\mathscr{X}}$ be as above.

    We first prove property (3). This follows from the relations
    \begin{align*}
      (f_{0}  \otimes \id)(\Gr{X}{k}{l}{m}{n}) &=
      \delta_{k,m}\delta_{l,n} \id_{\Gru{V}{k}{l}} =
      (\epsilon \otimes \id)(\Gr{X}{k}{l}{m}{n})
    \end{align*}
    and
    \begin{align*}
      (((f_{z}\otimes f_{z'})\circ \Delta) \otimes
      \id)(\Gr{X}{k}{l}{m}{n}) &=  \delta_{k,m}\delta_{l,n}(f_{z} \otimes f_{z'} \otimes
      \id)\big((\Gr{X}{k}{l}{k}{l})_{13}
      (\Gr{X}{k}{l}{k}{l})_{23}\big) \\
      &=  \delta_{k,m}\delta_{l,n}(\Gru{F}{k}{l})^{z}  \cdot (\Gru{F}{k}{l})^{z'} \\
      &= (f_{z+z'} \otimes \id)(\Gr{X}{k}{l}{m}{n}).
    \end{align*}
    Applying slice maps of the form $\id
    \otimes \omega_{\xi,\xi'}$ and invoking Theorem \ref{thm:rep-orthogonality}, this proves (3).

% Again? Check if this has already been used before   
    To prove (4), write again $ \Delta^{(2)} = (
    \Delta \otimes \id)\circ  \Delta = (\id \otimes 
    \Delta) \circ \Delta$, and put \[\theta_{z,z'}:=(f_{z'} \otimes \id
    \otimes f_{z})\circ  \Delta^{(2)}.\] Then
    \begin{align*}
      (\theta_{z,z'} \otimes \id)(\Gr{X}{k}{l}{m}{n}) &= (f_{z'} \otimes
      \id \otimes f_{z} \otimes
      \id)((\Gr{X}{k}{l}{k}{l})_{14}(\Gr{X}{k}{l}{m}{n})_{24}(\Gr{X}{m}{n}{m}{n})_{34})
      \\
      &= (1 \otimes (\Gru{F}{k}{l})^{z'}) \Gr{X}{k}{l}{m}{n} (1
      \otimes (\Gru{F}{m}{n})^{z}).
    \end{align*}
    We take $z=z'=1$, use Theorem \ref{thm:rep-orthogonality}, where
    now $d_F= d_G=d_{\mathscr{X}}$ by our scaling of $F$, and obtain
    \begin{eqnarray*}
     && \hspace{-2cm} (\phi \otimes \id \otimes
      \id)((\Gr{X}{k}{l}{m}{n})_{12}^{*}((\theta_{1,1} \otimes
      \id)(\Gr{X}{k}{l}{m}{n}))_{13})\\ && =d_{\mathscr{X}}^{-1}(\id \otimes
      \Gru{F}{k}{l}) (\id \otimes \Gru{(F^{-1})}{k}{l})
      \Sigma_{k,l,m,n} (\id \otimes
      \Gru{F}{m}{n}) \\
      &&=d_{\mathscr{X}}^{-1}(\Gru{F}{m}{n} \otimes \id) \Sigma_{k,l,m,n} \\
      &&= (\phi \otimes \id \otimes
      \id)((\Gr{X}{k}{l}{m}{n})_{13}(\Gr{X}{k}{l}{m}{n})_{12}^{*}).
    \end{eqnarray*}
    To conclude the proof of assertion (4), apply again slice maps of the form
    $\omega_{\xi,\xi'} \otimes \omega_{\eta,\eta'}$.

We have then already argued that the property (5) automatically holds. To show the property (6), note that by Proposition \ref{prop:rep-unitary-bidual} and the calculation above,
    \begin{align*}
      (S^{2} \otimes \id)(\Gr{X}{k}{l}{m}{n}) &= (1
      \otimes\Gru{F_{\mathscr{X}}}{k}{l})
      \Gr{X}{k}{l}{m}{n}(1 \otimes \Gru{F_{\mathscr{X}}}{m}{n})^{-1} 
      =(\theta_{-1,1}  \otimes \id)(\Gr{X}{k}{l}{m}{n}).
    \end{align*}
     Assertion (6) follows again by applying slice maps.
    
     Finally, (1), (2) and (4)
     immediately imply the relation
     $f_{z}(\UnitC{k}{m})=\delta_{k,m}$. As both $z \rightarrow f_{-z}$ and $z\rightarrow f_z\circ S$ satisfy the conditions (1)--(4) for $\mathscr{A}$ with the opposite product and coproduct (using the partial character property (5) and the invariance of $\phi$ with respect to $S$), we find $f_{-z} = f_{z} \circ S$
     
     %The concrete construction of $f_z$ combined with property (3), the identity \eqref{eq:rep-delta2} and the partial character property (5) gives the equality
     %\begin{align}
      % (f_{-z} \otimes \id) (\Gr{X}{k}{l}{k}{l})=
       %(\Gru{(F_{\mathscr{X}})}{k}{l})^{-z} &=\left( (f_{z} \otimes
       %\id)(\Gr{X}{k}{l}{k}{l})\right)^{-1} \\ &= (f_{z} \otimes
       %\id)(\Gr{(X^{-1})}{l}{k}{l}{k}) = ((f_{z} \circ S) \otimes
       %\id)(\Gr{X}{k}{l}{k}{l}).
     %\end{align}
%Therefore, .
\end{proof}
\begin{Cor} \label{cor:rep-characters} Let $\mathscr{A}$ be a 
  partial Hopf $^*$-algebra with positive invariant integral $\phi$ and define $\theta_{z,z'} \colon A
  \to A$ by $a \mapsto f_{z} \ast a \ast f_{z'}$ for each $z,z' \in
  \C$, where the functionals $f_{z}$ are as in Theorem
  \ref{thm:rep-characters}. Then for all $z,z',w,w'\in \C$, the
  following conditions hold.
  \begin{enumerate}[label={(\arabic*)}]
  \item $\theta_{z,z'}$ is an algebra automorphism and preserves
    each subspace $A(K)$. In particular,
    $\theta_{z,z'}(\lambda_{k}\rho_{m}) = \lambda_{k}\rho_{m}$ for all
    $k,m\in I$.
  \item $\theta_{z,z'}\circ \theta_{w,w'} = \theta_{z+w,z'+w'}$.
  \item $ (\theta_{w,z'} \otimes \theta_{z,-w}) \circ \Delta = \Delta
    \circ \theta_{z,z'}$, $\epsilon \circ \theta_{z,z'} = f_{z+z'}$,
    $\theta_{z,z'} \circ S = S \circ \theta_{-z',-z}$ and
    $\phi \circ \theta_{z,z'} = \phi$.
  \item For every linear map $\omega \colon A \to \C$ and every $a\in
    A$, the map $(z,z') \mapsto \omega(\theta_{z,z'}(a))$ is entire.
  \end{enumerate}
\end{Cor}
\begin{proof}
  All of this follows easily from Theorem \ref{thm:rep-characters}.
\end{proof}
Using the two-parameter group $\theta$, we define the \emph{modular
  automorphism group} $\sigma$, the \emph{scaling group} $\tau$   and
the \emph{unitary antipode} of a partial compact quantum group $A$ by
\begin{align} \label{eq:rep-groups}
  \sigma_{z} &:=\theta_{iz,iz}, & \tau_{z} &:=\theta_{iz,-iz}, & R&:=S
  \circ \tau_{i/2}.
\end{align}
Using Corollary \ref{cor:rep-characters}, one verifies that
$\sigma,\tau,R$ share all the main relations known for locally compact
quantum groups and measured quantum groupoids. For example, $\sigma$
and $\tau$ are complex one-parameter groups of algebra automorphisms
of $A$, the map $R$ is an anti-automorphism, the family  $\tau$ commutes with
$\sigma$ and with $R$ in the obvious sense, 
  \begin{gather}
    \begin{aligned} \label{eq:modular}
      \phi\circ \sigma_{z} &= \phi \circ \tau_{z} = \phi \circ R =
      \phi, & \phi(ab) &= \phi(b\sigma_{-i}(a)),
    \end{aligned}
\\ \label{eq:scaling-modular-delta}
    \begin{aligned} 
    \Delta \circ \tau_{z} &= (\tau_{z} \otimes \tau_{z}) \circ \Delta = (\sigma_{-z}\otimes \sigma_{z})\circ\Delta,
    & (\tau_{z} \otimes \sigma_{z}) \circ \Delta &= \Delta \circ
    \sigma_{z} = (\sigma_{-z} \otimes \tau_{z}) \circ \Delta,      
  \end{aligned} \\
  \begin{aligned} \label{eq:unitary-antipode}
    R^{2} &= \id_{A}, & \Delta \circ R &= (R \otimes R) \circ
    \Delta^{\op}.
  \end{aligned}
  \end{gather}

The relation with the $*$-structure is determined by the following proposition.

%Finally, we consider the functionals $f_{z}$, the automorphisms
%$\theta_{z,z'}$, $\tau_{z}$, $\sigma_{z}$ and the anti-isomorphism $R$
%introduced in Theorem \ref{thm:rep-characters}, Corollary
%\ref{cor:rep-characters} and \eqref{eq:rep-groups}.
\begin{Prop} \label{prop:rep-unitary-characters}  Let $\mathscr{A}$ be a 
  partial Hopf $^*$-algebra with positive invariant integral $\phi$.  Then $f_{z}(a^*) =
   \overline{f_{-\overline{z}}(a)}$ and $\theta_{z,w}(a^*) = \overline{\theta_{-\overline{z},-\overline{w}}(a)}$ for all $z,w\in \C$ and $a\in A$. In
  particular, $R$ is a $*$-anti-automorphism and $\theta_{it,is}$,
  $\tau_{t}$ and $\sigma_{t}$ are $*$-automorphisms for all $s,t\in
  \R$.
\end{Prop}
\begin{proof}
  We only have to prove the first equation.  Write $\bar{f}_z(a) =
  \overline{f_z(a^*)}$. Using the relations
  $ (\Gr{X}{k}{l}{k}{l})^{*}=(S \otimes \id)(\Gr{X}{k}{l}{k}{l})$,  $f_{z} \circ S=f_{-z}$
  (Theorem \ref{thm:rep-characters}) and
  positivity of $\Gr{F}{k}{l}{}{\mathscr{X}}$ (Proposition \ref{prop:rep-unitary-bidual}), we conclude
     \begin{align*}
       (\bar{f}_z \otimes
       \id)(\Gr{X}{k}{l}{k}{l})
&=       \left((f_{z} \otimes
       \id)((\Gr{X}{k}{l}{k}{l})^{*})\right)^{*} \\
& = \left((f_{-z} \otimes \id)(\Gr{X}{k}{l}{k}{l})\right)^{*} 
 =
((\Gr{F}{k}{l}{}{\mathscr{X}})^{-z})^{*} 
=       (\Gr{F}{k}{l}{}{\mathscr{X}})^{-\overline{z}} = (f_{-\overline{z}}
\otimes \id)(\Gr{X}{k}{l}{k}{l}),
     \end{align*}
whence $\bar{f}_z(a) = f_{-\overline{z}}(a)$ for all $a\in
\Gr{\mathcal{C}(\mathscr{X})}{k}{l}{k}{l}$. Since $f_{z}$ and
$f_{-\overline{z}}$ vanish on $\Gr{A}{k}{l}{m}{n}$ if $(k,l)\neq
(m,n)$ and the matrix coefficients of unitary 
corepresentations span $A$, we can conclude $\bar{f}_{z}=f_{-\overline{z}}$.
\end{proof}



%  consider
%   the family
%   \begin{align*}
%     (\Gru{F}{m}{n})^{\top} \circ  \overline{\Sigma_{m,n,k,l}} &=
% \phi((\Gr{X}{k}{l}{m}{n})_{12}^{*}(\Gr{X}{k}{l}{m}{n})_{13})^{\top}
% \circ  \overline{\Sigma_{m,n,k,l}} \\
%  &=
%  \phi((\Gr{X}{k}{l}{m}{n})_{12}^{*\circ (\id \otimes
%   \top)}(\overline{\Sigma_{k,l,k,l}})_{23} (\Gr{X}{k}{l}{m}{n})^{\id \otimes
%   \top}_{12})  \\
% &=\phi((\Gr{\overline{X}}{l}{k}{n}{m})_{12}(\overline{\Sigma_{k,l,k,l}})_{23}
%  (\Gr{\overline{X}}{l}{k}{n}{m})_{12}).
% \end{align*} \fxnote{Treat $X^{\id \otimes \top}$}
% By Lemma \ref{lem:rep-average}, this family is a morphism from to
% $(\overline{\mathcal{H}}\otimes \overline{K},\overline{X}^{-*} \otimes
% \id_{\overline{K}})$  to
% $(\overline{\mathcal{H}}\otimes \overline{K},\overline{X} \otimes
% \id_{\overline{K}})$ and hence of the form
% $(\Gru{\overline{F_{X}}}{n}{m})^{-1} \otimes T$ with $T \in
% \mathcal{B}(\overline{K})$ not depending on $m,n$.

% Thus,
% \begin{align*}
%  (\id \otimes R)  \circ \Sigma_{k,l,m,n} =   \Gru{F}{m}{n} =
%  \Sigma_{k,l,m,n} \circ (\Gru{\overline{F_{X}}}{n}{m})^{-\top} \otimes T^{\top})
% \end{align*}

  
%   We may assume $(k,l,m,n)=(p,q,r,s)$ because otherwise both sides of
%   the equation that we want to prove vanish.

%   Applying Lemma \ref{lem:rep-average} to the corepresentation $X$ and
%   the family $\Gru{T}{p}{q}=
%   \delta_{p,k}\delta_{q,l} |\eta\rangle\langle\xi|$, we obtain an
%   endomorphism $\check{T}$ of $(\mathcal{H},X)$ which
%   necessarily has the form $\check{T}=\lambda(\xi,\eta) \id$ for some
%   $\lambda(\xi,\eta) \in \C$. Inserting the definition of
%   $\check{T}$, we find
%   \begin{align}\nonumber
%     \phi(b^{*}a) &= \phi\big((\id \otimes
%     \omega_{\eta',\eta})((\Gr{X}{k}{l}{m}{n})^{*}) \cdot (\id \otimes
%     \omega_{\xi,\xi'})(\Gr{X}{k}{l}{m}{n})\big) \\  &= (\phi \otimes
%     \id)\left(\langle\eta'|_{2} \Gr{(X^{-1})}{m}{n}{k}{l}(1 \otimes
%       |\eta\rangle\langle \xi|)
%       \Gr{X}{k}{l}{m}{n}|\xi'\rangle_{2}\right) 
%     = \langle \eta'|_{2} \Gru{\check{T}}{m}{n}|\xi'\rangle_{2} =
%     \lambda(\xi,\eta) \langle\eta'|\xi'\rangle. \label{eq:rep-orthogonal-1}
%   \end{align}
%   Next, we apply Lemma \ref{lem:rep-average} to the corepresentations
%   $\overline{X}$ and $\overline{X}^{-*}$ and the family
%   $\Gru{R}{p}{q}=\delta_{p,m}\delta_{q,n}|\overline{
%     \xi'}\rangle\langle\overline{\eta'}|$, and obtain a morphism
%   $\hat{R}$ from $\overline{X}^{-*}$ to $\overline{X}$ which
%   necessarily has the form $\hat{R}=\mu(\eta',\xi')\overline{F_{X}}$
%   for some $\mu(\eta',\xi') \in \C$. Using the relation
%   \begin{align*}
%     a &= (\id \otimes
%     \omega_{\overline{\xi},\overline{\xi'}})(\Gr{\overline{X}}{l}{k}{n}{m})^{*},
%     & b&= (\id \otimes
%     \omega_{\overline{\eta},\overline{\eta'}})(\Gr{\overline{X}}{k}{l}{m}{n})^{*}
%   \end{align*}
%   and the definition of $\hat{R}$, we obtain
%   \begin{align}
%     \phi(b^{*}a) &= (\phi \otimes \id)\left(\langle
%       \overline{\eta}|_{2} \Gr{\overline{X}}{k}{l}{m}{n}(1 \otimes
%       |\overline{\eta}'\rangle\langle \overline{\xi'}|)
%       \Gr{(\overline{X}^{*})}{m}{n}{k}{l} \right) \nonumber \\
%     &=\langle \overline{\eta}| \Gru{\hat{R}}{k}{l}
%     |\overline{\xi}\rangle = \langle
%     \overline{\eta}|\Gru{\overline{F_{X}}}{k}{l}\overline{\xi}\rangle
%     \mu(\eta',\xi'). \label{eq:rep-orthogonal-2}
%   \end{align}
%   We choose a basis $(\zeta_{i})_{i}$ for
%   $\bigoplus_{k}\Gru{\mathcal{H}}{k}{l}$ and calculate
%   \begin{align*}
%  \langle
%     \eta'|\xi'\rangle &=  (\phi \otimes
%     \id)(\langle\eta'|_{2}(\lambda_{l}\rho_{n} \otimes
%     \id_{\Gru{\mathcal{H}}{m}{n}})|\xi'\rangle_{2}) \\
%  &=
%     \sum_{k} (\phi \otimes \id)\left(\langle\eta'|_{2}
%       \Gr{(X^{-1})}{m}{n}{k}{l}
%       \Gr{X}{k}{l}{m}{n}|\xi'\rangle_{2}\right)
%     \\ &=    \sum_{i} \lambda(\zeta_{i},\zeta_{i}) \langle
%     \eta'|\xi'\rangle 
%     \\
%     &=\sum_{i} \langle
%     \overline{\zeta_{i}}| \Gru{\overline{F_{X}}}{}{l}\overline{\zeta_{i}}\rangle
%     \mu(\eta',\xi') 
% \\ &    = \mu(\eta',\xi') \cdot \sum_{k} \Tr(\Gru{F_{(X)}}{k}{l}),
%   \end{align*}
% where $\Gru{\overline{F_{X}}}{}{l}=\bigoplus_{k}
% \Gru{\overline{F_{X}}}{k}{l}$. Inserting this relation into
% \eqref{eq:rep-orthogonal-2}, we finally obtain the assertion.


% We use here the standard leg numbering notation, e.g. $a_{12} =
% a\otimes 1$.


%%% Local Variables: 
%%% mode: latex
%%% TeX-master: "dyn-suq-main"
%%% End: 
