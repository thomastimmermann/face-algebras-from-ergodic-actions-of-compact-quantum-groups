\section{Partial compact quantum groups}

We generalize Hayashi's definition of a compact quantum group of face type \cite{Hay1} to the case where the commutative base algebra is no longer finite-dimensional. We will present two approaches, based on \emph{partial bialgebras} and \emph{weak multiplier bialgebras} \cite{Boh1,VDW1}. The first approach is piecewise and concrete, but requires some bookkeeping. The second approach is global but more abstract. As we will see from the general theory and the concrete examples, both approaches have their intrinsic value.

\subsection{Partial algebras}

Let $I$ be a set. We consider $I^2=I\times I$ as the pair groupoid with $\wmult$ denoting composition. That is, an element $K=(k,l)\in I^2$ has source $K_l = k$ and target $K_r=l$, and if $K=(k,l)$ and $L=(l,m)$ we write $K\wmult L = (k,m)$. 

\begin{Def} A \emph{partial algebra} $\mathscr{A}=(\mathscr{A},M)$ (over $\C$) is a set $I$ (the \emph{object} set) together with 
\begin{itemize}
\item[$\bullet$] for each $K=(k,l)\in I^2$ a vector space $A(K) = \Grs{A}{k}{l}=\!\!\GrDA{A}{k}{l}$ (possibly the zero vector space),
\item[$\bullet$] for each $K,L$ with $K_r = L_l$ a multiplication map \[M(K,L):A(K) \otimes A(L)\rightarrow A(K\cdot L),\qquad a\otimes b \mapsto ab\]  and 
\item[$\bullet$] elements $\Unit(k) = \Unit_k \in \Grs{A}{k}{k}$ (the units), % or the local units?
\end{itemize}
such that the obvious associativity and unit conditions are satisfied. 

By \emph{$I$-partial algebra} will be meant a partial algebra with object set $I$.
\end{Def}

\begin{Rem}
\begin{enumerate}\item It will be important to allow the local units $\Unit_k$ to be zero.
\item A partial algebra is by definition the same as a small $\C$-linear category. However, we do not emphasize this viewpoint, as the natural notion of morphism for partial algebras will be \emph{contravariant} on objects, see Definition \ref{DefMor}.% Is this correct, or are local units = 0 not allowed? 
\end{enumerate}
\end{Rem}

Let $\mathscr{A}$ be an $I$-partial algebra. We define $A(K\wmult L)$ to be $\{0\}$ when $K\wmult L$ is ill-defined, i.e. $K_r\neq L_l$. We then let $\Grs{M}{K}{L}$ be the zero map.

\begin{Def} The \emph{total algebra} $A$ of an $I$-partial algebra $\mathscr{A}$ is the vector space \[A = \oplus_{K\in I^2} A(K)\] endowed with the unique multiplication whose restriction to $A(K)\otimes A(L)$ concides with $M(K,L)$. 
\end{Def} 

Clearly $A$ is an associative algebra. If $A_0$ is infinite-dimensional it will not possess a unit, but it is a \emph{locally unital algebra} as there exist mutually orthogonal idempotents $\mathbf{1}_k$ with $A = \osum{k,l} \mathbf{1}_kA\mathbf{1}_l$. An element $a\in A$ can be interpreted as a function assigning to each element $(k,l)\in I^2$ an element $a_{kl}\in A(k,l)$, namely the $(k,l)$-th component of $a$. This identifies $A$ with finite support $I$-indexed matrices whose $(k,l)$-th entry lies in $A(k,l)$, equipped with the natural matrix multiplication. 

\begin{Rem}\label{RemGrad} When $\mathscr{A}$ is an $I$-partial algebra with total algebra $A$, then $A\otimes A$ can be naturally identified with the total algebra of an $I\times I$-partial algebra $\mathscr{A}\otimes \mathscr{A}$, where \[(A\otimes A)((k,k'),(l,l')) = A(k,l)\otimes A(k',l')\] with the obvious tensor product multiplications and the $\Unit_{k,k'} = \Unit_k\otimes \Unit_{k'}$ as units. 
\end{Rem}

Working with non-unital algebras necessitates the use of their \emph{multiplier algebra}. Let us first recall some general notions concerning non-unital algebras from \cite{Dau1,VDae1}.

\begin{Def} Let $A$ be an algebra over $\C$, not necessarily with unit. We call $A$ \emph{non-degenerate} if $A$ is faithfully represented on itself by left and right multiplication. It is called \emph{idempotent} if $A^2 = A$. 
\end{Def}

\begin{Def} Let $A$ be an algebra. A \emph{multiplier} $m$ for $A$ consists of a couple of maps \begin{eqnarray*} L_m:A\rightarrow A,\quad a\mapsto ma\\ R_m:A\rightarrow A,\quad a\mapsto am\end{eqnarray*} such that $(am)b = a(mb)$ for all $a,b\in A$. 

The set of all multipliers forms an algebra under composition for the $L$-maps and anti-composition for the $R$-maps. It is called the \emph{multiplier algebra} of $A$, and is denoted $M(A)$.
\end{Def}

One has a natural homomorphism $A\rightarrow M(A)$. When $A$ is non-degenerate,  this homomorphism is injective, and we can then identify $A$ as a subalgebra of the (unital) algebra $M(A)$. We then also have inclusions \[A\otimes A\subseteq M(A)\otimes M(A)\subseteq M(A\otimes A).\]

\begin{Exa}\label{ExaMult} 
\begin{enumerate}
\item Let $I$ be a set, and $\Fun_{\fin}(I)$ the algebra of all finite support functions on $I$. Then $M(\Fun_{\fin}(I)) = \Fun(I)$, the algebra of all functions on $I$. 
\item Let $A$ be the total algebra of an $I$-partial algebra $\mathscr{A}$. As $A$ has local units, it is non-degenerate and idempotent. Then one can identify $M(A)$ with \[M(A) = \left(\prod_l \oplus_k A(k,l)\right) \bigcap \left(\prod_k\oplus_l A(k,l)\right) \subseteq \prod_{k,l} A(k,l),\] i.e.~ with the space of functions \[m:I^2\rightarrow A,\quad m_{kl}\in A(k,l)\] which have finite support in either one of the variables when the other variable has been fixed. The multiplication is given by the formula \[(mn)_{kl} = \sum_p m_{kp}n_{pl}.\]
\item Let $m_i$ be any collection of multipliers of $A$, and assume that for each $a\in A$, $m_ia =0$ for almost all $i$, and similarly $am_i=0$ for almost all $i$. Then one can define a multiplier $\sum_i m_i$ in the obvious way by termwise multiplication. One says that the sum $\sum_i m_i$ converges in the \emph{strict} topology. 
\end{enumerate}
\end{Exa}

The condition appearing in the second example above will appear time and again, so we introduce it formally in the next definition.

\begin{Def} We will call any general assignment $(k,l)\rightarrow m_{kl}$ into a set with a distinguished zero element \emph{row-and column-finite} (rcf) if the assignment has finite support in either one of the variables when the other variable has been fixed. 0
\end{Def} 

Let us comment on the notion of morphism for partial algebras. We first introduce the piecewise definition.

\begin{Def}\label{DefMor} Let $\mathscr{A}$ and $\mathscr{B}$ be respectively $I$ and $J$-partial algebras. Let \[\phi: I \ni k \mapsto J_k \subseteq J\] with the $J_k$ disjoint. A \emph{homomorphism} (based on $\phi$) from $\mathscr{A}$ to $\mathscr{B}$ consists of linear maps \[\GrDA{f}{r}{s}: A(k\;l)\rightarrow B(r\;s),\quad a\mapsto \GrDA{f(a)}{r}{s}\] for all $r\in J_k, s\in J_l$, satisfying 
\begin{itemize}
\item[$\bullet$] (Unitality) $\GrDA{f(\Unit_{k})}{r}{s} = \delta_{rs}\Unit_r$ for all $r,s\in J_k$.
\item[$\bullet$] (Local finiteness) For each $k,l\in I$ and $a\in A(k\;l)$, the assigment $(r,s)\rightarrow \GrDA{f(a)}{r}{s}$ on $J_k\times J_l$ is rcf. 
\item[$\bullet$] (Multiplicativity) For all $k,l,m\in I$, all $r\in J_k$ and all $t\in J_m$, and all $a\in A(k\;l)$ and $b\in A(l\;m)$, one has \[\GrDA{f(ab)}{r}{t} = \sum_{s\in J_l} \GrDA{f(a)}{r}{s}\GrDA{f(b)}{s}{t}.\]
\end{itemize} 
The homomorphism is called \emph{unital} if $J=\bigcup \{J_k\mid k\in I\}$. % Can't call this a partition since $J_k$ can be empty 
\end{Def}
\begin{Rem}
\begin{enumerate}
\item
Note that the multiplicativity condition makes sense because of the local finiteness condition.
\item
If $J = \bigcup\{J_k\}$, we can interpret $\phi$ as a map \[J\rightarrow I,\quad r\mapsto k \iff r\in J_k.\] In the more general case, we obtain a function $J\rightarrow I^*$, where $I^*$ is $I$ with an extra point `at infinity' added.
\end{enumerate}
\end{Rem}

The following lemma provides the global viewpoint concerning homomorphisms. 

\begin{Lem} Let $\mathscr{A}$ and $\mathscr{B}$ be respective $I$- and $J$-partial algebras, and fix an assigment $\phi: k\mapsto J_k$. Then there is a one-to-one correspondence between homomorphisms $\mathscr{A}\rightarrow \mathscr{B}$ based on $\phi$ and homomorphisms $f:A\rightarrow M(B)$ with $f(\Unit_k) = \sum_{r\in J_k} \Unit_r$. 
\end{Lem} 
\begin{proof}
Straightforward, using the characterisation of the multiplier algebra provided in Remark \ref{ExaMult}.2.
\end{proof}

\subsection{Partial coalgebras}

The notion of partial algebra nicely dualizes, one of the main benefits of the local approach. For this we consider again $I^2$ as the pair groupoid, but now with elements considered as column vectors, and with $\bmult$ denoting the (vertical) composition. So $K=\Grt{}{k}{l}$ has source $K_u = k$ and target $K_d = l$, and if $K=\Grt{}{k}{l}$ and $L=\Grt{}{l}{m}$ then $K\bmult L = \Grt{}{k}{m}$.

\begin{Def} A \emph{partial coalgebra} $\mathscr{A}=(\mathscr{A},\Delta)$ (over $\C$) consists of a set $I$ (the object set) together with 
\begin{itemize}
\item[$\bullet$] for each $K=\Grru{k}{l}\in I^2$ a vector space $A(K) = \Grt{A}{k}{l}=\!\!\GrRA{A}{k}{l}$,
\item[$\bullet$] for each $K,L$ with $K_d = L_u$ a comultiplication map \[\Grt{\Delta}{K}{L}:A(K*L)\rightarrow A(K)\otimes A(L),\qquad a \mapsto a_{(1)K}\otimes a_{(2)L},\] and 
\item[$\bullet$] counit maps $\epsilon_k:\Grt{A}{k}{k}\rightarrow \C$,
\end{itemize} 
satisfying the obvious coassociativity and counitality conditions.

By \emph{$I$-partial coalgebra} will be meant a partial coalgebra with object set $I$.
\end{Def}

\begin{Not}\label{NotCom} As the index of $\epsilon_k$ is determined by the element to which it is applied, there is no harm in dropping the index $k$ and simply writing $\epsilon$.

Similarly, if $K = \Grt{}{k}{l}$ and $L = \Grt{}{l}{m}$, we abbreviate $\Delta_l = \Grt{\Delta}{K}{L}$, as the other indices are determined by the element to which $\Delta_l$ is applied.
\end{Not}

We also make again the convention that $A(K*L)=\{0\}$ and $\Grt{\Delta}{K}{L}$ the zero map when $K_d \neq L_u$. Similarly $\epsilon$ is seen as the zero functional on $A(K)$ when $K=\Grt{}{k}{l}$ with $k\neq l$. 

\subsection{Partial bialgebras}

We can now superpose the notions of partial algebra and partial coalgebra. Let $I$ be a set, and let $M_2(I)$ be the set of 4-tuples of elements of $I$ arranged as 2$\times$2-matrices. We can endow $M_2(I)$ with two compositions, namely $\cdot$ (viewing $M_2(I)$ as a row vector of column vectors) and $*$ (viewing $M_2(I)$ as a column vector of row vectors). When $K\in M_2(I)$, we will write $K = \Grs{}{K_l}{K_r} = \Grt{}{K_u}{K_d} = \eGr{}{K_{lu}}{K_{ru}}{K_{ld}}{K_{rd}}$. One can view $M_2(I)$ as a double groupoid, and in fact as a \emph{vacant} double groupoid in the sense of \cite{AN1}. 

In the following, a vector $(r,s)$ will sometimes be written simply as $r,s$ (without parentheses) or $rs$ in an index. We also follow Notation \ref{NotCom}, but the reader should be aware that the index of $\Delta$ will now be a 1$\times$2 vector in $I^2$ as we will work with partial coalgebras over $I^2$.

\begin{Def}\label{DefPartBiAlg} A \emph{partial bialgebra} $\mathscr{A}=(\mathscr{A},M,\Delta)$ consists of a set $I$ (the \emph{object set}) and a collection of vector spaces $A(K)$ for $K\in M_2(I)$ such that 
\begin{itemize}
\item[$\bullet$] the $\Grs{A}{K_l}{K_r}$ form an $I^2$-partial algebra,
\item[$\bullet$] the $\Grt{A}{K_u}{K_d}$ form an $I^2$-partial coalgebra,
\end{itemize} 
and for which the following compatibility relations are satisfied.
\begin{enumerate}[label=(\arabic*)]
\item\label{Propa} (Comultiplication of Units) For all $k,l,l',m\in I$, one has 
\[\Delta_{l,l'}(\UnitC{k}{m}) = \delta_{l,l'} \UnitC{k}{l}\otimes \UnitC{l}{m}.\]  
\item\label{Propb} (Counit of Multiplication) For all $K,L\in M_2(I)$ with $K_r = L_l$ and all $a\in A(K)$ and $b\in A(L)$, \[\epsilon(ab) = \epsilon(a)\epsilon(b).\]% Subtlety is that you already have to know composition to be able to apply this rule.
\item\label{Propc} (Non-degeneracy) For all $k\in I$, $\epsilon(\UnitC{k}{k})=1$. 
\item\label{Propd} (Finiteness) For each $K\in M_2(I)$ and each $a\in A(K)$, the assignment $(r,s)\rightarrow \Delta_{rs}(a)$ is rcf.
\item\label{Prope} (Comultiplication is multiplicative) For all $a\in A(K)$ and $b\in A(L)$ with $K_r= L_l$,  \[\Delta_{rs}(ab) = \sum_t \Delta_{rt}(a)\Delta_{ts}(b).\]
\end{enumerate}
\end{Def}

\begin{Rem}\begin{enumerate}
\item By assumption \ref{Propd}, the sum on the right hand side in condition \ref{Prope} is in fact finite and hence well-defined. 
\item Note that the object set of the above $\mathscr{A}$ as a partial bialgebra is $I$, but the object set of its underlying partial algebra (or coalgebra) is $I^2$.
\item Properties \ref{Propa}, \ref{Propd} and \ref{Prope} simply say that $\Delta$ is a homomorphism $\mathscr{A}\rightarrow \mathscr{A}\otimes \mathscr{A}$ of partial algebras based over the assignment $I^2\rightarrow \mathscr{P}(I^2\times I^2)$, the power set of $I^2\times I^2$, such that \[(I^2\times I^2)_{{\tiny \begin{pmatrix} k\\m \end{pmatrix}}} = \{\left(\begin{pmatrix} k \\ l \end{pmatrix},\begin{pmatrix} l \\ m \end{pmatrix}\right)\mid l\in I\}.\] 
\end{enumerate}
\end{Rem}

We relate the notion of partial bialgebra to the recently introduced notion of weak multiplier bialgebra \cite{Boh1}. Let us first introduce the following notation, using the notion introduced in Example \ref{ExaMult}.2.

\begin{Not}
If $\mathscr{A}$ is an $I$-partial bialgebra, we write \[\lambda_k = \sum_l \UnitC{k}{l},\qquad \rho_l = \sum_k\UnitC{k}{l} \qquad \in M(A).\]
\end{Not}

\begin{Rem} From Property \ref{Propc} of Definition \ref{DefPartBiAlg}, it follows that $\lambda_k\neq 0\neq \rho_k$ for any $k\in I$. 
\end{Rem} 

To show that the total algebra of a partial bialgebra becomes a weak multiplier bialgebra, we will need some easy lemmas. 

\begin{Lem} Let $\mathscr{A}$ be an $I$-partial bialgebra. Then for each $a\in A$, there exists a unique multiplier $\Delta(a) \in M(A\otimes A)$ such that \begin{align}\label{EqDel}
    \begin{aligned}
      \Delta_{rs}(a) &= (1\otimes \lambda_r)\Delta(a)(1\otimes
      \lambda_s) \\ &= (\rho_r\otimes 1)\Delta(a)(\rho_s\otimes 1)
    \end{aligned}
\end{align}  for all $r,s\in I$, all $K\in M_2(I)$ and all $a\in A(K)$. 

The resulting map \[\Delta:A\rightarrow M(A\otimes A),\quad a\mapsto \Delta(a)\] is a homomorphism.
\end{Lem} 
\begin{proof} For $a\in A$ homogeneous, we can define $\Delta(a) = \sum_{rs} \Delta_{rs}(a) \in M(A\otimes A)$, where the sum converges in the strict topology of $A\otimes A$ because of the property \ref{Propd} of Definition \ref{DefPartBiAlg}. This expression clearly satisfies the identities stated in the lemma. In turn, these identities uniquely define $\Delta(a)$ as a multiplier, as they determine the value of $\Delta(a)$ when cut down to the left and right with the local units of $\mathscr{A}\otimes \mathscr{A}$.

We can then extend $\Delta$ to $A$ by linearity. Since, for $a,b$ homogeneous, $\Delta_{rt}(a)\Delta_{t's}(b)=0$ unless $t=t'$, it follows from property \ref{Prope} of Definition \ref{DefPartBiAlg} that $\Delta$ is a homomorphism. 
\end{proof}

We will refer to $\Delta: A\rightarrow M(A\otimes A)$ as the \emph{total comultiplication} of $\mathscr{A}$. We will then also use the suggestive Sweedler notation for this map, \[\Delta(a) = a_{(1)}\otimes a_{(2)}.\] Note for example that \[\Delta(\UnitC{k}{m}) = \sum_{l}\UnitC{k}{l}\otimes \UnitC{l}{m} = \sum_l \lambda_k\rho_l\otimes \lambda_l\rho_m.\]

\begin{Lem} The element $E = \sum_{k,l,m} \UnitC{k}{l}\otimes \UnitC{l}{m}= \sum_l \rho_l\otimes \lambda_l$ is a well-defined idempotent in $A\otimes A$, and satisfies \[\Delta(A)(A\otimes A)=E(A\otimes A),\quad (A\otimes A)\Delta(A)= (A\otimes A)E.\]
\end{Lem} 
\begin{proof} Clearly the sum defining $E$ is strictly convergent, and makes $E$ into an idempotent. It is moreover immediate that $E\Delta(a)=\Delta(a) = \Delta(a)E$ for all $a\in A$. Since \[E(\UnitC{k}{l}\otimes \UnitC{m}{n}) = \Delta(\UnitC{k}{n})(\UnitC{k}{l}\otimes \UnitC{m}{n}) \] by the property \ref{Propa} of Definition \ref{DefPartBiAlg}, and analogously for multiplication with $E$ on the right, the lemma is proven. 
\end{proof} 

By \cite[Proposition A.3]{VDW2}, there is a unique homomorphism $\Delta:M(A)\rightarrow M(A\otimes A)$ extending $\Delta$ and such that $\Delta(1) = E$. Alternatively, if $m\in M(A)$, we can directly define $\Delta(m)$ as the strict limit of the series $\sum_{k,l,r,s} \Delta_{rs}(m_{kl})$. Similarly the maps $\id\otimes \Delta$ and $\Delta\otimes \id$ extend to maps from $M(A\otimes A)$ to $M(A\otimes A\otimes A)$. 

For example, note that
\begin{align} \label{eq:delta-lambda-rho} \Delta(\lambda_{k}) &=
  (\lambda_{k} \otimes 1)\Delta(1), & \Delta(\rho_{m}) &= (1 \otimes \rho_{m})\Delta(1).
\end{align}

The following proposition gathers the properties of $\Delta$, $\epsilon$ and $\Delta(1)$ which guarantee that $(A,\Delta)$ forms a weak multiplier bialgebra in the sense of \cite[Definition 2.1]{Boh1}. We will call it the \emph{total weak multiplier bialgebra} associated to $\mathscr{A}$.

\begin{Prop} Let $\mathscr{A}$ be a partial bialgebra with total algebra $A$, total comultiplication $\Delta$ and counit $\epsilon$. Then the following properties are satisfied.
\begin{enumerate}[label={(\arabic*)}]
\item Coassociativity: $(\Delta\otimes \id)\Delta = (\id\otimes \Delta)\Delta$ (as maps $M(A)\rightarrow M(A^{\otimes 3})$).
\item Counitality: $(\epsilon\otimes \id)(\Delta(a)(1\otimes b)) = ab = (\id\otimes \epsilon)((a\otimes 1)\Delta(b))$ for all $a,b\in A$.
\item Weak Comultiplicativity of Unit: \[(\Delta(1)\otimes 1)(1\otimes \Delta(1)) = (\Delta\otimes \id)\Delta(1) = (\id\otimes \Delta)\Delta(1) = (1\otimes \Delta(1))(\Delta(1)\otimes 1).\]
\item \label{WMC} Weak Multiplicativity of Counit: For all $a,b,c\in A$, one has \[(\epsilon\otimes \id)(\Delta(a)(b\otimes c)) = (\epsilon\otimes \id)((1\otimes a)\Delta(1)(b\otimes c))\] and 
\[(\epsilon\otimes \id)((a\otimes b)\Delta(c)) = (\epsilon\otimes \id)((a\otimes b)\Delta(1)(1\otimes c)).\]
\item Strong multiplier property: For all $a,b\in A$, one has \[\Delta(A)(1\otimes A)\cup (A\otimes 1)\Delta(A)\subseteq  A\otimes A.\] 
\end{enumerate}
\end{Prop}

\begin{proof} Most of these properties follow immediately from the definition of a partial bialgebra. For demonstrational purposes, let us check the first identity of property \ref{WMC}. Let us choose $a\in A(K)$, $b\in A(L)$ and $c\in A(M)$. Then \[\Delta(a)(b\otimes c) = \delta_{K_{ru},L_{lu}}\delta_{M_{lu},L_{ld}} \sum_r \Delta_{r,L_{ld}}(a)(b\otimes c).\]  Applying $(\epsilon\otimes \id)$ to both sides, we obtain by Proposition \ref{Propb} of Definition \ref{DefPartBiAlg} and counitality of $\epsilon$ that \[(\epsilon \otimes \id)(\Delta(a)(b\otimes c)) = \delta_{K_{ru},L_{lu},L_{ld},M_{lu}} \epsilon(b) ac.\] On the other hand, \begin{eqnarray*} (1\otimes a)\Delta(1)(b\otimes c) &=& \sum_{r,s,t} \UnitC{r}{s} b \otimes a\UnitC{s}{t}c \\ &=& \delta_{L_{ld},K_{ru},M_{lu}} b \otimes ac.\end{eqnarray*} Applying $(\epsilon\otimes \id)$, we find \begin{eqnarray*} (\epsilon\otimes \id)( (1\otimes a)\Delta(1)(b\otimes c) ) &=&  \delta_{L_{ld},K_{ru},M_{lu}}\delta_{L_{lu},L_{ld}}\delta_{L_{ru},L_{rd}} \epsilon(b)ac \\ &=&  \delta_{L_{ld},L_{lu},K_{ru},M_{lu}} \epsilon(b)ac,\end{eqnarray*} which agrees with the expression above.
\end{proof} 

\begin{Rem} 
Since also the expressions $\Delta(a)(b\otimes 1)$ and $(1\otimes a)\Delta(b)$ are in $A\otimes A$ for all $a,b\in A$, we see that $(A,\Delta)$ is in fact a \emph{regular} weak multiplier bialgebra \cite[Definition 2.3]{Boh1}.
\end{Rem} 

Recall from \cite[Section 3]{Boh1} that a regular weak multiplier
bialgebra admits four projections $A\rightarrow M(A)$, given
by \begin{align*} \bar{\Pi}^L(a) = (\epsilon\otimes \id)((a\otimes
  1)\Delta(1)),\quad & \bar{\Pi}^R(a) = (\id\otimes
  \epsilon)(\Delta(1)(1\otimes a)),\\ \Pi^L(a) = (\epsilon\otimes
  \id)(\Delta(1)(a\otimes 1)),\quad& \Pi^R(a) =
  (\id\otimes\epsilon)((1\otimes a)\Delta(1)),\end{align*} where the
right hand side expressions are interpreted as multipliers in the
obvious way. The relation  $\Delta(1)=\sum_{p} \rho_{p} \otimes \lambda_{p}$ and  condition (c) in Definition \ref{DefPartBiAlg} imply
\begin{align*}
  \bar \Pi^{L}(A) &=\mathrm{span}\{\lambda_{p}:p\in I\} =  \Pi^{L}(A), &
  \bar \Pi^{R}(A) &= \mathrm{span}\{\rho_{p}:p\in I\} =\Pi^{R}(A).
\end{align*}
The \emph{base algebra} of $(A,\Delta)$ is therefore just the algebra
$\Fun_{\fin}(I)$ of finite support functions on $I$, and the
comultiplication of $A$ is (left and right) \emph{full} (meaning
roughly that the legs of $\Delta(A)$ span $A$) by \cite[Theorem 3.13]{Boh1}.  

 The maps $\Pi^{L}$ and $\Pi^{R}$ can also
be written in the form
\begin{align} \label{eq:pi} 
    \Pi^L(a) & = \sum_{p}\epsilon(\lambda_{p}a)\lambda_p, & \Pi^R(a) & =    \sum_{p}\epsilon(a \rho_{p}) \rho_p
\end{align}
because $\epsilon(\lambda_{k}\rho_{m} a \lambda_{l}\rho_{n})=0$  if $(k,l)\neq(m,n)$. These relations and  \eqref{EqDel}, \eqref{eq:delta-lambda-rho} imply
\begin{align} \label{eq:pi-l-delta}
  (\Pi^{L} \otimes \id)(\Delta(a)) &= \sum_{p} \lambda_{p}\otimes \lambda_{p}a, &
  (\id \otimes \Pi^{L})(\Delta(a)) &= \sum_{p} \rho_{p}a \otimes \lambda_{p}, & \\ \label{eq:pi-r-delta}
  (\Pi^{R} \otimes \id)(\Delta(a)) &= \sum_{p} \rho_{p} \otimes a\lambda_{p}, &
  (\id \otimes \Pi^{R})(\Delta(a)) &= \sum_{p} a\rho_{p} \otimes \rho_{p}.
\end{align}

Let us now show a converse. If $(A,\Delta)$ is a regular weak multiplier bialgebra, let us write $A^L = \Pi^L(A) = \bar{\Pi}^L(A)\subseteq M(A)$ and $A^R = \Pi^R(A)= \bar{\Pi}^R(A)\subseteq M(A)$ for the base algebras, where the identities follow from \cite[Theorem 3.13]{Boh1}. Then if moreover $(A,\Delta)$ is left and right full, we have that $A^L$ is (canonically) anti-isomorphic to $A^R$ by the map \[\sigma: A^L \rightarrow A^R, \quad \bar{\Pi}^L(a) \rightarrow \Pi^R(a), \qquad a\in A,\] by \cite[Lemma 4.8]{Boh1}. We then simply refer to $A^L$ as \emph{the} base algebra. 

\begin{Rem}\label{RemNak} We could also have used the map $\bar{\sigma}(\Pi^L(a)) = \bar{\Pi}^R(a)$ to identify $A^L$ and $A^R$. As it turns out, $\bar{\sigma}^{-1}\sigma$ is the (unique) Nakayama automorphism for some functional $\varepsilon$ on $A^L$, cf. \cite[Proposition 4.9]{Boh1}. Hence if $A^L$ is commutative, it follows that $\sigma = \bar{\sigma}$.
\end{Rem} 

\begin{Prop}\label{PropCharPBA} Let $(A,\Delta)$ be a left and right full regular weak multiplier bialgebra whose base algebra is isomorphic to $\Fun_f(I)$ for some set $I$, and such that moreover $A^LA^R \subseteq A$. Then $(A,\Delta)$ is the total weak multiplier bialgebra of a uniquely determined partial bialgebra $\mathscr{A}$ over $I$.
\end{Prop} 

\begin{Rem} The condition $A^LA^R \subseteq A$ is of course essential, as we want $A$ to behave locally as a bialgebra, not a multiplier bialgebra. Indeed, in case $A^L= \C$, the condition simply says that $A$ is unital.
\end{Rem} 

\begin{proof} Let us write the standard generators (Dirac functions) of $A^L$ as $\lambda_k$ for $k\in I$, and write $\sigma(\lambda_k) = \rho_k\in A^R$. By assumption, $\UnitC{k}{l} = \lambda_k\rho_l\in A$. Further $A= AA^R = AA^L = A^LA=A^RA$, cf.~ the proof of \cite[Theorem 3.13]{Boh1}. Hence the $\UnitC{k}{l}$ make $A$ into a the total algebra of an $I\times I$-partial algebra, as $A^L$ and $A^R$ pointwise commute by \cite[Lemma 3.5]{Boh1}. 

Define \[\Delta_{rs}(a) = (\rho_r\otimes \lambda_r)\Delta(a)(\rho_s\otimes \lambda_s).\] From \cite[Lemma 3.3]{Boh1}, it follows that $\Delta_{rs}$ is a map from $\Gr{A}{k}{l}{m}{n}$ to $\Gr{A}{k}{l}{r}{s}\otimes \Gr{A}{r}{s}{m}{n}$. That same lemma, together with the coassociativity of $\Delta$, show that the $\Delta_{rs}$ form a coassociative family.  

Now by \cite[Lemma 3.9]{Boh1}, we have $(\rho_k\otimes 1)\Delta(a) = (1\otimes \lambda_k)\Delta(a)$ for all $a$. By that same lemma and Remark \ref{RemNak}, we have as well $\Delta(a)(\rho_k\otimes 1) = \Delta(a)(1\otimes \lambda_k)$. Hence we may as well write \begin{eqnarray*} \Delta_{rs}(a) &=& (\rho_r\otimes 1)\Delta(a)(\rho_s\otimes 1) \\ &=& (1\otimes \lambda_r)\Delta(a)(1\otimes \lambda_s)\end{eqnarray*}  It is now straightforward that the counit map of $(A,\Delta)$ also provides a counit for the $\Delta_{rs}$, hence the $\Gr{A}{k}{l}{m}{n}$ also form a partial coalgebra. 

As $\Delta(a)(1\otimes \lambda_s)$ and $(1\otimes \lambda_r)\Delta(a)$ are already in $A\otimes A$, it is also clear that $\Delta_{rs}(a)$ is rcf for each $a$. The multiplicativity of the $\Delta_{rs}$ is then immediate from the multiplicativity of $\Delta$.

To show that $\Delta_{ll'}(\UnitC{k}{m}) = \delta_{l,l'} \UnitC{k}{l}\otimes \UnitC{l}{m}$, it suffices to show that $\Delta(1) = \sum_k \rho_k\otimes \lambda_k$. Now as $\Delta(1)(A\otimes A)  = \Delta(A)(A\otimes A)$, and as clearly $\Delta(a) = \sum_{r,s}\Delta_{rs}(a)$ in the strict topology for all $a\in A$, it follows that \[\Delta(1) = \left(\sum_k \rho_k\otimes \lambda_k\right)\Delta(1).\]  Similarly, $\Delta(1) = \Delta(1)\left(\sum_k\rho_k\otimes \lambda_k\right)$. On the other hand, by \cite[Lemma 4.10]{Boh1} it follows that we can then write \[\sum_{k\in I'} \rho_k\otimes \lambda_k\] for some subset $I'\subseteq I$. As by definition $\bar{\Pi}^L(A) = \Fun_{\fin}(I)$, we deduce that $I=I'$. 

For $a\in \Gr{A}{k}{l}{p}{q}$ and $b\in \Gr{A}{l}{m}{q}{r}$, we then have $\epsilon(ab) = \epsilon(a\UnitC{l}{q}b) = \epsilon(a)\epsilon(b)$ by \cite[Proposition 2.6.(4)]{Boh1}, which shows the partial multiplicativity of $\epsilon$. 

Finally, assume that $k$ was such that $\epsilon(\UnitC{k}{k})=0$. Then by the partial multiplication law, $\epsilon$ is zero on all $\Gr{A}{k}{l}{k}{l}$. Applying $\Delta_{kl}$ to $\Gr{A}{k}{l}{m}{n}$ and using the counit property on the first leg, it follows that $\Gr{A}{k}{l}{m}{n}=0$ for all $l,m,n$. In particular, $\UnitC{k}{m}=0$ for all $m$. But this entails $\lambda_k=0$, a contradiction. Hence $\epsilon(\UnitC{k}{k})\neq 0$. From the partial multiplication law, it follows that $\epsilon(\UnitC{k}{k})^2 = \epsilon(\UnitC{k}{k})$, hence $\epsilon(\UnitC{k}{k})=1$.

This concludes the proof that $(A,\Delta)$ determines a partial bialgebra $\mathscr{A}$. It is immediate that $(A,\Delta)$ is in fact the total weak multiplier bialgebra of $\mathscr{A}$. 
\end{proof} 

% Mention precisely link to Hayashi

\subsection{Partial Hopf algebras}

We now formulate the notion of partial Hopf algebra, whose total form will correspond to a weak multiplier Hopf algebra \cite{Boh1,VDW2,VDW1}. We will mainly refer to \cite{Boh1} for uniformity.

 Let us denote $\circ$ for the inverse of $\wmult$, and $\bullet$ for the inverse of $\bmult$, so \[\begin{pmatrix} k & l \\ m & n \end{pmatrix}^{\circ} = \begin{pmatrix} l & k \\ n & m \end{pmatrix},\quad \begin{pmatrix} k & l \\ m & n \end{pmatrix}^{\bullet} = \begin{pmatrix} m & n \\ k & l \end{pmatrix},\quad \begin{pmatrix} k & l \\ m & n \end{pmatrix}^{\circ \bullet} = \begin{pmatrix} n & m \\ l & k \end{pmatrix}.\] The notation $\circ$ (resp. $\bullet$) will also be used for row vectors (resp. column vectors).

\begin{Def}\label{DefPartBiAlgAnt} An \emph{antipode} for an
  $I$-partial bialgebra $\mathscr{A}$ consists of linear
maps \[S:A(K)\rightarrow A(K^{\circ\bullet})\]
  such that the following property holds: for all $M,P\in M_2(I)$ and
  all $a\in A(M)$, \begin{align} \label{eq:antipode-pi-l}\underset{K\wmult
      L^{\circ\bullet}=P}{\sum_{K\bmult L = M}} a_{(1)K}S(a_{(2)L})&=
    \delta_{P_l,P_r}\epsilon(a)\mathbf{1}(P_l),
    \\ \label{eq:antipode-pi-r}
    \underset{K^{\circ\bullet}\wmult L=P}{\sum_{K\bmult L = M}}
    S(a_{(1)K})a_{(2)L}&=
    \delta_{P_l,P_r}\epsilon(a)\mathbf{1}(P_r).\end{align}

A partial bialgebra $\mathscr{A}$ is called a \emph{partial Hopf algebra} if it admits an antipode.
\end{Def} 

\begin{Rem} Note that condition \ref{Propd} of Definition \ref{DefPartBiAlg} again guarantees that the above sums are in fact finite.
\end{Rem}

If $S$ is an antipode for a partial bialgebra, we can extend $S$ to a
linear map \[S:A\rightarrow A\] on the total algebra $A$.  Conditions
\eqref{eq:antipode-pi-l} and \eqref{eq:antipode-pi-r} then take the
following simple form:
\begin{Lem} \label{lemma:antipode}
  A family of maps $S \colon A(K) \to A(K^{\circ\bullet})$ satisfies
  \eqref{eq:antipode-pi-l} and \eqref{eq:antipode-pi-r} if and only if
  the total map $S\colon A \to A$ satisfies 
  \begin{align} \label{eq:total-antipode}
 a_{(1)}S(a_{(2)}) &= \Pi^{L}(a), &
 S(a_{(1)})a_{(2)} &= \Pi^{R}(a)
  \end{align}
for all $a\in A$.
\end{Lem}

Note that these should be considered a priori as equalities of left (resp. right) multipliers on $A$.

\begin{proof}
For $M,P\in M_{2}(I)$ and $a\in A(M)$, the left and the right hand side  of \eqref{eq:antipode-pi-l} are the $P$-homogeneous components of $ a_{(1)}S(a_{(2)})$ and $\Pi^{L}(a)=\sum_{p} \epsilon(\lambda_{p}a)\lambda_{p}$, respectively.
\end{proof}

\begin{Lem}\label{LemAntiUnit} Let $\mathscr{A}$ be a partial Hopf algebra with antipode $S$. For all $k,l\in I$, $S(\UnitC{k}{l}) = \UnitC{l}{k}$.
\end{Lem}
\begin{proof} For example the first identity in Equation \eqref{eq:total-antipode} of Lemma \ref{lemma:antipode} applied to $\UnitC{k}{k}$ gives \[\sum_l S(\UnitC{l}{k}) = \sum_l \UnitC{k}{l}S(\UnitC{l}{k}) = \lambda_k,\] as $S(\UnitC{l}{k}) \in \Gr{A}{k}{k}{l}{l}$ and $\Pi^{L}(\UnitC{k}{k}) = \lambda_k$. This implies the lemma.
\end{proof} 
\begin{Rem} \label{remark:index-equivalence}
  Let $\mathscr{A}$ be an $I$-partial Hopf algebra. Then the relation
  on $I$ defined by
  \begin{align*}
    k \sim l \Leftrightarrow \UnitC{k}{l} \neq 0
  \end{align*}
is an equivalence relation. Indeed, it is reflexive and transitive by
assumptions (3) and (1) in Definition \ref{DefPartBiAlg}, and
symmetric by the preceding result. We call the set $\mathscr{I}$ of equivalence classes the \emph{hyperobject} set of $\mathscr{A}$. % Reference to Szlachanyi. 
\end{Rem}
The existence of an antipode is closely related to partial invertibility of
the maps $T_{1},T_{2} \colon A \otimes A \to A\otimes A$ given by
\begin{align} \label{eq:wt-12}
  T_{1} (a\otimes b)&= \Delta(a)(1 \otimes b), &
  T_{2} (a\otimes b)&= (a \otimes 1)\Delta(b).
 \end{align}
The precise formulation involves the linear maps $E_{i},G_{i}
 \colon A\otimes A\to A\otimes A$ given by
\begin{align} \label{eq:e1g1}
  G_{1}(a\otimes b) &=
 \sum_{p} a\rho_{p} \otimes \rho_{p}b, &  E_{1}(a \otimes b) &=\Delta(1)(a\otimes b)=\sum_{p} \rho_{p}a\otimes \lambda_{p}b, \\ \label{eq:e2g2}
 G_{2}(a \otimes b) &= \sum_{p} a\lambda_{p} \otimes
    \lambda_{p}b, &
E_{2}(a\otimes b) &= (a\otimes b)\Delta(1)=\sum_{p} a\rho_{p} \otimes b\lambda_{p}.
\end{align}
\begin{Prop} \label{prop:riti}
  Let $\mathscr{A}$ be a partial Hopf algebra with total algebra $A$,
  total comultiplication $\Delta$ and antipode  $S$. Then the maps
  $R_{1},R_{2} \colon A \otimes A \to M(A \otimes A)$ given by
  \begin{align*}
    R_{1}(a \otimes b) &= a_{(1)}\otimes S(a_{(2)})b, &
    R_{2}(a\otimes b) &= aS(b_{(1)})\otimes b_{(2)}
  \end{align*}
  take values in $A\otimes A$ and satisfy for $i=1,2$ the relations
  \begin{align} \label{eq:riti}
    T_{i}R_{i}&=E_{i}, & R_{i}T_{i}&= G_{i}, & T_{i}R_{i}T_{i}&= T_{i}, & R_{i}T_{i}R_{i} &= R_{i}.
  \end{align}
\end{Prop}
\begin{proof}
  The map $R_{1}$ takes values in $A\otimes A$ because
  \begin{align*}
  a_{(1)} \otimes
  S(a_{(2)})\lambda_{k}\rho_{l} =  a_{(1)} \otimes S(\rho_{l}\lambda_{k}a_{(2)}) \in A
  \otimes A
  \end{align*}
  for all $a\in A$, and Lemma
  \ref{lemma:antipode},  Equation \eqref{eq:pi-l-delta} and Lemma \ref{LemAntiUnit} imply
  \begin{align*}
    T_{1}R_{1}(a \otimes b)&= a_{(1)} \otimes a_{(2)}S(a_{(3)})b =
    a_{(1)} \otimes \Pi^{L}(a_{(2)})b =
    \sum_{p} \rho_{p}a \otimes \lambda_{p}b, \\
    R_{1}T_{1}(a \otimes b) &= a_{(1)} \otimes S(a_{(2)})a_{(3)}b =
    a_{(1)} \otimes \Pi^{R}(a_{(2)})b = \sum_{p} a\rho_{p}\otimes
    \rho_{p}b.
  \end{align*}
 The relations $T_{1}R_{1}T_{1}=T_{1}$ and
$R_{1}T_{1}R_{1}=R_{1}$ follow easily from  \eqref{EqDel} and
\eqref{eq:delta-lambda-rho}. The assertions concerning $R_{2}$ and
$T_{2}$ follow similarly.
\end{proof}
\begin{Theorem}  \label{theorem:partial-hopf-algebra}
  Let $\mathscr{A}$ be a partial bialgebra with total algebra $A$,
  total comultiplication $\Delta$ and counit $\epsilon$. Then the
  following conditions are equivalent:
  \begin{enumerate}[label={(\arabic*)}]
  \item\label{tph1} $\mathscr{A}$ is a partial Hopf algebra.
  \item\label{tph2} There exist linear maps $R_{1},R_{2} \colon A\otimes A\to
    A\otimes A$ satisfying  \eqref{eq:riti}.
  \item\label{tph3} $(A,\Delta,\epsilon)$  is a regular weak multiplier Hopf algebra in the sense of \cite{VDW1}.
  \end{enumerate}
  If these conditions hold, then the total  antipode of $\mathscr{A}$ coincides with the antipode of $(A,\Delta,\epsilon)$.
\end{Theorem}
\begin{proof}
\ref{tph1} implies \ref{tph2} by Proposition \ref{prop:riti}. \ref{tph2} is equivalent to \ref{tph3} by Definition
1.14 in \cite{VDW1}. Indeed, the maps $G_{1},G_{2}$ defined in \eqref{eq:e1g1} and \eqref{eq:e2g2} satisfy
\begin{align*}
  G_{1}(a_{(1)} \otimes b) \otimes a_{(2)}c &= \sum_{p} a_{(1)} \otimes \rho_{p}b
  \otimes a_{(2)}\lambda_{p}c, \\
  ac_{(1)} \otimes G_{2}(b\otimes c_{(2)}) &=\sum_{p} a\rho_{p}c_{(1)} \otimes b\lambda_{p} \otimes c_{(2)}
\end{align*}
and therefore coincide with the maps introduced in Proposition 1.14 in
\cite{VDW1}.  Finally, assume \ref{tph3}. Then
 Lemma 6.14 and equation (6.14) in \cite{Boh1} imply that the antipode
$S$ of $(A,\Delta)$ satisfies $S(A(K))\subseteq A(K^{\circ\bullet})$ and relation \eqref{eq:total-antipode}.  Now, Lemma \ref{lemma:antipode} implies \ref{tph1}.
\end{proof}

From \cite[Proposition 3.5 and Proposition 3.7]{VDW1} or \cite[Theorem
6.12 and Corollary 6.16]{Boh1}, we can conclude that the antipode of a
partial Hopf algebra reverses the multiplication and
comultiplication. Denote by $\Delta^{\op}$ the composition of
$\Delta$ with the flip map.

\begin{Cor} \label{corollary:antipode} Let $\mathscr{A}$ be a partial
  Hopf algebra. Then the total antipode $S:A\rightarrow A$ is uniquely determined and satisfies
  $S(ab) = S(b)S(a)$ and $\Delta(S(a)) = (S\otimes S)\Delta^{\op}(a)$
  for all $a,b\in A$.
\end{Cor} 
\begin{proof} Uniqueness of the antipode follows from the identities \eqref{eq:total-antipode}, see also \cite[Remark 2.8.(ii)]{VDW1}. 
\end{proof} 

We will need the following relation between $\epsilon$ and $S$ at some point.

\begin{Lem}\label{LemCoAnt} Let $(\mathscr{A},\Delta)$ be a partial Hopf algebra. Then $\epsilon\circ S = \epsilon$ on each $\Gr{A}{k}{l}{m}{n}$.
\end{Lem}

\begin{proof} Using the notation in Proposition \ref{prop:riti} and the discussion preceding it, we have that \[T_1: \sum_p(A\rho_p\otimes \rho_p A)\rightarrow \Delta(1)(A\otimes A)\] is a bijection with $R_1$ as inverse. As one easily verifies that $(\id\otimes \epsilon)T_1 = \id\otimes \epsilon$ by the partial multiplicativity and counit property of $\epsilon$, it follows that also $(\id\otimes \epsilon)R_1 = \id\otimes \epsilon$ on $\Delta(1)(A\otimes A)$. Applying both sides to $a\otimes \UnitC{k}{k}$ with $a\in \Gr{A}{k}{l}{k}{l}$, we find \[(\id\otimes (\epsilon\circ S))\Delta_{kl}(a) = a.\] Applying $\epsilon$ to this identity, we find $\epsilon\circ S = \epsilon$ on each $\Gr{A}{k}{l}{k}{l}$, and hence on all $\Gr{A}{k}{l}{m}{n}$.
\end{proof} 


In practice, it is convenient to have an \emph{invertible} antipode around. Although the invertibility often comes for free in case extra structure is around, we will mostly just impose it to make life easier. The following definition follows the terminology of \cite{VDae1}. 

\begin{Def} Let $\mathscr{A}$ be a partial Hopf algebra. We call $\mathscr{A}$ a \emph{regular} partial Hopf algebra if the antipode maps on $\mathscr{A}$ are invertible.
\end{Def}

From the uniqueness of the antipode, it follows immediately that $S^{-1}$ is then an antipode for $(\mathscr{A},\Delta^{\op})$. Conversely, if both $(\mathscr{A},\Delta)$ and $(\mathscr{A},\Delta^{\op})$ have antipodes, then $(\mathscr{A},\Delta)$ is a regular partial Hopf algebra. 

\subsection{Invariant integrals}

% Mention precisely link to Hayashi

\begin{Def}
  Let $\mathscr{A}$ be an $I$-partial bialgebra.  We call a family of
  functionals
\begin{align} \label{eq:functionals}
  \phic{k}{m} \colon A\pmat{k}{k}{m}{m} \to \C
\end{align}
a \emph{left invariant} \emph{integral} if
 $\phic{k}{k}(\UnitC{k}{k})=1$ for all $k\in
I$ and
\begin{align}
  \label{eq:integral}
   (\id \otimes \phic{l}{m})(\Delta_{ll}(a)) 
&= \delta_{k,p} \phic{k}{m}(a)
  \UnitC{k}{l} 
\end{align}
 for all $k,l,m,p\in I$, $a \in A\pmat{k}{p}{m}{m}$. 
 
 We call them a \emph{right invariant}  \emph{integral} if instead one has \begin{align}
  (\phic{k}{l} \otimes
  \id)(\Delta_{ll}(a))&= \delta_{m,p} \phic{k}{m}(a) \UnitC{l}{m}\end{align}
 for all $k,l,m,p\in I$, $a \in A\pmat{k}{k}{m}{p}$. 
 
 A left integral which is at the same time a right invariant integral will simply be called an \emph{invariant integral}.
\end{Def}

As before, we can extend a (left or right) invariant integral to a functional $\phi$ on $A$ by linearity and by putting $\phi=0$ on $\Gr{A}{k}{l}{m}{n}$ if $k\neq l$ or $m\neq n$. The total form of the invariance conditions
\eqref{eq:integral}  reads as follows. 

\begin{Lem} \label{lemma:total-integral}
  A family of functionals  as in   \eqref{eq:functionals}
  is left invariant
  if and only if
for all $a,b\in A$,
  \begin{align*}
(\id\otimes \phi)((b\otimes 1)\Delta(a)) &= \sum_{k}\phi(\lambda_{k}a)b\lambda_k.
      \end{align*}
      It defines a right invariant functional if and only if 
   \begin{align*}   (\phi\otimes \id)(\Delta(a)(1\otimes b)) &= \sum_{n}
\phi(\rho_{n} a)\rho_n b.\end{align*}
\end{Lem}

\begin{proof}
  Straightforward.
\end{proof}

We have the following form of \emph{strong invariance}.

\begin{Lem} \label{lemma:strong-invariance}
  Let $\mathscr{A}$ be a partial Hopf algebra with left invariant integral $\phi$. Then
  for all $a\in A$,
  \begin{align*}
    S\left(( \id\otimes
    \phi)(\Delta(b)(1 \otimes a))\right) &= (\id \otimes \phi)((1 \otimes b)\Delta(a)).
  \end{align*}
  Similarly, if $\mathscr{A}$ is a partial Hopf algebra with right invariant integral $\phi$, then 
   \begin{align*} S\left((\phi \otimes
    \id)(\Delta(a)(1\otimes b))\right) &= (\phi \otimes \id)(\Delta(a)(b \otimes 1)).\end{align*}
\end{Lem}
\begin{proof}
 The counit property, the relations \eqref{EqDel} and
 \eqref{eq:total-antipode} and Lemma \ref{lemma:total-integral} imply
  \begin{align*}
    a_{(1)}\phi(ba_{(2)}) &= \sum_{n}
    a_{(1)}\phi(\epsilon(b_{(1)}\rho_{n})b_{(2)}\lambda_{n}a_{(2)}) \\
&= \sum_{n} \epsilon(b_{(1)}\rho_{n})\rho_{n}a_{(1)}\phi(b_{(2)}a_{(2)})
\\
&= S(b_{(1)})b_{(2)}a_{(1)}\phi(b_{(3)}a_{(2)}) =
S(b_{(1)})\phi(b_{(2)}a)
  \end{align*}
for all $a,b \in A$. The second equation 
follows similarly.
\end{proof}

\begin{Lem} Assume that $\mathscr{A}$ is a regular  $I$-partial Hopf algebra which admits a left invariant integral $\phi$. Then the following hold.
\begin{enumerate}[label = {(\arabic*)}]
\item\label{LI1} $\phi(\UnitC{k}{m})=1$ for all $k,m\in I$ with $\UnitC{k}{m}\neq 0$.
\item\label{LI2} $\phi$ is uniquely determined.
\item\label{LI3} $\phi=\phi S$.
\item\label{LI4} $\phi$ is invariant.
\end{enumerate}
\end{Lem}

\begin{proof} 
To see \ref{LI1}, take $a=\UnitC{k}{k}$ in \eqref{eq:integral}. 

Now by Corollary \ref{corollary:antipode}, we have that $\phi S$ is right invariant. But assume that $\psi$ is any
 right invariant integral.     Then for all $k,l,m\in I$, $a\in A\pmat{k}{k}{m}{m}$,
    \begin{align*}
      \phic{k}{m}(a)  &= (\Grt{\psi}{k}{k} \otimes
      \phic{k}{m})(\Delta_{kk}(a)) = \Grt{\psi}{k}{m}(a)\Grt{\phi}{k}{m}(\UnitC{k}{m}) = \Grt{\psi}{k}{m}(a) .
    \end{align*}
  This proves \ref{LI2}, \ref{LI3} and \ref{LI4}.  
     \end{proof} 


We will need the following lemma at some point, cf.~ \cite[Proposition 3.4]{VDae2}.

\begin{Lem}\label{LemFaith} Let $\mathscr{A}$ be a regular partial Hopf
  algebra with an invariant integral $\phi$. Then
  $\phi$ is faithful in the following sense: if $a\in A$ and
  $\phi(ab) =0$ (resp. $\phi(ba)=0$) for all $b\in A$, then
  $a=0$.
\end{Lem} 

\begin{proof} Suppose $a\in A$ and $\phi(ba)=0$ for all $b\in A$. By the support condition of $\phi$, we may suppose $a$ is homogeneous, $a\in \Gr{A}{k}{l}{m}{n}$.

We will first show that necessarily $\epsilon(a)=0$, for which we may already assume $k=m$ and $l=n$. Indeed, the condition on $a$ implies also $(\id\otimes \phi)(\Delta_{lk}(b)(1\otimes a))=0$ for all $b\in \Gr{A}{s}{r}{l}{k}$. Applying the strong invariance identity, we deduce \begin{equation}\label{EqSwitch}(\id\otimes \phi)((1\otimes b)\Delta_{rs}(a))=0,\qquad \forall b\in \Gr{A}{s}{r}{l}{k}.\end{equation} Writing $\Delta_{rs}(a) = \sum_i p_i\otimes q_i$ with the $p_i$ linearly independent, we deduce $\phi(bq_i)=0$ for all $i$ and $b$, and so also $\sum_i \phi(S(p_i)q_i)=0$. Hence $0=\sum_r \phi(S(a_{(1){\tiny \begin{pmatrix} k & l \\ r & l \end{pmatrix}}})a_{(2){\tiny \begin{pmatrix} r & l \\ k & l\end{pmatrix}}}) = \phi(\epsilon(a) \UnitC{l}{l}) = \epsilon(a)$.

Note now that from \eqref{EqSwitch}, it follows that for any functional $\omega$ on $\Gr{A}{k}{l}{m}{n}$, also $a'=(\omega\otimes \id)\Delta_{mn}(a)$ satisfies $\phi(ba')=0$ for all $b\in A$. Hence, by what we have just shown, $\epsilon(a')=0$, i.e.~ $\omega(a)=0$. As $\omega$ was arbitrary, we deduce $a=0$.

The other case follows similarly, or by considering the opposite comultiplication.
\end{proof} 



\subsection{Partial compact quantum groups}

We now turn towards the structures which will allow us to build operator algebraic quantum groupoids out of our partial Hopf algebras (see Section 7).
 
\begin{Def} A \emph{partial $*$-algebra} $\mathscr{A}$ is a partial
  algebra whose total algebra $A$ is equipped with an antilinear,
  antimultiplicative involution $*\colon A\rightarrow A$, $ a\mapsto
  a^*$,  such that the $\mathbf{1}_k$ are selfadjoint for all $k$ in
  the object set. 
\end{Def} 

One can of course give an alternative definition directly in terms of the partial algebra structure by requiring that we are given antilinear maps $A(k,l)\rightarrow A(l,k)$ satisfying the obvious antimultiplicativity and involution properties.

\begin{Def} A \emph{partial $*$-bialgebra} $\mathscr{A}$ is a
 partial bialgebra whose underlying partial algebra has been
  endowed with a partial $*$-algebra structure such that
$\Delta_{rs}(a)^* = \Delta_{sr}(a^*)$ for all $a \in \Gr{A}{k}{l}{m}{n}$.
A \emph{partial Hopf $*$-algebra} is a partial bialgebra which is at the same time a partial $*$-bialgebra and a partial Hopf algebra.
\end{Def} 
Thus, a partial bialgebra is a partial
$*$-bialgebra if and only if the underlying weak multiplier bialgebra
 is a weak multiplier $*$-bialgebra.

From Theorem \ref{theorem:partial-hopf-algebra} and \cite{Boh1},
\cite{VDW1}, we can deduce:
\begin{Cor} \label{cor:involutive}
  An $I$-partial $*$-bialgebra $\mathscr{A}$ is an $I$-partial Hopf
  $*$-algebra if and only if the weak multiplier $*$-bialgebra
  $(A,\Delta)$ is a weak multiplier Hopf $*$-algebra. In that case,
  the counit and antipode satisfy
  $\epsilon(a^{*})=\overline{\epsilon(a)}$ and $S(S(a)^{*})^{*}=a$ for
  all $a\in A$. In particular, the total antipode is bijective.
\end{Cor}
\begin{proof}
  The if and only if part follows immediately from  Theorem
  \ref{theorem:partial-hopf-algebra}, the relation for the counit  from
uniqueness of the counit  \cite[Theorem 2.8]{Boh1}, and the relation
for the antipode from \cite[Proposition 4.11]{VDW1}.
\end{proof}


We are finally ready to formulate our main definition.
\begin{Def} A \emph{partial compact quantum group} $\mathscr{G}$ is a
  partial Hopf $*$-algebra $\mathscr{A} = P(\mathscr{G})$ with an invariant integral  $\phi$ that is positive in the sense  that $\phi(a^*a)\geq 0$ for all $a\in A$. We also say that $\mathscr{G}$ is the partial compact quantum group \emph{defined by} $\mathscr{A}$.
\end{Def} 

\begin{Rem} Following \cite{Hay1}, we could also have called our objects \emph{compact quantum groups of face type}, but we feel this gives the wrong impression when the base algebra is infinite dimensional (i.e.~ the object set is not compact). When referring to partial compact quantum groups, we feel that it is better reflected that only the \emph{parts} of this object are to be considered compact, not the total object. 
\end{Rem} 

%%% Local Variables: 
%%% mode: latex
%%% TeX-master: "dyn-suq-main"
%%% End: 
