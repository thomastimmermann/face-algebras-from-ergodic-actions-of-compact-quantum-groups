\section{Partial compact quantum groups}

We generalize Hayashi's definition of a compact quantum group of face type \cite{Hay1} to the case where the commutative base algebra is no longer finite-dimensional. We will present two approaches, based on \emph{partial bialgebras} and \emph{weak multiplier bialgebras} \cite{Boh1}. The first approach is piecewise and concrete, but requires some bookkeeping. The second approach is global but more abstract. As we will see from the general theory and the concrete examples, both approaches have their intrinsic value.

%\begin{Not} If $I$ is a set, we write $\Fun_{\fin}(I)$ for the algebra of \emph{finitely supported} $\C$-valued functions on $I$\\

%We write $\Fun(I)$ for the algebra of \emph{all} $\C$-valued functions on $I$. %We will always suppose that $I$ is at most countable.
%\end{Not}

% Notation $I$ ok, or better $O$?

Let $I$ be a set. We consider $I^2=I\times I$ as the pair groupoid with $\wmult$ denoting composition. That is, an element $K=(k,l)\in I^2$ has source $K_l = k$ and target $K_r=l$, and if $K=(k,l)$ and $L=(l,m)$ we write $K\wmult L = (k,m)$. %For general $K,L\in I^2$, we write the property `$K$ and $L$ are composable' as $K\wmate L$. 

\begin{Def} A \emph{partial algebra} $\mathscr{A}=(\mathscr{A},M)$ (over $\C$) is a small $\C$-linear category, that is, a set $I$ (the object set) together with %Change partial by face? Partial algebra might have distinct meaning
\begin{itemize}
\item[$\bullet$] for each $K=(k,l)\in I^2$ a vector space $A(K) = \Grs{A}{k}{l}=\!\!\GrDA{A}{k}{l}$ (possibly the zero vector space),
\item[$\bullet$] for each $K,L$ with $K_r = L_l$ a multiplication map \[M(K,L):A(K) \otimes A(L)\rightarrow A(K\cdot L),\qquad a\otimes b \mapsto ab\]  and 
\item[$\bullet$] elements $\Unit(k) = \Unit_k \in \Grs{A}{k}{k}$ (the units), % or the local units?
\end{itemize}
such that the obvious associativity and unit conditions are satisfied. 

By \emph{$I$-partial algebra} will be meant a partial algebra with object set $I$.
\end{Def}

\begin{Rem}We allow the local units $\Unit_k$ to be zero. %When none of the units are zero, we will call the partial algebra \emph{connected}. 
\end{Rem}

Let $\mathscr{A}$ be an $I$-partial algebra. We define $A(K\wmult L)$ to be $\{0\}$ when $K\wmult L$ is ill-defined, i.e. $K_r\neq L_l$. We then let $\Grs{M}{K}{L}$ be the zero map.

\begin{Def} The \emph{total algebra} $A$ of an $I$-partial algebra $\mathscr{A}$ is the vector space \[A = \oplus_{K\in I^2} A(K)\] endowed with the unique multiplication whose restriction to $A(K)\otimes A(L)$ concides with $M(K,L)$.  
\end{Def} 
% Should a reference to the notion of locally unital algebra be given? I had a reference to Quillen, but no precise article, and I don't seem to find the link anymore...
Clearly $A$ is an associative algebra. If $I$ is infinite it will not possess a unit, but it is a \emph{locally unital algebra} as there exist mutually orthogonal idempotents $\mathbf{1}_k$ with $A = \osum{k,l} \mathbf{1}_kA\mathbf{1}_l$. An element $a\in A$ can be interpreted as a function assigning to each element $(k,l)\in I^2$ an element $a_{kl}\in A(k,l)$, namely the $(k,l)$-th component of $a$. This identifies $A$ with finite support $I$-indexed matrices whose $(k,l)$-th entry lies in $A(k,l)$, equipped with the natural matrix multiplication. 

\begin{Rem}\label{RemGrad} When $\mathscr{A}$ is an $I$-partial algebra with total algebra $A$, then $A\otimes A$ can be naturally identified with the total algebra of an $I\times I$-partial algebra $\mathscr{A}\otimes \mathscr{A}$, where \[(A\otimes A)((k,k'),(l,l')) = A(k,l)\otimes A(k',l')\] with the obvious tensor product multiplications and the $\Unit_{k,k'} = \Unit_k\otimes \Unit_{k'}$ as units. 
\end{Rem} 

%There are two natural notions of morphism between partial algebras, functors and co-functors. %Different name? How natural are these?
%We will need the notion of \emph{relator} between partial algebras. When $R\subseteq I\times J$ is a relation, we write, for $k\in I$, $R_k= \{l'\in J\mid (k',l')\in R\}$. We write $D_R = \{k\in I\mid \#R_k\neq0\}$.  
%A \emph{functor} between an $I$-partial algebra $\mathscr{A}$ and $J$-partial algebra $\mathscr{B}$ consists of a map $F:I\rightarrow J$ and linear maps $F_{k,l}:A(k,l)\rightarrow B(F(k),F(l))$ such that the $F_{k,l}$ respect the algebra and unit maps in the obvious ways. 

%\begin{Def} A \emph{relator} between an $I$-partial algebra $\mathscr{A}$ and $J$-partial algebra $\mathscr{B}$ consists of a relation $F\subseteq I\times J$ and maps $F_{k',l'}:A(k,l)\rightarrow B(k',l')$ for $k'\in R_{k},l'\in R_{l}$ such that the following conditions hold.
%\begin{itemize}
%\item For all $k,l\in I$ and $a\in A(k,l)$, the $F_{k',l'}(a)$ are zero for almost all $k'\in R_{k}$ (resp. all $l'\in R_{l}$) when $l'$ (resp. $k'$) is fixed.  
%\item For all $a\in A(k,l)$ and $b\in A(l,m)$, and all $k'\in R_{k}, m'\in R_{m}$, one has \[F_{k',m'}(ab) = \sum_{l',l'\in R_{l}} F_{k',l'}(a)F_{l',m'}(b).\] 
%\item $F_{k',l'}(e_{k}) = \delta_{k',l'}e_{k'}$ for all $k'\in R_{k}$.  
%\end{itemize}
%\end{Def}

The notion of partial algebra dualizes. For this we consider again $I^2$ as the pair groupoid, but now with elements considered as column vectors, and with $\bmult$ denoting the (vertical) composition. So $K=\Grt{}{k}{l}$ has source $K_u = k$ and target $K_d = l$, and if $K=\Grt{}{k}{l}$ and $L=\Grt{}{l}{m}$ then $K\bmult L = \Grt{}{k}{n}$. %We write $K\bmate L$ if $K$ and $L$ are composable. 

%Then we have maps  \[\Grt{\Delta}{K}{L}: A(K\bmult L)\rightarrow A(K)\otimes A(L),\] which we interpret as zero maps when $\neg K\bmate L$. We also interpret $\Grt{\Delta}{K}{L}$ as the zero map on $A(M)$ if $M\neq K\bmult L$. The coassociativity condition can now be written  \[(\id\otimes \Grt{\Delta}{L}{M})\Grt{\Delta}{K}{L\bmult M} = ( \Grt{\Delta}{K}{L}\otimes \id)\Grt{\Delta}{K\bmult L}{M}.\]
% Should provide shortcut for notation where only the new indices appear at $\Delta$ (so disregard the source): write $\Delta(K over L)$ as $Delta_{rs}$ for $r=K_{ld}$ and $s=K_{rd}$. 
\begin{Def} A \emph{partial coalgebra} $\mathscr{A}=(\mathscr{A},\Delta)$ (over $\C$) consists of a set $I$ (the object set) together with 
\begin{itemize}
\item[$\bullet$] for each $K=\Grru{k}{l}\in I^2$ a vector space $A(K) = \Grt{A}{k}{l}=\!\!\GrRA{A}{k}{l}$,
\item[$\bullet$] for each $K,L$ with $K_d = L_u$ a comultiplication map \[\Grt{\Delta}{K}{L}:A(K*L)\rightarrow A(K)\otimes A(L),\qquad a \mapsto a_{(1)K}\otimes a_{(2)L},\] and 
\item[$\bullet$] counit maps $\epsilon_k:\Grt{A}{k}{k}\rightarrow \C$,
\end{itemize} 
satisfying the obvious coassociativity and counitality conditions.

By \emph{$I$-partial coalgebra} will be meant a partial coalgebra with object set $I$.
\end{Def}

\begin{Not}\label{NotCom} As the index of $\epsilon_k$ is determined by the element to which it is applied, there is no harm in dropping the index $k$ and simply writing $\epsilon$.

Similarly, if $K = \Grt{}{k}{l}$ and $L = \Grt{}{l}{m}$, we abbreviate $\Delta_l = \Grt{\Delta}{K}{L}$, as the other indices are determined by the element to which $\Delta_l$ is applied.
\end{Not}

We also make again the convention that $A(K*L)=\{0\}$ and $\Grt{\Delta}{K}{L}$ the zero map when $K_d \neq L_u$. Similarly $\epsilon$ is seen as the zero functional on $A(K)$ when $K=\Grt{}{k}{l}$ with $k\neq l$. 

%The $\Grt{A}{k}{l}$ can be combined into a comultiplication \[\Delta: \Prod_{K} A(K)\rightarrow \Prod_{L,M} \left(A(L)\otimes A(M)\right)\] which is coassociative in the obvious way. In the following, we will write elements of direct products as infinite sums.

We can now superpose the notions of partial algebra and partial coalgebra. To formulate the condition that the coalgebra maps form a `morphism of partial algebras', we will need to impose a finiteness condition which is automatically satisfied when the cardinality of $I$ is finite.
% Formally, a finite morphism between an $I$-partial algebra $\mathscr{A}$ and $J$-partial algebra $\mathscr{B}$ equipped with an injection $\phi$ of a subset of $J$ into $I$ consists of maps $F_{K,L}:A(\phi(K),\phi(L))\rightarrow B(K,L)$ for $K,L$ in the domain of $\phi$, such that the map $(K,L)\mapsto F_{K,L}(a)$ is finite in rows and columns when the values of $K,L$ are restricted to one fiber of$ \phi$, and such that then $F_{K,M}(ab) = \sum_L F_{K,L}(a)F_{L,M}(b)$. 
% Should this more general notion be included?

Let $I$ be a set, and let $M_2(I)$ be the set of 4-tuples of elements of $I$ arranged as 2$\times$2-matrices. We can endow $M_2(I)$ with two compositions, namely $\cdot$ (viewing $M_2(I)$ as a row vector of column vectors) and $*$ (viewing $M_2(I)$ as a column vector of row vectors). When $K\in M_2(I)$, we will write $K = \Grs{}{K_l}{K_r} = \Grt{}{K_u}{K_d} = \eGr{}{K_{lu}}{K_{ru}}{K_{ld}}{K_{rd}}$. One can view $M_2(I)$ as a double groupoid, and in fact as a \emph{vacant} double groupoid in the sense of \cite{AN1}. %, and many of the constructions below work with $M_2(I)$ replaced by an arbitrary double groupoid, if its vacant!

In the following, a vector $(r,s)$ will sometimes be written simply as $r,s$ (without parentheses) or $rs$ in an index. We also follow Notation \ref{NotCom}, but the reader should be aware that the index of $\Delta$ will now be a 1$\times$2 vector in $I^2$ as we will work with partial coalgebras over $I^2$.% Put less awkardly.
%\end{document}
\begin{Def}\label{DefPartBiAlg} A \emph{partial bialgebra} $\mathscr{A}=(\mathscr{A},M,\Delta)$ consists of a set $I$ and a collection of vector spaces $A(K)$ for $K\in M_2(I)$ such that 
\begin{itemize}
\item[$\bullet$] the $\Grs{A}{K_l}{K_r}$ form an $I^2$-partial algebra,
\item[$\bullet$] the $\Grt{A}{K_u}{K_d}$ form an $I^2$-partial coalgebra,
\end{itemize} 
and for which the following compatibility relations are satisfied.
\begin{enumerate}[label=(\alph*)]
\item\label{Propa} (Comultiplication of Units) For all $k,l,m\in I$, one has 
% $K = \begin{pmatrix} k & k\\p & p \end{pmatrix}$ and $L =\begin{pmatrix} p &p \\ l & l\end{pmatrix}$, one has 
\[\Delta_{l,l}(\UnitC{k}{m}) = \UnitC{k}{l}\otimes \UnitC{l}{m}.\]  
\item\label{Propb} (Counit of Multiplication) For all $K,L\in M_2(I)$ with $K_r = L_l$ and all $a\in A(K)$ and $b\in A(L)$, \[\epsilon(ab) = \epsilon(a)\epsilon(b).\]% Subtlety is that you already have to know composition to be able to apply this rule.
\item\label{Propc} (Non-degeneracy) For all $k\in I$, $\epsilon(\UnitC{k}{k})=1$. 
\item\label{Propd} (Finiteness) For each $K\in M_2(I)$ and each $a\in A(K)$, the element $\Delta_{rs}(a)$ is zero except for a finite number of indices $r$ (resp. $s$) when $s$ (resp. $r$) is fixed.
%\[\Grt{\Delta}{K}{L}(a)(1\otimes b) \neq0 \quad \textrm{or} \quad\Grt{\Delta}{K}{L}(a)(b\otimes 1) \neq 0.\]
% We encode regularity this way: if asymmetrically we ask finiteness only for $r$ fixed, we are in the non-regular situation. (?)
\item\label{Prope} (Comultiplication is multiplicative) For all $a\in A(K)$ and $b\in A(L)$ with $K_r= L_l$,  \[\Delta_{rs}(ab) = \sum_t \Delta_{rt}(a)\Delta_{ts}(b).\]
\end{enumerate}
\end{Def}

% To do: introduce notion of connectedness which should assure the trivial corepresentation is irreducible. This should correspond to the dual notion of pureness in [Bohm-Nill-Szlachanyi], i.e. to the left and right base algebras having $\C$ as intersection. I suspect this is the same as proving $\UnitC{k}{l}\neq 0$ for all $k,l$ (maybe in the presence of antipode integrals), but could not prove it immediately. Update: indeed, under the presence of an antipode the relation $k\sim l$ iff $\Unit{k}{l}\neq 0$ gives an equivalence relation, and then fixing a non-trivial class $P$ and putting $c_k = \delta_{k\in P}, d_l = \delta_{l\in P}$, we obtain a non-trivial element $\sum_k c_k\lambda_k = \sum_l d_l \rho_l$ in the intersection.

\begin{Rem} By assumption \ref{Propd}, the sum on the right hand side in condition \ref{Prope} is well-defined. 
\end{Rem}

%\begin{Rem}By assumption \ref{Propd}, the sum on the right hand side in condition \ref{Prope} is well-defined. We can write it less pedantically as \[\Delta_{rs}(ab) = \sum_t \Delta_{rt}(a)\Delta_{ts}(b),\] where we abbreviate $\Delta_{rs} = \Delta_{(r,s)}$.
%\end{Rem}

We want to relate the notion of partial bialgebra to the recently introduced notion of weak multiplier bialgebra \cite{Boh1}. We first recall some notions concerning non-unital algebras \cite{Dau1,VDae1}.

\begin{Def} Let $A$ be an algebra over $\C$, not necessarily with unit. We call $A$ \emph{non-degenerate} if $A$ is faithfully represented on itself by left and right multiplication. It is called \emph{idempotent} if $A^2 = A$. 
\end{Def}

\begin{Def} Let $A$ be an algebra. A \emph{multiplier} $m$ for $A$ consists of a couple of maps \begin{eqnarray*} L_m:A\rightarrow A,\quad a\mapsto ma\\ R_m:A\rightarrow A,\quad a\mapsto am\end{eqnarray*} such that $(am)b = a(mb)$ for all $a,b\in A$. 

The set of all multipliers forms an algebra under composition for the $L$-maps and anti-composition for the $R$-maps. It is called the \emph{multiplier algebra} of $A$, and is denoted $M(A)$.
\end{Def}

One has a natural homomorphism $A\rightarrow M(A)$. When $A$ is non-degenerate,  this homomorphism is injective, and we can then identify $A$ as a subalgebra of the (unital) algebra $M(A)$. We then also have inclusions \[A\otimes A\subseteq M(A)\otimes M(A)\subseteq M(A\otimes A).\]

\begin{Exa}\label{ExaMult} \begin{enumerate}
\item Let $A$ be the total algebra of an $I$-partial algebra $\mathscr{A}$. As $A$ has local units, it is non-degenerate and idempotent. Then one can identify $M(A)$ with \[M(A) = \left(\prod_l \oplus_k A(k,l)\right) \bigcap \left(\prod_k\oplus_l A(k,l)\right) \subseteq \prod_{k,l} A(k,l),\] i.e. with the space of functions \[m:I^2\rightarrow A,\quad m_{kl}\in A(k,l)\] which have finite support in either one of the variables when the other variable has been fixed. The multiplication is given by the formula \[(mn)_{kl} = \sum_p m_{kp}n_{pl}.\]
\item Let $m_i$ be any collection of multipliers of $A$, and assume that for each $a\in A$, $m_ia =0$ for almost all $i$, and similarly $am_i=0$ for almost all $i$. Then one can define a multiplier $\sum_i m_i$ in the obvious way by termwise multiplication. One says that the sum $\sum_i m_i$ converges in the \emph{strict} topology. 
\end{enumerate}
\end{Exa}

Using the notion introduced in Example \ref{ExaMult}.2, we can introduce the following notation.

\begin{Not}
If $\mathscr{A}$ is an $I$-partial bialgebra, we write \[\lambda_k = \sum_l \UnitC{k}{l},\qquad \rho_l = \sum_k\UnitC{k}{l} \qquad \in M(A).\]
\end{Not}

\begin{Rem} From Property \ref{Propc} of Definition \ref{DefPartBiAlg}, it follows that $\lambda_k\neq 0\neq \rho_k$ for any $k\in I$.
\end{Rem} 

To show that the total algebra of a partial bialgebra becomes a weak multiplier bialgebra, we will need some easy lemmas. 
%In the following, we will use the grading for tensor products as in Remark \ref{RemGrad}, coupled with the multiplier interpretation as in Example \ref{ExaMult}.

\begin{Lem} Let $\mathscr{A}$ be an $I$-partial bialgebra. Then for each $a\in A$, there exists a unique multiplier $\Delta(a) \in M(A\otimes A)$ such that \begin{eqnarray}\label{EqDel} \Delta_{rs}(a) &=& (1\otimes \lambda_r)\Delta(a)(1\otimes \lambda_s) \\ &=& (\rho_r\otimes 1)\Delta(a)(\rho_s\otimes 1)\end{eqnarray}  for all $r,s\in I$, all $K\in M_2(I)$ and all $a\in A(K)$. 

The resulting map \[\Delta:A\rightarrow M(A\otimes A),\quad a\mapsto \Delta(a)\] is a homomorphism.
\end{Lem}
\begin{proof} For $a\in A$ homogeneous, we can define $\Delta(a) = \sum_{rs} \Delta_{rs}(a) \in M(A\otimes A)$, where the sum converges in the strict topology of $A\otimes A$ because of the property \ref{Propd} of Definition \ref{DefPartBiAlg}. This expression clearly satisfies the identities stated in the lemma. In turn, these identities uniquely define $\Delta(a)$ as a multiplier, as they determine the value of $\Delta(a)$ when cut down to the left and right with the local units of $\mathscr{A}\otimes \mathscr{A}$.

We can then extend $\Delta$ to $A$ by linearity. Since, for $a,b$ homogeneous, $\Delta_{rt}(a)\Delta_{t's}(b)=0$ unless $t=t'$, it follows from property \ref{Prope} of Definition \ref{DefPartBiAlg} that $\Delta$ is a homomorphism. 
\end{proof}

We will refer to $\Delta: A\rightarrow M(A\otimes A)$ as the \emph{total comultiplication} of $\mathscr{A}$. We will then also use the suggestive Sweedler notation for this map, \[\Delta(a) = a_{(1)}\otimes a_{(2)}.\]

\begin{Lem} The element $E = \sum_{k,l,m} \UnitC{k}{l}\otimes \UnitC{l}{m}$ is a well-defined idempotent in $A\otimes A$, and satisfies \[\Delta(A)(A\otimes A)=E(A\otimes A),\quad (A\otimes A)\Delta(A)= (A\otimes A)E.\]
\end{Lem} 
\begin{proof} Clearly the sum defining $E$ is strictly convergent, and makes $E$ into an idempotent. It is moreover immediate that $E\Delta(a)=\Delta(a) = \Delta(a)E$ for all $a\in A$. Since \[E(\UnitC{k}{l}\otimes \UnitC{m}{n}) = \Delta(\UnitC{k}{n})(\UnitC{k}{l}\otimes \UnitC{m}{n}) \] by the property \ref{Propa} of Definition \ref{DefPartBiAlg}, and analogously for multiplication with $E$ on the right, the lemma is proven. 
\end{proof} 

By \cite[Proposition A.3]{VDW2}, there is a unique homomorphism $\Delta:M(A)\rightarrow M(A\otimes A)$ extending $\Delta$ and such that $\Delta(1) = E$. Alternatively, if $m\in M(A)$, we can directly define $\Delta(m)$ as the strict limit of the series $\sum_{k,l,r,s} \Delta_{rs}(m_{kl})$. Similarly the maps $\id\otimes \Delta$ and $\Delta\otimes \id$ extend to maps from $M(A\otimes A)$ to $M(A\otimes A\otimes A)$. The following proposition gathers the properties of $\Delta$, $\epsilon$ and $\Delta(1)$ which guarantee that $(A,\Delta)$ forms a weak multiplier bialgebra in the sense of \cite[Definition 2.1]{Boh1}.

\begin{Prop} Let $\mathscr{A}$ be a partial bialgebra with total algebra $A$, total comultiplication $\Delta$ and counit $\epsilon$. Then the following properties are satisfied.
\begin{enumerate}[label={(\arabic*)}]
\item Coassociativity: $(\Delta\otimes \id)\Delta = (\id\otimes \Delta)\Delta$ (as maps $M(A)\rightarrow M(A^{\otimes 3})$).
\item Counitality: $(\epsilon\otimes \id)(\Delta(a)(1\otimes b)) = ab = (\id\otimes \epsilon)((a\otimes 1)\Delta(b))$ for all $a,b\in A$.
\item Weak Comultiplicativity of Unit: \[(\Delta(1)\otimes 1)(1\otimes \Delta(1)) = (\Delta\otimes \id)\Delta(1) = (\id\otimes \Delta)\Delta(1) = (1\otimes \Delta(1))(\Delta(1)\otimes 1).\]
\item \label{WMC} Weak Multiplicativity of Counit: For all $a,b,c\in A$, one has \[(\epsilon\otimes \id)(\Delta(a)(b\otimes c)) = (\epsilon\otimes \id)((1\otimes a)\Delta(1)(b\otimes c))\] and 
\[(\epsilon\otimes \id)((a\otimes b)\Delta(c)) = (\epsilon\otimes \id)((a\otimes b)\Delta(1)(1\otimes c)).\]
\item Strong multiplier property: For all $a,b\in A$, one has \[\Delta(A)(1\otimes A)\cup (A\otimes 1)\Delta(A)\subseteq  A\otimes A.\] 
\end{enumerate}
\end{Prop}

\begin{proof} Most of these properties follow immediately from the definition of a partial bialgebra. For demonstrational purposes, let us check the first identity of property \ref{WMC}. Let us choose $a\in A(K)$, $b\in A(L)$ and $c\in A(M)$. Then \[\Delta(a)(b\otimes c) = \delta_{K_{ru},L_{lu}}\delta_{M_{lu},L_{ld}} \sum_r \Delta_{r,L_{ld}}(a)(b\otimes c).\]  Applying $(\epsilon\otimes \id)$ to both sides, we obtain by Proposition \ref{Propb} of Definition \ref{DefPartBiAlg} and counitality of $\epsilon$ that \[(\epsilon \otimes \id)(\Delta(a)(b\otimes c)) = \delta_{K_{ru},L_{lu},L_{ld},M_{lu}} \epsilon(b) ac.\] On the other hand, \begin{eqnarray*} (1\otimes a)\Delta(1)(b\otimes c) &=& \sum_{r,s,t} \UnitC{r}{s} b \otimes a\UnitC{s}{t}c \\ &=& \delta_{L_{ld},K_{ru},M_{lu}} b \otimes ac.\end{eqnarray*} Applying $(\epsilon\otimes \id)$, we find \begin{eqnarray*} (\epsilon\otimes \id)( (1\otimes a)\Delta(1)(b\otimes c) ) &=&  \delta_{L_{ld},K_{ru},M_{lu}}\delta_{L_{lu},L_{ld}}\delta_{L_{ru},L_{rd}} \epsilon(b)ac \\ &=&  \delta_{L_{ld},L_{lu},K_{ru},M_{lu}} \epsilon(b)ac,\end{eqnarray*} which agrees with the expression above.
\end{proof} 

\begin{Rem} 
%\begin{enumerate}\item 
Since also the expressions $\Delta(a)(b\otimes 1)$ and $(1\otimes a)\Delta(b)$ are in $A\otimes A$ for all $a,b\in A$, we see that $(A,\Delta)$ is in fact a \emph{regular} weak multiplier bialgebra \cite[Definition 2.3]{Boh1}.
%\item\label{RemStrongReg} In fact, for any multiplier $m$, all expressions of the form $\Delta(m)(1\otimes b)$ for $b\in A$ are already in $A\otimes A$. This follows from the fact that this holds for $m=1$. This is of course a very strong property for a general weak multiplier Hopf algebra.
%\end{enumerate} 
\end{Rem} 

%\begin{proof} This follows from equalities of the form $\Delta(a)(1\otimes \lambda_s) = \sum_r \Delta_{rs}(a)$, where the sum on the right is finite dimensional.
%\end{proof} 



% Need to give a converse of the theorem, show that any weak multiplier bialgebra with object algebra $F_f(I)$ is of the above form?

%\begin{Def}[B\"{o}hm] A \emph{multiplier weak bialgebra} consists of a non-degenerate idempotent algebra $A$, equipped with a homomorphism \[\Delta: A\rightarrow M(A\otimes A),\] a linear map $\varepsilon: A\rightarrow \C$ and an idempotent element $E\in M(A\otimes A)$ such that the following conditions are satisfied.
%\begin{itemize}
%\item[$\bullet$] For all $a,b\in A$, one has both $\Delta(a)(1\otimes b)$ and $(a\otimes 1)\Delta(b)$ inside $A\otimes A$. 
%\item[$\bullet$] \emph{Coassociativity}: for all $a,b,c\in A$, one has \[(b\otimes 1\otimes 1)(\Delta\otimes \id)\left(\Delta(a)(1\otimes c)\right) = (\id\otimes \Delta)\left((b\otimes 1)\Delta(a)\right)(1\otimes 1\otimes c)\]
%\item[$\bullet$] \emph{Counitality}: For all$ a,b\in A$, one has \[(\varepsilon \otimes \id)(\Delta(a)(1\otimes b) = ab = (\id\otimes \varepsilon)((a\otimes 1)\Delta(b)).\]
%\item[$\bullet$] The space $\Delta(A)(A\otimes A)$ equals $E(A\otimes A)$, and similarly the space $(A\otimes A)\Delta(A)$ equals $(A\otimes A)E$.  % Nicer formulation?
%\item[$\bullet$] 
%\end{itemize}

%\end{Def}

%The \emph{multiplier algebra} $M(A)$ of $A$ is the set One easily shows that this definition is independent of the realization of $A$ as a total algebra, and that it coincides with the notion of multiplier algebra found for example in \cite[Appendix]{VDae1}.

%In the following, we always view $A$ as a subalgebra of $M(A)$ in the natural way. Clearly $m$ is a multiplier if and only if for each $k$, the function $l\mapsto \eGrr{m}{k}{l}$ is almost everywhere zero, and similarly for each $l$, the function $k\mapsto \eGrr{m}{k}{l}$ is almost everywhere zero, while $m\in A$ if the function $(k,l)\mapsto \eGrr{m}{k}{l}$ is almost everywhere zero. The algebra $M(A)$ has a unit, $\eGrr{1}{k}{l} = \delta_{kl} e_k$. For $m\in M(A)$, we use the formal notation $m = \sum_{k,l} m_{kl}$ which can be justified as a topological sum using the multiplier topology. Then we may write the unit for example as $\sum_p e_p $.



%In the following, we will also use the notation \[\lambda_k = \sum_l e\Grru{k}{l},\qquad \rho_l = \sum_k e\Grru{k}{l} \qquad \in M(A).\]

%\begin{Lem} Let $(A(K),\Delta\Grru{K}{L})$ be a partial bialgebra. Then \[\Delta: M(A)\rightarrow M(A\otimes A),\quad a\mapsto \sum_{K,L,M} \Delta\Grru{K}{L}(a(M))\] is a well-defined homomorphism for which \begin{equation}\label{EqStrongMult} \Delta(A)(1\otimes A)\cup \Delta(A)(A\otimes 1)\cup (A\otimes 1)\Delta(A)\cup (1\otimes A)\Delta(A)\subseteq A\otimes A.\end{equation}
%Moreover, $\Delta$ satisfies the coassociativity assumption \[(c\otimes 1\otimes 1)(\Delta\otimes \id)\Delta(a)(1\otimes b)  = \left((\id\otimes \Delta)(c\otimes 1)\Delta(a)\right)(1\otimes 1\otimes b),\qquad \forall a,b,c\in A.\] 
%\end{Lem} 
%\begin{proof} The algebra $A\otimes A$ is a locally unital algebra with units $e_K\otimes e_L$ for column vectors $K,L$. Let us first verify that for $a\in A(M)$, one has $\Delta\Grru{K}{L}(a)$ almost always zero if $K_l$ and $L_l$ (resp. $K_r$ and $L_r$) are fixed. Since this expression is zero when $K\cdot L \neq M$ or $\neg K\bmate L$, it follows from condition b) in Definition \ref{DefPartBiAlg} that this condition is indeed satisfied. Hence $\Delta(a(M)) \in M(A\otimes A)$, and moreover, one obtains then also immediately that the inclusion \eqref{EqStrongMult} holds.

%If now $a\in M(A)$, we know that $a(M)$ is almost always zero if $m_{lu}$ and $m_{ru}$ (resp. $m_{ru}$ and $m_{rd}$) are fixed. It follows as before that $\Delta(a)$ is a well-defined multiplier.

%It is now easy to check that $\Delta$ is a homomorphism by condition c) in Definition \ref{DefPartBiAlg}, and that is coassociative by the partial coassociativity of the $\Delta\Grru{K}{L}$.
%\end{proof} 

%We can write then for example \[\Delta(1) = \sum_p \rho_p\otimes \lambda_p \in M(A\otimes A).\]

%In the following, we will use the Sweedler notation for $\Delta$, so \[\Delta(a) = a_{(1)}\otimes a_{(2)},\qquad a\in A.\] We will also use the notation \[\Delta\Gru{K}{L}(a) = a_{(1)K}\otimes a_{(2)L}.\]



%\begin{Def} A partial $^*$-algebra is a partial algebra equipped with anti-linear maps \[A_{K}\rightarrow A_{K^{\circ}},\quad a\mapsto a^*\] for which the total algebra $A$ becomes a $^*$-algebra.
%A partial $^*$-bialgebra is a partial bialgebra for which $\Grt{\Delta}{K}{L}(a)^* = \Grt{\Delta}{K^{\circ}}{L^{\circ}}(a^*)$ for all $a\in A$.
%\end{Def} 

%It is easily verified that in this case $\overline{\varepsilon_{K}(a)} = \varepsilon_{K^{\circ}}(a)$, and that $\Delta:A\rightarrow M(A\otimes A)$ becomes a $^*$-algebra homomorphism.

% To complete
\begin{Rem}\label{RemFull}
Recall from \cite[Section 3]{Boh1} that a regular weak multiplier bialgebra admits four projections $A\rightarrow M(A)$, given by \begin{align*} \bar{\Pi}^L(a) = (\epsilon\otimes \id)((a\otimes 1)\Delta(1)),\quad &  \bar{\Pi}^R(a) = (\id\otimes \epsilon)(\Delta(1)(1\otimes a)),\\ \Pi^L(a)  = (\epsilon\otimes \id)(\Delta(1)(a\otimes 1)),\quad& \Pi^R(a) = (\id\otimes\epsilon)((1\otimes a)\Delta(1)),\end{align*} where the right hand side expressions are interpreted as multipliers in the obvious way. A trivial computation shows that if $a\in \eGr{A}{k}{l}{m}{n}$, then \[ \overline{\Pi}^L(a) =\epsilon(a)\lambda_n,\quad \overline{\Pi}^R(a) =\epsilon(a)\rho_k,\quad \Pi^L(a) =\epsilon(a)\lambda_m,\quad \Pi^R(a) = \epsilon(a) \rho_l.\] In particular, we see that the \emph{base algebra} of $(A,\Delta)$ is just the algebra $\Fun_{\fin}(I)$ of finite support functions on $I$. From \cite[Theorem 3.13]{Boh1}, we conclude that the comultiplication of $A$ is (left and right) \emph{full} (meaning roughly that the legs of $\Delta(A)$ span $A$).
\end{Rem}

We now formulate the notion of partial Hopf algebra, whose total form will correspond to a weak multiplier Hopf algebra \cite{Boh1,VDW2,VDW1}. We will mainly refer to \cite{Boh1} for uniformity.% OK?

 Let us denote $\circ$ for the inverse of $\wmult$, and $\bullet$ for the inverse of $\bmult$, so \[\begin{pmatrix} k & l \\ m & n \end{pmatrix}^{\circ} = \begin{pmatrix} l & k \\ n & m \end{pmatrix},\quad \begin{pmatrix} k & l \\ m & n \end{pmatrix}^{\bullet} = \begin{pmatrix} m & n \\ k & l \end{pmatrix},\quad \begin{pmatrix} k & l \\ m & n \end{pmatrix}^{\circ \bullet} = \begin{pmatrix} n & m \\ l & k \end{pmatrix}.\] The notation $\circ$ (resp. $\bullet$) will also be used for row vectors (resp. column vectors).

% Write this out more explicitly
\begin{Def}\label{DefPartBiAlgAnt} An \emph{antipode} for an $I$-partial bialgebra $\mathscr{A}$ consists of maps \[S:A(K)\rightarrow A(K^{\circ\bullet})\]
% $S: \eGr{A}{k}{l}{m}{n} \rightarrow \eGr{A}{n}{m}{l}{k}$ 
such that the following property holds: for all $M,P\in M_2(I)$ and all $a\in A(M)$, \begin{eqnarray*} \underset{K\wmult L^{\circ\bullet}=P}{\sum_{K\bmult L = M}} a_{(1)K}S(a_{(2)L})= \delta_{P_l,P_r}\epsilon(a)\mathbf{1}(P_l),\\\underset{K^{\circ\bullet}\wmult L=P}{\sum_{K\bmult L = M}} S(a_{(1)K})a_{(2)L}= \delta_{P_l,P_r}\epsilon(a)\mathbf{1}(P_r).\end{eqnarray*}

A partial multiplier bialgebra $\mathscr{A}$ is called a \emph{partial Hopf algebra} if it admits an antipode.
\end{Def} 

\begin{Rem} Note that condition \ref{Propd} of Definition \ref{DefPartBiAlg} again guarantees that the above sums are in fact finite.
\end{Rem}

If $S$ is an antipode for a partial bialgebra, we can extends $S$ to a linear map \[S:A\rightarrow A\] on the total algebra $A$. 

\begin{Lem}\label{LemAnti} Let $S$ be an antipode for a partial Hopf algebra $\mathscr{A}$. Then for all $a,b,c\in A$, one has \begin{eqnarray*} (a_{(1)}c\otimes ba_{(2)}S(a_{(3)})) &=& (1\otimes b)\Delta(1)(ac\otimes 1)\\ (S(a_{(1)})a_{(2)}c\otimes ba_{(3)})&=& (1\otimes ba)\Delta(1)(c\otimes 1).\end{eqnarray*}
\end{Lem} 

%Note that we can here make use of Remark \ref{RemStrongReg} to make sense of the above expressions as elements of $A\otimes A$. 

\begin{proof} Take $a \in \eGr{A}{k}{l}{m}{n}$. Then in the strict topology, we obtain from the first identity in Definition \ref{DefPartBiAlgAnt} that 
\begin{eqnarray*}
a_{(1)}\otimes a_{(2)}S(a_{(3)}) 
&=& 
\sum_{r,s,t,u} a_{(1)\pmat{k}{l}{r}{s}} \otimes a_{(2)\pmat{r}{s}{t}{u}}S(a_{(3)\pmat{t}{u}{m}{n}})\\ 
&=& 
\sum_{r,t} a_{(1)\pmat{k}{l}{r}{n}} \otimes \left(\sum_{u} a_{(2)\pmat{r}{n}{t}{u}}S(a_{(3)\pmat{t}{u}{m}{n}})\right)\\ 
&=& 
\sum_{r,t} a_{(1)\pmat{k}{l}{r}{n}} \otimes \left(\delta_{r,m}\epsilon(a_{(2)\pmat{m}{n}{m}{n}})\UnitC{r}{t}\right)\\
&=&
a\otimes \lambda_m\\
&=&
\Delta(1)(a\otimes 1).
\end{eqnarray*}
The second identity is proven similarly.
\end{proof} 

\begin{Prop} A collection of maps $S:A(K)\rightarrow A(K^{\circ\bullet})$ defines an antipode for a partial bialgebra $\mathscr{A}$ if and only if the total map $S:A\rightarrow A$ defines an antipode for the total weak multiplier bialgebra $(A,\Delta)$ in the sense of \cite[Section 6]{Boh1}. 
\end{Prop} 

\begin{proof} We verify that the total antipode $S:A\rightarrow A$ satisfies the conditions of an antipode as stated in \cite[Theorem 6.8.(2)]{Boh1}. In fact, the two first identities of their Theorem 6.8.(2) are precisely our identities in Lemma \ref{LemAnti}. The third and last identity of their Theorem 6.8.(2) says that formally one should have \[\sum_kS(\rho_k a)\lambda_k = S(a),\quad \forall a\in A,\] but this follows immediately from the way $S$ acts on homogeneous components.% Should I spell out the definition of an antipode as given by Bohm et al?
\end{proof}

From the remarks following \cite[Theorem 6.8.(2)]{Boh1}, we conclude the following.

\begin{Cor} If $\mathscr{A}$ is a partial Hopf algebra, its total algebra $A$ with total comultiplication $\Delta$ forms a weak multiplier Hopf algebra in the sense of \cite[Definition 1.14]{VDW1}.
\end{Cor}

From \cite[Theorem 6.12 and Corollary 6.16]{Boh1}, we obtain the following corollary. The notation $\Delta_{rs}^{\op}$ signifies the composition of $\Delta_{rs}$ with the switch map.

\begin{Cor} The map $S:A\rightarrow A$ is an antihomomorphism s.t. \[\Delta_{rs}(S(a)) = (S\otimes S)\Delta_{sr}^{\op}(a).\]
\end{Cor} 


We now turn towards the structures which will allow us to build operator algebraic quantum groupoids out of our partial Hopf algebras.

\begin{Def} A partial $^*$-algebra $\mathscr{A}$ is a partial algebra whose total algebra $A$ is equipped with an antilinear, antimultiplicative involution \[*:A\rightarrow A,\quad a\mapsto a^*\] such that the $\mathbf{1}_k$ are selfadjoint for all $k$ in the object set. 
\end{Def} 

One can of course give an alternative definition directly in terms of the partial algebra structure by requiring that we are given antilinear maps $A(k,l)\rightarrow A(l,k)$ satisfying the obvious antimultiplicativity and involution properties.

\begin{Def} An $I$-partial $^*$-bialgebra $\mathscr{A}$ is an $I$-partial bialgebra whose underlying partial algebra has been endowed with a partial $^*$-algebra structure for which \[\Delta_{rs}(a)^* = \Delta_{sr}(a^*),\qquad \forall K\in M_2(I), \forall a\in A(K), \forall r,s\in I.\]

A partial Hopf $^*$-algebra is a partial bialgebra which is at the same time a partial $^*$-bialgebra and a partial Hopf algebra.
\end{Def} 

\begin{Rem}
\begin{enumerate}
\item By uniqueness of the counit \cite[Theorem 2.8]{Boh1}, it follows automatically that $\epsilon$ satisfies $\epsilon(a^*) = \overline{\epsilon(a)}$ for all $a$.
\item From \cite[Proposition 4.11]{VDW1}, it follows automatically that $S(S(a)^*)^* = a$ for all $a$. In particular, $S$ is invertible. % To substantiate
\end{enumerate}
\end{Rem}

\begin{Def} Let $\mathscr{A}$ be a partial bi-algebra. An \emph{invariant functional} $\varphi$ for $\mathscr{A}$ consists of a functional $\varphi:A\rightarrow \C$ %necessary to assume?
such that for all $a\in \eGr{A}{k}{l}{m}{n}$ and all $b$, one has  \[(\id\otimes \varphi)((b\otimes 1)\Delta(a)) = \varphi(a)b\lambda_k,\] and \[(\varphi\otimes \id)(\Delta(a)(1\otimes b)) = \varphi(a)\rho_nb.\]

We call $\varphi$ \emph{normalized} if $\varphi(\UnitC{k}{k}) = 1$ for all $k$. 
\end{Def} % More motivation for this notion to be provided?

\begin{Rem} 
\begin{enumerate}
\item It follows from the defining property of $\varphi$ (and the fact that $\lambda_k\neq 0\neq \rho_k$ for any $k\in I$) that $\varphi$ is zero on the $A(K)$ with $K_l\neq K_r$. 
\item If $\varphi$ is normalized, it also follows immediately from the definining property that $\varphi(\UnitC{k}{l})=1$ whenever $\UnitC{k}{l}\neq 0$.
\end{enumerate}
\end{Rem} 

We are finally ready to give our main definition.

\begin{Def} A \emph{partial compact quantum group} $\mathscr{G}$ is a partial Hopf $^*$-algebra $\mathscr{A} = P(\mathscr{G})$ with a \emph{positive} normalized invariant functional $\varphi$, that is $\varphi(a^*a)\geq 0$ for all $a\in A$. We also say that $\mathscr{G}$ is the partial compact quantum group \emph{defined by} $\mathscr{A}$.
%It is called \emph{connected} if the underlying partial algebra is connected, i.e.~ when $\UnitC{k}{l}\neq0$ for all $k,l$.  % Necessary to introduce?
\end{Def} 

\begin{Rem} Following \cite{Hay1}, we could also have called our objects \emph{compact quantum groups of face type}, but we feel this gives the wrong impression when the base algebra is infinite dimensional (i.e. the object set is not compact). When referring to partial compact quantum groups, we feel that it is better reflected that only the \emph{parts} of this object are to be considered compact, not the total object. % Is this a clear enough motivation? 
\end{Rem} 

%As in \cite[Example 2.14]{Boh1}, one can take direct sums of partial compact quantum groups. The following lemma shows that any partial compact quantum group is a direct sum of connected ones. FALSE!

%\begin{Lem} Let $\mathscr{A}$ define a partial compact quantum group. Then $\mathscr{A}\cong \oplus_k \mathscr{A}_j$ where the $\mathscr{A}_j$ define \emph{connected} partial compact quantum groups.
%\end{Lem} 
%\begin{proof} Let $I$ be the object set of $\mathscr{A}$, and say $k\sim l$ if $\UnitC{k}{l}\neq 0$. Then $\sim$ is an equivalence relation: as \[\Delta_{ll}(\UnitC{k}{m}) = \UnitC{k}{l}\otimes \UnitC{l}{m},\] it is transitive. As $S(\UnitC{k}{l}) = \UnitC{l}{k}, it is symmetric. And as $\varepsilon(\UnitC{1}{k}{k})=1$, it is reflexive.

%Let then $I = \sqcup_{\alpha\in \mathscr{I}} I_{\alpha}$ be a labeled partition associated to $\sim$. 

%\end{proof} 
 

We will need the following lemma at some point, which is an almost verbatim transcription of the argument in \cite[Proposition 3.4]{VDae2}.

\begin{Lem}\label{LemFaith} Let $\mathscr{A}$ be a partial Hopf algebra, and let $\varphi$ be a normalized invariant functional for $\mathscr{A}$. Then $\varphi$ is faithful: if $a\in A$ and $\varphi(ab) =0$ (resp. $\varphi(ba)=0$) for all $b\in A$, then $a=0$.
\end{Lem} 
% Ref to Timmermann for case of quantum groupoids? Can be explicitly used here? Ask Fons if he has already derived this result, and acknowledge? 
\begin{proof} Suppose  $\varphi(ba)=0$ for all $b$, we arrive at the conclusion that for all $d\in A$ and all functionals $\omega$ on $A$, the element $p = (\omega\otimes \id)((d\otimes 1)\Delta(b))$ satisfies \[(\id\otimes \varphi)((1\otimes c)\Delta(p)) = 0.\] Continuing as in the proof of \cite[Proposition 3.4]{VDae2}, we obtain from the antipode trick that \[\sum_n \varphi(cS(q)\rho_n)\epsilon(p\lambda_n)=0.\] Choosing now for $c$ and $q$ local units of the form $\lambda_k\rho_l$, the normalization condition on $\varphi$ gives that $\epsilon(p\lambda_n)=0$ for all $n$, hence $\epsilon(p)=0$. This implies $\omega(da)=0$. As $\omega$ and $d$ were arbitrary, it follows that $a=0$.

The other case follows similarly, or by considering the opposite comultiplication.
\end{proof} 




%%% Local Variables: 
%%% mode: latex
%%% TeX-master: "dyn-suq-main"
%%% End: 
