% Tannaka-Krein or Tannaka?

\documentclass[10pt]{article}

\usepackage{hyperref}
\usepackage{fixme}
\usepackage{mathrsfs}

\usepackage[a4paper]{geometry}
\usepackage{amssymb, amsthm, amsfonts, amsxtra, amsmath}
\usepackage{latexsym}
\usepackage{mathabx}
\usepackage{enumitem}
\usepackage[all]{xy}
\usepackage{graphics}
\usepackage{pdfpages}
\usepackage{epic}
\usepackage{fouridx}
\usepackage{parskip} % paragraphs have no indents and vertical spacings inbetween
\usepackage{bbm}


\makeatletter % need this to avoid the conflict between amsthm and parskip
\def\thm@space@setup{%
  \thm@preskip=\parskip \thm@postskip=0pt
}
\makeatother

\DeclareMathOperator{\adj}{\mathrm{adj}}
\DeclareMathOperator{\can}{\mathrm{can}}
\DeclareMathOperator{\Char}{\mathrm{Char}}
\DeclareMathOperator{\dyn}{\mathrm{dyn}}
\DeclareMathOperator{\ext}{\mathrm{e}}
\DeclareMathOperator{\End}{\mathrm{End}}
\DeclareMathOperator{\fin}{\mathrm{fd}}
\DeclareMathOperator{\hol}{\mathrm{hol}}
\DeclareMathOperator{\id}{id}
\DeclareMathOperator{\img}{img}
\DeclareMathOperator{\Ind}{\mathrm{Ind}}
\DeclareMathOperator{\Hom}{Hom}
\DeclareMathOperator{\Ker}{\mathrm{Ker}}
\DeclareMathOperator{\Mat}{\mathscr{M}\!\it{at}}
\DeclareMathOperator{\Matt}{\mathrm{Mat}}
\DeclareMathOperator{\Nat}{\mathrm{Nat}}
\DeclareMathOperator{\op}{\mathrm{op}}
\DeclareMathOperator{\Pol}{\mathrm{P}}
\DeclareMathOperator{\Par}{\mathrm{Par}}
\DeclareMathOperator{\Ran}{\mathrm{Ran}}
\DeclareMathOperator{\rcf}{\mathrm{rcfd}}
\DeclareMathOperator{\rd}{\mathrm{d}}
\DeclareMathOperator{\reg}{\mathrm{reg}}
\DeclareMathOperator{\sgn}{\mathrm{sgn}}
\DeclareMathOperator{\Span}{\mathrm{span}}
\DeclareMathOperator{\Spec}{\mathrm{Spec}}
\DeclareMathOperator{\tr}{\mathrm{tr}}
\DeclareMathOperator{\Zz}{\mathrm{Z}}


\newcommand{\dual}[1]{#1^{*}}
\newcommand{\duall}[1]{#1^{\wedge}}
\newcommand{\dualop}[1]{#1^{\tr}}
\newcommand{\dualco}[1]{\hat{#1}}
\newcommand{\dualcor}[1]{\check{#1}}
\newcommand{\predual}[1]{{^{\vee}\!#1}}
\newcommand{\co}{\mathrm{co}}
\newcommand{\Corep}{\mathrm{Corep}}
\newcommand{\Corepf}{\mathrm{Corep}^{f}}
\newcommand{\sff}{\textrm{s.f.~}}
\newcommand{\sfs}{\mathrm{sfs}}
\newcommand{\sfd}{\mathrm{sfd}}

\newcommand{\Circt}{{\mathop{\ooalign{$\ovoid$\cr\hidewidth\raise-.05ex\hbox{$\scriptstyle\mathsf T\mkern3.5mu$}\cr}}}} % Woronowicz style tensor product, USUAL SIZE
\newcommand{\Circtv}[1]{\underset{#1}{\mathop{\ooalign{$\ovoid$\cr\hidewidth\raise-.05ex\hbox{$\scriptstyle\mathsf T\mkern3.5mu$}\cr}}}} % Woronowicz style tensor product, USUAL SIZE
\newcommand{\smCirct}{\mathop{\ooalign{$\scriptstyle\ovoid$\cr\hidewidth\raise-.05ex\hbox{$\scriptscriptstyle\mathsf T\mkern2.8mu$}\cr}}}  % Woronowicz style tensor product, SCRIPT SIZE

\newcommand{\nc}{\R}
\newcommand{\g}{\mathfrak{g}}
\newcommand{\h}{\mathfrak{h}}

\newcommand{\kk}{\mathfrak{k}}
\newcommand{\ttt}{\mathfrak{t}}
\newcommand{\p}{\mathfrak{p}}
\newcommand{\n}{\mathfrak{n}}
\newcommand{\llll}{\mathfrak{l}}
\newcommand{\uu}{\mathfrak{u}}
\newcommand{\bb}{\mathfrak{b}}
\newcommand{\q}{\mathfrak{q}}
\newcommand{\su}{\mathfrak{su}}
\newcommand{\ssl}{\mathfrak{sl}}
\newcommand{\SSL}{\mathrm{SL}}
\newcommand{\so}{\mathfrak{so}}
\newcommand{\spp}{\mathfrak{sp}}
\newcommand{\G}{\mathbb{G}}
\newcommand{\e}{\mathfrak{e}}
\newcommand{\s}{\mathfrak{s}}
\newcommand{\C}{\mathbb{C}}
\newcommand{\R}{\mathbb{R}}
\newcommand{\Z}{\mathbb{Z}}
\newcommand{\N}{\mathbb{N}}
\newcommand{\X}{\mathbb{X}}
\newcommand{\Y}{\mathbb{Y}}
\newcommand{\Ss}{\mathbb{S}}
\newcommand{\ZZ}{\mathscr{Z}}
\newcommand{\ad}{\mathrm{ad}}
\newcommand{\Hsp}{\mathcal{H}}
\newcommand{\qn}[2]{\lbrack #1 \rbrack_{#2}}
\newcommand{\fqn}[2]{\lbrack #1 \rbrack_{#2}!}
\newcommand{\bqn}[3]{\left\lbrack \begin{array}{c} \!#1\! \\ \!#2\! \end{array}\right\rbrack_{#3}}
\newcommand{\Tr}{\mathrm{Tr}}
\newcommand{\RR}{\mathcal{R}}
\newcommand{\res}{\mathrm{res}}
\newcommand{\cop}{\mathrm{cop}}
\newcommand{\opp}{\mathrm{op}}
\newcommand{\coop}{\mathrm{coop}}
\newcommand{\Rm}{\mathcal{R}}
\newcommand{\wt}{\mathrm{wt}}
\newcommand{\Ad}{\mathrm{Ad}}
\newcommand{\CatC}{\mathcal{C}}
\newcommand{\CatD}{\mathcal{D}}
\newcommand{\CatCC}{\mathscr{C}}
\newcommand{\CatDD}{\mathscr{D}}
\newcommand{\Corr}{\mathrm{Corr}}

\newcommand{\Vectf}{\mathrm{Vect}_{f}}
\newcommand{\Vecti}{\mathrm{Vect}^{I\times I}}
\newcommand{\Vectif}{\mathrm{Vect}^{I\times I}_{\fin}}
\newcommand{\Vectrcf}{\mathrm{Vect}^{I\times I}_{\rcf}}
\newcommand{\Hilb}{\mathrm{Hilb}}
\newcommand{\Hilbf}{\mathrm{Hilb}_{\mathrm{f}}}
\newcommand{\Hilbi}{\mathrm{Hilb}^{I\times I}}
\newcommand{\Hilbif}{\mathrm{Hilb}^{I\times I}_{\fin}}
\newcommand{\Hilbrcf}{\mathrm{Hilb}^{I\times I}_{\rcf}}

\newcommand{\Star}[2]{{}_{#1}\!*_{#2}}
\newcommand{\vot}{\bar{\otimes}}
\newcommand{\A}{\mathcal{B}}
\newcommand{\Aa}{\mathscr{B}}
\newcommand{\Mor}{\mathrm{Mor}}
\newcommand{\alg}{\mathrm{alg}}
\newcommand{\Gg}{\mathscr{G}}
\newcommand{\ev}{\mathrm{ev}}
\newcommand{\coev}{\mathrm{coev}}
\newcommand{\Rtimes}{\underset{\R}{\times}}
\newcommand{\Rb}{\R^{\bullet}}
\newcommand{\vtimes}{\bar{\otimes}}
\newcommand{\Rr}{\mathscr{R}}
\newcommand{\Tt}{\mathscr{T}}
\newcommand{\Fun}{\mathrm{Fun}}
\newcommand{\Ff}{\Fun_{\fin}}
\newcommand{\itimes}{\underset{I}{\otimes}}
\newcommand{\osum}[1]{\underset{#1}{\sum}^{\oplus}}
\newcommand{\osumc}[1]{\underset{#1}{\sum}^{\bar{\oplus}}}
\newcommand{\oplusc}{\bar{\oplus}}
\newcommand{\wDelta}{\widetilde{\Delta}}
\newcommand{\f}{\mathrm{fin}}
\newcommand{\Rho}{\mathrm{P}}
\newcommand{\Rep}{\mathrm{Rep}}
\newcommand{\DA}{\mathcal{A}}
\newcommand{\even}{\mathrm{even}}
\newcommand{\odd}{\mathrm{odd}}
\newcommand{\fd}{\mathrm{fd}}
\newcommand{\Forget}{F}
\newcommand{\Vect}{\mathrm{Vect}}


\newcommand{\GrHA}[3]{#1{\begin{pmatrix} #2,  #3\end{pmatrix}}}% Horizontal grading ordinary style, with argument
\newcommand{\Grs}[3]{#1{\begin{pmatrix} #2,  #3\end{pmatrix}}}

\newcommand{\GrDA}[3]{{}_{#2}#1_{#3}} % Horizontal grading bottom style, with argument

\newcommand{\GrVA}[3]{#1{\tiny {\begin{pmatrix} #2\\#3\end{pmatrix}}}} % Vertical grading ordinary style, with argument
\newcommand{\Grt}[3]{#1{\tiny {\begin{pmatrix} #2\\#3\end{pmatrix}}}} 
\newcommand{\Grtt}[4]{#1{\tiny {\begin{pmatrix} #2\\#3\\#4\end{pmatrix}}}} 
\newcommand{\pms}[2]{{\tiny {\begin{pmatrix} #1\\#2\end{pmatrix}}}}

\newcommand{\GrRA}[3]{#1^{#2}_{#3}} % Vertical grading right style, with argument

\newcommand{\Unit}{\mathbf{1}}
\newcommand{\Unitb}{\mathbbm{1}}
\newcommand{\UnitC}[2]{\Grt{\mathbf{1}}{#1}{#2}} 
\newcommand{\IdCV}[2]{p_{#1#2}}
\newcommand{\Grru}[2]{{\tiny \begin{pmatrix} #1 \\ #2\end{pmatrix}}}

\newcommand{\eGr}[5]{#1{{\tiny \begin{pmatrix} #2 \quad #3 \\ #4 \quad #5\end{pmatrix}}}}

\newcommand{\pma}[4]{\begin{pmatrix} #1 \quad #2 \\ #3 \quad #4\end{pmatrix}}
\newcommand{\pmat}[4]{{\tiny \begin{pmatrix} #1 \quad #2 \\ #3 \quad #4\end{pmatrix}}}

\newcommand{\UT}[2]{#1{\tiny #2 }}
\newcommand{\Gr}[5]{\fourIdx{#2}{#4}{#3}{#5}{#1}}%TODO: better typesetting
\newcommand{\Grl}[3]{\Gr{#1}{#2}{}{#3}{}}%TODO: better typesetting
\newcommand{\Gru}[3]{\Gr{#1}{}{}{#2}{#3}}
\newcommand{\Grd}[3]{\Gr{#1}{#2}{#3}{}{}}
\newcommand{\gr}[5]{\;{}^{\;#2}_{#4}#1_{#5}^{#3}}%TODO: better typesetting
\newcommand{\eGrr}[3]{#1_{{\tiny \left(#2, #3\right)}}}
\newcommand{\eGrt}[4]{#1{{\tiny \begin{pmatrix} #2 \\ #3 \\ #4 \end{pmatrix}}}}
\newcommand{\Grr}[4]{\begin{pmatrix}#1 \quad #2\\#3&#4\end{pmatrix}}

\newcommand{\Grss}[3]{\UT{#1}{\begin{pmatrix} #2 \; #3\end{pmatrix}}}
\newcommand{\Grb}[7]{\UT{#1}{\begin{pmatrix} #2\quad #3 \\ #4 \quad #5\\ #6 \quad #7\end{pmatrix}}}
\newcommand{\un}[2]{e{{\tiny \begin{pmatrix}#1\\ #2\end{pmatrix}}}}
\newcommand{\unn}[3]{e{{\tiny \begin{pmatrix}#1\\ #2\\#3\end{pmatrix}}}}

\newcommand{\wmult}{\cdot}
\newcommand{\bmult}{*}
\newcommand{\wmate}{\rightarrow}% Change this to source/target notation l(eft) r(ight)
\newcommand{\bmate}{\downarrow}% Change this to source/target notation u(p) d(own)

\newcommand{\aste}[1]{\underset{#1}{\ast}}

\newcommand{\Vv}{\mathcal{V}}

\newcommand{\dT}{\dot T}

\newtheorem{Theorem}{Theorem}[section]
\newtheorem{Lem}[Theorem]{Lemma}
\newtheorem{Prop}[Theorem]{Proposition}
\newtheorem{Cor}[Theorem]{Corollary}

\theoremstyle{definition}
\newtheorem{Def}[Theorem]{Definition}
\newtheorem{Rem}[Theorem]{Remark}
\newtheorem{Exa}[Theorem]{Example}
\newtheorem{Not}[Theorem]{Notation}
\newtheorem{Que}[Theorem]{Question}
\newtheorem{Con}[Theorem]{Conjecture}

%%%%%%%%%%%%%%%%%%%
% Further notation for Section 1
\newcommand{\phic}[2]{\Grt{\phi}{#1}{#2}}

%%%%%%%%%%%%%%%%%%%
% Notation for Section 6
\newcommand{\LGtwo}{L^{2}(\mathscr{G})}
\newcommand{\LGinf}{L^{\infty}(\mathscr{G})}
\newcommand{\CrG}{C^{r}_{0}(\mathscr{G})}
\newcommand{\CuG}{C^{u}_{0}(\mathscr{G})}
\newcommand{\vnDelta}{\overline{\Delta}}
\newcommand{\vnE}{\overline{E}}
\newcommand{\astrl}{\underset{l^{\infty}(I)}{_{\rho}\ast_{\lambda}}}
\newcommand{\otimesrl}{\underset{\nu{}}{_{\rho}\otimes_{\lambda}}}
\newcommand{\vnphi}{\overline{\phi}}
\newcommand{\vnphic}[2]{\Grt{\vnphi}{#1}{#2}}
\newcommand{\vnR}{\overline{R}}
\newcommand{\vntau}{\overline{\tau}}

\date{}


\numberwithin{equation}{section}

\begin{document}
\title{Partial compact quantum groups}

\author{Kenny De Commer\thanks{Department of Mathematics, Vrije Universiteit Brussel, VUB, B-1050 Brussels, Belgium, email: {\tt kenny.de.commer@vub.ac.be}}
\and Thomas Timmermann\thanks{University of M\"{u}nster}}

\maketitle

\begin{abstract}
\noindent Compact quantum groups of face type, as introduced by
Hayashi, form a class of quantum groupoids with a classical, finite
set of objects. Using the notions of  weak multiplier bialgebras and
weak multiplier Hopf algebras (resp.~ due to
B{\"o}hm--G\'{o}mez-Torrecillas--L\'{o}pez-Centella and Van
Daele--Wang), we generalize Hayashi's definition to allow for an
infinite set of objects, and call the resulting objects partial
compact quantum groups. We prove a
Tannaka-Kre$\breve{\textrm{\i}}$n-Woronowicz reconstruction result for
such partial compact quantum groups using the notion of  partial fusion C$^*$-categories. As examples, we consider the dynamical quantum $SU(2)$-groups from the point of view of partial compact quantum groups.
\end{abstract}


%\emph{Keywords}:

%AMS 2010 \emph{Mathematics subject classification}:


%17B37: Quantum groups, quantized enveloping algebras
%20G42: quantized function algebras
%46L65: Functional analysis, deformations, quantizations
%81R50: Quantum groups and related algebraic methods
%16T05: Hopf algebras and their applications
%16T10: Bialgebras
%16T15: Coalgebras and comodules; corings
%46L08  $C^*$-modules
%58B32: Geometry of quantum groups


%\tableofcontents


\section*{Introduction}

The concept of \emph{face algebra} was introduced by T. Hayashi in \cite{Hay2}, motivated by the theory of solvable lattice models in statistical mechanics. It was further studied in \cite{Hay1,Hay3,Hay8,Hay6, Hay4, Hay5,Hay7}, where for example associated $^*$-structures and a canonical Tannaka duality were developed. This Tannaka duality allows one to construct a canonical face algebra from any (finite) fusion category. For example, a face algebra can be associated to the fusion category of a quantum group at root unity, for which no genuine quantum group implementation can be found. 

In \cite{Nil1,Sch1,Sch2}, it was shown that face algebras are particular kinds of $\times_R$-algebras \cite{Tak2} and of weak bialgebras \cite{Boh3,BCJ,Nik1}. More intuitively, they can be considered as quantum groupoids with  classical, finite object set. In this article, we want to extend Hayashi's theory by allowing an \emph{infinite} (but still discrete) object set. This requires passing from weak bialgebras to weak \emph{multiplier} bialgebras \cite{Boh1}. At the same time, our structures admit a piecewise description by what we call a \emph{partial bialgebra}, which is more in the spirit of Hayashi's original definition. In the presence of an antipode, an invariant integral and a compatible $^*$-structure, we call our structures \emph{partial compact quantum groups}. 

The passage to the infinite object case requires extra argument at certain points, as one has to impose the proper finiteness conditions on associated structures. However, once all conditions are in place, many of the proofs are similar in spirit to the finite object case. 

Our main result is a Tannaka-Kre$\breve{\textrm{\i}}$n-Woronowicz duality result which states that partial compact quantum groups are in one-to-one correspondence with \emph{concrete partial fusion C$^*$-categories}. In essence, a partial fusion C$^*$-category is a multifusion C$^*$-category \cite{ENO1}, except that (in a slight abuse of terminology) we allow an infinite number of irreducible objects as well as an infinite number of summands inside the unit object. By a \emph{concrete} multifusion C$^*$-category, we mean a multifusion C$^*$-category realized inside a category of (locally finite-dimensional) bigraded Hilbert spaces. Of course, Tannaka reconstruction is by now a standard procedure. For closely related results most relevant to our work, we mention \cite{Wor2,Sch3,Hay8,Ost1,Hai1,Szl1,Pfe1,DCY1,Nes1} as well as the surveys \cite{JoS1} and \cite[Section 2.3]{NeT1}.

As an application, we generalize Hayashi's Tannaka duality \cite{Hay8} (see also \cite{Ost1}) by showing that any module C$^*$-category over a multifusion C$^*$-category has an associated canonical partial compact quantum group. By the results of \cite{DCY1}, such data can be produced from ergodic actions of compact quantum groups. In particular,  we consider the case of ergodic actions of $SU_q(2)$ for $q$ a non-zero real. This will allow us to show that the construction of \cite{Hay1} generalizes to produce partial compact quantum group versions of the dynamical quantum $SU(2)$-group \cite{EtV1,KoR1}, see also \cite{Sto1} and references therein. This construction will provide the right setting for the operator algebraic versions of these dynamical quantum $SU(2)$-groups, which was the main motivation for writing this paper. These operator algebraic details will be studied elsewhere \cite{DCT2}.

The precise layout of the paper is as follows.

The \emph{first section} introduces the basic theory of the structures
which we will be concerned with. We introduce the notions of a
\emph{partial bialgebra}, \emph{partial Hopf algebra} and
\emph{partial compact quantum group}, and show how they are related to
the notion of a weak multiplier bialgebra \cite{Boh1}, weak multiplier
Hopf algebra \cite{VDW1,VDW2} and compact quantum group of face type
\cite{Hay1}. We also introduce the corresponding notions of a \emph{partial tensor category} and \emph{partial fusion C$^*$-category}. Most results are very similar to the ones in \cite{Hay1}, so not all proofs are spelled out.

In the next two sections, our main result is proven, namely the Tannaka-Kre$\breve{\textrm{\i}}$n-Woronowicz duality. In the \emph{second section} we develop the representation theory of partial compact quantum groups, and we show how it allows one to construct a concrete partial fusion C$^*$-category. Again we could refer to \cite{Hay1}, but we give our arguments in a little more detail. In the \emph{third} section, we show conversely how any concrete partial fusion C$^*$-category allows one to construct a partial compact quantum group, and we briefly show how the two constructions are inverses of each other.

In the final two sections, we provide some examples of our structures and applications of our main result. In the \emph{fourth section}, we first consider the construction of a canonical partial compact quantum group from any module C$^*$-category for a partial fusion C$^*$-category. We then introduce the notions of \emph{Morita}, \emph{co-Morita} and \emph{weak Morita equivalence} \cite{Mug1} of partial compact quantum groups, and show that two partial compact quantum groups are weakly Morita equivalent if and only if they can be connected by a string of Morita and co-Morita equivalences. In the \emph{fifth section}, we study in more detail a concrete example of a canonical partial compact quantum group, constructed from an ergodic action of quantum $SU(2)$. In particular, we obtain a partial compact quantum group version of the dynamical quantum $SU(2)$-group. 

\section{Partial compact quantum groups}

\subsection{Partial algebras}

\begin{Def} An \emph{$I$-partial algebra} $\mathscr{A}$ is a set $I=\{k,l,\cdots\}$, the \emph{object} set, together with $\C$-vector spaces $\GrDA{A}{k}{l}$, multiplication maps \[M=M_{k,l,m}:\GrDA{A}{k}{l} \otimes \GrDA{A}{l}{m}\rightarrow \GrDA{A}{k}{m},\qquad a\otimes b \mapsto ab\]  and \emph{unit elements} $\Unit_k \in \GrDA{A}{k}{k}$ such that the obvious associativity and unit conditions are satisfied. 
\end{Def}

We emphasize that $\Unit_k=0$ is allowed, in which case for example $\GrDA{A}{k}{k}=\{0\}$. Note that $I$-partial algebras can be seen as small $\C$-linear categories, but we will use a different notion of morphisms for them than the usual one of functor.

By making $M$ the zero map on all other tensor products, we can turn $A =  \underset{k,l}{\oplus}\,\GrDA{A}{k}{l}$ into an associative algebra, the \emph{total algebra} of $\mathscr{A}$.  It is a locally unital algebra by the orthogonal idempotents $\mathbf{1}_k$. 

For example, for any set $I$ we can define a partial algebra $\Mat_I$ with $\GrDA{\Matt}{k}{l} = \C$ for all $k,l$ and each $M_{k,l,m}$ scalar multiplication. The associated total algebra is the algebra of all finitely supported matrices based over $I$. For a general $\mathscr{A}$, one can identify $A$ with finite support $I$-indexed matrices $(a_{kl})_{k,l}$ with $a_{kl} \in \GrDA{A}{k}{l}$, equipped with the natural matrix multiplication. 

Working with non-unital algebras necessitates the use of their \emph{multiplier algebra} \cite{Dau1,VDae1}. Recall that a multiplier $m$ for an algebra $A$ consists of a couple of maps $a \mapsto L_m(a)=ma$ and $a\mapsto R_m(a)=am$ such that $(am)b = a(mb)$ for all $a,b\in A$. They form an algebra, the \emph{multiplier algebra} $M(A)$, under composition for the $L$-maps and anti-composition for the $R$-maps. In case $A$ is the total algebra of an $I$-partial algebra $\mathscr{A}$, the natural homomorphism $A\rightarrow M(A)$ is injective, and $M(A)$ can be identified with matrices $(m_{kl})_{kl}$ which are rcd in the sense of the following definition.

\begin{Def} An assignment $(k,l)\rightarrow m_{kl}$ into a set with distinguished zero element is called \emph{row-and column-finite} (rcf) if it has finite support in either one of the variables when the other variable has been fixed. 
\end{Def} 

When $m_i\in M(A)$ are such that for each $a\in A$ one has $m_ia =0=am_i$ for all but a finite set of $i$, one can define a multiplier $\sum_i m_i$ in the obvious way. One says that the sum $\sum_i m_i$ converges in the \emph{strict} topology. We will often use this kind of limit implicitly, for example in the next definition.
 
\begin{Def}\label{DefMor} Let $\mathscr{A}$ and $\mathscr{B}$ be respectively $I$ and $J$-partial algebras. A \emph{$\varphi$-morphism} from $\mathscr{A}$ to $\mathscr{B}$ consists of a map $\varphi:I \rightarrow \mathscr{P}(J)$, the power set of $J$, and a homomorphism $f:A\rightarrow M(B)$ such that $f(\Unit_k) = \sum_{r\in \varphi(k)} \Unit_r$.
\end{Def} 
The last condition implies $\varphi(k)\cap \varphi(l)=0$ for
$k\neq l$, so that $\varphi$ corresponds to an $I$-labeled partition of $J$.
Note also that  $f$ splits up into components $f_{rs}:
\GrDA{A}{k}{l}\rightarrow \GrDA{B}{r}{s}$ for all $r\in \varphi(k),s\in
\varphi(l)$. Then one has for example $f_{rt}(ab) = \sum_{s\in \varphi(l)}
f_{rs}(a)f_{st}(b)$ for all $a\in \GrDA{A}{k}{l}, b\in \GrDA{A}{l}{m},
r\in \varphi(k)$ and $t\in \varphi(m)$. This sum is meaningful
because of the rcd property of multipliers.  
 

\subsection{Partial coalgebras}

The notion of a partial algebra dualizes as follows.

\begin{Def}\label{DefCoAlg} An  $I$-\emph{partial coalgebra} $\mathscr{A}$ consists of a set $I=\{k,l,\ldots\}$, the \emph{object set}, together with vector spaces $\GrRA{A}{k}{l}$, comultiplication maps \[\Delta_{l} = \Grtt{\Delta}{k}{l}{m}:\GrRA{A}{k}{m}\rightarrow \GrRA{A}{k}{l}\otimes \GrRA{A}{l}{m},\qquad a \mapsto a_{(l;1)}\otimes a_{(l;2)},\] and counit maps $\epsilon =\epsilon_k:\GrRA{A}{k}{k}\rightarrow \C$ satisfying the obvious coassociativity and counitality conditions.
\end{Def}

We extend $\epsilon$ as the zero functional on $\GrRA{A}{k}{l}$ when $k\neq l$.

\subsection{Partial bialgebras}

We write $I^2$ for $I\times I$ seen as column vectors. To superimpose the notions of a partial algebra and  a partial coalgebra into that of a partial bialgebra, we need the maps
\[\varphi_{\Delta}: I^2 \rightarrow \mathscr{P}(I^2\times I^2),\quad
\varphi_{\Delta}(\pms{k}{m}) = \{\left(\pms{k}{l},\pms{l}{m}\right)\mid l\in I\},\] \[\varphi_{\epsilon}: I^2 \rightarrow \mathscr{P}(I),\quad \varphi_{\epsilon}(\pms{k}{l}) = \left\{\begin{array}{lll} \{k\}& \textrm{if } k= l,\\ \emptyset & \textrm{if } k\neq l. \end{array}\right.\]

We also use the natural tensor product of an $I$-partial algebra
$\mathscr{A}$ and $J$-partial algebra $\mathscr{B}$, which is an $I\times J$-partial algebra $\mathscr{A}\otimes \mathscr{B}$.

\begin{Def}\label{DefPartBiAlg} A \emph{partial bialgebra} $\mathscr{A}$ consists of a set $I$, the \emph{object set}, a collection of vector spaces $\Gr{A}{k}{l}{m}{n}$ with $I^2$-partial algebra structure on the $\GrDA{A}{\pms{k}{l}}{\pms{m}{n}} = \Gr{A}{k}{l}{m}{n}$, and a $\varphi_{\Delta}$-homomorphism $\mathscr{A}\rightarrow \mathscr{A}\otimes \mathscr{A}$ and $\varphi_{\epsilon}$-homomorphism $\epsilon:\mathscr{A}\rightarrow \Mat_I$ whose components turn $\GrRA{A}{(k\;l)}{(m\;n)}=\Gr{A}{k}{l}{m}{n}$ into an $I\times I$-partial coalgebra structure.\end{Def}

Spelled out, this means we have maps and elements \begin{align*} M: \Gr{A}{k}{l}{r}{s}\otimes \Gr{A}{l}{m}{s}{t} \rightarrow \Gr{A}{k}{m}{r}{t}, && \Delta_{rs}: \Gr{A}{k}{l}{m}{n}\rightarrow \Gr{A}{k}{l}{r}{s}\otimes \Gr{A}{r}{s}{m}{n}, && \UnitC{k}{l}\in \Gr{A}{k}{k}{l}{l}, && \epsilon: \Gr{A}{k}{l}{k}{l}\rightarrow \C\end{align*} satisfying (co)associativity and (co)unitality, and such that moreover $\epsilon(\UnitC{k}{k}) = 1$, $\epsilon(ab)=\epsilon(a)\epsilon(b)$ whenever $a,b$ are composable, $\Delta_{ll'}(\UnitC{k}{m}) = \delta_{l,l'}\UnitC{k}{l}\otimes \UnitC{l}{m}$ and $\Delta_{rs}(ab) = \sum_t \Delta_{rt}(a)\Delta_{ts}(b)$ whenever well-defined. Note that this last sum is finite, as it is implicit in the definition that the applications $(r,s)\rightarrow \Delta_{rs}(a)$ are rcf for each $a$. We will use the Sweedler notation $\Delta(a) = a_{(1)}\otimes a_{(2)}$ for the total comultiplication, and $\Delta_{rs}(a) = a_{(rs;1)}\otimes a_{(rs;2)}$ for its components.

It will be convenient to consider the multipliers $\lambda_k = \sum_l \UnitC{k}{l}$ and $\rho_l = \sum_k\UnitC{k}{l}$ in $M(A)$. Then for example by \cite[Proposition A.3]{VDW2}, there is a unique homomorphism $\Delta:M(A)\rightarrow M(A\otimes A)$ extending $\Delta$ and satisfying $\Delta(1) = \sum_k \rho_k\otimes \lambda_k$. It follows by elementary calculations that the total objects $(A,M,\Delta,\epsilon,\Delta(1))$ form a \emph{regular weak multiplier bialgebra} \cite[Definition 2.1 and Definition 2.3]{Boh1}. We will call it the \emph{total weak multiplier bialgebra} associated to $\mathscr{A}$.

Recall from \cite[Section 3]{Boh1} that a regular weak multiplier
bialgebra admits four projections $\Pi^L,\Pi^R,\bar{\Pi}^L,\bar{\Pi}^R:A\rightarrow M(A)$, where for example  $\bar{\Pi}^L(a) = (\epsilon\otimes \id)((a\otimes
  1)\Delta(1))$. One computes that for $a\in \Gr{A}{k}{l}{m}{n}$, one has \[ \Pi^L(a) = \epsilon(a)\lambda_k,\quad \bar{\Pi}^L(a) = \epsilon(a) \lambda_l, \quad \Pi^R(a) = \epsilon(a) \rho_n,\quad \bar{\Pi}^R(a) = \epsilon(a)\rho_m.\]

The \emph{base algebra} of $(A,\Delta)$ is therefore the algebra $\Fun_{f}(I)$ of finite support functions on $I$, and by \cite[Theorem 3.13]{Boh1} the comultiplication of $A$ is (left and right) \emph{full}, meaning that the legs of $\Delta(A)$ span $A$.

It is of more interest to consider the converse question. If $(A,\Delta)$ is a regular weak multiplier bialgebra, let us write $A^L = \Pi^L(A) = \bar{\Pi}^L(A)\subseteq M(A)$ and $A^R = \Pi^R(A)= \bar{\Pi}^R(A)\subseteq M(A)$ for the base algebras. If  $(A,\Delta)$ is full, we have by \cite[Lemma 4.8]{Boh1} that $A^L$ is anti-isomorphic to $A^R$ by the map $\sigma: A^L \rightarrow A^R$ sending $\bar{\Pi}^L(a)$ to $\Pi^R(a)$ . We then refer to $A^L$ as \emph{the} base algebra.

\begin{Prop}\label{PropCharPBA} Let $(A,\Delta)$ be a full regular weak multiplier bialgebra whose base algebra is isomorphic to $\Fun_f(I)$ for some set $I$, and such that moreover $A^LA^R \subseteq A$. Then $(A,\Delta)$ is the total weak multiplier bialgebra of a uniquely determined partial bialgebra $\mathscr{A}$ over $I$.
\end{Prop} 

The condition $A^LA^R \subseteq A$ is essential and should be considered as a \emph{properness} condition. % Ref to Timmermann?

\begin{proof} Write $\lambda_k \in A^L$ for the function $\lambda_k(l) = \delta_{kl}$, and write $\sigma(\lambda_k) = \rho_k\in A^R$. By assumption, $\UnitC{k}{l} = \lambda_k\rho_l\in A$. Further $A= AA^R = AA^L = A^LA=A^RA$, cf.~ the proof of \cite[Theorem 3.13]{Boh1}. Hence the $\UnitC{k}{l}$ make $A$ into the total algebra of an $I^2$-partial algebra, as $A^L$ and $A^R$ pointwise commute by \cite[Lemma 3.5]{Boh1}. 

Let us show first that $\Delta(1) = \sum_k \rho_k\otimes \lambda_k$. By \cite[Lemma 3.9]{Boh1}, we have $(\rho_k\otimes 1)\Delta(a) = (1\otimes \lambda_k)\Delta(a)$ for all $a$. By \cite[Lemma 4.10]{Boh1} and the fact that $\Delta(1)$ is an idempotent, we can then write $\Delta(1)=\sum_{k\in I'} \rho_k\otimes \lambda_k$ for some subset $I'\subseteq I$. As by definition $\bar{\Pi}^L(A) = \Fun_{f}(I)$, we deduce that $I=I'$. We then have as well that $\Delta(\UnitC{k}{m}) = \sum_l \UnitC{k}{l}\otimes \UnitC{l}{m}$ by \cite[Lemma 3.3]{Boh1}, and it follows that $\Delta$ is a $\varphi_{\Delta}$-homomorphism in the sense of Definition \ref{DefPartBiAlg}.

For $a\in \Gr{A}{k}{l}{p}{q}$ and $b\in \Gr{A}{l}{m}{q}{r}$, we then have $\epsilon(ab) = \epsilon(a\UnitC{l}{q}b) = \epsilon(a)\epsilon(b)$ by \cite[Proposition 2.6.(4)]{Boh1}, which shows the partial multiplicativity of $\epsilon$. Finally, assume that $k$ was such that $\epsilon(\UnitC{k}{k})=0$. By the partial multiplication law, $\epsilon$ is zero on all $\Gr{A}{k}{l}{k}{l}$. Applying $\Delta_{kl}$ to $\Gr{A}{k}{l}{m}{n}$ and using the counit property on the first leg, it follows that $\Gr{A}{k}{l}{m}{n}=0$ for all $l,m,n$. In particular, $\UnitC{k}{m}=0$ for all $m$. This entails $\lambda_k=0$, a contradiction. Hence $\epsilon(\UnitC{k}{k})\neq 0$, so $\epsilon(\UnitC{k}{k})=1$ by partial multiplicativity. This shows that $\epsilon$ is a $\varphi_{\epsilon}$-homomorphism.

As $\Delta$ is coassociative and $\epsilon$ satisfies the counit property, it is clear that the components of $\Delta$ and $\epsilon$ satisfy the conditions for a partial coalgebra, which finishes the proof.
\end{proof} 

\subsection{Partial Hopf algebras}

\begin{Def}\label{DefPartBiAlgAnt}A partial bialgebra $\mathscr{A}$ is called a \emph{partial Hopf algebra} if it admits an \emph{antipode}, a collection of linear
maps $S:\Gr{A}{k}{l}{m}{n}\rightarrow \Gr{A}{n}{m}{l}{k}$ such that $\sum_s a_{(rs;1)}S(a_{rs;2})= \epsilon(a)\UnitC{k}{r}$ and $\sum_r S(a_{(rs;1)})a_{(rs;2)}= \epsilon(a)\UnitC{s}{n}$ for all $a \in \Gr{A}{k}{l}{m}{n}$. It is called a \emph{regular} partial Hopf algebra if all $S$ are invertible.
\end{Def} 

An antipode can be linearly extended to a map $S: A\rightarrow A$, which then satisfies the defining relations $ba_{(1)}S(a_{(2)}) = b\Pi^{L}(a)$ and $S(a_{(1)})a_{(2)}b = \Pi^{R}(a)b$ for all $a,b\in A$. Taking $a= \UnitC{k}{k}$ and $b=\UnitC{l}{k}$, one obtains $S(\UnitC{k}{l}) = \UnitC{l}{k}$ for all $k,l\in I$.

\begin{Def} \label{remark:index-equivalence}
The \emph{hyperobject} set of a partial Hopf algebra $\mathscr{A}$ is the set of equivalence classes in $I$ for the equivalence relation $
    k \sim l \Leftrightarrow \UnitC{k}{l} \neq 0$.
\end{Def} 
Note that transitivity follows from the identity $\Delta_{ll}(\UnitC{k}{m}) = \UnitC{k}{l}\otimes \UnitC{l}{m}$.

\begin{Prop}  \label{theorem:partial-hopf-algebra}  Let $\mathscr{A}$ be a partial bialgebra. Then $S:A\rightarrow A$ is an antipode for $\mathscr{A}$ if and only if it is an antipode for the total weak multiplier bialgebra $(A,\Delta,\epsilon)$ in the sense of \cite[Section 6]{Boh1}.

\end{Prop}
\begin{proof} We will show equivalence with the antipode definition as in \cite[Theorem 6.8.(2)(vii)]{Boh1}. Assume first that $\mathscr{A}$ has an antipode. Then for $b\in \Gr{A}{k}{l}{m}{n}$ and $a,c\in A$ arbitrary we have $a_{(1)}b_{(1)} \otimes a_{(2)}b_{(2)}S(b_{(3)})c = a_{(1)}b_{(1)} \otimes a_{(2)}\Pi^L(b_{(2)})c$. As $b_{(1)}\otimes \Pi^L(b_{(2)}) = \rho_mb\otimes \lambda_m$, identity \cite[Theorem 6.8.(2)(vii)]{Boh1} follows. Identity \cite[Theorem 6.8.(2)(viii)]{Boh1} follows similarly, and \cite[Theorem 6.8.(2)(ix)]{Boh1} is immediate from the condition $S(\Gr{A}{k}{l}{m}{n}) = \Gr{A}{n}{m}{l}{k}$. 

Conversely, given a map $S$ satisfying the conditions of \cite[Theorem 6.8.(2)]{Boh1}, it follows from \cite[Lemma 6.14]{Boh1} that $S(\Gr{A}{k}{l}{m}{n}) = \Gr{A}{n}{m}{l}{k}$. From the identities (6.14) in \cite{Boh1}, we obtain that $ab_{(1)}S(b_{(2)})c = \epsilon(a_{(1)}b)a_{(2)}$ for all $a,b,c$. Taking $a=\UnitC{k}{r}$ and $c=\UnitC{m}{r}$, we find $\sum_s b_{(rs;1)}S(b_{rs;2)}) = \epsilon(b)\UnitC{m}{r}$ for all $b\in \Gr{A}{k}{l}{m}{n}$. The other antipode identity follows similarly.
\end{proof}

\begin{Cor} \label{corollary:antipode} Let $\mathscr{A}$ be a partial
  Hopf algebra. Then the total antipode $S:A\rightarrow A$ is uniquely determined and satisfies
  $S(ab) = S(b)S(a)$ and $\Delta(S(a)) = (S\otimes S)\Delta^{\op}(a)$
  for all $a,b\in A$.
\end{Cor} 
\begin{proof} This follows from \cite[Theorem 6.8, Theorem 6.12 and Corollary 6.16]{Boh1}. 
\end{proof} 

From the uniqueness of the antipode, it follows immediately that $S^{-1}$ is an antipode for $(\mathscr{A},\Delta^{\op})$ when $\mathscr{A}$ is regular. Conversely, if both $(\mathscr{A},\Delta)$ and $(\mathscr{A},\Delta^{\op})$ have antipodes, then $(\mathscr{A},\Delta)$ is a regular partial Hopf algebra. 

\begin{Lem}\label{LemCoAnt} Let $(\mathscr{A},\Delta)$ be a partial Hopf algebra. Then $\epsilon\circ S = \epsilon$ on each $\Gr{A}{k}{l}{m}{n}$.
\end{Lem}

\begin{proof} For $a\in \Gr{A}{k}{l}{k}{l}$, we have $\epsilon(S(a)) = \epsilon(a_{(kl;1)})\epsilon(S(a_{(kl;2)}) = \sum_s \epsilon(a_{(ks;1)})\epsilon(S(a_{(ks;2)}))$. By partial multiplicativity of $\epsilon$, this equals  $\epsilon\left( \sum_s a_{(ks;1)}S(a_{(ks;2)})\right) = \epsilon(a)\epsilon(\UnitC{k}{k})= \epsilon(a)$. 
\end{proof} 

\subsection{Invariant integrals}

\begin{Def}
  Let $\mathscr{A}$ be an $I$-partial bialgebra.  We call a family of
  functionals
\begin{align} \label{eq:functionals}
  \phi = \phic{k}{m} \colon \Gr{A}{k}{k}{m}{m} \to \C
\end{align}
an \emph{invariant} \emph{integral} if
 $\phi(\UnitC{k}{k})=1$ for all $k$ and
\begin{align}
  \label{eq:integral}
   (\id \otimes \phi)\Delta_{rr}(a) 
&= \phi(a)
  \UnitC{k}{r}, &   (\phi \otimes
  \id)\Delta_{rr}(a)&=\phi(a) \UnitC{r}{m}
\end{align}
 for all $a \in A\pmat{k}{l}{m}{n}$. 
\end{Def}

Here, as before, we interpret $\phi$ as the zero functional on the parts on which it is not defined. One easily checks that the linear extension of $\phi$ to $A$ satisfies the total invariance conditions \begin{align*}
(\id\otimes \phi)((b\otimes 1)\Delta(a)) &= \sum_{k}\phi(\lambda_{k}a)b\lambda_k,&  (\phi\otimes \id)(\Delta(a)(1\otimes b)) &= \sum_{m}
\phi(\rho_{m} a)\rho_m b.\end{align*}

Note that $\phi(\UnitC{k}{m})=1$ for all $k,m$ \emph{with $\UnitC{k}{m}\neq 0$}, by applying $(\id\otimes \phi)$ to $\Delta_{kk}(\UnitC{m}{m})$. One further has that $\phi$ is uniquely determined, since any other invariant integral $\psi$ satisfies \begin{align*}  \phi(a)  &= (\psi \otimes
      \phi)(\Delta_{kk}(a)) = \psi(a)\phi(\UnitC{k}{m}) = \psi(a),\qquad a\in \Gr{A}{k}{k}{m}{m}.
    \end{align*}
%Hence if $\mathscr{A}$ is a  \emph{regular}  partial Hopf algebra with antipode $S$, any invariant integral $\phi$ satisfies $\phi=\phi S$.

Note that our normalisation of $\phi$ is different from the one in \cite{Hay1}, where rather the $\lambda_k$ and $\rho_m$ were assigned weight one. That normalisation does however not make sense in case $I$ is infinite.

We have the following form of \emph{strong invariance}, see \cite[Lemma 3.4]{Hay1}

\begin{Lem} \label{lemma:strong-invariance}
  Let $\mathscr{A}$ be a partial Hopf algebra with invariant integral $\phi$. Then
  for all $a\in A$,
  \begin{align*}
    S\left(( \id\otimes
    \phi)(\Delta(b)(1 \otimes a))\right) &= (\id \otimes \phi)((1 \otimes b)\Delta(a)),\\  S\left((\phi \otimes
    \id)((a\otimes 1)\Delta(b))\right) &= (\phi \otimes \id)(\Delta(a)(b\otimes 1)).\end{align*}
\end{Lem}



\subsection{Partial compact quantum groups}
 
\begin{Def} A \emph{partial $*$-algebra} $\mathscr{A}$ is a partial
  algebra whose total algebra $A$ is equipped with an antilinear,
  antimultiplicative involution $ a\mapsto
  a^*$ with $\mathbf{1}_k^*=\mathbf{1}_k$ for all $k$ in
  the object set. 
\end{Def} 

This implies that $*$ restricts to antilinear maps $\GrDA{A}{k}{l}\rightarrow \GrDA{A}{l}{k}$.

\begin{Def} A \emph{partial $*$-bialgebra} $\mathscr{A}$ is a
 partial bialgebra whose underlying partial algebra has been
  endowed with a partial $*$-algebra structure such that
$\Delta_{rs}(a)^* = \Delta_{sr}(a^*)$ for all $a \in \Gr{A}{k}{l}{m}{n}$.
A \emph{partial Hopf $*$-algebra} is a partial bialgebra which is at the same time a partial $*$-bialgebra and a partial Hopf algebra.
\end{Def} 

\begin{Prop} \label{cor:involutive}
  An $I$-partial $*$-bialgebra $\mathscr{A}$ is an $I$-partial Hopf
  $*$-algebra if and only if the weak multiplier $*$-bialgebra
  $(A,\Delta)$ is a weak multiplier Hopf $*$-algebra. In that case,
  the counit and antipode satisfy
  $\epsilon(a^{*})=\overline{\epsilon(a)}$ and $S(S(a)^{*})^{*}=a$ for
  all $a\in A$. In particular, the total antipode is bijective.
\end{Prop}
\begin{proof}
  The if and only if part follows immediately from  Theorem
  \ref{theorem:partial-hopf-algebra}, the relation for the counit  from
uniqueness of the counit  \cite[Theorem 2.8]{Boh1}, and the relation
for the antipode from \cite[Proposition 4.11]{VDW1}.
\end{proof}

\begin{Def}\label{DefPCQG} A \emph{partial compact quantum group} $\mathscr{G}$ consists of a
  partial Hopf $*$-algebra $\mathscr{A} = P(\mathscr{G})$ with an invariant integral  $\phi$ that is positive in the sense  that $\phi(a^*a)\geq 0$ for all $a\in A$. We also say that $\mathscr{G}$ is the partial compact quantum group \emph{defined by} $\mathscr{A}$.
\end{Def} 

It will follow from Proposition \ref{prop:rep-cosemisimple} and  \cite[Theorem 3.3 and Theorem 4.4]{Hay1} that for $I$ finite, a partial compact quantum group is precisely a compact quantum group of face type \cite[Definition 4.1]{Hay1}. As the total structure should not be considered compact for $I$ infinite, we have changed the terminology to \emph{partial} compact quantum group to reflect that only the parts should be considered compact.

\subsection{Partial tensor categories}\label{SecPartTen}

We now categorify the notion of a partial algebra. By `category' and
`functor' we will by default mean a (small) $\C$-linear additive
category and a $\C$-linear functor. 

\begin{Def} A \emph{partial tensor category} $\CatCC$ over a set $\mathscr{I}$ consists of  a collection of categories $\mathcal{C}_{\alpha\beta}$ with $\alpha,\beta\in \mathscr{I}$, a family of $\C$-bilinear functors $\otimes = \otimes_{\alpha,\beta,\gamma}: \CatC_{\alpha\beta}\times \CatC_{\beta\gamma}\rightarrow \CatC_{\alpha\gamma}$, natural isomorphisms $a_{X,Y,Z}: (X\otimes Y)\otimes Z \rightarrow X\otimes (Y\otimes Z)$ for $X \in \CatC_{\alpha\beta},Y\in \CatC_{\beta\gamma},Z\in \CatC_{\gamma\delta}$, non-zero objects $\Unitb_{\alpha} \in \CatC_{\alpha\alpha}$ and natural isomorphisms $\lambda_X^{(\alpha)}:\Unitb_\alpha\otimes X \rightarrow X$ and $\rho_X^{(\beta)}:X\otimes \Unitb_\beta\rightarrow X$ for $X\in \CatC_{\alpha\beta}$, satisfying the obvious associativity and unit constraints. 
\end{Def}

Note that one can also interpret this structure as a 2-category, but again our notion of morphisms will be different from the usual one.

There is no problem in modifying Maclane's coherence theorem to partial tensor categories, and we will henceforth assume that our partial tensor categories are strict to lighten notation. 

The total notion corresponding to a partial tensor category is that of a \emph{tensor category with local units (indexed by $\mathscr{I}$)}, that is, a tensor category without unit $(\CatC,\otimes,a)$  for which there exists a collection $\{\Unitb_\alpha\}_{\alpha\in \mathscr{I}}$ of objects such that $\Unitb_\alpha\otimes \Unitb_\beta \cong 0$ for each $\alpha\neq \beta$, such that for each object $X$ one has $\Unitb_\alpha\otimes X \cong 0 \cong X\otimes \Unitb_\alpha$ for all but a finite set of $\alpha$, and with fixed natural isomorphisms $\lambda_X:\oplus_\alpha (\Unitb_\alpha\otimes X) \rightarrow X$ and $\rho_X:\oplus_\alpha(X\otimes \Unitb_\alpha)\rightarrow X$ satisfying the obvious unit conditions. For example, if $\CatC$ is a tensor category with local units, one can put \[X_{\alpha\beta} = \Unitb_\alpha\otimes X \otimes \Unitb_\beta, \quad \eta_{\alpha\beta}:X_{\alpha\beta} \rightarrow \oplus_{\gamma,\delta} \left(\Unitb_\gamma \otimes X \otimes \Unitb_\delta\right) \cong X.\]  Then the $\CatC_{\alpha\beta} = \{X \in \CatC\mid X_{\alpha\beta} \underset{\eta_{\alpha\beta}}{\cong} X\}$, seen as full subcategories of $\CatC$, form a partial tensor category upon restriction of $\otimes$, and one easily verifies that this defines an equivalence between partial tensor categories and tensor categories with local units. 

Continuing the analogy with the algebra case, one can define the enveloping \emph{multiplier tensor category} $M(\CatC)$ of a tensor category with local units. Its objects consist of formal rcf sums $\oplus_{\alpha,\beta\in \mathscr{I}} X_{\alpha\beta}$, with $\Mor(\oplus X_{\alpha\beta},\oplus Y_{\alpha\beta}) =\prod_{\alpha\beta} \Mor(X_{\alpha\beta},Y_{\alpha\beta})$ as morphism spaces, the composition being entry-wise. The tensor product of $\CatC$ extends to $M(\CatC)$ by putting $\left(\oplus X_{\alpha\beta}\right)\otimes \left(\oplus Y_{\alpha\beta}\right) = \oplus_{\alpha,\beta,\gamma} \left(X_{\alpha\beta}\otimes Y_{\beta\gamma}\right)$, and similarly for morphism spaces, using the rcf condition. The associativity constraints of the $\CatC_{\alpha\beta}$ can be summed to an associativity constraint for $M(\CatC)$, while $\Unitb := \oplus_{\alpha\in \mathscr{I}} \Unitb_\alpha$ becomes a unit for $M(\CatC)$, rendering $M(\CatC)$ into an ordinary tensor category with unit object.



As an example, consider a set $I$. Then we can form the partial tensor category $\CatCC = \{\Vect_{\fin}\}_{i,j\in I}$ where each $\CatC_{ij}$ is a copy of the category of finite-dimensional vector spaces $\Vect_{\fin}$, and with each $\otimes$ the ordinary tensor product. The total category $\CatC$ can be identified with the category $\Vectif$ of finite-dimensional bi-graded vector spaces with the `balanced' tensor product over $I$,  \[\Gru{(}{k}{}V\itimes W\Gru{)}{}{m} = \oplus_l \;(\Gru{V}{k}{l}\otimes \Gru{W}{l}{m})\subseteq V\otimes W.\] The multiplier category $M(\Vectif)$ equals $\Vectrcf$, the category of bigraded vector spaces which are rcfd in the sense of the following definition.

\begin{Def}\label{Defrcfd} Let $I$ be a set. An $I\times I$-graded vector space $V=\bigoplus_{k,l\in I} \GrDA{V}{k}{l}$ will be called \emph{row-and column finite-dimensional} (rcfd) if the $\oplus_{l\in I} \GrDA{V}{k}{l}$ (resp.~ $\oplus_{k\in I} \GrDA{V}{k}{l}$) are finite-dimensional for each $k$ (resp.~ $l$) fixed. 
\end{Def} 

Let us introduce morphisms between partial tensor categories.

\begin{Def} Let $\CatCC$ and $\CatDD$ be partial tensor categories over respective sets $\mathscr{I}$ and $\mathscr{J}$, and let $\varphi: \mathscr{I}\rightarrow \mathscr{P}(\mathscr{J})$. A \emph{$\varphi$-morphism} from $\mathscr{C}$ to $\mathscr{D}$ is a functor $F:\CatC \rightarrow M(\CatD)$ with isomorphisms $\iota_{X,Y}:F(X)\otimes F(Y)\cong F(X\otimes Y)$ and $\mu_\alpha:\oplus_{k\in \varphi(\alpha)} \Unitb_k \cong F(\Unitb_\alpha)$ satisfying the natural coherence conditions. 
\end{Def}

Note that necessarily $\varphi(\alpha)\cap \varphi(\beta)=\emptyset$ for
$\alpha\neq \beta$, so that $\varphi$ corresponds to an $I$-labeled
partition $\{I_{\alpha}\}$ of $J$ allowing however $I_{\alpha}=\emptyset$.  A $\varphi$-morphism splits into components $F_{kl}:
\CatC_{\alpha\beta} \rightarrow \CatD_{kl}$ for $k\in\varphi(\alpha),l\in
\varphi(\beta)$, with $(k,l)\mapsto F_{kl}(X)$ rcf on its domain for each
fixed $X$. One further obtains
\emph{monomorphisms} \[\iota^{(klm)}_{X,Y}:F_{kl}(X) \otimes F_{lm}(Y)
\hookrightarrow F_{km}(X\otimes Y), \qquad X\in
\CatC_{\alpha\beta},Y\in \CatD_{\beta\gamma}\] whose ranges form a
direct decomposition of the image when $l$ varies over
$\varphi(\beta)$. Similarly, we have isomorphisms $\mu_{k}: \Unitb_k
\cong F_ {kk}(\Unitb_{\alpha})$ for $k\in \varphi(\alpha)$, and
$F_{kl}(\Unitb_{\alpha})= 0$ if $k\neq l$ in $\varphi(\alpha)$.


\begin{Def} A partial tensor category $\CatCC$ is called \emph{semi-simple} if all $\CatC_{\alpha\beta}$ are semi-simple. A partial tensor category is said to have \emph{irreducible units} if all units $\Unitb_\alpha$ are irreducible. 
\end{Def}

One may always partition a semi-simple partial  tensor category into one with irreducible units.

\begin{Def} Let $\CatCC$ be a partial tensor category. An object $X\in \CatC_{\alpha\beta}$ is said to admit a \emph{left dual} if there exists an object $Y=\hat{X} \in \CatC_{\beta\alpha}$ and morphisms $\ev_{X}: Y\otimes X \rightarrow \Unitb_\beta$ and $\coev_X: \Unitb_\alpha\rightarrow X\otimes Y$ satisfying the obvious snake identities. We say $\CatCC$ \emph{admits left duality} if each object of each $\CatC_{\alpha\beta}$ has a left dual.
\end{Def}

Similarly, one defines right duality $X\rightarrow \check{X}$ and (two-sided) duality $X\rightarrow \bar{X}$. As for tensor categories with unit, if $X$ admits a (left or right) dual, it is unique up to canonical isomorphism, and if $X$ has left dual $\hat{X}$, then $X$ is a right dual to $\hat{X}$. Moreover, if $F$ is a morphism $\CatCC\rightarrow \CatDD$ based over $\varphi:\mathscr{I}\rightarrow \mathscr{P}(\mathscr{J})$, and if $X\in \CatC_{\alpha\beta}$ has a left dual, then $F_{lk}(\hat{X})$ is a left dual to $F_{kl}(X)$ for all $k\in \varphi(\alpha),l \in\varphi(\beta)$.

\begin{Def}\label{DefParFus} A \emph{partial fusion C$^*$-category} is
  a partial tensor category $(\CatC,\otimes)$ with duality such that
  all components $\CatC_{\alpha\beta}$ are semi-simple C$^*$-categories,  all functors $\otimes$ are $^*$-functors (in the sense that $(f\otimes g)^* = f^*\otimes g^*$ for morphisms), and the associativity and unit constraints are unitary.
\end{Def} 

Note that we slightly abuse the terminology `fusion', as strictly speaking this would require there to be only a finite set of mutually non-equivalent irreducible objects in each $\CatC_{\alpha\beta}$. 

The notion of a morphism for partial semi-simple tensor C$^*$-categories has to be adapted by requiring that all $F_{kl}$ are $^*$-functors and the $\iota$- and $\mu$-maps isometric.  Note that a $\varphi$-morphism of partial fusion C$^*$-categories is automatically faithful if $\varphi(\alpha)\neq \emptyset$ for all $\alpha$. Indeed, by semisimplicity a non-faithful morphism sends some irreducible object to zero. By the duality assumption this means that some irreducible unit $\mathbbm{1}_{\alpha}$ is sent to zero, which is impossible since $F(\mathbbm{1}_{\alpha})$ contains $\mathbbm{1}_k$ for some $k \in \varphi(\alpha)$.

As an example of partial fusion C$^*$-category, consider a set  $I$. Then we can form the partial fusion C$^*$-category $\CatCC = \{\Hilb_{\fin}\}_{I\times I}$ of finite-dimensional Hilbert spaces, with all $\otimes$ the ordinary tensor product. The associated global category is the category $\Hilbif$ of finite-dimensional bi-graded Hilbert spaces. The dual of a Hilbert space $\Hsp \in \CatC_{kl}$ is the ordinary dual Hilbert space $\Hsp^* \cong \overline{\Hsp}$, considered in the category $\CatC_{lk}$. 

% Ref to Gaby's comodule paper

\section{Representation theory of partial compact quantum groups}

\subsection{Corepresentations of partial bialgebras}

We denote by  $\Vectrcf$ the category whose objects are rcfd $I\times I$-graded vector spaces, see Definition \ref{Defrcfd}. Morphisms are linear maps $T$ that preserve the grading and therefore
can be written $T=\prod_{k,l\in I} \GrDA{T}{k}{l}$. 

\begin{Def} \label{definition:corep} Let $\mathscr{A}$ be an
  $I$-partial bialgebra. A \emph{corepresentation}
  $\mathscr{X}$ of $\mathscr{A}$ on an rcfd $I\times I$-graded vector space $V$
  is a family of elements $\Gr{X}{k}{l}{m}{n} \in \Gr{A}{k}{l}{m}{n} \otimes
  \Hom_\C(\Gru{V}{m}{n},\Gru{V}{k}{l})$
 satisfying $ (\Delta_{pq} \otimes
    \id)(\Gr{X}{k}{l}{m}{n}) =
    \Big{(}\Gr{X}{k}{l}{p}{q}\Big{)}_{13}\Big{(}\Gr{X}{p}{q}{m}{n}\Big{)}_{23}$ and
     $(\epsilon \otimes
  \id)(\Gr{X}{k}{l}{m}{n})=\delta_{k,m}\delta_{l,n}\id_{\Gru{V}{k}{l}}$.
\end{Def}
We use here the standard leg numbering notation, e.g.~ $a_{23}=1\otimes a$. Whenever we talk of a corepresentation, it will be assumed that the underlying bigraded vector space is rcfd.

 For example, equip the vector space
  $\C^{(I)}=\bigoplus_{k\in I} \C$ with the diagonal
  $I\times I$-grading. Then the family $\mathscr{U}$ given by $\Gr{U}{k}{l}{m}{n} = \delta_{k,l}\delta_{m,n} \UnitC{k}{m} \in
    \Gr{A}{k}{l}{m}{n}$
is a corepresentation of $\mathscr{A}$ on $\C^{(I)}$, called the
\emph{trivial corepresentation}. As another example,  assume  given an rcfd family of subspaces
  $\Gru{V}{m}{n} \subseteq \bigoplus_{k,l} \Gr{A}{k}{l}{m}{n}$ satisfying
  $\Delta_{pq}(\Gru{V}{m}{n}) \subseteq \Gru{V}{p}{q} \otimes
    \Gr{A}{p}{q}{m}{n}$ for all indices.
Then the  elements $\Gr{X}{k}{l}{m}{n} \in \Gr{A}{k}{l}{m}{n} \otimes
  \Hom_{\C}(\Gru{V}{m}{n},\Gru{V}{k}{l})$ defined by 
  \begin{align*}
    \Gr{X}{k}{l}{m}{n}(1 \otimes b) &= \Delta^{\op}_{kl}(b) \in
    \Gr{A}{k}{l}{m}{n} \otimes \Gru{V}{k}{l} \quad
    \text{for all } b\in \Gru{V}{m}{n}
  \end{align*}
  form a corepresentation $\mathscr{X}$ of $\mathscr{A}$ on
  $V$. Corepresentations of this form will be called
  \emph{regular corepresentations}. 

 A morphism  $T$ between corepresentations
  $(V,\mathscr{X})$ and $(W,\mathscr{Y})$ of $\mathscr{A}$ will be a family
  of linear maps $\Gru{T}{k}{l} \in
  \Hom_\C(\Gru{V}{k}{l},\Gru{W}{k}{l})$ satisfying the intertwiner property $(1 \otimes
  \Gru{T}{k}{l})\Gr{X}{k}{l}{m}{n} = \Gr{Y}{k}{l}{m}{n}(1 \otimes
  \Gru{T}{m}{n})$.  In this way corepresentations form a category which we will denote
$\Corep_{\rcf}(\mathscr{A})$.

We next consider the total form of a corepresentation. Let $V$ be an rcfd $I\times I$-graded vector space, and write $\IdCV{k}{l}$ for the projections on the component $\GrDA{V}{k}{l}$. Write $\End_0(V)$ for the algebra of endomorphisms on $V$ having finite-dimensional support. An element $X \in  M(A
  \otimes \End_{0}(V))$ is called a \emph{total corepresentation} if \begin{align*} \Gr{X}{k}{l}{m}{n} := (\UnitC{k}{m} \otimes \id){X}(\UnitC{l}{n}
    \otimes \id) = (1 \otimes \IdCV{k}{l}){X}(1 \otimes
    \IdCV{m}{n}) \in A \otimes \End_0(V),&& \forall k,l,m,n,\end{align*} \begin{align*} (\Delta\otimes \id)(X)=X_{13}X_{23}, && (\epsilon \otimes \id)({X}) = \id_{V},\end{align*} where the latter identities make sense as identities of multipliers by the first condition. It is easily verified that the $\Gr{X}{k}{l}{m}{n}$ then define a corepresentation $\mathscr{X}$ of $\mathscr{A}$, and that all corepresentations arise in this way from a unique total corepresentation. We call $X$ the \emph{corepresentation multiplier} of $\mathscr{X}$. It is not hard to show that total corepresentations are in unique correspondence with full comodule structures for $(A,\Delta)$ with rcfd induced bigrading, cf.~ \cite[Definition 2.2, Definition 4.2 and Theorem 4.5]{Boh2}.

The category $\Corep_{\rcf}(\mathscr{A})$ is an abelian category, with the forgetful functor from $\Corep_{\rcf}(\mathscr{A})$ to $\Vectrcf$  lifting kernels, cokernels and biproducts. In particular, one can call a corepresentation $V$ \emph{irreducible} if any morphism $T$ from (resp.~ into) $V$ either has all $T_{kl}$ zero or injective (resp.~ surjective).

Assume now that $\mathscr{A}$ is a partial Hopf algebra. Given a corepresentation $(V,\mathscr{X})$ of $\mathscr{A}$, the element \[X^{-1} =  (S \otimes \id)(X) \in M(A \otimes \End_{0}(V))\] is a generalized inverse of $X$ in the sense that $XX^{-1}X=X$ and $X^{-1}XX^{-1}=X^{-1}$. More precisely, with $\lambda_k^V$ (resp.~ $\rho_k^V$) denoting the projection in $\End_0(V)$ onto elements with left (resp.~ right) grading equal to $k$, one has $XX^{-1} = \sum_{k} \lambda_{k} \otimes \lambda^{V}_{k}$ and
    $X^{-1}X  = \sum_{l} \rho_{l} \otimes \rho^{V}_{l}$ w.r.t. strict convergence. We then write
   \[\Gr{(X^{-1})}{k}{l}{m}{n}=(S \otimes \id)(\Gr{X}{n}{m}{l}{k}) \in
   \Gr{A}{k}{l}{m}{n} \otimes \Hom_{\C}(\Gru{V}{l}{k},\Gru{V}{n}{m})\] for the components.

The following easy lemma will be very useful.
\begin{Lem} \label{lemma:rep-total-morphism}
 Let $\mathscr{A}$ be a partial Hopf algebra. A bigraded map $T$ defines a morphism from 
    $(V,\mathscr{X})$ to $(W,\mathscr{Y})$ if and only if one (and hence both) of the following relations hold:
    \begin{align*}
      Y^{-1}(1 \otimes T)X&=\sum_{m,n} \rho_{n} \otimes \Gru{T}{m}{n},
      &
    Y(1\otimes T)X^{-1} &=\sum_{k,l} \lambda_{k} \otimes \Gru{T}{k}{l}.
    \end{align*}
\end{Lem}


\subsection{Tensor products and duality}

Recall from Section \ref{SecPartTen} that the category $\Vectrcf$ of rcfd vector spaces is a tensor category for the $I$-balanced tensor product $\itimes$, the tensor product of morphisms being the restriction of the ordinary tensor product. For the sake of convenience we interpret this product as being strictly associative.  The unit for this product is the vector space $\C^{(I)}=\bigoplus_{k\in I} \C$. 

Given $V$ and $W$ in $\Vectrcf$, we identify $\Hom_\C(\Gru{V}{m}{n},\Gru{V}{k}{l})\otimes  \Hom_\C(\Gru{W}{n}{q},\Gru{W}{l}{p})$ with a subspace of $ \Hom_\C(\Gru{(}{m}{}V\itimes  W\Gru{)}{}{q},\Gru{(}{k}{}V\itimes W\Gru{)}{}{p})$. Then if $\mathscr{A}$ is a partial bialgebra, one defines a tensor product $(V\itimes W,\mathscr{X}\Circt \mathscr{Y})$ of corepresentations  $(V,\mathscr{X})$ and $(W,\mathscr{Y})$ by the formula \[\Gr{(X\Circt Y)}{k}{p}{m}{q} = \sum_{l,n}  \left(\Gr{X}{k}{l}{m}{n}\right)_{12}\left(\Gr{Y}{l}{p}{n}{q}\right)_{13},\] where the sum is seen to be finite and hence well-defined. Its associated corepresentation multiplier  is simply $X_{12}Y_{13}$. With this tensor product,  $\Corep_{\rcf}(\mathscr{A})$ becomes a strict tensor category with unit the trivial corepresentation $(\C^{(I)},\mathscr{U})$. Its natural forgetful functor into $\Vectrcf$ is a strong monoidal functor.

Assume now that $\mathscr{A}$ is an $I$-partial Hopf algebra.  If then $(I_{\alpha})_{\alpha\in \mathscr{I}}$ is the decomposition of $I$ into
  hyperobject classes, we have that each $\C^{(I_{\alpha})} \subseteq \C^{(I)}$ is invariant, and determines a decomposition $\mathscr{U}=\bigoplus_{\alpha\in\mathscr{I}}
  \mathscr{U_{\alpha}}$ of the trivial corepresentation into irreducibles. Denote by $\Corep(\mathscr{A})$ the category of corepresentations $(V,\mathscr{X})$ for which there exists a finite subset of the hyperobject set $\mathscr{I}$ such that $\Gru{V}{k}{l}=0$ for the equivalence classes of $k,l$ outside this subset.  Then it is easily seen that $\Corep(\mathscr{A})$ is  a tensor category with the $\mathscr{U}_{\alpha}$ as local units, and with $\Corep_{\rcf}(\mathscr{A})$ as its multiplier category. We will use the same notation $\Corep(\mathscr{A})$ for the associated partial tensor category. 
  
Moreover, still under the assumption that $\mathscr{A}$ is an $I$-partial Hopf algebra, we have that the tensor category $\Corep_{\rcf}(A)$ has left duality. To see this, let $(V,\mathscr{X})$ be a corepresentation. Denote the dual of vector spaces $V$ and  linear maps $T$ by
$\dual{V}$ and $\dualop{T}$, respectively, and define the dual of an
$I\times I$-graded vector space $V=\bigoplus_{k,l} \Gru{V}{k}{l}$ to be
the space $\duall{V}=\bigoplus_{k,l} \Gru{(\duall{V})}{k}{l}$ where $\Gru{(\duall{V})}{k}{l} = \dual{(\Gru{V}{l}{k})}$. Then using anti-comultiplicativity of $S$ and Lemma \ref{LemCoAnt}, we see that $\duall{V}$ and the family
  $\dualco{\mathscr{X}}$ given by \[\Gr{\dualco{X}}{k}{l}{m}{n}   :=  (S \otimes \dualop{-})(\Gr{X}{n}{m}{l}{k})\]
   form a corepresentation of $\mathscr{A}$. To see that it is a left dual of $\mathscr{X}$, consider the natural evaluation and coevaluation maps \begin{align*} \ev \colon \duall{V} \itimes V \to \C^{(I)},\quad \omega \otimes v\mapsto \omega(v), && \coev \colon \C^{(I)} \to V\itimes \duall{V}, \quad \delta_k \mapsto \sum_{l,i} e_i^{(kl)}\otimes \omega_i^{(lk)},\end{align*} where $\{e_i^{(kl)}\}$ and $\{\omega_i^{(lk)}\}$ are a basis and its dual basis for $\GrDA{V}{k}{l}$. These provide the duality between $V$ and $\duall{V}$ in $\Vectrcf$. It then suffices to show that they  are also morphisms from the trivial corepresentation to the tensor product representations of $\mathscr{X}$ with $\dualco{\mathscr{X}}$. But for example the intertwining property of $\ev$ follows from 
  \begin{align*}
    (1\otimes \Gru{\ev}{k}{k})
 \sum_{l,n}  \big(
\Gr{\dualco{X}}{k}{l}{m}{n}\big)_{12}
\big(\Gr{X}{l}{k}{n}{q}\big)_{13} &=
    (1\otimes \Gru{\ev}{k}{k})
 \sum_{l,n} 
(S \otimes \dualop{-})(\Gr{X}{n}{m}{l}{k})_{12}
    (\Gr{X}{l}{k}{n}{q})_{13} \\ &=
\delta_{m,q}(1\otimes \Gru{\ev}{m}{m})  \sum_{l,n}
      (S \otimes \id)(\Gr{X}{n}{m}{l}{k})_{13}(\Gr{X}{l}{k}{n}{q})_{13} \\
    &= \delta_{m,q}\UnitC{k}{q}\otimes \Gru{\ev}{m}{m} \\
    &= \Gr{U}{k}{k}{m}{q}(1 \otimes \Gru{\ev}{m}{m}).
  \end{align*}
  
Similarly, if $\mathscr{A}$ is a \emph{regular} partial Hopf algebra, then any corepresentation has right duals. It follows that in this case $\Corep(\mathscr{A})$ is a partial tensor category with left and right duals.


\subsection{Decomposition into irreducible corepresentations and matrix coefficients}

In the presence of an invariant integral, one can integrate morphisms of bigraded vector spaces to obtain morphisms of corepresentations. 
\begin{Lem} \label{lem:rep-average}  Let $(V,\mathscr{X})$ and
  $(W,\mathscr{Y})$ be corepresentations of  a partial
  Hopf algebra $\mathscr{A}$ with invariant integral $\phi$, and let
  $\Gru{T}{k}{l} \in \Hom_{\C}(\Gru{V}{k}{l},\Gru{W}{k}{l})$ for all $k,l\in I$. Then for each $m,n$ fixed, the families
  \begin{align*}
    \Gr{\check T}{m}{n}{k}{l} &:= (\phi \otimes
    \id)(\Gr{(Y^{-1})}{n}{m}{l}{k}(1\otimes
    \Gru{T}{m}{n})\Gr{X}{m}{n}{k}{l}), \\
    \Gr{\hat T}{m}{n}{k}{l} &:=(\phi \otimes
    \id)(\Gr{Y}{k}{l}{m}{n}(1\otimes
    \Gru{T}{m}{n})\Gr{(X^{-1})}{l}{k}{n}{m})
  \end{align*} 
form  morphisms $\Grd{\check{T}}{m}{n}$ and $\Grd{\hat{T}}{m}{n}$ from $(V,\mathscr{X})$ to $(W,\mathscr{Y})$. 

\end{Lem} 
\begin{proof} We may suppose that $T$ is supported only on the component at index $(m,n)$. We can then drop the upper indices and write $\Gru{\check{T}}{k}{l}$ and $\Gru{\hat{T}}{k}{l}$. Then 
 in total form, $\check{T}=(\phi \otimes \id)(Y^{-1}(1 \otimes T)X)$
  and $\hat{T}=(\phi \otimes \id)(Y(1 \otimes T)X^{-1})$.  We compute
  \begin{align*}
    Y^{-1}(1 \otimes \check{T})X &= (\phi \otimes \id \otimes
    \id)((Y^{-1})_{23}(Y^{-1})_{13}(1 \otimes 1
    \otimes T)X_{13}X_{23})  \\
    &= ((\phi \otimes\id)  \Delta  \otimes \id)(Y^{-1}(1 \otimes T)X) \\
    &= \sum_{l} \rho_{l} \otimes (\phi \otimes \id)((\rho_{l} \otimes
    1)Y^{-1}(1 \otimes T)X)  \\
    &= \sum_{k,l} \rho_{l} \otimes \Gru{\check T}{k}{l}.
  \end{align*}
  Hence $\check{T}$ is a morphism from $\mathscr{X}$ to $\mathscr{Y}$
  by Lemma \ref{lemma:rep-total-morphism}. The assertion for $\hat
  T$ follows similarly.
\end{proof}


\begin{Lem}\label{LemIm}
  Let $\mathscr{A}$ be an $I$-partial Hopf algebra with invariant integral $\phi$.
  Let $(V,\mathscr{X})$ be a corepresentation
  and $\Gru{W}{k}{l} \subseteq \Gru{V}{k}{l}$ an invariant family of
  subspaces. Then there exists an idempotent endomorphism $T$ of
  $(V,\mathscr{X})$ such that $\Gru{W}{k}{l}=\img\Gru{T}{k}{l}$ for
  all $k,l$.
\end{Lem}
\begin{proof}
  By a direct sum decomposition, we may assume that $V$ is in a fixed
  component $\Corep(\mathscr{A})_{\alpha\beta}$. For all $k\in
  I_{\alpha},l\in I_{\beta}$, choose idempotent endomorphisms
  $\Gru{T}{k}{l}$ of $\Gru{V}{k}{l}$ with image $\Gru{W}{k}{l}$.  Let
  $\mathscr{Y}$ be the restriction of $\mathscr{X}$ to $W$.  By Lemma
  \ref{lem:rep-average}, we obtain morphisms $\Grd{\check{T}}{m}{n}$
  of $(V,\mathscr{X})$ to $(W,\mathscr{Y})$, which we can also
  interpret as endomorphisms of $(V,\mathscr{X})$.  Fix $n\in
  I_{\beta}$ and write $\Grd{\check{T}}{}{n} = \sum_m
  \Grd{\check{T}}{m}{n}$ (using column-finiteness of $V$). We claim
  that $W$ is the image of $\Grd{\check{T}}{}{n}$.
  
   Invariance of $W$ implies  $(1 \otimes T)X(1
  \otimes T)=X(1\otimes T)$. Applying
 $(S \otimes \id)$, we get   $(1 \otimes T)X^{-1}(1
  \otimes T)=X^{-1}(1\otimes T)$. Then
  \begin{align*}
    \Grd{\check{T}}{}{n} T &= (\phi \otimes \id)(X^{-1}(1 \otimes
    \rho_{n}^{V}T)X(1 \otimes T))  \\ &= 
     (\phi \otimes \id)(X^{-1}(1 \otimes
    \rho_{n}^{V})X(1 \otimes T)) \\ 
    & = (\phi\otimes \id)(X^{-1}X(\lambda_n\otimes T))
    \\&
    =
  \sum_l \phi(\UnitC{n}{l}) \rho^{V}_{l}T  = \sum_{l\in I_{\beta}} \rho^V_lT = T,
  \end{align*}
 since $V\in \Corep(\mathscr{A})_{\alpha\beta}$. As $\Gr{\check{T}}{}{n}{k}{l}$ sends $\Gru{V}{k}{l}$ into $\Gru{W}{k}{l}$, it follows that $\img{\check{T}^{n}}=W$, which proves the claim.
\end{proof}

\begin{Cor}  \label{cor:rep-cosemisimple}
  Let $\mathscr{A}$ be a partial Hopf algebra with invariant integral.  Then
  every corepresentation of $\mathscr{A}$ decomposes into a (possibly infinite) direct
  sum of irreducible corepresentations.
\end{Cor} 

We can now prove that the category $\Corep(\mathscr{A})$ of a partial Hopf algebra with invariant integral is semisimple, that is, any object is a \emph{finite} direct sum of irreducible objects. We first state a lemma which will also be convenient at other occasions.

\begin{Lem}\label{LemInjMor}  Let $\mathscr{A}$ be a partial Hopf algebra, and fix $\alpha,\beta$ in the hyperobject set.  If $T$ is a morphism in $\Corep(\mathscr{A})_{\alpha\beta}$ and $\sum_{k\in I_\alpha} \Gru{T}{k}{l}=0$ for some $l \in I_\beta$, then $T=0$.
\end{Lem} 

\begin{proof} This follows from the equations in Lemma \ref{lemma:rep-total-morphism}.
\end{proof}

\begin{Prop}\label{prop:rep-cosemisimple} Let $\mathscr{A}$ be a partial Hopf algebra with invariant integral.   Then the components of the partial tensor category $\Corep(\mathscr{A})$ are semisimple.
\end{Prop}
\begin{proof} 

Let $V$ be in any object of $\Corep(\mathscr{A})_{\alpha\beta}$ for $\alpha,\beta\in \mathscr{I}$.  From Lemma \ref{LemInjMor}, we see that for $T$ a morphism in $\Corep(\mathscr{A})_{\alpha\beta}$, the map $T\mapsto \sum_{k\in I_\alpha} \Gru{T}{k}{l}$ is injective for any choice of $l\in I_\beta$. By column-finiteness of $V$, the algebra of self-intertwiners of $V$ is finite-dimensional. From Corollary \ref{cor:rep-cosemisimple}, we deduce that $V$ is a finite direct sum of irreducible invariant subspaces.
\end{proof} 


Our next goal is to obtain Schur orthogonality for matrix coefficients of corepresentations. We want to give a little more detail than provided in Hayashi's paper \cite{Hay1}.

Given finite-dimensional vector spaces $V$ and $W$, the dual space of
$\Hom_{\C}(V,W)$ is linearly spanned by functionals of the form $\omega_{f,v}(T) =  (f|Tv)$, where $v\in V$, $f\in \dual{W}$, and $(-|-)$ denotes the natural
pairing of $\dual{W}$ with $W$.
\begin{Def} Let $\mathscr{A}$ be a partial bialgebra. The space of
  \emph{matrix coefficients} $\mathcal{C}(\mathscr{X})$ of a
  corepresentation $(V,\mathscr{X})$ is the sum of the subspaces
\begin{align*}
  \Gr{\mathcal{C}(\mathscr{X})}{k}{l}{m}{n} &= \Span \left\{ (\id \otimes
    \omega_{f,v})(\Gr{X}{k}{l}{m}{n}) \mid v\in \Gru{V}{m}{n}, f \in
    \dual{(\Gru{V}{k}{l})} \right\} \subseteq \Gr{A}{k}{l}{m}{n}.
\end{align*}
\end{Def}
As $\Delta_{pq}(\Gr{\mathcal{C}(\mathscr{X})}{k}{l}{m}{n}) \subseteq
  \Gr{\mathcal{C}(\mathscr{X})}{k}{l}{p}{q} \otimes
  \Gr{\mathcal{C}(\mathscr{X})}{p}{q}{m}{n}$, the $\Gr{\mathcal{C}(\mathscr{X})}{k}{l}{m}{n}$ form a partial
coalgebra with respect to $\Delta$ and $\epsilon$.  Moreover, for each
$k,l$ the $I\times I$-graded vector  space $\Grd{\mathcal{C}(\mathscr{X})}{k}{l}:=\bigoplus_{m,n }
  \Gr{\mathcal{C}(\mathscr{X})}{k}{l}{m}{n}$ is rcfd, and the inclusion above shows that it supports a regular corepresentation.

\begin{Lem} \label{lemma:rep-regular-embedding}
  Let $(V,\mathscr{X})$ be a corepresentation
  of a partial bialgebra and let $f\in
  \dual{(\Gru{V}{k}{l})}$. Then the family of maps
  \begin{align*}
    \Gr{T}{}{\,(f)}{m}{n} \colon \Gru{V}{m}{n} \to
    \Gr{\mathcal{C}(\mathscr{X})}{k}{l}{m}{n}, \ w \mapsto (\id
    \otimes \omega_{f,w})(\Gr{X}{k}{l}{m}{n})=(\id \otimes
    f)(\Gr{X}{k}{l}{m}{n}(1 \otimes w)),
  \end{align*}
  is a morphism from $\mathscr{X}$ to the regular corepresentation on
  $\Grd{\mathcal{C}(\mathscr{X})}{k}{l}$. 
\end{Lem}
\begin{proof}
  Denote by $\mathscr{Y}$ the regular corepresentation on
  $\bigoplus_{m,n } \Gr{\mathcal{C}(\mathscr{X})}{k}{l}{m}{n}$. Then
for all $v \in \Gru{V}{m}{n}$, 
  \begin{align*}
 \Gr{Y}{p}{q}{m}{n}    (1\otimes \Gr{T}{}{\,(f)}{m}{n}(v)) &= 
(\Delta^{\op}_{pq} \otimes \omega_{f,v})( \Gr{X}{k}{l}{m}{n})  = (\id \otimes \id \otimes
f)((\Gr{X}{k}{l}{p}{q})_{23}(\Gr{X}{p}{q}{m}{n})_{13}(1 \otimes 1
 \otimes v)) \\&=(1 \otimes \Gr{T}{}{\,(f)}{p}{q})\Gr{X}{p}{q}{m}{n}(1 \otimes v).
  \end{align*}
\end{proof}
As before, we denote by $\dual{V}$ the dual of a vector space $V$.
\begin{Lem} \label{lemma:regular-corep} Let $\mathscr{A}$ be a partial
  Hopf algebra. Then for any $a \in \sum_{k,l} \Gr{A}{k}{l}{m}{n}$, the family of
  subspaces $\Gru{V}{p}{q} = \{ (\id\otimes f)(\Delta_{pq}(a)) : f \in
    \dual{(\Gr{A}{p}{q}{m}{n})}\}$
  supports a regular corepresentation with $a \in \Gru{V}{m}{n}$. If further $(W,\mathscr{Y})$ is an irreducible  regular corepresentation, then $(W,\mathscr{Y})$ is recovered by the above procedure from \emph{any} non-zero $a \in \Gru{W}{m}{n}$.
\end{Lem}
\begin{proof} Assume that $a$ and $V$ are as above. Taking $f=\epsilon$, one finds $a \in \Gru{V}{m}{n}$. Next, write
   $\Delta_{pq}(a)=\sum_{i} b_{pq}^{i} \otimes c^{i}_{pq}$ with linearly independent $(c_{pq}^{i})_{i}$. Then $ \Gru{V}{p}{q} =
  \mathrm{span}\{b_{pq}^{i} : i \}$, and  $\Delta_{rs}(\Gru{V}{p}{q}) \subseteq
  \Gru{V}{r}{s} \otimes \Gr{A}{r}{s}{p}{q}$ as $\sum_{i}
    \Delta_{rs}(b^{i}_{pq}) \otimes c^{i}_{pq}  = \sum_{j} b^{j}_{rs} \otimes
    \Delta_{pq}(c^{j}_{rs})$ by coassociativity. 
    
      If now  $W$ is an irreducible regular corepresentation and $a\in \Gru{W}{m}{n}$ non-zero, then 
    the associated $\Gru{V}{p}{q}$ are included in $\Gru{W}{p}{q}$. As $a\in \Gru{V}{m}{n}$, it follows by irreducibility and Lemma \ref{LemIm} that $\Gru{V}{p}{q} = \Gru{W}{p}{q}$ for all $p,q$.
\end{proof}
\begin{Prop} \label{prop:rep-weak-pw} Let $\mathscr{A}$ be a partial
  Hopf algebra with invariant integral. Then the total algebra $A$ is the sum
  of the matrix coefficients of irreducible corepresentations.
\end{Prop}
\begin{proof} 
  Let $a \in \Gr{A}{k}{l}{m}{n}$, define $\Gru{V}{p}{q}$ as in
  Lemma \ref{lemma:regular-corep} and let $\mathscr{X}$ be the regular
  corepresentation on $V$. Then
  $a = (\id \otimes \epsilon)(\Delta^{\op}_{kl}(a)) =
    (\id \otimes \epsilon)(\Gr{X}{k}{l}{m}{n}(1 \otimes a)) \in
    \Gr{\mathcal{C}(\mathscr{X})}{k}{l}{m}{n}$.
  Decomposing $(V,\mathscr{X})$, we find that
  $a$ is contained in the sum of matrix coefficients of irreducible  corepresentations.
\end{proof}

\begin{Prop} \label{prop:rep-orthogonality-1} Let $\mathscr{A}$ be a
  partial Hopf algebra with invariant integral $\phi$, and let
  $(V,\mathscr{X})$ and
  $(W,\mathscr{Y})$ be inequivalent irreducible corepresentations.  Then  for all
  $a\in \mathcal{C}(X), b \in \mathcal{C}(Y)$,
  \[\phi(S(b)a) = \phi(bS(a))=0.\]
\end{Prop}
\begin{proof}
Since $\phi$ vanishes on $S(\Gr{A}{k}{l}{m}{n})\Gr{A}{p}{q}{r}{s}$ and
on $\Gr{A}{p}{q}{r}{s}S(\Gr{A}{k}{l}{m}{n})$ unless
$(p,q,r,s) = (m,n,k,l)$, it suffices to prove the assertion for  elements of the form $
  a=(\id \otimes \omega_{f,v})(\Gr{X}{k}{l}{m}{n})$ and  $b =(\id \otimes \omega_{g,w})(\Gr{Y}{m}{n}{k}{l})$
where $f\in \dual{(\Gru{V}{k}{l})}, v \in \Gru{V}{m}{n}$ and $g \in
\dual{(\Gru{W}{m}{n})}, w \in \Gru{W}{k}{l}$.  Applying Lemma
\ref{lem:rep-average} to the family of maps $\Gru{T}{p}{q} \colon \Gru{V}{p}{q} \to \Gru{W}{p}{q}$ with $\Gru{T}{p}{q}(u) =  \delta_{p,k}\delta_{q,l}  f(u)w,$
  yields morphisms $\Grd{\check{T}}{k}{l},\Grd{\hat{T}}{k}{l}$ from $(V,\mathscr{X})$ to
  $(W,\mathscr{Y})$ which are necessarily $0$. Inserting the
  definition of $\Grd{\check{T}}{k}{l}$, we find
  \begin{align*}
    \phi(S(b)a) &= \phi\big((S \otimes
    \omega_{g,w})(\Gr{Y}{m}{n}{k}{l}) \cdot (\id \otimes
    \omega_{f,v})(\Gr{X}{k}{l}{m}{n})\big) \\ &= (\phi \otimes \omega_{g,v})\left(\Gr{(Y^{-1})}{l}{k}{n}{m}(1 \otimes
      \Gru{T}{k}{l} )     \Gr{X}{k}{l}{m}{n}\right) 
    = \omega_{g,v}( \Gr{\check{T}}{k}{l}{m}{n}) = 0.
  \end{align*}
  
  A similar calculation involving $\hat{T}$ shows that
  $\phi(bS(a))=0$.  
\end{proof}

%From now on, we will assume that our partial Hopf algebra is regular (i.e.~ has bijective antipode).

\begin{Cor}\label{CorIntAnt} Let $\mathscr{A}$ be a partial Hopf algebra with invariant integral $\phi$. Then $\phi = \phi \circ S$.
\end{Cor} 

\begin{proof} Assume that $a \in A$ lies in an irreducible regular corepresentation which is not a direct summand of the trivial corepresentation. Then $\phi(a) = \phi(S(a)) = 0$ by Proposition \ref{prop:rep-orthogonality-1}. As such $a$ together with the $\UnitC{k}{m}$ linearly span $A$, this proves the corollary.
\end{proof}

\begin{Theorem} \label{thm:rep-orthogonality} Let $\mathscr{A}$ be a partial Hopf algebra with invariant integral $\phi$. Let $\alpha,\beta\in \mathscr{I}$, and let $(V,\mathscr{X})$
  be an irreducible corepresentation of $\mathscr{A}$ inside $\Corep(\mathscr{A})_{\alpha\beta}$. Then there exists an isomorphism 
 $G$ from $(V,\hat{\hat{\mathscr{X}}})$ to $(V,\mathscr{X})$.
  Moreover, with $F$ denoting the inverse of $G$, the following hold.
  \begin{enumerate}[label=(\arabic*)]
  \item The numbers $d_G:=\sum_{k} \Tr (\Gru{G}{k}{l})$ and $d_F:=\sum_{n} \Tr (\Gru{F}{m}{n})$ are non-zero and do not depend on the choice of $l \in I_\beta$ or $m\in I_\alpha$.
    \item  For all $k,m \in I_\alpha$ and $l,n\in I_\beta$,
    \begin{align*}
      (\phi \otimes \id)(\Gr{(X^{-1})}{l}{k}{n}{m}\Gr{X}{k}{l}{m}{n})
      &=d_G^{-1}\Tr(\Gru{G}{k}{l})
      \id_{\Gru{V}{m}{n}}, \\
      (\phi \otimes \id)(\Gr{X}{k}{l}{m}{n}\Gr{(X^{-1})}{l}{k}{n}{m})
      &=d_F^{-1}\Tr(\Gru{F}{m}{n})
      \id_{\Gru{V}{k}{l}}.
    \end{align*}
  \item Denote by $\Sigma_{klmn}$ the flip map $\Gru{V}{k}{l}
    \otimes \Gru{V}{m}{n} \to \Gru{V}{m}{n}
    \otimes \Gru{V}{k}{l}$. Then
 \begin{align*}
   (\phi \otimes \id \otimes
   \id)((\Gr{(X^{-1})}{l}{k}{n}{m})_{12}(\Gr{X}{k}{l}{m}{n})_{13}) &=
   d_G^{-1}
   (\id_{\Gru{V}{m}{n}} \otimes \Gru{G}{k}{l})
   \circ \Sigma_{klmn}, \\
   (\phi \otimes \id \otimes
   \id)((\Gr{X}{k}{l}{m}{n})_{13}(\Gr{(X^{-1})}{l}{k}{n}{m})_{12}) &= d_F^{-1} (\Gru{F}{m}{n}
   \otimes \id_{\Gru{V}{k}{l}}) \circ \Sigma_{klmn}.
 \end{align*}
\end{enumerate}
  \end{Theorem}
\begin{proof}
  We prove the assertions and equations involving $d_G$ in (1), (2)
  and (3)  simultaneously; the assertions involving $d_F$ then  follow similarly.

  Consider
  the following endomorphism $F_{mnkl}$ of $\Gru{V}{m}{n}\otimes \Gru{V}{k}{l}$, 
  \begin{align*}
    F_{mnkl}
    &:=(\phi \otimes \id \otimes \id)\left((\Gr{(X^{-1})}{l}{k}{n}{m})_{12}(\Gr{X}{k}{l}{m}{n})_{13}\right)
    \circ \Sigma_{mnkl} \\ &= (\phi \otimes \id \otimes
    \id)\left((\Gr{(X^{-1})}{m}{n}{k}{l})_{12}
      \Sigma_{klkl,23}(\Gr{X}{k}{l}{m}{n})_{12}\right).
  \end{align*}
  By applying Lemma \ref{lem:rep-average} with respect to the flip map $\Sigma_{klkl}$, we see that the family $(F_{mnkl})_{m,n}$ is
  an endomorphism of $(V \otimes \Gru{V}{k}{l}, X\otimes \id)$ and hence
  $F_{mnkl} = \id_{\Gru{V}{m}{n}} \otimes \Gru{R}{k}{l}$ with some $\Gru{R}{k}{l} \in \End_{\C}(\Gru{V}{k}{l})$ not
  depending on $m,n$. 
  
  On the other hand, since $\phi = \phi S$,
  \begin{align*}
    F_{mnkl} &= (\phi \otimes \id \otimes \id)((S \otimes
    \id)(\Gr{X}{m}{n}{k}{l})_{12}(\Gr{X}{k}{l}{m}{n})_{13})
    \circ \Sigma_{mnkl} \\
    &= (\phi \otimes \id \otimes \id)\left(((S \otimes
      \id)(\Gr{X}{k}{l}{m}{n}))_{13}
      ((S^{2} \otimes \id)(\Gr{X}{m}{n}{k}{l}))_{12}\right)     \circ \Sigma_{mnkl}\\
    &= (\phi \otimes \id \otimes
    \id)\left((\Gr{(X^{-1})}{k}{l}{m}{n})_{13} (\Sigma_{mnmn})_{23}
      (\Gr{(X^{\!\wedge \!\wedge})}{m}{n}{k}{l})_{13}\right).
  \end{align*}
  Hence we can again apply Lemma \ref{lem:rep-average} and
  find that the family $(F_{mnkl})_{k,l}$ is a morphism from $(\Gru{V}{m}{n} \otimes V, \hat{\hat{X}}_{13})$ to $(\Gru{V}{m}{n} \otimes V,
 X_{13})$. Now as $\hat{\hat{\mathscr{X}}}$ is also irreducible, the space $\Mor(\mathscr{X},\hat{\hat{\mathscr{X}}})$ is linearly spanned by a single element $G$. Therefore $F_{mnkl} = \Gru{T}{m}{n} \otimes \GrDA{G}{k}{l}$
  with some $\Gru{T}{m}{n} \in \mathcal{\End_{\C}}(\Gru{V}{m}{n})$
  not depending on $k,l$. 
  
  We conclude from the above calculations, invoking once again the irreducibility of $\mathscr{X}$, that $F_{mnkl} = \lambda
  (\id_{\Gru{V}{m}{n}} \otimes \GrDA{G}{k}{l})$  for some $\lambda\in \C$.
   
  Choose now dual  bases
  $(v_{i})_{i}$ for $\Gru{V}{k}{l}$ and $(f_{i})_{i}$ for  $\dual{(\Gru{V}{k}{l})}$. Then
  \begin{align*}
    \lambda   \Tr( \GrDA{G}{k}{l}) \id_{\Gru{V}{m}{l}}
 &= \sum_{i} (\id \otimes
    \omega_{f_{i},v_{i}})(F_{mlkl}) = (\phi \otimes
    \id)((\Gr{(X^{-1})}{l}{k}{l}{m}) \Gr{X}{k}{l}{m}{l}).
  \end{align*}
   By Lemma \ref{LemInjMor}, we can choose $m\in I_{\alpha}$ with $\Gru{V}{m}{l}\neq 0$.   Then summing the previous relation over $k$, the relations $\sum_{k}
  (\Gr{(X^{-1})}{l}{k}{l}{m}) \Gr{X}{k}{l}{m}{l} = \UnitC{l}{l}
  \otimes \id_{\Gru{V}{m}{l}}$ and
  $\phi(\UnitC{l}{l})=1$ give $\lambda \cdot  \sum_{k} \Tr(\GrDA{G}{k}{l}) = 1.$ It follows at once that $G$ is not the zero morphism, and hence $\mathscr{X}$ is isomorphic to $\hat{\hat{\mathscr{X}}}$.  Now all assertions in (1)--(3) concerning $d_G$ follow.
\end{proof}

%Note that for semi-simple tensor categories with (left and right) duals, any object is
%isomorphic to its left bidual  \cite[Proposition
 % 2.1]{ENO1}. The same proof works for tensor categories with local units. Hence if $\mathscr{A}$ is regular, there always exists an $F$ as in the previous theorem. In fact, one can prove as in \cite{Lar1} that any partial Hopf algebra with invariant integral is regular. 

\begin{Cor}\label{CorOrth}
  Let $\mathscr{A}$ be a %regular 
partial Hopf algebra with invariant integral $\phi$, let
  $(V,\mathscr{X})$ be an irreducible corepresentation of
  $\mathscr{A}$, let $F$ be an isomorphism from
  $(V,\mathscr{X})$ to $(V,\dualco{\dualco{\mathscr{X}}})$ and
  $G=F^{-1}$, and let $a=(\id \otimes
  \omega_{f,v})(\Gr{X}{k}{l}{m}{n})$ and $b=(\id \otimes
  \omega_{g,w})(\Gr{X}{m}{n}{k}{l})$ for
  $f \in   \dual{(\Gru{V}{k}{l})}$, $v \in\Gru{V}{m}{n}$, $g \in
  \dual{(\Gru{V}{m}{n})}$, $w \in  \Gru{V}{k}{l}$.  Then
\begin{align*}
  \phi(S(b)a) &= \frac{(g|v)(f|Gw)}{\sum_{r}
    \Tr(\GrDA{G}{r}{n})}, & \phi(aS(b)) = \frac{(g|Fv)(f|w)}{\sum_{s}
    \Tr(\GrDA{F}{m}{s})}.
\end{align*}
\end{Cor}
\begin{proof}
Apply $\omega_{g,w} \otimes
    \omega_{f,v}$ to the formulas in  Theorem
    \ref{thm:rep-orthogonality}.(c).
\end{proof}

\begin{Cor} \label{cor:rep-pw}
  Let $\mathscr{A}$ be a partial Hopf algebra with invariant integral, and let
  $((V^{(a)},\mathscr{X_{a}}_{a}))_{a \in \mathcal{I}}$ be a maximal family of mutually non-isomorphic irreducible corepresentations of
  $\mathscr{A}$. Denote by $\Gr{\mathscr{Y}}{k}{l}{}{a}$
  the regular corepresentation on
  $\Grd{\mathcal{C}(\mathscr{X}_a)}{k}{l}$. Then there exists a
  linear isomorphism
  \begin{align*}
    \dual{( \Gr{V}{}{(a)}{k}{l})} \to
    \Mor(\mathscr{X}_{a},
    \Gr{\mathscr{Y}}{k}{l}{}{a})
  \end{align*}
  assigning to each $f\in     \dual{( \Gr{V}{}{(a)}{k}{l})}$ the morphism
  $T^{(f)}$ of Lemma \ref{lemma:rep-regular-embedding}, and with inverse the map $T \mapsto \epsilon \circ \GrDA{T}{k}{l}$. Furthermore, the map  
  \begin{align*}
    \bigoplus_{a \in \mathcal{I}} \bigoplus_{k,l,m,n}
    (\dual{(\Gr{V}{}{(a)}{k}{l})} \otimes
    \Gr{V}{}{(a)}{m}{n}) \to A,\quad f \otimes w \mapsto T^{(f)}(w)
  \end{align*}
  is a linear isomorphism. 
\end{Cor}
\begin{proof} The injectivity of the first map follows from Corollary \ref{CorOrth}. Its surjectivity follows immediately by checking that the map $T \mapsto \epsilon \circ \GrDA{T}{k}{l}$ is an inverse. 

The injectivity of the second map follows again from Corollary \ref{CorOrth}, and its surjectivity from Proposition \ref{prop:rep-weak-pw}.
\end{proof}

In particular, it follows from the previous two corollaries that an invariant integral $\phi$ is faithful, i.e.~ $\phi(ab)=0$ or $\phi(ba)=0$ for all $b$ implies $a=0$. This can also be proven more directly along the lines of \cite[Proposition 3.4]{VDae2}.

The following Corollary generalizes \cite[Theorem 3.3]{Lar1} and \cite[Corollary 3.6]{Hay1}.

\begin{Cor} Let $\mathscr{A}$ be a partial Hopf algebra with invariant integral $\phi$. Then $\mathscr{A}$ is regular.
\end{Cor}

\begin{proof} Injectivity of $S$ follows from Corollary \ref{CorOrth}. As further $\mathscr{X} \cong \hat{\hat{\mathscr{X}}}$ for any irreducible corepresentation $\mathscr{X}$, it follows that $S^2(\Gr{\mathcal{C}(\mathscr{X})}{k}{l}{m}{n}) \subseteq \Gr{\mathcal{C}(\mathscr{X})}{k}{l}{m}{n}$, hence $S$ is surjective by Proposition \ref{prop:rep-weak-pw} and finite dimensionality of each  $\Gr{\mathcal{C}(\mathscr{X})}{k}{l}{m}{n}$.
\end{proof} 


\subsection{Unitary corepresentations of partial compact quantum groups}


Let us write $B(\Hsp,\mathcal{G})$ for the space of
bounded morphisms between Hilbert spaces $\Hsp$ and $\mathcal{G}$. 

\begin{Def} Let $\mathscr{A}$ define a partial compact quantum
  group. A corepresentation $\mathscr{X}$ of $\mathscr{A}$ on a rcfd collection of Hilbert spaces $\Gru{\Hsp}{k}{l}$ is called
   \emph{unitary}
  if $\Gr{(X^{-1})}{k}{l}{m}{n}=(\Gr{X}{l}{k}{n}{m})^{*}$ as elements in $\Gr{A}{k}{l}{m}{n}\otimes
  B(\Gru{\Hsp}{l}{k},\Gru{\Hsp}{n}{m}).$
\end{Def} 

For example, viewing $\C^{(I)}$ as a direct sum of the trivial Hilbert spaces $\C$, the
  trivial corepresentation $\mathscr{U}$ on $\C^{(I)}$ is unitary. Moreover, the tensor product of corepresentations lifts to a tensor product
of unitary corepresentations.  We hence obtain a tensor C$^*$-category $\Corep_{u,\rcf}(\mathscr{A})$ of unitary corepresentations. We denote again by $\Corep_u(\mathscr{A})$ the subcategory of all corepresentations with finite support on the hyperobject set. It is the total tensor C$^*$-category with local units of a semi-simple partial tensor C$^*$-category. Our aim is to show that it is a partial fusion C$^*$-category. In the following, we use the physicist convention that inner products on Hilbert spaces are anti-linear in their \emph{first} argument. 

\begin{Lem} \label{lemma:rep-regular-unitary}
  Let $\mathscr{A}$ define a partial compact quantum group with
positive invariant  integral $\phi$, and let $\Gru{V}{m}{n} \subseteq
\bigoplus_{k,l} \Gr{A}{k}{l}{m}{n}$ define a regular corepresentation $\mathscr{X}$. Then with
    respect to the inner product given by $\langle
    a|b\rangle:=\phi(a^{*}b)$ each $\Gru{V}{k}{l}$ is a Hilbert space and $\mathscr{X}$ unitary.
\end{Lem}
\begin{proof}   Let  $a\in \Gru{V}{m}{n}$, $b\in \Gru{V}{m}{n'}$ and define $\omega_{b,a} \colon
\Hom_{\C}(\Gru{V}{m}{n},\Gru{V}{m}{n'}) \to \C$ by $T
\mapsto \langle b|Ta\rangle$. Then
\begin{eqnarray*}
\sum_{k }(\id \otimes \omega_{b,a})
((\Gr{X}{k}{l}{m}{n'})^* \Gr{X}{k}{l}{m}{n}))  &=& \sum_k
(\id\otimes \phi)(\Delta_{kl}^{\op}(b)^*\Delta_{kl}^{\op}(a))
  = \sum_k (\phi\otimes
  \id)(\Delta_{lk}(b^*)\Delta_{kl}(a)) \\ &=& (\phi\otimes
  \id)(\Delta_{ll}(b^*a))  = \phi(b^*a)\UnitC{l}{n} =
  \delta_{n',n} \UnitC{l}{n} \otimes \langle b|a\rangle.
\end{eqnarray*} Hence $ \sum_{k}
    (\Gr{X}{k}{l}{m}{n'})^* \Gr{X}{k}{l}{m}{n} =
    \delta_{n,n'}\UnitC{l}{n}\otimes
    \id_{\Gru{V}{m}{n}},$ and $\mathscr{X}$ is unitary by definition of the generalized inverse of a corepresentation.
\end{proof} 

\begin{Prop} \label{prop:rep-unitarisable} Every 
  corepresentation of a partial compact quantum group $\mathscr{A}$ is
  isomorphic to a unitary corepresentation.
\end{Prop}
\begin{proof}
  By Proposition \ref{prop:rep-cosemisimple} and Corollary
  \ref{cor:rep-pw}, every corepresentation is isomorphic to a direct
  sum of irreducible regular corepresentations, which are unitary by
  Lemma \ref{lemma:rep-regular-unitary}.
\end{proof}
\begin{Cor} The partial tensor C$^*$-category $\Corep_u(\mathscr{A})$ is a partial fusion C$^{*}$-category.
\end{Cor}

\begin{Prop} \label{prop:rep-unitary-bidual}
  Let $\mathscr{A}$ define a partial compact quantum group and let
  $(\Hsp,\mathscr{X})$ be an irreducible unitary corepresentation of
  $\mathscr{A}$.  Then there exists an isomorphism $F$
  from $(\Hsp,\mathscr{X})$ to 
  $(\Hsp,(S^{2} \otimes \id)(\mathscr{X}))$ in $\Corep(\mathscr{A})$ such
  that each $\Gru{F}{k}{l}$ is positive.
\end{Prop}
\begin{proof}
 By Proposition \ref{prop:rep-unitarisable}, there exists an
  isomorphism $T \colon \dualco{\mathscr{X}} \to \mathscr{Y}$ for some
  unitary corepresentation $\mathscr{Y}$ on $\dual{\Hsp}$, so that in total form,
  $(1\otimes T)\dualco{X} = Y(1 \otimes T)$.
We  apply   $S \otimes -^{\tr}$ and $-^{*} \otimes -^{*\tr}$,
respectively to find $ \dualco{\dualco{X}}(1 \otimes \dualop{T}) = (1 \otimes
  \dualop{T})\dualco{Y}$ and $(1 \otimes T^{*\tr})X=\dualco{Y}(1\otimes T^{*\tr}).$

Combining both equations, we
find $\dualco{\dualco{X}}(1 \otimes \dualop{T}T^{*\tr})=(1 \otimes
\dualop{T}T^{*\tr})X$. Thus, we can take
$F:=\dualop{T}T^{*\tr}$.
\end{proof}

The Schur orthogonality relations in Corollary \ref{CorOrth} can be
rewritten using the involution. If $v \in \Gru{\Hsp}{k}{l}$, $v' \in \Gru{\Hsp}{m}{n}$
and $\omega_{v,v'}(T) = \langle v|Tv'\rangle$, then 
\begin{align*}
  S((\id \otimes \omega_{v,v'})(\Gr{X}{k}{l}{m}{n})) &=
  (\id \otimes \omega_{v,v'}) (\Gr{(X^{-1})}{n}{m}{l}{k})) \\ & =
  (\id \otimes \omega_{v,v'})( (\Gr{X}{m}{n}{k}{l})^{*}) =
  (\id \otimes \omega_{v',v})(\Gr{X}{m}{n}{k}{l})^{*}.
\end{align*}
This equation and Corollary \ref{CorOrth} imply the following corollary.
\begin{Cor}\label{cor:rep-unitary-schur-orthogonality}
  Let $\mathscr{A}$ define a partial compact quantum group with
  positive invariant integral $\phi$, let $(\Hsp,\mathscr{X})$ be an irreducible
  unitary corepresentation of $\mathscr{A}$, let $F$ be a positive
  isomorphism from $(\Hsp,\mathscr{X})$ to
  $(\Hsp,\dualco{\dualco{\mathscr{X}}})$ and
  $G=F^{-1}$, and let $a=(\id \otimes
  \omega_{v,v'})(\Gr{X}{k}{l}{m}{n})$ and $b=(\id \otimes
  \omega_{w,w'})(\Gr{X}{k}{l}{m}{n})$, where $v,w \in
  \Gru{\Hsp}{k}{l}$ and $v',w' \in \Gru{\Hsp}{m}{n}$.  Then
\begin{align*}
  \phi(b^{*}a) &= \frac{\langle w|v'\rangle\langle v|Gw'\rangle}{\sum_{r}
    \Tr(\Gru{G}{r}{n})}, & \phi(ab^{*}) = \frac{\langle
    w|Fv'\rangle \langle v|w'\rangle}{\sum_{s}
    \Tr(\Gru{F}{m}{s})}.
\end{align*}
\end{Cor}

\begin{Rem}\label{RemPos} In fact, Proposition \ref{prop:rep-unitary-bidual} and Corollary \ref{cor:rep-unitary-schur-orthogonality} show the following. Let $\mathscr{A}$ be a partial Hopf $^*$-algebra admitting an invariant integral $\phi$, which a priori we do not assume to be positive. Suppose however that each irreducible corepresentation of $\mathscr{A}$ is equivalent to a unitary corepresentation. Then $\phi$ is necessarily positive.
\end{Rem} 

\subsection{Analogues of Woronowicz's  characters}

Let $\mathscr{A}$ be a partial bialgebra, and $a\in \Gr{A}{k}{l}{m}{n}$. Then for $\omega \in A^*$, we can define
\begin{align*} \omega \aste{p,q} a = (\id \otimes \omega) (\Delta_{pq}(a)),&&a \aste{r,s}
\omega:=(\omega \otimes \id)(\Delta_{rs}(a)),\end{align*}and this defines a bimodule structure with respect to the natural $I\times I$-partial convolution algebra structure on $\oplus \left(\Gr{A}{k}{l}{m}{n}\right)^*$. When $\omega$ has support on $\sum_{k,l}\Gr{A}{k}{l}{k}{l}$, it is meaningful to define \begin{align*} \omega\ast a := \sum_{p,q} \omega\aste{p,q}a && a\ast \omega = \sum_{r,s} a\aste{r,s}\omega \end{align*}

We recall that an entire function $f$ has \emph{exponential growth
  on the right half-plane} if there exist $C,d>0$ such that $|f(x+iy)|\leq
C\mathrm{e}^{dx}$  for all $x,y\in \R$ with $x>0$. 

\begin{Theorem} \label{thm:rep-characters} Let $\mathscr{A}$ define a
  partial compact quantum group with positive invariant integral $\phi$.  Then there
  exists a unique family of linear functionals $f_{z} \colon A\to \C$
  such that
\begin{enumerate}[label={(\arabic*)}]
  \item each $f_z$ has support on $\sum_{k,l}\Gr{A}{k}{l}{k}{l}$.
  \item for each $a\in A$, the function $z\mapsto f_{z}(a)$ is entire
    and of exponential growth on the right half-plane.
  \item $f_{0} = \epsilon$ and $(f_{z} \otimes f_{z'}) \circ 
    \Delta= f_{z+z'}$ for all $z,z' \in \C$.
  \item $\phi(ab)=\phi(b(f_{1} \ast a \ast f_{1}))$ for all $a,b\in A$.
  \end{enumerate}
  This family furthermore satisfies
  \begin{enumerate}[label={(\arabic*)}]\setcounter{enumi}{4}
  \item $f_z(ab) = f_z(a)f_z(b)$ for $a\in \Gr{A}{k}{l}{m}{n},b\in \Gr{A}{l}{p}{n}{q}$. 
  \item $S^{2}(a)=f_{-1} \ast a \ast f_{1}$ for all $a\in A$.
  \item $f_{z}(\UnitC{l}{n})=\delta_{l,n}$ and $f_{z} \circ S = f_{-z}$ for all $a\in A$.
  \item $\bar{f}_{z}=f_{-\overline{z}}$ if $\mathscr{A}$ is a partial
    Hopf $^*$-algebra and $\phi$ is positive.
\end{enumerate}
\end{Theorem}


Note that conditions (3), (4) and (6) are meaningful by condition (1).

\begin{proof}
  We first prove uniqueness.  Assume that $(f_{z})_{z}$ is a family of
  functionals satisfying (1)--(4).  Since $\phi$ is faithful, the map
  $\sigma\colon a \mapsto f_{1} \ast a \ast f_{1}$ is uniquely
  determined by $\phi$, and one easily sees that it is a homomorphism. Using
  (3), we find that $\epsilon \circ \sigma^n=f_{2n}$, which uniquely determines these functionals. Using (2) and the
  fact that every entire function of exponential growth on the right
  half-plane is uniquely determined by its values at $\N \subseteq \C$, we can conclude that the family $f_{z}$ is uniquely determined. Moreover, since the property (5) holds for $z = 2n$, we also conclude by the same argument as above that it holds for all $z\in \C$.

  Let us now prove existence.  By  Proposition \ref{prop:rep-unitary-bidual}, we can find non-zero positive invertible operators $\GrDA{F}{k}{l}$ defining a morphism $F$  from $(V,\mathscr{X})$ to
    $(V, \dualco{\dualco{\mathscr{X}}})$. We will scale the $F_{kl}$ such that
    \[d_{\mathscr{X}}:= \sum_{r} \Tr(\Gru{(F^{-1})}{r}{l}) = \sum_{s}
      \Tr(\Gru{F}{m}{s})\]
    for all $l$ in the right and all $m$ in the left hyperobject support of $\mathscr{X}$.
  
  Define now for each $z\in \C$ a functional $f_{z} \colon A \to \C$ such
  that for every 
  irreducible  corepresentation
  $(V,\mathscr{X})$ in $\Corep(\mathscr{A})$, \[ f_{z}((\id \otimes \omega_{\xi,\eta})(\Gr{X}{k}{l}{m}{n})) =
      \delta_{k,m}\delta_{l,n} \cdot
      \omega_{\xi,\eta}(\Gr{F}{}{z}{k}{l}), \qquad \forall \xi \in \Gru{V}{k}{l},\eta \in
      \Gru{V}{m}{n}.\] This is equivalent with $(f_{z} \otimes \id)(\Gr{X}{k}{l}{m}{n}) =
      \delta_{k,m}\delta_{l,n} \cdot \Gr{F}{}{z}{k}{l}$. By
    construction, (1) and (2) hold. We show that the $(f_{z})_{z}$ satisfy the
    assertions (3)--(7). 

        Throughout the following arguments, let 
    $(V,\mathscr{X})$ be an  irreducible corepresentation
 with $F$ as above.

    We first prove property (3). This follows from the relations $(f_{0}  \otimes \id)(\Gr{X}{k}{l}{m}{n}) =
      \delta_{k,m}\delta_{l,n} \id_{\Gru{V}{k}{l}} =
      (\epsilon \otimes \id)(\Gr{X}{k}{l}{m}{n})$ and the computation
    \begin{align*}
      (((f_{z}\otimes f_{z'})\circ \Delta) \otimes
      \id)(\Gr{X}{k}{l}{m}{n}) &=  \delta_{k,m}\delta_{l,n}(f_{z} \otimes f_{z'} \otimes
      \id)\big((\Gr{X}{k}{l}{k}{l})_{13}
      (\Gr{X}{k}{l}{k}{l})_{23}\big) \\
      &=  \delta_{k,m}\delta_{l,n}(\Gru{F}{k}{l})^{z}  \cdot (\Gru{F}{k}{l})^{z'} = (f_{z+z'} \otimes \id)(\Gr{X}{k}{l}{m}{n}).
    \end{align*}
    Applying slice maps of the form $\id
    \otimes \omega_{\xi,\xi'}$ and invoking Theorem \ref{thm:rep-orthogonality}, this proves (3).

To prove (4), write $ \Delta^{(2)} = (
    \Delta \otimes \id)\circ  \Delta = (\id \otimes 
    \Delta) \circ \Delta$, and put $\theta_{z,z'}:=(f_{z'} \otimes \id
    \otimes f_{z})\circ  \Delta^{(2)}.$ Then
    \begin{align*}
      (\theta_{z,z'} \otimes \id)(\Gr{X}{k}{l}{m}{n}) &= (f_{z'} \otimes
      \id \otimes f_{z} \otimes
      \id)((\Gr{X}{k}{l}{k}{l})_{14}(\Gr{X}{k}{l}{m}{n})_{24}(\Gr{X}{m}{n}{m}{n})_{34})
      \\
      &= (1 \otimes (\Gru{F}{k}{l})^{z'}) \Gr{X}{k}{l}{m}{n} (1
      \otimes (\Gru{F}{m}{n})^{z}).
    \end{align*}
    We take $z=z'=1$, use Theorem \ref{thm:rep-orthogonality}, where
    now $d_F= d_G=d_{\mathscr{X}}$ by our scaling of $F$, and obtain
    \begin{eqnarray*}
     && \hspace{-2cm} (\phi \otimes \id \otimes
      \id)((\Gr{(X^{-1})}{l}{k}{n}{m})_{12}((\theta_{1,1} \otimes
      \id)(\Gr{X}{k}{l}{m}{n}))_{13})\\ && =d_{\mathscr{X}}^{-1}(\id \otimes
      \Gru{F}{k}{l}) (\id \otimes \Gru{(F^{-1})}{k}{l})
      \Sigma_{k,l,m,n} (\id \otimes
      \Gru{F}{m}{n}) \\
      &&=d_{\mathscr{X}}^{-1}(\Gru{F}{m}{n} \otimes \id) \Sigma_{klmn} \\
      &&= (\phi \otimes \id \otimes
      \id)((\Gr{X}{k}{l}{m}{n})_{13}(\Gr{(X^{-1})}{l}{k}{n}{m})_{12}).
    \end{eqnarray*}
    To conclude the proof of assertion (4), apply again slice maps of the form
    $\omega_{\xi,\xi'} \otimes \omega_{\eta,\eta'}$.

We have then already argued that the property (5) automatically holds. To show the property (6), note that by Proposition \ref{prop:rep-unitary-bidual} and the calculation above,
    \begin{align*}
      (S^{2} \otimes \id)(\Gr{X}{k}{l}{m}{n}) &= (1
      \otimes\Gru{F}{k}{l})
      \Gr{X}{k}{l}{m}{n}(1 \otimes \Gru{F}{m}{n})^{-1} 
      =(\theta_{-1,1}  \otimes \id)(\Gr{X}{k}{l}{m}{n}).
    \end{align*}
     Assertion (6) follows again by applying slice maps.
    
     To check, (7), note that (1), (2) and (4) immediately imply
     $f_{z}(\UnitC{k}{m})=\delta_{k,m}$. As both $z \rightarrow
     f_{-z}$ and $z\rightarrow f_z\circ S$ satisfy the conditions
     (1)--(4) for $\mathscr{A}$ with the opposite product and
     coproduct (using the partial character property (5) and the
     invariance of $\phi$ with respect to $S$), we find $f_{-z} =
     f_{z} \circ S$.

     Finally, we prove
     (8).  Write
     $\bar{f}_z(a) = \overline{f_z(a^*)}$. Using the relations $
     (\Gr{X}{k}{l}{k}{l})^{*}=(S \otimes \id)(\Gr{X}{k}{l}{k}{l})$,
     $f_{z} \circ S=f_{-z}$  and
     positivity of $\Gr{F}{}{}{k}{l}$, we easily compute that $\bar{f}_z(a) = f_{-\overline{z}}(a)$ for all $a\in
\Gr{\mathcal{C}(\mathscr{X})}{k}{l}{k}{l}$. Since $f_{z}$ and
$f_{-\overline{z}}$ vanish on $\Gr{A}{k}{l}{m}{n}$ if $(k,l)\neq
(m,n)$ and the matrix coefficients of unitary 
corepresentations span $A$, we can conclude $\bar{f}_{z}=f_{-\overline{z}}$.
\end{proof}

Note that our formula for the Woronowicz characters is slightly different from the one in \cite{Hay1}, as we are using a different normalisation of the Haar functional.


\section{Tannaka-Kre$\breve{\textrm{\i}}$n-Woronowicz duality for partial compact quantum groups}

In the previous section, we showed how any partial compact quantum group gave rise to a partial fusion C$^*$-category with a faithful morphism into a partial tensor C$^*$-category of finite-dimensional Hilbert spaces. In this section we reverse this construction, and show that the two structures are in duality with each other. 

Fix a (strict) semi-simple $\mathscr{I}$-partial tensor category $\CatCC$ with irreducible units. Assume that we also have another set $I$ with partition $I = \sqcup_{\alpha\in \mathscr{I}} I_{\alpha}$. We write $k' = \alpha$ if $k\in I_{\alpha}$, and we denote by $\varphi: \mathscr{I}\rightarrow \mathscr{P}(I)$ the corresponding map $\alpha \mapsto I_{\alpha}$.  Let $F: \CatCC\rightarrow \{\Vect_{\fin}\}_{I\times I}$ be a $\varphi$-morphism with components $F_{kl}:\CatCC_{k'l'}\rightarrow \Vect_{\fin}$ and product and unit constraints $\iota$ and $\mu$. The unit constraint $\mu$ will often be used implicitly. For $X\in \CatC_{k'\beta}$ and $Y\in \CatC_{\beta m'}$, we write the projection maps associated to the identification $F_{km}(X\otimes Y)\cong \oplus_{l\in I_\beta} \left(F_{kl}(X)\otimes F_{lm}(Y)\right)$ as \[\pi^{(klm)}_{X,Y}=(\iota^{(klm)}_{X,Y})^{*}:F_{km}(X\otimes Y) \rightarrow F_{kl}(X)\otimes F_{lm}(Y).\]

We choose a set $\mathcal{I}$ parametrizing a maximal family of mutually inequivalent irreducible objects $\{u_a\}_{a\in \mathcal{I}}$ in $\CatC$. We assume that the $u_a$ include the unit objects $\Unitb_{\alpha}$ for $\alpha\in \mathscr{I}$, and we may hence identify $\mathscr{I}\subseteq \mathcal{I}$. For $a\in \mathcal{I}$, we will write $u_a \in \CatC_{\lambda_a,\rho_a}$ with $\lambda_a,\rho_a\in \mathscr{I}$. For $\alpha,\beta\in \mathscr{I}$ fixed, we write $\mathcal{I}_{\alpha\beta}$ for the set of all $a\in \mathcal{I}$ with $\lambda_a=\alpha$ and $\rho_a=\beta$. When $a,b,c\in \mathcal{I}$ with $a\in \mathcal{I}_{\alpha\beta},b\in \mathcal{I}_{\beta\gamma}$ and $c\in \mathcal{I}_{\alpha\gamma}$, we write $c\leq a\cdot b$ if $\Mor(u_c,u_a\otimes u_b)\neq \{0\}$. Note that with $a,b$ fixed, there is only a finite set of $c$ with $c\leq a\cdot b$. We also use this notation for multiple products.

\begin{Def} For $a\in \mathcal{I}$ and $k,l,m,n\in I$, define vector
  spaces \[\Gr{A}{k}{l}{m}{n}(a) =
  \delta_{k',m',\lambda_a}\delta_{l',n',\rho_a}
  \Hom_{\C}(F_{mn}(u_a),F_{kl}(u_a))^*,\] and write
  $\Gr{A}{k}{l}{m}{n} =\underset{a\in \mathcal{I}}{\oplus}\,
  \Gr{A}{k}{l}{m}{n}(a)$, $ A(a) = \underset{k,l,m,n}{\oplus}
  \Gr{A}{k}{l}{m}{n}(a)$, $A = \underset{k,l,m,n}{\oplus} \Gr{A}{k}{l}{m}{n}.$
\end{Def} 


\begin{Def} For $r,s\in I$, we define $\Delta_{rs}: \Gr{A}{k}{l}{m}{n}\rightarrow \Gr{A}{k}{l}{r}{s}\otimes \Gr{A}{r}{s}{m}{n}$ as the direct sum over $a$ of the duals of the composition \[\Hom_{\C}(F_{rs}(u_a),F_{kl}(u_a)) \otimes \Hom_{\C}(F_{mn}(u_a),F_{rs}(u_a))\rightarrow \Hom_{\C}(F_{mn}(u_a),F_{kl}(u_a)),\quad x\otimes y \mapsto x\circ y.\]
\end{Def} 

\begin{Lem} The couple $(\mathscr{A},\Delta)$ is an $I\times
  I$-partial coalgebra with counit map $\epsilon:\Gr{A}{k}{l}{k}{l}(a)\rightarrow \C$ sending $f$ to $f(\id_{F_{kl}(u_a)})$. Moreover, for each fixed $f\in \Gr{A}{k}{l}{m}{n}(a)$, the matrix $\left(\Delta_{rs}(f)\right)_{rs}$ is rcf.
\end{Lem} 
\begin{proof} Coassociativity and counitality are immediate by duality, as the $\Hom_{\C}(F_{mn}(u_a),F_{kl}(u_a))$ form a partial algebra with units $\id_{F_{kl}(u_a)}$ for each fixed $a$. The rcf condition follows immediately from the fact that the total $F(u_a)$ is rcfd.
\end{proof}

In the next step, we define a partial algebra structure on $\mathscr{A} = \{\Gr{A}{k}{l}{m}{n}\mid k,l,m,n\}$. First note that we can identify \[\Nat(F_{mn},F_{kl}) \cong \underset{\rho_a=l'=n'}{\underset{\lambda_a=k'=m'}{\prod_a}} \Hom_{\C}(F_{mn}(u_a),F_{kl}(u_a)),\] where $\Nat(F_{mn},F_{kl})$ denotes the space of natural transformations from $F_{mn}$ to $F_{kl}$ when $k'=m'$ and $l'=n'$. Similarly, we can identify \[\Nat(F_{mn}\otimes F_{pq},F_{kl}\otimes F_{rs}) \cong  \prod_{b,c} \Hom_{\C}(F_{mn}(u_b)\otimes F_{pq}(u_c) ,F_{kl}(u_b)\otimes F_{rs}(u_c)),\] with the product over the appropriate index set and where $F_{kl}\otimes F_{rs}:\CatC_{k'l'}\times \CatC_{r's'}\rightarrow \Vect_{\fin}$ sends $(X,Y)$ to $F_{kl}(X)\otimes F_{rs}(Y).$ As such, there is a natural pairing of these spaces with resp.~ $\Gr{A}{k}{l}{m}{n}$ and $\Gr{A}{k}{l}{m}{n}\otimes \Gr{A}{r}{s}{p}{q}$, and for example $\Gr{A}{k}{l}{m}{n}$ can be identified with the subspace of functionals on $\Nat(F_{mn},F_{kl})$ of finite support with respect to $\mathcal{I}$. 

\begin{Def} For $k'=r', l'=s'$ and $m'=t'$, we define a product
  map \[M:\Gr{A}{k}{l}{r}{s} \otimes \Gr{A}{l}{m}{s}{t}\rightarrow
  \Gr{A}{k}{m}{r}{t},\quad (f\cdot g)(x) = (f\otimes g)(
  \hat{\Delta}^{l}_{s}(x)) \qquad  \textrm{for }x \in
  \Nat(F_{rt},F_{km}),\] where $\hat{\Delta}^l_s(x)$ is the natural
  transformation \[\hat{\Delta}^l_s(x):  F_{rs}\otimes
  F_{st}\rightarrow F_{kl}\otimes F_{lm},\quad
  \hat{\Delta}^l_s(x)_{X,Y} = \pi^{(klm)}_{X,Y} \circ x_{X\otimes Y} \circ \iota^{(rst)}_{X,Y} \qquad \textrm{for }X\in \CatC_{k'l'},Y\in \CatC_{l'm'}.\]
\end{Def}

Note that indeed $f\cdot g$ has finite support as a functional on $\Nat(F_{rt},F_{km})$: if $f$ is supported at $b\in \mathcal{I}_{r's'}$ and $g$ at $c\in \mathcal{I}_{s't'}$, then $f\cdot g$ has support in the finite set of $a\in \mathcal{I}_{r't'}$ with $a\leq b\cdot c$, since if $x$ is a natural transformation with support outside this set, one has $x_{u_b\otimes u_c}=0$, and hence any of the $\big(\hat{\Delta}^l_s(x)\big)_{u_b,u_c} =0$.

\begin{Lem} The above product maps turn $(\mathscr{A},M)$ into an $I\times I$-partial algebra.
\end{Lem}
\begin{proof} We can extend the map $(\hat{\Delta}^l_s\otimes \id)$ on $\Nat(F_{rt},F_{km})\otimes \Nat(F_{tu},F_{mn})$ to a map \[(\hat{\Delta}^l_s\otimes \id): \Nat(F_{rt}\otimes F_{tu},F_{km}\otimes F_{mn}) \rightarrow  \Nat(F_{rs}\otimes F_{st}\otimes F_{tu},F_{kl}\otimes F_{lm}\otimes F_{mn}),\] with \[(\hat{\Delta}^l_s\otimes \id)(x)_{X,Y,Z} = \left(\pi^{(klm)}_{X,Y}\otimes \id_{F_{mn}(Z)}\right) x_{X\otimes Y, Z} \left(\iota^{(rst)}_{X,Y} \otimes \id_{F_{tu}(Z)}\right).\]
By finite support, we then have that \[((f\cdot g)\cdot h)(x) = (f\otimes g\otimes h)((\hat{\Delta}^l_s\otimes \id)\hat{\Delta}^m_t(x)), \quad \forall f\in \Gr{A}{k}{l}{r}{s},g\in \Gr{A}{l}{m}{s}{t},h\in \Gr{A}{m}{n}{t}{u}, x\in  \Nat(F_{ru},F_{kn}).\] Similarly, $((f\cdot g)\cdot h)(x) = (f\otimes g\otimes h)((\id\otimes \hat{\Delta}^m_t)\hat{\Delta}^l_s(x)).$ The associativity then follows from the 2-cocycle condition for the $\iota$- and $\pi$-maps. 

By a similar argument, one sees that the non-zero units are given by
$\UnitC{k}{l}\in \Gr{A}{k}{k}{l}{l}(\Unitb_{\alpha})$  (for
$\alpha=k'=l'$) corresponding to $1$ in the canonical
identifications  \[\Gr{A}{k}{k}{l}{l}(\alpha)
=\Hom_{\C}(F_{ll}(\Unitb_{\alpha}),F_{kk}(\Unitb_{\alpha}))^*\cong
\Hom_{\C}(\C,\C)^*  \cong \C. \qedhere\] 
\end{proof} 

\begin{Prop} The partial algebra and coalgebra structures on $\mathscr{A}$ define a partial bialgebra structure on $\mathscr{A}$. 
\end{Prop}
\begin{proof} We only show multiplicativity, which means showing that for each $x\in \Nat(F_{uw},F_{km})$ and $y\in \Nat(F_{rt},F_{uw})$ with all first or second indices in the same class of $\mathscr{I}$, one has pointwise that $\hat{\Delta}^l_s(x\circ y) = \sum_{v,v'=l'} \hat{\Delta}^v_s(x)\circ \hat{\Delta}^l_v(y)$ when  $l'=s'$. This follows from the fact that $\sum_v \pi^{(uvw)}_{X,Y}\iota^{(uvw)}_{X,Y} \cong \id_{F_{uw}(X\otimes Y)}$, where we again note that the left hand side sum is in fact finite.
\end{proof} 

\begin{Lem} Define $\phi: \Gr{A}{k}{k}{m}{m} \rightarrow \C$ as the functional which is zero on $\Gr{A}{k}{k}{m}{m}(a)$ with $a\neq \Unitb_{k'}$, and the canonical identification $\Gr{A}{k}{k}{m}{m}(k')\cong \C$ on the unit component (for $k'=m'$). Then the functional $\phi$ is an invariant integral.
\end{Lem}

\begin{proof} The normalisation condition $\phi(\UnitC{k}{k})=1$ is immediate by construction. Let $\hat{\phi}^k_l$ be the natural transformation from $F_{ll}$ to $F_{kk}$ which has support on multiples of $\Unitb_{k'}$, and with $(\hat{\phi}^k_l)_{\Unitb_{k'}} = 1$.  Then for $f\in \Gr{A}{k}{k}{l}{l}$, we have $\phi(f) = f(\hat{\phi}^k_l)$. The invariance of $\phi$ then follows from the easy verification that for $x\in \Nat(F_{ll},F_{kn})$ one has for example $x\circ \hat{\phi}^l_m =\delta_{k,n} \UnitC{k}{l}(x)\hat{\phi}^k_m$. 
\end{proof}

Let us further impose for the rest of this section that $\CatCC$ also admits (left and right) duality.  The following lemma just writes out how morphisms between partial tensor categories preserve duality.
\begin{Lem}
  For all $k,l$ and $X\in \mathcal{C}_{k',l'}$,  the maps
  \begin{align*}
    \coev^{kl}_{X}  &:=  \pi^{(klk)}_{X,\hat X} \circ F_{kk}(\coev_{X})\colon \C \to F_{kl}(X)
    \otimes F_{lk}(\hat X), \\
    \ev^{kl}_{X} &:=  F_{ll}(\ev_{X}) \circ \iota^{(lkl)}_{\hat X,X} \colon
    F_{lk}(\hat X) \otimes F_{kl}(X) \to \C
  \end{align*}
  define a duality between $F_{kl}(X)$ and $F_{lk}(\hat X)$.
\end{Lem}

\begin{Prop}\label{PropAnti} The partial bialgebra $\mathscr{A}$ is a regular partial Hopf algebra.
\end{Prop} 

\begin{proof} 
  For any $x\in \Nat(F_{mn},F_{kl})$, let us define $\hat{S}(x) \in
  \Nat(F_{lk},F_{nm})$ by \[\hat{S}(x)_X = 
(\id \otimes \ev^{lk}_{X}) \circ (\id \otimes x_{\hat X}
  \otimes \id) \circ (\coev^{nm}_{X} \otimes \id).\]
Then the assigment $\hat{S}$ dualizes to maps $S:\Gr{A}{k}{l}{m}{n} \rightarrow \Gr{A}{n}{m}{l}{k}$ by $S(f)(x) = f(\hat{S}(x))$. We claim that $S$ is an antipode for $\mathscr{A}$. Regularity will follow by considering $\mathscr{A}^{\textrm{op}}$ and right duality.

Let us check for example the formula $\sum_r f_{(nr;1)} S(f_{(nr;2)}) = \delta_{k,m}\epsilon(f)\UnitC{k}{n}$ for $f\in \Gr{A}{k}{l}{m}{l}$. By duality, this is equivalent to the pointwise identity of natural transformations $\sum_r\hat{M}^n_r(\id\otimes \hat{S})\hat{\Delta}^l_r(x) = \delta_{k,m}\UnitC{k}{n}(x) \id_{F_{kl}}$ for $x\in \Nat(F_{nn},F_{km})$, where $\hat{M}^n_r$ and $(\id\otimes \hat{S})$ are dual to respectively $\Delta_{nr}$ and $\id\otimes S$. 

Let us fix $X\in \mathcal{C}_{k'l'}$. Then for any $x\in
\Nat(F_{nr},F_{kl})$, $y\in \Nat(F_{rn},F_{lm})$, we have
\begin{align*}
  \left(\hat{M}^n_r(\id\otimes \hat{S})(x\otimes y)\right)_X =
\big(\id \otimes \ev_{X}^{ml}\big)  \big(x_{X} \otimes y_{\hat X} \otimes \id\big) 
  \big(\coev_{X}^{nr} \otimes \id\big).
\end{align*}
For any $x\in \Nat(F_{nn},F_{km})$, we therefore have
\begin{align*}
  \left(\hat{M}^n_r(\id\otimes \hat{S})\hat\Delta^{l}_{r}(x)\right)_X &=
\big(\id \otimes \ev^{ml}_{X}\big)  \big(\pi^{(klm)}_{X,\hat X}x_{X\otimes \hat
    X}\iota^{(nrm)}_{X,\hat X} \otimes \id\big) 
  \big(\coev^{nr}_{X} \otimes \id\big).
\end{align*}
We sum over $r$, use naturality of $x$, and obtain
\begin{align*}
\sum_{r}    \left(\hat{M}^n_r(\id\otimes
  \hat{S})\hat\Delta^{l}_{r}(x)\right)_X &=
\big(\id \otimes \ev_{X}^{ml}\big) \big(\pi^{(klm)}_{X,\hat X}x_{X\otimes \hat
    X}F_{nn}(\coev_{X}) \otimes \id\big) \\
  &=\delta_{k,m} \UnitC{k}{n}(x)
\big(\id \otimes \ev_{X}^{ml}\big) 
\big(\pi^{(mlm)}_{X,\hat X}F_{mm}(\coev_{X})
  \otimes \id\big) \\
  &=\delta_{k,m} \UnitC{k}{n}(x)
\big(\id \otimes \ev_{X}^{ml}\big) 
\big(\coev^{ml}_{X}
  \otimes \id\big) \\
  &= \delta_{k,m} \UnitC{k}{n}(x) \id. 
\end{align*}
\end{proof} 

Assume now that $\CatCC$ is a partial fusion C$^*$-category, and $F$ a $\varphi$-morphism from $\CatCC$ to $\{\Hilb_{\fin}\}_{I\times I}$. Let us show that $\mathscr{A}$, as constructed above, becomes a partial Hopf $^*$-algebra with positive invariant integral. In the following definition, we borrow the notation used in the proof of Proposition \ref{PropAnti}.

\begin{Def} We define $^*: \Gr{A}{k}{l}{m}{n}\rightarrow \Gr{A}{l}{k}{n}{m}$ by the formula $f^*(x) = \overline{f(\hat{S}(x)^*)}$ for $x\in \Nat(F_{nm},F_{lk}).$
\end{Def}

\begin{Lem} The operation $^*$ is an anti-linear, anti-multiplicative, comultiplicative involution.
\end{Lem}

\begin{proof} Anti-linearity is clear. Comultiplicativity follows from
  the fact that $(xy)^* = y^*x^*$ and $\hat{S}(xy) =
  \hat{S}(y)\hat{S}(x)$ for natural transformations. To see
  anti-multiplicativity of $^*$, note first that, since $S$ is
  anti-multiplicative for $\mathscr{A}$, the map $\hat{S}$ is anti-comultiplicative on natural transformations. Now as $(\iota_{X,Y}^{(klm)})^* = \pi_{X,Y}^{(klm)}$ by assumption, we also have $\hat{\Delta}^l_s(x)^* = \hat{\Delta}^s_l(x^*)$, which proves anti-multiplicativity of $^*$ on $\mathscr{A}$.  Finally, involutivity follows from the involutivity of $x\mapsto \hat{S}(x)^*$, which is a consequence of the fact that one can choose $\ev_{\bar{X}}^{kl} = (\coev_{X}^{lk})^*$ and $\coev_{\bar{X}}^{kl} = (\ev_X^{lk})^*$.
\end{proof}

\begin{Prop} The couple $(\mathscr{A},\Delta)$ with the above $^*$-structure defines a partial compact quantum group.
\end{Prop}
\begin{proof} The only thing which is left to prove is that our
  invariant integral $\phi$ is a positive functional. Now it is easily
  seen from the definition of $\phi$ that the $\Gr{A}{k}{l}{m}{n}(a)$
  are all mutually orthogonal. Hence it suffices to prove that the
  sesquilinear inner product $\langle f| g\rangle = \phi(f^*g)$ on
  $\Gr{A}{k}{l}{m}{n}(a)$ is positive-definite.

  Let us write $\bar{f}(x) = \overline{f(x^*)}$. Let again
  $\hat{\phi}^k_m$ be the natural transformation from $F_{mm}$ to
  $F_{kk}$ which is the identity on $\Unitb_{k'}$ and zero on other
  irreducible objects. Then by definition, $\phi(f^*g) =
  (\bar{f}\otimes g)((\hat{S}\otimes
  \id)\hat{\Delta}^k_m(\hat{\phi}^l_n)).$

Assume  that $f(x) = \langle v'| x_a v\rangle$ and
  $g(x) = \langle w' | x_aw\rangle$ for $v,w\in F_{mn}(u_a)$ and
  $v',w'\in F_{kl}(u_a)$. Then
  $\overline{f}(x) = \langle v|x_{a} v'\rangle$, and
 using the expression for $\hat{S}$ as
  in Proposition \ref{PropAnti} we find that
  \[
    \phi(f^*g) = \langle v \otimes w'|
    (\ev_{a}^{kl})_{23} 
    (\hat\Delta^{k}_{m}(\hat \phi^{l}_{n})_{\bar a, a})_{24} 
    (\coev^{mn}_{a})_{12} (v'\otimes
    w)\rangle.\]
  However, up to a positive non-zero scalar, which we may assume to be
  1 by proper rescaling, we
  have $\hat{\Delta}^k_m(\hat{\phi}^l_n)_{\bar{a}, a} =
  (\ev^{kl}_{a})^{*}(\ev^{kl}_{a}).$ Hence
  \begin{align*}
    \phi(f^*g) &=
\langle v \otimes w'|     (\ev^{kl}_{a})_{23}  (
  (\ev^{kl}_{a})^{*}(\ev^{kl}_{a}))_{24}
 (\coev^{mn}_{a})_{12} (v'\otimes w)\rangle \\
&= \langle v \otimes w'|        (\ev^{kl}_{a})_{23} 
  (\ev^{kl}_{a})^{*}_{24}
 (w\otimes v')\rangle \\
    &= \langle v|w\rangle (\ev^{kl}_{a}|v'\rangle_{2})
    (\ev_{a}^{kl}|w'\rangle_{2})^{*},
  \end{align*}
where $\ev_{a}^{kl}|z\rangle_{2}$ denotes the map $y \mapsto
\ev_{a}^{kl}(y\otimes z)$. This clearly defines a positive definite inner product on $\Gr{A}{k}{l}{m}{n}(a) \cong F_{mn}(u_a)\otimes F_{kl}(u_a)^*$.
\end{proof} 


Let us say that an $I$-partial compact quantum group with hyperobject set
$\mathscr{I}$ and corresponding partition function $\varphi \colon \mathscr{I} \to
\mathscr{P}(I)$  is \emph{based over $\varphi$}.
\begin{Theorem} \label{TheoTKPCQG} Fix sets $\mathcal{I}$ and $I$ with partition $\varphi:\mathscr{I}\rightarrow \mathscr{P}(I)$. Then the assigment $\mathscr{A}\rightarrow (\Corep_u(\mathscr{A}),F)$ is (up to isomorphism/equivalence) a one-to-one correspondence between partial compact quantum groups based over $\varphi$ and $\mathscr{I}$-partial fusion C$^*$-categories $\CatCC$ with $\varphi$-morphism $F$ to $\{\Hilb_{\fin}\}_{I\times I}$.
\end{Theorem} 

\begin{proof} Fix first $\mathscr{A}$, and let $\mathscr{B}$ be the
  partial Hopf $^*$-algebra constructed from $\Corep_u(\mathscr{A})$
  with its natural forgetful functor. Then we have a map $\mathscr{B}
  \rightarrow \mathscr{A}$ which piecewise goes from
  $\Gr{B}{k}{l}{m}{n}(a) =
  \Hom(\Gr{V}{}{(a)}{m}{n},\Gr{V}{}{(a)}{k}{l})^*$ to
  $\Gr{A}{k}{l}{m}{n}(a)$ sending $f$ to $(\id\otimes f)(X_a)$, where
  the $(V^{(a)},\mathscr{X}_a)$ run over all irreducible unitary
  corepresentations of $\mathscr{A}$. It is easy to check from the
  definition of $\mathscr{B}$ that this map is an morphism of partial
  Hopf $^*$-algebras.  By Corollary \ref{cor:rep-pw}, it is 
  bijective.

Conversely, let $\CatCC$ be an $\mathscr{I}$-partial fusion C$^*$-category with $\varphi$-morphism $F$ to $(\Hilb_f)_{I\times I}$. Let $\mathscr{A}$ be the associated partial Hopf $^*$-algebra. For each irreducible $u_a \in \CatCC$, let $V^{(a)} = F(u_a)$, and $\Gr{(X_a)}{k}{l}{m}{n} = \sum_i e_i^*\otimes e_i,$ where $e_i$ is a basis of $\Hom_{\C}(F_{mn}(u_{a}),F_{kl}(u_{a}))$ and $e_i^*$ a dual basis. Then from the definition of $\mathscr{A}$ it easily follows that $\mathscr{X}_a$ is a unitary corepresentation for $\mathscr{A}$. Clearly, $\mathscr{X}_a$ is irreducible. As the matrix coefficients of the $\mathscr{X}_a$ span $\mathscr{A}$, it follows that the $\mathscr{X}_a$ form a maximal class of non-isomorphic unitary corepresentations of $\mathscr{A}$. Hence we can make a unique equivalence $\CatCC\rightarrow \Corep_u(\mathscr{A})$ sending $u$ to $(F(u),\mathscr{X}_u)$ such that $u_a\rightarrow \mathscr{X}_a$. From the definitions of the coproduct and product in $\mathscr{A}$, it is readily verified that the natural morphisms $\iota^{(klm)}_{u,v}:F_{kl}(u)\otimes F_{lm}(v)\rightarrow F_{km}(u\otimes v)$ turn it into a monoidal equivalence. 
\end{proof}


\section{Examples}

\subsection{Hayashi's canonical partial compact quantum groups} \label{SubSecCan}

Let $\CatCC$ be an $\mathscr{I}$-partial fusion C$^*$-category. A \emph{semi-simple partial $\CatCC$-module C$^*$-category $\CatDD$} consists of a collection of (non-trivial) semi-simple C$^*$-categories $\{\CatD_{\alpha}\}_{\alpha\in \mathscr{I}}$ and  tensor products $\otimes: \CatC_{\alpha\beta}\times \CatD_{\beta}\rightarrow \CatD_{\alpha}$, together with unitary associativity constraints $X\otimes (Y\otimes V)\rightarrow (X\otimes Y)\otimes V$ and unitary unit constraints $\mathbbm{1}_{\alpha} \otimes V \rightarrow V$ for $V\in \CatD_{\alpha}$.

Choose a labeling $\mathcal{I}_{\alpha}$ for a distinguished maximal set $\{u_a\}$ of mutually non-isomorphic irreducible objects of $\CatD_{\alpha}$, and let $\mathcal{I} = \sqcup \mathcal{I}_{\alpha}$. Write $a' = \alpha$ if $a\in I_{\alpha}$. Then we can define \[F_{ab}(X)  = \Hom(u_a,  X\otimes u_b),\qquad X\in \CatC_{a'b'},\] which is a Hilbert space for the inner product $\langle f| g \rangle = f^*g$. It is easy to check that one obtains in a natural way a morphism from $\CatCC$ to $\{\Hilb_{\fin}\}_{\mathcal{I}\times \mathcal{I}}$ based over the partition $\mathcal{I} = \cup \mathcal{I}_{\alpha}$. The associated partial Hopf $^*$-algebra $\mathscr{A}_{(\CatCC,\CatDD)}$ will be called the \emph{canonical partial compact quantum group} associated with $(\CatCC,\CatDD)$. 

For example, given a $\varphi$-morphism $F:\CatCC \rightarrow  \{\Hilb_{\fin}\}_{I\times I}$ and defining $\CatDD_{\alpha} =\Hilb_{\fin}^{I_{\alpha}}$ the category of $I_{\alpha}$-graded finite-dimensional Hilbert spaces $V = \oplus_k\, \GrDA{V}{k}{}$ with $\GrDA{\left(X\otimes V\right)}{k}{} = \oplus_{l\in I_{\beta}} F_{kl}(X)\otimes \GrDA{V}{l}{}$ for $V\in \Hilb_{\fin}^{I_{\beta}}$ and $X\in \CatC_{\alpha\beta}$, we have the ordinary reconstruction obtained in the previous section. 

If $\CatCC$ is a partial fusion C$^*$-category, one can take $\CatD_{\beta}$ the full subcategory of $\CatC$ spanned by all $\CatC_{\alpha\beta}$, endowed with the module structure coming from the tensor product of $\CatCC$. The associated partial Hopf $^*$-algebra $\mathscr{A}_{\CatCC}$ coincides with Hayashi's construction \cite{Hay8} in case $\CatCC$ has only finitely many irreducible object classes. 

As a more concrete example, let $\G$ be a compact quantum group with ergodic action on a unital C$^*$-algebra $C(\mathbb{X})$. Then the collection of finitely generated $\G$-equivariant $C(\mathbb{X})$-Hilbert modules forms a module C$^*$-category over $\Rep_u(\G)$, cf.~ \cite{DCY1}. 

\subsection{Morita equivalence}

In this section, we will use the notation from Definition \ref{DefPCQG}, and we then also write $\Rep_u(\mathscr{G})$ in stead of $\Corep_u(P(\mathscr{G}))$.


\begin{Def} Two partial compact quantum groups $\mathscr{G}$ and $\mathscr{H}$ are said to be \emph{Morita equivalent} if there exists an equivalence $\Rep_u(\mathscr{G}) \rightarrow \Rep_u(\mathscr{H})$ of partial fusion C$^*$-categories. 
\end{Def} 

In particular, if $\mathscr{G}$ and $\mathscr{H}$ are Morita equivalent they have the same hyperobject set, but they need not share the same object set.

\begin{Def} A \emph{linking partial compact quantum group} consists of a partial compact quantum group $\mathscr{G}$ defined by a partial Hopf $^*$-algebra $\mathscr{A}$ over a set $I$ with a distinguished partition $I = I_1\sqcup I_2$ such that the units $\UnitC{i}{j} = \sum_{k\in I_i,l\in I_j} \UnitC{k}{l} \in M(A)$ are central, and such that for each $r\in I_i$, there exists $s\in I_{i+1}$ such that $\UnitC{r}{s}\neq 0$ (with the indices $i$ considered modulo 2).
\end{Def}

If $\mathscr{A}$ defines a linking partial compact quantum group, we can split $A$ into four components $A^i_j = A\UnitC{i}{j}$. It is readily verified that the $A^i_i$ together with all $\Delta_{rs}$ with $r,s \in I_i$ define themselves partial compact quantum groups, called the \emph{corner} partial compact quantum groups of $\mathscr{A}$. 

\begin{Prop} Two partial compact quantum groups are Morita equivalent iff they arise as the corners of a linking partial compact quantum group.
\end{Prop}

\begin{proof} Suppose first that $\mathscr{G}_1$ and $\mathscr{G}_2$ are Morita equivalent partial compact quantum groups with associated partial Hopf $^*$-algebras $\mathscr{A}_1$ and $\mathscr{A}_2$ over respective sets $I_1$ and $I_2$. Then we may identify their corepresentation categories with the same abstract partial tensor C$^*$-category $\CatCC$ based over their common hyperobject set $\mathscr{I}$. This $\CatCC$ comes endowed with two forgetful functors $F_i$ to $\{\Hilb_{\fin}\}_{I_i\times I_i}$ corresponding to the respective $\mathscr{A}_i$.

With $I = I_1\sqcup I_2$, we can combine the $F_i$ into a global morphism $F:\CatCC \rightarrow \{\Hilb_{\fin}\}_{I\times I}$, with $F_{kl}(X)=(F_i)_{kl}(X)$ if $k,l\in I_i$ and $F_{kl}(X)=0$ otherwise. Let $\mathscr{A}$ be the associated partial Hopf $^*$-algebra constructed from the Tannaka-Kre$\breve{\textrm{\i}}$n-Woronowicz reconstruction procedure. 

From the precise form of this reconstruction, it follows immediately that $\Gr{A}{k}{l}{m}{n} =0$ if either $k,l$ or $m,n$ do not lie in the same $I_i$. Hence the $\UnitC{i}{j} = \sum_{k\in I_i,l\in I_j} \UnitC{k}{l}$ are central. 

Moreover, fix $k\in I_i$ and any $l\in I_{i+1}$ with $k'=l'$. Then $\Nat(F_{ll},F_{kk})\neq \{0\}$. It follows that $\UnitC{k}{l}\neq 0$. Hence $\mathscr{A}$ is a linking compact quantum group. It is clear that $\mathscr{A}_1$ and $\mathscr{A}_2$ are the corners of $\mathscr{A}$. 

Conversely, suppose that $\mathscr{A}_1$ and $\mathscr{A}_2$ arise from the corners of a linking partial compact quantum group defined by $\mathscr{A}$ with invariant integral $\phi$. We will show that the associated partial compact quantum groups $\mathscr{G}$ and $\mathscr{G}_1$ are Morita equivalent. Then by symmetry $\mathscr{G}$ and $\mathscr{G}_2$ are Morita equivalent, and hence also $\mathscr{G}_1$ and $\mathscr{G}_2$.

For $(V,\mathscr{X}) \in \Corep_u(\mathscr{A})$, let $F(V,\mathscr{X}) = (W,\mathscr{Y})$ be the pair obtained from $(V,\mathscr{X})$ by restricting all indices to those appearing to $I_1$. It is immediate that $(W,\mathscr{Y})$ is a unitary corepresentation of $\mathscr{A}_1$, and that the functor $F$ hence becomes a unital morphism in a trivial way. What remains to show is that $F$ is an equivalence of categories, i.e.~ that $F$ is faithful and essentially surjective. 

By assumption, the hyperobject set of a linking partial compact group coincides with the hyperobject sets of its corners, hence $F$ is faithful by the remark under Definition \ref{DefParFus}.
To complete the proof, we need to show that $F$ induces a
bijection between isomorphism classes of irreducible unitary 
corepresentations of $\mathscr{A}$ and of $\mathscr{A}_{1}$. Note that
by Proposition \ref{prop:rep-cosemisimple} and Lemma
\ref{lemma:rep-regular-embedding}, each such class can be represented
by a restriction of the regular corepresentation of $\mathscr{A}$ or
$\mathscr{A}_{1}$, respectively.

So, let $(W,\mathscr{Y})$ be an irreducible restriction of the regular
corepresentation of $\mathscr{A}_{1}$. Pick a non-zero $a \in
\Gru{W}{m}{n}$, define $\Gru{V}{p}{q} \subseteq \bigoplus_{k,l}
\Gr{A}{k}{l}{p}{q}$ as in \eqref{lemma:regular-corep} and form the
regular corepresentation $(V,\mathscr{X})$ of $\mathscr{A}$. Then
$\Gru{V}{p}{q} = \Gru{W}{p}{q}$ for all $p,q\in I_{1}$ by Lemma
\ref{lemma:regular-corep} 2.\ and hence $F(V,\mathscr{X}) =
(W,\mathscr{Y})$.  Since $F$ is faithful, $(V,\mathscr{X})$ must be
irreducible.

Conversely, let $(V,\mathscr{X})$ be an irreducible restriction of the
regular corepresentation of $\mathscr{A}$. Since $F$ is faithful,
there exist $k,l\in I_{1}$ such that $\Gru{V}{k}{l}\neq 0$. Applying
Corollary \ref{cor:rep-pw}, we may assume that
$\Gru{V}{p}{q} \subseteq \Gr{A}{k}{l}{p}{q}$ for some $k,l\in I_{1}$
and all $p,q\in I$. But then $F(V,\mathscr{X})$ is a restriction of
the regular corepresentation of $\mathscr{A}_{1}$.  If
$F(V,\mathscr{X})$ would decompose into a direct sum of several
irreducible corepresentations, then the same would be true for
$(V,\mathscr{X})$ by the argument above. Thus, $F(V,\mathscr{X})$ is irreducible.

Finally, assume that
$(V,\mathscr{X})$ and $(W,\mathscr{Y})$ are 
inequivalent irreducible unitary corepresentations of $\mathscr{A}$. Then $\phi(\mathcal{C}(V,\mathscr{X})^*\mathcal{C}(W,\mathscr{Y}))=0$ by Corollary \ref{cor:rep-unitary-schur-orthogonality}. Since $\phi$ is faithful, $\mathcal{C}(V,\mathscr{X}) \cap
\mathcal{C}(W,\mathscr{Y})=0$, and hence $\mathcal{C}(F(V,\mathscr{X}))
\cap\mathcal{C}(F(W,\mathscr{Y})) =0$. So $F(V,\mathscr{X})$ and
$F(W,\mathscr{Y})$ are inequivalent.
\end{proof}

If $\mathscr{G}_1$ and $\mathscr{G}_2$ are Morita equivalent compact quantum groups, the total partial compact quantum group coincides with the co-groupoid constructed in \cite{Bic1}. 

\subsection{Weak Morita equivalence}

\begin{Def} A \emph{linking} partial fusion C$^*$-category consists of a partial fusion C$^*$-category with a distinguished partition $\mathscr{I} =\mathscr{I}_1 \cup \mathscr{I}_2$ such that for each $\alpha\in \mathscr{I}_1$, there exists $\beta \in \mathscr{I}_{2}$ with $\CatC_{\alpha\beta}\neq \{0\}$.

The \emph{corners} of $\CatCC$ are the restrictions of $\CatCC$ to $\mathscr{I}_1$ and $\mathscr{I}_2$.
\end{Def}

The following notion is essentially the same as the one by M. M\"{u}ger \cite{Mug1}. 

\begin{Def} Two partial semi-simple tensor C$^*$-categories $\CatCC_1$ and $\CatCC_2$ with duality over respective sets $\mathscr{I}_1$ and $\mathscr{I}_2$ are called \emph{Morita equivalent} if there exists a linking partial fusion C$^*$-category $\CatCC$ over the set $\mathscr{I}=\mathscr{I}_1\sqcup \mathscr{I}_2$ whose corners are isomorphic to $\CatCC_1$ and $\CatCC_2$.

We say two partial compact quantum groups $\mathscr{G}_1$ and $\mathscr{G}_2$ are \emph{weakly Morita equivalent} if their representation categories $\Rep_u(\mathscr{G}_i)$ are Morita equivalent. 
\end{Def} 

One can directly prove that this is indeed an equivalence relation, but it will follow indirectly from the discussion below.

The following notion is dual to that of linking partial compact quantum group.

\begin{Def}\label{DefCoLink} A \emph{co-linking partial compact quantum group} consists of a partial compact quantum group $\mathscr{G}$ defined by a partial Hopf $^*$-algebra $\mathscr{A}$ over an index set $I$, together with a distinguished partition $I = I_1\cup I_2$ such that  $\UnitC{k}{l}=0$ whenever $k\in I_i$ and $l\in I_{i+1}$, and such that for each $k\in I_i$, there exists $l\in I_{i+1}$ with $\Gr{A}{k}{l}{k}{l}\neq 0$.  
\end{Def} 

It is again easy to see that if we restrict all indices of a co-linking partial compact quantum group to one of the distinguished sets, we obtain a partial compact quantum group which we will call a corner. If we write $e_i = \sum_{k,l\in I_i} \UnitC{k}{l}$, we can decompose the total algebra $A$ into components $A_{ij} = e_{i}Ae_{j}$, and correspondingly write $A$ in matrix notation $A = \begin{pmatrix} A_{11} & A_{12}  \\ A_{21} & A_{22}\end{pmatrix}$.

\begin{Lem}  If $\mathscr{A}$ is a co-linking partial compact quantum group, then  $A_{ij}A_{jk} = A_{ik}$.
\end{Lem}
\begin{proof} It suffices to show $A_{12}A_{21} = A_{11}$. Take $k\in I_1$, and pick $l\in I_2$ with $\Gr{A}{k}{l}{k}{l}\neq \{0\}$. Then in particular, we can find an $a\in \Gr{A}{k}{l}{k}{l}$ with $\epsilon(a)\neq 0$. Hence for any $m\in I_1$, we have $\UnitC{k}{m} = \UnitC{k}{m} a_{(1)}S(a_{(2)}) \in A_{12}A_{21}$. As this latter space contains all local units of $A_{11}$ and is a right $A_{11}$-module, it follows that it is in fact equal to $A_{11}$. 
\end{proof}

It follows that $A_{11}$ and $A_{22}$ are Morita equivalent algebras.

\begin{Def} We call two partial compact quantum groups \emph{co-Morita equivalent} if there exists a \emph{co-linking partial compact quantum group} having these partial compact quantum groups as its corners.
\end{Def}

% We skip the proof of the following lemma, which can be proven along similar lines to the proof that Morita equivalence of algebras defines an equivalence relation.

\begin{Lem} Co-Morita equivalence is an equivalence relation. 
\end{Lem} 

\begin{proof} The idea is standard, and consists in building a 3 by 3 matrix $A_{\{1,2,3\}}= \begin{pmatrix} A_{11} & A_{12} &   A_{13} \\ A_{21} & A_{22} & A_{23} \\ A_{31} & A_{32} & A_{33} \end{pmatrix}$ having two co-linking partial quantum groups in the upper left and lower right 2 by 2 corners between resp.~ $\mathscr{G}_1$ and $\mathscr{G}_2$, and $\mathscr{G}_2$ and $\mathscr{G}_3$. For example, $A_{13}$ is constructed as $A_{12}\underset{A_{22}}{\otimes } A_{23}$. It is straightforward to define a regular weak multiplier Hopf $^*$-algebra on $A_{\{1,2,3\}}$ satisfying the conditions of Proposition \ref{PropCharPBA}. 

Let now $\phi$ be the functional which is zero on the off-diagonal entries $A_{ij}$ and which coincides with the invariant positive integrals on the $A_{ii}$. Then it is also easily checked that $\phi$ is invariant. To show that $\phi$ is positive, we invoke Remark \ref{RemPos}. Indeed, any irreducible corepresentation of $A_{\{1,2,3\}}$ has coefficients in a single $A_{ij}$. For those $i,j$ with $|i-j|\leq 1$, we know that the corepresentation is unitarizable by restricting to a corner $2\times 2$-block. If however the corepresentation $\mathscr{X}$ has coefficients living in (say) $A_{13}$, it follows from the identity $A_{12}A_{23}=A_{13}$ that the corepresentation is a direct summand of a product $\mathscr{Y}\Circt \mathscr{Z}$ of corepresentations with coefficients in respectively $A_{12}$ and $A_{23}$. This proves unitarizability of $\mathscr{X}$. It follows from Remark \ref{RemPos} that $\phi$ is positive, and hence $\mathscr{A}_{\{1,2,3\}}$ defines a partial compact quantum group.  

We claim that the subspace $\mathscr{A}_{\{1,3\}}$ (in the obvious notation) defines a co-linking compact quantum group between $\mathscr{G}_1$ and $\mathscr{G}_3$. In fact, it is clear that the $\mathscr{A}_{11}$ and $\mathscr{A}_{33}$ are corners of $\mathscr{A}_{\{1,3\}}$, and that $\UnitC{k}{l}=0$ for $k,l$ not both in $I_1$ and $I_{3}$. To finish the proof, it is sufficient to show now that for each $k\in I_1$, there exists $l\in I_{3}$ with $\Gr{A}{k}{l}{k}{l}\neq 0$, as the other case follows by symmetry using the antipode. But there exists $m\in I_2$ with $\Gr{A}{k}{m}{k}{m} \neq \{0\}$, and $l\in I_3$ with $\Gr{A}{m}{l}{m}{l}\neq\{0\}$. As in the discussion following Definition \ref{DefCoLink}, this implies that there exists $a\in \Gr{A}{k}{m}{k}{m}$ and $b\in \Gr{A}{m}{l}{m}{l}$ with $\epsilon(a)=\epsilon(b)=1$. Hence $\epsilon(ab)=1$, showing $\Gr{A}{k}{l}{k}{l}\neq \{0\}$.
\end{proof} 


\begin{Prop}\label{PropCoWeak} Assume that two partial compact quantum groups $\mathscr{G}_1$ and $\mathscr{G}_2$ are co-Morita equivalent. Then they are weakly Morita equivalent.
\end{Prop} 
\begin{proof} 
Consider the corepresentation category $\CatCC$ of a co-linking partial compact quantum group $\mathscr{A}$ over $I = I_1\cup I_2$. Let $\varphi:I\rightarrow \mathscr{I}$ define the corresponding partition along the hyperobject set. Then by the defining property of a co-linking partial compact quantum group, also $\mathscr{I} = \mathscr{I}_1\cup \mathscr{I}_2$ with $\mathscr{I}_i=\varphi(I_i)$ is a partition. Hence $\CatCC$ decomposes into parts $\CatCC_{ij}$ with $i,j\in \{1,2\}$ and $\CatC_{ii}\cong \Rep_u(\mathscr{G}_i)$. 

To show that $\mathscr{G}_1$ and $\mathscr{G}_2$ are weakly Morita equivalent, it thus suffices to show that $\{\CatCC_{ij}\}$ forms a linking partial fusion C$^*$-category. But fix $\alpha\in I_1$ and $k\in I_{\alpha}$. Then as $\mathscr{A}$ is co-linking, there exists $l \in I_2$ with $\Gr{A}{k}{l}{k}{l}\neq \{0\}$. Hence there exists a non-zero regular  unitary corepresentation inside $\oplus_{m,n}\Gr{A}{k}{l}{m}{n}$. If then $l\in I_{\beta}$ with $\beta\in \mathscr{I}_2$, it follows that $\CatC_{\alpha\beta}\neq 0$. By symmetry, we also have that for each $\alpha \in \mathscr{I}_2$ there exists $\beta \in \mathscr{I}_1$ with $\CatC_{\alpha\beta}\neq \{0\}$. This proves that the $\{\CatCC_{ij}\}$ forms a linking partial fusion C$^*$-category.
\end{proof}

\begin{Prop}\label{PropCoLink} Let $\CatCC$ be a linking $\mathscr{I}$-partial fusion C$^*$-category. Then the associated canonical partial compact quantum group is a co-linking partial compact quantum group. 
\end{Prop} 

\begin{proof} Let $\mathscr{I}= \mathscr{I}_1\cup \mathscr{I}_2$ be the associated partition of $\mathscr{I}$. Let $\mathscr{A} = \mathscr{A}_{\CatCC}$ define the canonical partial compact quantum group with object set $I$ and hyperobject partition $\varphi:I\rightarrow \mathscr{I}$. Let $I=I_1\cup I_2$ with $I_i = \varphi^{-1}(\mathscr{I}_i)$ be the corresponding decomposition of $I$. By construction, $\UnitC{k}{l}=0$ if $k$ and $l$ are not both in $I_1$ or $I_2$. 

Fix now $k\in I_{\alpha}$ for some $\alpha \in I_i$. Pick $\beta\in I_{i+1}$ with $\CatC_{\alpha\beta}\neq\{0\}$, and let $(V,\mathscr{X})$ be a non-zero irreducible corepresentation inside $\CatC_{\alpha\beta}$. Then by irreducibility, we know that $\oplus_l \Gru{V}{k}{l} \neq \{0\}$, hence there exists $l\in I_{\beta}$ with $\Gru{V}{k}{l}\neq \{0\}$. As $(\epsilon\otimes \id)\Gr{X}{k}{l}{k}{l} = \id_{\Gru{V}{k}{l}}$, it follows that $\Gr{A}{k}{l}{k}{l} \neq 0$. This proves that $\mathscr{A}$ defines a co-linking partial compact quantum group.
\end{proof} 

Note that the corners of the canonical partial compact quantum group associated to linking $\mathscr{I}$-partial fusion C$^*$-category \emph{are not} the canonical partial compact quantum groups associated to the corners of the linking $\mathscr{I}$-partial fusion C$^*$-category. Rather, they are Morita equivalent copies of these.

\begin{Theorem} Two partial compact quantum groups $\mathscr{G}_1$ and $\mathscr{G}_2$ are weakly Morita equivalent if and only if they are connected by a string of Morita and co-Morita equivalences. 
\end{Theorem}

\begin{proof} Clearly if two partial compact quantum groups are Morita
  equivalent, they are weakly Morita equivalent. By Proposition
  \ref{PropCoWeak}, the same is true for co-Morita equivalence. This proves one direction of the theorem. 

Conversely, assume $\mathscr{G}_1$ and $\mathscr{G}_2$ are weakly Morita equivalent. Let $\CatCC$ be a linking fusion C$^*$-category between $\Rep_u(\mathscr{G}_1)$ and $\Rep_u(\mathscr{G}_2)$. Then $\mathscr{G}_i$ are Morita equivalent with the corners of the canonical partial compact quantum group associated to $\CatCC$. But Proposition \ref{PropCoLink} shows that these corners are co-Morita equivalent. 
\end{proof} 


\section{Partial compact quantum groups from reciprocal random walks}\label{SecDyn}

In this section, we investigate a special class of partial compact quantum groups constructed from \emph{$t$-reciprocal random walks} (\cite{DCY1}). We first recall this notion, slightly changing the terminology for the sake of convenience.

\begin{Def} Let $t\in \R_0$. A \emph{$t$-reciprocal random walk} consists of a quadruple $(\Gamma,w,\sgn,i)$ where $\Gamma=(\Gamma^{(0)},\Gamma^{(1)},s,t)$ is a graph with source and target maps $s$ and $t$, where $w$ is a weight function $w:\Gamma^{(1)}\rightarrow \R_0^+$ and $\sgn$ a sign function $\sgn:\Gamma^{(1)}\rightarrow \{\pm 1\}$, and where $i$ is an involution  $e\mapsto \overline{e}$ on $\Gamma^{(1)}$ interchanging source and target,  satisfying for all $e$ the weight reciprocality $w(e)w(\bar{e}) = 1$, the sign reciprocality $\sgn(e)\sgn(\bar{e}) = \sgn(t)$, and the random walk property $\sum_{s(e)=v}  \frac{1}{|t|}w(e) = 1$ for all $v\in \Gamma^{(0)}$.
\end{Def}
 

By \cite[Proposition 3.1]{DCY1}, there is a uniform bound on the number of edges leaving from any given vertex $v$, i.e.~ $\Gamma$ has a finite degree. For examples of $t$-reciprocal random walks, we refer to \cite{DCY1}. 

Let now $0<|q|\leq 1$, and let $\mathcal{T}_q$ be the Temperley-Lieb C$^*$-tensor category, which is the universal tensor C$^*$-category with irreducible unit and duality, generated by a single self-adjoint object $X$ and duality morphism $R: \mathbbm{1} \rightarrow  X\otimes X$ satisfying 
\begin{align*} R^*R= |q|+|q|^{-1}, &&(R^*\otimes \id_X)(\id_X\otimes R) = -\sgn(q)\id_X.\end{align*} Then if $\Gamma = (\Gamma,w,\sgn,i)$ is a $-(q+q^{-1})$-reciprocal random walk, we have a morphism $F_{\Gamma}$ from $\mathcal{T}_q$ into $\Hilbrcf$ with $I= \Gamma^{(0)}$, by sending $X$ to the bigraded Hilbert space  $\Hsp^{\Gamma}=l^2(\Gamma^{(1)})$, where the $\Gamma^{(0)}$-bigrading is given by $\delta_e \in \Gru{\Hsp^{\Gamma}}{s(e)}{t(e)}$, and $R$ to the morphism \[R_{\Gamma}:l^2(\Gamma^{(0)})\rightarrow \Hsp^{\Gamma}\underset{\Gamma^{(0)}}{\otimes} \Hsp^{\Gamma},\quad R_{\Gamma} \delta_v =\sum_{e,s(e) = v} \sgn(e)\sqrt{w(e)}\delta_e \otimes \delta_{\bar{e}}.\] Note that $\Hsp^{\Gamma}$ is rcfd as $\Gamma$ has finite degree.  Up to equivalence, $F_{\Gamma}$ only depends upon the isomorphism class of $(\Gamma,w)$, and is independent of the chosen involution or sign structure. Conversely, every (unital) morphism from $\mathcal{T}_q$ into $\Hilbrcf$ for some set $I$ arises in this way \cite{DCY2}.

Let us denote by $\mathscr{A}(\Gamma)$ the $I$-partial compact quantum group associated to this morphism by the Tannaka-Kre$\breve{\textrm{\i}}$n-Woronowicz reconstruction result. Our aim is to give a direct representation of the associated algebra $A(\Gamma)$ by generators and relations. We will write $\Gamma_{vw}\subseteq \Gamma^{(1)}$ for the set of edges with source $v$ and target $w$.

\begin{Theorem}\label{TheoGenRel} The $^*$-algebra $A(\Gamma)$ is the universal $^*$-algebra generated by self-adjoint orthogonal idempotents $\UnitC{v}{w}$ for $v,w\in \Gamma^{(0)}$ and elements $(u_{e,f})_{e,f\in \Gamma^{(1)}}$ where the $u_{e,f}\in \Gr{A(\Gamma)}{s(e)}{t(e)}{s(f)}{t(f)}$ satisfy $u_{e,f}^* = \sgn(e)\sgn(f)\sqrt{\frac{w(f)}{w(e)}} u_{\bar{e},\bar{f}}$ and 
\begin{eqnarray} 
\label{EqUni1} \sum_{t(g)=w} u_{g,e}^*u_{g,f} = \delta_{e,f}\mathbf{1}\Grru{w}{t(e)}, \qquad \forall w\in \Gamma^{(0)}, e,f\in \Gamma^{(1)},\\ 
\label{EqUni2}\sum_{s(g)=v} u_{e,g}u_{f,g}^* = \delta_{e,f} \mathbf{1}\Grru{s(e)}{v}\qquad \forall v\in \Gamma^{(0)}, e,f\in \Gamma^{(1)}.
\end{eqnarray}

The partial Hopf $^*$-algebra structure is given by $\Delta_{vw}(u_{e,f}) = \underset{t(g) = w}{\underset{s(g) = v}{\sum}} u_{e,g}\otimes u_{g,f}$, $\varepsilon(u_{e,f}) = \delta_{e,f}$ and $S(u_{e,f}) = u_{f,e}^*$. 
\end{Theorem} 

Note that the sums in \eqref{EqUni1} and \eqref{EqUni2} are in fact finite, as $\Gamma$ has finite degree. 

\begin{proof} Let $(\Hsp,V)$ be the generating unitary corepresentation of $A(\Gamma)$ on $\Hsp = l^2(\Gamma^{(1)})$. Then $V$ decomposes into parts $\Gr{V}{k}{l}{m}{n} = \sum_{e,f} v_{e,f} \otimes E_{e,f} \in \Gr{A}{k}{l}{m}{n}\otimes B(\Gru{\Hsp}{m}{n},\Gru{\Hsp}{k}{l}),$ where the $E_{e,f}$ are elementary matrix coefficients and with the sum over all $e$ with $s(e)=k,t(e)=l$ and all $f$ with $s(f) = m, t(f)=n$. By construction $V$ defines a unitary corepresentation of $A(\Gamma)$, hence the relations \eqref{EqUni1} and \eqref{EqUni2} are satisfied for the $v_{e,f}$. Now as $R_{\Gamma}$ is an intertwiner between the trivial representation on $\C^{(\Gamma^{(0)})} = \oplus_{v\in \Gamma^{(0)}} \C$ and $V\Circtv{\Gamma^{(0)}} V$, we have for all $v\in \Gamma^{(0)}$ that \begin{equation}\label{EqMorR}\underset{t(f)=s(h),t(e)=s(g)}{\sum_{e,f,g,h\in \Gamma^{(1)}}} v_{e,f}v_{g,h}\otimes \left((E_{e,f}\otimes E_{g,h}) R_{\Gamma} \delta_v\right) = \sum_w \UnitC{w}{v}\otimes R_{\Gamma}\delta_v,\end{equation} hence
\[\underset{t(e)=s(g),s(k)=v}{\sum_{e,g,k}} \sgn(k)\sqrt{w(k)}\left( v_{e,k}v_{g,\bar{k}} \otimes \delta_e\otimes \delta_{g}\right) =\underset{s(k)=w}{\sum_{w,k}}\sgn(k)\sqrt{w(k)} \left(\UnitC{w}{v} \otimes \delta_k\otimes \delta_{\bar{k}}\right).\] So if $t(e) = s(g)=z$, we have $\sum_{k,s(k)=v} \sgn(k)\sqrt{w(k)} v_{e,k}v_{g,\bar{k}} =  \delta_{e,\bar{g}} \sgn(e)\sqrt{w(e)}\UnitC{s(e)}{v}.$ Multiplying to the left with $v_{e,l}^*$ and summing over all $e$ with $t(e) = z$, we see from \eqref{EqUni1} that also $v_{e,f}^* = \sgn(e)\sgn(f)\sqrt{\frac{w(f)}{w(e)}} v_{\bar{e},\bar{f}}$ holds. Hence the $v_{e,f}$ satisfy the universal relations in the statement of the theorem. The formulas for comultiplication, counit and antipode then follow immediately from the fact that $V$ is a unitary corepresentation.

Let us now a priori denote by $B(\Gamma)$ the $^*$-algebra determined by the relations \eqref{EqUni1},\eqref{EqUni2} and the relation for the adjoint as above, and write $\mathscr{B}(\Gamma)$ for the associated $\Gamma^{(0)}\times \Gamma^{(0)}$-partial $^*$-algebra induced by the local units $\UnitC{v}{w}$. Write $\Delta(\UnitC{v}{w}) = \sum_{z\in \Gamma^{(0)}} \UnitC{v}{z}\otimes \UnitC{z}{w}$ and $\Delta(u_{e,f}) = \sum_{g\in \Gamma^{(1)}} u_{e,g}\otimes u_{g,f},$ which makes sense in $M(B(\Gamma)\otimes B(\Gamma))$ as the degree of $\Gamma$ is finite. Then an easy computation shows that $ \sum_{t(g)=w}\Delta(u_{g,e})^*\Delta(u_{g,f})= \delta_{e,f} \Delta(\UnitC{w}{t(e)})$. Similarly, the analogue of \eqref{EqUni2} holds for $\Delta(u_{e,f})$. As also the relation for the adjoint holds trivially for $\Delta(u_{e,f})$, it follows that we can define a $^*$-algebra homomorphism $\Delta:B(\Gamma)\rightarrow M(B(\Gamma)\otimes B(\Gamma))$ sending $u_{e,f}$ to $\Delta(u_{e,f})$ and $\UnitC{v}{w}$ to $\Delta(\UnitC{v}{w})$. Cutting down, we obtain maps $\Delta_{vw}:\Gr{B(\Gamma)}{r}{s}{t}{z}\rightarrow \Gr{B(\Gamma)}{r}{s}{v}{w}\otimes \Gr{B(\Gamma)}{v}{w}{t}{z}$ which are easily seen to satisfy Definition \ref{DefPartBiAlg}. Moreover, the $\Delta_{vw}$ are coassociative as they are coassociative on generators.

Let now $E_{v,w}$ be the matrix units for $l^2(\Gamma^{(0)})$. Then one verifies again directly from the defining relations of $B(\Gamma)$ that one can define a $^*$-homomorphism $\widetilde{\varepsilon}: B(\Gamma)\rightarrow B(l^2(\Gamma^{(0)}))$ sending $\UnitC{v}{w}$ to $\delta_{v,w}\, e_{v,v}$ and $u_{e,f}$ to $\delta_{e,f}\, e_{s(e),t(e)}$. We can hence define a map $\varepsilon: B(\Gamma)\rightarrow \C$ such that $\widetilde{\varepsilon}(x) = \varepsilon(x) e_{v,w}$ for all $x\in \Gr{B(\Gamma)}{v}{w}{v}{w}$, and which is zero elsewhere. Clearly it defines a morphism on the partial algebra $\mathscr{B}(\Gamma)$. As $\varepsilon$ satisfies the counit condition on generators, it follows by partial multiplicativity that it satisfies the counit condition on the whole of $B(\Gamma)$, i.e.~ $B(\Gamma)$ is a partial $^*$-bialgebra. 

It is clear now that the $u_{e,f}$ define a unitary corepresentation $U$ of $B(\Gamma)$ on $\Hsp^{\Gamma}$. Moreover, from \eqref{EqUni1} and the formula for $u_{e,f}^*$ we can deduce that $R_{\Gamma}: \C_{\Gamma^{(0)}}\rightarrow \Hsp^{\Gamma}\underset{\Gamma^{(0)}}{\otimes}\Hsp^{\Gamma}$ is a morphism from $\C^{(\Gamma^{(0)})}$ to $U\Circtv{\Gamma^{(0)}} U$ in $\Corep_{\rcf,u}(\mathscr{B}(\Gamma))$, cf.~ \eqref{EqMorR}. From the universal property of $\mathcal{T}_q$, it then follows that we have a morphism $G^{\Gamma}: \mathcal{T}_q \rightarrow \Corep_{\rcf,u}(\mathscr{B}(\Gamma))$ with $G^{\Gamma}(X) = U$. On the other hand, as we have a $\Delta$-preserving $^*$-homomorphism $B(\Gamma)\rightarrow A(\Gamma)$ by the universal property of $\mathscr{B}(\Gamma)$, we have a strongly monoidal $^*$-functor $H^{\Gamma}:  \Corep_{\rcf,u}(\mathscr{B}(\Gamma))\rightarrow \Corep_u(\mathscr{A}(\Gamma)) = \mathcal{T}_q$ which is inverse to $G^{\Gamma}$. Since the commutation relations of $\mathscr{A}(\Gamma)$ are completely determined by the morphism spaces of $\mathcal{T}_q$, it follows that we have a $^*$-homomorphism $\mathscr{A}(\Gamma)\rightarrow \mathscr{B}(\Gamma)$ sending $v_{e,f}$ to $u_{e,f}$. This proves the theorem. 
\end{proof}

We remark that for finite graphs with their canonical weights coming from the Perron-Frobenius eigenvalues (\cite[Section 3.1]{DCY2}), these partial compact quantum groups were considered in \cite[Section 6]{Hay1}.


Let us now consider a particular class of `homogeneous' $-(q+q^{-1})$-reciprocal random walks. Namely, assume that there exists a finite set $T$ partitioning $\Gamma^{(1)} = \cup_a \Gamma^{(1)}_a$ such that for each $a\in T$ and $v\in \Gamma^{(0)}$, there exists a unique $e_a(v)\in \Gamma^{(1)}_a$ with source $v$. Write $av$ for the range of $e_a(v)$. Assume moreover that $T$ has an involution $a\mapsto \bar{a}$ such that $\overline{e_a(v)} = e_{\bar{a}}(av)$. Then for each $a$, the map $v\mapsto av$ is a bijection on $\Gamma^{(0)}$ with inverse $v\mapsto \bar{a}v$. In particular, also for each $w\in \Gamma^{(0)}$ there exists a unique $f_w(a) \in \Gamma^{(1)}_a$ with target $w$.

Let us further denote $w_a(v) = w(e_a(v))$ and $\sgn_a(v) = \sgn(e_a(v))$. Let again $A(\Gamma)$ be the total $^*$-algebra of the associated partial compact quantum group. Using Theorem \ref{TheoGenRel}, we have the following presentation of $A(\Gamma)$: it is generated by self-adjoint mutually orthogonal idempotents $\UnitC{v}{w}$ and elements $(u_{a,b})_{v,w} := u_{e_a(v),e_b(v)}$ for $a,b\in T$ and $v,w\in \Gamma^{(0)}$ with defining relations $(u_{a,b})_{v,w}^* = \frac{\sgn_b(w)\sqrt{w_b(w)}}{\sgn_a(v)\sqrt{w_a(v)}}(u_{\bar{a},\bar{b}})_{av,bw}$ and \begin{eqnarray*} \sum_{a\in T} (u_{a,b})_{\bar{a}v,w}^* (u_{a,c})_{\bar{a}v,z}= \delta_{w,z} \delta_{b,c} \UnitC{v}{bw},& \sum_{a\in T} (u_{b,a})_{w,v} (u_{c,a})_{z,v}^* = \delta_{b,c}\delta_{w,z} \UnitC{w}{v}.\end{eqnarray*} The element $(u_{a,b})_{v,w}$ lives inside the component $\Gr{A(\Gamma)}{v}{av}{w}{bw}$.

Let us now consider $M(A(\Gamma))$, the multiplier algebra of $A(\Gamma)$. For a function $f$ on $\Gamma^{(0)}\times \Gamma^{(0)}$, write $f(\lambda,\rho) = \sum_{v,w} f(v,w)\UnitC{v}{w} \in M(A(\Gamma))$. Similarly, for a function $f$ on $\Gamma^{(0)}$ we write $f(\lambda) = \sum_{v,w} f(v)\UnitC{v}{w}$ and $f(\rho) = \sum_{v,w}f(w)\UnitC{v}{w}$. We then write for example $f(a\lambda,\rho)$ for the element corresponding to the function $(v,w)\mapsto f(av,w)$.

We can further form in $M(A(\Gamma))$ the elements $u_{a,b} = \sum_{v,w} (u_{a,b})_{v,w}$. Then $u=(u_{a,b})$ is a unitary m$\times$m matrix for $m=\#T$.  Moreover, \begin{equation}\label{EqAdju}u_{a,b}^* =
  u_{\bar{a},\bar{b}}\frac{\gamma_b(\rho)}{\gamma_a(\lambda)},\end{equation}
where $\gamma_a(v) = \sgn_a(v)\sqrt{w_a(v)}$.   We then have the
following commutation relations between functions on
$\Gamma^{(0)}\times \Gamma^{(0)}$ and the entries of
$u$: \begin{equation}\label{EqGradu} f(\lambda,\rho)u_{a,b} =
  u_{a,b}f(\bar{a}\lambda,\bar{b}\rho),\end{equation} where
$f(\bar{a}\lambda,\bar{b}\rho)$ is given by $(v,w) \mapsto f(\bar{a}v,\bar{b}w)$. Further, $\Delta(u_{a,b}) = \Delta(1) \sum_c(u_{a,c}\otimes u_{c,b})$. Note that the $^*$-algebra generated by the $u_{a,b}$ is no longer a weak Hopf $^*$-algebra when $\Gamma^{(0)}$ is infinite, but rather one can turn it into a Hopf algebroid.

\begin{Rem}
  The weak multiplier Hopf algebra $A(\Gamma)$ is related to the free
  orthogonal dynamical quantum groups introduced in
  \cite{timmermann:free} as follows.  Denote by $G$ the free group
  generated by the elements of $T$ subject to the relation
  $\bar{a}=a^{-1}$ for all $a\in T$. By assumption on $\Gamma$, the
  formula $(af)(v):=f(\bar{a}v)$ defines a left action of $G$ on
  $\Fun(\Gamma^{(0)})$. Denote by $C\subseteq \Fun(\Gamma^{(0)})$ the
  unital subalgebra generated by all $\gamma_{a}$ and their inverses
  and translates under $G$,  write the
  elements of $T \subseteq G$ as a tuple in the form
  $\nabla=(a_{1},\bar{a_{1}},\ldots,a_{n},\bar{a_{n}})$, and define a
  $\nabla\times\nabla$ matrix $F$ with values in $C$ by $F_{a,b} :=
  \delta_{b,\bar{a}} \gamma_{a}$.  Then the free orthogonal dynamical
  quantum group $A_{\mathrm{o}}^{C}(\nabla,F,F)$ introduced in
  \cite{timmermann:free} is the universal unital $*$-algebra generated
  by a copy of $C\otimes C$ and the entries of a unitary $\nabla\times\nabla$-matrix
  $v=(v_{a,b})$ satisfying
  \begin{align*}
    v_{a,b}(f \otimes g) &= (af\otimes bg) v_{a,b}, &
    (aF_{a,\bar{a}}\otimes 1)v_{\bar{a},\bar{b}}^{*} &=
    v_{a,b}(1\otimes F_{b,\bar{b}})
  \end{align*}
  for all $f,g\in C$ and $a,b\in \nabla$. The second equation
  can be rewritten as
  $v_{\bar{a},\bar{b}}^{*}=v_{a,b}(\gamma_{a}^{-1} \otimes
  \gamma_{b})$.   Comparing with
  \eqref{EqAdju} and \eqref{EqGradu}, we see that there exists a
  $*$-homomorphism
  \begin{align*}
  A^{C}_{\mathrm{o}}(\nabla,F,F)  \to
  M(A(\Gamma)), \quad
  \begin{cases}
    f\otimes g&
\mapsto f(\lambda)g(\rho), \\
    v_{a,b} &\mapsto u_{\bar{a},\bar{b}}.
  \end{cases}
\end{align*}
 The two quantum groupoids are related by an
analogue of the unital base changes considered for dynamical quantum
groups in \cite[Proposition 2.1.12]{timmermann:free}. Indeed, Theorem
\ref{TheoGenRel} shows that $A(\Gamma)$ is the image of
$A^{C}_{\mathrm{o}}(\nabla,F,F)$ under a non-unital base change from
$C$ to $\Fun_{f}(\Gamma^{(0)})$ along the natural map $C \to
M(\Fun_{f}(\Gamma^{(0)}))$.

\end{Rem}


As a particular example, take $0<|q|<1$ and $x>0$, and consider the graph with $\Gamma_x^{(0)} = x|q|^{\Z}$ and $\Gamma_x^{(1)} = \{(y,z)\mid y/z \in \{|q|,|q|^{-1}\}\}$. Endow $\Gamma$ with the involution $(y,z) \mapsto (z,y)$, the weight $w(y,z) = \frac{z+z^{-1}}{y+y^{-1}}$ and the sign $\sgn(y,z) = \sigma_{\mu}$ if $y/z = |q|^{\mu}$, where $\sigma_{+} = 1$ and $\sigma_- = -\sgn(q)$.  Consider further the set $T = \{+,-\}$ with the non-trivial involution, and label the edges $(y,z)$ with $\mu$ if $y/z = |q|^{\mu}$. Write $F(y) = |q|^{-1}\frac{|q|y+|q|^{-1}y^{-1}}{y+y^{-1}},$ and put $\alpha = \frac{F^{1/2}(\rho-1)}{F^{1/2}(\lambda-1)}u_{--}$ and $\beta = \frac{1}{F^{1/2}(\lambda-1)}u_{-+}.$ Then our relations for the $u_{\epsilon,\nu}$ are equivalent to the commutation relations \begin{equation}\label{EqqCom} \alpha \beta = qF(\rho-1)\beta\alpha \qquad \alpha\beta^* = qF(\lambda)\beta^*\alpha\end{equation} \begin{equation}\label{EqDet} \alpha\alpha^* +F(\lambda)\beta^*\beta = 1,\qquad \alpha^*\alpha+q^{-2}F(\rho-1)^{-1}\beta^*\beta = 1,\end{equation}\begin{equation*} F(\rho-1)^{-1}\alpha\alpha^* +\beta\beta^* = F(\lambda-1)^{-1},\qquad  F(\lambda)\alpha^*\alpha +q^{-2}\beta\beta^* = F(\rho),\end{equation*} \begin{equation}\label{EqGrad} f(\lambda)g(\rho)\alpha =
\alpha f(\lambda+1)g(\rho+1),\qquad f(\lambda)g(\rho)\beta = \beta f(\lambda+1)g(\rho-1).\end{equation}

These are precisely the commutation relations for the dynamical quantum $SU(2)$-group as in \cite[Definition 2.6]{KoR1}, except that the precise value of $F$ has been changed by a shift in the parameter domain. The (total) coproduct on $A_x$ also agrees with the one on the dynamical quantum $SU(2)$-group. Note also that the case of $q$ a root of unity case was considered in \cite[Section 5]{Hay7}, see also \cite{Hay4, EtN1} for generalizations to higher rank (in resp.~ the unitary and non-unitary case).



\bibliographystyle{habbrv}
%\bibliographystyle{hplain}
%\bibliographystyle{kp}
\bibliography{references}

\end{document}