\section{C$^*$-partial compact quantum groups and their representations}


\subsection{Definition}

For $A$ a C$^*$-algebra, we denote by $M(A)$ the multiplier C$^*$-algebra of $A$. All tensor products of C$^*$-algebras in this paper will be minimal. We denote by $[\,\cdot\,]$ the closed linear span of a subset.

\begin{Def}\label{DefCpcqg} Let $I$ be a set. We call \emph{C$^*$-algebraic $I$-partial compact quantum group}, or \emph{C$^*$-pcqg} (over $I$) for short, a triple consisting of 
\begin{itemize}
\item a (not necessarily unital) C$^*$-algebra $A$,
\item a family of orthogonal self-adjoint projections $\UnitC{k}{l}\in A$ for $k,l\in I$, some of which are possibly zero, and
\item  a (not necessarily unital) $^*$-homomorphism \[\Delta: A\rightarrow M(A\otimes A),\] 
\end{itemize}
satisfying the following conditions:
\begin{enumerate}[(a)]
\item[Ui)] $\UnitC{k}{k}\neq 0$ for all $k\in I$. 
\item[Uii)] $\sum_{k,l} \UnitC{k}{l}$ strictly converges to the unit in $M(A)$.
\item[Uiii)] $\Delta(\UnitC{k}{l}) = \sum_{m}\UnitC{k}{m}\otimes \UnitC{m}{l}$ strictly for all $k,l$. 
\item[Di)] With $\Delta(1) = \sum_{k,l,m} \UnitC{k}{m}\otimes \UnitC{m}{l}$, we have \begin{equation}\label{CondDi}(A\otimes A)\Delta(1) = [(A\otimes 1)\Delta(A)] = [(1\otimes A)\Delta(A)].\end{equation} 
\item[Dii)] With $P=\sum_{k} \UnitC{k}{k}$, and $A_P = PAP$, we have \[[(\omega\otimes \id)\Delta(A_P)\mid \omega \in A^*] = [(\id\otimes \omega)\Delta(A_P)\mid \omega \in A^*] = A.\]
\item[C)] $\Delta$ is coassociative: for all $a,b,c\in A$, we have \[(a\otimes 1\otimes 1)(\Delta\otimes \id)(\Delta(b)(1\otimes c)) = (\id\otimes \Delta)((a\otimes 1)\Delta(b))(1\otimes 1\otimes c).\] 
\end{enumerate}
\end{Def} 

Note that $\Delta(1)$ is a well-defined projection in $M(A\otimes A)$. By condition \eqref{CondDi}, $\Delta$ extends uniquely to a $^*$-homomorphism \[\Delta: M(A)\rightarrow M(A\otimes A)\] with value in the unit precisely $\Delta(1)$. In the same way, $(\id\otimes \Delta)$ and $(\Delta\otimes \id)$ extend to $M(A\otimes A)$, and we can write the coassociativity condition in the usual form \[(\Delta\otimes \id)\Delta = (\id\otimes \Delta)\Delta,\] valid also on $M(A)$.

In the following, we write \[\Gr{A}{k}{l}{m}{n} = \UnitC{k}{m}A\UnitC{l}{n},\] each of which is a Banach subspace of $A$. In particular, each corner $\Gr{A}{k}{k}{m}{m}$ is a unital C$^*$-algebra with unit $\UnitC{k}{m}$. We will write also \[\lambda_k = \sum_{m}\UnitC{k}{m},\qquad \rho_m = \sum_{k}\UnitC{k}{m},\] which give well-defined projections in $M(A)$. Finally, we will write, for $a\in A$, \[\Delta_{rs}(a) = (\rho_r\otimes 1)\Delta(a)(\rho_s\otimes 1) = (1\otimes \lambda_r)\Delta(a)(1\otimes \lambda_s),\] which are well-defined elements in $A\otimes A$ by Definition \ref{DefCpcqg}.Uii). We then have that  \[\Delta(a) = \sum_{r,s} \Delta_{rs}(a)\] in the strict topology. 

\begin{Lem}
Let $I$ be a set and let $(A,\Delta)$   be a $C^{*}$-algebraic $I$-partial compact quantum group. Then the relation $\sim$ on $I$ given by $k\sim l :\Leftrightarrow \UnitC{k}{l}\neq 0$ is an equivalence relation.
\end{Lem}
\begin{proof}
  The relation is reflexive by Ui) and transitive by Uiii). Finally, if $\UnitC{k}{l}\neq 0$, then by Dii) it must be contained in $[(\omega \otimes \id)(\Delta(\Gr{A}{l}{l}{l}{l})) | \omega \in A^{*}]$, whence $\Gr{A}{l}{l}{k}{k}\neq 0$ and $\UnitC{l}{k} \neq 0$.
\end{proof}
\begin{Def}
  The \emph{hyperobject set} of a   $C^{*}$-algebraic $I$-partial compact quantum group $(A,\Delta)$ is the set $I/\sim$.
\end{Def}

\subsection{Invariant integrals}
\begin{Def} Let $(A,\Delta)$ be a C$^*$-pcqg. An \emph{invariant integral} on $A$ consists of a weight $\phi: A^+ \rightarrow [0,+\infty]$ satisfying the following conditions:
\begin{enumerate}[i)]
\item For all $k,m$ with $\UnitC{k}{m}\neq 0$, \[\phi(\UnitC{k}{m}) = 1.\]
\item For all $a\in A^+$, \[\phi(a) = \sum_{k,m} \phi\left(\UnitC{k}{m}a\UnitC{k}{m}\right).\]
\item For all $a\in A^+$ and all states $\omega\in A^*$, \begin{equation}\label{EqInvL} \phi((\omega \otimes \id)\Delta(a)) = \sum_{k} \omega(\lambda_k)\phi(\lambda_ka\lambda_k),\end{equation} \begin{equation}\label{EqInvR} \phi((\id\otimes \omega)\Delta(a)) = \sum_{m}\omega(\rho_m)\phi(\rho_ma\rho_m).\end{equation}
\end{enumerate} 
\end{Def}


%family of states \[\phi_{km}: \Gr{A}{k}{k}{m}{m}\rightarrow \C\] for each $k,m$ with $\UnitC{k}{m}\neq 0$, so that for each $a\in \Gr{A}{k}{l}{m}{n}$ and each $r$, \[(\id\otimes \phi_{rm})\Delta_{rr}(a) = \phi_{km}(a)\UnitC{k}{r},\qquad (\phi_{kr}\otimes \id)\Delta_{rr}(a) = \phi_{kr}(a)\UnitC{r}{m}.\]
%\end{Def} 

%State zero functional if algebra zero

%Here we interpret $\phi_{rs}(a) =0$ for $a\notin \Gr{A}{r}{r}{s}{s}$. 

Clearly, the formula \[\phi_{km}(a) = \phi(\UnitC{k}{m}a\UnitC{k}{m})\] defines a bounded weight $\phi_{km}$ on $A$, which hence can be seen as a positive functional on $A$. If $\UnitC{k}{m}\neq 0$ it is a state, otherwise it is the zero functional. By abuse of language, we will in the following refer to the complete family of $\phi_{km}$ as `states', so the reader should bear in mind that some of them can be zero functionals.

It is also clear that $\phi$ is completely determined by the family
$\{\phi_{km}\}$.  

In terms of the $\phi_{km}$, the left and right invariance properties
\eqref{EqInvL} and \eqref{EqInvR} take the following form.  For all
$a\in A$,
\begin{align} \label{eq:invariance}
\sum_{k}  \lambda_{k} \phi_{km}(a) &= \sum_{k}  (\id \otimes
\phi_{km})(\Delta(a)), &
\sum_{m} \rho_{m} \phi_{km}(a) &= \sum_{m} (\phi_{km} \otimes \id)(\Delta(a)),
\end{align}
where the sums converge strictly.

These relations can also  be rewritten in terms of the associative
convolution product on $A^*$  defined by \[(\chi*\omega)(a) =
(\chi\otimes \omega)\Delta(a).\] Let us write \begin{equation}\label{DefSpB} \Gr{B}{k}{m}{l}{n} = \{\omega \in A^* \mid \forall a\in A, \omega(a) = \omega\left(\UnitC{k}{m}a\UnitC{l}{n}\right)\}.\end{equation}Then the convolution product restricts to products \[\Gr{B}{k}{m}{l}{n}\times \Gr{B}{m}{p}{n}{q}\rightarrow \Gr{B}{k}{p}{l}{q},\] all other products being zero. The left and right invariance properties \eqref{EqInvL} and \eqref{EqInvR} can then be written in terms of the $\phi_{km}$ as \begin{equation}\label{EqInvLp}\omega*\phi_{km} = \omega(\UnitC{p}{k})\phi_{pm},\qquad \forall \omega \in \Gr{B}{p}{k}{q}{k},\end{equation}
\begin{equation}\label{EqInvRp}\phi_{km}*\omega = \omega(\UnitC{m}{q})\phi_{kq},\qquad \forall \omega \in \Gr{B}{m}{p}{m}{q}.\end{equation}


We will refer to families of states satisfying \eqref{EqInvLp} as a \emph{left invariant integral}, and to those satisfying \eqref{EqInvRp} as \emph{right invariant integral}.

\begin{Theorem}\label{TheoInvInt} Each $C^*$-pcqg admits a unique invariant integral.
\end{Theorem} 

We will split the proof of the Theorem into several steps, setting the stage so that eventually the arguments of \cite{MVD1} can be applied almost verbatim. 

\begin{Lem} Let $\{\phi_{km}\}$ be a a left invariant integral, and $\{\psi_{km}\}$ a right invariant integral. Then $\phi_{km}= \psi_{km}$ for all $k,m$. 
\end{Lem} 
\begin{proof} By the invariance properties, and the fact that $\UnitC{k}{k}\neq 0$, we have \[\phi_{km}  = \psi_{kk}\left(\UnitC{k}{k}\right)\phi_{km} = \psi_{kk}*\phi_{km}= \phi_{km}\left(\UnitC{k}{m}\right)\psi_{km} = \psi_{km}.\]

\end{proof} 

By the previous Lemma, the unicity in Theorem \ref{TheoInvInt} already follows. It implies as well that it is sufficient to find an invariant left integral for $(A,\Delta)$.

The following lemma will be crucial.

\begin{Lem}\label{LemRefSep} Let $\omega \in \Gr{B}{k}{m}{l}{n}$, and assume $\chi*\omega =  0$ for all $\chi \in \Gr{B}{m}{k}{n}{l}$. Then $\omega =0$.
\end{Lem} 
\begin{proof} By assumption, we have for all $\chi\in A^*$ that \[(\chi\otimes \omega)((\UnitC{m}{k}\otimes \UnitC{k}{m})\Delta(A)(\UnitC{n}{l}\otimes \UnitC{l}{n}))=0.\] But since $\omega = \omega(\UnitC{k}{m}\,\cdot\,\UnitC{l}{n})$, this means, using the notation from Definition \ref{DefCpcqg}.Dii), \[\omega((\chi\otimes \id)(\Delta(A_P))) =(\chi\otimes \omega)((\sum_{k'}\UnitC{m}{k'}\otimes \UnitC{k'}{m})\Delta(A)(\sum_{l'}\UnitC{n}{l'}\otimes \UnitC{l'}{n})) =0.\] By Definition \ref{DefCpcqg}.Dii), we conclude $\omega=0$.
\end{proof} 

\begin{Cor} If $\UnitC{k}{m}=0$, then $\UnitC{m}{k}=0$. 
\end{Cor}
\begin{proof} If $\UnitC{k}{l}=0$, this implies $\Gr{B}{k}{l}{k}{l}=0$. By the previous lemma, this forces also $\Gr{B}{l}{k}{l}{k}=0$, and so $\UnitC{l}{k}=0$. 
\end{proof} 

\begin{Lem} Assume that there exists a family of states $\{\phi_{kk}\}$ in $\Gr{B}{k}{k}{k}{k}$ such that, for any $\omega \in \Gr{B}{k}{k}{k}{k}$, one has \[\omega*\phi_{kk} = \omega\left(\UnitC{k}{k}\right)\phi_{kk}.\] Then $(A,\Delta)$ admits a left invariant state.
\end{Lem}
\begin{proof} Let $\theta_{rm}$ be an arbitary collection of states in $\Gr{B}{r}{m}{r}{m}$ (whenever this algebra is not zero), and write \[\phi_{rm} = \theta_{rm}*\phi_{mm}.\] By assumption, this notation is consistent in the case $r=m$. 

Assume now that $\omega \in \Gr{B}{k}{r}{l}{r}$ and $\chi \in \Gr{B}{m}{k}{m}{l}$. Assume first that $\UnitC{r}{m}\neq 0$ and $\UnitC{k}{m}\neq 0$. Then \begin{eqnarray*} \chi*(\omega*\phi_{rm}) &=& (\chi*\omega*\theta_{rm})*\phi_{mm} \\ &=&  (\chi*\omega*\theta_{rm})(\UnitC{m}{m}) \phi_{mm} \\ &=&  \chi(\UnitC{m}{k})\omega(\UnitC{k}{r})\phi_{mm} \\ &=&  \omega(\UnitC{k}{r}) \; (\chi*\theta_{km})\left(\UnitC{m}{m}\right)\phi_{mm}
\\ &=& \omega(\UnitC{k}{r}) \; (\chi*\theta_{km})*\phi_{mm} \\ &=&  \omega(\UnitC{k}{r}) \; \chi*\phi_{km} .\end{eqnarray*} As $\chi$ was arbitrary, we find by Lemma \ref{LemRefSep} that \begin{equation}\label{EqInvL2} \omega*\phi_{rm} =  \omega(\UnitC{k}{r}) \phi_{km}.\end{equation}

Assume now that $\UnitC{r}{m}=0$. Then also $\Delta_{kk}(\UnitC{r}{m}) = \UnitC{r}{k}\otimes \UnitC{k}{m}=0$, and hence $\UnitC{r}{k}=0$ or $\UnitC{k}{m}=0$. By the previous lemma, we obtain either $\UnitC{k}{r}=0$ or $\UnitC{k}{m}=0$. In either case, both sides of \eqref{EqInvL2} are zero. 

Similarly, if $\UnitC{k}{m}=0$, we conclude that either $\UnitC{k}{r}=0$ or $\UnitC{r}{m}=0$, and again both sides of \eqref{EqInvL2} are zero.

This shows that \eqref{EqInvL2} holds for all indices, and hence $\{\phi_{km}\}$ is a left invariant integral. 
\end{proof} 

Hence Theorem \ref{TheoInvInt} will be proven once we can produce a family of invariant states $\phi_{kk}$ as in the previous lemma. For this, one can follow the proof as in \cite{MVD1} for the existence of an invariant state for a compact quantum group.

\begin{Prop} For each $k\in I$, there exists a state $\phi_{kk}$ in $\Gr{B}{k}{k}{k}{k}$ such that, for any state $\omega \in \Gr{B}{k}{k}{k}{k}$, one has \[\omega*\phi_{kk} =\phi_{kk}.\]
\end{Prop} 
\begin{proof} Let $k\in I$, and $\omega$ a state in $\Gr{B}{k}{k}{k}{k}$. By \cite[Lemma 4.2]{MVD1},
  there exists a state $h_{kk} \in \Gr{B}{k}{k}{k}{k}$ with \[\omega *h_{kk}= h_{kk} =
  h_{kk}*\omega.\]

  Assume now that $\rho\in \Gr{B}{k}{k}{k}{k}$ and $0\leq \rho\leq \omega$. Take $a\in A$. Then the
  beginning of the proof of \cite[Lemma 4.3]{MVD1}, applied with $b= (\id\otimes h_{kk})\Delta(a)$,
  shows that, for all $c\in A$, 
  \begin{equation}\label{EqAuxId} (h_{kk}\otimes
    (\rho*h_{kk}))((c\otimes 1)\Delta(a)) = \rho(1) (h_{kk}\otimes h_{kk})((c\otimes
    1)\Delta(a)).\end{equation} 
Since $(A\otimes A)\Delta(1) = [(A\otimes 1)\Delta(A)]$, we may
  replace $(c\otimes 1)\Delta(a)$ with $\UnitC{k}{k}\otimes a$ for $a\in \Gr{A}{k}{k}{k}{k}$. Then
  \eqref{EqAuxId} becomes $(\rho*h_{kk})(a) = \rho(1)h_{kk}(a)$. Hence \[\rho*h_{kk} =
  \rho(1)h_{kk}.\]

A compactness argument as in \cite[Theorem 4.4]{MVD1} lets us conclude that there exists a state $\phi_{kk}$ as in the statement of the proposition.
\end{proof}


  Assume now that $(A,\Delta)$ is the C$^*$-algebra with comultiplication arising from a $^*$-algebraic pcqg with invariant integral $\phi$. It is easy to see that the conditions in Definition \ref{DefCpcqg} are satisfied. To see that $[(\omega\otimes \id)\Delta(pAp)\mid \omega \in A^*] = A$ for example, for $p = \sum_{k} \UnitC{k}{k}$, note that for the matrix coefficient $(\Gr{X}{k}{l}{m}{n})_{pj}$ of a unitary corepresentation, we have \[ (\Gr{X}{k}{l}{m}{n})_{pj} \in \sum_{q,r,s} \phi((\Gr{X}{m}{n}{r}{s})_{jq}A) (\Gr{X}{r}{s}{m}{n})_{qj} =(\phi(\,\cdot\, A)\otimes \id)\Delta((\Gr{X}{m}{n}{m}{n})_{jj})\] by the orthogonality relations. 


%Also note that for the $(A,\Delta)$ arising from $^*$-pcqg with invariant integral, the above conditions are all satisfied - the fourth condition can be checked using the orthogonality relations for matrix coefficients of unitary corepresentations. 


\subsection{The $C^{*}$-tensor category of corepresentations}

Let $I$ be a set. Given an $I^{2}$-graded Hilbert space
$H=\bigoplus_{k,l} \Grd{H}{k}{l}$, we denote by
$p_{kl}^{H} \in \mathcal{B}(H)$ the projections onto the homogeneous components and let
\begin{align*}
  \lambda^{H}_{k} &= \sum_{l} p_{kl}^{H}, &
  \rho^{H}_{l} &= \sum_{k} p_{kl}^{H},
\end{align*}
 the sums converging in the strong operator topology.


 \begin{Def} \label{def:corepresentation} Let $(A,\Delta)$ be a
   $C^{*}$-algebraic $I$-partial compact quantum group. 

   A \emph{unitary corepresentation} of $(A,\Delta)$ on a Hilbert
   space $H$ is a partial isometry $X \in M(A \otimes \mathcal{K}(H))$
   satisfying the following conditions:
   \begin{enumerate}
   \item with $\Delta \otimes \id$ extended to the multiplier algebra, we have
     \begin{align} \label{eq:corep}
     (\Delta \otimes \id)(X) = X_{13}X_{23},  
   \end{align}
 \item there exists an $I^{2}$-grading on $H$ such that
     \begin{align} \label{eq:corep-pi}
       X^{*}X= \sum_{n}\rho_{n} \otimes \rho^{H}_{n} \quad \text{and}
       \quad XX^{*} = \sum_{k} \lambda_{k} \otimes \lambda^{H}_{k}.
     \end{align}
   \end{enumerate}
   We call such a unitary corepresentation \emph{row- and column-finite-dimensional}, briefly
   \emph{rcfd}, if $\lambda^{H}_{k}$ and $\rho^{H}_{m}$ have finite rank for all $k,m\in I$.
\end{Def}
Note that the $I^{2}$-grading on $H$ in  condition 2.\ is uniquely
determined by $X$.

\begin{Exa} \label{exa:corep-trivial}
  Denote by $(e_{km})_{k,m\in I}$ the matrix units in $\mathcal{B}(l^{2}(I))$. Then the sum
  \begin{align*}
    E = \sum_{k,m} \UnitC{k}{m} \otimes e_{km} 
  \end{align*}
  converges strictly in $M(A\otimes \mathcal{K}(l^{2}(I)))$ to a unitary rcfd corepresentation, as
  one can easily check. The associated $I^{2}$-grading on $l^{2}(I)$ is the diagonal grading, that
  is, $p^{l^{2}(I)}_{km} =\delta_{k,m} e_{kk}$. We call $E$ the \emph{trivial corepresentation} of
  $(A,\Delta)$. 
If we restrict the sum above to $k,m \in \alpha$ for some hyperobject $\alpha \in I/\sim$, we obtain a unitary rcfd corepresentation $E_{\alpha}$ on $l^{2}(\alpha)$.
\end{Exa}

\begin{Lem} \label{lem:corep-intertwine}
 Let $X$ be a unitary corepresentation of $(A,\Delta)$ on a Hilbert space $H$. Then  
    \begin{align*}
      (\UnitC{k}{m} \otimes 1)X(\UnitC{l}{n} \otimes 1) = (1 \otimes p^{H}_{kl})X(1 \otimes
      p^{H}_{mn})
    \end{align*}
   for all $k,l,m,n \in I$.
\end{Lem}
\begin{proof}
 Conditions 1.\ and 2.\ in the definition above imply
  \begin{align*}
 \sum_{l,n} \rho_{l} \otimes
  \UnitC{l}{n} \otimes \rho_{n}^{H} &=
    \sum_{n} \Delta(\rho_{n}) \otimes \rho_{n}^{H} \\ &= (\Delta \otimes
    \id)(X^{*}X) \\ &= X^{*}_{23}X^{*}_{13}X_{13}X_{23}  =
    \sum_{l}\rho_{l} \otimes X^{*}(1 \otimes \rho_{l}^{H})X.
  \end{align*}
  We multiply on the left by $X$, use  condition 2.\, and deduce
  $X(\lambda_{l}\otimes 1)=(1\otimes \rho_{l}^{H})X$ for all $l$.
  This relation and \eqref{eq:corep} imply
 \begin{align*}
   X_{13}(1\otimes 1 \otimes \rho_{l}^{H})X_{23} &=X_{13}X_{23}(1
   \otimes \lambda_{l} \otimes 1) \\ &= X_{13}X_{23}(\rho_{l} \otimes 1
   \otimes 1) = X_{13} (\rho_{l} \otimes 1
   \otimes 1) X_{23},
 \end{align*}
 and multiplying by $X_{23}^{*}$ on the right we conclude $X(1\otimes
 \rho^{H}_{l})=X(\rho_{l} \otimes 1)$.


  Similarly,  the relation
  \begin{align*}
    \sum_{k,m} \UnitC{k}{m} \otimes \lambda_{m} \otimes
    \lambda_{m}^{H} &= (\Delta \otimes \id)(XX^{*}) = \sum_{m}
    X_{13}(1 \otimes \lambda_{m} \otimes \lambda_{m}^{H})X_{13}^{*}
  \end{align*}
  implies $(\rho_{m} \otimes 1)X=X(1\otimes \lambda^{H}_{m})$ for all
  $m$, and then
  \begin{align*}
    X_{13}(1 \otimes 1 \otimes \lambda^{H}_{m})X_{23} &= (
    \rho_{m} \otimes 1 \otimes 1)X_{13}X_{23} \\ &= (1\otimes
    \lambda_{m}\otimes 1)X_{13}X_{23} = X_{13}(1 \otimes \lambda_{m}
    \otimes 1)X_{23}
  \end{align*}
  implies, after multiplying by $X_{13}^{*}$ on the left, $(1 \otimes
  \lambda_{m}^{H})X=(\lambda_{m} \otimes 1)X$.
\end{proof}

\begin{Def} \label{def:intertwiner}
  Let $(A,\Delta)$ be a $C^{*}$-pcqg with unitary corepresentations $X$ and $Y$ on Hilbert spaces
  $H$ and $K$, respectively.  Then an \emph{intertwiner} from $X$ to $Y$ is an operator $T\in
  \mathcal{B}(H,K)$ satisfying   $Y(1\otimes T)=(1 \otimes T)X$.
\end{Def}
The unitary corepresentations of $(A,\Delta)$ with intertwiners as morphisms form a category.  We
denote by $\Corep(A,\Delta)$ this category, and by $\Corepf(A,\Delta)$ the full subcategory formed
by all unitary rcfd corepresentations. The following result shows that for every intertwiner $T$,
the adjoint $T^{*}$ is an intertwiner as well.
\begin{Lem} \label{lem:def-intertwiner}
  Let $(A,\Delta)$ be a $C^{*}$-pcqg with unitary corepresentations $X$ and $Y$ on Hilbert spaces
  $H$ and $K$, respectively, and let $T\in \mathcal{B}(H,K)$. Then the following conditions are
  equivalent:
  \begin{enumerate}
  \item $T$ intertwines $X$ and $Y$, that is, $Y(1\otimes T)=(1 \otimes T)X$;
  \item $Y^{*}(1\otimes T)X=\sum_{l} \rho_{l} \otimes \rho^{K}_{l}T$ and $\lambda_{m}^{K}T=T\lambda_{m}^{H}$ for all $l$;
  \item $Y(1\otimes T)X^{*} = \sum_{k} \lambda_{k} \otimes \lambda^{K}_{k}T$ and
    $\rho_{n}^{K}T=\rho_{n}^{H}$ for all $k$.
  \end{enumerate}
  In particular,  these conditions imply that $Tp_{mn}^{H}=p_{mn}^{K}T$ for all $m,n$ and that
  $T^{*}$ intertwines $Y$ and $X$.
\end{Lem}
\begin{proof}
  Assume 1.\ holds. Then $Y^{*}(1\otimes T)X=\sum_{l} \rho_{l} \otimes \rho^{K}_{l}T$ by condition
  2.\ in Definition \ref{def:corepresentation}, and
  \begin{align*}
    Y(1\otimes T\lambda^{H}_{m}) = (\rho_{m} \otimes T)X = (\rho_{m} \otimes 1)Y(1\otimes T) =
    Y(1\otimes \lambda^{K}_{m}T)
  \end{align*}
which implies $T\lambda^{H}_{m} = \lambda^{K}_{m}T$ because of  condition
  2.\ in Definition \ref{def:corepresentation} again. 
Conversely, assume 2.\ holds. Then the first equation implies $YY^{*}(1\otimes T)X = Y(1\otimes T)$,
and the second implies $YY^{*}(1\otimes T)X=(1\otimes T)X$.

 A similar argument shows that conditions 1.\ and 3.\ are equivalent.
\end{proof}

The following implication will be used in the next subsection.
\begin{Prop} \label{prop:restrict-morphisms}
  Let $(A,\Delta)$ be a $C^{*}$-pcqg and let $T$ be a morphism of unitary corepresentations $X$ and $Y$. Suppose that $l_{\beta} \in \beta$ for each hyperobject $\beta$ and that $\Grd{T}{k}{l_{\beta}}=0$ for all $k\in I$ and all $\beta\in I/\sim$. Then $T=0$.
\end{Prop} 
\begin{proof}
  By Lemma \ref{lem:corep-intertwine} and \ref{lem:def-intertwiner}, 
  \begin{align*}
    \UnitC{l}{n} \otimes \Grd{T}{m}{n} = \sum_{k}
    (\Gr{X}{k}{l}{m}{n})^{*}(1\otimes \Grd{T}{k}{l})\Gr{Y}{k}{l}{m}{n}.
  \end{align*}
For $l=l_{\beta}$   and $n\in \beta$, the right hand side vanishes but $\UnitC{l}{n}\neq 0$, whence $\Grd{T}{m}{n}=0$.
\end{proof}

Given a family of unitary corepresentations $(H^{(x)},X_{x})_{x \in \mathcal{I}}$, we define a direct sum
\begin{align*}
  \bigoplus_{x\in \mathcal{I}} X_{x} \in M(A \otimes \mathcal{K}(H)), \quad \text{where } H = \bigoplus_{x} H^{(x)},
\end{align*}
in a straightforward way, and this construction evidently is functorial. 

\begin{Exa} \label{exa:decompose-trivial}
 The isomorphism
  \begin{align*}
    l^{2}(I) \cong \bigoplus_{\alpha \in I/\sim} l^{2}(\alpha)
  \end{align*}
  identifies the trivial corepresentation $E$ on $l^{2}(I)$ with the
  direct sum of the family of corepresentations
  $(l^{2}(\alpha),E_{\alpha})_{\alpha \in I/\sim}$ introduced in
  Example \ref{exa:corep-trivial}.

More generally,  for any unitary corepresentation $X$ on a Hilbert space $H$, we can write
\begin{align*}
  H \cong \bigoplus_{\alpha,\beta \in I/\sim} H^{(\alpha,\beta)}, \  \quad \text{where} \quad H^{(\alpha,\beta)} = \bigoplus_{k\in \alpha,\beta\in l} \Grd{H}{k}{l},
\end{align*}
and $X$ restricts to a unitary corepresentation $\Grd{X}{\alpha}{\beta}$ on each $H^{(\alpha,\beta)}$ such that
the isomorphism above identifies $X$ with the direct sum $\bigoplus_{\alpha,\beta} \Grd{X}{\alpha}{\beta}$.
\end{Exa}
\begin{Def}
 The \emph{hyperobject support}
  of a unitary corepresentation $X$ is the set of all hyperobject
  pairs $(\alpha,\beta)$ such that $\Grd{X}{\alpha}{\beta}$ is
  non-zero.
\end{Def}
Given two unitary corepresentations, we define a tensor product as follows.
\begin{Lem}
  Let $X$ and $Y$ be unitary corepresentations of a $C^{*}$-pcqg $(A,\Delta)$ on Hilbert spaces $H$,
  $K$. Then
    \begin{align*}
      X \Circt Y:= X_{12}Y_{13} \in M(A \otimes \mathcal{K}(H \otimes K))
    \end{align*}
    is a unitary corepresentation of $(A,\Delta)$ on the Hilbert space
    \begin{align*}
      H \itimes K := \bigoplus_{k,l,m} \Grd{H}{k}{l} \otimes \Grd{K}{l}{m},
    \end{align*}
    where the  associated $I^{2}$-grading is given by  $k$ and $m$.
\end{Lem}
\begin{proof}
Clearly,
$(\Delta \otimes \id)(X_{12}Y_{13}) = X_{13}X_{23}Y_{14}Y_{24} =
    X_{13}Y_{14}X_{23}Y_{24}$,
and Lemma \ref{lem:corep-intertwine} implies 
\begin{align*} % Changed order X and X^* in second part of first line
  X_{12}Y_{13}Y_{13}^{*}X_{12}^{*} &= \sum_{l} X_{12}(\lambda_{l}
  \otimes 1)X_{12}^* \otimes \lambda_{l}^{K} = \sum_{k,l} \lambda_{k}
  \otimes p_{kl}^{H} \otimes \lambda_{l}^{K}, \\
  Y_{13}^{*}X_{12}^{*}X_{12}Y_{13} &= \sum_{l} Y_{13}^{*}(\rho_{l}
  \otimes \rho_{l}^{H}\otimes 1)Y_{13}  = \sum_{l} \rho_{m} \otimes
  \rho_{l}^{H} \otimes p_{lm}^{K}. 
\end{align*}
The assertions follow. 
\end{proof}
If $S$ and $T$ are intertwiners between unitary corepresentations $X,Y$ and $U,V$, respectively,
then $S\otimes T$ restricts to an intertwiner between $X\Circt U$ and $Y\Circt V$.  

Thus, the category $\Corep(A,\Delta)$ becomes a $C^{*}$-tensor category with the trivial corepresentation $E$ as the unit. 
Indeed, for any unitary corepresentation $X$ on a Hilbert space $H$, the natural isomorphisms
\begin{align*}
  l^{2}(I) \itimes H \cong H  \quad \text{and} \quad H \itimes l^{2}(I)\cong H
\end{align*}
intertwine $E \Circt X$ and $X\Circt E$, respectively, with $X$, as one can easily check, and induce an  isomorphism between  $ E_{\alpha} \Circt X \Circt E_{\beta}$ and the  component  $\Grd{X}{\alpha}{\beta}$  defined in Example \ref{exa:decompose-trivial}.



If $X$ and $Y$ are unitary rcfd corepresentations, then so is $X\Circt Y$, and therefore,  the subcategory
$\Corepf(A,\Delta)$ is a $C^{*}$-tensor category as well.


\subsection{Decomposition into irreducible corepresentations}

We show that  every unitary
corepresentation of a $C^{*}$-partial compact quantum group decomposes into a direct sum of irreducible unitary corepresentations, following
the approach in \cite{MVD1}, and that every irreducible unitary corepresentation is rcfd.

\begin{Def}
Let $X$ be a unitary corepresentation of a $C^{*}$-pcqg $(A,\Delta)$ on a
  Hilbert space $H$. We call a closed subspace $K\subseteq H$
  \emph{invariant} if $K=\bigoplus_{k,l} (K \cap \Grd{H}{k}{l})$ % State that this is completed sum?
  and
  the orthogonal projection $P\in \mathcal{B}(H)$ onto $K$ satisfies
  $(1\otimes P)X(1\otimes P)=X(1\otimes P)$.  We call $X$
  \emph{irreducible} if there exists no non-trivial closed invariant
  subspace $K\subseteq H$.
\end{Def}
As in \cite{MVD1}, the  following result will imply that every invariant closed subspace is complemented:
 \begin{Prop}\label{prop:corep-complemented}
   Let $(A,\Delta)$ be a $C^{*}$-pcqg with invariant integral $h$ and
   let $X$ be a unitary corepresentation of $(A,\Delta)$. Then the
   subspace % Change notation B to C as to not have overlap with notation from first section? DONE
   \begin{align*}
     C &:= [ \{(\phi_{km} \otimes
     \id)(X(a\otimes 1)) : a \in A,\, k,m\in I\}] \subseteq \mathcal{B}(H)
   \end{align*}
   is a non-degenerate $C^{*}$-subalgebra and $X\in M(A\otimes C)$.
 \end{Prop}
  \begin{proof}
We first show that $[C^{*}C]= C$. This relation implies $C=C^{*}$ and
$[CC]=C$ so that $C$ is a $C^{*}$-subalgebra of $\mathcal{B}(H)$. 

 Let $a\in \Gr{A}{k}{l}{m}{n}$, $b\in \Gr{A}{p}{q}{r}{s}$ and
 \begin{align*}
   S&:=(\phi_{ln} \otimes \id)(X(a\otimes 1)), &
   T&:=(\phi_{qs} \otimes \id)(X(b\otimes 1)).
 \end{align*}
Then  $S \in \mathcal{B}(\Grd{H}{n}{m},\Grd{H}{l}{k})$, $T\in
\mathcal{B}(\Grd{H}{s}{r},\Grd{H}{q}{p})$ and
\begin{align*}
  S^{*}T = \delta_{q,l}\delta_{p,k} (\phi_{ln} \otimes \id)((a^{*}\otimes
  1)X^{*}(1\otimes T)).
\end{align*}
Assume that $(p,q)=(k,l)$.  Then \eqref{eq:invariance} implies
\begin{align*}
 X^{*}(\rho_{n}\otimes T) &=  X^{*}(\UnitC{l}{n} \otimes (\phi_{ls} \otimes
  \id)(X(b\otimes 1))) \\
  &=  X^{*} \sum_{n'}(\id \otimes \phi_{n's}\otimes
  \id)((\rho_n\otimes 1\otimes 1)(\Delta\otimes \id)(X(b\otimes 1))(\lambda_l\otimes 1 \otimes 1))\\
  &=  X^{*}(\id \otimes \phi_{ns}\otimes
  \id)((\Delta\otimes \id)(X(b\otimes 1))) \\
  &=    (\id \otimes \phi_{ns}\otimes
  \id)(X^{*}_{13}X_{13}X_{23}(\Delta(b)\otimes 1)) \\
  &= \sum_{t}(\id \otimes \phi_{ns} \otimes
  \id)((\rho_{t} \otimes 1\otimes \rho^{H}_{t})X_{23}(\Delta(b)\otimes
  1)) \\
&= \sum_{t}(\id \otimes \phi_{ns} \otimes
  \id)(X_{23}((\rho_{t}\otimes \lambda_{t})\Delta(b)\otimes
  1)), 
\end{align*}
that is,
\begin{align}\label{eq:corep-complemented}
  X^{*}(\rho_{n}\otimes T) &=
 (\id \otimes \phi_{ns} \otimes
  \id)(X_{23}(\Delta(b)\otimes 1)).
\end{align}
Thus,
\begin{align*}
  S^{*}T &= (\phi_{ln} \otimes \id)((a^{*}\otimes
  1)X^{*}(1\otimes T))  \\ &= (\phi_{ln} \otimes \phi_{ns} \otimes
  \id)((a^{*} \otimes 1\otimes 1)X_{23}(\Delta(b) \otimes 1)) 
  =(\phi_{ns} \otimes \id)(X(c\otimes 1)), 
\end{align*}
where $c=(\phi_{ln} \otimes \id)((a^{*}\otimes 1)\Delta(b))$. Therefore,
$S^{*}T \in C$. Since $[(A\otimes 1)\Delta(A)]=[(A\otimes
A)\Delta(A)]$, we can conclude that $[C^{*}C]=C$.


To see that $C$ is non-degenerate, observe that $X(A\otimes
\mathcal{K}(H))$ contains $\lambda_{k}A \otimes
\lambda_{k}^{H}\mathcal{K}(H)$ for all $k$ and hence
$[C\mathcal{K}(H)]=\mathcal{K}(H)$.

We finally show that $X \in M(A\otimes C)$.  Let $a,b$ as above but
allow $(p,q)\neq (k,l)$ again. 
Then
equation \eqref{eq:corep-complemented}  implies 
\begin{align*}
  X^{*}(a\otimes T) = X^{*}(\rho_{m}a\otimes T) = (\id \otimes \phi_{ms}
  \otimes \id)(X_{23}(\Delta(b)(a\otimes 1))\otimes 1).
\end{align*}
Since $\Delta(b)(a\otimes 1) \subseteq A\otimes A$, the right hand
side belongs to $A\otimes C$. Thus, $X^{*}(A\otimes C) \subseteq
A\otimes C$. On the other hand, 
\begin{align*}
  X(a\otimes T) &= (\id \otimes \phi_{qs} \otimes
  \id)(X_{13}X_{23}(a\otimes b\otimes 1)) \\ &= (\id \otimes \phi_{qs}
  \otimes \id)((\Delta \otimes \id)(X)(a\otimes b\otimes 1)) 
\end{align*}
Since $\Delta(1)(A \otimes A) = [\Delta(A)(A \otimes 1)]$, we can
approximate $(\Delta \otimes \id)(X)(a\otimes b\otimes 1)$ in norm by sums of
products of the form $(\Delta \otimes\id)(X(c\otimes 1))(d \otimes
1\otimes 1)$, where %Changed lower indices c
$c\in \Gr{A}{k}{t}{r}{s}$, $d \in
\Gr{A}{t}{l}{}{n}$ and $t\in I$. But \eqref{eq:invariance} implies that for such $c,d,t$,
\begin{align*}
(\id \otimes \phi_{qs} 
  \otimes \id)(  (\Delta \otimes\id)(X(c\otimes 1)))(d \otimes %in next line changed index k into t
1) &= \UnitC{t}{q}d \otimes (\phi_{ts} \otimes \id)(X(c\otimes 1)) \in A
\otimes C.
\end{align*}
Therefore, $X(A \otimes C) \subseteq A\otimes C$.
  \end{proof}

Similarly as in the case of compact quantum groups, Proposition \ref{prop:corep-complemented}
and a straightforward application of Zorn's Lemma imply:
\begin{Cor} \label{cor:invariant-complemented}
  Let $X$ be a unitary corepresentation. Then every orthogonal projection onto a closed invariant
  subspace intertwines $X$.
\end{Cor}

% The next corollary fails badly in general I would say :) Moved it further down the line

%\begin{Cor} \label{cor:corep-decompose}
 % Every unitary corepresentation of a $C^{*}$-pcqg is a direct sum of
 % irreducible unitary corepresentations.
%\end{Cor}
We next show that every irreducible unitary corepresentation
is rcfd, using an averaging construction, which
will also be used to prove the Schur orthogonality relations later on.
\begin{Lem} \label{lem:intertwiner-averaged}
  Let $X$ and $Y$ be unitary corepresentations of $(A,\Delta)$ on Hilbert spaces $H$ and $K$,
  respectively, and let $T \in \mathcal{B}(H,K)$. Then for all $m,l \in I$, the sums
  \begin{align*}
    \hat T^{(m)} &:= \sum_{k} (\phi_{km} \otimes \id)(Y(1 \otimes T)X^{*}) \quad \text{and} \quad
    \check T^{(l)}=\sum_{n} (\phi_{ln}\otimes \id)(Y^{*}(1 \otimes T)X)
  \end{align*}
  converge in the strong operator topology to intertwiners $\hat
  T^{(m)}$ and $\check T^{(l)}$ from $X$ to $Y$.
\end{Lem}
\begin{proof}
  We only prove the assertion concerning $\hat T^{(m)}$; the proof for $\check T^{(l)}$ is similar.
  
  The sum defining $\hat T^{(m)}$ converges in the strong operator topology because by
  \eqref{eq:corep-pi},
  \begin{align*}
    (\phi_{km} \otimes \id)(Y(1 \otimes T)X^{*}) = \lambda^{H}_{k}
    (\phi_{km} \otimes \id)(Y(1 \otimes T)X^{*}) \lambda^{H}_{k}.
  \end{align*}
  To see that $\hat T^{(m)}$ is an intertwiner, we use invariance of $h$ and find
  \begin{align*}
    Y(1 \otimes \hat T^{(m)})X^{*} &=
    \sum_{k} (\id \otimes \phi_{km} \otimes \id)(Y_{13}Y_{23}(1 \otimes T)X^{*}_{23}X^{*}_{13}) \\
    &= \sum_{k} ((\id \otimes \phi_{km})\circ \Delta \otimes \id)(Y(1 \otimes T)X^{*}) \\
    &=\sum_{l} \lambda_{l} \otimes (\phi_{lm} \otimes \id)(Y(1\otimes T)X^{*}) \\
    &= \sum_{l} \lambda_{l}   \otimes \lambda^{H}_{l} \hat T^{(m)}.
  \end{align*}%Added parenthesis in next line
  With \eqref{eq:corep-pi}, we conclude that $Y(1\otimes \hat T^{(m)})
  = (1 \otimes \hat T^{(m)})X$.
\end{proof} 
\begin{Lem} \label{lem:intertwiner-compact} Let $X$ and $Y$ be unitary corepresentations of
  $(A,\Delta)$ on Hilbert spaces $H$ and $K$, respectively, let $T \in \mathcal{K}(H,K)$ and define
  $\hat T^{(m)}$ and $\check T^{(l)}$ as above. Then $\rho^{H}_{n}\hat T^{(m)}$ and
  $\lambda^{H}_{k}\check T^{(l)}$ are compact for all $k,n\in I$.
\end{Lem}
\begin{proof}
  We only prove the assertion concerning $\hat T^{(m)}$; a similar reasoning applies to $\check
  T^{(l)}$.

By Lemma \ref{lem:corep-intertwine} and
  \begin{align*}
    \rho^{H}_{n}(\phi_{km} \otimes \id)(Y(1\otimes T)X^{*}) = 
    (\phi_{km} \otimes \id)(Y(\lambda_{n} \otimes T)X^{*}),
  \end{align*}
and  by \eqref{eq:corep-pi}, 
\begin{align*}
    Y(\lambda_{n} \otimes T)X^{*} =  \sum_{p} Y(\UnitC{n}{p} \otimes \rho^{H}_{p}T)X^{*}.
  \end{align*}
  Since $T$ is compact, the sum converges in norm and each summand $Y(\UnitC{n}{p} \otimes
  \rho^{H}_{p}T)X^{*}$ lies in $A \otimes \mathcal{K}(H,K)$. Thus,
  \begin{align*}
    Y(\lambda_{n} \otimes T)X^{*} \in A \otimes \mathcal{K}(H,K).
  \end{align*}
 We can therefore approximate $Y(\lambda_{n}
  \otimes T)X^{*}$ by finite sums $\sum_{i} a_{i} \otimes S_{i}$,
  where $a_{i} \in A$, $S_{i} \in \mathcal{K}(H,K)$, and by
  \eqref{eq:corep-pi}, we may assume $a_{i}=\lambda_{k_{i}}a_{i}
  \lambda_{l_{i}}$ and $S_{i}=\lambda_{k_{i}}^{K}S\lambda_{l_{i}}^{H}$
  for some $k_{i},l_{i}\in I$. Then
  \begin{align*}
\sum_{i,k}    (\phi_{km} \otimes \id)(a_{i} \otimes S_{i}) = \sum_{i}
\phi_{k_{i}m}(a_{i})S_{i} \in \mathcal{K}(H,K)
  \end{align*}
and  $  \rho^{H}_{n}\hat T^{(m)} - \sum_{i}
\phi_{k_{i}m}(a_{i})S_{i}$ is equal to the sum of the elements
\begin{align*}
  D_{k}:=   (\phi_{km} \otimes \id)\left(Y(\lambda_{n} \otimes T)X^{*} -
    \sum_{i}a_{i} \otimes S_{i}\right).
\end{align*}
Now, \eqref{eq:corep-pi} and the assumption on the $a_{i}$ and $S_{i}$ %Changed H to K in next line
implies $D_{k} = \lambda_{k}^{K}D_{k}\lambda_{k}^{H}$, and  since each $\phi_{km}$ is a state,
\begin{align*}
  \|D_{k}\| \leq \left\|Y(\lambda_{n} \otimes T)X^{*} -
    \sum_{i}a_{i} \otimes S_{i}\right\| \quad\text{for all } k.
\end{align*}
Therefore, 
\begin{align*}
\left\|    \rho^{H}_{n}\hat T^{(m)} - \sum_{i}
\phi_{k_{i}m}(a_{i})S_{i}\right\| \leq \left\|\sum_{k} D_{k}\right\| \leq \left\|Y(\lambda_{n} \otimes T)X^{*} -
    \sum_{i}a_{i} \otimes S_{i}\right\|.
\end{align*}
Since the right hand side can be made arbitrarily small, we can
conclude that $\rho^{H}_{n}\hat T^{(m)}$ is compact. 
\end{proof}
\begin{Prop} \label{prop:corep-rcfd}
Every irreducible unitary corepresentation of a  $C^{*}$-pcqg is rcfd.
\end{Prop}
\begin{proof}
  Let $(A,\Delta)$ be a $C^{*}$-pcqg with invariant integral $h$ and
  let $X$ be a unitary irreducible corepresentation of $(A,\Delta)$ on
  a Hilbert space $H$.

  We claim that there exists a non-zero intertwiner of $S$ of $X$ such
  that $\rho^{H}_{n}S$ is compact for all $n$.  Indeed, choose
  projections $T_{i} \in \mathcal{K}(H)$ converging strictly to
  $\id_{H}$ and fix $m\in I$. Then products $X(1\otimes T_{i})X^{*}$
  converge strictly to $\sum_{k} \lambda_{k} \otimes \lambda_{k}^{H}$
  and the sums
  \begin{align*}
    \hat T_{i} := \sum_{k}(\phi_{km} \otimes \id)(X(1\otimes T_{i})X^{*})
  \end{align*}
  converge strictly to $\id_{H}$ in $M(\mathcal{K}(H))$. In  particular, we find some $i$ such that $\hat T_{i}\neq 0$.  By Lemma  \ref{lem:intertwiner-averaged} and  Lemma   \ref{lem:intertwiner-compact}, we can now take $S:=\hat T_{i}$.

%Removed `we find an n'
  Since $\rho^{H}_{n}S$ is compact for all $n$ and $S$ is non-zero, we  find a non-zero $\lambda\in \C$ such that the  intertwiner $S-\lambda\id_{H}$ of $X$ has non-trivial kernel.  But  this kernel is an invariant subspace and therefore must be $H$.  Thus, $S=\lambda\id_{H}$ and hence $\rho^{H}_{n}$ is compact for all  $n$.

%removed S
  A similar argument shows $\lambda^{H'}_{l}$ is compact for all  $l$. Therefore, $X$ is rcfd.
\end{proof}

% Moved corollary, added proof, to complete (hyperobject set should be mentioned earlier)


\begin{Cor} \label{cor:corep-decompose} Every unitary corepresentation
  of a $C^{*}$-pcqg  is a direct sum of irreducible unitary
  corepresentations. If the unitary corepresentation has finite hyperobject support, then the
  direct sum is finite.
\end{Cor}
\begin{proof} It follows from Lemma \ref{lem:intertwiner-compact} that any unitary corepresentation is a direct sum of rcfd corepresentations, and any rcfd corepresentation is a direct sum of rcfd corepresentations with singletons as hyperobject support. But any such rcfd corepresentation has finite-dimensional intertwiner space since the restriction to a non-zero $\Hsp_n$ is faithful by Proposition \ref{prop:restrict-morphisms}, hence is a finite direct sum of irreducible unitary corepresentations. 
\end{proof} 

We finally note how  Schur's lemma translates into the present context. % Can be formulated as a corollary and I think the proof can be skipped
\begin{Cor} \label{lem:schur}
  \begin{enumerate}
  \item   A unitary corepresentation $X$ is irreducible if and only if every
  intertwiner of $X$ is a scalar multiple of the identity.
\item Let $X$ and $Y$ be irreducible unitary corepresentations on Hilbert spaces $H$ and $K$,
  respectively.  Then either $\dim \Hom(X,Y) = 0$ or $\dim\Hom(X,Y)=1$, and in the second case,
  there exists a unitary intertwiner from $X$ to $Y$.
  \end{enumerate}
\end{Cor}
% \begin{proof}%changed ref to corollary
%   1. The ``if'' part follows from Corollary \ref{cor:invariant-complemented}.  Assume that $X$ is
%   irreducible and $T$ intertwines $X$. By \ref{prop:corep-rcfd}, the components $\Grd{T}{k}{l}$ are
%   compact operators and hence there exists a $\lambda\in \C$ such that the intertwiner $T-\lambda
%   \id$ has non-trivial kernel. This kernel is easily seen to be invariant, hence $T-\lambda \id =
%   0$.

%   2. If $T$ is a non-zero intertwiner from $X$ to $Y$, then $T^{*}T$ intertwines $X$ and $TT^{*}$
%   intertwines $Y$, so that $T^{*}T=\lambda \id$ and $TT^{*}=\mu\id$ for some $\lambda,\mu\in\C$ by
%   1.\ But since $T^{*}T$ and $TT^{*}$ have the same spectrum away from $0$, we get $\lambda=\mu$ and
%   $\lambda^{-1/2}T$ is a unitary intertwiner. If $S$ is another intertwiner from $X$ to $Y$, then $T^{*}S$
%   intertwines $X$ and hence is a multiple of $\id$, whence $\dim\Hom(X,Y)=1$.
% \end{proof}
We will need the following sharpening. % Proof of following lemma should be cleaned up a bit
\begin{Lem} \label{lem:schur-algebraic}
  Let $X$ and $Y$ be irreducible unitary corepresentations on Hilbert spaces $H$ and $K$,
  respectively, and assume that operators $ \Grd{T}{k}{l} \in
  \mathcal{B}(\Grd{H}{k}{l},\Grd{K}{k}{l}) $ satisfy $(1 \otimes \Grd{T}{k}{l})\Gr{X}{k}{l}{m}{n} =
  \Gr{Y}{k}{l}{m}{n}(1\otimes \Grd{T}{m}{n})$ for all $k,l,m,n$. Then they are the restrictions of
  an intertwiner $T$ from $X$ to $Y$. 
\end{Lem}
\begin{proof}
  We only need to show that $\sup_{k,l} \|\Grd{T}{k}{l}\|<\infty$. Replacing $\Grd{T}{k}{l}$ by
  $(\Grd{T}{k}{l})^{*}\Grd{T}{k}{l}$, if necessary, we may assume $X=Y$. If $T$ is non-zero, choose $k,l$ such that $\Grd{T}{k}{l}$ is  non-zero,  and $\lambda \in \C$ such that $\Grd{T}{k}{l}-\lambda \id_{\Grd{H}{k}{l}}$ has non-trivial kernel. Then the direct sum of the kernels of $\Grd{T}{m}{n} - \lambda \id_{\Grd{H}{m}{n}}$, as $m$ and $n$ vary, is  a non-trivial invariant subspace for $X$.  As $X$ is irreducible, this kernel must $H$ and so $\Grd{T}{m}{n} = \lambda \id_{\Grd{H}{m}{n}}$ for all $m,n$.
\end{proof}


\subsection{The regular  corepresentation and duals of corepresentations}

Let $(A,\Delta)$ be a $C^{*}$-pcqg with an invariant integral $\phi$.

We  construct a left-regular corepresentation of $(A,\Delta)$ and
use it to associate to every irreducible unitary corepresentation a
dual one as follows.

 
 Denote by $H_{h}$ the associated GNS-space, by $\zeta^k_m \in H_{h}$ the image of $\UnitC{k}{m}
 \in A$, and write the GNS-representation of $A$ on $H_{h}$ as left multiplication.   Equip
 $H_{h}$ with the $I^{2}$-grading given by
 \begin{align*}
   \lambda^{H_{h}}_{m}  (a\zeta^{k}_{l}) &=  \rho_{m}a \zeta^{k}_{l}, &
   \rho^{H_{h}}_{m} (a\zeta^{k}_{l}) &= a\rho_{m} \zeta^{k}_{l} = \delta_{m,l} a\zeta^{k}_{l}
 \end{align*}
 for all $k,l,m\in I$ and $a\in A$.
 Choose a non-degenerate, faithful representation of $A$ on some Hilbert space $K$ and identify $A$
 with its image.
 \begin{Lem}\label{lem:reg-corep-pi}
   There exists a unique partial isometry $\tilde V$ on $H_{h} \otimes
   K$ such
   that
   \begin{align} \label{eq:regular-corep-pi}
     \tilde V(a \zeta^k_m \otimes \eta) &=
     \Delta(a) \sum_{l}(\zeta^{k}_{l} \otimes \UnitC{l}{m}\eta)
   \end{align}
   for all $a\in A$, $k,m\in I$ and $\eta\in K$. Its support and range
   projections are given by
   \begin{align*}
     \tilde V^{*}\tilde V &= \sum_{l} \rho^{H_{h}}_{l} \otimes \rho_{l}, &
     \tilde V \tilde V^{*} &= \sum_{k} \lambda^{H_{h}}_{k} \otimes \lambda_{k}.
   \end{align*}
 \end{Lem}
 \begin{proof}
   Let $p,q\in I$ and $a,k,m,\eta$ as above. Given $\xi \in K$, denote by $\omega_{\xi}\in A^{*}$ the vector
   functional $a \mapsto \langle \xi|a\xi\rangle$.% Ref to notation in next line
   Then, using the notation \eqref{DefSpB}, $\omega_{\UnitC{l}{m}\eta} \in \Gr{B}{l}{m}{l}{m}$ and by \eqref{EqInvRp},
   \begin{align*}
  \left\|\Delta(a)\sum_{l}(\zeta^{k}_{l}\otimes \UnitC{l}{m}\eta)\right\|^{2} &= \sum_{l} (\phi_{kl}
  \otimes \omega_{\UnitC{l}{m}\eta})(\Delta(a^{*}a)) \\
  &= \sum_{l}\omega_{\eta}(1\Grru{l}{m}) \phi_{km}(a^{*}a) 
  = \left\|a\zeta^{k}_{m} \otimes  \rho_{m}\eta \right\|^{2}.
\end{align*}
Therefore, the formula above defines a partial isometry $\tilde V$ with support projection $\sum_{m}
\rho^{H_{h}}_{m} \otimes \rho_{m}$. %Changed wording somewhat 
As the image of $\tilde V$ is closed, it contains by its definition the closed linear span of all elements of the form
$\Delta(a) (\zeta^{k}_{l} \otimes b\eta)$, where $a,b\in A$, $k,l,m\in I$, $\eta\in K$. By  \eqref{CondDi}, this is equal to the closed linear span of all
elements of the form $\rho_{m}c \zeta^{k}_{l} \otimes \lambda_{m}d\eta$,
where $c,d\in A$, $k,l,m \in I$, $\eta\in K$. Since $\rho_{m}c\zeta^{k}_{l} =%Added tildes in next line
\lambda^{H_{h}}_{m}(c\zeta^{k}_{l})$, we can conclude that $\tilde V \tilde V^{*}=\sum_{m} \lambda^{H_{h}}_{m}
\otimes \lambda_{m}$ as claimed.
 \end{proof}
 \begin{Lem}\label{lem:reg-corep-mult}
   The products $ \tilde V(\mathcal{K}(H_{h})\otimes A)$ and 
   $(\mathcal{K}(H_{h}) \otimes A)\tilde V$ lie in %changed order
  $\mathcal{K}(H_h)\otimes A$ so that $\tilde V \in M(\mathcal{K}(H_h)\otimes A)$.
 \end{Lem}
 \begin{proof}
   Given $\xi,\eta\in H_{h}$, denote by $|\xi\rangle\langle \eta| \in \mathcal{K}(H_{h})$ ket-bra   operator $\vartheta \mapsto \langle \eta|\vartheta\rangle \xi$. 

   Let $a\in A$, $k,m,r,s\in I$, $\xi,\vartheta \in H_{h}$ and   $\eta\in K$. 
   
   We first show that $\tilde V(\mathcal{K}(H_{h})\otimes A) \subseteq
   \mathcal{K}(H_{h}) \otimes A$. By definition,
\begin{align*}
     \tilde V(|a\zeta^{k}_{m}\rangle\langle \xi| \otimes \UnitC{r}{s})(\vartheta \otimes \eta) 
     &=     \delta_{s,m} \langle \xi|\vartheta\rangle \Delta(a)(1 \otimes \UnitC{r}{s})(\zeta^{k}_{r}
     \otimes \eta).
   \end{align*} 
   Here, we can approximate $\Delta(a)(1\otimes \UnitC{r}{s}) \in A\otimes A$ by finite sums $\sum_{i} c_{i} \otimes
   d_{i}$ with $c_{i},d_{i} \in A$, and
   \begin{align*}
 \langle \xi|\vartheta\rangle (c_{i} \otimes
d_{i})(\zeta^{k}_{r} \otimes \eta) = 
 (|c_{i}\zeta^{k}_{r}\rangle\langle \xi| \otimes d_{i}) (\vartheta \otimes \eta).
   \end{align*}
  But  $\sum_{i} (|c_{i}\zeta^{k}_{r}\rangle\langle \xi| \otimes d_{i}) \in \mathcal{K}(H_{h}) \otimes
A$  and  one can check that, %added $m=s$, also added extra norm of $\xi$ in right hand side
for $m=s$, 
   \begin{align*}
     \left\|\tilde V(|a\zeta^{k}_{m}\rangle\langle \xi| \otimes \UnitC{r}{s}) -      \sum_{i}
       (|c_{i}\zeta^{k}_{r}\rangle\langle \xi| \otimes d_{i})\right\| \leq
     \left\|\Delta(a)(1\otimes \UnitC{r}{s}) -  \sum_{i} c_{i} \otimes
   d_{i}\right\| \|\xi\|.
   \end{align*}
   Therefore, $\tilde V(|a\zeta^{k}_{m}\rangle\langle \xi| \otimes \UnitC{r}{s})$ lies in   $\mathcal{K}(H_{h}) \otimes A$ and hence $\tilde V(\mathcal{K}(H_{h})\otimes A) \subseteq
   \mathcal{K}(H_{h}) \otimes A$.
   

   We next show that $\tilde V^{*}(\mathcal{K}(H_{h}) \otimes A)
   \subseteq \mathcal{K}(H_{h}) \otimes A$.  By Lemma
   \ref{lem:reg-corep-pi},
   \begin{align*}
     \tilde V^{*}(|a\zeta^{k}_{m}\rangle\langle\xi| \otimes
     \UnitC{r}{s})(\vartheta\otimes \eta) &=  \langle
     \xi|\vartheta\rangle \tilde V^{*}(\rho_{r}a\UnitC{k}{m}\otimes
     \UnitC{r}{s})(\zeta^{k}_{m} \otimes \eta).
   \end{align*}
   By assumption \eqref{CondDi}, we can approximate $\rho_{r}a
   \UnitC{k}{m}\otimes
   \UnitC{r}{s}$ by finite sums $\sum_{i} \Delta(c_{i})(1 \otimes d_{i})$ with
   $c_{i}, d_{i} \in A$, such that $c_{i} =c_{i}\rho_{l_{i}}$ and   $d_{i}=\rho_{l_{i}}d_{i}$ for some $l_{i} \in I$, and then
   \begin{align*}
     \langle \xi|\vartheta\rangle \tilde
     V^{*}\Delta(c_{i})(\zeta^{k}_{m}\otimes d_{i}\eta) = \langle
     \xi|\vartheta\rangle c_{i}\zeta^{k}_{l_{i}} \otimes d_{i}\eta =
     (|c_{i}\zeta^{k}_{l_{i}}\rangle\langle\xi| \otimes
     d_{i})(\vartheta\otimes \eta).
   \end{align*}
But $\sum_{i}     (|c_{i}\zeta^{k}_{l_{i}}\rangle\langle\xi| \otimes
     d_{i}) \in \mathcal{K}(H_{h}) \otimes A$ and again, one can check   that%added norm of $\xi$
     \begin{align*}
       \left\|      \tilde V^{*}(|a\zeta^{k}_{m}\rangle\langle\xi| \otimes
     \UnitC{r}{s}) - \sum_{i}     (|c_{i}\zeta^{k}_{l_{i}}\rangle\langle\xi| \otimes
     d_{i})  \right\| \leq \left\|
\rho_{r}a
   \UnitC{k}{m}\otimes
   \UnitC{r}{s} - \sum_{i} \Delta(c_{i})(1 \otimes d_{i})
   \right\|\|\xi\|.
     \end{align*}
     Therefore, $   \tilde V^{*}(|a\zeta^{k}_{m}\rangle\langle\xi| \otimes
     \UnitC{r}{s})$ lies in $\mathcal{K}(H_{h}) \otimes A$ and the    second inclusion follows.
\end{proof}
\begin{Rem}
  We also have $(1 \otimes A)\tilde V(\mathcal{K}(H_{h})\otimes 1)
  \subseteq \mathcal{K}(H_{h}) \otimes A$. Indeed, for all $r,s\in I$,
   \begin{align*}
     (1 \otimes \UnitC{r}{s})\tilde
     V(|a\zeta^{k}_{m}\rangle\langle\xi|\otimes 1)(\vartheta \otimes
     \eta) = \langle \xi|\vartheta \rangle(1 \otimes
     \UnitC{r}{s})\Delta(a) \sum_{l} (\zeta^{k}_{l} \otimes
     \UnitC{l}{m}\eta),
   \end{align*}
where we can approximate $(1
   \otimes \UnitC{r}{s})\Delta(a) \in A\otimes A$ by finite sums
   $\sum_{i} c_{i} \otimes d_{i}$ with homogeneous $c_{i},d_{i} \in
   A$, and
   \begin{align*}
     \sum_{i,l} \langle \xi|\vartheta\rangle (c_{i} \otimes d_{i})(\zeta^{k}_{l} \otimes
     \UnitC{l}{m} \eta) = \sum_{i,l} (|c_{i}\zeta^{k}_{l}\rangle\langle \xi| \otimes
     d_{i}\UnitC{l}{m}) (\vartheta \otimes \eta),
   \end{align*}
   where the sum becomes finite by homogenity of the $c_{i}$ and $d_{i}$. Now, one concludes   similarly as above that $(1 \otimes \UnitC{r}{s})\tilde
   V(|a\zeta^{k}_{m}\rangle\langle\xi|\otimes 1) \in \mathcal{K}(H_{h}) \otimes A$ and consequently
   $(1\otimes A)\tilde V(\mathcal{K}(H_{h}) \otimes 1) \subseteq \mathcal{K}(H_{h}) \otimes A$.
  
    % To look into
However,   we do not know whether $(\mathcal{K}(H_{h}) \otimes 1)
   \tilde V(1\otimes A)$ is contained in $A\otimes \mathcal{K}(H)$.
\end{Rem}

As before, we write
\begin{align*}
  \Grd{B}{l}{n}  =\{ \omega \in A^{*} : \omega(a)=\omega(a\UnitC{l}{n}) \text{ for all  }a\in A\} \subseteq A^{*},
\end{align*}
and denote by  $\zeta^{k}_{m}$  the  image of $\UnitC{k}{m}$ in $H_{h}$. 

Flipping $\tilde V$, we obtain the \emph{regular corepresentation} $V$ of $(A,\Delta)$.
 \begin{Prop}
Let $(A,\Delta)$ be a $C^{*}$-pcqg with invariant integral $h$.
\begin{enumerate}
\item  There exists a unique unitary corepresentation $V$ of  $(A,\Delta)$ on the associated GNS-space $H_{h}$ such that  for all
$\omega \in \Grd{B}{l}{n}$ and $a \in A$,
  \begin{align} \label{eq:regular-corep}
    (\omega \otimes \id)(V) a\zeta^{k}_{m} &= \delta_{m,n} (\id \otimes \omega)(\Delta(a)) \zeta^{k}_{l}.
  \end{align}
\item Denote by $\Gr{H}{p}{q}{}{h}
  \subseteq H_{h}$ the closure of the image of  $\Gr{A}{p}{q}{}{}$ in $H_{h}$. Then  $\Gr{H}{p}{q}{}{h}$ is invariant with  respect to $H_{h}$ and $V$ restricts to a unitary corepresentation $V^{(pq)}$ on $\Gr{H}{p}{q}{}{h}$.
\end{enumerate}
 \end{Prop}
 \begin{proof}
(1)   Denote by $\sigma$ the flip $ M(\mathcal{K}(H) \otimes A) \to M(A
   \otimes \mathcal{K}(H))$ and let $V=\sigma(\tilde V)$. Then the   preceding lemmas show that $V$ lies in $M(A\otimes
   \mathcal{K}(H_{h}))$. Given an element $\omega \in \Grd{B}{l}{n}$, we can assume without loss of generality that $\omega=\omega_{\xi,\eta}$ with $\xi,\eta\in K$ such that $\eta=\UnitC{l}{n}\eta$, and then \eqref{eq:regular-corep} follows easily from \eqref{eq:regular-corep-pi}. Let us finally show that $(\Delta \otimes \id)(V)=V_{13}V_{23}$. Let $\omega \in \Grd{B}{l}{n}, \omega' \in \Grd{B}{p}{q}$ and $a\in A$. Then
$\omega \ast\omega' \in \Grd{B}{l}{q}$ is zero unless $n=p$, and
   \begin{align*}
     (\omega \otimes \omega' \otimes \id)(V_{13}V_{23})a\zeta^{k}_{m} &= (\omega \otimes \id)(V)(\omega' \otimes \id)(V) a\zeta^{k}_{m}\\
     &= \delta_{q,m} (\omega \otimes \id)(V)(\id \otimes \omega')(\Delta(a))\zeta^{k}_{p} \\
     &= \delta_{q,m}\delta_{n,p}  (\id \otimes \omega \otimes \omega')((\Delta\otimes \id)(\Delta(a)))\zeta^{k}_{l} \\
     &= \delta_{q,m} (\id \otimes \omega \ast \omega')(\Delta(a))\zeta^{k}_{l} \\
     &= (\omega\ast\omega' \otimes \id)(V)a\zeta^{k}_{m} \\
     &= (\omega \otimes \omega' \otimes \id)((\Delta \otimes \id)(V))a\zeta^{k}_{m}.
   \end{align*}

(2) Straightforward.
 \end{proof}

Given a Hilbert space $H$, we denote by $\overline{H}$ the conjugate Hilbert space and by $\eta\mapsto \overline{\eta}$ the canonical anti-unitary $H\to\overline{H}$. Given a second Hilbert space $K$ and $T \in \mathcal{B}(H,K)$, we define $\overline{T}\in
\mathcal{B}(\overline{H},\overline{K})$ by $\overline{\eta} \mapsto \overline{(T\eta)}$.  If $H$ is $I^{2}$-graded, we equip $\overline{H}$ with the reversed $I^{2}$-grading, that is, $\Grd{(\overline{H})}{k}{l}=\overline{(\Grd{H}{l}{k})}$.

Let now $X$ be an irreducible unitary corepresentation on a Hilbert space $H$, which is automatically rcfd by Proposition \ref{prop:corep-rcfd}. Then for each $k,l,m,n$, we have components $\Gr{X}{k}{l}{m}{n} \in \Gr{A}{k}{l}{m}{n}\otimes \mathcal{B}(\Grd{H}{m}{n},\Grd{H}{k}{l})$, and we
define
\begin{align*}
  \Gr{\overline{X}}{k}{l}{m}{n} &:=  (\Gr{X}{l}{k}{n}{m})^{*\otimes \overline{(\phantom{T})}} \in
  \Gr{A}{k}{l}{m}{n} \otimes \mathcal{B}(\Grd{\overline{H}}{m}{n},\Grd{\overline{H}}{k}{l}).
\end{align*}
Thus, if $ \Gr{X}{l}{k}{n}{m} = \sum_{i} a_{i} \otimes T_{i}$, then $ \Gr{\overline{X}}{k}{l}{m}{n}
= \sum_{i} a_{i}^{*} \otimes \overline{T_{i}}$.

A straightforward calculation shows that the relation $(\Delta\otimes \id)(X)=X_{13}X_{23}$ implies
  \begin{align*}
    (\Delta_{pq} \otimes \id)(\Gr{\overline{X}}{k}{l}{m}{n}) =
     (\Gr{\overline{X}}{k}{l}{p}{q})_{13}(\Gr{\overline{X}}{p}{q}{m}{n})_{23}.
  \end{align*}
We want to show that after rescaling, the components  $\overline{X}$ is, in a sense, equivalent to an irreducible unitary  corepresentation. The idea is to embed $\overline{X}$ into the regular corepresentation.
\begin{Lem} \label{lem:construct-intertwiner} Let $K$ be an $I^{2}$-graded Hilbert space, assume   given elements
  \begin{align*}
    \Gr{Y}{k}{l}{m}{n}\in \Gr{A}{k}{l}{m}{n} \otimes \mathcal{B}(\Grd{K}{m}{n},\Grd{K}{k}{l})
  \end{align*}
   satisfying $    (\Delta_{kl} \otimes \id)(\Gr{Y}{p}{q}{m}{n}) =
     (\Gr{Y}{p}{q}{k}{l})_{13}(\Gr{Y}{k}{l}{m}{n})_{23}
$. Then for each $\eta\in \Grd{K}{p}{q}$,  the operators
\begin{align*}
  \Gr{T}{}{(\eta)}{m}{n} \colon
\Grd{K}{m}{n} \to  \Grd{(H_{h})}{m}{n}, \quad \xi \mapsto (\id \otimes \omega_{\eta,\xi})(\Gr{Y}{p}{q}{m}{n}) \zeta^{q}_{n},
\end{align*}
satisfy
  \begin{align*}
    \Gr{V}{k}{l}{m}{n}(1\otimes \Gr{T}{}{(\eta)}{m}{n}) = (1 \otimes
    \Gr{T}{}{(\eta)}{k}{l})\Gr{Y}{k}{l}{m}{n}.
  \end{align*}
\end{Lem}
\begin{proof}
Let $\eta \in \Grd{K}{p}{q}$ and  $\xi \in \Grd{K}{m}{n}$. Then
  \begin{align*}
    \Gr{V}{k}{l}{m}{n}(1\otimes \Gr{T}{}{(\eta)}{m}{n})(\UnitC{l}{n}\otimes \xi) &= \Gr{V}{k}{l}{m}{n}(\UnitC{l}{n}
    \otimes (\id \otimes \omega_{\eta,\xi})(\Gr{Y}{p}{q}{m}{n})\zeta^{q}_{n}) \\ &= (\Delta_{kl}^{\op}    \otimes
    \omega_{\eta,\xi})(\Gr{Y}{p}{q}{m}{n})(\UnitC{l}{n}\otimes \zeta^{q}_{l}) \\
    &= (\id \otimes \id \otimes \omega_{\eta,\xi})((\Gr{Y}{p}{q}{k}{l})_{23}(\Gr{Y}{k}{l}{m}{n})_{13})(\UnitC{l}{n}\otimes
    \zeta^{q}_{l}) \\
    &= (\id \otimes \Gr{T}{}{(\eta)}{k}{l})\Gr{Y}{k}{l}{m}{n}(\UnitC{l}{n}\otimes \xi)
  \end{align*}
and the claim follows.
\end{proof}


\begin{Prop} \label{prop:corep-dual}
There exist invertible operators $\Grd{T}{k}{l} \in \mathcal{B}(\Grd{\overline{H}}{k}{l})$ such that the sum
\begin{align} \label{eq:dual}
   \sum_{k,l,m,n} (1\otimes \Grd{T}{k}{l}) \Gr{\overline{X}}{k}{l}{m}{n} (1 \otimes
  \Grd{T}{m}{n})^{-1}
\end{align}
converges strictly to an irreducible unitary corepresentation. This corepresentation is uniquely
determined by $X$ up to equivalence and the operators 
\begin{align*} 
  \Grd{F}{k}{l} := \overline{(\Grd{T}{l}{k})^{*}\Grd{T}{l}{k}} \in \mathcal{B}(\Grd{H}{k}{l})
\end{align*}
are uniquely determined by $X$ up to a scalar factor that does not depend on $k,l$.
\end{Prop}
\begin{proof}
Let us first prove uniqueness. Assume that $\Grd{T}{k}{l},\Grd{S}{k}{l} \in
\mathcal{B}(\Grd{\overline{H}}{k}{l})$ are invertible operators such that
\begin{align*}
  Y = \sum_{k,l,m,n} (1\otimes \Grd{S}{k}{l}) \Gr{\overline{X}}{k}{l}{m}{n} (1 \otimes
  \Grd{S}{m}{n})^{-1}  \text{ and }
  Z = \sum_{k,l,m,n} (1\otimes \Grd{T}{k}{l}) \Gr{\overline{X}}{k}{l}{m}{n} (1 \otimes
  \Grd{T}{m}{n})^{-1}
\end{align*}
are irreducible unitary corepresentations. Then Lemma \ref{lem:schur-algebraic}, applied to $Y,Z$
and the operators $(\Grd{S}{k}{l}(\Grd{T}{k}{l})^{-1})_{k,l}$, shows that these operators are the
components of an invertible intertwiner $R$ between $Y$ and $Z$, and by Lemma \ref{lem:schur} 2.,
this intertwiner is a scalar multiple of a unitary. The uniqueness assertions follow.

Let us now prove existence.  Choose $p,q$ and $\eta\in \Grd{\overline{H}}{p}{q}$ such that the
family of operators $\Gr{T}{}{(\eta)}{m}{n} \in
\mathcal{B}(\Grd{\overline{H}}{m}{n},\Grd{(H_{h})}{m}{n})$ constructed in Lemma
\ref{lem:construct-intertwiner} is non-zero.

We claim that then all $\Gr{T}{}{(\eta)}{m}{n}$ are injective.  Indeed, denote by $\Grd{P} {m}{n}$
the orthogonal projection onto the kernel of $\Gr{T}{}{(\eta)}{m}{n}$. Then the relation
\begin{align} \label{eq:dual-intertwine-aux}
  \Gr{V}{k}{l}{m}{n}(1\otimes \Gr{T}{}{(\eta)}{m}{n}) = (1\otimes \Gr{T}{}{(\eta)}{k}{l})\Gr{\overline{X}}{k}{l}{m}{n}
\end{align}
implies $(1\otimes \Gr{T}{}{(\eta)}{k}{l})\Gr{\overline{X}}{k}{l}{m}{n} (1 \otimes \Grd{P}{m}{n}) =
0$ and hence $\Gr{\overline{X}}{k}{l}{m}{n} (1 \otimes \Grd{P}{m}{n}) = (1 \otimes \Grd{P}{k}{l})
\overline{X}(1 \otimes P)$, and therefore,
\begin{align*}
  (1 \otimes P)X(1\otimes P) = X(1\otimes P),
\end{align*}
where $P=\sum_{m,n} \Grd{P}{m}{n}$.  By irreducibility of $X$, we can conclude that $P=0$ or
$P=\id_{H}$. Since some $\Gr{T}{}{(\eta)}{m}{n}$ is non-zero, we must have $P=0$ and hence each $\Gr{T}{}{(\eta)}{m}{n}$ is injective.

Now, take the polar decomposition $\Gr{T}{}{(\eta)}{m}{n}=\Grd{U}{m}{n} |\Gr{T}{}{(\eta)}{m}{n}|$
and let $\Grd{T}{m}{n}=|\Gr{T}{}{(\eta)}{m}{n}|$.   Equation \eqref{eq:dual-intertwine-aux} shows
that the image of $U:=\sum_{m,n} \Grd{U}{m}{n}$ is a closed invariant subspace for $V$, and by Corollary
\ref{cor:invariant-complemented}, $V$ restricts to a unitary corepresentation on this subspace.
Now, the sum \eqref{eq:dual} is equal to the unitary corepresentation $(1\otimes U^{*})V(1\otimes
U)$, and a similar argument like the one involving $P$ shows that this unitary corepresentation is irreducible.
\end{proof}

\subsection{Matrix coefficients, Schur orthogonality relations and Peter-Weyl}


To every unitary corepresentation $X$ of $(A,\Delta)$ on a Hilbert space $H$, we associate a space
of \emph{matrix coefficients}
\begin{align*}
\mathcal{C}(X) = \sum_{k,l,m,n}  \Gr{\mathcal{C}(X)}{k}{l}{m}{n},
\text{ where } \Gr{\mathcal{C}(X)}{k}{l}{m}{n} = \span \{ (\id \otimes
  \omega_{\eta,\xi})(\Gr{X}{k}{l}{m}{n}) : \eta,\xi \in H\} \subseteq
  \Gr{A}{k}{l}{m}{n}.
\end{align*}
%We call the matrix coefficients contained in the spaces $\Gr{\mathcal{C}(X)}{k}{l}{m}{n}$
%\emph{homogeneous}.

\begin{Prop} \label{prop:matrix-coefficients-dense}    Let $(A,\Delta)$ be a $C^{*}$-partial compact quantum group.   Then the matrix coefficients of irreducible unitary rcfd corepresentations
  are linearly dense in $A$. 
\end{Prop}
\begin{proof}
  Consider the regular corepresentation $V$. Let $a\in \Gr{A}{k}{l}{m}{n}$, $b\in
  \Gr{A}{p}{q}{r}{s}$ and $\eta=a\zeta^{k}_{m}$, $\eta=b\zeta^{q}_{s}$. Then by definition,
  \begin{align*}
    (\id \otimes \omega_{\eta,\xi})(V) = (\phi_{qs}\otimes \id)((b^{*} \otimes
    1)\Delta(a))
  \end{align*}
  and since $[(A\otimes 1)\Delta(A)]=[(A\otimes A)\Delta(1)]$, we can conclude that $\mathcal{C}(V)$
  is dense in $A$. But by Corollary \ref{cor:corep-decompose} and Proposition \ref{prop:corep-rcfd}, $V$ decomposes as a direct sum of
  irreducible unitary rcfd corepresentations, whence the assertion follows.
\end{proof}

 Matrix coefficients  satisfy the following Schur orthogonality relations.
\begin{Lem}\label{lem:schur-1}
  Let $X$ and $Y$ be inequivalent unitary irreducible corepresentations of $(A,\Delta)$. Then
  \begin{align*}
    \phi_{km}(a^{*}b) = 0 \quad \text{for all } a\in \mathcal{C}(X), \ b\in \mathcal{C}(Y).
  \end{align*}
\end{Lem}
\begin{proof}
It suffices to  consider elements of the form
  \begin{align*}
    a = (\id\otimes \omega_{\xi,\eta})(X) \quad\text{and}  \quad b=(\id \otimes \omega_{\zeta,\theta})(Y),
  \end{align*}
  where $\xi \in \Grd{H}{k}{l}$, $\eta\in \Grd{H}{m}{n}$, $\zeta \in \Grd{K}{k}{l}$, $\theta \in
  \Grd{K}{m}{n}$. Define $T\in \mathcal{B}(K,H)$ by $T\omega = \langle \zeta|\omega\rangle \xi$. 
  Then $a^{*}=(\id \otimes \omega_{\eta,\xi})(X^{*})$ and, in the notation of Lemma \ref{lem:intertwiner-averaged},
  \begin{align*}
    \phi_{km}(a^{*}b) &= (\phi_{km} \otimes \omega_{\eta,\theta})(X^{*}(1\otimes T)Y) 
 = \langle \eta|
     \check{T}^{(k)} \theta\rangle = 0,
  \end{align*}
  because $\check{T}^{(k)}$ intertwines $X$  and $Y$  and hence is $0$.
\end{proof}

\begin{Theorem} \label{thm:schur}
  Let $X$ be an irreducible unitary corepresentation on a Hilbert space $H$ with left hyperobject
  support  $\alpha$ and right hyperobject support $\beta$. Let $\Grd{F}{k}{l}
  \in \mathcal{B}(\Grd{H}{k}{l})$ be as in Proposition \ref{prop:corep-dual} and let
  $\Grd{G}{k}{l}:=\Gr{F}{}{-1}{k}{l}$.
  \begin{enumerate}
  \item For $l \in \beta$ and $m\in\alpha$, the numbers $d_G:=\sum_{k} \Tr (\Grd{G}{k}{l})$ and $d_F:=\sum_{n} \Tr (\Grd{F}{m}{n})$ are non-zero and do not depend on the choice of $l$ or $m$.
    \item  For all $k,m \in \alpha$ and $l,n\in \beta$,
    \begin{align*}
      (\phi_{ln} \otimes \id)((\Gr{X}{k}{l}{m}{n})^{*}\Gr{X}{k}{l}{m}{n})
      &=d_G^{-1}\Tr(\Gru{G}{k}{l})
      \id_{\Grd{H}{m}{n}}, \\
      (\phi_{km} \otimes \id)(\Gr{X}{k}{l}{m}{n}(\Gr{X}{k}{l}{m}{n})^{*})
      &=d_F^{-1}\Tr(\Grd{F}{m}{n})
      \id_{\Grd{H}{k}{l}}.
    \end{align*}
  \item    Denote by $\Sigma_{mnkl} \colon \Grd{H}{m}{n}\otimes \Grd{H}{k}{l} \to \Grd{H}{k}{l}\otimes
  \Grd{H}{m}{n}$ the flip map. Then
    \begin{align*}
      (\phi_{ln} \otimes \id \otimes
    \id)((\Gr{X}{k}{l}{m}{n})^{*}_{12}(\Gr{X}{k}{l}{m}{n})_{13})  &=  d_{G}^{-1}\cdot
    (\id_{\Grd{H}{m}{n}} \otimes \Grd{G}{k}{l}) \circ \Sigma_{klmn}, \\
   (\phi_{km} \otimes \id \otimes
   \id)((\Gr{X}{k}{l}{m}{n})_{13}(\Gr{X}{k}{l}{m}{n})^{*}_{12}) &= d_F^{-1} (\Grd{F}{m}{n}
   \otimes \id_{\Grd{H}{k}{l}}) \circ \Sigma_{klmn}.
    \end{align*}
  \end{enumerate}
\end{Theorem}
\begin{proof}
  We proceed similarly as in the proof of  Theorem 2.22 in \cite{}, and only prove the assertions
  involving $d_{G}$.

  Consider the operators
  \begin{align*}
    P_{klmn} &:= (\phi_{ln} \otimes \id \otimes
    \id)((\Gr{X}{k}{l}{m}{n})^{*}_{12}(\Gr{X}{k}{l}{m}{n})_{13})  \in
    \mathcal{K}(\Grd{H}{k}{l} \otimes \Grd{H}{m}{n}, \Grd{H}{m}{n}\otimes \Grd{H}{k}{l}).
  \end{align*}
  Then
  \begin{align*}
    P_{klmn} \circ \Sigma_{mnkl} = (\phi_{ln} \otimes \id \otimes \id)((\Gr{X}{k}{l}{m}{n})^{*}_{12}(\Sigma_{klkl})_{23}(\Gr{X}{k}{l}{m}{n})_{12})
  \end{align*}
  For fixed $k,l$, the multiplier $X \otimes \id_{\Grd{H}{k}{l}}$ is a unitary corepresentation on
  $H \otimes \Grd{H}{k}{l}$, and by Lemma \ref{lem:intertwiner-averaged}, the sum $\sum_{m,n}
  (P_{klmn}\circ \Sigma_{mnkl})$ intertwines $X \otimes \id_{\Grd{H}{k}{l}}$. Since $X$ is
  irreducible, the sum must take the form $\id_{H} \otimes S$ with some $S\in
  \mathcal{B}(\Grd{H}{k}{l})$ and then for all $m,n$,
  \begin{align} \label{eq:schur-aux-1}
    P_{klmn} = (\phi_{ln} \otimes \id \otimes
    \id)((\Gr{X}{k}{l}{m}{n})^{*}_{12}(\Gr{X}{k}{l}{m}{n})_{13})= (\id_{\Grd{H}{m}{n}} \otimes
    S) \circ \Sigma_{klmn}.
  \end{align}

  A similar argument shows that there exists a $T\in \mathcal{B}(\Grd{H}{m}{n})$ such that for all
  $k,l$,
\begin{align} \label{eq:schur-aux-2} 
  (\phi_{km} \otimes \id \otimes
  \id)((\Gr{X}{k}{l}{m}{n})_{12}(\Gr{X}{k}{l}{m}{n})^{*}_{13})= (\id_{\Grd{H}{k}{l}} \otimes T) \circ
  \Sigma_{mnkl}.
  \end{align}

  Now, choose  invertible operators $\Grd{T}{k}{l} \in
  \mathcal{B}(\Grd{\overline{H}}{k}{l})$ as in Proposition \ref{prop:corep-dual} such that the components
  \begin{align*}
    \Gr{Y}{k}{l}{m}{n} &:= (1 \otimes \Grd{T}{k}{l}) \Gr{\overline{X}}{k}{l}{m}{n} (1 \otimes
    \Grd{T}{m}{n})^{-1} 
  \end{align*}
  sum up to an irreducible unitary corepresentation $Y$ on $\overline{H}$. 
  Given an operator $T$ between Hilbert spaces $K$ and $L$,  denote by
  $T^{\transpose}:=\overline{T}^{*}$ the transpose, which maps $\overline{L}$ to
  $\overline{K}$. Then
  \begin{align*}
    (\Gr{X}{k}{l}{m}{n})^{*}  = (\Gr{\overline{X}}{l}{k}{n}{m})^{\id \otimes \transpose} =
    ((1 \otimes \Grd{T}{l}{k})^{-1}\cdot \Gr{Y}{l}{k}{n}{m} \cdot (1\otimes \Grd{T}{n}{m}))^{\id \otimes \transpose}
  \end{align*}
  and hence
  \begin{align*}
    P_{klmn} &= ((\Gr{T}{}{-1}{l}{k} \otimes \Gr{T}{}{*}{n}{m})(\phi_{ln} \otimes \id \otimes
    \id)((\Gr{Y}{l}{k}{n}{m})_{12}(\Gr{Y}{l}{k}{n}{m})^{*}_{13})(\Grd{T}{n}{m} \otimes \Gr{T}{}{-*}{l}{k}))^{\transpose\otimes \transpose}.
  \end{align*}
  Now, \eqref{eq:schur-aux-2} for $Y$ instead of $X$ shows that there exists an $R\in
  \mathcal{B}(\Grd{\overline{H}}{n}{m})$ such that 
  \begin{align*}
    (\phi_{ln} \otimes \id \otimes
    \id)((\Gr{Y}{l}{k}{n}{m})_{12}(\Gr{Y}{l}{k}{n}{m})^{*}_{13}) = (\id_{\Grd{\overline{H}}{l}{k}}
    \otimes R)\circ \overline{\Sigma_{mnkl}},
  \end{align*}
  and hence,
  \begin{align*}
    P_{klmn} &= ((\Gr{T}{}{-1}{l}{k} \otimes \Gr{T}{}{*}{n}{m})(\id \otimes
    R)\overline{\Sigma_{mnkl}} (\Grd{T}{n}{m} \otimes \Gr{T}{}{-*}{l}{k}))^{\transpose\otimes
      \transpose} \\
    &= ((\Gr{T}{}{*}{n}{m}R \Grd{T}{n}{m})^{\transpose} \otimes
    (\Gr{T}{}{-1}{l}{k}\Gr{T}{}{-*}{l}{k})^{\transpose})\circ \Sigma_{mnkl}^{*} \\
    &=  ((\Gr{T}{}{*}{n}{m}R \Grd{T}{n}{m})^{\transpose} \otimes \Grd{G}{k}{l}) \circ \Sigma_{klmn}.
  \end{align*}
  Comparing with \eqref{eq:schur-aux-1}, we find that 
  $P_{klmn} = \lambda \cdot (\id_{\Grd{H}{m}{n}} \otimes \Grd{G}{k}{l}) \circ \Sigma_{klmn}$
  for some $\lambda \in \C$.
  
  Now, one finishes the proof similarly like the proof of Theorem 2.22 in \cite{}.
\end{proof}
One can now deduce the following Schur orthogonality relations as in \cite{}:
\begin{Cor} \label{cor:schur-2}
  Let $X$ an irreducible
  unitary corepresentation on a Hilbert space $H$, let $F$ be defined as in Proposition \ref{prop:corep-dual} and
  $G=F^{-1}$, and let $a=(\id \otimes \omega_{v,v'})(\Gr{X}{k}{l}{m}{n})$ and $b=(\id \otimes
  \omega_{w,w'})(\Gr{X}{k}{l}{m}{n})$, where $v,w \in \Grd{H}{k}{l}$ and $v',w' \in
  \Grd{H}{m}{n}$.  Then
  \begin{align*} 
    \phi_{ln}(b^{*}a) &= \frac{\langle w|v'\rangle\langle v|\Grd{G}{m}{n}w'\rangle}{\sum_{r}
      \Tr(\Grd{G}{r}{n})}, &  \phi_{km}(ab^{*}) = \frac{\langle w|\Grd{F}{m}{n}v'\rangle \langle v|w'\rangle}{\sum_{s}
      \Tr(\Grd{F}{m}{s})}.
  \end{align*}
\end{Cor}

\begin{Cor} \label{cor:pw}
  Let $(A,\Delta)$ be a $C^{*}$-partial compact quantum group and let
  $((K^{(x)}, \mathscr{X_{x}}_{x}))_{x \in \mathcal{I}}$ be a maximal
  family of mutually non-isomorphic irreducible corepresentations of
  $(A,\Delta)$.  Then for each $k,l,m,n$, the linear maps
      \begin{align*}
        \label{eq:1}
        \bigoplus_{x\in \mathcal{I}} 
        \overline{\Gr{K}{}{(x)}{k}{l}} \otimes \Gr{K}{}{(x)}{m}{n} &\to
        \Gr{A}{k}{l}{m}{n}, &
        \bigoplus_{x\in \mathcal{I}} 
        \overline{\Gr{K}{}{(x)}{k}{l}} \otimes \Gr{K}{}{(x)}{m}{n} &\to
        \Grd{{(\Gr{H}{k}{l}{}{h})}}{m}{n}, &
      \end{align*}
      given on each summand by $\overline{\eta} \otimes \xi \mapsto (\id \otimes
      \omega_{\eta,\xi})(X_{x})$ and $\overline{\eta} \otimes \xi \mapsto T^{(\eta)}\xi$, respectively,  are injective and
      have dense image.
\end{Cor}
\begin{proof}
For the first map, injectivity   follows easily from Lemma  \eqref{lem:schur-1} and Corollary \ref{cor:schur-2}, and density  of the image from Proposition \ref{prop:matrix-coefficients-dense}. For the second map, note that $T^{(\eta)}\xi = (\id \otimes
      \omega_{\eta,\xi})(X_{x})\zeta^{l}_{n}$.
\end{proof}
 Finally, we find the multiplicity of each irreducible corepresentation in the regular corepresentation. 
\begin{Cor} 
Let $(A,\Delta)$ be a $C^{*}$-pcqg with regular corepresentation $V$. Then for every irreducible unitary corepresentation   $(K,X)$ and each $k,l$, the map
  \begin{align*}
\overline{ \Grd{K}{k}{l}} \to \Mor(X,V^{(kl)}), \quad \eta \mapsto T^{(\eta)} = \sum_{m,n} \Gr{T}{}{(\eta)}{m}{n},
  \end{align*}
  is a linear isomorphism. In particular, $\Mor(X,V^{(kl)})$ has finite dimension.
\end{Cor}
\begin{proof}
  Injectivity of the map follows easily from Corollary \ref{cor:schur-2}.  Let us prove surjectivity.  Since the morphism space
  $\Mor(X,V^{(kl)})$ carries an inner product given by $\langle
  S|T\rangle =\lambda$ iff $S^{*}T=\lambda \id_{K}$, it suffices to  prove that the orthogonal complement of the map $\eta\mapsto
  T^{(\eta)}$ is zero. Now,  suppose that $S \in \Mor(X,V^{(kl)})$ is non-zero. If $Y$ is an  irreducible unitary corepresentation which is not equivalent to $X$, then $S^{*}T=0$    for every $T \in \Mor(Y,V^{(kl)})$. From  Corollary \ref{cor:schur-2}, we deduce that the image of  $\Grd{S}{m}{n}$ must be contained in the span of the images of the  $\Gr{T}{}{(\eta)}{m}{n}$, where $\eta \in \Grd{K}{k}{l}$, and hence  $S^{*}T^{(\eta)}$ is non-zero for some $\eta$.
\end{proof}


%%% Local Variables: 
%%% mode: latex
%%% TeX-master: "dynamical-SUq-file"
%%% End: 
