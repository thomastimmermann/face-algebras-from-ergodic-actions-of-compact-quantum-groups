% Remarks:
% Podles sphere can be defined as an algebra in $\Rep(SU_q(2))$. Hence, it should make sense as an algebra under the forgetful functor to SU_q(2)-dynamical. By duality for coideals, the same should hold for passage to SU_q(1,1) in fact...
% link with article `Racah - Wigner quantum 6j Symbols, Ocneanu Cells for AN diagrams and quantum groupoids' by Coquereaux?
% Thanks to Makoto, Piotr (?), Leonid
% Donin, J.(IL-BILN); Mudrov, A.(IL-BILN), Quantum groupoids and dynamical categories. 
% Leonid, cf. talk www.fields.utoronto.ca/programs/scientific/13-14/.../Vainerman.pdf
% Level 2 Hecke algebras Brundan
% Concerning locally unital algebras ask J. Vercruysse on recent work 
% Say construction groupoids from forgetful functors on temperley-Lieb already in Enock-Ostrik remark
% Say partial compact quantum groups are proper locally compact quantum groupoids with a discrete base set. Can this be made precise? Is e.g. every measured quantum groupoid which has commutative discrete base and which is `proper' (finiteness integrals on components) of this type? Also compare with properness Timmermann.
% Mention generalization of Neshveyev
% Uniformize enumerates and itemizes
% Reference to Larson paper on cosemisimplicity
% Van Daele and Khang reference?
% More direct references to Gaby's paper on comodules? 
% Caenepeel paper on Galois theory weak Hopf algebras.
% Ref to Vainerman?
% Ref to dynamical functor Mudrov

\documentclass[10pt]{article}

\usepackage{hyperref}
\usepackage{fixme}
\usepackage{mathrsfs}

\usepackage[a4paper]{geometry}
\usepackage{amssymb, amsthm, amsfonts, amsxtra, amsmath}
\usepackage{latexsym}
\usepackage{mathabx}
\usepackage{enumitem}
\usepackage[all]{xy}
\usepackage{graphics}
\usepackage{pdfpages}
\usepackage{epic}
\usepackage{fouridx}
\usepackage{parskip} % paragraphs have no indents and vertical spacings inbetween
\usepackage{bbm}


\makeatletter % need this to avoid the conflict between amsthm and parskip
\def\thm@space@setup{%
  \thm@preskip=\parskip \thm@postskip=0pt
}
\makeatother

\DeclareMathOperator{\adj}{\mathrm{adj}}
\DeclareMathOperator{\can}{\mathrm{can}}
\DeclareMathOperator{\Char}{\mathrm{Char}}
\DeclareMathOperator{\dyn}{\mathrm{dyn}}
\DeclareMathOperator{\ext}{\mathrm{e}}
\DeclareMathOperator{\End}{\mathrm{End}}
\DeclareMathOperator{\fin}{\mathrm{fd}}
\DeclareMathOperator{\hol}{\mathrm{hol}}
\DeclareMathOperator{\id}{id}
\DeclareMathOperator{\img}{img}
\DeclareMathOperator{\Ind}{\mathrm{Ind}}
\DeclareMathOperator{\Hom}{Hom}
\DeclareMathOperator{\Ker}{\mathrm{Ker}}
\DeclareMathOperator{\Mat}{\mathscr{M}\!\it{at}}
\DeclareMathOperator{\Matt}{\mathrm{Mat}}
\DeclareMathOperator{\Nat}{\mathrm{Nat}}
\DeclareMathOperator{\op}{\mathrm{op}}
\DeclareMathOperator{\Pol}{\mathrm{P}}
\DeclareMathOperator{\Par}{\mathrm{Par}}
\DeclareMathOperator{\Ran}{\mathrm{Ran}}
\DeclareMathOperator{\rcf}{\mathrm{rcfd}}
\DeclareMathOperator{\rd}{\mathrm{d}}
\DeclareMathOperator{\reg}{\mathrm{reg}}
\DeclareMathOperator{\sgn}{\mathrm{sgn}}
\DeclareMathOperator{\Span}{\mathrm{span}}
\DeclareMathOperator{\Spec}{\mathrm{Spec}}
\DeclareMathOperator{\tr}{\mathrm{tr}}
\DeclareMathOperator{\Zz}{\mathrm{Z}}


\newcommand{\dual}[1]{#1^{*}}
\newcommand{\duall}[1]{#1^{\wedge}}
\newcommand{\dualop}[1]{#1^{\tr}}
\newcommand{\dualco}[1]{\hat{#1}}
\newcommand{\dualcor}[1]{\check{#1}}
\newcommand{\predual}[1]{{^{\vee}\!#1}}
\newcommand{\co}{\mathrm{co}}
\newcommand{\Corep}{\mathrm{Corep}}
\newcommand{\Corepf}{\mathrm{Corep}^{f}}
\newcommand{\sff}{\textrm{s.f.~}}
\newcommand{\sfs}{\mathrm{sfs}}
\newcommand{\sfd}{\mathrm{sfd}}

\newcommand{\Circt}{{\mathop{\ooalign{$\ovoid$\cr\hidewidth\raise-.05ex\hbox{$\scriptstyle\mathsf T\mkern3.5mu$}\cr}}}} % Woronowicz style tensor product, USUAL SIZE
\newcommand{\Circtv}[1]{\underset{#1}{\mathop{\ooalign{$\ovoid$\cr\hidewidth\raise-.05ex\hbox{$\scriptstyle\mathsf T\mkern3.5mu$}\cr}}}} % Woronowicz style tensor product, USUAL SIZE
\newcommand{\smCirct}{\mathop{\ooalign{$\scriptstyle\ovoid$\cr\hidewidth\raise-.05ex\hbox{$\scriptscriptstyle\mathsf T\mkern2.8mu$}\cr}}}  % Woronowicz style tensor product, SCRIPT SIZE

\newcommand{\nc}{\R}
\newcommand{\g}{\mathfrak{g}}
\newcommand{\h}{\mathfrak{h}}

\newcommand{\kk}{\mathfrak{k}}
\newcommand{\ttt}{\mathfrak{t}}
\newcommand{\p}{\mathfrak{p}}
\newcommand{\n}{\mathfrak{n}}
\newcommand{\llll}{\mathfrak{l}}
\newcommand{\uu}{\mathfrak{u}}
\newcommand{\bb}{\mathfrak{b}}
\newcommand{\q}{\mathfrak{q}}
\newcommand{\su}{\mathfrak{su}}
\newcommand{\ssl}{\mathfrak{sl}}
\newcommand{\SSL}{\mathrm{SL}}
\newcommand{\so}{\mathfrak{so}}
\newcommand{\spp}{\mathfrak{sp}}
\newcommand{\G}{\mathbb{G}}
\newcommand{\e}{\mathfrak{e}}
\newcommand{\s}{\mathfrak{s}}
\newcommand{\C}{\mathbb{C}}
\newcommand{\R}{\mathbb{R}}
\newcommand{\Z}{\mathbb{Z}}
\newcommand{\N}{\mathbb{N}}
\newcommand{\X}{\mathbb{X}}
\newcommand{\Y}{\mathbb{Y}}
\newcommand{\Ss}{\mathbb{S}}
\newcommand{\ZZ}{\mathscr{Z}}
\newcommand{\ad}{\mathrm{ad}}
\newcommand{\Hsp}{\mathcal{H}}
\newcommand{\qn}[2]{\lbrack #1 \rbrack_{#2}}
\newcommand{\fqn}[2]{\lbrack #1 \rbrack_{#2}!}
\newcommand{\bqn}[3]{\left\lbrack \begin{array}{c} \!#1\! \\ \!#2\! \end{array}\right\rbrack_{#3}}
\newcommand{\Tr}{\mathrm{Tr}}
\newcommand{\RR}{\mathcal{R}}
\newcommand{\res}{\mathrm{res}}
\newcommand{\cop}{\mathrm{cop}}
\newcommand{\opp}{\mathrm{op}}
\newcommand{\coop}{\mathrm{coop}}
\newcommand{\Rm}{\mathcal{R}}
\newcommand{\wt}{\mathrm{wt}}
\newcommand{\Ad}{\mathrm{Ad}}
\newcommand{\CatC}{\mathcal{C}}
\newcommand{\CatD}{\mathcal{D}}
\newcommand{\CatCC}{\mathscr{C}}
\newcommand{\CatDD}{\mathscr{D}}
\newcommand{\Corr}{\mathrm{Corr}}

\newcommand{\Vectf}{\mathrm{Vect}_{f}}
\newcommand{\Vecti}{\Grd{\mathrm{Vect}}{I}{I}}
\newcommand{\Vectif}{\Gr{\mathrm{Vect}}{I}{I}{}{\fin}}
\newcommand{\Vectrcf}{\Gr{\mathrm{Vect}}{I}{I}{}{\rcf}}
\newcommand{\Hilb}{\mathrm{Hilb}}
\newcommand{\Hilbf}{\mathrm{Hilb}_{\mathrm{f}}}
\newcommand{\Hilbi}{\Grd{\mathrm{Hilb}}{I}{I}}
\newcommand{\Hilbif}{\Gr{\mathrm{Hilb}}{I}{I}{}{\fin}}
\newcommand{\Hilbrcf}{\Gr{\mathrm{Hilb}}{I}{I}{}{\rcf}}

\newcommand{\Star}[2]{{}_{#1}\!*_{#2}}
\newcommand{\vot}{\bar{\otimes}}
\newcommand{\A}{\mathcal{B}}
\newcommand{\Aa}{\mathscr{B}}
\newcommand{\Mor}{\mathrm{Mor}}
\newcommand{\alg}{\mathrm{alg}}
\newcommand{\Gg}{\mathscr{G}}
\newcommand{\ev}{\mathrm{ev}}
\newcommand{\coev}{\mathrm{coev}}
\newcommand{\Rtimes}{\underset{\R}{\times}}
\newcommand{\Rb}{\R^{\bullet}}
\newcommand{\vtimes}{\bar{\otimes}}
\newcommand{\Rr}{\mathscr{R}}
\newcommand{\Tt}{\mathscr{T}}
\newcommand{\Fun}{\mathrm{Fun}}
\newcommand{\Ff}{\Fun_{\fin}}
\newcommand{\itimes}{\underset{I}{\otimes}}
\newcommand{\osum}[1]{\underset{#1}{\sum}^{\oplus}}
\newcommand{\osumc}[1]{\underset{#1}{\sum}^{\bar{\oplus}}}
\newcommand{\oplusc}{\bar{\oplus}}
\newcommand{\wDelta}{\widetilde{\Delta}}
\newcommand{\f}{\mathrm{fin}}
\newcommand{\Rho}{\mathrm{P}}
\newcommand{\Rep}{\mathrm{Rep}}
\newcommand{\DA}{\mathcal{A}}
\newcommand{\even}{\mathrm{even}}
\newcommand{\odd}{\mathrm{odd}}
\newcommand{\fd}{\mathrm{fd}}
\newcommand{\Forget}{F}
\newcommand{\Vect}{\mathrm{Vect}}


\newcommand{\GrHA}[3]{#1{\begin{pmatrix} #2,  #3\end{pmatrix}}}% Horizontal grading ordinary style, with argument
\newcommand{\Grs}[3]{#1{\begin{pmatrix} #2,  #3\end{pmatrix}}}

\newcommand{\GrDA}[3]{{}_{#2}#1_{#3}} % Horizontal grading bottom style, with argument

\newcommand{\GrVA}[3]{#1{\tiny {\begin{pmatrix} #2\\#3\end{pmatrix}}}} % Vertical grading ordinary style, with argument
\newcommand{\Grt}[3]{#1{\tiny {\begin{pmatrix} #2\\#3\end{pmatrix}}}} 
\newcommand{\Grtt}[4]{#1{\tiny {\begin{pmatrix} #2\\#3\\#4\end{pmatrix}}}} 
\newcommand{\pms}[2]{{\tiny {\begin{pmatrix} #1\\#2\end{pmatrix}}}}

\newcommand{\GrRA}[3]{#1^{#2}_{#3}} % Vertical grading right style, with argument

\newcommand{\Unit}{\mathbf{1}}
\newcommand{\Unitb}{\mathbbm{1}}
\newcommand{\UnitC}[2]{\Grt{\mathbf{1}}{#1}{#2}} 
\newcommand{\Grru}[2]{{\tiny \begin{pmatrix} #1 \\ #2\end{pmatrix}}}

\newcommand{\eGr}[5]{#1{{\tiny \begin{pmatrix} #2 \quad #3 \\ #4 \quad #5\end{pmatrix}}}}

\newcommand{\pma}[4]{\begin{pmatrix} #1 \quad #2 \\ #3 \quad #4\end{pmatrix}}
\newcommand{\pmat}[4]{{\tiny \begin{pmatrix} #1 \quad #2 \\ #3 \quad #4\end{pmatrix}}}

\newcommand{\UT}[2]{#1{\tiny #2 }}
\newcommand{\Gr}[5]{\fourIdx{#2}{#4}{#3}{#5}{#1}}%TODO: better typesetting
\newcommand{\Grl}[3]{\Gr{#1}{#2}{}{#3}{}}%TODO: better typesetting
\newcommand{\Gru}[3]{\Gr{#1}{}{}{#2}{#3}}
\newcommand{\Grd}[3]{\Gr{#1}{#2}{#3}{}{}}
\newcommand{\gr}[5]{\;{}^{\;#2}_{#4}#1_{#5}^{#3}}%TODO: better typesetting
\newcommand{\eGrr}[3]{#1_{{\tiny \left(#2, #3\right)}}}
\newcommand{\eGrt}[4]{#1{{\tiny \begin{pmatrix} #2 \\ #3 \\ #4 \end{pmatrix}}}}
\newcommand{\Grr}[4]{\begin{pmatrix}#1 \quad #2\\#3&#4\end{pmatrix}}

\newcommand{\Grss}[3]{\UT{#1}{\begin{pmatrix} #2 \; #3\end{pmatrix}}}
\newcommand{\Grb}[7]{\UT{#1}{\begin{pmatrix} #2\quad #3 \\ #4 \quad #5\\ #6 \quad #7\end{pmatrix}}}
\newcommand{\un}[2]{e{{\tiny \begin{pmatrix}#1\\ #2\end{pmatrix}}}}
\newcommand{\unn}[3]{e{{\tiny \begin{pmatrix}#1\\ #2\\#3\end{pmatrix}}}}

\newcommand{\wmult}{\cdot}
\newcommand{\bmult}{*}
\newcommand{\wmate}{\rightarrow}% Change this to source/target notation l(eft) r(ight)
\newcommand{\bmate}{\downarrow}% Change this to source/target notation u(p) d(own)

\newcommand{\aste}[1]{\underset{#1}{\ast}}

\newcommand{\Vv}{\mathcal{V}}

\newcommand{\dT}{\dot T}

\newtheorem{Theorem}{Theorem}[section]
\newtheorem{Lem}[Theorem]{Lemma}
\newtheorem{Prop}[Theorem]{Proposition}
\newtheorem{Cor}[Theorem]{Corollary}

\theoremstyle{definition}
\newtheorem{Def}[Theorem]{Definition}
\newtheorem{Rem}[Theorem]{Remark}
\newtheorem{Exa}[Theorem]{Example}
\newtheorem{Not}[Theorem]{Notation}
\newtheorem{Que}[Theorem]{Question}
\newtheorem{Con}[Theorem]{Conjecture}

%%%%%%%%%%%%%%%%%%%
% Further notation for Section 1
\newcommand{\phic}[2]{\Grt{\phi}{#1}{#2}}

%%%%%%%%%%%%%%%%%%%
% Notation for Section 6
\newcommand{\LGtwo}{L^{2}(\mathscr{G})}
\newcommand{\LGinf}{L^{\infty}(\mathscr{G})}
\newcommand{\CrG}{C^{r}_{0}(\mathscr{G})}
\newcommand{\CuG}{C^{u}_{0}(\mathscr{G})}
\newcommand{\vnDelta}{\overline{\Delta}}
\newcommand{\vnE}{\overline{E}}
\newcommand{\astrl}{\underset{l^{\infty}(I)}{_{\rho}\ast_{\lambda}}}
\newcommand{\otimesrl}{\underset{\nu{}}{_{\rho}\otimes_{\lambda}}}
\newcommand{\vnphi}{\overline{\phi}}
\newcommand{\vnphic}[2]{\Grt{\vnphi}{#1}{#2}}
\newcommand{\vnR}{\overline{R}}
\newcommand{\vntau}{\overline{\tau}}

\date{}


\numberwithin{equation}{section}

\begin{document}
\title{Partial compact quantum groups}

\author{Kenny De Commer\thanks{Department of Mathematics, Vrije Universiteit Brussel, VUB, B-1050 Brussels, Belgium, email: {\tt kenny.de.commer@vub.ac.be}}
\and Thomas Timmermann\thanks{University of M\"{u}nster}}

\maketitle

\begin{abstract}
\noindent Compact quantum groups of face type, as introduced by Hayashi, form a class of quantum groupoids with a classical, finite set of objects. Using the notions of weak multiplier bialgebra and weak multiplier Hopf algebra (resp.~ due to B{\"o}hm--G\'{o}mez-Torrecillas--L\'{o}pez-Centella and Van Daele--Wang), we generalize Hayashi's definition to allow for an infinite set of objects, and call the resulting objects partial compact quantum groups. We prove a Tannaka-Krein-Woronowicz reconstruction result for such partial compact quantum groups using the notion of partial fusion C$^*$-category. As examples, we consider the dynamical quantum $SU(2)$-groups from the point of view of partial compact quantum groups.
 %We also study their connection to the Lesieur--Enock theory of measured quantum groupoids.
\end{abstract}


%\emph{Keywords}:

%AMS 2010 \emph{Mathematics subject classification}:


%17B37: Quantum groups, quantized enveloping algebras
%20G42: quantized function algebras
%46L65: Functional analysis, deformations, quantizations
%81R50: Quantum groups and related algebraic methods
%16T05: Hopf algebras and their applications
%16T10: Bialgebras
%16T15: Coalgebras and comodules; corings
%46L08  $C^*$-modules
%58B32: Geometry of quantum groups


\tableofcontents


\section*{Introduction}

The concept of \emph{face algebra} was introduced by T. Hayashi in \cite{Hay2}, motivated by the theory of solvable lattice models in statistical mechanics. It was further studied in \cite{Hay1,Hay3,Hay4,Hay5,Hay6,Hay7,Hay8}, where for example associated $^*$-structures and a canonical Tannaka duality were developed. This canonical Tannaka duality allows one to construct a canonical face algebra from any (finite) fusion category. For example, a face algebra can be associated to the fusion category of a quantum group at root unity, for which no genuine quantum group implementation can be found. 

In \cite{Nil1,Sch1,Sch2}, it was shown that face algebras are particular kinds of $\times_R$-algebras \cite{Tak2} and of weak bialgebras \cite{Boh3}. More intuitively, they can be considered as quantum groupoids with a classical, finite object set. In this article, we want to extend Hayashi's theory by allowing an \emph{infinite} (but still discrete) object set. This requires passing from weak bialgebras to weak \emph{multiplier} bialgebras \cite{Boh1}. At the same time, our structures admit a piecewise description by what we call a \emph{partial bialgebra}, which is more in the spirit of Hayashi's original definition. In the presence of an antipode, an invariant integral and a compatible $^*$-structure, we call our structures \emph{partial compact quantum groups}. 

The passage to the infinite object case is delicate at points, and requires imposing the proper finiteness conditions on associated structures. However, once all conditions are in place, many of the proofs are similar in spirit to the finite object case. % Rephrase? Don't downplay too much?

Our main result is a Tannaka-Krein-Woronowicz duality result which states that partial compact quantum groups are in one-to-one correspondence with \emph{concrete partial fusion C$^*$-categories}. In its essence, a partial fusion C$^*$-category is a multifusion C$^*$-category \cite{ENO1}, except that (in a slight abuse of terminology) we allow an infinite number of irreducible objects as well as an infinite number of summands inside the unit object. By a \emph{concrete} multifusion C$^*$-category, we mean a multifusion C$^*$-category realized inside a category of (locally finite dimensional) bigraded Hilbert spaces. Of course, Tannaka reconstruction is by now a standard procedure. % Almost standard procedure?
For closely related results most relevant to our work, we mention \cite{Wor2,Sch3,Hay8,Ost1,Hai1,Szl1,Pfe1,DCY1,Nes1} as well as the surveys \cite{JoS1} and \cite[Section 2.3]{NeT1}.

As an application, we generalize Hayashi's Tannaka duality \cite{Hay8} (see also \cite{Ost1}) by showing that any module C$^*$-category over a multifusion C$^*$-category has an associated canonical partial compact quantum group. By the results of \cite{DCY1}, such data can be produced from ergodic actions of compact quantum groups. In particular,  we consider the case of ergodic actions of $SU_q(2)$ for $q$ a non-zero real. This will allow us to show that the construction of \cite{Hay4} generalizes to produce partial compact quantum group versions of the dynamical quantum $SU(2)$-groups of \cite{EtV1,KoR1}, see also \cite{Sto1}. This construction will immediately provide the right setting for the operator algebraic versions of these dynamical quantum $SU(2)$-groups, which was the main motivation for writing this paper. These operator algebraic details will however be studied elsewhere \cite{DCT2}.

The precise layout of the paper is as follows.

The \emph{first section} introduces the basic theory of the structures which we will be concerned with in this paper. We introduce the notions of \emph{partial bialgebra}, \emph{partial Hopf algebra} and \emph{partial compact quantum group}, and show how they are related to the notion of weak multiplier bialgebra \cite{Boh1}, weak multiplier Hopf algebra \cite{VDW1,VDW2} and compact quantum group of face type \cite{Hay1}. We also introduce the corresponding notions of \emph{partial tensor category} and \emph{partial fusion C$^*$-category}. 

In the next two sections, our main result is proven, namely the Tannaka-Krein-Woronowicz duality. In the \emph{third section} we develop the corepresentation theory of partial Hopf algebras and the representation theory of partial compact quantum groups, and we show that the latter allows one to construct a concrete partial fusion C$^*$-category. In the \emph{fourth} section, we show conversely how any concrete partial fusion C$^*$-category allows one to construct a partial compact quantum group, and we briefly show how the two constructions are inverses of each other.

In the final two sections, we provide some examples of our structures and applications of our main result. In the \emph{fifth section}, we first consider the construction of a canonical partial compact quantum group from any partial module C$^*$-category for a partial fusion C$^*$-category. We then introduce the notions of \emph{Morita}, \emph{co-Morita} and \emph{weak Morita equivalence} \cite{Mug1} of partial compact quantum groups, and show that two partial compact quantum groups are weakly Morita equivalent if and only if they can be connected by a string of Morita and co-Morita equivalences. In the \emph{sixth section}, we study in more detail a concrete example of a canonical partial compact quantum group, constructed from an ergodic action of quantum $SU(2)$. In particular, we obtain a partial compact quantum group version of the dynamical quantum $SU(2)$-group. 

\emph{Note}: we follow the physicist's convention that inner products on Hilbert spaces are anti-linear in their \emph{first} argument. 


\section{Partial compact quantum groups}

We generalize Hayashi's definition of a compact quantum group of face type \cite{Hay1} to the case where the commutative base algebra is no longer finite-dimensional. We will present two approaches, based on \emph{partial bialgebras} and \emph{weak multiplier bialgebras} \cite{Boh1,VDW1}. The first approach is piecewise and concrete, but requires some bookkeeping. The second approach is global but more abstract. As we will see from the general theory and the concrete examples, both approaches have their intrinsic value.

\subsection{Partial algebras and partial coalgebras}

%Let $I$ be a set. We consider $I^2=I\times I$ as the pair groupoid with $\wmult$ denoting composition. That is, an element $(k,l)\in I^2$ has source $k$ and target $l$, and $(k,l)\cdot (l,m)= (k,m)$. 

\begin{Def} An \emph{$I$-partial algebra} $\mathscr{A}=(\mathscr{A},M)$ (over $\C$) is a set $I=\{k,l,\cdots\}$ (the \emph{object} set) together with vector spaces $\GrDA{A}{k}{l}$ (allowed to be $\{0\}$), multiplication maps \[M=M_{k,l,m}:\GrDA{A}{k}{l} \otimes \GrDA{A}{l}{m}\rightarrow \GrDA{A}{k}{m},\qquad a\otimes b \mapsto ab\]  and \emph{unit elements} $\Unit(k) = \Unit_k \in \Grs{A}{k}{k}$ (allowed to be zero),
such that the obvious associativity and unit conditions are satisfied. 
\end{Def}

%\begin{Rem}
%\begin{enumerate}\item It will be important to allow the local units $\Unit_k$ to be zero.
%\item A partial algebra is by definition the same as a small $\C$-linear category. However, we do not emphasize this viewpoint, as the natural notion of morphism for partial algebras will be \emph{contravariant} on objects, see Definition \ref{DefMor}.% Is this correct, or are local units = 0 not allowed? 
%\end{enumerate}
%\end{Rem}

By extending $M$ as the zero map to tensor products on which it is not defined, we can define an associative algebra structure on $A =  \oplus_{k,l\in I} \GrDA{A}{k}{l}$, which we call the \emph{total algebra}.  It is a locally unital algebra by the orthogonal idempotents $\mathbf{1}_k$. One can identify $A$ with finite support $I$-indexed matrices $(a_{kl})_{k,l}$ with $a_{kl} \in \GrDA{A}{k}{l}$, equipped with the natural matrix multiplication. 

For example, for any set $I$ we can define a partial algebra $\Mat_I$ with $\GrDA{\Matt}{k}{l} = \C$ for all $k,l$ and each $M_{k,l,m}$ scalar multiplication. The associated total algebra is the algebra of all finitely supported matrices based over $I$.

We have a natural notion of tensor product for partial algebras, where $\GrDA{(A\otimes B)}{(k,k')}{(l,l')} = \GrDA{A}{k}{l}\otimes \GrDA{B}{k'}{l'}$, indexed by the Cartesian product of the object sets. 

Working with non-unital algebras necessitates the use of their \emph{multiplier algebra} \cite{Dau1,VDae1}. Recall that a multiplier $m$ for an algebra $A$ consists of a couple of maps $L_m, R_m\in \End(A)$ such that $(am)b = a(mb)$ for all $a,b\in A$. They form an algebra, the \emph{multiplier algebra} $M(A)$, under composition for the $L$-maps and anti-composition for the $R$-maps. One has a natural homomorphism $A\rightarrow M(A)$ which is injective when $A$ has local units. More precisely, when $A$ is the total algebra of an $I$-partial algebra, we can identify $M(A)$ with matrices $(a_{kl})_{kl}$ which have finite support in either one of the variables when the other variable has been fixed. The multiplication is matrix multiplication, $(mn)_{kl} = \sum_p m_{kp}n_{pl}$. Whenever $m_i$ is a collection of multipliers of $A$, such that for each $a\in A$ one has $m_ia =0=am_i$ for all but a finite set of $i$, then one can define a multiplier $\sum_i m_i$ in the obvious way. One says that the sum $\sum_i m_i$ converges in the \emph{strict} topology. 
 
The finiteness condition characterising multipliers as above will appear time and again, so we formalize it in the next definition. 
 
\begin{Def} An assignment $(k,l)\rightarrow m_{kl}$ into a set with distinguished zero element is called \emph{row-and column-finite} (rcf) if it has finite support in either one of the variables when the other variable has been fixed. 
\end{Def} 

%Let us comment on the notion of morphism for partial algebras. We first introduce the piecewise definition.

The following definition provides the notion of morphism for partial algebras. We denote by $\mathscr{P}(I)$ the power set of a set $I$.

\begin{Def}\label{DefMor} Let $\mathscr{A}$ and $\mathscr{B}$ be respectively $I$ and $J$-partial algebras. Let $\phi: I \ni k \mapsto J_k \in \mathscr{P}(J)$ with the $J_k$ disjoint. A $\phi$-\emph{homomorphism} from $\mathscr{A}$ to $\mathscr{B}$ consists of linear maps $f_{rs}: \GrDA{A}{k}{l}\rightarrow \GrDA{B}{r}{s}$ for all $r\in J_k, s\in J_l$ such that the applications $(r,s) \mapsto f_{rs}(a)$ is pointwise rcf on $J_k\times J_l$ for each $a\in \GrDA{A}{k}{l}$, and such that $f_{rs}(\Unit_{k}) = \delta_{rs}\Unit_r$ and $f_{rt}(ab) = \sum_{s\in J_l} f_{rs}(a)f_{st}(b)$ for all $a\in \GrDA{A}{k}{l}, b\in \GrDA{A}{l}{m}, r\in J_k$ and $t\in J_m$.
\end{Def}

Note that the above sum is meaningful because of the rcf condition.  It is easy to see that a $\phi$-homomorphism $\mathscr{A}\rightarrow \mathscr{B}$ is nothing but a homomorphism $f:A\rightarrow M(B)$ with $f(\Unit_k) = \sum_{r\in J_k} \Unit_r$. 

\subsection{Partial coalgebras and partial bialgebras}

The notion of partial algebra nicely dualizes, one of the main benefits of the local approach. 
%For this we consider again $I^2$ as the pair groupoid, but now with elements considered as column vectors, and with $\bmult$ denoting the (vertical) composition. So $K=\Grt{}{k}{l}$ has source $K_u = k$ and target $K_d = l$, and if $K=\Grt{}{k}{l}$ and $L=\Grt{}{l}{m}$ then $K\bmult L = \Grt{}{k}{m}$.

\begin{Def}\label{DefCoAlg} An  $I$-\emph{partial coalgebra} $\mathscr{A}=(\mathscr{A},\Delta)$ (over $\C$) consists of a set $I=\{k,l,\ldots\}$ (the object set) together with vector space $\GrRA{A}{k}{l}$, comultiplication maps \[\Delta_{l} = \Grtt{\Delta}{k}{l}{m}:\GrRA{A}{k}{m}\rightarrow \GrRA{A}{k}{l}\otimes \GrRA{A}{l}{m},\qquad a \mapsto a_{(l;1)}\otimes a_{(l;2)},\] and counit maps $\epsilon =\epsilon_k:\GrRA{A}{k}{k}\rightarrow \C$ satisfying the obvious coassociativity and counitality conditions.
\end{Def}

We extend $\epsilon$ as the zero functional on $\GrRA{A}{k}{l}$ when $k\neq l$.

The notions of partial algebra and partial coalgebra can be superimposed in the following way.  We will write 
\[\phi_{\Delta}: I^2 \rightarrow \mathscr{P}(I^2\times I^2),\quad (I^2\times I^2)_{{\tiny \begin{pmatrix} k\\m \end{pmatrix}}} = \{\left(\pms{k}{l},\pms{l}{m}\right)\mid l\in I\},\] \[\phi_{\epsilon}: I^2 \rightarrow \mathscr{P}(I),\quad I_{{\tiny \begin{pmatrix} k\\m \end{pmatrix}}} = \left\{\begin{array}{lll} \{k\}& \textrm{if } k= l,\\ \emptyset & \textrm{if } k\neq l. \end{array}\right.\]

\begin{Def}\label{DefPartBiAlg} A \emph{partial bialgebra} $\mathscr{A}=(\mathscr{A},M,\Delta)$ consists of a set $I$ (the \emph{object set}) and a collection of vector spaces $\Gr{A}{k}{l}{m}{n}$ with $I^2$-partial algebra structure on $\GrDA{A}{\pms{k}{l}}{\pms{m}{n}} = \Gr{A}{k}{l}{m}{n}$ and $I^2$-partial coalgebra structure $(\Delta_{rs})$ on $\GrRA{A}{(k\;l)}{(m\;n)}=\Gr{A}{k}{l}{m}{n}$, such that $\Delta$ is a $\phi_{\Delta}$-homomorphism $\mathscr{A}\rightarrow \mathscr{A}\otimes \mathscr{A}$ and $\epsilon$ is a $\phi_{\epsilon}$-homomorphism $\mathscr{A}\rightarrow \Mat_I$.\end{Def}

For example, this gives that the comultiplication gives maps \[\Delta_{rs}: \Gr{A}{k}{l}{m}{n}\rightarrow \Gr{A}{k}{l}{r}{s}\otimes \Gr{A}{r}{s}{m}{n},\quad a\mapsto a_{(rs;1)}\otimes a_{(rs;2)}\] such that the maps $(r,s) \mapsto \Delta_{rs}(a)$ are rcf and satisfy the multiplicativity property $\Delta_{rs}(ab) = \sum_t \Delta_{rt}(a)\Delta_{ts}(b)$ for any $a\in \Gr{A}{k}{l}{m}{n}, b\in \Gr{A}{l}{p}{n}{q}$. The accompanying total map $\Delta: A\rightarrow M(A\otimes A)$ as the \emph{total comultiplication} of $\mathscr{A}$, and we use for it the ordinary Sweedler notation $\Delta(a) = a_{(1)}\otimes a_{(2)}$.  Similarly, $\epsilon(\UnitC{k}{k})=1$ and $\epsilon(ab) = \epsilon(a)\epsilon(b)$ for $a,b$ of the above form. 

We relate the notion of partial bialgebra to the recently introduced notion of regular weak multiplier bialgebra \cite[Definition 2.1 and Definition 2.3]{Boh1}. For  $\mathscr{A}$ an $I$-partial bialgebra we write \[\lambda_k = \sum_l \UnitC{k}{l},\qquad \rho_l = \sum_k\UnitC{k}{l} \qquad \in M(A).\]  Then $E= \sum_l \rho_l\otimes \lambda_l$ is a well-defined idempotent in $M(A\otimes A)$, and satisfies $\Delta(A)(A\otimes A)=E(A\otimes A)$ and $(A\otimes A)\Delta(A)= (A\otimes A)E$. 
By \cite[Proposition A.3]{VDW2}, there is a unique homomorphism $\Delta:M(A)\rightarrow M(A\otimes A)$ extending $\Delta$ and such that $\Delta(1) = E$. It is simply the continuous extension of $\Delta$ to $M(A)$ for the strict topology. Similarly the maps $\id\otimes \Delta$ and $\Delta\otimes \id$ extend to maps from $M(A\otimes A)$ to $M(A\otimes A\otimes A)$. 

%The following proposition gathers the properties of $\Delta$, $\epsilon$ and $\Delta(1)$ which guarantee that $(A,\Delta)$ forms a regular weak multiplier bialgebra in the sense of \cite{Boh1}.

\begin{Prop} Let $\mathscr{A}$ be a partial bialgebra with total algebra $A$, total comultiplication $\Delta$ and counit $\epsilon$. Then $(A,\Delta,\epsilon,\Delta(1)$ is a regular weak multiplier bialgebra.
\end{Prop}

\begin{proof} The only property which is not clear on sight is the weak multiplicativity condition \[(\epsilon\otimes \id)(\Delta(a)(b\otimes c)) = (\epsilon\otimes \id)((1\otimes a)\Delta(1)(b\otimes c)),\qquad \forall a,b,c\in A,\] and its symmetric companion. But it follows from a straightforward computation using that $\epsilon$ is a partial homomorphism.  
\end{proof} 

We will call $(A,\Delta,\epsilon,\Delta(1))$ the \emph{total weak multiplier bialgebra} associated to $\mathscr{A}$.

Recall from \cite[Section 3]{Boh1} that a regular weak multiplier
bialgebra admits four projections $A\rightarrow M(A)$, given
by \begin{align*} \bar{\Pi}^L(a) = (\epsilon\otimes \id)((a\otimes
  1)\Delta(1)),\quad & \bar{\Pi}^R(a) = (\id\otimes
  \epsilon)(\Delta(1)(1\otimes a)),\\ \Pi^L(a) = (\epsilon\otimes
  \id)(\Delta(1)(a\otimes 1)),\quad& \Pi^R(a) =
  (\id\otimes\epsilon)((1\otimes a)\Delta(1)),\end{align*} where the
right hand side expressions are interpreted as multipliers in the
obvious way. Hence if $a\in \Gr{A}{k}{l}{m}{n}$, one has \[ \Pi^L(a) = \epsilon(a)\lambda_m,\quad \bar{\Pi}^L(a) = \epsilon(a) \lambda_n, \quad \Pi^R(a) = \epsilon(a) \rho_l,\quad \bar{\Pi}^R(a) = \epsilon(a)\rho_k.\]

The \emph{base algebra} of $(A,\Delta)$ is therefore just the algebra
$\Fun_{f}(I)$ of finite support functions on $I$, and the
comultiplication of $A$ is (left and right) \emph{full} (meaning that the legs of $\Delta(A)$ span $A$) by \cite[Theorem 3.13]{Boh1}.  

Let us now show a converse. If $(A,\Delta)$ is a regular weak multiplier bialgebra, let us write $A^L = \Pi^L(A) = \bar{\Pi}^L(A)\subseteq M(A)$ and $A^R = \Pi^R(A)= \bar{\Pi}^R(A)\subseteq M(A)$ for the base algebras. If moreover $(A,\Delta)$ is full, we have that $A^L$ is anti-isomorphic to $A^R$ by the map \[\sigma: A^L \rightarrow A^R, \quad \bar{\Pi}^L(a) \rightarrow \Pi^R(a), \qquad a\in A,\] by \cite[Lemma 4.8]{Boh1}. We then refer to $A^L$ as \emph{the} base algebra. Note that one could also have used the map $\bar{\sigma}(\Pi^L(a)) = \bar{\Pi}^R(a)$ to identify $A^L$ and $A^R$. However, $\bar{\sigma}^{-1}\sigma$ is the (unique) Nakayama automorphism for some functional $\varepsilon$ on $A^L$, cf.~ \cite[Proposition 4.9]{Boh1}, hence $\sigma = \bar{\sigma}$ if $A^L$ is commutative.


\begin{Prop}\label{PropCharPBA} Let $(A,\Delta)$ be a full regular weak multiplier bialgebra whose base algebra is isomorphic to $\Fun_f(I)$ for some set $I$, and such that moreover $A^LA^R \subseteq A$. Then $(A,\Delta)$ is the total weak multiplier bialgebra of a uniquely determined partial bialgebra $\mathscr{A}$ over $I$.
\end{Prop} 

The condition $A^LA^R \subseteq A$ is of course essential. It should be considered as a \emph{properness} condition. % Ref to Timmermann?


\begin{proof} Write $\lambda_k \in A^L$ for the function $\lambda_k(l) = \delta_{kl}$, and write $\sigma(\lambda_k) = \rho_k\in A^R$. By assumption, $\UnitC{k}{l} = \lambda_k\rho_l\in A$. Further $A= AA^R = AA^L = A^LA=A^RA$, cf.~ the proof of \cite[Theorem 3.13]{Boh1}. Hence the $\UnitC{k}{l}$ make $A$ into the total algebra of an $I\times I$-partial algebra, as $A^L$ and $A^R$ pointwise commute by \cite[Lemma 3.5]{Boh1}. 

Define \[\Delta_{rs}(a) = (\rho_r\otimes \lambda_r)\Delta(a)(\rho_s\otimes \lambda_s).\] From \cite[Lemma 3.3]{Boh1}, it follows that $\Delta_{rs}$ is a map from $\Gr{A}{k}{l}{m}{n}$ to $\Gr{A}{k}{l}{r}{s}\otimes \Gr{A}{r}{s}{m}{n}$. That same lemma, together with the coassociativity of $\Delta$, show that the $\Delta_{rs}$ form a coassociative family.  

Now by \cite[Lemma 3.9]{Boh1}, we have $(\rho_k\otimes 1)\Delta(a) = (1\otimes \lambda_k)\Delta(a)$ for all $a$. By that same lemma and the remark preceding our proof, we have as well $\Delta(a)(\rho_k\otimes 1) = \Delta(a)(1\otimes \lambda_k)$. Hence we can write \begin{eqnarray*} \Delta_{rs}(a) &=& (\rho_r\otimes 1)\Delta(a)(\rho_s\otimes 1) \\ &=& (1\otimes \lambda_r)\Delta(a)(1\otimes \lambda_s)\end{eqnarray*}  It is now straightforward that the counit map of $(A,\Delta)$ provides a counit for the $\Delta_{rs}$. Hence the $\Gr{A}{k}{l}{m}{n}$ form a partial coalgebra. 

As $\Delta(a)(1\otimes \lambda_s)$ and $(1\otimes \lambda_r)\Delta(a)$ are already in $A\otimes A$, it is also clear that $\Delta_{rs}(a)$ is rcf for each $a$. The multiplicativity of the $\Delta_{rs}$ is immediate from the multiplicativity of $\Delta$.

To show that $\Delta_{ll'}(\UnitC{k}{m}) = \delta_{l,l'} \UnitC{k}{l}\otimes \UnitC{l}{m}$, it suffices to show that $\Delta(1) = \sum_k \rho_k\otimes \lambda_k$. As $\Delta(1)(A\otimes A)  = \Delta(A)(A\otimes A)$, and as $\Delta(a) = \sum_{r,s}\Delta_{rs}(a)$ in the strict topology for all $a\in A$, it follows that $\Delta(1) = \left(\sum_k \rho_k\otimes \lambda_k\right)\Delta(1)$, and similarly $\Delta(1) = \Delta(1)\left(\sum_k\rho_k\otimes \lambda_k\right)$. On the other hand, by \cite[Lemma 4.10]{Boh1} we can write $\Delta(1)=\sum_{k\in I'} \rho_k\otimes \lambda_k$ for some subset $I'\subseteq I$. As by definition $\bar{\Pi}^L(A) = \Fun_{\fin}(I)$, we deduce that $I=I'$. 

For $a\in \Gr{A}{k}{l}{p}{q}$ and $b\in \Gr{A}{l}{m}{q}{r}$, we then have $\epsilon(ab) = \epsilon(a\UnitC{l}{q}b) = \epsilon(a)\epsilon(b)$ by \cite[Proposition 2.6.(4)]{Boh1}, which shows the partial multiplicativity of $\epsilon$. 

Finally, assume that $k$ was such that $\epsilon(\UnitC{k}{k})=0$. By the partial multiplication law, $\epsilon$ is zero on all $\Gr{A}{k}{l}{k}{l}$. Applying $\Delta_{kl}$ to $\Gr{A}{k}{l}{m}{n}$ and using the counit property on the first leg, it follows that $\Gr{A}{k}{l}{m}{n}=0$ for all $l,m,n$. In particular, $\UnitC{k}{m}=0$ for all $m$. This entails $\lambda_k=0$, a contradiction. Hence $\epsilon(\UnitC{k}{k})\neq 0$. From the partial multiplication law it follows that $\epsilon(\UnitC{k}{k})^2 = \epsilon(\UnitC{k}{k})$, hence $\epsilon(\UnitC{k}{k})=1$.

This concludes the proof that $(A,\Delta)$ determines a partial bialgebra $\mathscr{A}$. It is immediate that $(A,\Delta)$ is in fact the total weak multiplier bialgebra of $\mathscr{A}$. 
\end{proof} 

% Mention precisely link to Hayashi

\subsection{Partial Hopf algebras}

%We now formulate the notion of partial Hopf algebra, whose total form will correspond to a weak multiplier Hopf algebra \cite{Boh1,VDW2,VDW1}. We will mainly refer to \cite{Boh1} for uniformity.

% Let us denote $\circ$ for the inverse of $\wmult$, and $\bullet$ for the inverse of $\bmult$, so \[\begin{pmatrix} k & l \\ m & n \end{pmatrix}^{\circ} = \begin{pmatrix} l & k \\ n & m \end{pmatrix},\quad \begin{pmatrix} k & l \\ m & n \end{pmatrix}^{\bullet} = \begin{pmatrix} m & n \\ k & l \end{pmatrix},\quad \begin{pmatrix} k & l \\ m & n \end{pmatrix}^{\circ \bullet} = \begin{pmatrix} n & m \\ l & k \end{pmatrix}.\] The notation $\circ$ (resp. $\bullet$) will also be used for row vectors (resp. column vectors).

\begin{Def}\label{DefPartBiAlgAnt}A partial bialgebra $\mathscr{A}$ is called a \emph{partial Hopf algebra} if it admits an \emph{antipode}, that is, a collection of linear
maps $S:\Gr{A}{k}{l}{m}{n}\rightarrow \Gr{A}{n}{m}{l}{k}$ such that $\sum_s a_{(rs;1)}S(a_{rs;(2)})= \epsilon(a)\UnitC{k}{r}$ and $\sum_r S(a_{(rs;1)})a_{(rs;2)}= \epsilon(a)\UnitC{s}{n}$ for all $a \in \Gr{A}{k}{l}{m}{n}$.
   
\end{Def} 

%\begin{Rem} Note that condition \ref{Propd} of Definition \ref{DefPartBiAlg} again guarantees that the above sums are in fact finite.
%\end{Rem}

An antipode can be linearly extended to a map $S: A\rightarrow A$, which then satisfies the defining relations $a_{(1)}S(a_{(2)}) = \Pi^{L}(a)$ and $
 S(a_{(1)})a_{(2)} = \Pi^{R}(a)$. These identities imply immediately that $S(\UnitC{k}{l}) = \UnitC{l}{k}$ for all $k,l\in I$.
%\end{Lem}
%\begin{proof} For example the first identity in Equation \eqref{eq:total-antipode} of Lemma \ref{lemma:antipode} applied to $\UnitC{k}{k}$ gives \[\sum_l S(\UnitC{l}{k}) = \sum_l \UnitC{k}{l}S(\UnitC{l}{k}) = \lambda_k,\] as $S(\UnitC{l}{k}) \in \Gr{A}{k}{k}{l}{l}$ and $\Pi^{L}(\UnitC{k}{k}) = \lambda_k$. This implies the lemma.
%\end{proof} 
\begin{Def} \label{remark:index-equivalence}
The \emph{hyperobject} set of a partial Hopf algebra $\mathscr{A}$ is the set of equivalence classes in $I$ for the equivalence relation $
    k \sim l \Leftrightarrow \UnitC{k}{l} \neq 0$.
\end{Def} 


The existence of an antipode is closely related to partial invertibility of
the maps $T_{1},T_{2} \colon A \otimes A \to A\otimes A$ given by
\begin{align} \label{eq:wt-12}
  T_{1} (a\otimes b)&= \Delta(a)(1 \otimes b), &
  T_{2} (a\otimes b)&= (a \otimes 1)\Delta(b).
 \end{align}
Namely, for any partial bialgebra we can define linear maps $E_{i},G_{i}
 \colon A\otimes A\to A\otimes A$ by
\begin{align} \label{eq:e1g1}
  G_{1}(a\otimes b) &=
 \sum_{p} a\rho_{p} \otimes \rho_{p}b, &  E_{1}(a \otimes b) &=\Delta(1)(a\otimes b)=\sum_{p} \rho_{p}a\otimes \lambda_{p}b, \\ \label{eq:e2g2}
 G_{2}(a \otimes b) &= \sum_{p} a\lambda_{p} \otimes
    \lambda_{p}b, &
E_{2}(a\otimes b) &= (a\otimes b)\Delta(1)=\sum_{p} a\rho_{p} \otimes b\lambda_{p}.
\end{align}

Then if $\mathscr{A}$ is a partial Hopf algebra with antipode $S$, the maps
  $R_{1},R_{2} \colon A \otimes A \to M(A \otimes A)$ given by
  \begin{align*}
    R_{1}(a \otimes b) &= a_{(1)}\otimes S(a_{(2)})b, &
    R_{2}(a\otimes b) &= aS(b_{(1)})\otimes b_{(2)}
  \end{align*}
  take values in $A\otimes A$ and satisfy for $i=1,2$ the relations
  \begin{align} \label{eq:riti}
    T_{i}R_{i}&=E_{i}, & R_{i}T_{i}&= G_{i}, & T_{i}R_{i}T_{i}&= T_{i}, & R_{i}T_{i}R_{i} &= R_{i},
  \end{align}
by a straightforward calculation using the definition of antipode.


\begin{Theorem}  \label{theorem:partial-hopf-algebra}
  Let $\mathscr{A}$ be a partial bialgebra. Then the
  following conditions are equivalent:
  \begin{enumerate}[label={(\arabic*)}]
  \item\label{tph1} $\mathscr{A}$ is a partial Hopf algebra.
 % \item\label{tph2} There exist linear maps $R_{1},R_{2} \colon A\otimes A\to
    %A\otimes A$ satisfying  \eqref{eq:riti}.
  \item\label{tph3} $(A,\Delta,\epsilon)$  is a weak multiplier Hopf algebra in the sense of \cite{VDW1}.
  \end{enumerate}
  If these conditions hold, then the total  antipode of $\mathscr{A}$ coincides with the antipode of $(A,\Delta,\epsilon)$.
\end{Theorem}
\begin{proof}
If \ref{tph1} holds, then the existence of linear maps $R_{1},R_{2} \colon A\otimes A\to A\otimes A$ satisfying  \eqref{eq:riti} implies \ref{tph3} by Definition
1.14 in \cite{VDW1}, as the maps $G_{1},G_{2}$ coincide with the maps introduced in Proposition 1.14 of \cite{VDW1}.  Conversely, assume \ref{tph3} holds.
 Then Lemma 6.14 and equation (6.14) of \cite{Boh1} imply that the antipode
$S$ of $(A,\Delta)$ satisfies $S(\Gr{A}{k}{l}{m}{n})\subseteq \Gr{A}{n}{m}{l}{k}$ and relation \eqref{eq:total-antipode}, hence \ref{tph1} holds.
\end{proof}

\begin{Cor} \label{corollary:antipode} Let $\mathscr{A}$ be a partial
  Hopf algebra. Then the total antipode $S:A\rightarrow A$ is uniquely determined and satisfies
  $S(ab) = S(b)S(a)$ and $\Delta(S(a)) = (S\otimes S)\Delta^{\op}(a)$
  for all $a,b\in A$.
\end{Cor} 
As usual, $\Delta^{\op}$ denotes  the composition of $\Delta$ with the flip map.
\begin{proof} This follows from \cite[Proposition 3.5 and Proposition 3.7]{VDW1} or \cite[Theorem
6.12 and Corollary 6.16]{Boh1}. Uniqueness of the antipode follows from the identities \eqref{eq:total-antipode}, see also \cite[Remark 2.8.(ii)]{VDW1}. 
\end{proof} 

%We will need the following relation between $\epsilon$ and $S$ at some point.

\begin{Lem}\label{LemCoAnt} Let $(\mathscr{A},\Delta)$ be a partial Hopf algebra. Then $\epsilon\circ S = \epsilon$ on each $\Gr{A}{k}{l}{m}{n}$.
\end{Lem}

\begin{proof} Using the notation in Proposition \ref{prop:riti} and the discussion preceding it, we have that \[T_1: \sum_p(A\rho_p\otimes \rho_p A)\rightarrow \Delta(1)(A\otimes A)\] is a bijection with $R_1$ as inverse. As one easily verifies that $(\id\otimes \epsilon)T_1 = \id\otimes \epsilon$ by the partial multiplicativity and counit property of $\epsilon$, it follows that also $(\id\otimes \epsilon)R_1 = \id\otimes \epsilon$ on $\Delta(1)(A\otimes A)$. Applying both sides to $a\otimes \UnitC{k}{k}$ with $a\in \Gr{A}{k}{l}{k}{l}$, we find \[(\id\otimes (\epsilon\circ S))\Delta_{kl}(a) = a.\] Applying $\epsilon$ to this identity, we find $\epsilon\circ S = \epsilon$ on each $\Gr{A}{k}{l}{k}{l}$, and hence on all $\Gr{A}{k}{l}{m}{n}$.
\end{proof} 

%In practice, it is convenient to have an \emph{invertible} antipode around. Although the invertibility often comes for free in case extra structure is around, we will mostly just impose it to make life easier. The following definition follows the terminology of \cite{VDae1}. 

\begin{Def} Let $\mathscr{A}$ be a partial Hopf algebra. We call $\mathscr{A}$ a \emph{regular} partial Hopf algebra if the antipode maps on $\mathscr{A}$ are invertible.
\end{Def}

From the uniqueness of the antipode, it follows immediately that $S^{-1}$ is then an antipode for $(\mathscr{A},\Delta^{\op})$. Conversely, if both $(\mathscr{A},\Delta)$ and $(\mathscr{A},\Delta^{\op})$ have antipodes, then $(\mathscr{A},\Delta)$ is a regular partial Hopf algebra. 

\subsection{Invariant integrals}

% Mention precisely link to Hayashi

\begin{Def}
  Let $\mathscr{A}$ be an $I$-partial bialgebra.  We call a family of
  functionals
\begin{align} \label{eq:functionals}
  \phi = \phic{k}{m} \colon A\pmat{k}{k}{m}{m} \to \C
\end{align}
an \emph{invariant} \emph{integral} if
 $\phi(\UnitC{k}{k})=1$ for all $k\in
I$ and
\begin{align}
  \label{eq:integral}
   (\id \otimes \phi)(\Delta_{rr}(a)) 
&= \delta_{k,l} \phi(a)
  \UnitC{k}{r}, &   (\phi \otimes
  \id)(\Delta_{rr}(a))&= \delta_{m,n} \phi(a) \UnitC{r}{m}
\end{align}
 for all $k,l,m,n,r\in I$ and $a \in A\pmat{k}{l}{m}{n}$. 
\end{Def}

As before, we interpret $\phi$ as the zero functional on the parts on which it is not defined. One easily checks that the linear extension of $\phi$ to $A$ satisfies the total invariance conditions \begin{align*}
(\id\otimes \phi)((b\otimes 1)\Delta(a)) &= \sum_{k}\phi(\lambda_{k}a)b\lambda_k,&  (\phi\otimes \id)(\Delta(a)(1\otimes b)) &= \sum_{n}
\phi(\rho_{n} a)\rho_n b.\end{align*}

Note that $\phi(\UnitC{k}{m})=1$ for all $k,m\in I$ with $\UnitC{k}{m}\neq 0$, by applying $(\id\otimes \phi)$ to $\Delta_{kk}(\UnitC{m}{m})$. One further has that $\phi$ is uniquely determined, since any other invariant function $\psi$ satisfies \begin{align*}  \phi(a)  &= (\psi \otimes
      \phi)(\Delta_{kk}(a)) = \psi(a)\phi(\UnitC{k}{m}) = \psi(a) .
    \end{align*}
Hence if $\mathscr{A}$ is a  regular  $I$-partial Hopf algebra, any invariant integral $\phi$ satisfies $\phi=\phi S$.

We have the following form of \emph{strong invariance}.

\begin{Lem} \label{lemma:strong-invariance}
  Let $\mathscr{A}$ be a partial Hopf algebra with invariant integral $\phi$. Then
  for all $a\in A$,
  \begin{align*}
    S\left(( \id\otimes
    \phi)(\Delta(b)(1 \otimes a))\right) &= (\id \otimes \phi)((1 \otimes b)\Delta(a)),\\  S\left((\phi \otimes
    \id)((a\otimes 1)\Delta(b))\right) &= (\phi \otimes \id)(\Delta(a)(b\otimes 1)).\end{align*}
\end{Lem}
\begin{proof}
 For example, definition of the antipode and left invariance of $\phi$ give
  \begin{align*}
    a_{(1)}\phi(ba_{(2)}) &= \sum_{n}
    a_{(1)}\phi(\epsilon(b_{(1)}\rho_{n})b_{(2)}\lambda_{n}a_{(2)}) 
= \sum_{n} \epsilon(b_{(1)}\rho_{n})\rho_{n}a_{(1)}\phi(b_{(2)}a_{(2)})
\\
&= S(b_{(1)})b_{(2)}a_{(1)}\phi(b_{(3)}a_{(2)}) =
S(b_{(1)})\phi(b_{(2)}a).
  \end{align*}
\end{proof}


\subsection{Partial compact quantum groups}

%We now turn towards the structures which will allow us to build operator algebraic quantum groupoids out of our partial Hopf algebras (see Section 7).
 
\begin{Def} An \emph{$I$-partial $*$-algebra} $\mathscr{A}$ is a partial
  algebra whose total algebra $A$ is equipped with an antilinear,
  antimultiplicative involution $*\colon A\rightarrow A$, $ a\mapsto
  a^*$,  such that the $\mathbf{1}_k$ are selfadjoint for all $k$ in
  the object set. 
\end{Def} 

This implies that $*$ restricts to antilinear maps $\GrDA{A}{k}{l}\rightarrow \GrDA{A}{l}{k}$

%One can of course give an alternative definition directly in terms of the partial algebra structure by requiring that we are given antilinear maps  satisfying the obvious antimultiplicativity and involution properties.

\begin{Def} A \emph{partial $*$-bialgebra} $\mathscr{A}$ is a
 partial bialgebra whose underlying partial algebra has been
  endowed with a partial $*$-algebra structure such that
$\Delta_{rs}(a)^* = \Delta_{sr}(a^*)$ for all $a \in \Gr{A}{k}{l}{m}{n}$.
A \emph{partial Hopf $*$-algebra} is a partial bialgebra which is at the same time a partial $*$-bialgebra and a partial Hopf algebra.
\end{Def} 

%From Theorem \ref{theorem:partial-hopf-algebra} and \cite{Boh1},
%\cite{VDW1}, we can deduce:
\begin{Prop} \label{cor:involutive}
  An $I$-partial $*$-bialgebra $\mathscr{A}$ is an $I$-partial Hopf
  $*$-algebra if and only if the weak multiplier $*$-bialgebra
  $(A,\Delta)$ is a weak multiplier Hopf $*$-algebra. In that case,
  the counit and antipode satisfy
  $\epsilon(a^{*})=\overline{\epsilon(a)}$ and $S(S(a)^{*})^{*}=a$ for
  all $a\in A$. In particular, the total antipode is bijective.
\end{Prop}
\begin{proof}
  The if and only if part follows immediately from  Theorem
  \ref{theorem:partial-hopf-algebra}, the relation for the counit  from
uniqueness of the counit  \cite[Theorem 2.8]{Boh1}, and the relation
for the antipode from \cite[Proposition 4.11]{VDW1}.
\end{proof}

%We are finally ready to formulate our main definition.
\begin{Def} A \emph{partial compact quantum group} $\mathscr{G}$ consists of a
  partial Hopf $*$-algebra $\mathscr{A} = P(\mathscr{G})$ with an invariant integral  $\phi$ that is positive in the sense  that $\phi(a^*a)\geq 0$ for all $a\in A$. We also say that $\mathscr{G}$ is the partial compact quantum group \emph{defined by} $\mathscr{A}$.
\end{Def} 

\begin{Rem} It will follow from our Proposition \ref{prop:rep-cosemisimple} and  \cite[Theorem 3.3 and Theorem 4.4]{Hay1} that for $I$ finite, a partial compact quantum group is precisely a compact quantum group of face type \cite[Definition 4.1]{Hay1}. As the total object should not be considered compact for $I$ infinite, we have changed the terminology to \emph{partial} compact quantum group to reflect that only the parts should be considered compact.
\end{Rem} 

\subsection{Partial tensor categories}\label{SecPartTen}

The notion of partial algebra has a nice categorification. Recall first that the appropriate (vertical) categorification of a unital $\C$-algebra is a $\C$-linear additive tensor category. From now on, by `category' we will by default mean a $\C$-linear additive category. 

\begin{Def} A \emph{partial tensor category} $\CatCC$ over a set $\mathscr{I}$ consists of  a collection of (small) categories $\mathcal{C}_{\alpha\beta}$ with $\alpha,\beta\in \mathscr{I}$, a family of $\C$-bilinear functors $\otimes = \otimes_{\alpha,\beta,\gamma}: \CatC_{\alpha\beta}\times \CatC_{\beta\gamma}\rightarrow \CatC_{\alpha\gamma}$, natural isomorphisms $a_{X,Y,Z}: (X\otimes Y)\otimes Z \rightarrow X\otimes (Y\otimes Z)$ for $X \in \CatC_{\alpha\beta},Y\in \CatC_{\beta\gamma},Z\in \CatC_{\alpha\beta}$, non-zero objects $\Unitb_{\alpha} \in \CatC_{\alpha\alpha}$ and natural isomorphisms $\lambda_X^{(\alpha)}:\Unitb_\alpha\otimes X \rightarrow X$ and $\rho_X^{(\beta)}:X\otimes \Unitb_\beta\rightarrow X$ for $X\in \CatC_{\alpha\beta}$, satisfying the obvious associativity and unit constraints. 
\end{Def}

There is no problem in modifying Maclane's coherence theorem to partial tensor categories, and we will henceforth assume that our partial tensor categories and tensor categories with local units are strict to lighten notation. 

The total notion corresponding to a partial tensor category is that of a \emph{tensor category with local units (indexed by $\mathscr{I}$)}, that is, a tensor category without unit $(\CatC,\otimes,a)$  for which there exists a collection $\{\Unitb_\alpha\}_{\alpha\in \mathscr{I}}$ of objects such that $\Unitb_\alpha\otimes \Unitb_\beta \cong 0$ for each $\alpha\neq \beta$, such that for each object $X$ one has $\Unitb_\alpha\otimes X \cong 0 \cong X\otimes \Unitb_\alpha$ for all but a finite set of $\alpha$, and with fixed natural isomorphisms $\lambda_X:\oplus_\alpha (\Unitb_\alpha\otimes X) \rightarrow X$ and $\rho_X:\oplus_\alpha(X\otimes \Unitb_\alpha)\rightarrow X$ satisfying the obvious unit conditions. For example, if $\CatC$ is a tensor category with local units, one can put \[X_{\alpha\beta} = \Unitb_\alpha\otimes X \otimes \Unitb_\beta, \quad \eta_{\alpha\beta}:X_{\alpha\beta} \rightarrow \oplus_{\gamma,\delta} \left(\Unitb_\gamma \otimes X \otimes \Unitb_\delta\right) \cong X.\]  Then the $\CatC_{\alpha\beta} = \{X \in \CatC\mid X_{\alpha\beta} \underset{\eta_{\alpha\beta}}{\cong} X\}$, seen as full subcategories of $\CatC$, form a partial tensor category upon restriction of $\otimes$, and one easily verifies that this defines an equivalence between partial tensor categories and tensor categories with local units. 

Continuing the analogy with the algebra case, one can define the enveloping \emph{multiplier tensor category} $M(\CatC)$ of a tensor category with local units. Its objects consist of formal rcf sums $\oplus_{\alpha,\beta\in \mathscr{I}} X_{\alpha\beta}$, with $\Mor(\oplus X_{\alpha\beta},\oplus Y_{\alpha\beta}) = \left(\prod_\beta\oplus_\alpha  \Mor(X_{\alpha\beta},Y_{\alpha\beta}) \right) \cap \left(\prod_\alpha\oplus_\beta \Mor(X_{\alpha\beta},Y_{\alpha\beta})\right)$ as morphism spaces, the composition being entry-wise. Note that the rcf condition on objects allows us to write simply $\Mor(\oplus X_{\alpha\beta},\oplus Y_{\alpha\beta}) = \prod_{\alpha\beta} \Mor(X_{\alpha\beta},Y_{\alpha\beta})$.

The tensor product of $\CatC$ extends to $M(\CatC)$ by putting $\left(\oplus X_{\alpha\beta}\right)\otimes \left(\oplus Y_{\alpha\beta}\right) = \oplus_{\alpha,\beta,\gamma} \left(X_{\alpha\beta}\otimes Y_{\beta\gamma}\right)$, and similarly for morphism spaces. The associativity constraints of the $\CatC_{\alpha\beta}$ can be summed to an associativity constraint for $M(\CatC)$, while $\Unitb := \oplus_{\alpha\in \mathscr{I}} \Unitb_\alpha$ becomes a unit for $M(\CatC)$, rendering $M(\CatC)$ into an ordinary tensor category (with unit object).

%\begin{Rem} With some effort, a more intrinsic construction of the multiplier tensor category can be given in terms of couples of endofunctors, in the same vein as the construction of the multiplier algebra of a non-unital algebra.
%\end{Rem} 

As an example, consider a set $I$. Then we can form the partial tensor category $\CatCC = \{\Vect_{\fin}\}_{i,j\in I}$ where each $\CatC_{ij}$ is a copy of the category of finite-dimensional vector spaces $\Vect_{\fin}$, and with each $\otimes$ the ordinary tensor product. The total category $\CatC$ can then be identified with the category $\Vectif$ of finite-dimensional bi-graded vector spaces with the `balanced' tensor product over $I$,  $\Gru{(}{k}{}V\itimes W\Gru{)}{}{m} = \oplus_l \;(\Gru{V}{k}{l}\otimes \Gru{W}{l}{m})\subseteq V\otimes W.$ The multiplier category $M(\Vectif)$ equals $\Vectrcf$, the category of bigraded vector spaces which are rcfd (i.e.~ finite-dimensional on each row and column).

%We now formulate the appropriate notion of functor between partial tensor categories. When $\CatCC$ is a partial tensor category over $\mathscr{I}$ and $\mathscr{J}\subseteq \mathscr{I}$, we call $\CatDD = \{\CatC_{\alpha\beta}\}_{\alpha,\beta\in \mathscr{J}}$ a \emph{restriction} of $\CatCC$. 

\begin{Def} Let $\CatCC$ and $\CatDD$ be partial tensor categories over respective sets $\mathscr{I}$ and $\mathscr{J}$, and let $\phi: k \mapsto k'$ determine a decomposition $\mathscr{J} = \{\mathscr{J}_\alpha\mid \alpha\in \mathscr{I}\}$ with $k\in \mathscr{J}_\alpha \iff \phi(k)=\alpha$. 

A \emph{$\phi$-morphism} from $\CatCC$ to $\CatDD$ consists of $\C$-linear functors $F_{kl}: \CatC_{k'l'}\rightarrow \CatD_{kl}$, natural monomorphisms $\iota^{(klm)}_{X,Y}:F_{kl}(X) \otimes F_{lm}(Y) \hookrightarrow F_{km}(X\otimes Y)$ for $X\in \CatC_{k'l'},Y\in \CatD_{l'm'}$ and isomorphisms $\mu_{k}:  \Unitb_k \cong F_ {kk}(\Unitb_{k'})$ with $F_{kl}(\Unitb_{\alpha})= 0$ if $k\neq l$ in $\mathscr{J}_\alpha$, with $(k,l)\mapsto F_{kl}(X)$ rcf for each fixed $X$, and with $\oplus_{l\in \mathscr{J}_\beta} \iota^{(klm)}_{X,Y}: \left(\oplus_{l\in \mathscr{J}_\beta}F_{kl}(X) \otimes F_{lm}(Y)\right) \cong F_{km}(X\otimes Y)$ for all $X\in \CatC_{k'\beta}$ and $Y\in \CatC_{\beta m'}$. Moreover, the $\iota^{(klm)}$ satisfy the 2-cocycle condition making \[\xymatrix{F_{kl}(X)\otimes F_{lm}(Y)\otimes F_{mn}(Z) \ar[rr]^{\id\otimes \iota^{(lmn)}_{Y,Z}} \ar[d]_{\iota^{(klm)}_{X,Y}\otimes\id}&& F_{kl}(X)\otimes F_{ln}(Y\otimes Z)\ar[d]^{\iota^{(kln)}_{X,Y\otimes Z}}\\ F_{km}(X\otimes Y)\otimes F_{mn}(Z) \ar[rr]_{\iota^{(kmn)}_{X\otimes Y,Z}}&& F_{kn}(X\otimes Y \otimes Z)}\] commute for all $X\in \CatC_{k'l'},Y\in \CatC_{l'm'}, Z\in \CatC_{m'n'}$, and the $\mu_k$ satisfy satisfy a similar coherence with respect to the unit and $\iota$-maps. 
\end{Def}

Note that if $J_{\alpha}=\emptyset$, this means that $\Unitb_{\alpha}$ is sent to the zero object. In case $\phi$ is not everywhere defined on $\mathscr{J}$, a $\phi$-morphism is defined as above but with $\CatDD$ replaced by its restriction to the domain of $\phi$, where restriction means that components with indices outside the subset are forgotten.

The corresponding global notion of $\phi$-morphism (for $\phi$ everywhere defined) is that of a functor $F:\CatC \rightarrow M(\CatD)$ with isomorphisms $\iota_{X,Y}:F(X)\otimes F(Y)\cong F(X\otimes Y)$ and $\mu_\alpha:\oplus_{k\in \mathscr{J}_\alpha} \Unitb_k \cong F(\Unitb_\alpha)$ satisfying the natural coherence conditions. 

%We can use the notion of morphism to introduce the following weak form of equivalence, corresponding to chopping up a partial tensor category into smaller pieces (or, vice versa, gluing certain blocks of a partial tensor category together). %Let us formalize this in the following definition.

%\begin{Def} Let $\CatCC$ and $\CatDD$ be partial tensor categories. We say $\CatDD$ is a \emph{partitioning} of $\CatCC$ (or $\CatCC$ a \emph{globalisation} of $\CatDD$) if there exists a unital morphism $\CatCC\rightarrow \CatDD$ inducing an equivalence of categories $\CatC\rightarrow \CatD$.
%\end{Def}

%The partial tensor categories that we will be interested in will be required to have some further structure. 

\begin{Def} A partial tensor category $\CatCC$ is called \emph{semi-simple} if all $\CatC_{\alpha\beta}$ are semi-simple. A partial tensor category is said to have \emph{indecomposable units} if all units $\Unitb_\alpha$ are indecomposable. 
\end{Def}

%It is easy to see that any semi-simple tensor category can be partitioned into a semi-simple tensor category with indecomposable units.  Hence we will from now on only consider semi-simple partial tensor categories with indecomposable units.
 
%The following definition introduces the notion of duality for partial tensor categories.

\begin{Def} Let $\CatCC$ be a partial tensor category. An object $X\in \CatC_{\alpha\beta}$ is said to admit a \emph{left dual} if there exists an object $Y=\hat{X} \in \CatC_{\beta\alpha}$ and morphisms $\ev_{X}: Y\otimes X \rightarrow \Unitb_\beta$ and $\coev_X: \Unitb_\alpha\rightarrow X\otimes Y$ satisfying the obvious snake identities. We say $\CatCC$ \emph{admits left duality} if each object of each $\CatC_{\alpha\beta}$ has a left dual.
\end{Def}

Similarly, one defines right duality $X\rightarrow \check{X}$ and (two-sided) duality $X\rightarrow \bar{X}$. As for tensor categories with unit, if $X$ admits a (left or right) dual, it is unique up to isomorphism. As for usual tensor categories, one easily verifies that if $X$ has left dual $\hat{X}$, then $X$ is a right dual to $\hat{X}$. Moreover, if $F$ is a morphism $\CatCC\rightarrow \CatDD$ based over $\phi:\mathscr{J}\rightarrow \mathscr{I}$, and if $X\in \CatC_{k'l'}$ has a left dual, then $F_{lk}(\hat{X})$ is a left dual to $F_{kl}(X)$.

\begin{Def} A \emph{partial fusion C$^*$-category} is a partial tensor category $(\CatC_{\alpha\beta},\otimes)$ with duality such that all $\CatC_{\alpha\beta}$ are semi-simple C$^*$-categories, such that all functors $\otimes$ are $^*$-functors (in the sense that $(f\otimes g)^* = f^*\otimes g^*$ for morphisms), and such that the associativity and unit constraints are unitary.
\end{Def} 

Note that we slightly abuse the terminology `fusion', as strictly speaking this would require there to be only a finite set of mutually non-equivalent irreducible objects in each $\CatC_{\alpha\beta}$. For the corresponding total (pre-)C$^*$-category with local units associated to a partial fusion C$^*$-category, we will use the terminology \emph{multiplier fusion C$^*$-category}.

The notion of morphism for partial semi-simple tensor C$^*$-categories has to be adapted by requiring that all $F_{kl}$ are $^*$-functors and the $\iota$- and $\mu$-maps isometric.  Note that a morphism of partial fusion C$^*$-categories based over a \emph{surjective} map $\varphi: \mathscr{J}\rightarrow \mathscr{I}$ is automatically faithful. Indeed, by semisimplicity a non-faithful morphism sends some irreducible object to zero. By the duality assumption this means that some irreducible unit is sent to zero, which is excluded by surjectivity of $\varphi$ and the definition of morphism.

As an example of partial fusion C$^*$-category, consider a set  $I$. Then we can consider the partial fusion C$^*$-category $\CatCC = \{\Hilb_{\fin}\}_{I\times I}$ of finite-dimensional Hilbert spaces, with all $\otimes$ the ordinary tensor product. The associated global category is the category $\Hilbif$ of finite-dimensional bi-graded Hilbert spaces. The dual of a Hilbert space $\Hsp \in \CatC_{kl}$ is just the ordinary dual Hilbert space $\Hsp^* \cong \overline{\Hsp}$, but considered in the category $\CatC_{lk}$. 

% Ref to Gaby's comodule paper

\section{Representation theory of partial compact quantum groups}

\subsection{Corepresentations of partial bialgebras}

Let $\mathscr{A}$ be an $I$-partial bialgebra. We write
$\Hom_\C(V,W)$ for the vector space of linear maps between two vector
spaces $V$ and $W$.

%Let us first formally introduce a notion which already appeared in the previous section.

\begin{Def} Let $I$ be a set. An $I^{2}$-graded vector space $V=\bigoplus_{k,l\in I} \Gru{V}{k}{l}$ will be called \emph{row-and column finite dimensional} (rcfd) if the $\oplus_l V_{kl}$ (resp.~ $\oplus_k V_{kl}$) are finite dimensional for each $k$ (resp.~ $l$) fixed. 
\end{Def} 

We denote by  $\Vectrcf$ the category whose objects are rcfd $I^{2}$-graded vector spaces. Morphisms are linear maps $T$ that preserve the grading and therefore
can be written $T=\prod_{k,l\in I} \Gru{T}{k}{l}$. 

\begin{Def} \label{definition:corep} Let $\mathscr{A}$ be an
  $I$-partial bialgebra and let $V=\bigoplus_{k,l} \Gru{V}{k}{l}$
   be
an rcfd $I^{2}$-graded vector space.  A \emph{corepresentation}
  $\mathscr{X}=(\Gr{X}{k}{l}{m}{n})_{k,l,m,n}$ of $\mathscr{A}$ on $V$
  is a family of elements $\Gr{X}{k}{l}{m}{n} \in \Gr{A}{k}{l}{m}{n} \otimes
  \Hom_\C(\Gru{V}{m}{n},\Gru{V}{k}{l})$
 satisfying $ (\Delta_{pq} \otimes
    \id)(\Gr{X}{k}{l}{m}{n}) =
    \Big{(}\Gr{X}{k}{l}{p}{q}\Big{)}_{13}\Big{(}\Gr{X}{p}{q}{m}{n}\Big{)}_{23}$ and
     $(\epsilon \otimes
  \id)(\Gr{X}{k}{l}{m}{n})=\delta_{k,m}\delta_{l,n}\id_{\Gru{V}{k}{l}}$
% We also call $(V,\mathscr{X})$ an
 % \emph{rcfd corepresentation}.
\end{Def}
Here, we use here the standard leg numbering notation, e.g.~ $a_{23}=1\otimes a$. 

 For example, equip the vector space
  $\C^{(I)}=\bigoplus_{k\in I} \C$ with the diagonal
  $I^{2}$-grading. Then the family $\mathscr{U}$ given by $\Gr{U}{k}{l}{m}{n} = \delta_{k,l}\delta_{m,n} \UnitC{k}{m} \in
    \Gr{A}{k}{l}{m}{n}$
is a corepresentation of $\mathscr{A}$ on $\C^{(I)}$, called the
\emph{trivial corepresentation}. As another example,  assume  given an rcfd family of subspaces
  $\Gru{V}{m}{n} \subseteq \bigoplus_{k,l} \Gr{A}{k}{l}{m}{n}$ satisfying
  $\Delta_{pq}(\Gru{V}{m}{n}) \subseteq \Gru{V}{p}{q} \otimes
    \Gr{A}{p}{q}{m}{n}.$
Then the  elements $\Gr{X}{k}{l}{m}{n} \in \Gr{A}{k}{l}{m}{n} \otimes
  \Hom_{\C}(\Gru{V}{m}{n},\Gru{V}{k}{l})$ defined by 
  \begin{align*}
    \Gr{X}{k}{l}{m}{n}(1 \otimes b) &= \Delta^{\op}_{kl}(b) \in
    \Gr{A}{k}{l}{m}{n} \otimes \Gru{V}{k}{l} \quad
    \text{for all } b\in \Gru{V}{m}{n}
  \end{align*}
  form a corepresentation $\mathscr{X}$ of $\mathscr{A}$ on
  $V$, called the 
  \emph{regular corepresentation on $V$}. 

 A morphism  $T$ between rcfd corepresentations
  $(V,\mathscr{X})$ and $(W,\mathscr{Y})$ of $\mathscr{A}$ will be a family
  of linear maps $\Gru{T}{k}{l} \in
  \Hom_\C(\Gru{V}{k}{l},\Gru{W}{k}{l})$ satisfying $(1 \otimes
  \Gru{T}{k}{l})\Gr{X}{k}{l}{m}{n} = \Gr{Y}{k}{l}{m}{n}(1 \otimes
  \Gru{T}{m}{n})$.  In this way rcfd corepresentations form a category which we will denote
$\Corep_{\rcf}(\mathscr{A})$.

We next consider the total form of a corepresentation. Let $V$ be an rcfd $I^{2}$-graded vector space, and write $\lambda_k^V,\rho_k^V$ for the projections on the components with resp.~ left and right grading equal to $k$. Write $\End_0(V)$ for the algebra of endomorphisms on $V$ having finite dimensional support. We can in a natural way consider the map $\Delta \otimes \id \colon M(A \otimes \End_{0}(V)) \to M(A \otimes A \otimes \End_{0}(V))$. An element $X \in  M(A
  \otimes \End_{0}(V))$ is then called a \emph{total corepresentation} if $(\lambda_{k}\rho_{m} \otimes \id){X}(\lambda_{l}\rho_{n}
    \otimes \id) = (1 \otimes \lambda^{V}_{k}\rho^{V}_{l}){X}(1 \otimes
    \lambda^{V}_{m}\rho^{V}_{n}) \in A \otimes \End_0(V)$,  $(\Delta\otimes \id)(X)=X_{13}X_{23}$ and $(\epsilon \otimes \id)({X}) = \id_{V}$ as a multiplier of $\End_0(V)$. It is easily verified that the $\Gr{X}{k}{l}{m}{n} = (1 \otimes \lambda^{V}_{k}\rho^{V}_{l}){X}(1 \otimes
    \lambda^{V}_{m}\rho^{V}_{n})$ then define a corepresentation $\mathscr{X}$ of $\mathscr{A}$, and that all corepresentations arise in this way from a unique total corepresentation. We call $X$ the \emph{corepresentation multiplier} of $\mathscr{X}$. It is also not hard to show that total corepresentations are in unique correspondence with full comodule structures  whose induced bigrading is rcfd, cf.~ \cite[Definition 2.2, Definition 4.2 and Theorem 4.5]{Boh2}.


There is no problem in showing that $\Corep_{\rcf}(\mathscr{A})$ is a $\C$-linear abelian category, and that the forgetful functor $\Corep_{\rcf}(\mathscr{A}) \to \Vectrcf$ into rcfd vector spaces lifts kernels, cokernels and biproducts. In particular, one can call a corepresentation $V$ \emph{irreducible} if any morphism $T$ from (resp.~ into) $V$ either has all $T_{kl}$ zero or injective (resp.~ surjective).

In case $\mathscr{A}$ is a partial Hopf algebra, the defining property of the antipode shows that every
corepresentation multiplier has a generalized inverse. Namely, if $(V,\mathscr{X})$ is an rcfd corepresentation of an $I$-partial Hopf
  algebra $\mathscr{A}$, then $X^{-1} =  (S \otimes \id)(X) \in M(A \otimes \Hom_{\C}^{0}(V))$ satisfies $XX^{-1} = \sum_{k} \lambda_{k} \otimes \lambda^{V}_{k}$ and
    $X^{-1}X  = \sum_{l} \rho_{l} \otimes \rho^{V}_{l}$, and  hence $XX^{-1}X=X$ and $X^{-1}XX^{-1}=X^{-1}$. We then write
   \[\Gr{(X^{-1})}{k}{l}{m}{n}=(S \otimes \id)(\Gr{X}{l}{k}{n}{m}) \in
   \Gr{A}{k}{l}{m}{n} \otimes \Hom_{\C}(\Gru{V}{l}{k},\Gru{V}{n}{m})\] for the components.

The following easy lemma will be very useful.
\begin{Lem} \label{lemma:rep-total-morphism}
 A bigraded map $T$ defines a morphism from 
    $(V,\mathscr{X})$ to $(W,\mathscr{Y})$ if and only if one of the following relations hold:
    \begin{align*}
      Y^{-1}(1 \otimes T)X&=\sum_{m,n} \rho_{n} \otimes \Gru{T}{m}{n},
      &
    Y(1\otimes T)X^{-1} &=\sum_{k,l} \lambda_{k} \otimes \Gru{T}{k}{l}.
    \end{align*}
\end{Lem}


\subsection{Tensor product and duality}

Recall from Section \ref{SecPartTen} that the category $\Vectrcf$ of rcfd vector spaces is a tensor category for the $I$-balanced tensor product $\itimes$, the tensor product of morphisms being the restriction of the ordinary tensor product. For the sake of convenience we interpret this product as being strictly associative.  The unit for this product is the vector space $\C^{(I)}=\bigoplus_{k\in I} \C$. 

Given $V$ and $W$ in $\Vectrcf$, we identify $\Hom_\C(\Gru{V}{m}{n},\Gru{V}{k}{l})\otimes  \Hom_\C(\Gru{W}{n}{q},\Gru{W}{l}{p})$ with a subspace of $ \Hom_\C(\Gru{(}{m}{}V\itimes  W\Gru{)}{}{q},\Gru{(}{k}{}V\itimes W\Gru{)}{}{p})$. Then if $\mathscr{A}$ is a partial bialgebra, one define a tensor product of copresentations  $\mathscr{X}$ and $\mathscr{Y}$ by the formula $\Gr{(X\Circt Y)}{k}{p}{m}{q} = \sum_{l,n}  \left(\Gr{X}{k}{l}{m}{n}\right)_{12}\left(\Gr{Y}{l}{p}{n}{q}\right)_{13}$, where the sum is seen to be finite and hence well-defined. Its associated corepresentation multiplier  is simply $X_{12}Y_{13}$.
    
With this tensor product,  $\Corep_{\rcf}(\mathscr{A})$ becomes a strict tensor category with unit the trivial corepresentation $(\C^{(I)},\mathscr{U})$. Its natural forgetful functor into $\Vecti$ is a tensor functor.

Assume now that $\mathscr{A}$ is an $I$-partial Hopf algebra.  If then $(I_{\alpha})_{\alpha\in \mathscr{I}}$ is the decomposition of $I$ into
  hyperobject classes, we have that each $\C^{(I_{\alpha})} \subseteq \C^{(I)}$ is invariant, and determines a decomposition $\mathscr{U}=\bigoplus_{\alpha\in\mathscr{I}}
  \mathscr{U_{\alpha}}$ of the trivial corepresentation into irreducibles. Denote by $\Corep(\mathscr{A})$ the category of rcfd corepresentations $(V,\mathscr{X})$ for which there exists a finite subset of the hyperobject set $\mathscr{I}$ such that $\Gru{V}{k}{l}=0$ for the equivalence classes of $k,l$ outside this subset.  Then it is easily seen that $\Corep(\mathscr{A})$ is  a tensor category with the $\mathscr{U}_{\alpha}$ as local units, and with $\Corep_{\rcf}(\mathscr{A})$ as its multiplier category. We will use the same notation $\Corep(\mathscr{A})$ for the associated partial tensor category. 
  
Moreover, still under the assumption that $\mathscr{A}$ is an $I$-partial Hopf algebra, we have that $\Corep_{\rcf}(A)$ has left duality. To see this, let $(V,\mathscr{X})$ be an rcfd corepresentation. Denote the dual of vector spaces $V$ and  linear maps $T$ by
$\dual{V}$ and $\dualop{T}$, respectively, and define the dual of an
$I^{2}$-graded vector space $V=\bigoplus_{k,l} \Gru{V}{k}{l}$ to be
the space $\duall{V}=\bigoplus_{k,l} \Gru{(\duall{V})}{k}{l}$ where $\Gru{(\duall{V})}{k}{l} = \dual{(\Gru{V}{l}{k})}$. Then using anti-comultiplicativity of $S$ and Lemma \ref{LemCoAnt}, we see that $\duall{V}$ and the family
  $\dualco{\mathscr{X}}$ given by $\Gr{\dualco{X}}{k}{l}{m}{n}   :=  (S \otimes \dualop{-})(\Gr{X}{n}{m}{l}{k})$
   form an rcfd corepresentation of $\mathscr{A}$. To see that it is a left dual of $\mathscr{X}$, it suffices to show that the natural evaluation and coevaluation maps $\ev \colon \duall{V} \itimes V \to \C^{(I)}$ and $\coev \colon \C^{(I)} \to V\itimes \duall{V}$ are morphisms from the trivial corepresentation to the tensor product representations of $\mathscr{X}$ with $\dualco{\mathscr{X}}$. But for example the intertwining property of $\ev$ follows from 
  \begin{align*}
    (1\otimes \Gru{\ev}{k}{k})
 \sum_{l,n}  \big(
\Gr{\dualco{X}}{k}{l}{m}{n}\big)_{12}
\big(\Gr{X}{l}{k}{n}{q}\big)_{13} &=
    (1\otimes \Gru{\ev}{k}{k})
 \sum_{l,n} 
(S \otimes \dualop{-})(\Gr{X}{n}{m}{l}{k})_{12}
    (\Gr{X}{l}{k}{n}{q})_{13} \\ &=
(1\otimes \Gru{\ev}{m}{m})  \sum_{l,n}
      (S \otimes \id)(\Gr{X}{n}{m}{l}{k})_{13}(\Gr{X}{l}{k}{n}{q})_{13} \\
    &= \delta_{m,q}\UnitC{k}{q}\otimes \Gru{\ev}{m}{m} \\
    &= \Gr{U}{k}{k}{m}{q}(1 \otimes \Gru{\ev}{m}{m}).
  \end{align*}
  
Similarly, if $\mathscr{A}$ is a \emph{regular} partial Hopf algebra, then any rcfd corepresentation has right duals. It follows that in this case $\Corep_{\rcf}(\mathscr{A})$ is a partial tensor category with left and right duals.


\subsection{Decomposition into irreducible corepresentations and matrix coefficients}

In the presence of an invariant integral, one can integrate morphisms of vector spaces to obtain morphisms of corepresentations. 
\begin{Lem} \label{lem:rep-average}  Let $(V,\mathscr{X})$ and
  $(W,\mathscr{Y})$ be rcfd corepresentations of  a partial
  Hopf algebra $\mathscr{A}$ with an invariant integral $\phi$, and let
  $\Gru{T}{k}{l} \in \Hom_{\C}(\Gru{V}{k}{l},\Gru{W}{k}{l})$ for all $k,l\in I$. Then for each $m,n$ fixed, the families
  \begin{align*}
    \Gr{\check T}{m}{n}{k}{l} &:= (\phi \otimes
    \id)(\Gr{(Y^{-1})}{n}{m}{l}{k}(1\otimes
    \Gru{T}{m}{n})\Gr{X}{m}{n}{k}{l}), \\
    \Gr{\hat T}{m}{n}{k}{l} &:=(\phi \otimes
    \id)(\Gr{Y}{k}{l}{m}{n}(1\otimes
    \Gru{T}{m}{n})\Gr{(X^{-1})}{l}{k}{n}{m})
  \end{align*} % think it's ok to keep this notation as we did not apply the previous hat and check to operators...
form  morphisms $\Grd{\check{T}}{m}{n}$ and $\Grd{\hat{T}}{m}{n}$ from $(V,\mathscr{X})$ to $(W,\mathscr{Y})$. 
% Slightly changed the averaging so that I do not need a restriction of finite support on $T$.
\end{Lem} 
\begin{proof} We may suppose that $T$ is supported only on the component at index $(m,n)$. We can then drop the upper indices and write $\Gru{\check{T}}{k}{l}$ and $\Gru{\hat{T}}{k}{l}$. Then 
 in total form, $\check{T}=(\phi \otimes \id)(Y^{-1}(1 \otimes T)X)$
  and $\hat{T}=(\phi \otimes \id)(Y(1 \otimes T)X^{-1})$.  We then compute
  \begin{align*}
    Y^{-1}(1 \otimes \check{T})X &= (\phi \otimes \id \otimes
    \id)((Y^{-1})_{23}(Y^{-1})_{13}(1 \otimes 1
    \otimes T)X_{13}X_{23})  \\
    &= ((\phi \otimes\id)  \Delta  \otimes \id)(Y^{-1}(1 \otimes T)X) \\
    &= \sum_{l} \rho_{l} \otimes (\phi \otimes \id)((\rho_{l} \otimes
    1)Y^{-1}(1 \otimes T)X)  \\
    &= \sum_{k,l} \rho_{l} \otimes \Gru{\check T}{k}{l}.
  \end{align*}
  Hence $\check{T}$ is a morphism from $\mathscr{X}$ to $\mathscr{Y}$
  by Lemma \ref{lemma:rep-total-morphism}. The assertion for $\hat
  T$ follows similarly.
\end{proof}


\begin{Lem}
  Let $\mathscr{A}$ be an $I$-partial Hopf algebra with an invariant integral $\phi$.
  Let $(V,\mathscr{X})$ be an rcfd corepresentation
  and $\Gru{W}{k}{l} \subseteq \Gru{V}{k}{l}$ an invariant family of
  subspaces. Then there exists an idempotent endomorphism $T$ of
  $(V,\mathscr{X})$ such that $\Gru{W}{k}{l}=\img\Gru{T}{k}{l}$ for
  all $k,l$.
\end{Lem}
\begin{proof}
By a direct sum decomposition, we may assume that $V$ is in a fixed component $\Corep(\mathscr{A})_{\alpha\beta}$. For all $k\in I_{\alpha},l\in I_{\beta}$, choose idempotent endomorphisms $\Gru{T}{k}{l}$ of $\Gru{V}{k}{l}$
  with image $\Gru{W}{k}{l}$. By Lemma \ref{lem:rep-average}, we obtain
  endomorphisms $\Grd{\check{T}}{m}{n}$ of $(V,\mathscr{X})$. We want to show
  that linear combinations of these provide the sought-after morphism.
  
    In
  total form, invariance of $W$ implies  $(1 \otimes T)X(1
  \otimes T)=X(1\otimes T)$. Applying
 $(S \otimes \id)$, we get   $(1 \otimes T)X^{-1}(1
  \otimes T)=X^{-1}(1\otimes T)$.
Now choose $n\in I_{\beta}$ and write $\Grd{\check{T}}{}{n} = \sum_m \Grd{\check{T}}{m}{n}$ (using column-finiteness of $V$). Then
  \begin{align*}
    \Grd{\check{T}}{}{n} T &= (\phi \otimes \id)(X^{-1}(1 \otimes
    \rho_{n}^{V}T)X(1 \otimes T))  \\ &= 
     (\phi \otimes \id)(X^{-1}(1 \otimes
    \rho_{n}^{V})X(1 \otimes T)) 
    =
  \sum_l \phi(\UnitC{n}{l}) \rho^{V}_{l}T  =T,
  \end{align*}
 since we only have to sum over $l\in I_{\beta}$ as $n\in I_{\beta}$ by assumption. 
 
 Now as $W$ is invariant and $T$ sends $V$ into $W$, we have that $\Gr{\check{T}}{}{n}{k}{l}$ sends $\Gru{V}{k}{l}$ into $\Gru{W}{k}{l}$. Hence it follows that $\img{\check{T}^{n}}=\img T$, and $\check{T}^{n}$ is the sought-after intertwiner.
\end{proof}

\begin{Cor}  \label{cor:rep-cosemisimple}% Changed! See if later refs are still ok.
  Let $\mathscr{A}$ be a partial Hopf algebra with an invariant integral.  Then
  every rcfd corepresentation of $\mathscr{A}$ decomposes into a (possibly infinite) direct
  sum of irreducible rcfd corepresentations.
\end{Cor} 

% Semi-simplicity requires that each element is a finite direct sum! But possibly this is convention
We can now prove that the category $\Corep(\mathscr{A})$ of a partial Hopf algebra with invariant integral is semisimple, that is, any object is a finite direct sum of irreducible objects. We first state a lemma which will also be convenient at other occasions.
%If one allows a more relaxed definition of semisimplicity allowing infinite direct sums, this will be true also for the a potentially bigger category $\Corep_{\rcf}(\mathscr{A})$.

\begin{Lem}\label{LemInjMor}  Let $\mathscr{A}$ be a partial Hopf algebra with an invariant integral, and fix $\alpha,\beta$ in the hyperobject set.  Then if $T$ is a morphism in $\Corep(\mathscr{A})_{\alpha\beta}$ and $\sum_{k\in I_\alpha} \Gru{T}{k}{l}=0$ for some $l \in I_\beta$, then $T=0$.
\end{Lem} 

\begin{proof} This follows from the equations in Lemma \ref{lemma:rep-total-morphism}
\end{proof}

\begin{Prop}\label{prop:rep-cosemisimple} Let $\mathscr{A}$ be a partial Hopf algebra with an invariant integral.   Then the components of the partial tensor category $\Corep(\mathscr{A})$ are semisimple.
\end{Prop}
\begin{proof} 

Let $V$ be in any object of $\Corep(\mathscr{A})_{\alpha\beta}$ for $\alpha,\beta\in \mathscr{I}$.  From Lemma \ref{LemInjMor}, we see that for $T$ a morphism in $\Corep(\mathscr{A})_{\alpha\beta}$, the map $T\mapsto \sum_{k\in I_\alpha} \Gru{T}{k}{l}$ is injective for any choice of $l\in I_\beta$. By column-finiteness of $V$, the algebra of self-intertwiners of $V$ is finite-dimensional. From Corollary \ref{cor:rep-cosemisimple}, we deduce that $V$ is a finite direct sum of irreducible invariant subspaces.
\end{proof} 

%\subsection{Matrix coefficients of irreducible corepresentations}

Our next goal is to obtain Schur orthogonality for matrix coefficients of corepresentations. We want to give a little more detail than provided in Hayashi's paper \cite{Hay1}.

Given finite dimensional vector spaces $V$ and $W$, the dual space of
$\Hom_{\C}(V,W)$ is linearly spanned by functionals of the form $\omega_{f,v}(T) =  (f|Tv)$, where $v\in V$, $f\in \dual{W}$, and $(-|-)$ denotes the natural
pairing of $\dual{W}$ with $W$.
\begin{Def} Let $\mathscr{A}$ be a partial bialgebra. The space of
  \emph{matrix coefficients} $\mathcal{C}(\mathscr{X})$ of an rcfd
  corepresentation $(V,\mathscr{X})$ is the sum of the subspaces
\begin{align*}
  \Gr{\mathcal{C}(\mathscr{X})}{k}{l}{m}{n} &= \Span \left\{ (\id \otimes
    \omega_{f,v})(\Gr{X}{k}{l}{m}{n}) \mid v\in \Gru{V}{m}{n}, f \in
    \dual{(\Gru{V}{k}{l})} \right\} \subseteq \Gr{A}{k}{l}{m}{n}.
\end{align*}
\end{Def}
As $\Delta_{pq}(\Gr{\mathcal{C}(\mathscr{X})}{k}{l}{m}{n}) \subseteq
  \Gr{\mathcal{C}(\mathscr{X})}{k}{l}{p}{q} \otimes
  \Gr{\mathcal{C}(\mathscr{X})}{p}{q}{m}{n}$, the $\Gr{\mathcal{C}(\mathscr{X})}{k}{l}{m}{n}$ form a partial
coalgebra with respect to $\Delta$ and $\epsilon$.  Moreover, for each
$k,l$ the $I^{2}$-graded vector  space $\Grd{\mathcal{C}(\mathscr{X})}{k}{l}:=\bigoplus_{m,n }
  \Gr{\mathcal{C}(\mathscr{X})}{k}{l}{m}{n}$ is rcfd, and the inclusion above shows that one can
form the regular corepresentation on this space.

\begin{Lem} \label{lemma:rep-regular-embedding}
  Let $(V,\mathscr{X})$ be an rcfd corepresentation
  of a partial bialgebra and let $f\in
  \dual{(\Gru{V}{k}{l})}$. Then the family of maps
  \begin{align*}
    \Gr{T}{}{}{m}{n(f)} \colon \Gru{V}{m}{n} \to
    \Gr{\mathcal{C}(\mathscr{X})}{k}{l}{m}{n}, \ w \mapsto (\id
    \otimes \omega_{f,w})(\Gr{X}{k}{l}{m}{n})=(\id \otimes
    f)(\Gr{X}{k}{l}{m}{n}(1 \otimes w)),
  \end{align*}
  is a morphism from $\mathscr{X}$ to the regular corepresentation on
  $\Grd{\mathcal{C}(\mathscr{X})}{k}{l}$. 
\end{Lem}
\begin{proof}
  Denote by $\mathscr{Y}$ the regular corepresentation on
  $\bigoplus_{m,n } \Gr{\mathcal{C}(\mathscr{X})}{k}{l}{m}{n}$. Then
for all $v \in \Gru{V}{m}{n}$, 
  \begin{align*}
 \Gr{Y}{p}{q}{m}{n}    (1\otimes \Gr{T}{}{}{m}{n(f)}(v)) &= 
(\Delta^{\op}_{pq} \otimes \omega_{f,v})( \Gr{X}{k}{l}{m}{n})  = (\id \otimes \id \otimes
f)((\Gr{X}{k}{l}{p}{q})_{23}(\Gr{X}{p}{q}{m}{n})_{13}(1 \otimes 1
 \otimes v)) \\&=(1 \otimes \Gr{T}{}{}{p}{q(f)})\Gr{X}{p}{q}{m}{n}(1 \otimes v).
  \end{align*}
\end{proof}
As before, we denote by $\dual{V}$ the dual of a vector space $V$.
\begin{Lem} \label{lemma:regular-corep} Let $\mathscr{A}$ be a partial
  Hopf algebra. Then for any $a \in \bigoplus_{k,l} \Gr{A}{k}{l}{m}{n}$, the family of
  subspaces $\Gru{V}{p}{q} = \{ (\id\otimes f)(\Delta_{pq}(a)) : f \in
    \dual{(\Gr{A}{p}{q}{m}{n})}\}$
  defines an rcfd  regular corepresentation with $a \in \Gru{V}{m}{n}$. If further $(W,\mathscr{Y})$ is an irreducible restriction of the
  regular corepresentation, then $(W,\mathscr{Y})$ is recovered by the above procedure from \emph{any} non-zero $a \in \Gru{W}{m}{n}$.
\end{Lem}
\begin{proof} Assume $a$ and $V$ as above. Taking $f=\epsilon$, one finds $a \in \Gru{V}{m}{n}$. Next, write
   $\Delta_{pq}(a)=\sum_{i} b_{pq}^{i} \otimes c^{i}_{pq}$ with linearly independent $(c_{pq}^{i})_{i}$. Then $ \Gru{V}{p}{q} =
  \mathrm{span}\{b_{pq}^{i} : i \}$, and  $\Delta_{rs}(\Gru{V}{p}{q}) \subseteq
  \Gru{V}{r}{s} \otimes \Gr{A}{r}{s}{p}{q}$ as $\sum_{i}
    \Delta_{rs}(b^{i}_{pq}) \otimes c^{i}_{pq}  = \sum_{j} b^{j}_{rs} \otimes
    \Delta_{pq}(c^{j}_{rs})$ by coassociativity. 
    
      If now  $W$ is an irreducible restriction of the regular corepresentation and $a\in \Gru{W}{m}{n}$ non-zero, then 
    the associated $\Gru{V}{p}{q}$ are included in $\Gru{W}{p}{q}$. As $a\in \Gru{V}{m}{n}$, it follows by irreducibility that $\Gru{V}{p}{q} = \Gru{W}{p}{q}$ for all $p,q$.
\end{proof}
\begin{Prop} \label{prop:rep-weak-pw} Let $\mathscr{A}$ be a partial
  Hopf algebra with an invariant integral. Then the total algebra $A$ is the sum
  of the matrix coefficients of irreducible rcfd corepresentations.
\end{Prop}
\begin{proof} 
  Let $a \in \Gr{A}{k}{l}{m}{n}$, define $\Gru{V}{p}{q}$ as in
  Lemma \ref{lemma:regular-corep} and form the restriction of the regular
  corepresentation $(V,\mathscr{X})$. Then
  $a = (\id \otimes \epsilon)(\Delta^{\op}_{kl}(a)) =
    (\id \otimes \epsilon)(\Gr{X}{k}{l}{m}{n}(1 \otimes a)) \in
    \Gr{\mathcal{C}(\mathscr{X})}{k}{l}{m}{n}$.
  Decomposing $(V,\mathscr{X})$, we find that
  $a$ is contained in the sum of matrix coefficients of irreducible
rcfd  corepresentations.
\end{proof}


%The first part of the orthogonality relations concerns matrix
%coefficients of inequivalent irreducible corepresentations. 
\begin{Prop} \label{prop:rep-orthogonality-1} Let $\mathcal{A}$ be a
  partial Hopf algebra with an invariant integral $\phi$ and inequivalent
  irreducible rcfd corepresentations $(V,\mathscr{X})$ and
  $(W,\mathscr{Y})$.  Then  for all
  $a\in \mathcal{C}(X), b \in \mathcal{C}(Y)$,
  \[\phi(S(b)a) = \phi(bS(a))=0.\]
\end{Prop}
\begin{proof}
Since $\phi$ vanishes on $S(\Gr{A}{k}{l}{m}{n})\Gr{A}{p}{q}{r}{s}$ and
on $\Gr{A}{p}{q}{r}{s}S(\Gr{A}{k}{l}{m}{n})$ unless
$(p,q,r,s) = (m,n,k,l)$, it suffices to prove the assertion for  elements of the form $
  a=(\id \otimes \omega_{f,v})(\Gr{X}{k}{l}{m}{n})$ and  $b =(\id \otimes \omega_{g,w})(\Gr{Y}{m}{n}{k}{l})$
where $f\in \dual{(\Gru{V}{k}{l})}, v \in \Gru{V}{m}{n}$ and $g \in
\dual{(\Gru{W}{m}{n})}, w \in \Gru{W}{k}{l}$.  Applying Lemma
\ref{lem:rep-average} to the family of maps $\Gru{T}{p}{q} \colon \Gru{V}{p}{q} \to \Gru{W}{p}{q}$ with $\Gru{T}{p}{q}(u) =  \delta_{p,k}\delta_{q,l}  f(u)w,$
  yields morphisms $\Grd{\check{T}}{k}{l},\Grd{\hat{T}}{k}{l}$ from $(V,\mathscr{X})$ to
  $(W,\mathscr{Y})$ which necessarily are $0$. Inserting the
  definition of $\Grd{\check{T}}{k}{l}$, we find
  \begin{align*}
    \phi(S(b)a) &= \phi\big((S \otimes
    \omega_{g,w})(\Gr{Y}{m}{n}{k}{l}) \cdot (\id \otimes
    \omega_{f,v})(\Gr{X}{k}{l}{m}{n})\big) \\ &= (\phi \otimes \omega_{g,v})\left(\Gr{(Y^{-1})}{l}{k}{n}{m}(1 \otimes
      \Gru{T}{k}{l} )     \Gr{X}{k}{l}{m}{n}\right) 
    = \omega_{g,v}( \Gr{\check{T}}{k}{l}{m}{n}) = 0.
  \end{align*}
  
  A similar calculation involving $\hat{T}$ shows that
  $\phi(bS(a))=0$.  
\end{proof}

From now on, we will assume that our partial Hopf algebra is regular (i.e.~ has bijective antipode).

\begin{Theorem} \label{thm:rep-orthogonality} Let $\mathscr{A}$ be a
  regular partial Hopf algebra with an invariant integral $\phi$. Let $\alpha,\beta\in \mathscr{I}$, and let $(V,\mathscr{X})$
  be an irreducible rcfd corepresentation of $\mathscr{A}$ inside $\Corep(\mathscr{A})_{\alpha\beta}$. Suppose
  $F_{\mathscr{X}}=F$ is an isomorphism from $(V,\mathscr{X})$ to
  $(V,\hat{\hat{\mathscr{X}}})$ with inverse
  $G_{\mathscr{X}}= G$. Then the following hold.
  \begin{enumerate}[label=(\arabic*)]
  \item The numbers $d_G:=\sum_{k} \Tr (\Gru{G}{k}{l})$ and $d_F:=\sum_{n} \Tr (\Gru{F}{m}{n})$ are non-zero and do not depend on the choice of $l \in I_\beta$ or $m\in I_\alpha$.
    \item  For all $k,m \in I_\alpha$ and $l,n\in I_\beta$,
    \begin{align*}
      (\phi \otimes \id)(\Gr{(X^{-1})}{l}{k}{n}{m}\Gr{X}{k}{l}{m}{n})
      &=d_G^{-1}\Tr(\Gru{G}{k}{l})
      \id_{\Gru{V}{m}{n}}, \\
      (\phi \otimes \id)(\Gr{X}{k}{l}{m}{n}\Gr{(X^{-1})}{l}{k}{n}{m})
      &=d_F^{-1}\Tr(\Gru{F}{m}{n})
      \id_{\Gru{V}{k}{l}}.
    \end{align*}
  \item Denote by $\Sigma_{klmn}$ the flip map $\Gru{V}{k}{l}
    \otimes \Gru{V}{m}{n} \to \Gru{V}{m}{n}
    \otimes \Gru{V}{k}{l}$. Then
 \begin{align*}
   (\phi \otimes \id \otimes
   \id)((\Gr{(X^{-1})}{l}{k}{n}{m})_{12}(\Gr{X}{k}{l}{m}{n})_{13}) &=
   d_G^{-1}
   (\id_{\Gru{V}{m}{n}} \otimes \Gru{G}{k}{l})
   \circ \Sigma_{klmn}, \\
   (\phi \otimes \id \otimes
   \id)((\Gr{X}{k}{l}{m}{n})_{13}(\Gr{(X^{-1})}{l}{k}{n}{m})_{12}) &= d_F^{-1} (\Gru{F}{m}{n}
   \otimes \id_{\Gru{V}{k}{l}}) \circ \Sigma_{klmn}.
 \end{align*}
\end{enumerate}
  \end{Theorem}
\begin{proof}
  We prove the assertions and equations involving $d_G$ in (1), (2)
  and (3)  simultaneously; the assertions involving $d_F$  follow similarly.

  Consider
  the following endomorphism $F_{m,n,k,l}$ of $\Gru{V}{m}{n}\otimes \Gru{V}{k}{l}$, 
  \begin{align*}
    F_{m,n,k,l}
    &:=(\phi \otimes \id \otimes \id)\left((\Gr{(X^{-1})}{l}{k}{n}{m})_{12}(\Gr{X}{k}{l}{m}{n})_{13}\right)
    \circ \Sigma_{mnkl} \\ &= (\phi \otimes \id \otimes
    \id)\left((\Gr{(X^{-1})}{m}{n}{k}{l})_{12}
      \Sigma_{klkl,23}(\Gr{X}{k}{l}{m}{n})_{12}\right).
  \end{align*}
  By applying Lemma \ref{lem:rep-average} with respect to the flip map $\Sigma_{klkl}$, we see that the family $(F_{m,n,k,l})_{m,n}$ is
  an endomorphism of $(V \otimes \Gru{V}{k}{l}, X\otimes \id)$ and hence
  $F_{m,n,k,l} = \id_{\Gru{V}{m}{n}} \otimes \Gru{R}{k}{l}$ with some $\Gru{R}{k}{l} \in \Hom_{\C}(\Gru{V}{k}{l})$ not
  depending on $m,n$. 
  
  On the other hand, since $\phi = \phi S$,
  \begin{align*}
    F_{m,n,k,l} &= (\phi \otimes \id \otimes \id)((S \otimes
    \id)(\Gr{X}{m}{n}{k}{l})_{12}(\Gr{X}{k}{l}{m}{n})_{13})
    \circ \Sigma_{mnkl} \\
    &= (\phi \otimes \id \otimes \id)\left(((S \otimes
      \id)(\Gr{X}{k}{l}{m}{n}))_{13}
      ((S^{2} \otimes \id)(\Gr{X}{m}{n}{k}{l}))_{12}\right)     \circ \Sigma_{mnkl}\\
    &= (\phi \otimes \id \otimes
    \id)\left((\Gr{(X^{-1})}{k}{l}{m}{n})_{13} (\Sigma_{mnmn})_{23}
      (\Gr{(\dual{\dual{X}{}\!})}{m}{n}{k}{l})_{13}\right).
  \end{align*}
  Hence we can again apply Lemma \ref{lem:rep-average} and
  find that the family $(F_{m,n,k,l})_{k,l}$ is a morphism from $(\Gru{V}{m}{n} \otimes V, \hat{\hat{X}}_{13})$ to $(\Gru{V}{m}{n} \otimes V,
 X_{13})$. Therefore also $F_{m,n,k,l} = \Gru{T}{m}{n} \otimes \Gr{G}{k}{l}{}{\mathscr{X}}$
  with some $\Gru{T}{m}{n} \in \mathcal{\Hom_{\C}}(\Gru{V}{m}{n})$
  not depending on $k,l$. We hence conclude that in fact $F_{m,n,k,l} = \lambda
  (\id_{\Gru{V}{m}{n}} \otimes \Gr{G}{k}{l}{}{\mathscr{X}})$  for some $\lambda\in \C$
  
  Choose dual  bases
  $(v_{i})_{i}$ for $\Gru{V}{k}{l}$ and $(f_{i})_{i}$ for  $\dual{(\Gru{V}{k}{l})}$. Then
  \begin{align*}
    \lambda   \Tr( \Gr{G}{k}{l}{}{\mathscr{X}}) \id_{\Gru{V}{m}{n}}
 &= \sum_{i} (\id \otimes
    \omega_{f_{i},v_{i}})(F_{m,n,k,l}) = (\phi \otimes
    \id)((\Gr{(X^{-1})}{l}{k}{n}{m}) \Gr{X}{k}{l}{m}{n}).
  \end{align*}
  Take now $n=l$.  By Lemma \ref{LemInjMor}, we can choose $m\in I_{\alpha}$ with $\Gru{V}{m}{n}\neq 0$.   Then summing the previous relation over $k$, the relations $\sum_{k}
  (\Gr{(X^{-1})}{l}{k}{n}{m}) \Gr{X}{k}{l}{m}{n} = \UnitC{l}{n}
  \otimes \id_{\Gru{V}{m}{n}}$ and
  $\phi(\UnitC{l}{l})=1$ give $\lambda \cdot  \sum_{k} \Tr(\Gr{G}{k}{l}{}{\mathscr{X}}) = 1.$
Now all assertions in (1)--(3) concerning $d_G$ follow.
\end{proof}

Note that for semi-simple tensor categories with duals, any object is isomorphic to its left bidual (and this holds as well for tensor categories with local units by a local argument). Hence there always exists an $F_{\mathscr{X}}$ as in the previous theorem. 

\begin{Cor}\label{CorOrth}
  Let $\mathscr{A}$ be a regular partial Hopf algebra with an invariant integral $\phi$, let
  $(V,\mathscr{X})$ be an irreducible rcfd corepresentation of
  $\mathscr{A}$, let $F_{\mathscr{X}}$ be an isomorphism from
  $(V,\mathscr{X})$ to $(V,\dualco{\dualco{\mathscr{X}}})$ and
  $G_{\mathscr{X}}=F^{-1}_{{\mathscr{X}}}$, and let $a=(\id \otimes
  \omega_{f,v})(\Gr{X}{k}{l}{m}{n})$ and $b=(\id \otimes
  \omega_{g,w})(\Gr{X}{m}{n}{k}{l})$, where 
  $f \in   \dual{(\Gru{V}{k}{l})}$, $v \in\Gru{V}{m}{n}$, $g \in
  \dual{(\Gru{V}{m}{n})}$, $w \in  \Gru{V}{k}{l}$.  Then
\begin{align*}
  \phi(S(b)a) &= \frac{(g|v)(f|G_{\mathscr{X}}w)}{\sum_{r}
    \Tr(\Gr{G}{r}{n}{}{\mathscr{X}})}, & \phi(aS(b)) = \frac{(g|F_{\mathscr{X}}v)(f|w)}{\sum_{s}
    \Tr(\Gr{F}{m}{s}{}{\mathscr{X}})}.
\end{align*}
\end{Cor}
\begin{proof}
Apply $\omega_{g,w} \otimes
    \omega_{f,v}$ to the formulas in  Theorem
    \ref{thm:rep-orthogonality}.(c).
\end{proof}



\begin{Cor} \label{cor:rep-pw}
  Let $\mathscr{A}$ be a partial Hopf algebra with an invariant integral and let
  $((V_{i},\mathscr{X}_{i}))_{i \in \mathcal{I}}$ be a maximal family of mutually non-isomorphic irreducible rcfd corepresentations of
  $\mathscr{A}$. Then the map
  \begin{align*}
    \bigoplus_{i \in \mathcal{I}} \bigoplus_{k,l,m,n}
    (\dual{(\Gr{V}{k}{l}{}{i})} \otimes
    \Gr{V}{m}{n}{}{i}) \to A
  \end{align*}
  that sends $f \otimes w \in
  \dual{(\Gr{V}{k}{l}{}{\alpha})} \otimes
  \Gr{V}{m}{n}{}{i}$ to $ (\id \otimes
  \omega_{f,w})(\Gr{(X_{i})}{k}{l}{m}{n})$,
  is a linear isomorphism. 
\end{Cor}
\begin{proof} This follows from Proposition \ref{prop:rep-weak-pw}, Proposition \ref{prop:rep-orthogonality-1} and Corollary \ref{CorOrth}.
\end{proof}

In particular, it follows from the previous two corollaries that an invariant integral $\phi$ is faithful, i.e.~ $\phi(ab)=0$ or $\phi(ba)=0$ for all $b$ implies $a=0$. This can also be proven more directly.

\begin{Cor} \label{cor:rep-pw-morphisms}
  Let $\mathscr{A}$ be a regular partial Hopf algebra with an invariant integral, let
  $((V_{i},\mathscr{X}_{i}))_{i\in \mathcal{I}}$ be a maximal
  family of mutually non-isomorphic irreducible rcfd corepresentations of $\mathscr{A}$,
  fix $i \in \mathcal{I}$ and $k,l\in I$, and denote by $\Gr{\mathscr{Y}}{k}{l}{}{i}$
  the regular corepresentation on
  $\Grd{\mathcal{C}(\mathscr{X}_i)}{k}{l}$. Then there exists a
  linear isomorphism
  \begin{align*}
    \dual{( \Gru{V}{k}{l})} \to
    \Mor((V_{i},\mathscr{X}_{i}),
    (\Grd{\mathcal{C}(\mathscr{X}_i)}{k}{l},\Gr{\mathscr{Y}}{k}{l}{}{i}))
  \end{align*}
  assigning to each $f\in     \dual{( \Gru{V}{k}{l})}$ the morphism
  $T_{(f)}$ of Lemma \ref{lemma:rep-regular-embedding}.
\end{Cor}



\subsection{Unitary corepresentations of partial compact quantum groups}


Let us write $B(\Hsp,\mathcal{G})$ for the space of
bounded morphisms between Hilbert spaces $\Hsp$ and $\mathcal{G}$. 

\begin{Def} Let $\mathscr{A}$ define a partial compact quantum
  group. An rcfd corepresentation $\mathscr{X}$ of $\mathscr{A}$ on a collection of Hilbert spaces $\Gru{\Hsp}{k}{l}$ is called
   \emph{unitary}
  if $\Gr{(X^{-1})}{k}{l}{m}{n}=(\Gr{X}{l}{k}{n}{m})^{*}$ as elements in $\Gr{A}{k}{l}{m}{n}\otimes
  B(\Gru{\Hsp}{l}{k},\Gru{\Hsp}{n}{m}).$
\end{Def} 

For example, viewing $\C^{(I)}$ as a direct sum of the trivial Hilbert spaces $\C$, the
  trivial corepresentation $\mathscr{U}$ on $\C^{(I)}$ is unitary. Moreover, also the tensor product of rcfd corepresentations lifts to a tensor product
of unitary corepresentations as follows.  We hence obtain a tensor C$^*$-category $\Corep_{u,\rcf}(\mathscr{A})$ of unitary rcfd corepresentations. We denote again by $\Corep_u(\mathscr{A})$ the subcategory of all corepresentations with finite support on the hyperobject set. It is the total tensor C$^*$-category with local units of a semi-simple partial tensor C$^*$-category.

\begin{Lem} \label{lemma:rep-regular-unitary}
  Let $\mathscr{A}$ define a partial compact quantum group with
positive invariant  integral $\phi$, and let $\Gru{V}{m}{n} \subseteq
\bigoplus_{k,l} \Gr{A}{k}{l}{m}{n}$ be subspaces such that
$\Delta_{pq}(\Gru{V}{m}{n}) \subseteq \Gru{V}{p}{q} \otimes
    \Gr{A}{p}{q}{m}{n}$ and $V=\bigoplus_{k,l} \Gru{V}{k}{l}$ is rcfd. Then each $\Gru{V}{k}{l}$ is a Hilbert space with
    respect to the inner product given by $\langle
    a|b\rangle:=\phi(a^{*}b)$, and the regular corepresentation
    $\mathscr{X}$ on $V$ is unitary.
\end{Lem}
\begin{proof}   Let  $a\in \Gru{\Hsp}{m}{n}$, $b\in \Gru{\Hsp}{m}{n'}$ and define $\omega_{b,a} \colon
\Hom_{\C}(\Gru{\Hsp}{m}{n},\Gru{\Hsp}{m}{n}) \to \C$ by $T
\mapsto \langle b|Ta\rangle$. Then
\begin{eqnarray*}
\sum_{k }(\id \otimes \omega_{b,a})
((\Gr{X}{k}{l}{m}{n'})^* \Gr{X}{k}{l}{m}{n}))  &=& \sum_k
(\id\otimes \phi)(\Delta_{kl}^{\op}(b)^*\Delta_{kl}^{\op}(a))
  = \sum_k (\phi\otimes
  \id)(\Delta_{lk}(b^*)\Delta_{kl}(a)) \\ &=& (\phi\otimes
  \id)(\Delta_{ll}(b^*a))  = \phi(b^*a)\UnitC{l}{n} =
  \delta_{n',n} \UnitC{l}{n} \otimes \langle b|a\rangle.
\end{eqnarray*} Hence $ \sum_{k}
    (\Gr{X}{k}{l}{m}{n'})^* \Gr{X}{k}{l}{m}{n} =
    \delta_{n,n'}\UnitC{l}{n}\otimes
    \id_{\Gru{\Hsp}{m}{n}},$ and $\mathscr{X}$ is unitary by definition of the generalized inverse of a corepresentation.
\end{proof} 

\begin{Prop} \label{prop:rep-unitarisable} Every  rcfd
  corepresentation of a partial compact quantum group $\mathscr{A}$ is
  isomorphic to a unitary rcfd corepresentation.
\end{Prop}
\begin{proof}
  By Proposition \ref{prop:rep-cosemisimple}, it suffices to prove the
  assertion for every rcfd corepresentation $(V,\mathscr{X})$ that is
  irreducible.  For some $k,l$ and $f \in
  \dual{(\Gru{V}{k}{l})}$, the operator $T_{(f)}$ defined in
  Lemma \ref{lemma:rep-regular-embedding} has to be non-zero and
  hence, by irreducibility, injective. Thus, it forms an equivalence
  between $(V,\mathscr{X})$ and a restriction of the regular
  corepresentation on $\Grd{\mathcal{C}(\mathscr{X})}{k}{l}$, which is
  unitary by Lemma \ref{lemma:rep-regular-unitary}.
\end{proof}
\begin{Cor} The partial C$^*$-tensor category $\Corep_u(\mathscr{A})$ is a partial fusion category.
\end{Cor}

\begin{Prop} \label{prop:rep-unitary-bidual}
  Let $\mathscr{A}$ define a partial compact quantum group and let
  $(\Hsp,\mathscr{X})$ be an irreducible unitary rcfd corepresentation of
  $\mathscr{A}$.  Then there exists an isomorphism $F=F_{\mathscr{X}}$
  from $(\Hsp,\mathscr{X})$ to 
  $(\Hsp,(S^{2} \otimes \id)(\mathscr{X}))$ in $\Corep(\mathscr{A})$ such
  that each $\Gru{F}{k}{l}$ is positive.
\end{Prop}
\begin{proof}
 By Proposition \ref{prop:rep-unitarisable}, there exists an
  isomorphism $T \colon \dualco{\mathscr{X}} \to \mathscr{Y}$ for some
  unitary rcfd corepresentation $\mathscr{Y}$ on $\dual{\Hsp}$, so that in total form,
  $(1\otimes T)\dualco{X} = Y(1 \otimes T)$.
We  apply   $S \otimes -^{\tr}$ and $-^{*} \otimes -^{*\tr}$,
respectively to find $ \dualco{\dualco{X}}(1 \otimes \dualop{T}) = (1 \otimes
  \dualop{T})\dualco{Y}$ and $(1 \otimes T^{*\tr})X=\dualco{Y}(1\otimes T^{*\tr}).$

Combining both equations, we
find $\dualco{\dualco{X}}(1 \otimes \dualop{T}T^{*\tr})=(1 \otimes
\dualop{T}T^{*\tr})X$. Thus, we can take
$F_{\mathscr{X}}:=\dualop{T}T^{*\tr}$.
\end{proof}

The Schur orthogonality relations in Corollary \ref{CorOrth} can be
rewritten using the involution. If $v \in \Gru{\Hsp}{k}{l}$, $v' \in \Gru{\Hsp}{m}{n}$
and $\omega_{v,v'}(T) = \langle v|Tv'\rangle$, then 
\begin{align*}
  S((\id \otimes \omega_{v,v'})(\Gr{X}{k}{l}{m}{n})) &=
  (\id \otimes \omega_{v,v'}) (\Gr{(X^{-1})}{n}{m}{l}{k})) \\ & =
  (\id \otimes \omega_{v,v'})( (\Gr{X}{m}{n}{k}{l})^{*}) =
  (\id \otimes \omega_{v',v})(\Gr{X}{m}{n}{k}{l})^{*}.
\end{align*}
This equation and Corollary \ref{CorOrth} imply the following corollary.
\begin{Cor}\label{cor:rep-unitary-schur-orthogonality}
  Let $\mathscr{A}$ define a partial compact quantum group with
  positive invariant integral $\phi$, let $(\Hsp,\mathscr{X})$ be an irreducible
  unitary rcfd corepresentation of $\mathscr{A}$, let $F_{\mathscr{X}}$ be a positive
  isomorphism from $(\Hsp,\mathscr{X})$ to
  $(\Hsp,\dualco{\dualco{\mathscr{X}}})$ and
  $G_{\mathscr{X}}=F^{-1}_{{\mathscr{X}}}$, and let $a=(\id \otimes
  \omega_{v,v'})(\Gr{X}{k}{l}{m}{n})$ and $b=(\id \otimes
  \omega_{w,w'})(\Gr{X}{k}{l}{m}{n})$, where $v,w \in
  \Gru{\Hsp}{k}{l}$ and $v',w' \in \Gru{\Hsp}{m}{n}$.  Then
\begin{align*}
  \phi(b^{*}a) &= \frac{\langle w|v'\rangle\langle v|G_{\mathscr{X}}w'\rangle}{\sum_{r}
    \Tr(\Gr{G}{r}{n}{}{\mathscr{X}})}, & \phi(ab^{*}) = \frac{\langle
    w|F_{\mathscr{X}}v'\rangle \langle v|w'\rangle}{\sum_{s}
    \Tr(\Gr{F}{m}{s}{}{\mathscr{X}})}.
\end{align*}
\end{Cor}
As a consequence of Proposition \ref{prop:rep-weak-pw} and Proposition
\ref{prop:rep-unitarisable} or Lemma \ref{lemma:rep-regular-unitary},
the matrix coefficients of irreducible unitary rcfd corepresentations
span $\mathscr{A}$, and in the Corollary \ref{cor:rep-pw}, we may
assume the irreducible rcfd corepresentations
$(V_{i},\mathscr{X}_{i})$ to be unitary if $\mathscr{A}$
defines a partial compact quantum group.

\begin{Rem}\label{RemPos} In fact, Proposition \ref{prop:rep-unitary-bidual} and Corollary \ref{cor:rep-unitary-schur-orthogonality} show the following. Let $\mathscr{A}$ be a partial Hopf $^*$-algebra admitting an invariant integral $\phi$, which a priori we do not assume to be positive. Suppose however that each irreducible corepresentation of $\mathscr{A}$ is equivalent to a unitary corepresentation. Then $\phi$ is necessarily positive.
\end{Rem} 

\subsection{Analogues of Woronowicz's  characters}

Let $\mathscr{A}$ be a partial bialgebra, and $a\in \Gr{A}{k}{l}{m}{n}$. Then for $\omega \in A^*$, we can define
$\omega \aste{p,q} a = (\id \otimes \omega) (\Delta_{pq}(a)),$ and $a \aste{r,s}
\omega:=(\omega \otimes \id)(\Delta_{rs}(a)).$ Clearly we can define
$ \omega \aste{p,q} a \aste{r,s}
\omega':= (\omega \aste{p,q} a)\aste{r,s} \omega' = \omega \aste{p,q}(a \aste{r,s} \omega').$
When $\omega$ has support on the $\Gr{A}{k}{l}{k}{l}$, we can write, for $a\in \Gr{A}{k}{l}{m}{n}$, $\omega\ast a := \sum_{p,q} \omega\aste{p,q}a = \omega\aste{m,n}a,$ and $a\ast \omega = \sum_{r,s} a\aste{r,s}\omega = a\aste{k,l}\omega.$ 

We recall that an entire function $f$ has \emph{exponential growth
  on the right half-plane} if there exist $C,d>0$ such that $|f(x+iy)|\leq
C\mathrm{e}^{dx}$  for all $x,y\in \R$ with $x>0$. 

\begin{Theorem} \label{thm:rep-characters} Let $\mathscr{A}$ be a
  partial Hopf algebra with an invariant integral $\phi$.  Then there
  exists a unique family of linear functionals $f_{z} \colon A\to \C$
  such that
\begin{enumerate}[label={(\arabic*)}]
  \item each $f_z$ has support on $\sum_{k,l}\Gr{A}{k}{l}{k}{l}$.
  \item for each $a\in A$, the function $z\mapsto f_{z}(a)$ is entire
    and of exponential growth on the right half-plane.
  \item $f_{0} = \epsilon$ and $(f_{z} \otimes f_{z'}) \circ 
    \Delta= f_{z+z'}$ for all $z,z' \in \C$.
  \item $\phi(ab)=\phi(b(f_{1} \ast a \ast f_{1}))$ for all $a,b\in A$.
  \end{enumerate}
  This family furthermore satisfies
  \begin{enumerate}[label={(\arabic*)}]\setcounter{enumi}{4}
  \item $f_z(ab) = f_z(a)f_z(b)$ for $a\in \Gr{A}{k}{l}{m}{n},b\in \Gr{A}{l}{p}{n}{q}$. 
  \item $S^{2}(a)=f_{-1} \ast a \ast f_{1}$ for all $a\in A$.
  \item $f_{z}(\UnitC{l}{n})=\delta_{l,n}$ and $f_{z} \circ S = f_{-z}$ for all $a\in A$.
  \item $\bar{f}_{z}=f_{-\overline{z}}$ if $\mathscr{A}$ is a partial
    Hopf $^*$-algebra and $\phi$ is positive.
\end{enumerate}
\end{Theorem}


Note that conditions (3), (4) and (6) are meaningful by condition (1).

\begin{proof}
  We first prove uniqueness.  Assume that $(f_{z})_{z}$ is a family of
  functionals satisfying (1)--(4).  Since $\phi$ is faithful, the map
  $\sigma\colon a \mapsto f_{1} \ast a \ast f_{1}$ is uniquely
  determined by $\phi$, and one easily sees that it is a homomorphism. Using
  (3), we find that $\epsilon \circ \sigma^n=f_{2n}$, which uniquely determines these functionals. Using (2) and the
  fact that every entire function of exponential growth on the right
  half-plane is uniquely determined by its values at $\N \subseteq \C$, we can conclude that the family $f_{z}$ is uniquely determined. Moreover, since the property (5) holds for $z = 2n$, we also conclude by the same argument as above that it holds for all $z\in \C$.

  Let us now prove existence.  By Theorem \ref{thm:rep-orthogonality}, Corollary \ref{cor:rep-pw} and Proposition \ref{prop:rep-unitary-bidual}, we can
  define for each $z\in \C$ a functional $f_{z} \colon A \to \C$ such
  that for every 
  irreducible rcfd corepresentation
  $(V,\mathscr{X})$ in $\Corep(\mathscr{A})$, $ f_{z}((\id \otimes \omega_{\xi,\eta})(\Gr{X}{k}{l}{m}{n})) =
      \delta_{k,m}\delta_{l,n} \cdot
      \omega_{\xi,\eta}((\Gr{F}{k}{l}{}{\mathscr{X}})^{z})$ for all $\xi \in \Gru{V}{k}{l},\eta \in
      \Gru{V}{m}{n}$. This is equivalent with $(f_{z} \otimes \id)(\Gr{X}{k}{l}{m}{n}) =
      \delta_{k,m}\delta_{l,n} \cdot (\Gru{(F_{\mathscr{X}})}{k}{l})^{z}$, 
    where $F_{\mathscr{X}}$ is a non-zero operator implementing a morphism from $(V,\mathscr{X})$ to
    $(V, \dualco{\dualco{\mathscr{X}}})$, scaled such that
    $d_{\mathscr{X}}:= \sum_{r} \Tr(\Gru{(F_{\mathscr{X}}^{-1})}{r}{l}) = \sum_{s}
      \Tr(\Gru{(F_{\mathscr{X}})}{m}{s})$
    for all $l$ in the right and all $m$ in the left hyperobject support of $\mathscr{X}$. By
    construction, (1) and (2) hold. We show that the $(f_{z})_{z}$ satisfy the
    assertions (3)--(7). 

        Throughout the following arguments, let 
    $(V,\mathscr{X})$ be an  irreducible corepresentation
    $(V,\mathscr{X})$ and let $F=F_{\mathscr{X}}$ be as above.

    We first prove property (3). This follows from the relations $(f_{0}  \otimes \id)(\Gr{X}{k}{l}{m}{n}) =
      \delta_{k,m}\delta_{l,n} \id_{\Gru{V}{k}{l}} =
      (\epsilon \otimes \id)(\Gr{X}{k}{l}{m}{n})$ and the computation
    \begin{align*}
      (((f_{z}\otimes f_{z'})\circ \Delta) \otimes
      \id)(\Gr{X}{k}{l}{m}{n}) &=  \delta_{k,m}\delta_{l,n}(f_{z} \otimes f_{z'} \otimes
      \id)\big((\Gr{X}{k}{l}{k}{l})_{13}
      (\Gr{X}{k}{l}{k}{l})_{23}\big) \\
      &=  \delta_{k,m}\delta_{l,n}(\Gru{F}{k}{l})^{z}  \cdot (\Gru{F}{k}{l})^{z'} = (f_{z+z'} \otimes \id)(\Gr{X}{k}{l}{m}{n}).
    \end{align*}
    Applying slice maps of the form $\id
    \otimes \omega_{\xi,\xi'}$ and invoking Theorem \ref{thm:rep-orthogonality}, this proves (3).

To prove (4), write $ \Delta^{(2)} = (
    \Delta \otimes \id)\circ  \Delta = (\id \otimes 
    \Delta) \circ \Delta$, and put $\theta_{z,z'}:=(f_{z'} \otimes \id
    \otimes f_{z})\circ  \Delta^{(2)}.$ Then
    \begin{align*}
      (\theta_{z,z'} \otimes \id)(\Gr{X}{k}{l}{m}{n}) &= (f_{z'} \otimes
      \id \otimes f_{z} \otimes
      \id)((\Gr{X}{k}{l}{k}{l})_{14}(\Gr{X}{k}{l}{m}{n})_{24}(\Gr{X}{m}{n}{m}{n})_{34})
      \\
      &= (1 \otimes (\Gru{F}{k}{l})^{z'}) \Gr{X}{k}{l}{m}{n} (1
      \otimes (\Gru{F}{m}{n})^{z}).
    \end{align*}
    We take $z=z'=1$, use Theorem \ref{thm:rep-orthogonality}, where
    now $d_F= d_G=d_{\mathscr{X}}$ by our scaling of $F$, and obtain
    \begin{eqnarray*}
     && \hspace{-2cm} (\phi \otimes \id \otimes
      \id)((\Gr{(X^{-1})}{l}{k}{n}{m})_{12}((\theta_{1,1} \otimes
      \id)(\Gr{X}{k}{l}{m}{n}))_{13})\\ && =d_{\mathscr{X}}^{-1}(\id \otimes
      \Gru{F}{k}{l}) (\id \otimes \Gru{(F^{-1})}{k}{l})
      \Sigma_{k,l,m,n} (\id \otimes
      \Gru{F}{m}{n}) \\
      &&=d_{\mathscr{X}}^{-1}(\Gru{F}{m}{n} \otimes \id) \Sigma_{k,l,m,n} \\
      &&= (\phi \otimes \id \otimes
      \id)((\Gr{X}{k}{l}{m}{n})_{13}(\Gr{(X^{-1})}{l}{k}{n}{m})_{12}).
    \end{eqnarray*}
    To conclude the proof of assertion (4), apply again slice maps of the form
    $\omega_{\xi,\xi'} \otimes \omega_{\eta,\eta'}$.

We have then already argued that the property (5) automatically holds. To show the property (6), note that by Proposition \ref{prop:rep-unitary-bidual} and the calculation above,
    \begin{align*}
      (S^{2} \otimes \id)(\Gr{X}{k}{l}{m}{n}) &= (1
      \otimes\Gru{F_{\mathscr{X}}}{k}{l})
      \Gr{X}{k}{l}{m}{n}(1 \otimes \Gru{F_{\mathscr{X}}}{m}{n})^{-1} 
      =(\theta_{-1,1}  \otimes \id)(\Gr{X}{k}{l}{m}{n}).
    \end{align*}
     Assertion (6) follows again by applying slice maps.
    
     To check, (7), note that (1), (2) and (4) immediately imply
     $f_{z}(\UnitC{k}{m})=\delta_{k,m}$. As both $z \rightarrow
     f_{-z}$ and $z\rightarrow f_z\circ S$ satisfy the conditions
     (1)--(4) for $\mathscr{A}$ with the opposite product and
     coproduct (using the partial character property (5) and the
     invariance of $\phi$ with respect to $S$), we find $f_{-z} =
     f_{z} \circ S$.

     Finally, we assume that $\mathscr{A}$ is a partial Hopf
     $^*$-algebra with positive invariant integral $\phi$ and prove
     (8).  By Proposition \ref{prop:rep-unitary-bidual}, we can assume
     $\Gr{F}{k}{l}{}{\mathscr{X}}$ to be positive.  Write
     $\bar{f}_z(a) = \overline{f_z(a^*)}$. Using the relations $
     (\Gr{X}{k}{l}{k}{l})^{*}=(S \otimes \id)(\Gr{X}{k}{l}{k}{l})$,
     $f_{z} \circ S=f_{-z}$  and
     positivity of $\Gr{F}{k}{l}{}{\mathscr{X}}$ (Proposition
     \ref{prop:rep-unitary-bidual}), we easily compute that $\bar{f}_z(a) = f_{-\overline{z}}(a)$ for all $a\in
\Gr{\mathcal{C}(\mathscr{X})}{k}{l}{k}{l}$. Since $f_{z}$ and
$f_{-\overline{z}}$ vanish on $\Gr{A}{k}{l}{m}{n}$ if $(k,l)\neq
(m,n)$ and the matrix coefficients of unitary 
corepresentations span $A$, we can conclude $\bar{f}_{z}=f_{-\overline{z}}$.
\end{proof}




% Correct writing Krein

\section{Tannaka-Krein-Woronowicz duality for partial compact quantum groups}

% See `Galois reconstruction of finite quantum groups' of Bichon for additional refs

In the previous section, we showed how any partial compact quantum group gave rise to a partial fusion C$^*$-category with a unital morphism into a partial tensor C$^*$-category of finite dimensional Hilbert spaces. In this section we reverse this construction, and show that the two structures are in duality with each other. The proof does not differ much from the usual Tannaka-Krein reconstruction process, but one has to pay some extra care to the well-definedness of certain constructions. Implicitly, we build our reconstruction process by passing first through the construction of the discrete dual of a partial compact quantum group, which we however refrain from formally introducing.

Let us at first fix an indecomposable semi-simple partial tensor category $\CatCC$ over a base set $\mathscr{I}$. We will again view the tensor product of $\CatC$ as being strict, for notational convenience. 

Assume that we also have another set $I$ and a partition $I = \{I_\alpha\mid \alpha\in \mathscr{I}\}$ with associated \emph{surjective} function \[\varphi:I\rightarrow \mathscr{I}, \quad k\mapsto k'.\] Let $F: \CatCC\rightarrow \{\Vect_{\fin}\}_{I\times I}$ be a morphism based on $\varphi$, cf.~ Example \ref{ExaVectBiGr}.  We will again denote by $F_{kl}:\CatCC_{k'l'}\rightarrow \Vect_{\fin}$ the components of $F$ at index $(k,l)$, and by $\iota$ and $\mu$ resp.~ the product and unit constraints.  For $X\in \CatC_{k'\beta}$ and $Y\in \CatC_{\beta m'}$, we write the projection maps associated to the identification $F_{km}(X\otimes Y)\cong \oplus_{l\in I_\beta} \left(F_{kl}(X)\otimes F_{lm}(Y)\right)$ as \[\pi^{(klm)}_{X,Y}:F_{km}(X\otimes Y) \rightarrow F_{kl}(X)\otimes F_{lm}(Y).\]

We choose a maximal family of mutually inequivalent irreducible objects $\{u_a\}_{a\in \mathcal{I}}$ in $\CatC$. We assume that the $u_a$ include the unit objects $\Unitb_{\alpha}$ for $\alpha\in \mathscr{I}$, and we may hence identify $\mathscr{I}\subseteq \mathcal{I}$. For $a\in \mathcal{I}$, we will write $u_a \in \CatC_{\lambda_a,\rho_a}$ with $\lambda_a,\rho_a\in \mathscr{I}$. For $\alpha,\beta\in \mathscr{I}$ fixed, we write $\mathcal{I}_{\alpha\beta}$ for the set of all $a\in \mathcal{I}$ with $\lambda_a=\alpha$ and $\rho_a=\beta$. When $a,b,c\in \mathcal{I}$ with $a\in \mathcal{I}_{\alpha\beta},b\in \mathcal{I}_{\beta\gamma}$ and $c\in \mathcal{I}_{\gamma\delta}$, we write $c\leq a\cdot b$ if $\Mor(u_c,u_a\otimes u_b)\neq \{0\}$. Note that with $a,b$ fixed, there is only a finite set of $c$ with $c\leq a\cdot b$. We also use this notation for multiple products.

\begin{Def} For $a\in \mathcal{I}$ and $k,l,m,n\in I$, define vector spaces \[\Gr{A}{k}{l}{m}{n}(a) =  \delta_{k,m,\lambda_a}\delta_{l,n,\rho_a} \Hom_{\C}(F(u_a)_{mn},F(u_a)_{kl})^*.\] Write \[\Gr{A}{k}{l}{m}{n} =\underset{a\in \mathcal{I}}{\oplus}\, \Gr{A}{k}{l}{m}{n}(a),\quad A(a) = \underset{k,l,m,n}{\oplus} \Gr{A}{k}{l}{m}{n}(a),\quad A = \underset{k,l,m,n}{\oplus} \Gr{A}{k}{l}{m}{n}.\] 
\end{Def} 

We first turn the $\Gr{A}{k}{l}{m}{n}$ into a partial coalgebra $\mathscr{A}$ over $I^2$.

\begin{Def} For $r,s\in I$, we define \[\Delta_{rs}: \Gr{A}{k}{l}{m}{n}\rightarrow \Gr{A}{k}{l}{r}{s}\otimes \Gr{A}{r}{s}{m}{n}\] as the direct sums of the duals of the composition maps \[\Hom_{\C}(F_{rs}(u_a),F_{kl}(u_a)) \otimes \Hom_{\C}(F_{mn}(u_a),F_{rs}(u_a))\rightarrow \Hom_{\C}(F_{mn}(u_a),F_{kl}(u_a)),\]\[x\otimes y \mapsto x\circ y.\]
\end{Def} 

\begin{Lem} The couple $(\mathscr{A},\Delta)$ is a partial coalgebra with counit map \[\epsilon:\Gr{A(a)}{k}{l}{k}{l}\rightarrow \C,\quad f\mapsto f(\id_{F_{kl}(u_a)}).\] Moreover, for each fixed $f\in \Gr{A(a)}{k}{l}{m}{n}$, the matrix $\left(\Delta_{rs}(f)\right)_{rs}$ is rcf.
\end{Lem} 
\begin{proof} Coassociativity and counitality are immediate by duality, as for each $a$ fixed the $\Hom_{\C}(F_{mn}(u_a),F_{kl}(u_a))$ form a partial algebra with units $\id_{F_{kl}(u_a)}$. The rcf condition follows immediately from the rcf condition for the morphism $F$.
\end{proof}

In the next step, we define a partial algebra structure on $\mathscr{A} = \{\Gr{A}{k}{l}{m}{n}\mid k,l,m,n\}$. First note that we can identify \[\Nat(F_{mn},F_{kl}) \cong \underset{\rho_a=l'=n'}{\underset{\lambda_a=k'=m'}{\prod_a}} \Hom_{\C}(F_{mn}(u_a),F_{kl}(u_a)),\] where $\Nat(F_{mn},F_{kl})$ denotes the space of natural transformations from $F_{mn}$ to $F_{kl}$ when $k'=m'$ and $l'=n'$. Similarly, we can identify \[\Nat(F_{mn}\otimes F_{pq},F_{kl}\otimes F_{rs}) \cong  \prod_{b,c} \Hom_{\C}(F_{mn}(u_b)\otimes F_{pq}(u_c) ,F_{kl}(u_b)\otimes F_{rs}(u_c)),\] with the product over the appropriate index set and where \[F_{kl}\otimes F_{rs}:\CatC_{k'l'}\times \CatC_{r's'}\rightarrow \Vect_{\fin},\quad (X,Y) \mapsto F_{kl}(X)\otimes F_{rs}(Y).\] As such, there is a natural pairing of these spaces with resp.~ $\Gr{A}{k}{l}{m}{n}$ and $\Gr{A}{k}{l}{m}{n}\otimes \Gr{A}{r}{s}{p}{q}$. 

\begin{Def} For $k'=r', l'=s'$ and $m'=t'$, we define a product map \[M:\Gr{A}{k}{l}{r}{s} \otimes \Gr{A}{l}{m}{s}{t}\rightarrow  \Gr{A}{k}{m}{r}{t},\quad f\otimes g \mapsto f\cdot g\] by the formula \[(f\cdot g)(x) = (f\otimes g)( \hat{\Delta}^{l}_{s}(x)), \qquad  x \in \Nat(F_{rt},F_{km}),\] where $\hat{\Delta}^l_s(n)$ is the natural transformation\[\hat{\Delta}^l_s(x):  F_{rs}\otimes F_{st}\rightarrow F_{kl}\otimes F_{lm},\quad \hat{\Delta}^l_s(x)_{X,Y} = \pi^{(klm)}_{X,Y}x_{X\otimes Y} \iota^{(rst)}_{X,Y},\quad X\in \CatC_{k'l'},Y\in \CatC_{l'm'}.\]
\end{Def}

\begin{Rem} It has to be argued that $f\cdot g$ has finite support (over $\mathcal{I})$ as a functional on $\Nat(F_{rt},F_{km})$. In fact, if $f$ is supported at $b\in \mathcal{I}_{r's'}$ and $g$ at $c\in \mathcal{I}_{s't'}$, then $f\cdot g$ has support in the finite set of $a\in \mathcal{I}_{r't'}$ with $a\leq b\cdot c$, since if $x$ is a natural transformation with support outside this set, one has $x_{u_b\otimes u_c}=0$, and hence any of the $\left(\hat{\Delta}^l_s(x)\right)_{u_b,u_c} =0$.
\end{Rem}

\begin{Lem} The above product maps turn $(\mathscr{A},M)$ into an $I^2$-partial algebra.
\end{Lem}
\begin{proof} We can extend the map $(\hat{\Delta}^l_s\otimes \id)$ on $\Nat(F_{rt},F_{km})\otimes \Nat(F_{tu},F_{mn})$ to a map \[(\hat{\Delta}^l_s\otimes \id): \Nat(F_{rt}\otimes F_{tu},F_{km}\otimes F_{mn}) \rightarrow  \Nat(F_{rs}\otimes F_{st}\otimes F_{tu},F_{kl}\otimes F_{lm}\otimes F_{mn}),\] \[(\hat{\Delta}^l_s\otimes \id)(x)_{X,Y,Z} = \left(\pi^{(klm)}_{X,Y}\otimes \id_{F_{mn}(Z)}\right) x_{X\otimes Y, Z} \left(\iota^{(rst)}_{X,Y} \otimes \id_{F_{tu}(Z)}\right).\]
By finite support, we then have that \[((f\cdot g)\cdot h)(x) = (f\otimes g\otimes h)((\hat{\Delta}^l_s\otimes \id)\hat{\Delta}^m_t(x))\] for all $f\in \Gr{A}{k}{l}{r}{s},g\in \Gr{A}{l}{m}{s}{t},h\in \Gr{A}{m}{n}{t}{u}$ and $x\in  \Nat(F_{ru},F_{kn})$. Similarly, \[(f\cdot g)\cdot h)(x) = (f\otimes g\otimes h)((\id\otimes \hat{\Delta}^m_t)\hat{\Delta}^l_s(x)).\] The associativity then follows from the 2-cocycle condition for the $\iota$- and $\pi$-maps. 

By a similar argument, one sees that the (non-zero) units are given by $\UnitC{k}{l}\in \Gr{A}{k}{k}{l}{l}(\Unitb_{\alpha})$  (for $\alpha=k'=l'$) corresponding to $1$ in the canonical identifications  \[\Gr{A}{k}{k}{l}{l}(\alpha) \cong \Hom_{\C}(F_{ll}(\Unitb_{\alpha}),F_{kk}(\Unitb_{\alpha}))^*\cong \Hom_{\C}(\C,\C)^*  \cong \C^* \cong \C.\] Indeed, to prove for example the right unit property, we use that (essentially) $\pi_{u_a,\Unitb_{\alpha}}^{(kll)} =(\id\otimes \mu_l)$ and $\iota_{u_a,\Unitb_{\alpha}}^{(kll)} = (\id\otimes \mu_l^{-1})$, while \[\UnitC{k}{l}(\mu_kx_{\Unitb_{\alpha}}\mu_l^{-1}) = x_{\Unitb_{\alpha}} \in \C,\quad x\in \Nat(F_{ll},F_{kk}).\] % to complete
\end{proof} 

\begin{Prop} The partial algebra and coalgebra structures on $\mathscr{A}$ define a partial bialgebra structure on $\mathscr{A}$. 
\end{Prop}
\begin{proof} Let us check the properties in Definition \ref{DefPartBiAlg}. Properties \ref{Propa} and \ref{Propc} are left to the reader. Property \ref{Propd} was proven above. Property \ref{Propb} follows from the fact that for $k'=l'=s'=m'$, \[\hat{\Delta}^{l}_s(\id_{F_{km}}) = \delta_{ls} \id_{F_{kl}}\otimes \id_{F_{lm}}.\] 
It remains to show the multiplicativity property \ref{Prope}. This is equivalent with proving that, for each $x\in \Nat(F_{uw},F_{km})$ and $y\in \Nat(F_{rt},F_{uw})$ (with all first or second indices in the same class of $\mathscr{I}$), one has (pointwise) that (for $l'=s'$) \[ \hat{\Delta}^l_s(x\circ y) = \sum_{v,v'=l'} \hat{\Delta}^v_s(x)\circ \hat{\Delta}^l_v(y).\] This follows from the fact that $\sum_v \pi^{(uvw)}_{X,Y}\iota^{(uvw)}_{X,Y} \cong \id_{F_{uw}(X\otimes Y)}$ (where we again note that the left hand side sum is in fact finite).
\end{proof} 

Let us show now that the resulting partial bialgebra $\mathscr{A}$ has an invariant integral.

\begin{Def} Define $\phi: \Gr{A}{k}{k}{m}{m} \rightarrow \C$ as the functional which is zero on $\Gr{A}{k}{k}{m}{m}(a)$ with $a\neq \Unitb_{k'}$, and the canonical identification $\Gr{A}{k}{k}{m}{m}(k')\cong \C$ on the unit component (for $k'=m'$).
\end{Def}

\begin{Lem} The functional $\phi$ is an invariant integral.
\end{Lem}

\begin{proof} The normalisation condition $\phi(\UnitC{k}{k})=1$ is immediate by construction. Let us check left invariance, as right invariance will follow similarly.

Let $\hat{\phi}^k_l$ be the natural transformation from $F_{ll}$ to $F_{kk}$ which has support on multiples of $\Unitb_{k'}$, and with $(\hat{\phi}^k_l)_{\Unitb_{k'}} = 1$.  Then for $f\in \Gr{A}{k}{k}{l}{l}$, we have $\phi(f) = f(\hat{\phi}^k_l)$. The left invariance of $\phi$ then follows from the easy verification that for $x\in \Nat(F_{ll},F_{kn})$, \[x\circ \hat{\phi}^l_m =\delta_{k,n} \UnitC{k}{l}(x)\hat{\phi}^k_m.\] 
\end{proof}

So far, we have constructed from $\CatCC$ and $F$ a partial bialgebra
$\mathscr{A}$ with invariant integral $\phi$. Let us further impose
for the rest of this section that $\CatCC$ admits duality.  We shall
use the following straightforward observation.
\begin{Lem}
  For all $k,l$ and $X\in \mathcal{C}_{k',l'}$,  the maps
  \begin{align*}
    \coev^{kl}_{X}  &:=  \pi^{(klk)}_{X,\hat X} \circ F_{kk}(\coev_{X})\colon \C \to F_{kl}(X)
    \otimes F_{lk}(\hat X), \\
    \ev^{kl}_{X} &:=  F_{ll}(\ev_{X}) \circ \iota^{(lkl)}_{\hat X,X} \colon
    F_{lk}(\hat X) \otimes F_{kl}(X) \to \C
  \end{align*}
  define a duality between $F_{kl}(X)$ and $F_{lk}(\hat X)$.
\end{Lem}


\begin{Prop}\label{PropAnti} The partial bialgebra $\mathscr{A}$ is a regular partial Hopf algebra.
\end{Prop} 

\begin{proof} 
  For any $x\in \Nat(F_{mn},F_{kl})$, let us define $\hat{S}(x) \in
  \Nat(F_{lk},F_{nm})$ by
\begin{align*}
  \hat{S}(x)_X &= % \big(\id \otimes F_{kk}(\ev_{X})\iota^{(klk)}_{\hat
  %   X,X}\big) \circ \big(\id \otimes x_{\hat X} \otimes \id\big) \circ
  % \big(\pi^{(nmn)}_{X,\hat X} F_{nn}(\coev_{X}) \otimes \id\big) \\
 % &= 
(\id \otimes \coev^{lk}_{X}) \circ (\id \otimes x_{\hat X}
  \otimes \id) \circ (\coev^{nm}_{X} \otimes \id).
\end{align*}
Then the assigment $\hat{S}$ dualizes to maps $S:\Gr{A}{k}{l}{m}{n} \rightarrow \Gr{A}{n}{m}{l}{k}$ by $S(f)(x) = f(\hat{S}(x))$. We claim that $S$ is an antipode for $\mathscr{A}$. 

Let us check for example the formula \[\sum_r f_{(1){\scriptscriptstyle \begin{pmatrix}k&l\\n & r\end{pmatrix}}} S(f_{(2){\scriptscriptstyle \begin{pmatrix} n & r \\ m & l\end{pmatrix}}}) = \delta_{k,m}\epsilon(f)\UnitC{k}{n}\] for $f\in \Gr{A}{k}{l}{m}{l}$. The other antipode identity follows similarly.

By duality, this is equivalent to the pointwise identity of natural transformations \[\sum_r\hat{M}^n_r(\id\otimes \hat{S})\hat{\Delta}^l_r(x) = \delta_{k,m}\UnitC{k}{n}(x) \id_{F_{kl}},\quad x\in \Nat(F_{nn},F_{km})\] where $\hat{M}^n_r$ and $(\id\otimes \hat{S})$ are dual to respectively $\Delta_{nr}$ and $\id\otimes S$. 

Let us fix $X\in \mathcal{C}_{k'l'}$. Then for any $x\in
\Nat(F_{nr},F_{kl})$, $y\in \Nat(F_{rn},F_{lm})$, we have
\begin{align*}
  \left(\hat{M}^n_r(\id\otimes \hat{S})(x\otimes y)\right)_X =
\big(\id \otimes \ev_{X}^{ml}\big)  \big(x_{X} \otimes y_{\hat X} \otimes \id\big) 
  \big(\coev_{X}^{nr} \otimes \id\big).
\end{align*}
For any $x\in \Nat(F_{nn},F_{km})$, we therefore have
\begin{align*}
  \left(\hat{M}^n_r(\id\otimes \hat{S})\hat\Delta^{l}_{r}(x)\right)_X &=
\big(\id \otimes \ev^{ml}_{X}\big)  \big(\pi^{(klm)}_{X,\hat X}x_{X\otimes \hat
    X}\iota^{(nrm)}_{X,\hat X} \otimes \id\big) 
  \big(\coev^{nr}_{X} \otimes \id\big).
\end{align*}
We sum over $r$, use naturality of $x$, and obtain
\begin{align*}
\sum_{r}    \left(\hat{M}^n_r(\id\otimes
  \hat{S})\hat\Delta^{l}_{r}(x)\right)_X &=
\big(\id \otimes \ev_{X}^{ml}\big) \big(\pi^{(klm)}_{X,\hat X}x_{X\otimes \hat
    X}F_{nn}(\coev_{X}) \otimes \id\big) \\
  &=\delta_{k,m} \UnitC{k}{n}(x)
\big(\id \otimes \ev_{X}^{ml}\big) 
\big(\pi^{(mlm)}_{X,\hat X}F_{mm}(\coev_{X})
  \otimes \id\big) \\
  &=\delta_{k,m} \UnitC{k}{n}(x)
\big(\id \otimes \ev_{X}^{ml}\big) 
\big(\coev^{ml}_{X}
  \otimes \id\big) \\
  &= \delta_{k,m} \UnitC{k}{n}(x) \id.
\end{align*}
Similarly, one shows that $\mathscr{A}$ with the opposite multiplication has an antipode, using right duality. It follows that $\mathscr{A}$ is a regular partial Hopf algebra.  
\end{proof} 

Now let us upgrade $\CatCC$ to a partial fusion C$^*$-category, and $F$ to a morphism from $\CatCC$ to $\{\Hilb_{\fin}\}_{I\times I}$. Then we can of course still form the partial Hopf algebra $\mathscr{A}$ as above. Let us show that it becomes a partial Hopf $^*$-algebra with positive invariant integral.

We first introduce a $^*$-structure on $\mathscr{A}$. 

\begin{Def} We define $^*: \Gr{A}{k}{l}{m}{n}\rightarrow \Gr{A}{l}{k}{n}{m}$ by the formula \[f^*(x) = \overline{f(\hat{S}(x)^*)},\qquad x\in \Nat(F_{nm},F_{lk}).\]
\end{Def}

\begin{Lem} The operation $^*$ is an anti-linear, anti-multiplicative, comultiplicative involution.
\end{Lem}

\begin{proof} Anti-linearity is clear. Comultiplicativity follows from the fact that $(xy)^* = y^*x^*$ and $\hat{S}(xy) = \hat{S}(y)\hat{S}(x)$ for natural transformations. To see anti-multiplicativity of $^*$, note first that, since $S$ is anti-multiplicative for $\mathscr{A}$, $\hat{S}$ is anti-comultiplicative on natural transformations. Now as $(\iota_{X,Y}^{(klm)})^* = \pi_{X,Y}^{(klm)}$ by assumption, we also have $\hat{\Delta}^l_s(x)^* = \hat{\Delta}^s_l(x^*)$, which proves anti-multiplicativity of $^*$ on $\mathscr{A}$.  Finally, involutivity follows from the involutivity of $x\mapsto \hat{S}(x)^*$, which is a consequence of the identity $d_{\hat{X}} = d_X^{\tr}$. 
\end{proof}

\begin{Prop} The couple $(\mathscr{A},\Delta)$ with the above $^*$-structure defines a partial compact quantum group.
\end{Prop}
\begin{proof} The only thing which is left to prove is that our
  invariant integral $\phi$ is a positive functional. Now it is easily
  seen from the definition of $\phi$ that the $\Gr{A}{k}{l}{m}{n}(a)$
  are all mutually orthogonal. Hence it suffices to prove that the
  sesquilinear inner product \[\langle f| g\rangle = \phi(f^*g)\] on
  $\Gr{A}{k}{l}{m}{n}(a)$ is positive-definite.

  Let us write $\bar{f}(x) = \overline{f(x^*)}$. Let again
  $\hat{\phi}^k_m$ be the natural transformation from $F_{mm}$ to
  $F_{kk}$ which is the identity on $\Unitb_{k'}$ and zero on other
  irreducible objects. Then by definition, \[\phi(f^*g) =
  (\bar{f}\otimes g)((\hat{S}\otimes
  \id)\hat{\Delta}^k_m(\hat{\phi}^l_n)).\] 

Assume  that $f(x) = \langle v'| x_a v\rangle$ and
  $g(x) = \langle w' | x_aw\rangle$ for $v,w\in F_{mn}(u_a)$ and
  $v',w'\in F_{kl}(u_a)$. Then
  $\overline{f}(x) = \langle v|x_{a} v'\rangle$ and
 using the expression for $\hat{S}$ as
  in Proposition \ref{PropAnti}, we find that
  \begin{align*}
    \phi(f^*g) &= \langle v \otimes w'|
    (\ev_{a}^{kl})_{23} 
    (\hat\Delta^{k}_{m}(\hat \phi^{l}_{n})_{\bar a, a})_{24} 
    (\coev^{mn}_{a})_{12} (v'\otimes
    w)\rangle.
  \end{align*}
  However, up to a positive non-zero scalar, which we may assume to be
  1 by proper rescaling, we
  have 
  \[\hat{\Delta}^k_m(\hat{\phi}^l_n)_{\bar{a}, a} =
  (\ev^{kl}_{a})^{*}(\ev^{kl}_{a}).\] Hence
  \begin{align*}
    \phi(f^*g) &=
\langle v \otimes w'|     (\ev^{kl}_{a})_{23}  (
  (\ev^{kl}_{a})^{*}(\ev^{kl}_{a}))_{24}
 (\coev^{mn}_{a})_{12} (v'\otimes w)\rangle \\
&= \langle v \otimes w'|        (\ev^{kl}_{a})_{23} 
  (\ev^{kl}_{a})^{*}_{24}
 (w\otimes v')\rangle \\
    &= \langle v|w\rangle (\ev^{kl}_{a}|v'\rangle_{2})
    (\ev_{a}^{kl}|w'\rangle_{2})^{*},
  \end{align*}
where $\ev_{a}^{kl}|z\rangle_{2}$ denotes the map $y \mapsto
\ev_{a}^{kl}(y\otimes z)$.
If $v=w$ and $v'=w'$, the expression above clearly becomes positives.
\end{proof} 

\begin{Theorem} \label{TheoTKPCQG}

The assigment $\mathscr{A}\rightarrow (\Corep_u(\mathscr{A}),F)$ is (up to isomorphism/equivalence) a one-to-one correspondence between partial compact quantum groups based over $\varphi:I\twoheadrightarrow \mathscr{I}$ and $\mathscr{I}$-partial fusion C$^*$-categories $\CatCC$ with unital morphism $F$ to $(\Hilb_f)_{I\times I}$ based over $\varphi$. 
\end{Theorem} 

\begin{proof} Fix first $\mathscr{A}$, and let $\mathscr{B}$ be the partial Hopf $^*$-algebra constructed from $\Corep_u(\mathscr{A})$ with its natural forgetful functor. Then we have a map $\mathscr{B} \rightarrow \mathscr{A}$ by \[ \Gr{B}{k}{l}{m}{n}(a) = \Hom(\Gr{V}{}{(a)}{m}{n},\Gr{V}{}{(a)}{k}{l})^* \rightarrow \Gr{A}{k}{l}{m}{n}(a):  f \mapsto (\id\otimes f)X_a,\] where the $(V^{(a)},X_a)$ run over all irreducible unitary corepresentations of $\mathscr{A}$. It is easy to check from the definition of $\mathscr{B}$ that this map is an morphism of partial Hopf $^*$-algebras. As the matrix coefficients of irreducible unitary corepresentations span $\mathscr{A}$, the map is surjective. By the Schur orthogonality relations, it is bijective.

Conversely, let $\CatCC$ be an $\mathscr{I}$-partial fusion C$^*$-category with unital morphism $F$ to $(\Hilb_f)_{I\times I}$ based over $\varphi$. Let $\mathscr{A}$ be the associated partial Hopf $^*$-algebra. For each irreducible $u_a \in \CatCC$, let $V^{(a)} = F(u_a)$, and \[\Gr{(X_a)}{k}{l}{m}{n} = \sum_i e_i^*\otimes e_i,\] where $e_i$ is a basis of $\Nat(F_{mn},F_{kl})$ and $e_i^*$ a dual basis. Then from the definition of $\mathscr{A}$ it easily follows that $X_a$ is a unitary corepresentation for $\mathscr{A}$. Since $\Nat(F_{mn},F_{kl})$ spans the total space of operators at any irreducible object, it follows that $X_a$ is irreducible. As the matrix coefficients of the $X_a$ span $\mathscr{A}$, it follows that the $X_a$ form a maximal class of non-isomorphic unitary corepresentations of $\mathscr{A}$. Hence we can make a unique equivalence \[\CatCC\rightarrow \Corep_u(\mathscr{A}), \quad u \mapsto (F(u),X_u)\] such that $u_a\rightarrow X_a$. From the definitions of the coproduct and product in $\mathscr{A}$, it is readily verified that the natural morphisms $\iota^{(klm)}_{u,v}:F_{kl}(u)\otimes F{lm}(v)\rightarrow F_{km}(u\otimes v)$ turn it into a monoidal equivalence. 
\end{proof}


\section{Examples}

\subsection{Classical examples}

\begin{Exa} Let $G$ be a discrete groupoid with object set $I = G^{(0)}$. Consider for $r,s\in I$ the vector space $\Gru{A}{r}{s}= \mathbb{C}\lbrack \Gru{G}{r}{s}\rbrack$ which has a basis $\{\lambda_g\}$ spanned by the morphisms $g$ from $r$ to $s$. Linearly extending the product of $G$ to the $\Gru{A}{r}{s}$ turns $\mathscr{A}$ into a partial algebra. It becomes a partial $^*$-algebra by putting $\lambda_g^* = \lambda_{g^{-1}}$. 

Extend the $I^2$-bigrading to an $I^4$-bigrading by putting $\Gr{A}{k}{l}{m}{n} = \delta_{kl}\delta_{mn}\Gru{A}{k}{l}$. Then together with the coproducts \[ \Delta: \Gru{A}{r}{s}\rightarrow \Gru{A}{r}{s}\otimes \Gru{A}{r}{s},\quad \lambda_g\mapsto \lambda_g\otimes \lambda_g,\] $(\mathscr{A},\Delta)$ defines a partial compact quantum group, the invariant functional $\phi$ being given by \[\phi(\lambda_g) = \delta_{rs}\delta_{g,\id_r},\quad r,s\in I,g\in \Mor(r,s).\]
\end{Exa}

\begin{Exa} Let $G$ be a proper locally compact groupoid with discrete object space $I=G^{(0)}$. Then each $\Gru{G}{r}{s} = \Mor(r,s)$ is a compact space. The $\Gru{G}{r}{r}$ are compact groups, hence come with probability Haar measures $\mu_{rr}$. The $\Gru{G}{r}{s}$ are $\Gru{G}{r}{r}$-$\Gru{G}{s}{s}$-bitorsors, and as such admit a bi-invariant probability measure $\mu_{rs}$. Let $A^r_s \subseteq C(\Gru{G}{r}{s})$ be the spaces of functions which transform as finite-dimensional representations of $\Gru{G}{r}{r}\times \Gru{G}{s}{s}$ under the natural actions. Then the $A^r_s$ form a partial coalgebra over $I$ by putting \[\Delta_{s}: A^r_t \rightarrow A^r_s\otimes A^s_t,\quad f\mapsto \left(\Delta_s(f):\Gru{G}{r}{s}\times \Gru{G}{s}{t}\mapsto \Gru{G}{r}{t},\quad (g,h)\mapsto f(gh)\right).\] We can extend the $I^2$-grading to an $I^4$-grading by putting \[\Gr{A}{k}{l}{m}{n} = \delta_{kl}\delta_{mn} A^l_n,\] and the resulting $\mathscr{A}$ becomes a partial $^*$-algebra by endowing each $A^r_s$ with the pointwise product and $^*$-algebra structure. The couple $(\mathscr{A},\Delta)$ then defines a partial compact quantum group with invariant integral \[\phi: A^r_s \mapsto \C,\quad f\mapsto \int_{\Gru{G}{r}{s}} f(g) \rd \Gru{\mu}{r}{s}(g).\]

\end{Exa}

\begin{Exa} Let $\mathscr{A}$ and $\mathscr{B}$ define two partial compact quantum groups $\mathscr{G}$ and $\mathscr{H}$ over respective sets $I$ and $J$. Then we can make a tensor product partial Hopf $^*$-algebra $\mathscr{A}\otimes \mathscr{B}$ over the index set $I\times J$ by putting \[\Gr{(A\otimes B)}{(k,k')}{(l,l')}{(m,m')}{(n,n')} = \Gr{A}{k}{l}{m}{n}\otimes \Gr{B}{k'}{l'}{m'}{n'}\] with the factorwise product and with coproducts \[\Delta_{(r,r'),(s,s')} = \sigma_{23}(\Delta_{rs}\otimes \Delta_{r',s'}),\] $\sigma$ being the switch map. It is easily seen that the tensor products of the positive invariant integrals for $\mathscr{A}$ and $\mathscr{B}$ produce a positive invariant integral on $\mathscr{A}\otimes \mathscr{B}$. Hence $\mathscr{A}\otimes \mathscr{B}$ defines a partial compact quantum group, which we will denote $\mathscr{G}\times \mathscr{H}$.
\end{Exa}

\subsection{Canonical partial compact quantum groups} \label{SubSecCan}

The following generalizes Hayashi's original construction.

\begin{Exa} 
Let $\CatCC$ be an $\mathscr{I}$-partial fusion C$^*$-category. Let $\mathcal{I}$ label a distinguished maximal set $\{u_k\}$ of mutually non-isomorphic irreducible objects of $\CatC$, with associated bigrading $\Gru{\mathcal{I}}{\alpha}{\beta}$ over $\mathscr{I}$. Define \[F_{kl}(X)  = \Hom(u_k,  X\otimes u_l),\qquad X\in \Gru{\CatC}{\alpha}{\beta}, k\in \Gru{\mathcal{I}}{\alpha}{\gamma},l\in \Gru{\mathcal{I}}{\beta}{\gamma}.\] Then each $F_{kl}(X)$ is a Hilbert space by the inner product $\langle f,g\rangle = f^*g$. Put $F_{kl}(X) = 0$ for $k,l$ outside their proper domains. Then clearly the application $(k,l)\mapsto F_{kl}(X)$ is rcf. Moreover, we have isometric compatibility morphisms \[F_{kl}(X)\otimes F_{lm}(Y)\rightarrow F_{km}(X\otimes Y),\quad f\otimes g \mapsto (\id\otimes g)f,\] while $F_{kl}(\Unitb_{\alpha}) \cong \delta_{kl} \C$ for $k,l\in \Gru{\mathcal{I}}{\alpha}{\alpha}$. 

It is readily verified that $F$ defines a unital morphism from $\CatCC$ to $\{\Hilb_{\fin}\}_{\mathcal{I}\times \mathcal{I}}$ based over the partition \[\mathcal{I}_{\alpha} = \bigcup_{\beta} \Gru{\mathcal{I}}{\alpha}{\beta},\quad\alpha\in \mathscr{I}.\] From the Tannaka-Krein-Woronowicz reconstruction result, we obtain a partial compact quantum group $\mathscr{A}_{\CatCC}$ with object set $\mathcal{I}$, which we call the \emph{canonical partial compact quantum group} associated with $\CatCC$. 
\end{Exa} 

\begin{Exa} More generally, let $\CatCC$ be an $\mathscr{I}$-partial fusion C$^*$-category, and let $\CatDD$ be a \emph{semi-simple partial $\CatCC$-module C$^*$-category} based over a set $\mathscr{J}$ and function $\phi:\mathscr{J}\rightarrow \mathscr{I},k\mapsto k'$. That is, $\CatDD$ consists of a collection of semi-simple C$^*$-categories $\CatD_{k}$ with $k\in \mathscr{J}$, together with tensor products $\otimes: \CatC_{k'l'}\times \CatD_{l}\rightarrow \CatC_{k}$ satisfying the appropriate associativity and unit constraints. Then if $\mathcal{I}$ labels a distinguished maximal set $\{u_a\}$ of mutually non-isomorphic irreducible objects of $\CatD$, with associated grading $\mathcal{I}_{k}$ over $\mathscr{J}$, we can again define \[F_{ab}(X)  = \Hom(u_a,  X\otimes u_b),\qquad X\in \Gru{\CatC}{k'}{l'}, a\in \mathcal{I}_{k},b\in\mathcal{I}_{l},\] and we obtain a unital morphism from $\CatCC$ to $\{\Hilb_{\fin}\}_{\mathcal{I}\times \mathcal{I}}$. The associated partial compact quantum group $\mathscr{A}_{\CatCC}$ will be called the \emph{canonical partial compact quantum group} associated with $(\CatCC,\CatDD)$. The previous construction coincides with the special case $\CatCC= \CatDD$ with $\mathscr{J} = \mathscr{I}\times \mathscr{I}$ and $\phi$ projection to the first factor.
\end{Exa}

\begin{Exa}\label{ExaErgo} As a particular instance, let $\G$ be a compact quantum group, and consider an ergodic action of $\G$ on a unital C$^*$-algebra $C(\mathbb{X})$. Then the collection of finitely generated $\G$-equivariant $C(\mathbb{X})$-Hilbert modules forms a module C$^*$-category over $\Rep_u(\G)$, cf.~ \cite{DCY1}. 
\end{Exa}

\subsection{Morita equivalence}

% Caenepeel Galois theory weak Hopf algebras

\begin{Def} Two partial compact quantum groups $\mathscr{G}$ and $\mathscr{H}$ are said to be \emph{Morita equivalent} if there exists an equivalence $\Rep_u(\mathscr{G}) \rightarrow \Rep_u(\mathscr{H})$ of partial fusion C$^*$-categories. 
\end{Def} 

In particular, if $\mathscr{G}$ and $\mathscr{H}$ are Morita equivalent they have the same hyperobject set, but they need not share the same object set.

Our goal is to give a concrete implementation of Morita equivalence, as has been done for compact quantum groups \cite{BDV1}. Note that we slightly changed their terminology of monoidal equivalence into Morita equivalence, as we feel the monoidality is intrinsic to the context. We introduce the following definition, in which indices are considered modulo 2. 

\begin{Def} A \emph{linking partial compact quantum group} consists of a partial compact quantum group $\mathscr{G}$ defined by a partial Hopf $^*$-algebra $\mathscr{A}$ over a set $I$ with a distinguished partition $I = I_1\sqcup I_2$ such that the units $\UnitC{i}{j} = \sum_{k\in I_i,l\in I_j} \UnitC{k}{l} \in M(A)$ are central, and such that for each $r\in I_i$, there exists $s\in I_{i+1}$ such that $\UnitC{r}{s}\neq 0$.
\end{Def}

If $\mathscr{A}$ defines a linking partial compact quantum group, we can split $A$ into four components $A^i_j = A\UnitC{i}{j}$. It is readily verified that the $A^i_i$ together with all $\Delta_{rs}$ with $r,s \in I_i$ define themselves partial compact quantum groups, which we call the \emph{corner} partial compact quantum groups of $\mathscr{A}$. 

\begin{Prop} Two partial compact quantum groups are Morita equivalent iff they arise as the corners of a linking partial compact quantum group.
\end{Prop}

\begin{proof} Suppose first that $\mathscr{G}_1$ and $\mathscr{G}_2$ are Morita equivalent partial compact quantum groups with associated partial Hopf $^*$-algebras $\mathscr{A}_1$ and $\mathscr{A}_2$ over respective sets $I_1$ and $I_2$. Then we may identify their corepresentation categories with the same abstract partial tensor C$^*$-category $\CatCC$ based over their common hyperobject set $\mathscr{I}$. Then $\CatCC$ comes endowed with two forgetful functors $F_i$ to $\{\Hilb_{\fin}\}_{I_i\times I_i}$ corresponding to the respective $\mathscr{A}_i$.

With $I = I_1\sqcup I_2$, we may then as well combine the $F_i$ into a global unital morphism $F:\CatCC \rightarrow \{\Hilb_{\fin}\}_{I\times I}$, with $F_{kl}(X)=F_i(X)$ if $k,l\in I_i$ and $F_{kl}(X)=0$ otherwise. Let $\mathscr{A}$ be the associated partial Hopf $^*$-algebra constructed from the Tannaka-Krein-Woronowicz reconstruction procedure. 

From the precise form of this reconstruction, it follows immediately that $\Gr{A}{k}{l}{m}{n} =0$ if either $k,l$ or $m,n$ do not lie in the same $I_i$. Hence the $\UnitC{i}{j} = \sum_{k\in I_i,l\in I_j} \UnitC{k}{l}$ are central. 

Moreover, fix $k\in I_i$ and any $l\in I_{i+1}$ with $k'=l'$. Then $\Nat(F_{ll},F_{kk})\neq \{0\}$. It follows that $\UnitC{k}{l}\neq 0$. Hence $\mathscr{A}$ is a linking compact quantum groupoid. It is clear that $\mathscr{A}_1$ and $\mathscr{A}_2$ are the corners of $\mathscr{A}$. 

Conversely, suppose that $\mathscr{A}_1$ and $\mathscr{A}_2$ arise from the corners of a linking partial compact quantum groupoid defined by $\mathscr{A}$ with invariant integral $\phi$. We will show that in fact the associated partial compact quantum groups $\mathscr{G}$ and $\mathscr{G}_1$ are Morita equivalent. Then by symmetry $\mathscr{G}$ and $\mathscr{G}_2$ are Morita equivalent, and hence also $\mathscr{G}_1$ and $\mathscr{G}_2$.


For $(V,\mathscr{X}) \in \Corep_u(\mathscr{A})$, let $F(V,\mathscr{X}) = (W,\mathscr{Y})$ be the pair obtained from $(V,\mathscr{X})$ by restricting all indices to those appearing to $I_1$. It is immediate that $(W,\mathscr{Y})$ is a unitary rcfd corepresentation of $\mathscr{A}_1$, and that the functor $F$ hence becomes a unital morphism in a trivial way. What remains to show is that $F$ is an equivalence of categories, i.e.~ that $F$ is faithful and essentially surjective. 

Let us first show that $F$ is faithful.  Lemma
\ref{lemma:rep-invertible} implies that for every $(V,\mathscr{X}) \in
\Corep_u(\mathscr{A})$, we have $\Gru{V}{k}{l}=0$ whenever $k\in
I_{i}$ and $l\in I_{i+1}$.  If $T$ is a morphism in
$\Corep_u(\mathscr{A}_{\alpha\beta}$ and $\Grd{T}{k}{l}=0$ for all
$k,l \in I_{1}$, we therefore get $\Grd{T}{k}{l}=0$ for all $k\in I$
and $l\in I_{1}$. Since $I_{\beta}\cap I_{1}$ is non-empty by
assumption, we can apply Lemma \ref{LemInjMor} and conclude that
$T=0$.

To complete the proof, we only need to show that $F$ induces a
bijection between isomorphism classes of irreducible unitary rcfd
corepresentations of $\mathcal{A}$ and of $\mathcal{A}_{1}$. Note that
by Proposition \ref{prop:rep-cosemisimple} and Lemma
\ref{lemma:rep-regular-embedding}, each such class can be represented
by a restriction of the regular corepresentation of $\mathcal{A}$ or
$\mathcal{A}_{1}$, respectively.

If $(V,\mathscr{X})$ is an irreducible restriction of the regular
corepresentation of $\mathcal{A}$, then $F(V,\mathscr{X})$ is a
restriction of the regular corepresentation of $\mathcal{A}_{1}$, and
Lemma \ref{lemma:regular-corep} implies that it is irreducible.  
If $(V,\mathscr{X}), (W,\mathscr{Y}) \in \Corep_{u}(\mathcal{A})$ are
irreducible and inequivalent, then $\mathcal{C}(V,\mathscr{X}) \cap
\mathcal{C}(W,\mathscr{Y})=0$ and hence $\mathcal{C}(F(V,\mathscr{X}))
\cap\mathcal{C}(F(W,\mathscr{Y})) =0$, whence $F(V,\mathscr{X})$ and
$F(W,\mathscr{Y})$ are inequivalent.
Finally, assume that $(W,\mathscr{Y})$ is a restriction of the regular
corepresentation of $\mathcal{A}_{1}$. Pick a non-zero $a \in
\Gru{W}{m}{n}$, define $\Gru{V}{p}{q} \subseteq \Gr{A}{k}{l}{p}{q}$ as
in \eqref{eq:element-reg-corep} and form the regular corepresentation
$(V,\mathscr{X})$ of $\mathscr{A}$. Then Lemma
\ref{lemma:regular-corep} 2., applied to the corepresentation
$(W,\mathscr{Y})$ of $\mathcal{A}_{1}$, shows that $\Grd{V}{p}{q} =
\Grd{W}{p}{q}$ for all $p,q\in I_{1}$ so that $F(V,\mathscr{X}) =
(W,\mathscr{Y})$. Since $F$ is faithful, $(V,\mathscr{X})$ must be
irreducible.
\end{proof}

\begin{Exa} If $\mathscr{G}_1$ and $\mathscr{G}_2$ are Morita equivalent compact quantum groups, the total partial compact quantum group is the co-groupoid constructed in \cite{Bic1}. 
\end{Exa}

\begin{Exa}  Let $\G$ be a compact quantum group with ergodic action on a unital C$^*$-algebra $C(\mathbb{X})$. Consider the module C$^*$-category $\CatD$ of finitely generated $\G$-equivariant Hilbert $C(\mathbb{X})$-modules as in Example \ref{ExaErgo}. Then $\G$ is Morita equivalent with the canonical partial compact quantum group constructed from $(\Rep_u(\G),\CatD)$. The off-diagonal part of the associated linking partial compact quantum group was studied in \cite{DCY1}. We will make a detailed study of the case $\G = SU_q(2)$ in \cite{DCT2}, in particular for $\X$ a Podle\'{s} sphere. This will lead us to partial compact quantum group versions of the dynamical quantum $SU(2)$-group.
\end{Exa}


\subsection{Weak Morita equivalence}

\begin{Def} A \emph{linking} partial fusion C$^*$-category consists of a partial fusion C$^*$-category with a distinguished partition $\mathscr{I} =\mathscr{I}_1 \cup \mathscr{I}_2$ such that for each $\alpha\in \mathscr{I}_1$, there exists $\beta \in \mathscr{I}_{2}$ with $\CatC_{\alpha\beta}\neq \{0\}$.

The \emph{corners} of $\CatCC$ are the restrictions of $\CatCC$ to $\mathscr{I}_1$ and $\mathscr{I}_2$.
\end{Def}

The following notion is essentially the same as the one by M. M\"{u}ger \cite{Mug1}. 

\begin{Def} Two partial semi-simple tensor C$^*$-categories $\CatCC_1$ and $\CatCC_2$ with duality over respective sets $\mathscr{I}_1$ and $\mathscr{I}_2$ are called \emph{Morita equivalent} if there exists a linking partial fusion C$^*$-category $\CatCC$ over the set $\mathscr{I}=\mathscr{I}_1\sqcup \mathscr{I}_2$ whose corners are isomorphic to $\CatCC_1$ and $\CatCC_2$.

We say two partial compact quantum groups $\mathscr{G}_1$ and $\mathscr{G}_2$ are \emph{weakly Morita equivalent} if their corepresentation categories $\Corep_u(\mathscr{G}_i)$ are Morita equivalent. 
\end{Def} 

One can prove that this is indeed an equivalence relation. %ref to add! 

\begin{Def}\label{DefCoLink} A \emph{co-linking partial compact quantum group} consists of a partial compact quantum group $\mathscr{G}$ defined by a Hopf $^*$-algebra $\mathscr{A}$ over an index set $I$, together with a distinguished partition $I = I_1\cup I_2$ such that each $\UnitC{k}{l}=0$ for $k\in I_i$ and $l\in I_{i+1}$, and such that for each $k\in I_i$, there exists $l\in I_{i+1}$ with $\Gr{A}{k}{l}{k}{l}\neq 0$.  
\end{Def} 

It is again easy to see that if we restrict all indices of a co-linking partial compact quantum group to one of the distinguished sets, we obtain a partial compact quantum group which we will call a corner. In fact, write $e_i = \sum_{k,l\in I_i} \UnitC{k}{l}$. Then we can decompose the total algebra $A$ into components $A_{ij} = e_{i}Ae_{j}$, and correspondingly write $A$ in matrix notation \[ A = \begin{pmatrix} A_{11} & A_{12}  \\ A_{21} & A_{22}\end{pmatrix},\] where multiplication is matrixwise and where comultiplication is entrywise. Note that we have $A_{12}A_{21} = A_{11}$, and similarly $A_{21}A_{12} = A_{22}$. Indeed, take $k\in I_1$, and pick $l\in I_2$ with $\Gr{A}{k}{l}{k}{l}\neq \{0\}$. Then in particular, we can find an $a\in \Gr{A}{k}{l}{k}{l}$ with $\epsilon(a)\neq 0$. Hence for any $m\in I_1$, we have $\UnitC{k}{m} = \UnitC{k}{m} a_{(1)}S(a_{(2)}) \in A_{12}A_{21}$. As this latter space contains all local units of $A_{11}$ and is a right $A_{11}$-module, it follows that it is in fact equal to $A_{11}$. We hence deduce that in fact $A_{11}$ and $A_{22}$ are Morita equivalent algebras, with the Morita equivalence implemented by $A$. % Cf. Abrams? 

\begin{Rem} For finite partial compact quantum groups, one can then easily show that the notion of co-linking partial compact quantum group is dual to the notion of linking partial compact quantum group.\end{Rem}

\begin{Def} We call two partial compact quantum groups \emph{co-Morita equivalent} if there exists a \emph{co-linking partial compact quantum group} having these partial compact quantum groups as its corners.
\end{Def}

\begin{Lem} Co-Morita equivalence is an equivalence relation. 
\end{Lem} 

\begin{proof} Symmetry is clear. Co-Morita equivalence of $\mathscr{A}$ with itself follows by considering as co-linking quantum groupoid the product of $\mathscr{A}$ with the partial compact quantum groupoid $M_2(\C)$, where $\Delta(e_{ij}) = e_{ij}\otimes e_{ij}$. 

Let us show the main elements to prove transitivity. Let us assume $\mathscr{G}_1$ and $\mathscr{G}_2$ as well as $\mathscr{G}_2$ and $\mathscr{G}_3$ are co-Morita equivalent. Let us write the global $^*$-algebras of the associated co-linking quantum groupoids as \[A_{\{1,2\}} = \begin{pmatrix} A_{11} & A_{12} \\ A_{21} & A_{22} \end{pmatrix}, \quad A_{\{2,3\}} = \begin{pmatrix} A_{22} & A_{23} \\ A_{32} & A_{33}\end{pmatrix}.\] Then we can define a new $^*$-algebra $A_{\{1,2,3\}}$ as \[ A_{\{1,2,3\}} = \begin{pmatrix} A_{11} & A_{12} &   A_{13} \\ A_{21} & A_{22} & A_{23} \\ A_{31} & A_{32} & A_{33} \end{pmatrix},\] where $A_{13} = A_{12}\underset{A_{22}}{\otimes } A_{23}$ and $A_{31} = A_{32}\underset{A_{22}}{\otimes} A_{21}$, and with multiplication and $^*$-structure defined in the obvous way. 

It is straightforward to verify that there exists a unique $^*$-homomorphism $\Delta: A_{\{1,2,3\}} \rightarrow M(A_{\{1,2,3\}}\otimes A_{\{1,2,3\}})$ whose restrictions to the $A_{ij}$ with $|i-j|\leq 1$ coincide with the already defined coproducts. We leave it to the reader to verify that $(A,\Delta)$ defines a regular weak multiplier Hopf $^*$-algebra satisfying the conditions of Proposition \ref{PropCharPBA}, and hence arises from a regular partial weak Hopf $^*$-algebra. 

Let now $\phi$ be the functional which is zero on the off-diagonal entries $A_{ij}$ and coincides with the invariant positive integrals on the $A_{ii}$. Then it is also easily checked that $\phi$ is invariant. To show that $\phi$ is positive, we invoke Remark \ref{RemPos}. Indeed, any irreducible corepresentation of $A_{\{1,2,3\}}$ has coefficients in a single $A_{ij}$. For those $i,j$ with $|i-j|\leq 1$, we know that the corepresentation is unitarizable by restricting to a corner $2\times 2$-block. If however the corepresentation $\mathscr{X}$ has coefficients living in (say) $A_{13}$, it follows from the identity $A_{12}A_{23}=A_{13}$ that the corepresentation is a direct summand of a product $\mathscr{Y}\Circt \mathscr{Z}$ of corepresentations with coefficients in respectively $A_{12}$ and $A_{23}$. This proves unitarizability of $\mathscr{X}$. It follows from Remark \ref{RemPos} that $\phi$ is positive, and hence $\mathscr{A}_{\{1,2,3\}}$ defines a partial compact quantum group.  

We claim that the subspace $\mathscr{A}_{\{1,3\}}$ (in the obvious notation) defines a co-linking compact quantum group between $\mathscr{G}_1$ and $\mathscr{G}_2$. In fact, it is clear that the $\mathscr{A}_{11}$ and $\mathscr{A}_{33}$ are corners of $\mathscr{A}_{\{1,3\}}$, and that $\UnitC{k}{l}=0$ for $k,l$ not both in $I_1$ and $I_{3}$. To finish the proof, it is sufficient to show now that for each $k\in I_1$, there exists $l\in I_{3}$ with $\Gr{A}{k}{l}{k}{l}\neq 0$, as the other case follows by symmetry using the antipode. But there exists $m\in I_2$ with $\Gr{A}{k}{m}{k}{m} \neq \{0\}$, and $l\in I_3$ with $\Gr{A}{m}{l}{m}{l}\neq\{0\}$. As in the discussion following Definition \ref{DefCoLink}, this implies that there exists $a\in \Gr{A}{k}{m}{k}{m}$ and $b\in \Gr{A}{m}{l}{m}{l}$ with $\epsilon(a)=\epsilon(b)=1$. Hence $\epsilon(ab)=1$, showing $\Gr{A}{k}{l}{k}{l}\neq \{0\}$.
\end{proof} 

\begin{Prop}\label{PropCoWeak} Assume that two partial compact quantum groups $\mathscr{G}_1$ and $\mathscr{G}_2$ are co-Morita equivalent. Then they are weakly Morita equivalent.
\end{Prop} 
\begin{proof} 
Consider the corepresentation category $\CatCC$ of a co-linking partial compact quantum group $\mathscr{A}$ over $I = I_1\cup I_2$. Let $\varphi:I\rightarrow \mathscr{I}$ define the corresponding partition along the hyperobject set. Then by the defining property of a co-linking partial compact quantum group, also $\mathscr{I} = \mathscr{I}_1\cup \mathscr{I}_2$ with $\mathscr{I}_i=\varphi(I_i)$ is a partition. Hence $\CatCC$ decomposes into parts $\CatCC_{ij}$ with $i,j\in \{1,2\}$ and $\CatC_{ii}\cong \Rep_u(\mathscr{G}_i)$. 

To show that $\mathscr{G}_1$ and $\mathscr{G}_2$ are weakly Morita equivalent, it thus suffices to show that $\{\CatCC_{ij}\}$ forms a linking partial fusion C$^*$-category. But fix $\alpha\in I_1$ and $k\in I_{\alpha}$. Then as $\mathscr{A}$ is co-linking, there exists $l \in I_2$ with $\Gr{A}{k}{l}{k}{l}\neq \{0\}$. Hence there exists a non-zero regular rcfd unitary corepresentation inside $\oplus_{m,n}\Gr{A}{k}{l}{m}{n}$. If then $l\in I_{\beta}$ with $\beta\in \mathscr{I}_2$, it follows that $\CatC_{\alpha\beta}\neq 0$. By symmetry, we also have that for each $\alpha \in \mathscr{I}_2$ there exists $\beta \in \mathscr{I}_1$ with $\CatC_{\alpha\beta}\neq \{0\}$. This proves that the $\{\CatCC_{ij}\}$ forms a linking partial fusion C$^*$-category.
\end{proof}

\begin{Prop}\label{PropCoLink} Let $\CatCC$ be a linking $\mathscr{I}$-partial fusion C$^*$-category. Then the associated canonical partial compact quantum group is a co-linking partial compact quantum group. 
\end{Prop} 

\begin{proof} Let $\mathscr{I}= \mathscr{I}_1\cup \mathscr{I}_2$ be the associated partition of $\mathscr{I}$. Let $\mathscr{A} = \mathscr{A}_{\CatCC}$ define the canonical partial compact quantum group with object set $I$ and hyperobject partition $\varphi:I\rightarrow \mathscr{I}$. Let $I=I_1\cup I_2$ with $I_i = \varphi^{-1}(\mathscr{I}_i)$ be the corresponding decomposition of $I$. By construction, $\UnitC{k}{l}=0$ if $k$ and $l$ are not both in $I_1$ or $I_2$. 

Fix now $k\in I_{\alpha}$ for some $\alpha \in I_i$. Pick $\beta\in I_{i+1}$ with $\CatC_{\alpha\beta}\neq\{0\}$, and let $(V,\mathscr{X})$ be a non-zero irreducible corepresentation inside $\CatC_{\alpha\beta}$. Then by irreducibility, we know that $\oplus_l \Gru{V}{k}{l} \neq \{0\}$, hence there exists $l\in I_{\beta}$ with $\Gru{V}{k}{l}\neq \{0\}$. As $(\epsilon\otimes \id)\Gr{X}{k}{l}{k}{l} = \id_{\Gru{V}{k}{l}}$, it follows that $\Gr{A}{k}{l}{k}{l} \neq 0$. This proves that $\mathscr{A}$ defines a co-linking partial compact quantum group.
\end{proof} 

\begin{Rem} Note however that the corners of the canonical partial compact quantum group associated to linking $\mathscr{I}$-partial fusion C$^*$-category \emph{are not} the canonical partial compact quantum groups associated to the corners of the linking $\mathscr{I}$-partial fusion C$^*$-category. Rather, they are Morita equivalent copies of these.
\end{Rem} 

\begin{Theorem} Two partial compact quantum groups $\mathscr{G}_1$ and $\mathscr{G}_2$ are weakly Morita equivalent if and only if they are connected by a string of Morita and co-Morita equivalences. 
\end{Theorem}

\begin{proof} Clearly if two partial compact quantum groups are Morita equivalent, they are weakly Morita equivalent. By Proposition \ref{PropCoWeak}, the same goes for co-Morita equivalence. This proves one direction of the theorem. 

Conversely, assume $\mathscr{G}_1$ and $\mathscr{G}_2$ are weakly Morita equivalent. Let $\CatCC$ be a linking fusion C$^*$-category between $\Rep_u(\mathscr{G}_1)$ and $\Rep_u(\mathscr{G}_2)$. Then $\mathscr{G}_i$ are Morita equivalent with the corners of the canonical partial compact quantum group associated to $\CatCC$. But Proposition \ref{PropCoLink} shows that these corners are co-Morita equivalent. 
\end{proof} 

\begin{Rem} 
\begin{enumerate}
\item Note that it is essential that we allow the string of equivalences to pass through partial compact quantum groups, even if we start out with (genuine) compact quantum groups.
\item One can show that if $\mathscr{G}$ is a finite partial compact quantum group, then $\mathscr{G}$ is weakly Morita equivalent with its dual $\widehat{\mathscr{G}}$ (defined by the dual weak Hopf $^*$-algebra). In fact, if $\mathscr{G}$ is the canonical partial compact quantum group associated to a finite partial fusion C$^*$-category, then $\mathscr{G}$ is isomorphic to the co-opposite of its dual, e.g.~ the case of dynamical quantum $SU(2)$ at roots of unity. In any case, it follows that two finite quantum groups $H$ and $G$ are weakly Morita equivalent if and only if they can be connected by a string of 2-cocycle-elements and 2-cocycle functionals. % Ref to add! 
\end{enumerate}
\end{Rem}


\section{Partial compact quantum groups from reciprocal random walks}

In this section, we study in more detail the construction from Section \ref{SubSecCan} in case the category $\CatC$ is the Temperley-Lieb C$^*$-category.


\subsection{Reciprocal random walks}

We recall some notions introduced in \cite{DCY1}. We slightly change the terminology for the sake of convenience.

\begin{Def} Let $t\in \R_0$. A \emph{$t$-reciprocal random walk} consists of a quadruple $(\Gamma,w,\sgn,i)$ with \begin{itemize}
\item[$\bullet$] $\Gamma=(\Gamma^{(0)},\Gamma^{(1)},s,t)$ a graph with \emph{source} and \emph{target} maps \[s,t:\Gamma^{(1)}\rightarrow \Gamma^{(0)},\]
\item[$\bullet$] $w$ a function (the \emph{weight} function) $w:\Gamma^{(1)}\rightarrow \R_0^+$,
\item[$\bullet$] $\sgn$ a function (the \emph{sign} function) $\sgn:\Gamma^{(1)}\rightarrow \{\pm 1\}$,
\item[$\bullet$] $i$ an involution \[i:\Gamma^{(1)} \rightarrow \Gamma^{(1)},\quad e\mapsto \overline{e}\] with $s(\bar{e}) = t(e)$ for all edges $e$,
\end{itemize}
such that the following conditions are satisfied:
\begin{enumerate}[label=(\arabic*)]
\item (weight reciprocality) $w(e)w(\bar{e}) = 1$ for all edges $e$,
\item (sign reciprocality) $\sgn(e)\sgn(\bar{e}) = \sgn(t)$ for all edges $e$,
\item (random walk property) $p(e) = \frac{1}{|t|}w(e)$ satisfies $\sum_{s(e)=v} p(e) = 1$ for all $v\in \Gamma^{(0)}$.
\end{enumerate}
\end{Def}

%Isomorphisms between $t$-random walks will simply be isomorphisms of weighted graphs (and hence do not remember any structure concerning involutions or signedness). 

Note that, by \cite[Proposition 3.1]{DCY1}, there is a uniform bound on the number of edges leaving from any given vertex $v$, i.e.~ $\Gamma$ has a finite degree.

For examples of $t$-reciprocal random walks, we refer to \cite{DCY1}. One particular example (which will be needed for our construction of dynamical quantum $SU(2)$) is the following.

\begin{Exa}\label{ExaGraphPod} Take $0<|q|<1$ and $x\in \R$. Write $2_q = q+q^{-1}$. Then we have the reciprocal $-2_q$-random walk \[\Gamma_x =(\Gamma_x,w,\sgn,i)\] with \[ \Gamma^{(0)} = \Z,\quad \Gamma^{(1)} = \{(k,l)\mid |k-l|= 1\}\subseteq \Z\times \Z\] with projection on the first (resp. second) leg as source (resp. target) map, with weight function \[w(k,k\pm 1) = \frac{|q|^{x+k\pm 1}+|q|^{-(x+k\pm 1)}}{|q|^{x+k}+|q|^{-(x+k)}},\] sign function \[\sgn(k,k+1) = 1,\quad \sgn(k,k-1) = -\sgn(q),\] and involution $\overline{(k,k+1)} = (k+1,k)$. 

By translation we can shift the value of $x$ by an integer. By a point reflection and changing the direction of the arrows, we can change $x$ into $-x$. It follows that by some (unoriented) graph isomorphism, we can always arrange to have $x\in \lbrack 0,\frac{1}{2}\rbrack$.
\end{Exa} 

\subsection{Temperley-Lieb categories}

Let now $0<|q|\leq 1$, and let $SU_q(2)$ be Woronowicz's twisted $SU(2)$ group \cite{Wor1}. Then $SU_q(2)$ is a compact quantum group whose category of finite-dimensional unitary representations $\Rep(SU_q(2))$ is generated by the spin $1/2$-representation $\pi_{1/2}$ on $\C^2$. It has the same fusion rules as $SU(2)$, and conversely any compact quantum group with the fusion rules of $SU(2)$ has its representation category equivalent to $\Rep(SU_q(2))$ as a tensor C$^*$-category. Abstractly, these tensor C$^*$-categories are referred to as the \emph{Temperley-Lieb C$^*$-categories}.

Let now $\Gamma = (\Gamma,w,\sgn,i)$ be a $-2_q$-reciprocal random walk. Define $\Hsp^{\Gamma}$ as the $\Gamma^{(0)}$-bigraded Hilbert space $l^2(\Gamma^{(1)})$, where the $\Gamma^{(0)}$-bigrading is given by \[\delta_e \in \Gru{\Hsp^{\Gamma}}{s(e)}{t(e)}\] for the obvious Dirac functions. Note that, because $\Gamma$ has finite degree, $\Hsp^{\Gamma}$ is \emph{row- and column finite dimensional} (rcfd), i.e.~ $\oplus_{v\in \Gamma^{(0)}} \Gru{\Hsp^{\Gamma}}{v}{w}$ (resp.~ $\oplus_{w\in \Gamma^{(0)}} \Gru{\Hsp^{\Gamma}}{v}{w}$) is finite dimensional for all $w$ (resp.~ all $v$). 

Consider now $R_{\Gamma}$ as the (bounded) map \[R_{\Gamma}:l^2(\Gamma^{(0)})\rightarrow \Hsp^{\Gamma}\underset{\Gamma^{(0)}}{\otimes} \Hsp^{\Gamma}\] given by \begin{eqnarray*} R_{\Gamma} \delta_v &=& \sum_{e,s(e) = v} \sgn(e)\sqrt{w(e)}\delta_e \otimes \delta_{\bar{e}}.\end{eqnarray*} Then $R_{\Gamma}^*R_{\Gamma} = |q|+|q|^{-1}$ and \[(R_{\Gamma}^*\underset{\Gamma^{(0)}}{\otimes} \id_{\Hsp^{\Gamma}})(\id_{\Hsp^{\Gamma}}\underset{\Gamma^{(0)}}{\otimes} R_{\Gamma}) = -\sgn(q)\id.\]


%Then we can define a couple $\Hsp(\Gamma) = (\Hsp,R)$ where $\Hsp$ is $l^2(\Gamma^{(1)})$ with the $\Gamma^{(0)}$-bigrading $\delta_e \in \Hsp(s(e),t(e))$ for the obvious Dirac functions, and where  

% Dropped Temperley-Lieb terminology

Hence, by the universal property of $\Rep(SU_q(2))$ (\cite[Theorem 1.4]{DCY1}, based on \cite{Tur1,EtO1,Yam1,Pin2,Pin3}), we have a strongly monoidal $^*$-functor
\begin{equation}\label{EqForget} F_{\Gamma}: \Rep(SU_q(2)) \rightarrow {}^{\Gamma^{(0)}}\Hilb_{\rcf}^{\Gamma^{(0)}}\end{equation} into the tensor C$^*$-category of rcfd $\Gamma^{(0)}$-bigraded Hilbert spaces such that $F_{\Gamma}(\pi_{1/2}) = \Hsp_{\Gamma}$ and $F_{\Gamma}(\mathscr{R}) = R_{\Gamma}$, with \[(\pi_{1/2},\mathscr{R},-\sgn(q)\mathscr{R})\] a solution for the conjugate equations for $\pi_{1/2}$. Up to equivalence, $F_{\Gamma}$ only depends upon the isomorphism class of $(\Gamma,w)$, and is independent of the chosen involution or sign structure. Conversely, any strong monoidal $^*$-functor from $\Rep(SU_q(2))$ into $\Gr{\Hilb}{I}{I}{}{\rcf}$ for some set $I$ arises in this way \cite{DCY2}.

\subsection{Universal orthogonal partial compact quantum groups}

% Concrete reference to be added.

It follows from the previous subsection and the Tannaka-Krein-Woronowicz duality of \cite[Theorem 4.14]{DCT1} that for each reciprocal random walk on a graph $\Gamma$, one obtains a $\Gamma^{(0)}$-partial compact quantum group $\mathscr{G}$, and conversely every partial compact quantum group $\mathscr{G}$ with the fusion rules of $SU(2)$ arises in this way. Our first aim is to give a direct representation of the associated algebras $A(\Gamma) = P(\mathscr{G})$ by generators and relations. We will write $\Gamma_{vw}\subseteq \Gamma^{(1)}$ for the set of edges with source $v$ and target $w$.


\begin{Theorem}\label{TheoGenRel} Let $0<|q|\leq 1$, and let $\Gamma = (\Gamma,w,\sgn,i)$ be a $-2_q$-reciprocal random walk. Let $A(\Gamma)$ be the total $^*$-algebra associated to the $\Gamma^{(0)}$-partial compact quantum group constructed from the fiber functor $F_{\Gamma}$ as in \eqref{EqForget}. Then $A(\Gamma)$ is the universal $^*$-algebra generated by a copy of the $^*$-algebra of finite support functions on $\Gamma^{(0)}\times \Gamma^{(0)}$ (with the Dirac functions written as $\UnitC{v}{w}$) and elements $(u_{e,f})_{e,f\in \Gamma^{(1)}}$ where $u_{e,f}\in \Gr{A(\Gamma)}{s(e)}{t(e)}{s(f)}{t(f)}$ and 
\begin{eqnarray} 
\label{EqUni1}\sum_{v\in \Gamma^{(0)}}\sum_{g\in \Gamma_{vw}} u_{g,e}^*u_{g,f} = \delta_{e,f}\mathbf{1}\Grru{w}{t(e)}, \qquad \forall w\in \Gamma^{(0)}, e,f\in \Gamma^{(1)},\\ % just sum over $g$ with $t(g) =w$
\label{EqUni2}\sum_{w\in \Gamma^{(0)}} \sum_{g\in \Gamma_{vw}} u_{e,g}u_{f,g}^* = \delta_{e,f} \mathbf{1}\Grru{s(e)}{v}\qquad \forall v\in \Gamma{(0)}, e,f\in \Gamma^{(1)},\\ 
\label{EqInt}u_{e,f}^* \;=\; \sgn(e)\sgn(f)\sqrt{\frac{w(f)}{w(e)}} u_{\bar{e},\bar{f}},\qquad \forall e,f\in \Gamma^{(1)}.
\end{eqnarray}

If moreover $v,w\in \Gamma^{(0)}$ and $e,f\in \Gamma^{(1)}$, we have \[\Delta_{vw}(u_{e,f}) = \underset{t(g) = w}{\sum_{s(g) = v}} u_{e,g}\otimes u_{g,f},\]
\[\varepsilon(u_{e,f}) = \delta_{e,f}\] and \[S(u_{e,f}) = u_{f,e}^*.\] 
\end{Theorem} 

Note that the sums in \eqref{EqUni1} and \eqref{EqUni2} are in fact finite, as $\Gamma$ has finite degree. 

\begin{proof} Let $(\Hsp,V)$ be the generating unitary corepresentation of $A(\Gamma)$ on $\Hsp = l^2(\Gamma^{(1)})$. Then $V$ decomposes into parts \[ \Gr{V}{k}{l}{m}{n} = \sum_{e,f} v_{e,f} \otimes e_{e,f} \in \Gr{A}{k}{l}{m}{n}\otimes B(\Gru{\Hsp}{m}{n},\Gru{\Hsp}{k}{l}),\] where the $e_{e,f}$ are elementary matrix coefficients and with the sum over all $e$ with $s(e)=k,t(e)=l$ and all $f$ with $s(f) = m, t(f)=n$. By construction $V$ defines a unitary corepresentation of $A(\Gamma)$, hence the relations \eqref{EqUni1} and \eqref{EqUni2} are satisfied for the $v_{e,f}$. Now as $R_{\Gamma}$ is an intertwiner between the trivial representation on $\C^{(\Gamma^{(0)})} = \oplus_{v\in \Gamma^{(0)}} \C$ and $V\Circtv{\Gamma^{(0)}} V$, we have for all $v\in \Gamma^{(0)}$ that \begin{equation}\label{EqMorR}\underset{t(f)=s(h),t(e)=s(g)}{\sum_{e,f,g,h\in \Gamma^{(1)}}} v_{e,f}v_{g,h}\otimes \left((e_{e,f}\otimes e_{g,h})\circ R_{\Gamma} \delta_v\right) = \sum_w \UnitC{w}{v}\otimes R_{\Gamma}\delta_v,\end{equation} hence
\[\underset{t(e)=s(g),s(k)=v}{\sum_{e,g,k}} \sgn(k)\sqrt{w(k)}\left( v_{e,k}v_{g,\bar{k}} \otimes \delta_e\otimes \delta_{g}\right) =\underset{s(k)=w}{\sum_{w,k}}\sgn(k)\sqrt{w(k)} \left(\UnitC{w}{v} \otimes \delta_k\otimes \delta_{\bar{k}}\right).\] Hence if $t(e) = s(g)=z$, we have \[\sum_{k,s(k)=v} \sgn(k)\sqrt{w(k)} v_{e,k}v_{g,\bar{k}} =  \delta_{e,\bar{g}} \sgn(e)\sqrt{w(e)}\UnitC{s(e)}{v}.\] Multiplying to the left with $v_{e,l}^*$ and summing over all $e$ with $t(e) = z$, we see from \eqref{EqUni1} that also relation \eqref{EqInt} is satisfied. Hence the $v_{e,f}$ satisfy the universal relations in the statement of the theorem. The formulas for comultiplication, counit and antipode then follow immediately from the fact that $V$ is a unitary corepresentation.
% In fact, I have a small problem with the indices of source and target in the above derivation when summing...

Let us now a priori denote by $B(\Gamma)$ the $^*$-algebra determined by the relations \eqref{EqUni1},\eqref{EqUni2} and \eqref{EqInt} above, and write $\mathscr{B}(\Gamma)$ for the associated $\Gamma^{(0)}\times \Gamma^{(0)}$-partial $^*$-algebra induced by the local units $\UnitC{v}{w}$. Write $\Delta(\UnitC{v}{w}) = \sum_{z\in \Gamma^{(0)}} \UnitC{v}{z}\otimes \UnitC{z}{w}$ and \[\Delta(u_{e,f}) = \sum_{g\in \Gamma^{(1)}} u_{e,g}\otimes u_{g,f},\] which makes sense in $M(B(\Gamma)\otimes B(\Gamma))$ as the degree of $\Gamma$ is finite. Then we compute for $w\in \Gamma^{(0)}$ and $e,f\in \Gamma^{(1)}$ that \begin{eqnarray*} \sum_{v\in \Gamma^{(0)}}\sum_{g\in \Gamma_{vw}}\Delta(u_{g,e})^*\Delta(u_{g,f}) &=& \sum_{v\in \Gamma^{(0)}}\sum_{g\in \Gamma_{vw}} \sum_{h,k\in \Gamma^{(1)}} u_{g,h}^*u_{g,k}\otimes u_{h,e}^*u_{k,f}\\ &=& \sum_{h,k\in \Gamma^{(1)}} \delta_{h,k} \UnitC{w}{t(h)}\otimes u_{h,e}^*u_{k,f}\\ &=&  \sum_{z\in \Gamma^{(0)}}\underset{t(h)=z}{\sum_{h\in \Gamma^{(1)}}} \UnitC{w}{z}\otimes u_{h,e}^*u_{h,f} \\ &=& \delta_{e,f} \sum_{z\in \Gamma^{(0)}} \UnitC{w}{z}\otimes \UnitC{z}{t(e)}\\ &=& \delta_{e,f} \Delta(\UnitC{w}{t(e)}).\end{eqnarray*}  Similarly, the analogue of \eqref{EqUni2} holds for $\Delta(u_{e,f})$. As also \eqref{EqInt} holds trivially for $\Delta(u_{e,f})$, it follows that we can define a $^*$-algebra homomorphism \[\Delta:B(\Gamma)\rightarrow M(B(\Gamma)\otimes B(\Gamma))\] sending $u_{e,f}$ to $\Delta(u_{e,f})$ and $\UnitC{v}{w}$ to $\Delta(\UnitC{v}{w})$. Cutting down, we obtain maps \[\Delta_{vw}:\Gr{B(\Gamma)}{r}{s}{t}{z}\rightarrow \Gr{B(\Gamma)}{r}{s}{v}{w}\otimes \Gr{B(\Gamma)}{v}{w}{t}{z}\] which then satisfy the properties \ref{Propa}, \ref{Propd} and \ref{Prope} of Definition \ref{DefPartBiAlg}. Moreover, the $\Delta_{vw}$ are coassociative as they are coassociative on generators.

% Remark this in beginning that one can construct hom from counit
Let now $e_{v,w}$ be the matrix units for $l^2(\Gamma^{(0)})$. Then one verifies again directly from the defining relations of $B(\Gamma)$ that one can define a $^*$-homorphism \[\widetilde{\varepsilon}: B(\Gamma)\rightarrow B(l^2(\Gamma^{(0)})),\quad \left\{\begin{array}{lll} \UnitC{v}{w}&\mapsto &\delta_{v,w}\, e_{v,v}\\ u_{e,f}&\mapsto& \delta_{e,f}\, e_{s(e),t(e)}\end{array}\right.\] We can hence define a map $\varepsilon: B(\Gamma)\rightarrow \C$ such that \[\widetilde{\varepsilon}(x) = \varepsilon(x) e_{v,w},\qquad  \forall x\in \Gr{B(\Gamma)}{v}{w}{v}{w},\] and which is zero elsewhere. Clearly it satisfies the conditions \ref{Propb} and \ref{Propc} of Definition \ref{DefPartBiAlg}. As $\varepsilon$ satisfies the counit condition on generators, it follows by partial multiplicativity that it satisfies the counit condition on the whole of $B(\Gamma)$, i.e.~ $B(\Gamma)$ is a partial $^*$-bialgebra. 

It is clear now that the $u_{e,f}$ define a unitary corepresentation $U$ of $B(\Gamma)$ on $\Hsp^{\Gamma}$. Moreover, from \eqref{EqUni1} and \eqref{EqInt} we can deduce that $R_{\Gamma}: \C_{\Gamma^{(0)}}\rightarrow \Hsp^{\Gamma}\underset{\Gamma^{(0)}}{\otimes}\Hsp^{\Gamma}$ is a morphism from $\C^{(\Gamma^{(0)})}$ to $U\Circtv{\Gamma^{(0)}} U$ in $\Corep_{\rcf,u}(\mathscr{B}(\Gamma))$, cf.~ \eqref{EqMorR}. From the universal property of $\Rep(SU_q(2))$, it then follows that we have a (unique and faithful) strongly monoidal $^*$-functor \[G^{\Gamma}: \Rep(SU_q(2)) \rightarrow \Corep_{\rcf,u}(\mathscr{B}(\Gamma))\] such that $G^{\Gamma}(\pi_{1/2}) = U$. On the other hand, as we have a $\Delta$-preserving $^*$-homomorphism $B(\Gamma)\rightarrow A(\Gamma)$ by the universal property of $\mathscr{B}(\Gamma)$, we have a strongly monoidal $^*$-functor $H^{\Gamma}:  \Corep_{\rcf,u}(\mathscr{B}(\Gamma))\rightarrow \Corep_u(\mathscr{A}(\Gamma)) = \Rep(SU_q(2))$ which is inverse to $G^{\Gamma}$. Then since the commutation relations of $\mathscr{A}(\Gamma)$ are completely determined by the morphism spaces of $\Rep(SU_q(2))$, it follows that we have a $^*$-homomorphism $\mathscr{A}(\Gamma)\rightarrow \mathscr{B}(\Gamma)$ sending $v_{e,f}$ to $u_{e,f}$. This proves the theorem. 
\end{proof}

\subsection{Dynamical quantum $SU(2)$ from the Podle\'{s} graph}

%Let us now consider the particular case where $\Gamma$ is a graph with uniform degree $m$. 

Let us now fix a $-2_q$-reciprocal random walk, and assume further that there exists a finite set $T$ partitioning $\Gamma^{(1)} = \cup_a \Gamma^{(1)}_a$ such that for each $a\in T$ and $v\in \Gamma^{(0)}$, there exists a unique $e_a(v)\in \Gamma^{(1)}_a$ in $\Gamma^{(1)}_a$) with source $v$. Write $av$ for the range of $e_a(v)$. Assume moreover that $T$ has an involution $a\mapsto \bar{a}$ such that $\overline{e_a(v)} = e_{\bar{a}v}$. Then for each $a$, the map $v\mapsto av$ is a bijection on $\Gamma^{(0)}$ with inverse $v\mapsto \bar{a}v$. In particular, also for each $w\in \Gamma^{(0)}$ there exists a unique $f_v(a) \in \Gamma^{(1)}_a$ with target $w$.

Let us further denote $w_a(v) = w(e_a(v))$ and $\sgn_a(v) = \sgn(e_a(v))$. Let $A(\Gamma)$ be the total $^*$-algebra of the associated partial compact quantum group. Using Theorem \ref{TheoGenRel}, we have the following presentation of $A(\Gamma)$. Let $B$ be the $^*$-algebra of finite support functions on $\Gamma^{(0)}\times \Gamma^{(0)}$, whose Dirac functions we write as $\UnitC{v}{w}$. Then $A(\Gamma)$ is generated by a copy of $B$ and elements \[(u_{a,b})_{v,w} := u_{e_a(v),e_b(v)}\] for $a,b\in T$ and $v,w\in \Gamma^{(0)}$ with defining relations \begin{eqnarray*} \sum_{a\in T} (u_{a,b})_{\bar{a}v,w}^* (u_{a,c})_{\bar{a}v,z}&=& \delta_{w,z} \delta_{b,c} \UnitC{v}{bw},\\ \sum_{a\in T} (u_{b,a})_{w,v} (u_{c,a})_{z,v}^* &=& \delta_{b,c}\delta_{w,z} \UnitC{w}{v}\\ (u_{a,b})_{v,w}^* &=& \frac{\sgn_b(w)\sqrt{w_b(w)}}{\sgn_a(v)\sqrt{w_a(v)}}(u_{\bar{a},\bar{b}})_{av,bw}.\end{eqnarray*} The element $(u_{a,b})_{v,w}$ lives inside the component $\Gr{A(\Gamma)}{v}{av}{w}{bw}$.

Let us now consider $M(A(\Gamma))$, the multiplier algebra of $A(\Gamma)$. For a function $f$ on $\Gamma{(0)}\times \Gamma^{(0)}$, write $f(\lambda,\rho) = \sum_{v,w} f(v,w)\UnitC{v}{w} \in M(A(\Gamma))$. Similarly, for a function $f$ on $\Gamma^{(0)}$ we write $f(\lambda) = \sum_{v,w} f(v)\UnitC{v}{w}$ and $f(\rho) = \sum_{v,w}f(w)\UnitC{v}{w}$. We then write for example $f(a\lambda,\rho)$ for the element corresponding to the function $(v,w)\mapsto f(av,w)$.

We can further form in $M(A(\Gamma))$ the elements $u_{a,b} = \sum_{v,w} (u_{a,b})_{v,w}$. Then $u=(u_{a,b})$ is a unitary m$\times$m matrix for $m=\#T$. Moreover, \begin{equation}\label{EqAdju}u_{a,b}^* =  u_{\bar{a},\bar{b}}\frac{\gamma_b(\rho)}{\gamma_a(\lambda)},\end{equation} where $\gamma_a(v) = \sgn_a(v)\sqrt{w_a(v)}$.   We then have the following commutation relations between functions on $\Gamma^{(0)}\times \Gamma^{(0)}$ and the entries of $u$: \begin{equation}\label{EqGradu} f(\lambda,\rho)u_{a,b} = u_{a,b}f(\bar{a}v,\bar{b}w).\end{equation} The coproduct is given by $\Delta(u_{a,b}) = \Delta(1) \sum_c(u_{a,c}\otimes u_{c,b})$. Note that the $^*$-algebra generated by the $u_{a,b}$ is no longer a weak Hopf $^*$-algebra when $\Gamma^{(0)}$ is infinite, but rather one can turn it into a Hopf algebroid. %ref.

As a particular example, consider the Podle\'{s} graph of Example \ref{ExaGraphPod} at parameter $x\in \lbrack 0,\frac{1}{2}\rbrack$. Then one can take $T = \{+,-\}$ with the non-trivial involution, and label the edges $(k,k+1)$ with $+$ and the edges $(k+1,k)$ with $-$. Let us write \[F(k) = |q|^{-1}w_+(k) =  |q|^{-1}\frac{|q|^{x+k+1}+|q|^{-x-k-1}}{|q|^{x+k}+|q|^{-x-k}},\] and further put\[\alpha = \frac{F^{1/2}(\rho-1)}{F^{1/2}(\lambda-1)}u_{--},\qquad \beta = \frac{1}{F^{1/2}(\lambda-1)}u_{-+}.\] Then the unitarity of $(u_{\epsilon,\nu})_{\epsilon,\nu}$ together with \eqref{EqAdju} and \eqref{EqGradu} are equivalent to the commutation relations \begin{equation}\label{EqqCom} \alpha \beta = qF(\rho-1)\beta\alpha \qquad \alpha\beta^* = qF(\lambda)\beta^*\alpha\end{equation} \begin{equation}\label{EqDet} \alpha\alpha^* +F(\lambda)\beta^*\beta = 1,\qquad \alpha^*\alpha+q^{-2}F(\rho-1)^{-1}\beta^*\beta = 1,\end{equation}\begin{equation*} F(\rho-1)^{-1}\alpha\alpha^* +\beta\beta^* = F(\lambda-1)^{-1},\qquad  F(\lambda)\alpha^*\alpha +q^{-2}\beta\beta^* = F(\rho),\end{equation*} \begin{equation}\label{EqGrad} f(\lambda)g(\rho)\alpha =
\alpha f(\lambda+1)g(\rho+1),\qquad f(\lambda)g(\rho)\beta = \beta f(\lambda+1)g(\rho-1).\end{equation}%layout might be nicer

These are precisely the commutation relations for the dynamical quantum $SU(2)$-group as in \cite[Definition 2.6]{KoR1}, except that the precise value of $F$ has been changed by a shift in the parameter domain by a complex constant. The (total) coproduct on $A_x$ also agrees with the one on the dynamical quantum $SU(2)$-group, namely \begin{eqnarray*} \Delta(\alpha) &=& \Delta(1) (\alpha\otimes \alpha - q^{-1}\beta\otimes \beta^*),\\ \Delta(\beta) &=& \Delta(1)(\beta\otimes \alpha^* +\alpha\otimes \beta)\end{eqnarray*} where $\Delta(1) = \sum_{k\in \Z} \rho_k\otimes \lambda_k$.



\bibliographystyle{habbrv}
%\bibliographystyle{hplain}
%\bibliographystyle{kp}
\bibliography{references}

\end{document}