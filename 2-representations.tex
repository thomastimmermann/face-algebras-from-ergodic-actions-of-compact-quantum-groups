\section{Representation theory of partial compact quantum groups}

In this section, the representation theory of partial compact quantum groups is investigated. As the situation is quite similar to the case already studied by Hayashi \cite{Hay1}, we do not always provide fully written out proofs, but only draw attention to those parts of the theory which need modification.

In what follows, the homogeneous component $A(K) = \eGr{A}{k}{l}{m}{n}$ of a partial bialgebra will now be mainly written as $A(K) = \Gr{A}{k}{l}{m}{n}$. 

\subsection{Corepresentations of partial Hopf algebras}


Let $\mathscr{A}$ be an $I$-partial bialgebra. We write
$\Hom_\C(V,W)$ for the vector space of linear maps between two vector
spaces $V$ and $W$.



Denote by $\Vecti$ the category whose objects are $I^{2}$-graded
vector spaces $V=\bigoplus_{k,l\in I} \Gru{V}{k}{l}$ and whose
morphisms are linear maps $T$ that preserve the grading and therefore
can be written $T=\bigoplus_{k,l\in I} \Gru{T}{k}{l}$.  Clearly, this
category is abelian and $\C$-linear.  We call an $I^{2}$-graded vector
space $V=\bigoplus_{k,l\in I} \Gru{V}{k}{l}$ \emph{row- and
  column-finite} if $\oplus_l \Gru{V}{k}{l}$ (resp. $\oplus_l
\Gru{V}{k}{l}$) is finite-dimensional for $k$ (resp. $l$) fixed.
We henceforth abbreviate ``row- and column-finite'' by rcf.

\begin{Def} \label{definition:corep} Let $\mathscr{A}$ be an
  $I$-partial bialgebra and let $V=\bigoplus_{k,l} \Gru{V}{k}{l}$ be
an rcf $I^{2}$-graded vector space.  A \emph{corepresentation}
  $\mathscr{X}=(\Gr{X}{k}{l}{m}{n})_{k,l,m,n}$ of $\mathscr{A}$ on $V$
  is a family of elements
 \begin{align} \label{eq:rep-blocks}
   \Gr{X}{k}{l}{m}{n} \in \Gr{A}{k}{l}{m}{n} \otimes
  \Hom_\C(\Gru{V}{m}{n},\Gru{V}{k}{l})
 \end{align}
 satisfying 
 \begin{align}
   \label{eq:rep-comultiplication}
    (\Delta_{pq} \otimes
    \id)(\Gr{X}{k}{l}{m}{n}) &=
    \Big{(}\Gr{X}{k}{l}{p}{q}\Big{)}_{13}\Big{(}\Gr{X}{p}{q}{m}{n}\Big{)}_{23},
    \\ \label{eq:rep-counit}
(\epsilon \otimes
  \id)(\Gr{X}{k}{l}{m}{n})&=\delta_{k,m}\delta_{l,n}\id_{\Gru{V}{k}{l}}
 \end{align}
  for all possible indices. We also call $(V,\mathscr{X})$ an
  \emph{rcf corepresentation}.
\end{Def}
Here, we use here the standard leg numbering notation, e.g $a_{23}=1\otimes a$.
\begin{Exa} \label{example:rep-triv} Equip the vector space
  $\C^{(I)}=\bigoplus_{k\in I} \C$ with the diagonal
  $I^{2}$-grading. Then the family $\mathscr{U}$ given by
  \begin{align} \label{eq:rep-triv}
    \Gr{U}{k}{l}{m}{n} = \delta_{k,l}\delta_{m,n} \UnitC{k}{m} \in
    \Gr{A}{k}{l}{m}{n}
  \end{align}
is a corepresentation of $\mathscr{A}$ on $\C^{(I)}$. We call it the
\emph{trivial corepresentation}.
\end{Exa}

\begin{Exa} \label{example:rep-regular}
  Assume  given an rcf family of subspaces
  \begin{align*}
    \Gru{V}{m}{n} \subseteq \bigoplus_{k,l} \Gr{A}{k}{l}{m}{n}
  \end{align*}
  satisfying
  \begin{align} \label{eq:rep-regular-inclusion}
    \Delta_{pq}(\Gru{V}{m}{n}) &\subseteq \Gru{V}{p}{q} \otimes
    \Gr{A}{p}{q}{m}{n}.
  \end{align}
Then the  elements $\Gr{X}{k}{l}{m}{n} \in \Gr{A}{k}{l}{m}{n} \otimes
  \Hom_{\C}(\Gru{V}{m}{n},\Gru{V}{k}{l})$ defined by 
  \begin{align*}
    \Gr{X}{k}{l}{m}{n}(1 \otimes b) &= \Delta^{\op}_{kl}(b) \in
    \Gr{A}{k}{l}{m}{n} \otimes \Gru{V}{k}{l} \quad
    \text{for all } b\in \Gru{V}{m}{n}
  \end{align*}
  form a corepresentation $\mathscr{X}$ of $\mathscr{A}$ on
  $V$. Indeed, 
  \begin{align*}
    (\Delta_{pq} \otimes \id)(\Gr{X}{k}{l}{m}{n})(1 \otimes 1 \otimes
    b) &=(\Delta_{pq}\otimes \id)(\Delta^{\op}_{kl}(b)) =
    \Big{(}\Gr{X}{k}{l}{p}{q}\Big{)}_{13}\Big{(}\Gr{X}{p}{q}{m}{n}\Big{)}_{23}(1
    \otimes 1 \otimes b), \\
    (\epsilon \otimes \id)(\Gr{X}{k}{l}{m}{n})b &= (\epsilon \otimes
    \id)(\Delta^{\op}_{kl}(b)) = b
  \end{align*}
  for all $b\in \Gru{V}{m}{n}$.  We call $\mathscr{X}$ the
  \emph{regular corepresentation on $V$}. 
\end{Exa}

We next consider the total form of a corepresentation.

Let $\mathscr{A}$ be a partial bialgebra with total algebra $A$ and
let $V$ be an rcf $I^{2}$-graded vector space.
Denote by $\lambda^{V}_{k},\rho^{V}_{l} \in \Hom_{\C}(V)$ the
projections onto the summands $\Gru{V}{k}{} = \bigoplus_{q}
\Gru{V}{k}{q}$ and $\Gru{V}{}{l}=\bigoplus_{p}\Gru{V}{p}{l}$,
respectively, identify $\Hom_{\C}(\Gru{V}{m}{n},\Gru{V}{k}{l})$ with
$\lambda^{V}_{k}\rho^{V}_{l}\Hom_{\C}(V)\lambda^{V}_{m}\rho^{V}_{n}$,
denote by $\Hom_{\C}^{0}(V) \subseteq \Hom_{\C}(V)$ the sum of all
these subspaces, and define a homomorphism
\begin{align*}
  \Delta \otimes \id \colon M(A \otimes \Hom_{\C}^{0}(V)) \to M(A
  \otimes A \otimes \Hom_{\C}^{0}(V))
\end{align*}
similarly as we defined $ \Delta \colon A \to M(A\otimes A)$.
\begin{Lem} \label{lemma:rep-multiplier}
  Let  $\mathscr{A}$ be an $I$-partial bialgebra and $V$  an rcf $I^{2}$-graded vector space.  If $\mathscr{X}$ is a
  corepresentation of  $\mathscr{A}$ on $V$, then the sum
  \begin{align}
    \label{eq:rep-multiplier}
  X:=\sum_{k,l,m,n} \Gr{X}{k}{l}{m}{n} \in  M(A
  \otimes \Hom_{\C}^{0}(V))
  \end{align}
 converges strictly and satisfies the following conditions:
  \begin{enumerate}\setcounter{enumi}{-1}
  \item $(\lambda_{k}\rho_{m} \otimes \id){X}(\lambda_{l}\rho_{n}
    \otimes \id) = (1 \otimes \lambda^{V}_{k}\rho^{V}_{l}){X}(1 \otimes
    \lambda^{V}_{m}\rho^{V}_{n}) = \Gr{X}{k}{l}{m}{n}$,
  \item $(A \otimes 1){X}$, $ {X}(A \otimes 1)$ and $(1 \otimes
    \Hom^{0}_{\C}(V))X(1 \otimes \Hom^{0}_{\C}(V))$ lie in $A \otimes \Hom_{\C}^{0}(V)$,
  \item $(\Delta\otimes \id)(X)=X_{13}X_{23}$, 
  \item the sum $(\epsilon \otimes \id)({X}) :=\sum (\epsilon \otimes
    \id)(\Gr{X}{k}{l}{m}{n})$ converges in $M(\Hom^{0}_{\C}(V))$ strictly
    to $\id_{V}$.
  \end{enumerate}
  Conversely, if $ X \in M(A \otimes \Hom_{\C}^{0}(V))$ satisfies
  0.--3.\ with $\Gr{X}{k}{l}{m}{n}$ defined by 0., then
  $\mathscr{X}=(\Gr{X}{k}{l}{m}{n})_{k,l,m,n}$ is a corepresentation
  of $\mathscr{A}$ on $V$.
\end{Lem}
\begin{proof}
 Straightforward.
\end{proof}



If $\mathscr{A}$ is a partial Hopf algebra,  then every
corepresentation multiplier has a generalized inverse.
\begin{Lem} \label{lemma:rep-invertible}
  Let $(V,\mathscr{X})$ be an rcf corepresentation of an $I$-partial Hopf
  algebra $\mathscr{A}$. Then
  \begin{align*}
    \Gr{X}{k}{l}{m}{n}  (S \otimes
      \id)(\Gr{X}{p}{q}{r}{s}) &=0 \text{ if } (l,m,n)\neq(s,p,q), &
      \sum_{n} \Gr{X}{k}{l}{m}{n} \cdot (S \otimes
      \id)(\Gr{X}{m}{n}{k'}{l}) &= \delta_{k,k'}\UnitC{k}{m} \otimes
      \id_{\Gru{V}{k}{l}}, \\
      (S \otimes \id)(\Gr{X}{k}{l}{m}{n}) \Gr{X}{p}{q}{r}{s} &= 0
      \text{ if } (k,m,n)\neq (r,p,q), & \sum_{m}
      (S \otimes \id)(\Gr{X}{k}{l}{m}{n}) \Gr{X}{m}{n}{k}{l'} &=
      \delta_{l,l'} \UnitC{n}{l} \otimes \id_{\Gru{V}{k}{l}}.
  \end{align*}
  In particular, the multiplier $Z:=     (S \otimes
  \id)(X) \in M(A \otimes \Hom_{\C}^{0}(V))$
  satisfies
  \begin{align} \label{eq:rep-generalized-inverse}
    XZ &= \sum_{k} \lambda_{k} \otimes \lambda^{V}_{k}, &
    ZX &= \sum_{n} \rho_{n} \otimes \rho^{V}_{n},
  \end{align}
  and is a generalized inverse of $X$ in the sense that $XZX=X$ and $ZXZ=Z$.
\end{Lem}
\begin{proof}
  The first equation follows from \eqref{eq:rep-blocks} and the
  relation  $S(\Gr{A}{p}{q}{r}{s})\subseteq \Gr{A}{s}{r}{q}{p}$. To
  verify the second one, we use relations 2.\ and 3.\ in Definition
  \ref{definition:corep} and \eqref{eq:antipode-pi-l} and find
  \begin{align*}
      \sum_{n} \Gr{X}{k}{l}{m}{n} \cdot (S \otimes
      \id)(\Gr{X}{m}{n}{k'}{l}) &= \sum_{n} (m_{A} \circ (\id \otimes S)
      \otimes \id)((\Gr{X}{k}{l}{m}{n})_{13}(\Gr{X}{m}{n}{k'}{l})_{23})
 \\ &= \sum_{n} (m_{A} \circ (\id \otimes S) \circ \Delta_{m,n} \otimes
      \id)(\Gr{X}{k}{l}{k'}{l}) \\
      &= \delta_{k,k'} \UnitC{k}{l} \otimes (\epsilon \otimes
      \id)(\Gr{X}{k}{l}{k'}{l})
      \\ &=
\delta_{k,k'}\UnitC{k}{m} \otimes
      \id_{\Gru{V}{k}{l}},  \end{align*}
where $m_{A}$ denotes the multiplication of $A$. The third and fourth
equation follow similarly, and the assertions concerning $Z$ are
direct consequences.
\end{proof}
\begin{Def}
  Let $\mathscr{X}$ be an rcf corepresentation of a  partial Hopf
  algebra.  We  denote the generalized inverse $(S \otimes \id)(X)$
  of $X$  by $X^{-1}$ and let
  \begin{align*}
   \Gr{(X^{-1})}{k}{l}{m}{n}=(S \otimes \id)(\Gr{X}{k}{l}{m}{n}) \in
   \Gr{A}{n}{m}{l}{k} \otimes \Hom_{\C}(\Gru{V}{m}{n},\Gru{V}{k}{l})
  \end{align*}
\end{Def}
For completeness, we mention the following following converse to Lemma \ref{lemma:rep-invertible}
\begin{Lem}
  Let $\mathscr{A}$ be an $I$-partial bialgebra, $V$ an rcf $I^{2}$-graded vector space and $X,Z \in M(A \otimes
  \Hom_{\C}^{0}(V))$. If conditions 0.--2.\ in Lemma
  \ref{lemma:rep-multiplier} and
  \eqref{eq:rep-generalized-inverse} hold, then the corresponding
  family $\mathscr{X}=(\Gr{X}{k}{l}{m}{n})_{k,l,m,n}$ is a
  corepresentation of $\mathscr{A}$ on $V$.
\end{Lem}
\begin{proof}
  We have to verify condition 3.\ in Lemma
  \ref{lemma:rep-multiplier}.  If $(k,l) \neq (p,q)$, then
  $\epsilon(\Gr{A}{k}{l}{p}{q})=0$ and hence $(\epsilon
  \otimes \id)(\Gr{X}{k}{l}{p}{q}) =0$. The counit property and condition
  2.\ in Lemma \ref{lemma:rep-multiplier} imply 
\begin{align*}
  \Gr{X}{k}{l}{m}{n} &= ((\epsilon\otimes \id)\circ  \Delta \otimes
  \id)(  \Gr{X}{k}{l}{m}{n}) 
\\ &  = \sum_{p,q} (\epsilon\otimes \id \otimes
  \id)\left((\Gr{X}{k}{l}{p}{q})_{13}(\Gr{X}{p}{q}{m}{n})_{23}\right)
  =  (1 \otimes \Gru{T}{k}{l})\Gr{X}{k}{l}{m}{n},
\end{align*}
where $\Gru{T}{k}{l}=(\epsilon \otimes \id)(\Gr{X}{k}{l}{k}{l}) \in
\Hom_{\C}(\Gru{V}{k}{l})$.  Therefore,  $T=\sum_{k,l} T_{k,l}$  satisfies $(1 \otimes T)X =
X$. Multiplying on the right by $Z$, we find
$T\lambda^{V}_{k}=\lambda^{V}_{k}$ for all $k$. Thus, $T=\id_{V}$.
\end{proof}

Morphisms of corepresentations are defined as follows.
\begin{Def}
  Let $\mathscr{A}$ be an $I$-partial bialgebra.  A \emph{morphism}
  $T$ between rcf corepresentations
  $(V,\mathscr{X})$ and $(W,\mathscr{Y})$ of $\mathscr{A}$ is a family
  of linear maps
  \[\Gru{T}{k}{l} \in
  \Hom_\C(\Gru{V}{k}{l},\Gru{W}{k}{l})\] satisfying \[(1 \otimes
  \Gru{T}{k}{l})\Gr{X}{k}{l}{m}{n} = \Gr{Y}{k}{l}{m}{n}(1 \otimes
  \Gru{T}{m}{n})\]
\end{Def}
We denote the category of all corepresentations of $\mathscr{A}$ by
$\Corep(\mathscr{A})$.
\begin{Rem} \label{remark:rep-total-morphism}
 Equivalently, a morphism between
    $(V,\mathscr{X})$ and $(W,\mathscr{Y})$ is just a morphism of
    $I^{2}$-graded vector spaces $T\colon V\to W$ satisfying
    $(1\otimes T) X= Y(1 \otimes T)$. If $\mathscr{A}$ is  a
    partial Hopf algebra, this condition is equivalent to each of the relations
    \begin{align*}
      Y^{-1}(1 \otimes T)X&=\sum_{m,n} \rho_{n} \otimes \Gru{T}{m}{n},
      &
    Y(1\otimes T)X^{-1} &=\sum_{k,l} \lambda_{k} \otimes \Gru{T}{k}{l}.
    \end{align*}
\end{Rem}

The category $\Corep(\mathscr{A})$ is evidently $\C$-linear and the
forgetful functor $\Corep(\mathscr{A})\to \Vecti$ is faithful.  

Given an $I^{2}$-graded vector space $V=\bigoplus_{k,l} \Gru{V}{k}{l}$
and a family of subspaces $\Gru{W}{k}{l} \subseteq \Gru{V}{k}{l}$, we
denote by $\iota_{W}\colon W\to V$ and $\pi_{W} \colon V \to
V/W=\bigoplus_{k,l} \Gru{V}{k}{l}/\Gru{W}{k}{l}$ the embedding and the
quotient map.
\begin{Def} Let $(V,\mathscr{X})$ be an rcf
  corepresentation of a partial bialgebra $\mathscr{A}$.  We call a
  family of subspaces $\Gru{W}{k}{l} \subseteq \Gru{V}{k}{l}$
  \emph{invariant (w.r.t.\ $\mathscr{X}$)} if
 \begin{align} \label{eq:rep-invariant} (1\otimes
   \Gr{\pi}{k}{l}{}{W})\Gr{X}{k}{l}{m}{n}(1 \otimes
   \Gr{\iota}{m}{n}{}{W}) =0,
  \end{align}
and $(V,\mathscr{X})$ 
 \emph{irreducible} if the only invariant families of subspaces are
 $(0)_{k,l}$ and $(\Gru{V}{k}{l})_{k,l}$.
\end{Def}


If a corepresentation has an invariant subspace, it restricts to it
and factorizes to the quotient:
\begin{Lem}
  Let $(V,\mathscr{X})$ be an rcf corepresentation
  of a partial bialgebra and let $\Gru{W}{k}{l}
  \subseteq \Gru{V}{k}{l}$ be an invariant family of subspaces. Then
  there exist unique rcf corepresentations
  $(W,\iota_{W}^{*}\mathscr{X})$ and $(V/W,(\pi_{W})_{*}\mathscr{X})$ 
  such that $\iota_{W}$  and  $\pi_{W}$  are  morphisms
  $(W,\iota_{W}^{*}\mathscr{X}) \to (V,\mathscr{X}) \to (V/W,(\pi_{W})_{*}\mathscr{X})$.
\end{Lem}
\begin{proof}
  Straightforward.
\end{proof}
The following analogue of Schur's Lemma holds.
\begin{Lem} Let $T$ be a morphism of rcf
  corepresentations $(V,\mathscr{X})$ and $(W,\mathscr{Y})$ of a
  partial bialgebra. Then the families of subspaces $\ker
  \Gru{T}{k}{l} \subseteq \Gru{V}{k}{l}$ and $\img\Gru{T}{k}{l}
  \subseteq \Gru{W}{k}{l}$ are invariant.  In particular, if
  $(V,\mathscr{X})$ and $(W,\mathscr{Y})$ are irreducible, then either
  all $\Gru{T}{k}{l}$ are zero or all $\Gru{T}{k}{l}$ are
  isomorphisms.
\end{Lem} 
\begin{proof}
  Straightforward again.
\end{proof}


Given corepresentations $\mathscr{X}$ and $\mathscr{Y}$ of
a partial bialgebra $\mathscr{A}$ on respective rcf $I^{2}$-graded vector spaces $V$ and $W$,
we  obtain an $I^{2}$-graded vector space $V\oplus W$ by taking
component-wise direct sums, and use the canonical embedding 
\begin{align*}
  \Hom(\Gru{V}{m}{n},\Gru{V}{k}{l}) \oplus
  \Hom(\Gru{W}{m}{n},\Gru{W}{k}{l}) \hookrightarrow
  \Hom(\Gru{V}{m}{n} \oplus \Gru{W}{m}{n},\Gru{V}{k}{l} \oplus
  \Gru{W}{k}{l})
\end{align*}
to define the \emph{direct sum} $\mathscr{X} \oplus \mathscr{Y}$,
which is a corepresentation of $\mathscr{A}$ on $V\oplus W$. Then the
natural embeddings from $V$ and $W$ into $V\oplus W$ and the
projections onto $V$ and $W$ are evidently morphisms of
corepresentations.  More generally, given a family of rcf corepresentations
$((V_{\alpha},\mathscr{X}_{\alpha}))_{\alpha}$ such that the sum
$\bigoplus_{\alpha} V_{\alpha}$ is rcf again, one
can form the direct sum $\bigoplus_{\alpha} \mathscr{X}_{\alpha}$,
which is a corepresentation on $\bigoplus_{\alpha} V_{\alpha}$.
\begin{Prop}
  Let $\mathscr{A}$ be an $I$-partial bialgebra. Then $\Corep(\mathscr{A})$
  is a $\C$-linear abelian category and the forgetful functor
  $\Corep(\mathscr{A}) \to \Vecti$ lifts kernels, cokernels and biproducts.
\end{Prop}
\begin{proof}
  The preceding considerations show that the forgetful functor lifts
  kernels, cokernels and biproducts. Moreover, in
  $\Corep(\mathscr{A})$, every monic is a kernel
  and every epic is a cokernel because the same is true in $\Vecti$
  and because kernels and cokernels lift.
\end{proof}


\subsection{The tensor product and duality}
The category $\Vecti$ is a tensor category, where
the product of $I^{2}$-graded vector spaces $V$ and $W$ is the sum of
the subspaces
\begin{align*}
 \Gru{(V\itimes W)}{k}{l} =
  \bigoplus_{p} (\Gru{V}{k}{p}\otimes \Gru{V}{p}{l}) \subseteq
  V\otimes W,
\end{align*}
which we denote by $V \itimes W$, and the product of morphisms is the
restriction of the ordinary tensor product.  We pretend this product
to be strictly associative.  The unit for this product is the vector
space $\C^{(I)}=\bigoplus_{k\in I} \C$. Indeed, for every
$I^{2}$-graded vector space $V$, there exist obvious natural
isomorphisms $\C^{(I)} \itimes V \cong V \cong V \itimes \C^{(I)}$.

Note that $V\itimes W$ is  rcf if $V$ and $W$ are.

Given $V$ and $W$ as above, we identify $\Hom_\C(\Gru{V}{m}{n},\Gru{V}{k}{l})\otimes
   \Hom_\C(\Gru{W}{n}{q},\Gru{W}{l}{p})$ with a subspace of
\begin{align*}
   \Hom_\C(\Gru{V}{m}{n}\otimes
   \Gru{W}{n}{q},\Gru{V}{k}{l}\otimes \Gru{W}{l}{p})\subseteq
   \Hom_\C(\Gru{(V\itimes
     W)}{m}{q},\Gru{(V\itimes W)}{k}{p}).
\end{align*}


We can now construct a product of corepresentations as follows.
\begin{Lem} Let $\mathscr{X}$ and $\mathscr{Y}$ be copresentations of
  $\mathscr{A}$ on respective  rcf $I^{2}$-graded vector spaces $V$ and
  $W$. Then the sum
  \begin{align} \label{eq:rep-product-blocks}
     \Gr{(X\Circt Y)}{k}{p}{m}{q} := \sum_{l,n}
    \left(\Gr{X}{k}{l}{m}{n}\right)_{12}\left(\Gr{Y}{l}{p}{n}{q}\right)_{13}
  \end{align}
  has only finitely many non-zero terms and the elements
 \[\Gr{(X\Circt
    Y)}{k}{p}{m}{q}\in \Gr{A}{k}{p}{m}{q} \otimes
  \Hom_\C(\Gru{(V\itimes W)}{m}{q},\Gru{(V\itimes W)}{k}{p})
\]
define an rcf corepresentation $\mathscr{X} \Circt \mathscr{Y}$ of
$\mathscr{A}$ on $V\itimes W$. 
\end{Lem} 
\begin{proof}
  The sum \eqref{eq:rep-product-blocks} is finite because $V$ and
  $W$ are  rcf. Using the identification above, we
  see that
 $
  \left(\Gr{X}{k}{l}{m}{n}\right)_{12}\left(\Gr{Y}{l}{p}{n}{q}\right)_{13}
  $ lies in the tensor product $ \Gr{A}{k}{p}{m}{q} \otimes \Hom_\C(\Gru{(V\itimes
    W)}{m}{q},\Gru{(V\itimes W)}{k}{p})$. Now,   the fact that $\Gr{(X\Circt
    Y)}{k}{p}{m}{q}$ is a corepresentation follows easily
  from the multiplicativity of $\Delta$ and the weak multiplicativity
  of $\epsilon$.
\end{proof}
\begin{Rem} \label{remark:rep-tensor-multiplier}
  The ``total'' multiplier associated to $\mathscr{X}\Circt
  \mathscr{Y}$   is  just $X_{12}Y_{13}$.
\end{Rem}

\begin{Prop} \label{prop:rep-tensor} Let $\mathscr{A}$ be an
  $I$-partial bialgebra. Then  $\Corep(\mathscr{A})$ carries the
  structure of strict tensor category such that the product of rcf corepresentations $(V,\mathscr{X})$ and
  $(W,\mathscr{Y})$ is the corepresentation $(V\itimes
  W,\mathscr{X}\Circt \mathscr{Y})$, the unit is the trivial
  corepresentation $(\C^{(I)},\mathscr{U})$, and the forgetful functor
  $\Corep(\mathscr{A}) \to \Vecti$ is a strict tensor functor.
\end{Prop}
\begin{proof}
This is clear.
\end{proof}

Given a corepresentation of a partial Hopf algebra, one can use the
antipode to define a contragredient corepresentation on a dual space.
Denote the dual of vector spaces $V$ and  linear maps $T$ by
$\dual{V}$ and $\dual{T}$, respectively, and define the dual of an
$I^{2}$-graded vector space $V=\bigoplus_{k,l} \Gru{V}{k}{l}$ to be
the space
\begin{align*}
  \dual{V}=\bigoplus_{k,l} \Gru{(\dual{V})}{k}{l}, \quad \text{where }
\Gru{(\dual{V})}{k}{l} = \dual{(\Gru{V}{k}{l})}.
\end{align*}


Recall that an object $X$ in a strict tensor category is called a
\emph{right dual} of an object $Y$ and $Y$ is called a \emph{left
  dual} of $X$, if there are morphisms $X \otimes Y \to 1$ and $1 \to
Y \otimes X$, where $1$ denotes the tensor unit, such
that the obvious compositions
  \begin{gather*}
 X \otimes 1 \to X\otimes Y\otimes X
  \to 1 \otimes X  \quad\text{and} \quad
     1 \otimes Y \to Y \otimes X \otimes Y \to Y
    \otimes 1 
  \end{gather*}
  are the identity of $X$ and $Y$,
  respectively.  


\begin{Prop}
  Let $\mathscr{A}$ be an $I$-partial Hopf algebra with antipode $S$
  and  let $(V,\mathscr{X})$ be an rcf
  corepresentation of $\mathscr{A}$. Then $\dual{V}$ and the family
  $\dual{\mathscr{X}}$ given by
   \begin{align} \label{eq:rep-right-dual}
\Gr{(\dual{X})}{k}{l}{m}{n}   :=  (S \otimes \dual{-})(\Gr{X}{n}{m}{l}{k}) 
   \end{align} 
   form an rcf corepresentation of $\mathscr{A}$  which is a right dual of $(V,\mathscr{X})$. If the antipode
   $S$ of $\mathscr{A}$ is bijective, then $\dual{V}$ and the family
   $\predual{\mathscr{X}}$ given by 
   \begin{align} \label{eq:rep-left-dual}
 \Gr{(\predual{X})}{k}{l}{m}{n} :=(S^{-1}
   \otimes \dual{-})(\Gr{X}{n}{m}{l}{k})    
   \end{align}
 form an rcf corepresentation
 of $\mathscr{A}$ which is a
   left dual of $(V,\mathscr{X})$.
  \end{Prop}
  \begin{proof}
    We only prove the assertion concerning
    $(\dual{V},\dual{\mathscr{X}})$. To see that this is a corepresentation, note that the element
    \eqref{eq:rep-right-dual} belongs to $\Gr{A}{k}{l}{m}{n} \otimes
    \Hom_{\C}(\Gru{(\dual{V})}{m}{n},\Gru{(\dual{V})}{k}{l})$ and use
    the relations $\Delta \circ S = (S \otimes S)\Delta^{\op}$ and
    $\epsilon \circ S = \epsilon$ from Corollary
    \ref{corollary:antipode}.  
    Let us show that $(\dual{V},\dual{\mathscr{X}})$ is a right dual
    of $(V,\mathscr{X})$.

    Given a finite-dimensional vector space $W$, denote by $T_{W}
    \colon \dual{W} \otimes W \to \C$ the evaluation map and by $R_{W}
    \colon \C \to W \otimes \dual{W}$ the coevaluation map, given by
    $1\mapsto \sum_{i} w_{i} \otimes \dual{w_{i}}$ if $(w_{i})_{i}$
    and $(\dual{w_{i}})_{i}$ are dual bases of $W$ and
    $\dual{W}$. With respect to these maps, $\dual{W}$ is a right dual
    of $W$. If $F\colon W_{1}\to W_{2}$ is a linear map between
    finite-dimensional spaces, then
\begin{align} \label{eq:coev-vee} (\id_{W_{2}} \otimes F^{\vee}) \circ R_{W_{2}} &= (F \otimes \id_{W_{1}^{\vee}})\circ
  R_{W_{1}}, &
T_{W_{1}}(F^{\vee}
  \otimes \id_{W_{2}})&=  T_{W_{2}}(\id_{W_{2}^{\vee}} \otimes F).
\end{align}

Now, define morphisms $R \colon \C^{(I)} \to V\itimes \dual{V}$ and
$T \colon \dual{V} \itimes V \to \C^{(I)}$ by
\begin{align*}
  \Gru{R}{k}{l} &= \delta_{k,l} \sum_{p} R_{\footnotesize\Gru{V}{k}{p}} \colon
  \C \to 
    \Gru{(V\itimes \dual{V})}{k}{l}, &
  \Gru{T}{k}{l} &= \delta_{k,l} \sum_{k,p} T_{(\Gru{V}{p}{k})} \colon
    \Gru{(V\itimes \dual{V})}{k}{l} \to \C.
\end{align*}
One easily checks that with respect to these maps, $\dual{V}$ is a
right dual of $V$ in $\Vecti$. 

We therefore only need to show that $R$ is a morphism from
$\mathscr{X}\Circt\dual{\mathscr{X}}$ to $\mathscr{U}$ and that $T$ is
a morphism from $\mathscr{U}$ to
$\dual{\mathscr{X}}\Circt{\mathscr{X}}$.  But \eqref{eq:coev-vee} and
Lemma \ref{lemma:rep-invertible} imply
  \begin{align*}
    (1\otimes \Gru{T}{k}{k})
 \sum_{l,n}  \big(
\Gr{(\dual{X})}{k}{l}{m}{n}\big)_{12}
\big(\Gr{X}{l}{k}{n}{q}\big)_{13} &=
    (1\otimes \Gru{T}{k}{k})
 \sum_{l,n} 
(S \otimes \dual{-})(\Gr{X}{n}{m}{l}{k})_{12}
    (\Gr{X}{l}{k}{n}{q})_{13} \\ &=
(\id\otimes \Gru{T}{m}{m})  \sum_{l,n}
      (S \otimes \id)(\Gr{X}{n}{m}{l}{k})_{13}(\Gr{X}{l}{k}{n}{q})_{13} \\
    &= \delta_{m,q}\UnitC{k}{q}\otimes \Gru{T}{m}{m} \\
    &= \Gr{U}{k}{k}{m}{q}(1 \otimes \Gru{T}{m}{m}),
  \end{align*}
and a similar  calculation shows that 
\begin{align*}
 \sum_{l,n}  \big(
\Gr{X}{k}{l}{m}{n}\big)_{12}
\big(\Gr{(\dual{X})}{l}{p}{n}{m}\big)_{13} 
    (1\otimes \Gru{R}{m}{m})
&= \delta_{k,p} \UnitC{k}{m} \otimes \Gru{R}{k}{k} = (1 \otimes
\Gru{R}{k}{k}) \Gr{U}{k}{p}{m}{m},
\end{align*}
whence the claim follows.
\end{proof}
\begin{Cor} \label{cor:rep-tensor-duality}
  Let $\mathscr{A}$ be a partial Hopf algebra. Then
  $\Corep(\mathscr{A})$ is a tensor category with right
  duals and, if the antipode of $\mathscr{A}$ is invertible, with left duals.
\end{Cor}

Let $\mathscr{A}$ be an $I$-partial Hopf algebra.  Then the tensor
unit in $\Corep(\mathscr{A})$, which is the trivial corepresentation
$\mathscr{U}$ on $\C^{(I)}$, need not be irreducible and decomposes
into irreducible corepresentations that correspond to equivalence
classes for the relation $\sim$ on $I$ given by  $k \sim l \leftrightarrow
  \UnitC{k}{l}\neq 0$ (see Remark
\ref{remark:index-equivalence}).
\begin{Lem}
  Let $\mathscr{A}$ be an $I$-partial Hopf algebra and let
  $(I_{\alpha})_{\alpha}$ be a labelled partition of $I$ into
  equivalence classes for the relation $\sim$.  Then for each $\alpha$, the subspace
  $\C^{(I_{\alpha})} \subseteq \C^{(I)}$ is invariant and the restriction
  $\mathscr{U_{\alpha}}$ of $\mathscr{U}$ to $\C^{(I_{\alpha})}$ is
  irreducible. In particular, $\mathscr{U}=\bigoplus_{\alpha}
  \mathscr{U_{\alpha}}$ is a decomposition into irreducible corepresentations.
\end{Lem}
\begin{proof}
Immediate from the fact that $\Gr{U}{k}{k}{m}{m} = 
  \UnitC{k}{m}$  is $1$  if $k\sim m$  and $0$ if $k\not\sim m$. 
\end{proof}
The decomposition of the tensor unit leads to a decomposition of the
whole tensor category into full subcategories, where the tensor
product acts like the multiplication in a partial algebra. Let us
briefly sketch the general categorical idea, which can conveniently be
formulated using enriched categories, see, e.g.\ \cite{}, and apply it
to $\Corep(\mathscr{A})$.  Denote by $\C$-$\mathbf{Cat}$ the category
of all small $\C$-linear abelian categories. This category is monoidal
with respect to the cartesian product, where the unit is the category
with just one object with endomorphism algebra $\C$.
\begin{Def}
  A \emph{strict partial  tensor category} is a small
  $\C$-$\mathbf{Cat}$-category $\mathscr{C}$, that is, a set $J$ (the object set)
  together with
  \begin{itemize}
  \item for each $\alpha,\beta \in J$, a $\C$-linear abelian category
    $\Gru{C}{\alpha}{\beta}$,
  \item for each $\alpha,\beta,\gamma\in J$, a functor
    \begin{align*}
      -\underset{\beta}{\otimes}- \colon
      \Gru{C}{\alpha}{\beta} \times
      \Gru{C}{\beta}{\gamma} \to
      \Gru{C}{\alpha}{\gamma},
    \end{align*}
  \item for each $\alpha \in J$, a unit object $U_{\alpha} \in
    \Gru{C}{\alpha}{\alpha}$,
  \end{itemize}
satisfying the obvious strict associativity and strict unitality conditions.
\end{Def}

Let $C$ be a strict tensor category, where the tensor unit $U$
decomposes as a direct sum $U=\bigoplus_{\alpha \in J}
U_{\alpha}$. Then there exists a unique strict partial tensor category
$\mathscr{C}$, where each $\Gru{C}{\alpha}{\beta}$ is the full
subcategory of $C$ formed by all objects $U_{\alpha} \otimes X \otimes
U_{\beta}$ for $X \in C$, and $-\underset{\beta}{\otimes}-$ is the
restriction of the tensor product on $C$. Note that the natural maps
$U_{\alpha} \leftrightarrows U$ induce corresponding natural
transformations $C \leftrightarrows \Gru{C}{\alpha}{\beta}$ for all $\alpha,\beta$.

Conversely, to every strict partial  tensor category
$\mathscr{C}$, one can associate a ``total''  
category $C$, where the objects are all families of objects
$\Gru{X}{\alpha}{\beta} \in \Gru{C}{\alpha}{\beta}$ with only finitely
many non-zero components, and morphisms are all
families of morphisms of the components. This total category $C$
carries a  product given by
\begin{align*}
  (\Gru{X}{\alpha}{\beta})_{\alpha,\beta} \otimes
  (\Gru{Y}{\beta}{\gamma})_{\beta,\gamma} = \big(\bigoplus_{\beta}
(\Gru{X}{\alpha}{\beta} \underset{\beta}{\otimes}   \Gru{X}{\beta}{\gamma})\big)_{\alpha,\gamma},
\end{align*}
which we can can pretend to be strict. Note that in general,
this product has no unit but that the $U_{\alpha}$ can be used to from  ``local units''. 





\subsection{Decomposition into irreducible corepresentations}



An integral allows to average morphisms of vector spaces so that one
obtains morphisms of corepresentations. 
\begin{Lem} \label{lem:rep-average}  Let $(V,\mathscr{X})$ and
  $(W,\mathscr{Y})$ be rcf corepresentations of  a partial
  Hopf algebra $\mathscr{A}$ with an integral $\phi$ and let
  $\Gru{T}{k}{l} \in \Hom_{\C}(\Gru{V}{k}{l},\Gru{W}{k}{l})$ for all $k,l\in I$. Then the families
  \begin{align*}
    \Gr{\check T}{k}{l}{}{n} &:= \sum_{m} (\phi \otimes
    \id)(\Gr{(Y^{-1})}{k}{l}{m}{n}(1\otimes
    \Gru{T}{m}{n})\Gr{X}{m}{n}{k}{l}), \\
    \Gr{\hat T}{k}{l}{m}{} &:= \sum_{n} (\phi \otimes
    \id)(\Gr{Y}{k}{l}{m}{n}(1\otimes
    \Gru{T}{m}{n})\Gr{(X^{-1})}{m}{n}{k}{l})
  \end{align*}
form  morphisms $\Grd{\check{T}}{}{n}$ and $\Grd{\hat{T}}{m}{}$ from $(V,\mathscr{X})$ to $(W,\mathscr{Y})$. % Slightly changed the averaging so that I do not need a restriction of finite support on $T$.
\end{Lem} 
\begin{proof}
  In total form, $\check{T}=(\phi \otimes \id)(Y^{-1}(1 \otimes T)X)$
  and $\hat{T}=(\phi \otimes \id)(Y(1 \otimes T)X^{-1})$.  Now, Lemma
  \ref{lemma:rep-multiplier} and \ref{lemma:total-integral} 
  imply
  \begin{align*}
    Y^{-1}(1 \otimes \check{T})X &= (\phi \otimes \id \otimes
    \id)((Y^{-1})_{23}(Y^{-1})_{13}(1 \otimes 1
    \otimes T)X_{13}X_{23})  \\
    &= ((\phi \otimes\id) \circ \Delta  \otimes \id)(Y^{-1}(1 \otimes T)X) \\
    &= \sum_{n} \rho_{n} \otimes (\phi \otimes \id)((\rho_{n} \otimes
    1)Y^{-1}(1 \otimes T)X)  \\
    &= \sum_{m,n} \rho_{n} \otimes \Gru{\check T}{m}{n},
  \end{align*}
  whence $\check{T}$ is a morphism from $\mathscr{X}$ to $\mathscr{Y}$
  by Remark \ref{remark:rep-total-morphism}. The assertion on $\hat
  T$ follows similarly.
\end{proof}

% Choose a representative family of unitary irreducible locally finite
% corepresentations
% $(\Grd{\mathcal{H}}{(\alpha)}{},{_{(\alpha)}X})_{\alpha}$ and a basis
% $(\Gr{\zeta}{k}{l}{(\alpha)}{i})_{i}$ for each
% $\Gr{\mathcal{H}}{k}{l}{(\alpha)}{}$, and let
% \begin{align*}
%   (\Gr{(u_{\alpha})}{k}{l}{m}{n}){i,j} &:= (\id \otimes
%   \Gr{\omega}{k}{l}{(\alpha)}{i,j})( )
% \end{align*}

\begin{Lem}
  Let $\mathscr{A}$ be a partial Hopf algebra with an integral $\phi$.
  Let $(V,\mathscr{X})$ be an rcf corepresentation
  and $\Gru{W}{k}{l} \subseteq \Gru{V}{k}{l}$ and invariant family of
  subspaces. Then there exists an idempotent endomorphism $T$ of
  $(V,\mathscr{X})$ such that $\Gru{W}{k}{l}=\img\Gru{T}{k}{l}$ for
  all $k,l$.
\end{Lem}
\begin{proof}
  Choose idempotent endomorphisms $\Gru{T}{k}{l}$ of $\Gru{V}{k}{l}$
  with image $\Gru{W}{k}{l}$. We apply the preceding lemma, obtain an
  endomorphism $\Grd{\check{T}}{}{n}$ of $(V,\mathscr{X})$, and show
  that $\img \Gr{\check{T}}{k}{l}{}{n} = \img\Gru{T}{k}{l}= \Gru{W}{k}{l}$.   In
  total form, invariance of $W$ implies  $(1 \otimes T)X(1
  \otimes T)=X(1\otimes T)$. Applying
 $(S \otimes \id)$, we get   $(1 \otimes T)X^{-1}(1
  \otimes T)=X^{-1}(1\otimes T)$.
Now, we combine  Lemma
  \ref{lemma:rep-multiplier}, Lemma \ref{lemma:rep-invertible} and
  normalisation of $\phi$, and find
  \begin{align*}
    \check{T}_{n} T &= (\phi \otimes \id)(X^{-1}(1 \otimes
    T\rho_{n}^{V})X(1 \otimes T)) \\ &=
(\phi \otimes \id)(X^{-1}X(\lambda_{n} \otimes T))
 = \sum_{m} \phi(\UnitC{n}{m}) \rho^{V}_{m}T =T
  \end{align*}
  and similarly $T\check{T}_{n}=T$. Therefore,
  $\img{\check{T}_{n}}=\img T$. \texttt{Need to insert a remark on the
  choice of $n$, taking into account the decomposition into components
of corepresentations --- namely, there exists an $n$ such that
$\Gru{V}{}{m}=0$ whenever $\UnitC{n}{m}=0$}
\end{proof}
\begin{Prop}  \label{prop:rep-cosemisimple}
  Let $\mathscr{A}$ be a partial Hopf algebra with an integral.  Then
  every rcf corepresentation of $\mathscr{A}$ decomposes into a direct
  sum of irreducible rcf corepresentations.
\end{Prop} 
\begin{proof} 
The preceding lemma shows that  every corepresentation is either
irreducible or the direct sum of two corepresentations. 

% Trivial rep should maybe be introduced in a more conspicuous place
\texttt{Put that into the first or second subsection}
Let us now first show that the trivial representation decomposes into irreducibles. Let $I$ be the object set of $\mathscr{A}$, and say $k\sim l$ if $\UnitC{k}{l}\neq 0$. Then $\sim$ is an equivalence relation: as \[\Delta_{ll}(\UnitC{k}{m}) = \UnitC{k}{l}\otimes \UnitC{l}{m},\] the relation $\sim$ is transitive. As $S(\UnitC{k}{l}) = \UnitC{l}{k}$, we have that $\sim$ is symmetric. And as $\varepsilon(\UnitC{k}{k})=1$, we also have that $\sim$ is reflexive. 

Let then $I = \sqcup_{\alpha\in \mathscr{I}} I_{\alpha}$ be a labeled partition associated to $\sim$. Define $\C_{I_{\alpha}}\subseteq \C_I$ as the linear span of the homogeneous components with index in $\alpha$. It is clear then that the $\C_{I_{\alpha}}$ are invariant and irreducible.

Consider now a general corepresentation $(X,\Hsp)$. Let $\Grd{\Hsp}{\alpha}{\beta}$ be the closed linear span of the homogeneous components with index in $\alpha\times \beta$. As we can identify \[\Grd{\Hsp}{\alpha}{\beta} \cong \C_{I_{\alpha}}\,\Circt\, \Hsp\,\Circt\, \C_{I_{\beta}},\] we see that $\Grd{\Hsp}{\alpha}{\beta}$ is an invariant subspace of $\Hsp$. Hence we may as well suppose that $\Hsp = \Grd{\Hsp}{\alpha}{\beta}$. 

But let then $T$ be a bounded self-intertwiner of $\Hsp$. Then from the two equations in Remark \ref{RemMorRep}, we see that $T\rightarrow \Gru{T}{k}{l}$ is injective for any choice of $k\in \alpha,l\in \beta$. It follows that the algebra of self-intertwiners of $\Hsp$ is finite-dimensional. We then immediately conclude that $\Hsp$ is a finite direct sum of irreducible invariant subspaces.
\end{proof} 

\begin{Cor} \label{cor:rep-irreducible-bidual}
  Let $\mathscr{A}$ be a partial Hopf algebra with an integral. Then
  every irreducible rcf corepresentation $(V,\mathscr{X})$ of
  $\mathscr{A}$ is equivalent to its right bidual
  $(V,\dual{\dual{\mathscr{X}}{}\!})$.
\end{Cor}
\begin{proof}
  The corepresentations above are a left and a right dual of
  $(\dual{V},\dual{\mathscr{X}})$. But in every semi-simple tensor
  category, left and right duals are isomorphic \cite{}. 
\end{proof}

\subsection{Matrix coefficients of irreducible corepresentations}

Our next goal is to obtain the analogue of Schur's orthogonality
relations for matrix coefficients of corepresentations.

Given finite-dimensional vector spaces $V$ and $W$, the dual space of
$\Hom_{\C}(V,W)$ is linearly spanned by functionals of the form
\begin{align*}
  \omega_{f,v} \colon \Hom_{\C}(V,W) \to \C, \quad T \mapsto  (f|Tv),
\end{align*}
where $v\in V$, $f\in \dual{W}$, and $(-|-)$ denotes the natural
pairing of $\dual{W}$ with $W$.
\begin{Def} Let $\mathscr{A}$ be a partial bialgebra. The space of
  \emph{matrix coefficients} $\mathcal{C}(\mathscr{X})$ of an rcf
  corepresentation $(V,\mathscr{X})$ is the sum of the subspaces
\begin{align*}
  \Gr{\mathcal{C}(\mathscr{X})}{k}{l}{m}{n} &= \span \left\{ (\id \otimes
    \omega_{f,v})(\Gr{X}{k}{l}{m}{n}) \mid v\in \Gru{V}{m}{n}, f \in
    \dual{(\Gru{V}{k}{l})} \right\} \subseteq \Gr{A}{k}{l}{m}{n}.
\end{align*}
\end{Def}
Let $(V,\mathscr{X})$ be  an rcf corepresentation of a partial bialgebra
$\mathscr{A}$.  Condition 2 in Definition \ref{definition:corep}
implies
\begin{align} \label{eq:rep-matrix-delta}
  \Delta_{pq}(\Gr{\mathcal{C}(\mathscr{X})}{k}{l}{m}{n}) \subseteq
  \Gr{\mathcal{C}(\mathscr{X})}{k}{l}{p}{q} \otimes
  \Gr{\mathcal{C}(\mathscr{X})}{p}{q}{m}{n}.
\end{align}
Thus, the $\Gr{\mathcal{C}(\mathscr{X})}{k}{l}{m}{n}$ form a partial
coalgebra with respect to $\Delta$ and $\epsilon$.  Moreover, for each
$k,l$, the $I^{2}$-graded vector  space
\begin{align*}
  \Gru{\mathcal{C}(\mathscr{X})}{k}{l}:=\bigoplus_{m,n }
  \Gr{\mathcal{C}(\mathscr{X})}{k}{l}{m}{n}
\end{align*}
is rcf, and the inclusion above shows that one can
form the regular corepresentation on this space.
\begin{Lem} \label{lemma:rep-regular-embedding}
  Let $(V,\mathscr{X})$ be an rcf corepresentation
  of a partial bialgebra and let $f\in
  \dual{(\Gru{V}{k}{l})}$. Then the family of maps
  \begin{align*}
    \Gr{T}{m}{n}{}{(f)} \colon \Gru{V}{m}{n} \to
    \Gr{\mathcal{C}(\mathscr{X})}{k}{l}{m}{n}, \ w \mapsto (\id
    \otimes \omega_{f,w})(\Gr{X}{k}{l}{m}{n})=(\id \otimes
    f)(\Gr{X}{k}{l}{m}{n}(1 \otimes w)),
  \end{align*}
  is a morphism from $\mathscr{X}$ to the regular corepresentation on
  $\Gru{\mathcal{C}(\mathscr{X})}{k}{l}$.
\end{Lem}
\begin{proof}
  Denote by $\mathscr{Y}$ the regular corepresentation on
  $\bigoplus_{m,n } \Gr{\mathcal{C}(\mathscr{X})}{k}{l}{m}{n}$. Then
  \begin{align*}
    \label{eq:1}
 \Gr{Y}{p}{q}{m}{n}    (1\otimes \Gr{T}{m}{n}{}{(f)}(v)) &= 
(\Delta^{\op}_{pq} \otimes \omega_{f,v})( \Gr{X}{k}{l}{m}{n}) 
\\ & = (\id \otimes \id \otimes
f)((\Gr{X}{k}{l}{p}{q})_{23}(\Gr{X}{p}{q}{m}{n})_{13}(1 \otimes 1
 \otimes v)) \\ &=(1 \otimes \Gr{T}{p}{q}{}{(f)})\Gr{X}{p}{q}{m}{n}(1 \otimes v)
  \end{align*}
for all $v \in \Gru{V}{m}{n}$.
\end{proof}
\begin{Prop} \label{prop:rep-weak-pw} Let $\mathscr{A}$ be a partial
  Hopf algebra with an integral. Then the total algebra $A$ is the sum
  of the matrix coefficients of irreducible rcf corepresentations.
\end{Prop}
\begin{proof} 
  Let $a \in \Gr{A}{k}{l}{m}{n}$. Write
  \begin{align*}
    \Delta_{pq}(a)=\sum_{i} b_{pq}^{i} \otimes c^{i}_{pq}
  \end{align*}
 with linearly independent
  $(c_{pq}^{i})_{i}$. Then the family of subspaces
  \begin{align*}
    \Gru{V}{p}{q} = \mathrm{span}\{b_{pq}^{i} : i \}\subseteq \bigoplus_{k,l}
  \Gr{A}{k}{l}{m}{n}
  \end{align*}
is rcf, and the relation
  \begin{align*}
 \sum_{i}
    \Delta_{rs}(b^{i}_{pq}) \otimes c^{i}_{pq} =
    (\Delta_{rs} \otimes \id)\Delta_{pq}(a) = (\id \otimes
    \Delta_{pq}) \Delta_{rs}(a) = \sum_{j} b^{j}_{rs} \otimes
    \Delta_{pq}(c^{j}_{rs})
  \end{align*}
  implies $\Delta_{rs}(\Gru{V}{p}{q}) \subseteq \Gru{V}{r}{s} \otimes
  \Gr{A}{r}{s}{p}{q}$.  We can therefore form the regular
  corepresentation $\mathscr{X}$ on $V$ as in Example \ref{example:rep-regular}, and
  \begin{align*}
    a = (\id \otimes \epsilon)(\Delta^{\op}_{mn}(a)) =
    (\id \otimes \epsilon)(\Gr{X}{m}{n}{m}{n}(1 \otimes a)) \in
    \Gr{\mathcal{C}(\mathscr{X})}{m}{n}{m}{n}.
  \end{align*}
  Decomposing $(V,\mathscr{X})$, we find that
  $a$ is contained in the sum of matrix coefficients of irreducible
rcf  corepresentations.
\end{proof}


The first part of the orthogonality relations concerns matrix
coefficients of inequivalent irreducible corepresentations. 
\begin{Prop} \label{prop:rep-orthogonality-1} Let $\mathcal{A}$ be a
  partial Hopf algebra with an integral $\phi$ and inequivalent
  irreducible rcf corepresentations $(V,\mathscr{X})$ and
  $(W,\mathscr{Y})$.  Then  for all
  $a\in \mathcal{C}(X), b \in \mathcal{C}(Y)$,
  \[\phi(S(b)a) = \phi(bS(a))=0.\]
\end{Prop}
\begin{proof}
Since $\phi$ vanishes on $S(\Gr{A}{k}{l}{m}{n})\Gr{A}{p}{q}{r}{s}$ and
on $\Gr{A}{p}{q}{r}{s}S(\Gr{A}{k}{l}{m}{n})$ unless
$(p,q,r,s) = (m,n,k,l)$, it suffices to prove the assertion for  elements of the form
\begin{align*}
  a&=(\id \otimes \omega_{f,v})(\Gr{X}{k}{l}{m}{n})  && \text{and} &
  b&=(\id \otimes \omega_{g,w})(\Gr{Y}{k}{l}{m}{n})
\end{align*}
where $f\in \dual{(\Gru{V}{k}{l})}, v \in \Gru{V}{m}{n}$ and $g \in
\dual{(\Gru{W}{k}{l})}, w \in \Gru{W}{m}{n}$.  Lemma
\ref{lem:rep-average}, applied to the family
  \begin{align*}
    \Gru{T}{p}{q} \colon \Gru{V}{p}{q} \to \Gru{W}{p}{q}, \quad x
    \mapsto  \delta_{p,k}\delta_{q,l}  f(x)w,
  \end{align*}
  yields morphisms $\check{T}_l,\hat{T}_k$ from $(V,\mathscr{X})$ to
  $(W,\mathscr{Y})$ which necessarily are $0$. Inserting the
  definition of $\check{T}_l$, we find
  \begin{align*}
    \phi(S(b)a) &= \phi\big((S \otimes
    \omega_{g,w})(\Gr{Y}{m}{n}{k}{l}) \cdot (\id \otimes
    \omega_{f,v})(\Gr{X}{k}{l}{m}{n})\big) \\ &= (\phi \otimes \omega_{g,v})\left(\Gr{(Y^{-1})}{l}{k}{n}{m}(1 \otimes
      \Gru{T}{k}{l} )     \Gr{X}{k}{l}{m}{n}\right) 
    = \omega_{g,v}( \Gru{\check{T}_l}{m}{n}) = 0.
  \end{align*}% Resort notation on using leg notation, physics bra-ket, etc.
  
  A similar calculation involving $\hat{T}$ shows that
  $\phi(bS(a))=0$.  
\end{proof}

\begin{Theorem} \label{thm:rep-orthogonality} Let $\mathcal{A}$ be a
  partial Hopf algebra with an integral $\phi$. Let $(V,\mathscr{X})$
  be an irreducible rcf corepresentation of $\mathscr{A}$, let
  $F_{\mathscr{X}}$ be an isomorphism from $(V,\mathscr{X})$ to
  $(V,\dual{\dual{\mathscr{X}}\!{}})$ and let
  $G_{\mathscr{X}}=F_{\mathscr{X}}^{-1}$.
  \begin{enumerate}
  \item The numbers $\alpha:=\sum_{k} \Tr (\Gr{G}{k}{l}{}{\mathscr{X}})$
    and $\beta:=\sum_{n} \Tr (\Gr{F}{m}{n}{}{\mathscr{X}})$ do not depend on $l$
    or $n$.
  \item  For all $k,l,m,n$,
    \begin{align*}
      (\phi \otimes \id)(\Gr{(X^{-1})}{m}{n}{k}{l}\Gr{X}{k}{l}{m}{n})
      &=\alpha^{-1}\Tr(\Gr{G}{k}{l}{}{\mathscr{X}})
      \id_{\Gru{V}{m}{n}}, \\
      (\phi \otimes \id)(\Gr{X}{k}{l}{m}{n}\Gr{(X^{-1})}{m}{n}{k}{l})
      &=\beta^{-1}\Tr(\Gr{F}{m}{n}{}{\mathscr{X}})
      \id_{\Gru{V}{k}{l}}.
    \end{align*}
  \item Denote by $\Sigma_{klmn}$ the flip map $\Gru{V}{k}{l}
    \otimes \Gru{V}{m}{n} \to \Gru{V}{m}{n}
    \otimes \Gru{V}{k}{l}$. Then
 \begin{align*}
   (\phi \otimes \id \otimes
   \id)((\Gr{(X^{-1})}{m}{n}{k}{l})_{12}(\Gr{X}{k}{l}{m}{n})_{13}) &=
   \alpha^{-1}
   (\id_{\Gru{V}{m}{n}} \otimes \Gr{G}{k}{l}{}{\mathscr{X}})
   \circ \Sigma_{klmn}, \\
   (\phi \otimes \id \otimes
   \id)((\Gr{X}{k}{l}{m}{n})_{13}(\Gr{(X^{-1})}{m}{n}{k}{l})_{12}) &= \beta^{-1} (\Gr{F}{m}{n}{}{\mathscr{X}}
   \otimes \id_{\Gru{V}{k}{l}}) \circ \Sigma_{klmn}.
 \end{align*}
\end{enumerate}
  \end{Theorem}
\begin{proof}
  We prove the assertions and equations involving $\alpha$ in (1), (2)
  and (3)  simultaneously; the assertions involving $\beta$  follow similarly.

  %As above, we denote by $\Sigma_{p,q,r,s}$ the flip
  %$\Gru{\mathcal{H}}{p}{q} \otimes \Gru{\mathcal{H}}{r}{s} \to
  %\Gru{\mathcal{H}}{r}{s} \otimes \Gru{\mathcal{H}}{p}{q}$.  
  Consider
  the following endomorphism $F_{m,n,k,l}$ of $\Gru{V}{m}{n}\otimes \Gru{V}{k}{l}$, 
  \begin{align*}
    F_{m,n,k,l}
    &:=(\phi \otimes \id \otimes \id)(\Gr{(X^{-1})}{m}{n}{k}{l})_{12}(\Gr{X}{k}{l}{m}{n})_{13})
    \circ \Sigma_{mnkl} \\ &= (\phi \otimes \id \otimes
    \id)\left(\Gr{(X^{-1})}{m}{n}{k}{l})_{12}
      \Sigma_{klkl,23}(\Gr{X}{k}{l}{m}{n})_{12}\right).
  \end{align*}
  By applying Lemma \ref{lem:rep-average} with respect to the flip map $\Sigma_{klkl}$, we see that the family $(F_{m,n,k,l})_{m,n}$ is
  an endomorphism of $(V \otimes \Gru{V}{k}{l}, X_{12})$ and hence
  \begin{align}
    F_{m,n,k,l} &= \id_{\Gru{V}{m}{n}} \otimes \Gru{R}{k}{l} \label{eq:rep-orthogonal-1}
  \end{align}
  with some $\Gru{R}{k}{l} \in \Hom_{\C}(\Gru{V}{k}{l})$ not
  depending on $m,n$. % In using irreducibility, do we not miss any subtlety in allowing non-bounded morphisms?
  On the other hand, 
  \begin{align*}
    F_{m,n,k,l} &= (\phi \otimes \id \otimes \id)((S \otimes
    \id)(\Gr{X}{m}{n}{k}{l})_{12}(\Gr{X}{k}{l}{m}{n})_{13})
    \circ \Sigma_{mnkl} \\
    &= (\phi\circ S^{-1} \otimes \id \otimes \id)\left(((S \otimes
      \id)(\Gr{X}{k}{l}{m}{n}))_{13}
      ((S^{2} \otimes \id)(\Gr{X}{m}{n}{k}{l}))_{12}\right)     \circ \Sigma_{mnkl}\\
    &= (\phi\circ S^{-1} \otimes \id \otimes
    \id)\left((\Gr{(X^{-1})}{k}{l}{m}{n})_{13} (\Sigma_{mnmn})_{23}
      (\Gr{(\dual{\dual{X}{}\!})}{m}{n}{k}{l})_{13}\right).
  \end{align*}
  Since $\phi\circ S^{-1}$ is an invariant functional for
  $\mathscr{A}$, we can again apply Lemma \ref{lem:rep-average} and
  find that the family $(F_{m,n,k,l})_{k,l}$ is a morphism \[(F_{m,n,k,l})_{k,l}:
  (\Gru{V}{m}{n} \otimes V, (\dual{\dual{\mathscr{X}}{}\!})_{13})\rightarrow (\Gru{V}{m}{n} \otimes V,
 \mathscr{X}_{13}).\] Therefore,
  \begin{align}
    F_{m,n,k,l} &= \Gru{T}{m}{n} \otimes \Gr{G}{k}{l}{}{\mathscr{X}} \label{eq:rep-orthogonal-2}
  \end{align}
  with some $\Gru{T}{m}{n} \in \mathcal{\Hom_{\C}}(\Gru{V}{m}{n})$
  not depending on $k,l$. Combining \eqref{eq:rep-orthogonal-1} and
  \eqref{eq:rep-orthogonal-2}, we conclude that, for some $\lambda\in \C$, \[F_{m,n,k,l} = \lambda
  (\id_{\Gru{V}{m}{n}} \otimes \Gr{G}{k}{l}{}{\mathscr{X}})\]
  
  Choose dual  bases
  $(v_{i})_{i}$ for $\Gru{V}{k}{l}$ and $(f_{i})_{i}$ for  $\dual{(\Gru{V}{k}{l})}$. Then
  \begin{align*}
    \lambda   \Tr( \Gr{G}{k}{l}{}{\mathscr{X}}) \id_{\Gru{V}{m}{n}}
 &= \sum_{i} (\id \otimes
    \omega_{f_{i},v_{i}})(F_{m,n,k,l}) = (\phi \otimes
    \id)((\Gr{X}{k}{l}{m}{n})^{*} \Gr{X}{k}{l}{m}{n}).
  \end{align*}
  Taking $n=l$ and summing over $k$, the relations $\sum_{k}
  (\Gr{X}{k}{l}{m}{n})^{*} \Gr{X}{k}{l}{m}{n} = \UnitC{l}{n}
  \otimes \id_{\Gru{V}{m}{n}}$ and
  $\phi(\UnitC{l}{l})=1$ give
\begin{align*}
\lambda \cdot  \sum_{k} \Tr(\Gr{G}{k}{l}{}{\mathscr{X}}) = 1.
\end{align*}
Now all assertions in (1)--(3) concerning $\alpha$ follow.
 % the second formula, we use the first formula for the
 %    opposite of $(A,\Delta)$. For this opposite, $\phi$ still is a
 %    faithful, positive, normalized invariant functional and
 %    $(\mathcal{H},X)$ still is a unitary irreducible locally finite
 %    corepresentation, but the antipode $S$ gets replaced by $S^{-1}$
 %    and therefore $F_{X}$ gets replaced by $F_{X}^{-1}$.
\end{proof}
\begin{Cor}\label{CorOrth}
  Let $\mathscr{A}$ be a partial Hopf algebra with an integral $\phi$, let
  $(V,\mathscr{X})$ be an irreducible rcf corepresentation of
  $\mathscr{A}$, let $F_{\mathscr{X}}$ be an isomorphism from
  $(V,\mathscr{X})$ to $(V,\dual{\dual{\mathscr{X}}{}\!})$ and
  $G_{\mathscr{X}}=F^{-1}_{{\mathscr{X}}}$, and let $a=(\id \otimes
  \omega_{f,v})(\Gr{X}{k}{l}{m}{n})$ and $b=(\id \otimes
  \omega_{g,w})(\Gr{X}{m}{n}{k}{l})$, where 
  $f \in   \dual{(\Gru{V}{k}{l})}$, $v \in\Gru{V}{m}{n}$, $g \in
  \dual{(\Gru{V}{m}{n})}$, $w \in  \Gru{V}{k}{l}$.  Then
\begin{align*}
  \phi(S(b)a) &= \frac{(g|v)(v|G_{\mathscr{X}}w)}{\sum_{m}
    \Tr(\Gr{G}{m}{n}{}{\mathscr{X}})}, & \phi(aS(b)) = \frac{(g|F_{\mathscr{X}}v)(f|w)}{\sum_{n}
    \Tr(\Gr{F}{m}{n}{}{\mathscr{X}})}.
\end{align*}
\end{Cor}
\begin{proof}
Apply $\omega_{g,w} \otimes
    \omega_{f,v}$ to the formulas in  Theorem
    \ref{thm:rep-orthogonality} 3.
\end{proof}
\begin{Cor} \label{cor:rep-pw}
  Let $\mathscr{A}$ be a partial Hopf algebra with an integral and let
  $((V_{\alpha},\mathscr{X}_{\alpha}))_{\alpha}$ be a representative
  family of all irreducible rcf corepresentations of
  $\mathscr{A}$. Then the map
  \begin{align*}
    \bigoplus_{\alpha} \bigoplus_{k,l,m,n}
    (\dual{(\Gr{V}{k}{l}{}{\alpha})} \otimes
    \Gr{V}{m}{n}{}{\alpha}) \to A
  \end{align*}
  that sends $\dual{v} \otimes w \in
  \dual{(\Gr{V}{k}{l}{}{\alpha})} \otimes
  \Gr{V}{m}{n}{}{\alpha}$ to $ (\id \otimes
  \omega_{\dual{v},w})(\Gr{(X_{\alpha})}{k}{l}{m}{n})$,
  is a linear isomorphism. 
\end{Cor}
\begin{proof} This follows from Proposition \ref{prop:rep-weak-pw}, Proposition \ref{prop:rep-orthogonality-1} and Corollary \ref{CorOrth}.
\end{proof}
\begin{Cor} \label{cor:rep-pw-morphisms}
  Let $\mathscr{A}$ be a partial Hopf algebra with an integral, let
  $((V_{\alpha},\mathscr{X}_{\alpha}))_{\alpha}$ be a representative
  family of all irreducible rcf corepresentations of $\mathscr{A}$,
  fix $\alpha,k,l$ and denote by $\Gr{\mathscr{Y}}{k}{l}{}{\alpha}$
  the regular corepresentation on
  $\Gru{\mathcal{C}(\mathscr{X}_{\alpha})}{k}{l}$. Then there exists a
  linear isomorphism
  \begin{align*}
    \dual{( \Gru{V}{k}{l})} \to
    \Mor((V_{\alpha},\mathscr{X}_{\alpha}),
    (\Gru{\mathcal{C}(\mathscr{X}_{\alpha})}{k}{l},\Gr{\mathscr{Y}}{k}{l}{}{\alpha}))
  \end{align*}
  assigning to each $v\in     \dual{( \Gru{V}{k}{l})}$ the morphism
  $T_{(v)}$ of Lemma \ref{lemma:rep-regular-embedding}.
\end{Cor}

\subsection{Analogues of Woronowicz's  characters}

Suppose now $a\in \Gr{A}{k}{l}{m}{n}$ for some partial bialgebra $\mathscr{A}$. Then for $\omega \in \Hom_{\C}(A,\C)$, we can define
\begin{align*}
  \omega \aste{p,q} a
&:= (\id \otimes \omega) (\Delta_{pq}(a)), & a \aste{r,s}
\omega&:=(\omega \otimes \id)(\Delta_{rs}(a)).\end{align*} Clearly we can define
\begin{align*} \omega \aste{p,q} a \aste{r,s}
\omega'&:= (\omega \aste{p,q} a)\aste{r,s} \omega' = \omega \aste{p,q}(a \aste{r,s} \omega').\end{align*}
When $\omega$ has support on the $A(K)$ with $K_u=K_d$, we can write, for $a\in \Gr{A}{k}{l}{m}{n}$, \[\omega\ast a := \sum_{p,q} \omega\aste{p,q}a = \omega\aste{m,n}a,\quad  a\ast \omega = \sum_{r,s} a\aste{r,s}\omega = a\aste{k,l}\omega.\] 

We shall say that an entire function $f$ has \emph{exponential growth
  on the right half-plane} if there exist $C,d>0$ such that $|f(x+iy)|\leq
C\mathrm{e}^{dx}$  for all $x,y\in \R$ with $x>0$. 

\begin{Theorem} \label{thm:rep-characters} Let $\mathscr{A}$ be a
  partial Hopf algebra with an integral $\phi$.  Then there exists a unique
  family of linear functionals $f_{z} \colon A\to \C$ such that
\begin{enumerate}[label={(\arabic*)}]
  \item $f_z$ vanishes on $A(K)$ when $K_u\neq K_d$.
  \item for each $a\in A$, the function $z\mapsto f_{z}(a)$ is entire
    and of exponential growth on the right half-plane.
  \item $f_{0} = \epsilon$ and $(f_{z} \otimes f_{z'}) \circ 
    \Delta= f_{z+z'}$ for all $z,z' \in \C$.
  \item $\phi(ab)=\phi(b(f_{1} \ast a \ast f_{1}))$ for all $a,b\in A$.
  \end{enumerate}
  This family furthermore satisfies
  \begin{enumerate}[label={(\arabic*)}]\setcounter{enumi}{4}
  \item $f_z(ab) = f_z(a)f_z(b)$ for $a\in A(K)$ and $b\in A(L)$ with $K_r = L_l$. 
  \item $S^{2}(a)=f_{-1} \ast a \ast f_{1}$ for all $a\in A$.
  \item $f_{z}(\UnitC{l}{n})=\delta_{l,n}$ and $f_{z} \circ S = f_{-z}$ for all $a\in A$.
\end{enumerate}
\end{Theorem}


Note that condition (3) is meaningful by condition (1).

\begin{proof}
  We first prove uniqueness.  Assume that $(f_{z})_{z}$ is a family of
  functionals satisfying (1)--(4).  Since $\phi$ is faithful, the map
  $\sigma\colon a \mapsto f_{1} \ast a \ast f_{1}$ is uniquely
  determined by $\phi$, and one easily sees that it is a homomorphism. Using
  (3), we find that $\epsilon \circ \sigma^n=f_{2n}$, which uniquely determines these functionals. Using (2) and the
  fact that every entire function of exponential growth on the right
  half-plane is uniquely determined by its values at $\N \subseteq \C$, we can conclude that the family $f_{z}$ is uniquely determined. Moreover, since the property (5) holds for $z = 2n$, we also conclude by the same argument as above that it holds for all $z\in \C$.

  Let us now prove existence.  By Corollary \ref{cor:rep-pw}, we can
  define for each $z\in \C$ a functional $f_{z} \colon A \to \C$ such
  that for every irreducible rcf corepresentation
  $(V,\mathscr{X})$,
    \begin{align*}
      f_{z}((\id \otimes \omega_{\xi,\eta})(\Gr{X}{k}{l}{m}{n})) &=
      \delta_{k,m}\delta_{l,n} \cdot
      \omega_{\xi,\eta}((\Gr{F}{k}{l}{}{\mathscr{X}})^{z}) \quad \text{for all }
      \xi \in \Gru{V}{k}{l},\eta \in
      \Gru{V}{m}{n},
    \end{align*}
    or, equivalently,
    \begin{align*}
      (f_{z} \otimes \id)(\Gr{X}{k}{l}{m}{n}) =
      \delta_{k,m}\delta_{l,n} \cdot (\Gr{F}{k}{l}{}{\mathscr{X}})^{z},
    \end{align*}
    where $F_{\mathscr{X}}$ is a non-zero positive operator implementing a morphism from $(V,\mathscr{X})$ to
    $(V, \dual{\dual{\mathscr{X}}\!{}})$, scaled such that
    \begin{align*}
      \alpha_{X}:= \sum_{k} \Tr(\Gru{(F_{X}^{-1})}{k}{l}) = \sum_{n}
      \Tr(\Gru{F_{X}}{m}{n})
    \end{align*}
    for all $l,n$ (see Proposition \ref{prop:rep-f} and Theorem \ref{thm:rep-orthogonality}). By
    construction, (1) and (2) hold. We show that the $(f_{z})_{z}$ satisfy the
    assertions (3)--(7). 
    %We have already argued that (5) is satisfied
    %$f_{z}$ is a character. 
    Throughout the following arguments, let 
    $(V,\mathscr{X})$ be a unitary irreducible corepresentation
    $(V,\mathscr{X})$ and let $F_{\mathscr{X}}$ be as above.

    We first prove property (3). This follows from the relations
    \begin{align*}
      (f_{0}  \otimes \id)(\Gr{X}{k}{l}{m}{n}) &=
      \delta_{k,m}\delta_{l,n} \id_{\Gru{V}{k}{l}} =
      (\epsilon \otimes \id)(\Gr{X}{k}{l}{m}{n})
    \end{align*}
    and
    \begin{align*}
      (((f_{z}\otimes f_{z'})\circ \Delta) \otimes
      \id)(\Gr{X}{k}{l}{m}{n}) &=  \delta_{k,m}\delta_{l,n}(f_{z} \otimes f_{z'} \otimes
      \id)\big((\Gr{X}{k}{l}{k}{l})_{13}
      (\Gr{X}{k}{l}{k}{l})_{23}\big) \\
      &=  \delta_{k,m}\delta_{l,n}(\Gru{F_{\mathscr{X}}}{k}{l})^{z}  \cdot (\Gru{F_{\mathscr{X}}}{k}{l})^{z'} \\
      &= (f_{z+z'} \otimes \id)(\Gr{X}{k}{l}{m}{n}).
    \end{align*}
    Applying slice maps of the form $\id
    \otimes \omega_{\xi,\xi'}$ and invoking Theorem \ref{thm:rep-orthogonality}, this proves (3).

% Again? Check if this has already been used before   
    To prove (4), write again $ \Delta^{(2)} = (
    \Delta \otimes \id)\circ  \Delta = (\id \otimes 
    \Delta) \circ \Delta$, and put \[\theta_{z,z'}:=(f_{z'} \otimes \id
    \otimes f_{z})\circ  \Delta^{(2)}.\] Then
    \begin{align*}
      (\theta_{z,z'} \otimes \id)(\Gr{X}{k}{l}{m}{n}) &= (f_{z'} \otimes
      \id \otimes f_{z} \otimes
      \id)((\Gr{X}{k}{l}{k}{l})_{14}(\Gr{X}{k}{l}{m}{n})_{24}(\Gr{X}{m}{n}{m}{n})_{34})
      \\
      &= (1 \otimes (\Gru{F_{\mathscr{X}}}{k}{l})^{z'}) \Gr{X}{k}{l}{m}{n} (1
      \otimes (\Gru{F_{\mathscr{X}}}{m}{n})^{z}).
    \end{align*}
    We take $z=z'=1$, use Theorem \ref{thm:rep-orthogonality}, where
    now $\alpha= \beta$ by our scaling of $F_{\mathscr{X}}$, and obtain
    \begin{eqnarray*}
     && \hspace{-2cm} (\phi \otimes \id \otimes
      \id)((\Gr{X}{k}{l}{m}{n})_{12}^{*}((\theta_{1,1} \otimes
      \id)(\Gr{X}{k}{l}{m}{n}))_{13})\\ && =\alpha^{-1}(\id \otimes
      \Gru{F_{\mathscr{X}}}{k}{l}) (\id \otimes \Gru{(F_{\mathscr{X}}^{-1})}{k}{l})
      \Sigma_{k,l,m,n} (\id \otimes
      \Gru{F_{\mathscr{X}}}{m}{n}) \\
      &&=\beta^{-1}(\Gru{F_{\mathscr{X}}}{m}{n} \otimes \id) \Sigma_{k,l,m,n} \\
      &&= (\phi \otimes \id \otimes
      \id)((\Gr{X}{k}{l}{m}{n})_{13}(\Gr{X}{k}{l}{m}{n})_{12}^{*}).
    \end{eqnarray*}
    To conclude the proof of assertion (4), apply again slice maps of the form
    $\omega_{\xi,\xi'} \otimes \omega_{\eta,\eta'}$.

We have then already argued that the property (5) automatically holds. To show the property (6), note that by Proposition \ref{prop:rep-f} and the calculation above,
    \begin{align*}
      (S^{2} \otimes \id)(\Gr{X}{k}{l}{m}{n}) &= (1
      \otimes\Gru{F_{\mathscr{X}}}{k}{l})
      \Gr{X}{k}{l}{m}{n}(1 \otimes \Gru{F_{\mathscr{X}}}{m}{n})^{-1} 
      =(\theta_{-1,1}  \otimes \id)(\Gr{X}{k}{l}{m}{n}).
    \end{align*}
     Assertion (6) follows again by applying slice maps.
    
     Finally, (1), (2) and (4)
     immediately imply the relation
     $f_{z}(\UnitC{k}{m})=\delta_{k,m}$. The concrete construction of $f_z$ combined with property (3), the identity \eqref{eq:rep-delta2} and the partial character property (5) gives the equality
     \begin{align}
       (f_{-z} \otimes \id) (\Gr{X}{k}{l}{k}{l})=
       (\Gru{(F_{\mathscr{X}})}{k}{l})^{-z} &=\left( (f_{z} \otimes
       \id)(\Gr{X}{k}{l}{k}{l})\right)^{-1} \\ &= (f_{z} \otimes
       \id)(\Gr{(X^{-1})}{l}{k}{l}{k}) = ((f_{z} \circ S) \otimes
       \id)(\Gr{X}{k}{l}{k}{l}).
     \end{align}
Therefore, $f_{-z} = f_{z} \circ S$.
\end{proof}
\begin{Cor} \label{cor:rep-characters} Let $\mathscr{A}$ be a partial
  Hopf algebra with integral $\phi$ and define $\theta_{z,z'} \colon A
  \to A$ by $a \mapsto f_{z} \ast a \ast f_{z'}$ for each $z,z' \in
  \C$, where the functionals $f_{z}$ are as in Theorem
  \ref{thm:rep-characters}. Then for all $z,z',w,w'\in \C$, the
  following conditions hold:
  \begin{enumerate}
  \item $\theta_{z,z'}$ is an algebra automorphism and preserves
    each subspace $A(K)$; in particular,
    $\theta_{z,z'}(\lambda_{k}\rho_{m}) = \lambda_{k}\rho_{m}$ for all
    $k,m\in I$;
  \item $\theta_{z,z'}\circ \theta_{w,w'} = \theta_{z+w,z'+w'}$;
  \item $ (\theta_{w,z'} \otimes \theta_{z,-w}) \circ \Delta = \Delta
    \circ \theta_{z,z'}$, $\epsilon \circ \theta_{z,z'} = f_{z+z'}$,
    $\theta_{z,z'} \circ S = S \circ \theta_{-z',-z}$ and
    $\phi \circ \theta_{z,z'} = \phi$;
  \item for every linear map $\omega \colon A \to \C$ and every $a\in
    A$, the map $(z,z') \mapsto \omega(\theta_{z,z'}(a))$ is entire.
  \end{enumerate}
\end{Cor}
\begin{proof}
  All of this follows easily from Theorem \ref{thm:rep-characters}.
\end{proof}
Using the two-parameter group $\theta$, we define the \emph{modular
  automorphism group} $\sigma$, the \emph{scaling group} $\tau$   and
the \emph{unitary antipode} of a partial compact quantum group $A$ by
\begin{align} \label{eq:rep-groups}
  \sigma_{z} &:=\theta_{iz,iz}, & \tau_{z} &:=\theta_{iz,-iz}, & R&:=S
  \circ \tau_{i/2}.
\end{align}
Using Corollary \ref{cor:rep-characters}, one verifies that
$\sigma,\tau,R$ share all the main relations known for locally compact
quantum groups and measured quantum groupoids, for example, $\sigma$
and $\tau$ are complex one-parameter groups of algebra automorphisms
of $A$, the map $R$ is an anti-automorphism,  $\tau_{t}$ and
$\sigma_{t}$ are automorphisms for all $t\in \R$, the family  $\tau$ commutes with
$\sigma$ and with $R$ in the obvious sense, 
  \begin{gather}
    \begin{aligned} \label{eq:modular}
      \phi\circ \sigma_{z} &= \phi \circ \tau_{z} = \phi \circ R =
      \phi, & \phi(ab) &= \phi(b\sigma_{-i}(a)),
    \end{aligned}
\\ \label{eq:scaling-modular-delta}
    \begin{aligned} 
    \Delta \circ \tau_{z} &= (\tau_{z} \otimes \tau_{z}) \circ \Delta,
    & (\tau_{z} \otimes \sigma_{z}) \circ \Delta &= \Delta \circ
    \sigma_{z} = (\sigma_{-z} \otimes \tau_{z}) \circ \Delta,      
  \end{aligned} \\
  \begin{aligned} \label{eq:unitary-antipode}
    R^{2} &= \id_{A}, & \Delta \circ R &= (R \otimes R) \circ
    \Delta^{\op}.
  \end{aligned}
  \end{gather}



\subsection{Unitary corepresentations of partial compact quantum groups}


Let us now enhance our partial Hopf algebras to partial compact
quantum groups. One then considers corepresentations on sfd bigraded
\emph{Hilbert spaces} such that the inverse of the corepresentation
coincides with its adjoint. More precisely, we have the following
definition. We write $B(\Hsp,\mathcal{G})$ for the linear space of
bounded morphisms between Hilbert spaces $\Hsp$ and $\mathcal{G}$.

Unitary corepresentations act on $I^{2}$-graded Hilbert spaces $\Hsp =
\bigoplus_{k,l} \Gru{\Hsp}{k}{l}$ which are row- and column-finite,
where the sum is a direct sum of Hilbert spaces. Associated to each
such $I^{2}$-graded Hilbert space is the $I^{2}$-graded vector space
which is obtained by taking the algebraic direct sum of the
components. 

\begin{Def} Let $\mathscr{A}$ define a partial compact quantum
  group. We call an rcf corepresentation $\mathscr{X}$ on an rcf
  $I^2$-graded Hilbert space $\mathcal{H}$ \emph{unitary}
  if \[\Gr{(X^{-1})}{n}{m}{l}{k}=(\Gr{X}{l}{k}{n}{m})^{*}\quad
  \textrm{in }\Gr{A}{k}{l}{m}{n}\otimes
  B(\Gru{\Hsp}{l}{k},\Gru{\Hsp}{n}{m}).\]
\end{Def} 
\begin{Rem} \begin{enumerate} \item In the Hilbert space setting, it is more natural to let $\Hsp$ be the \emph{closed} (instead of the purely algebraic) direct sum of all (finite-dimensional) $\Gru{\Hsp}{k}{l}$. This does not change the notion of corepresentation, which had a local definition.
\item Concerning morphisms, we will say a collection of $\Gru{T}{k}{l}$ defines a \emph{bounded} intertwiner or morphism if the total operator $T= \oplus \Gru{T}{k}{l}$ is bounded. We will denote by $\Corep_{u}(\mathscr{A})$ the category of unitary rcf corepresentations with arbitrary morphisms, and $\Corep_{u}^{\infty}(\mathscr{A})$ for the category with bounded morphisms.
% Give a more prominent place, or check later if it is actually worthwhile to make this distinction.
\end{enumerate}
\end{Rem}

\begin{Exa}\label{example:rep-trivial-unitary}
  Regard $\C$ as a Hilbert space in the canonical way. Then the
  trivial corepresentation $\mathscr{U}$ on $l^{2}(I)$ is unitary.
\end{Exa}
Our aim now is to show that every irreducible rcf corepresentation is
equivalent to a unitary one. We show this by embedding the
corepresentation into a restriction of the regular corepresentation.
\begin{Lem} \label{lemma:rep-regular-unitary}
  Let $\mathscr{A}$ define a partial compact quantum group with
positive  integral $\phi$ and let $\Gru{V}{m}{n} \subseteq
\bigoplus_{k,l} \Gr{A}{k}{l}{m}{n}$ be subspaces such that
$\Delta_{pq}(\Gru{V}{m}{n}) \subseteq \Gru{V}{p}{q} \otimes
    \Gr{A}{p}{q}{m}{n}$ and $V=\bigoplus_{k,l} \Gru{V}{k}{l}$ is row-
    and column-finite. Then each $\Gru{V}{k}{l}$ is a Hilbert space with
    respect to the inner product given by $\langle
    a|b\rangle:=\phi(a^{*}b)$, and the regular corepresentation
    $\mathscr{X}$ on $V$ is unitary.
\end{Lem}
\begin{proof} 
By Lemma \ref{lemma:rep-invertible},  it suffices to show that
  \begin{equation}\label{EqUnit} \sum_{k}
    (\Gr{X}{k}{l}{m}{n'})^* \Gr{X}{k}{l}{m}{n} =
    \delta_{n,n'}\UnitC{l}{n}\otimes
    \id_{\Gru{\Hsp}{m}{n}}.
  \end{equation} 
Let  $a\in \Gru{\Hsp}{m}{n}$, $b\in \Gru{\Hsp}{m}{n'}$ and define $\omega_{b,a} \colon
\Hom_{\C}(\Gru{\Hsp}{m}{n},\Gru{\Hsp}{m}{n}) \to \C$ by $T
\mapsto \langle b|Ta\rangle$. Then
\begin{eqnarray*}
\sum_{k }(\id \otimes \omega_{b,a})
((\Gr{X}{k}{l}{m}{n'})^* \Gr{X}{k}{l}{m}{n}))  &=& \sum_k
(\id\otimes \phi)(\Delta_{kl}^{\op}(b)^*\Delta_{kl}^{\op}(a))\\
  &=& \sum_k (\phi\otimes
  \id)(\Delta_{lk}(b^*)\Delta_{kl}(a)) \\ &=& (\phi\otimes
  \id)(\Delta_{ll}(b^*a)) \\ &=& \phi(b^*a)\UnitC{l}{n} \\&=&
  \delta_{n',n} \UnitC{l}{n} \otimes \langle b|a\rangle.
\end{eqnarray*} 
This proves \eqref{EqUnit}.
\end{proof} 

\begin{Prop} \label{prop:rep-unitarisable} Every  rcf
  corepresentation of a partial compact quantum group $\mathscr{A}$ is
  isomorphic to a unitary rcf corepresentation.
\end{Prop}
\begin{proof}
  By Proposition \ref{prop:rep-cosemisimple}, it suffices to prove the
  assertion for every sfd corepresentation $(V,\mathscr{X})$ that is
  irreducible.  For some $k,l$ and $\dual{v} \in
  \dual{(\Gru{V}{k}{l})}$, the operator $T_{(\dual{v})}$ defined in
  Lemma \ref{lemma:rep-regular-embedding} has to be non-zero and
  hence, by Schur's Lemma, injective. Thus, it forms an equivalence
  between $(V,\mathscr{X})$ and a restriction of the regular
  corepresentation on $\Gru{\mathcal{C}(\mathscr{X})}{k}{l}$, which is
  unitary by Lemma \ref{lemma:rep-regular-unitary}.
\end{proof}
This result and Proposition \ref{prop:rep-cosemisimple}
imply that the category $\Corep_{u}(\mathscr{A})$ is semisimple:
\begin{Cor}
  Every unitary rcf corepresentation of a partial compact quantum
  group decomposes into a direct sum of irreducible unitary rcf
  corepresentations.
\end{Cor}
 
The tensor product of rcf corepresentations lifts to a tensor product
of unitary corepresentations as follows.  We define the tensor product
of rcf $I^{2}$-graded Hilbert spaces similarly as for rcf
$I^{2}$-graded vector spaces and pretend it to be strict again.
\begin{Lem}\label{lemma:rep-unitary-tensor}
  Let $(\Hsp,\mathscr{X})$ and $(\mathcal{G},\mathscr{Y})$ be unitary
  rcf corepresentations of a partial compact quantum group. Then the
  tensor product $(\Hsp \itimes \mathcal{G},\mathscr{X} \Circt
  \mathscr{Y})$ is unitary again.
\end{Lem}
\begin{proof}
In total form,  $(X\Circt Y)^{-1} = Y_{13}^{-1}X_{12}^{-1}
  =Y_{13}^{*}X_{12}^{*} = (X \Circt Y)^{*}$ by Remark \ref{remark:rep-tensor-multiplier}.
\end{proof}
With the evident definition on morphisms, we obtain a tensor product
on $\Corep_{u}(\mathscr{A})$.  This tensor
category is \emph{rigid} in the sense that every object has a
left and a right dual:
\begin{Cor}
  Let $\mathscr{A}$ define a partial compact quantum group. Then the
  category $\Corep_{u}(\mathscr{A})$ is a rigid tensor category.
\end{Cor}
\begin{proof}
  Since the antipode of $\mathscr{A}$ is invertible (Corollary
  \ref{cor:involutive}), every unitary rcf corepresentation
  $(\Hsp,\mathscr{X})$ has a left and a right dual in
  $\Corep(\mathscr{A})$ (Corollary \ref{cor:rep-tensor-duality}),
  and these are equivalent isomorphic to unitary rcf corepresentations
  (Proposition \ref{prop:rep-unitarisable}).
\end{proof}
Note that in $\Corep(\mathscr{A})$, the right dual of a unitary rcf
corepresentation $(\Hsp,\mathscr{X})$ is given by the $I^{2}$-graded
vector space
$\dual{\Hsp}$  and the corepresentation multiplier
\begin{align} \label{eq:rep-unitary-right-dual}
  (S\otimes -^{\vee})(X) = (\id \otimes -^{\vee})(X^{-1}) )=
  (-^{*}\otimes -^{*\vee})(X).
\end{align}
By Corollary \ref{cor:rep-irreducible-bidual},
$(\mathcal{H},\mathscr{X})$ is isomorphic to the right bidual, which
is given by the $\Hsp$ and $(S^{2} \otimes \id)(X)$. This isomorphism can be
chosen to be positive:
\begin{Prop} \label{prop:rep-unitary-bidual}
  Let $\mathscr{A}$ define a partial compact quantum group and let
  $(\Hsp,\mathscr{X})$ be an irreducible unitary rcf corepresentation of
  $\mathscr{A}$.  Then there exists an isomorphism $F_{\mathscr{X}}$
  from $(\Hsp,\mathscr{X})$ to 
  $(\Hsp,(S^{2} \otimes \id)(\mathscr{X}))$ in $\Corep(\mathscr{A})$ such
  that each $\Gr{F}{k}{l}{}{\mathscr{X}}$ is positive.
\end{Prop}
\begin{proof}
  By Proposition \ref{prop:rep-unitarisable}, there exists an
  isomorphism $T \colon \dual{\mathscr{X}} \to \mathscr{Y}$ for some
  unitary rcf corepresentation $\mathscr{Y}$, so that in total form,
  $(1\otimes T)\dual{X} = Y(1 \otimes T)$.
We  apply   $S \otimes -^{\vee}$ and $-^{*} \otimes -^{*\vee}$,
respectively, use \eqref{eq:rep-unitary-right-dual}, and  find 
\begin{align*}
  \dual{\dual{X}\!{}}(1 \otimes \dual{T}) &= (1 \otimes
  \dual{T})\dual{Y}, & (1 \otimes T^{*\vee})X=\dual{Y}(1\otimes T^{*\vee}).
\end{align*}
Here, we identify the the dual of a Hilbert space with its conjugate
Hilbert space to make sense of $T^{*\vee}$.  Combining both equations, we
find $\dual{\dual{X}\!{}}(1 \otimes \dual{T}T^{*\vee}))=(1 \otimes
\dual{T}T^{*\vee})X$. Thus, we can take
$F_{\mathscr{X}}:=\dual{T}T^{*\vee}$.
\end{proof}

The Schur orthogonality relations in Corollary \ref{CorOrth} can be
rewritten using the involution instead of the antipode as follows.
Let $(\Hsp,\mathscr{X})$ be a unitary rcf corepresentation of
$\mathscr{A}$. Since $(S\otimes \id)(X)=X^{-1}=X^{*}$, the space of
matrix coefficients $\mathcal{C}(\mathscr{X})$ satisfies
\begin{align} \label{eq:rep-unitary-matrix-coefficients}
  S(\Gr{\mathcal{C}(\mathscr{X})}{k}{l}{m}{n}) &=
  (\Gr{\mathcal{C}(\mathscr{X})}{m}{n}{k}{l})^{*} \subseteq \Gr{A}{n}{m}{l}{k}.
\end{align}
More precisely, let $v \in \Gru{\Hsp}{k}{l}$, $v' \in \Gru{\Hsp}{m}{n}$
and denote by $\omega_{v,v'}$ and $\omega_{v',v}$ the functionals
given by $T \mapsto \langle v|Tv'\rangle$ and $T\mapsto \langle
v'|Tv\rangle$, respectively. Then
\begin{align*}
  S((\id \otimes \omega_{v,v'})(\Gr{X}{k}{l}{m}{n})) &=
  (\id \otimes \omega_{v,v'}) (\Gr{(X^{-1})}{k}{l}{m}{n})) \\ & =
  (\id \otimes \omega_{v,v'})( (\Gr{X}{m}{n}{k}{l})^{*}) =
  (\id \otimes \omega_{v',v})(\Gr{X}{m}{n}{k}{l})^{*}.
\end{align*}
This equation and Corollary \ref{CorOrth} imply:
\begin{Cor}\label{cor:rep-unitary-schur-orthogonality}
  Let $\mathscr{A}$ define a partial compact quantum group with
  invertible integral $\phi$, let $(\Hsp,\mathscr{X})$ be an irreducible
  rcf corepresentation of $\mathscr{A}$, let $F_{\mathscr{X}}$ be an
  isomorphism from $(\Hsp,\mathscr{X})$ to
  $(\Hsp,\dual{\dual{\mathscr{X}}{}\!})$ and
  $G_{\mathscr{X}}=F^{-1}_{{\mathscr{X}}}$, and let $a=(\id \otimes
  \omega_{v,v'})(\Gr{X}{k}{l}{m}{n})$ and $b=(\id \otimes
  \omega_{w,w'})(\Gr{X}{k}{l}{m}{n})$, where $v,w \in
  \Gru{\Hsp}{k}{l}$ and $v',w' \in \Gru{\Hsp}{m}{n}$.  Then
\begin{align*}
  \phi(b^{*}a) &= \frac{\langle w|v'\rangle\langle v|G_{\mathscr{X}}w'\rangle}{\sum_{m}
    \Tr(\Gr{G}{m}{n}{}{\mathscr{X}})}, & \phi(ab^{*}) = \frac{\langle
    w|F_{\mathscr{X}}v'\rangle \langle v|w'\rangle}{\sum_{n}
    \Tr(\Gr{F}{m}{n}{}{\mathscr{X}})}.
\end{align*}
\end{Cor}
As a consequence of Proposition \ref{prop:rep-weak-pw} and Proposition
\ref{prop:rep-unitarisable} or Lemma \ref{lemma:rep-regular-unitary},
the matrix coefficients of irreducible unitary rcf corepresentations
span $\mathscr{A}$, and in the Corollary \ref{cor:rep-pw}, we may
assume the irreducible rcf corepresentations
$(V_{\alpha},\mathscr{X}_{\alpha})$ to be unitary if $\mathscr{A}$
defines a partial compact quantum group.


Finally, we consider the functionals $f_{z}$, the automorphisms
$\theta_{z,z'}$, $\tau_{z}$, $\sigma_{z}$ and the anti-isomorphism $R$
introduced in Theorem \ref{thm:rep-characters}, Corollary
\ref{cor:rep-characters} and \eqref{eq:rep-groups}.
\begin{Prop} \label{prop:rep-unitary-characters} Let $\mathscr{A}$
  define a partial compact quantum group.  Then $f_{z} \circ \ast =
  \ast \circ f_{-\overline{z}}$ and $\theta_{z,z'} \circ * = * \circ
  \theta_{-\overline{z},-\overline{z'}}$ for all $z\in \C$. In
  particular, $R$ is a $*$-anti-automorphism and $\theta_{it,is}$,
  $\tau_{t}$ and $\sigma_{t}$ are $*$-automorphisms for all $s,t\in
  \R$,
\end{Prop}
\begin{proof}
  We only have to prove the first equation.  Write $\bar{f}_z(a) =
  \overline{f_z(a^*)}$. Using the relations
  $ (\Gr{X}{k}{l}{k}{l})^{*}=(S \otimes \id)(\Gr{X}{k}{l}{k}{l})$,  $f_{z} \circ S=f_{-z}$
  (Theorem \ref{thm:rep-characters}) and
  positivity of $\Gr{F}{k}{l}{}{\mathscr{X}}$ (Proposition \ref{prop:rep-unitary-bidual}), we conclude
     \begin{align*}
       (\bar{f}_z \otimes
       \id)(\Gr{X}{k}{l}{k}{l})
&=       \left((f_{z} \otimes
       \id)((\Gr{X}{k}{l}{k}{l})^{*})\right)^{*} \\
& = \left((f_{-z} \otimes \id)(\Gr{X}{k}{l}{k}{l})\right)^{*} 
 =
((\Gr{F}{k}{l}{}{\mathscr{X}})^{-z})^{*} 
=       (\Gr{F}{k}{l}{}{\mathscr{X}})^{-\overline{z}} = (f_{-\overline{z}}
\otimes \id)(\Gr{X}{k}{l}{k}{l}),
     \end{align*}
whence $\bar{f}_z(a) = f_{-\overline{z}}(a)$ for all $a\in
\Gr{\mathcal{C}(\mathscr{X})}{k}{l}{k}{l}$. Since $f_{z}$ and
$f_{-\overline{z}}$ vanish on $\Gr{A}{k}{l}{m}{n}$ if $(k,l)\neq
(m,n)$ and the matrix coefficients of unitary 
corepresentations span $A$, we can conclude $\bar{f}_{z}=f_{-\overline{z}}$.
\end{proof}



%  consider
%   the family
%   \begin{align*}
%     (\Gru{F}{m}{n})^{\top} \circ  \overline{\Sigma_{m,n,k,l}} &=
% \phi((\Gr{X}{k}{l}{m}{n})_{12}^{*}(\Gr{X}{k}{l}{m}{n})_{13})^{\top}
% \circ  \overline{\Sigma_{m,n,k,l}} \\
%  &=
%  \phi((\Gr{X}{k}{l}{m}{n})_{12}^{*\circ (\id \otimes
%   \top)}(\overline{\Sigma_{k,l,k,l}})_{23} (\Gr{X}{k}{l}{m}{n})^{\id \otimes
%   \top}_{12})  \\
% &=\phi((\Gr{\overline{X}}{l}{k}{n}{m})_{12}(\overline{\Sigma_{k,l,k,l}})_{23}
%  (\Gr{\overline{X}}{l}{k}{n}{m})_{12}).
% \end{align*} \fxnote{Treat $X^{\id \otimes \top}$}
% By Lemma \ref{lem:rep-average}, this family is a morphism from to
% $(\overline{\mathcal{H}}\otimes \overline{K},\overline{X}^{-*} \otimes
% \id_{\overline{K}})$  to
% $(\overline{\mathcal{H}}\otimes \overline{K},\overline{X} \otimes
% \id_{\overline{K}})$ and hence of the form
% $(\Gru{\overline{F_{X}}}{n}{m})^{-1} \otimes T$ with $T \in
% \mathcal{B}(\overline{K})$ not depending on $m,n$.

% Thus,
% \begin{align*}
%  (\id \otimes R)  \circ \Sigma_{k,l,m,n} =   \Gru{F}{m}{n} =
%  \Sigma_{k,l,m,n} \circ (\Gru{\overline{F_{X}}}{n}{m})^{-\top} \otimes T^{\top})
% \end{align*}

  
%   We may assume $(k,l,m,n)=(p,q,r,s)$ because otherwise both sides of
%   the equation that we want to prove vanish.

%   Applying Lemma \ref{lem:rep-average} to the corepresentation $X$ and
%   the family $\Gru{T}{p}{q}=
%   \delta_{p,k}\delta_{q,l} |\eta\rangle\langle\xi|$, we obtain an
%   endomorphism $\check{T}$ of $(\mathcal{H},X)$ which
%   necessarily has the form $\check{T}=\lambda(\xi,\eta) \id$ for some
%   $\lambda(\xi,\eta) \in \C$. Inserting the definition of
%   $\check{T}$, we find
%   \begin{align}\nonumber
%     \phi(b^{*}a) &= \phi\big((\id \otimes
%     \omega_{\eta',\eta})((\Gr{X}{k}{l}{m}{n})^{*}) \cdot (\id \otimes
%     \omega_{\xi,\xi'})(\Gr{X}{k}{l}{m}{n})\big) \\  &= (\phi \otimes
%     \id)\left(\langle\eta'|_{2} \Gr{(X^{-1})}{m}{n}{k}{l}(1 \otimes
%       |\eta\rangle\langle \xi|)
%       \Gr{X}{k}{l}{m}{n}|\xi'\rangle_{2}\right) 
%     = \langle \eta'|_{2} \Gru{\check{T}}{m}{n}|\xi'\rangle_{2} =
%     \lambda(\xi,\eta) \langle\eta'|\xi'\rangle. \label{eq:rep-orthogonal-1}
%   \end{align}
%   Next, we apply Lemma \ref{lem:rep-average} to the corepresentations
%   $\overline{X}$ and $\overline{X}^{-*}$ and the family
%   $\Gru{R}{p}{q}=\delta_{p,m}\delta_{q,n}|\overline{
%     \xi'}\rangle\langle\overline{\eta'}|$, and obtain a morphism
%   $\hat{R}$ from $\overline{X}^{-*}$ to $\overline{X}$ which
%   necessarily has the form $\hat{R}=\mu(\eta',\xi')\overline{F_{X}}$
%   for some $\mu(\eta',\xi') \in \C$. Using the relation
%   \begin{align*}
%     a &= (\id \otimes
%     \omega_{\overline{\xi},\overline{\xi'}})(\Gr{\overline{X}}{l}{k}{n}{m})^{*},
%     & b&= (\id \otimes
%     \omega_{\overline{\eta},\overline{\eta'}})(\Gr{\overline{X}}{k}{l}{m}{n})^{*}
%   \end{align*}
%   and the definition of $\hat{R}$, we obtain
%   \begin{align}
%     \phi(b^{*}a) &= (\phi \otimes \id)\left(\langle
%       \overline{\eta}|_{2} \Gr{\overline{X}}{k}{l}{m}{n}(1 \otimes
%       |\overline{\eta}'\rangle\langle \overline{\xi'}|)
%       \Gr{(\overline{X}^{*})}{m}{n}{k}{l} \right) \nonumber \\
%     &=\langle \overline{\eta}| \Gru{\hat{R}}{k}{l}
%     |\overline{\xi}\rangle = \langle
%     \overline{\eta}|\Gru{\overline{F_{X}}}{k}{l}\overline{\xi}\rangle
%     \mu(\eta',\xi'). \label{eq:rep-orthogonal-2}
%   \end{align}
%   We choose a basis $(\zeta_{i})_{i}$ for
%   $\bigoplus_{k}\Gru{\mathcal{H}}{k}{l}$ and calculate
%   \begin{align*}
%  \langle
%     \eta'|\xi'\rangle &=  (\phi \otimes
%     \id)(\langle\eta'|_{2}(\lambda_{l}\rho_{n} \otimes
%     \id_{\Gru{\mathcal{H}}{m}{n}})|\xi'\rangle_{2}) \\
%  &=
%     \sum_{k} (\phi \otimes \id)\left(\langle\eta'|_{2}
%       \Gr{(X^{-1})}{m}{n}{k}{l}
%       \Gr{X}{k}{l}{m}{n}|\xi'\rangle_{2}\right)
%     \\ &=    \sum_{i} \lambda(\zeta_{i},\zeta_{i}) \langle
%     \eta'|\xi'\rangle 
%     \\
%     &=\sum_{i} \langle
%     \overline{\zeta_{i}}| \Gru{\overline{F_{X}}}{}{l}\overline{\zeta_{i}}\rangle
%     \mu(\eta',\xi') 
% \\ &    = \mu(\eta',\xi') \cdot \sum_{k} \Tr(\Gru{F_{(X)}}{k}{l}),
%   \end{align*}
% where $\Gru{\overline{F_{X}}}{}{l}=\bigoplus_{k}
% \Gru{\overline{F_{X}}}{k}{l}$. Inserting this relation into
% \eqref{eq:rep-orthogonal-2}, we finally obtain the assertion.


% We use here the standard leg numbering notation, e.g. $a_{12} =
% a\otimes 1$.


%%% Local Variables: 
%%% mode: latex
%%% TeX-master: "dyn-suq-main"
%%% End: 
