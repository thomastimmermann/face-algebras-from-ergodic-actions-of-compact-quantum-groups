\section{Representation theory}

\subsection{Corepresentations of generalized compact Hopf face
  algebras}

Let $(A,\Delta)$ be a generalized compact Hopf face algebra over an
index set $I$. % A \emph{locally finite corepresentation}
% of $(A,\Delta)$ consists of a row- and column-finite $I^{2}$-graded
% Hilbert space $\mathcal{H}=\bigoplus_{k,l} \Gru{\mathcal{H}}{k}{l}$
% together with a family $X=(\Gr{X}{k}{l}{m}{n})_{k,l,m,n}$ of elements $\Gr{X}{k}{l}{m}{n} \in \Gr{A}{k}{l}{m}{n}
% \otimes
% \mathcal{B}(\Gru{\mathcal{H}}{m}{n},\Gru{\mathcal{H}}{k}{l})$ such that
% \begin{enumerate}
% \item $(\tilde \Delta \otimes \id)(\Gr{X}{k}{l}{m}{n}) = \sum_{p,q}
%   \left(\Gr{X}{k}{l}{p}{q}\right)_{13}\left(\Gr{X}{p}{q}{m}{n}\right)_{23}$ and
% \item $X$ is invertible in the sense that there exist elements $\Gr{Z}{k}{l}{m}{n} \in \Gr{A}{n}{m}{l}{k}
%   \otimes
%   \mathcal{B}(\Gru{\mathcal{H}}{m}{n},\Gru{\mathcal{H}}{k}{l})$ such
%   that
%   \begin{align*}
%     \sum_{k} \Gr{Z}{m}{n'}{k}{l} \Gr{X}{k}{l}{m}{n} &= \delta_{n,n'}
%     \lambda_{l} \rho_{n} \otimes \id_{\Gru{\mathcal{H}}{m}{n}}, &
%     \sum_{n} \Gr{X}{k}{l}{m}{n} \Gr{Z}{m}{n}{k'}{l} &= \delta_{k,k'}
%     \lambda_{k} \rho_{m} \otimes \id_{\Gru{\mathcal{H}}{k}{l}}.
%   \end{align*}
% \end{enumerate}
% Here, the infinite sum in 1.\ makes sense in the multiplier algebra,
% and the sums in 2.\ are finite. The family
% $Z=(\Gr{Z}{k}{l}{m}{n})_{k,l,m,n}$ in 2.\ is uniquely
% determined by $X$ and will be denoted by $X^{-1}$.
\begin{Lem} \label{lemma:rep-corep}
  Let $\mathcal{H}=\bigoplus_{k,l} \Gru{\mathcal{H}}{k}{l}$ be a row-
  and column-finite $I^{2}$-graded Hilbert space and let
  $X=(\Gr{X}{k}{l}{m}{n})_{k,l,m,n}$ be a family of elements
  $\Gr{X}{k}{l}{m}{n} \in \Gr{A}{k}{l}{m}{n} \otimes
  \mathcal{B}(\Gru{\mathcal{H}}{m}{n},\Gru{\mathcal{H}}{k}{l})$
  satisfying
  \begin{align} \label{eq:rep-delta}
    (\tilde \Delta \otimes \id)(\Gr{X}{k}{l}{m}{n}) = \sum_{p,q}
    \left(\Gr{X}{k}{l}{p}{q}\right)_{13}\left(\Gr{X}{p}{q}{m}{n}\right)_{23}. 
  \end{align}
  Then the following conditions are equivalent:
  \begin{enumerate}
  \item $(\epsilon \otimes
    \id)(\Gr{X}{k}{l}{m}{n})=\delta_{k,m}\delta_{l,n}\id_{\Gru{\mathcal{H}}{k}{l}}$;
  \item there exist elements $\Gr{Z}{k}{l}{m}{n} \in
    \Gr{A}{n}{m}{l}{k} \otimes
    \mathcal{B}(\Gru{\mathcal{H}}{m}{n},\Gru{\mathcal{H}}{k}{l})$ such
    that
  \begin{align*}
    \sum_{k} \Gr{Z}{m}{n'}{k}{l} \Gr{X}{k}{l}{m}{n} &= \delta_{n,n'}
    \lambda_{l} \rho_{n} \otimes \id_{\Gru{\mathcal{H}}{m}{n}}, &
    \sum_{n} \Gr{X}{k}{l}{m}{n} \Gr{Z}{m}{n}{k'}{l} &= \delta_{k,k'}
    \lambda_{k} \rho_{m} \otimes \id_{\Gru{\mathcal{H}}{k}{l}}.
  \end{align*}
  \end{enumerate}
  If these conditions hold, then the family $X^{-1}:=Z$ is unique and
  given by $\Gr{(X^{-1})}{k}{l}{m}{n}=(S \otimes
  \id)(\Gr{X}{k}{l}{m}{n})$. In particular, it satisfies
  \begin{align*}
    (\tilde \Delta \otimes \id)(\Gr{(X^{-1})}{k}{l}{m}{n}) &=
    \sum_{p,q}
    \left(\Gr{(X^{-1})}{k}{l}{m}{n}\right)_{23}\left(\Gr{(X^{-1})}{m}{n}{p}{q}\right)_{13}.
  \end{align*}
\end{Lem}
Note that the sum in \eqref{eq:rep-delta} makes sense in the
multiplier algebra, and that the sums in (2) are finite because
$\mathcal{H}$ is row- and column-finite. If it exists, we denote the family $Z$ in condition (2)  by $X^{-1}$.
\begin{proof}
   Assume that (1) holds and let  $\Gr{Z}{k}{l}{m}{n}=(S \otimes
   \id)(\Gr{X}{k}{l}{m}{n})$. Then the antipode axiom implies
   \begin{align*}
     \sum_{k} \Gr{Z}{m}{n'}{k}{l} \Gr{X}{k}{l}{m}{n} &= \sum_{k}
     (S\otimes \id)(\Gr{X}{m}{n'}{k}{l}) \Gr{X}{k}{l}{m}{n}
     \\
     &=\sum_{k} (m_{A} \otimes \id)(S\otimes \id \otimes \id) (
     (\Gr{X}{m}{n
       '}{k}{l})_{13}(\Gr{X}{k}{l}{m}{n})_{23}) \\
     &= (m_{A} \otimes \id)(S\otimes \id \otimes \id)((\tilde \Delta
     \otimes \id)(\Gr{X}{m}{n'}{m}{n})(1 \otimes \lambda_{l}\otimes
     1)) \\
     &= \sum_{p} \rho_{p}\lambda_{l} \otimes (\epsilon(-\lambda_{p}
    ) \otimes \id)(\Gr{X}{m}{n'}{m}{n}) \\
     &= \rho_{n}\lambda_{l} \otimes \delta_{n',n}
     \id_{\Gru{\mathcal{H}}{m}{n}},
   \end{align*}
   which is the first equation in (2). The second one follows
   similarly.

   Conversely, assume that (2) holds. Then uniqueness of the family
   $Z$ is easily verified. Let $\Gr{c}{k}{l}{m}{n}:=(\epsilon \otimes
   \id)(\Gr{X}{k}{l}{m}{n})$.  If $(k,l)\neq (m,n)$, then
   $\Gr{c}{k}{l}{m}{n} = 0$ because
   $\epsilon(\Gr{A}{k}{l}{m}{n})=0$. Relation \eqref{eq:rep-delta} and
   the counit property imply
\begin{align*}
  \Gr{X}{k}{l}{m}{n} &=
  (\lambda_{k} \otimes 1)\Gr{X}{k}{l}{m}{n}(\lambda_{l}\otimes 1) \\
  &= (\epsilon \otimes \id \otimes \id)(((1 \otimes \lambda_{k} \otimes
  1)(\hat \Delta \otimes \id)(\Gr{X}{k}{l}{m}{n})(1 \otimes \lambda_{l}
  \otimes 1)) \\ &=(\epsilon \otimes \id \otimes \id)
  \left(\left(\Gr{X}{k}{l}{k}{l}\right)_{13}\left(\Gr{X}{k}{l}{m}{n}\right)_{23}\right)
\\ &  = (1 \otimes \Gr{c}{k}{l}{k}{l}) \Gr{X}{k}{l}{m}{n}.
   \end{align*}
   We multiply on the right by $\Gr{Z}{m}{n}{k}{l}$, sum over $n$, use
   condition (2) and find
   $\Gr{c}{k}{l}{k}{l}=\id_{\Gru{\mathcal{H}}{k}{l}}$.
\end{proof}

A \emph{locally finite corepresentation of $(A,\Delta)$} is a pair
$(\mathcal{H},X)$ as above satisfying conditions (1) and (2). A
\emph{morphism} $T$ between two such corepresentations
$(\mathcal{H},X)$ and $(\mathcal{K},Y)$ is a family of morphisms
$\Gru{T}{k}{l} \in
\mathcal{B}(\Gru{\mathcal{H}}{k}{l},\Gru{\mathcal{K}}{k}{l})$
satisfying $(1 \otimes \Gru{T}{k}{l})\Gr{X}{k}{l}{m}{n} =
\Gr{Y}{k}{l}{m}{n}(1 \otimes \Gru{T}{m}{n})$.  Using condition (2) in
Lemma \ref{lemma:rep-corep}, one finds that the last equation holds if
and only if
  \begin{align} 
    \sum_{m} \Gr{(Y^{-1})}{k}{l'}{m}{n}(1 \otimes
    \Gru{T}{m}{n})\Gr{X}{m}{n}{k}{l} &= \delta_{l,l'}
    \lambda_{n}\rho_{l} \otimes \Gru{T}{k}{l}, \\
    \sum_{n} \Gr{Y}{k'}{l}{m}{n}(1 \otimes
    \Gru{T}{m}{n})\Gr{(X^{-1})}{m}{n}{k}{l} &= \delta_{k,k'}
    \lambda_{m} \rho_{k} \otimes \Gru{T}{k}{l}.
 \end{align}
For example, if the first equation holds, then
\begin{align*}
  \Gr{Y}{m}{n}{k}{l} (1 \otimes \Gru{T}{k}{l}) &=
  \sum_{m}  \Gr{Y}{m}{n}{k}{l} \Gr{(Y^{-1})}{k}{l}{m}{n}(1 \otimes
  \Gru{T}{m}{n})\Gr{X}{m}{n}{k}{l} = (\lambda_{k}\rho_{m}\otimes \Gru{T}{m}{n})\Gr{X}{m}{n}{k}{l}.
\end{align*}
Conversely, if $T$ is a morphism, then
\begin{align*}
  \sum_{m} \Gr{(Y^{-1})}{k}{l'}{m}{n}(1 \otimes
  \Gru{T}{m}{n})\Gr{X}{m}{n}{k}{l} &= \sum_{m}
  \Gr{(Y^{-1})}{k}{l'}{m}{n} \Gr{Y}{m}{n}{k}{l}(1 \otimes
  \Gru{T}{k}{l}) = \delta_{l',l} \lambda_{n}\rho_{l} \otimes
  \Gru{T}{k}{l}.
\end{align*}

We denote by $\Corep(A,\Delta)$ the category of locally finite
corepresentations of $(A,\Delta)$ with morphisms as above.


Let $(\mathcal{H},X)$ be a locally finite
corepresentation. We call a family of subspaces
$\Gru{\mathcal{K}}{k}{l} \subseteq \Gru{\mathcal{H}}{k}{l}$
\emph{invariant} if $\Gr{X}{k}{l}{m}{n}(1 \otimes \Gru{P}{m}{n}) =
(1\otimes \Gru{P}{k}{l})\Gr{X}{k}{l}{m}{n}$, where $\Gru{P}{k}{l}$
denotes the projection
$\Gru{\mathcal{H}}{k}{l}\to\Gru{\mathcal{K}}{k}{l}$.

The following analogue of Schur's Lemma holds.  Let $T$ be a morphism
of locally finite corepresentations $(\mathcal{H},X)$ and
$(\mathcal{K},Y)$. Then $ \bigoplus_{k,l} \ker \Gru{T}{k}{l}$ and
$\bigoplus_{k,l} \img \Gru{T}{k}{l}$ are invariant subspaces of
$\mathcal{H}$ and $\mathcal{K}$, respectively. In particular, if
$(\mathcal{H},X)$ and $(\mathcal{K},Y)$ are irreducible, then $T$ is
either $0$ or an isomorphism.


\paragraph{Unitarity of corepresentations}
We call a locally finite corepresentation $(\mathcal{H},X)$
\emph{unitary} if $\Gr{(X^{-1})}{k}{l}{m}{n}=(\Gr{X}{m}{n}{k}{l})^{*}$.  In this paragraph, we assume that
$(A,\Delta)$ has a faithful positive functional, and show that then
every irreducible locally finite-dimensional corepresentation is
equivalent to a unitary one, by embedding it into a restriction of the
regular corepresentation.
\begin{Exa} \label{exa:rep-regular}
  Assume that $(A,\Delta)$ has a positive invariant functional $\phi$
  that is faithful. 

  Let $\Gru{\mathcal{H}}{k}{l} \subseteq \bigoplus_{m,n}
  \Gr{A}{m}{n}{k}{l}$ be a row- and column-finite family of
  finite-dimensional subspaces satisfying
\begin{align*}
 \tilde  \Delta^{\co} (\Gru{\mathcal{H}}{m}{n}) \subseteq \sum_{p,q}
 \Gr{A}{p}{q}{m}{n} \otimes \Gru{\mathcal{H}}{p}{q}.
\end{align*}
Equip each $\Gru{\mathcal{H}}{k}{l}$ with the scalar product $\langle
a|b\rangle:=\phi(a^{*}b)$ and take the Hilbert space direct sum
$\mathcal{H}:=\bigoplus_{k,l} \Gru{\mathcal{H}}{k}{l}$.
Define $\Gr{V}{k}{l}{m}{n} \in \Gr{A}{k}{l}{m}{n} \otimes
\mathcal{B}(\Gru{\mathcal{H}}{m}{n},\Gru{\mathcal{H}}{k}{l})$ by the
equation
\begin{align*}
  \Gr{V}{k}{l}{m}{n}|a\rangle_{2} &= \tilde \Delta^{\co}(a),
\end{align*}
where $\Gr{V}{k}{l}{m}{n}|a\rangle_{2}$  denotes the application of
the second leg of
$\Gr{V}{k}{l}{m}{n}$  to $a \in \Gru{\mathcal{H}}{m}{n}$. Then
$(\mathcal{H},V)$ is a unitary corepresentation. We call it the 
\emph{locally finite restriction of the regular corepresentation}
determined by the family $(\Gru{\mathcal{H}}{k}{l})_{k,l}$. 
\fxnote{proof this}
\end{Exa}

\begin{Lem} \label{lem:rep-morphism-regular}
  Assume that $(A,\Delta)$ has a faithful and positive invariant
  functional.  Let $(\mathcal{H},X)$ be a locally
  finite-dimensional corepresentation and let $\xi \in
  \Gru{\mathcal{H}}{k}{l}$. Then the family of finite-dimensional
  subspaces
  \begin{align*}
   \Gru{\mathcal{K}}{m}{n}  &=  \{ (\id \otimes
   \omega_{\xi,\eta})(\Gr{X}{k}{l}{m}{n}) : \eta \in
   \Gru{\mathcal{H}}{m} {n}\} \subseteq \Gr{A}{k}{l}{m}{n}
  \end{align*}
  defines a restriction of the regular corepresentation
  $(\mathcal{K},V)$, and the family of maps
  \begin{align*}
    \Gru{T_{(\xi)}}{m}{n} \colon \Gru{\mathcal{H}}{m}{n} \to
    \Gru{\mathcal{K}}{m}{n}, \ \eta \mapsto (\id \otimes
    \omega_{\xi,\eta})(\Gr{X}{k}{l}{m}{n}),
  \end{align*}
  is a morphism from $(\mathcal{H},X)$ to $(\mathcal{K},V)$.
\end{Lem}
Note that the family $(\Gru{\mathcal{K}}{m}{n})_{m,n}$ is row- and
column-finite because $(\Gru{\mathcal{H}}{m}{n})_{m,n}$
is. 
\begin{proof} Both assertions  follow from the fact
  that for all $\eta \in \Gru{\mathcal{H}}{p}{q}$,
\begin{align*}
  (1 \otimes \rho_{m})\cdot \tilde \Delta^{\co}\big((\id \otimes
  \omega_{\xi,\eta})(\Gr{X}{k}{l}{p}{q})\big) \cdot (1 \otimes
  \rho_{n}) &= (\id \otimes \id \otimes \omega_{\xi,\eta})\big(
  (\Gr{X}{k}{l}{m}{n})_{23}(\Gr{X}{m}{n}{p}{q})_{13})\big) \\ &=(1
  \otimes \Gru{T_{(\xi)}}{m}{n})\Gr{X}{m}{n}{p}{q} |\eta\rangle_{2}.
\end{align*}
\end{proof}
\begin{Prop}  \label{prop:rep-unitarisable}
  Assume that $(A,\Delta)$ has a faithful and positive invariant
  functional. Then every irreducible locally finite corepresentation
  is equivalent to a unitary one.
\end{Prop}
\begin{proof}
  Let $(\mathcal{H},X)$ be an irreducible locally finite
  corepresentation. Then for some $k,l$ and $\xi \in
  \Gru{\mathcal{H}}{k}{l}$,  the operator  $T_{(\xi)}$  defined in
  Lemma \ref{lem:rep-morphism-regular} has to be non-zero and hence,
  by Schur's Lemma, injective. Thus, it forms an equivalence between
  $(\mathcal{H},X)$ and a sub-corepresentation of a locally finite
  restriction of the regular corepresentation, which is unitary by
  Example \ref{exa:rep-regular}.
\end{proof}


\paragraph{Schur orthogonality relations}
In this paragraph, we obtain the analogue of Schur's orthogonality
relations for matrix coefficients of corepresentations.

The space of \emph{matrix coefficients} $\mathcal{C}(X)$ of a locally
finite corepresentation $(\mathcal{H},X)$ is the sum of
the subspaces
\begin{align*}
  \Gr{\mathcal{C}(X)}{k}{l}{m}{n} &= \span \left\{ (\id \otimes
  \omega_{\xi,\eta})(\Gr{X}{k}{l}{m}{n}) \mid \xi \in
  \Gru{\mathcal{H}}{k}{l}, \eta \in \Gru{\mathcal{H}}{m}{n} \right\}
\subseteq \Gr{A}{k}{l}{m}{n}.
\end{align*}

\begin{Prop} \label{prop:rep-weak-pw}
  Assume that $(A,\Delta)$ has a faithful and positive invariant
  functional. Then $A$ is the sum of the matrix coefficients of
unitary  irreducible locally finite corepresentations.
\end{Prop}
\begin{proof}
  Let $a \in \Gr{A}{k}{l}{m}{n}$. Then
  $(1 \otimes \rho_{p})\tilde \Delta^{\co}(a)(1 \otimes \rho_{q}) \in
  \Gr{A}{p}{q}{m}{n} \otimes \Gr{A}{k}{l}{p}{q}$ and the subspace
    \begin{align*}
    \Gru{\mathcal{H}}{p}{q} &:= \span \{ (\omega \otimes \id)(\tilde
    \Delta^{\co}(a)) : \omega \in (\Gr{A}{p}{q}{m}{n})' \} \subseteq
    \Gr{A}{k}{l}{p}{q}
  \end{align*}
  has finite dimension. Since $(1 \otimes \rho_{p})\tilde
  \Delta^{\co}(a)$ and $\tilde \Delta^{\co}(a)(1 \otimes \rho_{q})$
  lie in the tensor product $A \otimes A$, the family
  $(\Gru{\mathcal{H}}{p}{q})_{p,q}$ is row- and column-finite. Using
  co-associativity, one checks that this family defines a locally
  finite restriction $(\mathcal{H},V)$ of the regular
  corepresentation. Evidently, $a \in \mathcal{C}(V)$. Decomposing
  $(\mathcal{H},V)$, we find that
  $a$ is contained in the sum of matrix coefficients of unitary
  irreducible corepresentations.
\end{proof}
The key to the orthogonality relations is the following averaging procedure.
\begin{Lem} \label{lem:rep-average}
  Let $\phi$ be an invariant functional for $(A,\Delta)$, let
  $(\mathcal{H},X)$ and $(\mathcal{K},Y)$ be locally
  finite-dimensional corepresentations of $(A,\Delta)$ and let $T$ be
  a family of operators $\Gru{T}{k}{l} \in
  \mathcal{B}(\Gru{\mathcal{H}}{k}{l},\Gru{\mathcal{K}}{k}{l})$ which
  are $0$ for all but finitely many $k,l$. Then the families $\check T$ and
  $\hat T$ given by
  \begin{align*}
    \Gru{\check T}{k}{l} &:= \sum_{m,n} (\phi \otimes
    \id)(\Gr{(Y^{-1})}{k}{l}{m}{n}(1\otimes
    \Gru{T}{m}{n})\Gr{X}{m}{n}{k}{l}), \\
    \Gru{\hat T}{k}{l} &:= \sum_{m,n} (\phi \otimes
    \id)(\Gr{Y}{k}{l}{m}{n}(1\otimes
    \Gru{T}{m}{n})\Gr{(X^{-1})}{m}{n}{k}{l})
  \end{align*}
  are morphisms from $(\mathcal{H},X)$ to $(\mathcal{K},Y)$.
\end{Lem}
\begin{proof}
 The assertion converning $\check{T}$ follows
  from the calculation
  \begin{align*}
    &\sum_{m} \Gr{(Y^{-1})}{k}{l'}{m}{n}(1 \otimes
    \Gru{\check{T}}{m}{n})\Gr{X}{m}{n}{k}{l} = \\
    &=\sum_{m,p,q} (\phi \otimes \id \otimes
    \id)\left(\left(\Gr{(Y^{-1})}{k}{l'}{m}{n}\right)_{23}
      \left(\Gr{(Y^{-1})}{m}{n}{p}{q}\right)_{13}(1 \otimes 1 \otimes
      \Gru{T}{p}{q})\left(\Gr{X}{p}{q}{m}{n}\right)_{13}\left(\Gr{X}{m}{n}{k}{l}\right)_{23}\right)
    \\
    % &= \sum_{m,p,q}(\phi \otimes \id \otimes \id)\left((1 \otimes
    %   \lambda_{n} \otimes 1)(\tilde \Delta \otimes
    %   \id)\left(\Gr{(Y^{-1})}{k}{l'}{p}{q}\right)(\rho_{m} \otimes 1
    %   \otimes \Gru{T}{p}{q})(\tilde \Delta \otimes
    %   \id)\left(\Gr{(X}{p}{q}{k}{l}\right)(1 \otimes\lambda_{n}
    %   \otimes 1)\right) \\
    &= \sum_{p,q} (\lambda_{n} \otimes 1) \cdot ((\phi \otimes \id)
    \circ \tilde \Delta \otimes \id)\left(\Gr{(Y^{-1})}{k}{l'}{p}{q}(1
      \otimes \Gru{T}{p}{q})\Gr{X}{p}{q}{k}{l}\right) \cdot
    (\lambda_{n} \otimes 1)
    \\
    &= \sum_{r,p,q} (\lambda_{n} \otimes 1) \cdot \left(\rho_{r}
      \otimes (\phi \otimes \id)\left(\Gr{(Y^{-1})}{k}{l'}{p}{q}(1 
      \otimes \Gru{T}{p}{q})\Gr{X}{p}{q}{k}{l}(\rho_{r} \otimes
      1)\right)\right) \cdot (\lambda_{n} \otimes 1) \\
  &= \delta_{l,l'}\lambda_{n}\rho_{l} \otimes \Gru{\check{T}}{k}{l},
  \end{align*}
  where we used the relation $\phi(\Grd{A}{l'}{l})=0$ for $l'\neq l$
  for the last equality. A similar calculation proves the assertion
  concerning $\hat{T}$. 
\end{proof}

% Choose a representative family of unitary irreducible locally finite
% corepresentations
% $(\Grd{\mathcal{H}}{(\alpha)}{},{_{(\alpha)}X})_{\alpha}$ and a basis
% $(\Gr{\zeta}{k}{l}{(\alpha)}{i})_{i}$ for each
% $\Gr{\mathcal{H}}{k}{l}{(\alpha)}{}$, and let
% \begin{align*}
%   (\Gr{(u_{\alpha})}{k}{l}{m}{n}){i,j} &:= (\id \otimes
%   \Gr{\omega}{k}{l}{(\alpha)}{i,j})( )
% \end{align*}

The first part of the orthogonality relations concerns matrix
coefficients of inequivalent irreducible corepresentations.
\begin{Prop} \label{prop:rep-orthogonality-1}
  Let $(\mathcal{H},X)$ and $(\mathcal{K},Y)$ be inequivalent unitary
  irreducible locally finite-dimensional corepresentations and let
  $\phi$ be an invariant functional for $(A,\Delta)$.  Then
  $\phi(S(b)a) =\phi(b^{*}a) = \phi(bS(a))=\phi(ba^{*})=0$ for all
  $a\in \mathcal{C}(X), b \in \mathcal{C}(Y)$.
\end{Prop}
\begin{proof}
  Let $a=(\id \otimes \omega_{\xi,\xi'})(\Gr{X}{k}{l}{m}{n})$ and
  $b=(\id \otimes \omega_{\eta,\eta'})(\Gr{Y}{p}{q}{r}{s})$, where
  $\xi \in \Gru{\mathcal{H}}{k}{l}, \xi' \in \Gru{\mathcal{H}}{m}{n}$
  and $\eta \in \Gru{\mathcal{K}}{p}{q}, \eta' \in
  \Gru{\mathcal{K}}{r}{s}$. We may assume $(p,q,r,s) = (m,n,k,l)$
  because $\phi(S(a)b) = 0 = \phi(aS(b))$ otherwise.  
Lemma \ref{lem:rep-average}, applied to the 
  family $\Gru{T}{p}{q}= \delta_{p,k}\delta_{q,l}
  |\eta'\rangle\langle\xi|$, yields  morphisms $\check{T},\hat{T}$
  from $(\mathcal{H},X)$ to $(\mathcal{K},Y)$ which necessarily are
  $0$. Inserting the definition of $\check{T}$, we find
  \begin{align*}
    \phi(S(b)a) &= \phi\big((S \otimes
    \omega_{\eta,\eta'})(\Gr{Y}{m}{n}{k}{l}) \cdot (\id \otimes
    \omega_{\xi,\xi'})(\Gr{X}{k}{l}{m}{n})\big) \\ &= (\phi \otimes
    \id)\left(\langle\eta|_{2} \Gr{(Y^{-1})}{m}{n}{k}{l}(1 \otimes
      |\eta'\rangle\langle \xi|)
      \Gr{X}{k}{l}{m}{n}|\xi'\rangle_{2}\right) 
    = \langle \eta|_{2} \Gru{\check{T}}{m}{n}|\xi'\rangle_{2} = 0.
  \end{align*}
  A similar calculation involving $\hat{T}$ shows that
  $\phi(aS(b))=0$.  Using the relation $X^{*}=X^{-1}=(S\otimes
  \id)(X)$ and $Y^{*}=(S\otimes \id)(Y)$, we conclude
  $\phi(a^{*}b)=\phi(ab^{*})=0$.
\end{proof}

The second part of the orthogonality relations concerns inner products
as above but with $a,b\in \mathcal{C}(X)$ for some irreducible
corepresentation  $X$  and involves the conjugate corepresentation,
which is defined as follows.



Given a Hilbert spaces $H,K$, we denote by $\overline{H},\overline{K}$
the conjugate Hilbert spaces, by $T \mapsto \overline{T}$ the
canonical conjugate-linear isomorphism $\mathcal{B}(H,K) \to
\mathcal{B}(\overline{H},\overline{K})$, and by $T \mapsto
T^{\top}:=\overline{T}^{*}$ the linear anti-isomorphism
$\mathcal{B}(H,K) \to \mathcal{B}(\overline{K},\overline{H})$.

\begin{Lem} \label{lemma:rep-functors}
   On the category
  $\Corep(A,\Delta)$, there exist
  \begin{enumerate}
  \item a covariant functor $(\mathcal{H},X) \mapsto
    (\overline{\mathcal{H}},\overline{X})$ and $T \mapsto
    \overline{T}$, where
    \begin{align*}
      \Gru{\overline{\mathcal{H}}}{k}{l} &= \overline{\Gru{\mathcal{H}}{l}{k}},
      & \Gr{\overline{X}}{k}{l}{m}{n} &= (\Gr{X}{l}{k}{n}{m})^{(*
        \otimes \overline{(\ \cdot \ ) })}
      =((\Gr{X}{l}{k}{n}{m})^{*})^{\id \otimes \top}, &
      \Gru{\overline{T}}{k}{l} &= \overline{\Gru{T}{l}{k}};
    \end{align*}
  \item a contravariant functor $(\mathcal{H},X) \mapsto
    (\overline{\mathcal{H}},X^{S\otimes \top})$ and
    $T\mapsto T^{\top}$, where 
    \begin{align*}
      \Gr{(X^{S\otimes \top})}{k}{l}{m}{n} &= (S \otimes (\ \cdot \
      )^{\top})(\Gr{X}{n}{m}{l}{k}), & \Gru{(T^{\top})}{k}{l}
      &=(\Gru{T}{l}{k})^{\top};
    \end{align*}
  \item a covariant functor  $(\mathcal{H},X) \mapsto (\mathcal{H},X^{S^{2}\otimes \id})$
    and $T\mapsto T$, where $\Gr{(X^{S^{2}\otimes \id})}{k}{l}{m}{n}=(S^{2} \otimes
  \id)(\Gr{X}{k}{l}{m}{n})$.
  \end{enumerate}
  If $(\mathcal{H},X)$ is unitary, then $\overline{X}=X^{S\otimes \top}$.\end{Lem}
\begin{proof}
  Let $(\mathcal{H},X)$ be a locally finite corepresentation. Using the fact that $\tilde \Delta$ and $\epsilon$ are
  $*$-homomorphisms, one easily verifies
  \begin{align*}
    (\tilde \Delta \otimes \id)( \Gr{\overline{X}}{k}{l}{m}{n}) &=
    \sum_{p,q} ( \Gr{\overline{X}}{k}{l}{p}{q})_{13}
    ( \Gr{\overline{X}}{p}{q}{m}{n})_{23}, \\
    (\epsilon \otimes \id)(\Gr{\overline{X}}{k}{l}{m}{n}) &=
    \overline{(\epsilon \otimes \id)(\Gr{X}{l}{k}{n}{m})}
    = \delta_{l,n}\delta_{k,m}
    \overline{\id_{\Gru{\mathcal{H}}{l}{k}}}
    = \delta_{l,n}\delta_{k,m} \id_{\Gru{\overline{\mathcal{H}}}{k}{l}}.
  \end{align*}
  A similar calculation shows that
  $(\overline{\mathcal{H}},X^{S\otimes \top})$ is a
  corepresentation. The assertions (1) and (2) follow
  immediately. The square of the functor in (2) yields the functor in
  (3).  
\end{proof}
We call $(\overline{\mathcal{H}},\overline{X})$ the \emph{conjugate} of 
$(\mathcal{H},X)$.   
\begin{Prop} \label{prop:rep-f}
  Assume that $(A,\Delta)$ has a faithful, positive, normalized
  invariant functional $\phi$ and let $(\mathcal{H},X)$ be a unitary
  irreducible locally finite corepresentation. 
  \begin{enumerate}
  \item $\overline{\mathcal{H}}$ and the family
    $\Gr{(\overline{X}^{-*})}{k}{l}{m}{n}:=\big(\Gr{(\overline{X}^{-1})}{m}{n}{k}{l}\big)^{*}$
    form a locally finite corepresentation and there
    exists an invertible, positive isomorphism $\overline{F_{X}}$ from
    $(\overline{\mathcal{H}},\overline{X})$ to
    $(\overline{\mathcal{H}},\overline{X}^{-*})$.
  \item The family
    $\Gru{F_{X}}{k}{l}:=\overline{\Gru{\overline{F_{X}}}{l}{k}}$ is an
    invertible, positive isomorphism from $(\mathcal{H},X)$ to
    $(\mathcal{H},X^{S^{2} \otimes \id})$.
  \end{enumerate}
\end{Prop}
\begin{proof}
(1) By Proposition \ref{prop:rep-unitarisable},
$(\overline{\mathcal{H}},\overline{X})$ is equivalent to a unitary
corepresentation, that is, there exists a family of operators
$\Gru{T}{k}{l} \in \mathcal{B}(\Gru{\overline{\mathcal{H}}}{k}{l})$
such that the family 
\begin{align*}
\Gr{Z}{k}{l}{m}{n}:= (1 \otimes
\Gru{T}{k}{l})\Gr{\overline{X}}{k}{l}{m}{n}(1 \otimes
\Gru{T}{m}{n})^{-1} 
\end{align*}
 is a unitary corepresentation. The relation
 $\Gr{(Z^{-1})}{k}{l}{m}{n}=\big(\Gr{Z}{m}{n}{k}{l})^{*}$ then implies
 \begin{align*}
   \Gr{Z}{k}{l}{m}{n} = (1 \otimes
   \Gru{T}{k}{l})^{-*}\big(\Gr{(\overline{X}^{-1})}{m}{n}{k}{l}\big)^{*}(1
   \otimes \Gru{T}{m}{n})^{*}
 \end{align*}
 and hence the family 
 $\Gr{(\overline{X}^{-*})}{k}{l}{m}{n}:=\big(\Gr{(\overline{X}^{-1})}{m}{n}{k}{l}\big)^{*}$
 is an irreducible locally finite corepresentation and the family
 $\Gru{\overline{F}_{X}}{k}{l}:=(\Gru{T}{k}{l})^{*}\Gru{T}{k}{l} \in
  \mathcal{B}(\Gru{\overline{\mathcal{H}}}{k}{l})$
is an isomorphism from $(\overline{\mathcal{H}},\overline{X})$ to
$(\overline{\mathcal{H}},\overline{X}^{-*})$.   

(2) The morphism $T$  from $(\overline{\mathcal{H}},\overline{X})$  to
$(\overline{\mathcal{H}},Z)$ yields morphisms $\overline{T}$ from
$(\mathcal{H},X)$ to $(\mathcal{H},\overline{Z})$ and $T^{\top}$ from $(\mathcal{H},Z^{S\otimes\top})$ to
$(\mathcal{H},\overline{X}{S \otimes \top})$. Since $X$ and $Z$ are
unitary, $\overline{Z}=Z^{S\otimes \top}$ and  $\overline{X}^{S \otimes
  \top} = X^{S^{2} \otimes \top}$. Thus $T^{\top}\overline{T} =
\overline{T^{*}T}$ is a morphism from $(\mathcal{H},X)$ to
$(\mathcal{H},X^{S^{2}\otimes \id})$.
\end{proof}
\begin{Theorem} \label{thm:rep-orthogonality} Assume that $(A,\Delta)$
  has a faithful, positive, normalized invariant functional
  $\phi$. Let $(\mathcal{H},X)$ be a unitary irreducible locally
  finite corepresentation of $(A,\Delta)$ and let $F_{X}$ be a
  non-zero morphism from $(\mathcal{H},X)$ to $(\mathcal{H},(S^{2}
  \otimes \id)(X))$.
  \begin{enumerate}
  \item The numbers $\alpha:=\sum_{k} \Tr(\Gru{(F_{X}^{-1})}{k}{l})$
    and $\beta:=\sum_{n} \Tr(\Gru{F_{X}}{m}{n})$ do not depend on $l$
    or $n$.
  \item  For all $k,l,m,n$,
    \begin{align*}
      (\phi \otimes \id)((\Gr{X}{k}{l}{m}{n})^{*}\Gr{X}{k}{l}{m}{n})
      &=\alpha^{-1}{\Tr(\Gru{(F^{-1}_{X})}{k}{l})} \cdot
      \id_{\Gru{\mathcal{H}}{m}{n}}, \\
      (\phi \otimes \id)(\Gr{X}{k}{l}{m}{n}(\Gr{X}{k}{l}{m}{n})^{*})
      &=\beta^{-1}{\Tr( \Gru{(F_{X})}{m}{n})} \cdot
      \id_{\Gru{\mathcal{H}}{k}{l}}.
    \end{align*}
  \item Denote by $\Sigma_{k,l,m,n}$ the flip $\Gru{\mathcal{H}}{k}{l}
    \otimes \Gru{\mathcal{H}}{m}{n} \to \Gru{\mathcal{H}}{m}{n}
    \otimes \Gru{\mathcal{H}}{k}{l}$. Then
 \begin{align*}
   (\phi \otimes \id \otimes
   \id)((\Gr{X}{k}{l}{m}{n})_{12}^{*}(\Gr{X}{k}{l}{m}{n})_{13}) &=
   \alpha^{-1}
   (\id_{\Gru{\mathcal{H}}{m}{n}} \otimes \Gru{(F_{X}^{-1})}{k}{l})
   \circ \Sigma_{k,l,m,n}, \\
   (\phi \otimes \id \otimes
   \id)((\Gr{X}{k}{l}{m}{n})_{13}(\Gr{X}{k}{l}{m}{n})_{12}^{*}) &= \beta^{-1} (\Gru{F_{X}}{m}{n}
   \otimes \id_{\Gru{\mathcal{H}}{k}{l}}) \circ \Sigma_{k,l,m,n}.
 \end{align*}
\end{enumerate}
  \end{Theorem}
\begin{proof}
  We prove the assertions and equations involving $\alpha$ in (1), (2)
  and (3)  simultaneously; the assertions involving $\beta$  follow similarly.

  As above, we denote by $\Sigma_{p,q,r,s}$ the flip
  $\Gru{\mathcal{H}}{p}{q} \otimes \Gru{\mathcal{H}}{r}{s} \to
  \Gru{\mathcal{H}}{r}{s} \otimes \Gru{\mathcal{H}}{p}{q}$.  Consider
  the operator
  \begin{align*}
    F_{m,n,k,l}
    &:=(\phi \otimes \id \otimes \id)((\Gr{X}{k}{l}{m}{n})_{12}^{*}(\Gr{X}{k}{l}{m}{n})_{13})
    \circ \Sigma_{m,n,k,l} \\ &= (\phi \otimes \id \otimes
    \id)\left((\Gr{(X^{-1})}{m}{n}{k}{l})_{12}
      (\Sigma_{k,l,k,l})_{23}(\Gr{X}{k}{l}{m}{n})_{12}\right).
  \end{align*}
  By Lemma \ref{lem:rep-average}, the family $(F_{m,n,k,l})_{m,n}$ is
  an endomorphism of $(\mathcal{H} \otimes \Gru{\mathcal{H}}{k}{l}, (X)_{12})$ and hence
  \begin{align}
    F_{m,n,k,l} &= \id_{\Gru{\mathcal{H}}{m}{n}} \otimes \Gru{R}{k}{l} \label{eq:rep-orthogonal-1}
  \end{align}
  with some $\Gru{R}{k}{l} \in \mathcal{B}(\Gru{\mathcal{H}}{k}{l})$ not
  depending on $m,n$.
  On the other hand, 
  \begin{align*}
    F_{m,n,k,l} &= (\phi \otimes \id \otimes \id)((S \otimes
    \id)(\Gr{X}{m}{n}{k}{l})_{12}(\Gr{X}{k}{l}{m}{n})_{13})
    \circ \Sigma_{m,n,k,l} \\
    &= (\phi\circ S^{-1} \otimes \id \otimes \id)\left(((S \otimes
      \id)(\Gr{X}{k}{l}{m}{n}))_{13}
      ((S^{2} \otimes \id)(\Gr{X}{m}{n}{k}{l}))_{12}\right)     \circ \Sigma_{m,n,k,l}\\
    &= (\phi\circ S^{-1} \otimes \id \otimes \id)\left(((S \otimes
      \id)(\Gr{X}{k}{l}{m}{n}))_{13} (\Sigma_{m,n,m,n})_{23} ((S^{2}
      \otimes \id)(\Gr{X}{m}{n}{k}{l}))_{12}\right).
  \end{align*}
  Since $\phi\circ S^{-1}$ is an invariant functional for
  $(A,\Delta)$, we can again apply Lemma \ref{lem:rep-average} and
  find that the family $(F_{m,n,k,l})_{k,l}$ is a morphism from
  $(\Gru{\mathcal{H}}{m}{n} \otimes \mathcal{H}, (X^{S^{2} \otimes
    \id})_{13})$ to $(\Gru{\mathcal{H}}{m}{n} \otimes \mathcal{H},
  (X)_{13})$. Therefore,
  \begin{align}
    F_{m,n,k,l} &= \Gru{T}{m}{n} \otimes (\Gru{F_{X}}{k}{l})^{-1} \label{eq:rep-orthogonal-2}
  \end{align}
  with some $\Gru{T}{m}{n} \in \mathcal{B}(\Gru{\mathcal{H}}{m}{n})$
  not depending on $k,l$. Combining \eqref{eq:rep-orthogonal-1} and
  \eqref{eq:rep-orthogonal-2}, we conclude that $F_{m,n,k,l} = \lambda
  (\id_{\Gru{\mathcal{H}}{m}{n}} \otimes (\Gru{F_{X}}{k}{l})^{-1})$
  for some $\lambda \in \C$.  Choose a basis
  $(\zeta_{i})_{i}$ for $\Gru{\mathcal{H}}{k}{l}$. Then
  \begin{align*}
    \lambda \cdot \id_{\Gru{\mathcal{H}}{m}{n}} \cdot
    \Tr((\Gru{F_{X}}{k}{l})^{-1}) &= \sum_{i} (\id \otimes
    \omega_{\zeta_{i},\zeta_{i}})(F_{m,n,k,l}) = (\phi \otimes
    \id)((\Gr{X}{m}{n}{k}{l})^{*} \Gr{X}{k}{l}{m}{n}).
  \end{align*}
  We sum over $k$, use the relations $\sum_{k}
  (\Gr{X}{m}{n}{k}{l})^{*} \Gr{X}{k}{l}{m}{n} = \lambda_{l}\rho_{n}
  \otimes \id_{\Gru{\mathcal{H}}{m}{n}}$ and
  $\phi(\lambda_{l}\rho_{n})=1$, and find
\begin{align*}
\lambda \cdot  \sum_{k} \Tr((\Gru{F_{X}}{k}{l})^{-1}) = 1.
\end{align*}
Now all assertions in (1)--(3) concerning $\alpha$ follow.
 % the second formula, we use the first formula for the
 %    opposite of $(A,\Delta)$. For this opposite, $\phi$ still is a
 %    faithful, positive, normalized invariant functional and
 %    $(\mathcal{H},X)$ still is a unitary irreducible locally finite
 %    corepresentation, but the antipode $S$ gets replaced by $S^{-1}$
 %    and therefore $F_{X}$ gets replaced by $F_{X}^{-1}$.
\end{proof}
\begin{Cor}
  Assume that $(A,\Delta)$ has a faithful, positive, normalized
  invariant functional $\phi$. Let $(\mathcal{H},X)$ be a unitary
  irreducible locally finite corepresentation of $(A,\Delta)$, let
  $F_{X}$ be a non-zero morphism from $(\mathcal{H},X)$ to
  $(\mathcal{H},(S^{2} \otimes \id)(X))$, and let $a=(\id \otimes
  \omega_{\xi,\xi'})(\Gr{X}{k}{l}{m}{n})$ and $b=(\id \otimes
  \omega_{\eta,\eta'})(\Gr{X}{p}{q}{r}{s})$, where $\xi \in
  \Gru{\mathcal{H}}{k}{l}, \xi' \in \Gru{\mathcal{H}}{m}{n}$ and $\eta
  \in \Gru{\mathcal{H}}{p}{q}, \eta' \in \Gru{\mathcal{H}}{r}{s}$.
  Then
\begin{align*}
  \phi(b^{*}a) &= \frac{\langle \eta'|\xi'\rangle \langle
    \xi|F^{-1}_{X}\eta\rangle}{\sum_{k}
    \Tr(\Gru{(F^{-1}_{X})}{k}{l})}, & \phi(ab^{*}) = \frac{\langle
    \eta'|F_{X}\xi'\rangle \langle \xi|\eta\rangle}{\sum_{n}
    \Tr(\Gru{F_{X}}{m}{n})}.
\end{align*}
\end{Cor}
\begin{proof}
  By Theorem \ref{thm:rep-orthogonality}, 
  \begin{align*}
    \phi(b^{*}a) &= (\phi \otimes \omega_{\eta',\eta} \otimes
    \omega_{\xi,\xi'})((\Gr{X}{k}{l}{m}{n})_{12}^{*}\Gr{X}{k}{l}{m}{n})
    \\
    &= \frac{1}{\sum_{k} \Tr(\Gru{(F_{X}^{-1})}{k}{l})} 
    (\omega_{\eta',\eta} \otimes
    \omega_{\xi,\xi'})(   (
    \id_{\Gru{\mathcal{H}}{m}{n}} \otimes \Gru{(F_{X}^{-1})}{k}{l})
    \circ \Sigma_{k,l,m,n}). 
  \end{align*}
  The formula for $\phi(ab^{*})$ follows similarly or by considering
  the opposite of $(A,\Delta)$.
\end{proof}
\begin{Cor} \label{cor:rep-pw}
  Assume that $(A,\Delta)$ has a faithful, positive, normalized
  invariant functional $\phi$. Let
  $(\mathcal{H}_{\alpha},X_{\alpha})_{\alpha}$ be a representative
  family of all irreducible locally finite corepresentations of
  $(A,\Delta)$. Then the map
  \begin{align*}
    \bigoplus_{\alpha} \bigoplus_{k,l,m,n}
    (\overline{\Gru{\mathcal{H}_{\alpha}}{k}{l}} \otimes
    \Gru{\mathcal{H}_{\alpha}}{m}{n}) \to A
  \end{align*}
  that sends $\overline{\xi} \otimes \eta \in
  \overline{\Gru{\mathcal{H}_{\alpha}}{k}{l}} \otimes
  \Gru{\mathcal{H}_{\alpha}}{m}{n}$ to $ (\id \otimes
  \omega_{\xi_{\alpha},\eta_{\alpha}})(\Gr{(X_{\alpha})}{k}{l}{m}{n})$,
  is a linear isomorphism.
\end{Cor}

Given $\omega,\omega' \in A'$ and $a \in A$, we define convolution
products
\begin{align*}
  \omega \ast a
&:= (\id \otimes \omega) (\tilde \Delta(a)), & a \ast
\omega'&:=(\omega' \otimes \id)(\Delta(a)), & \omega \ast a \ast
\omega'&:= (\omega \ast a)\ast \omega' = \omega \ast (a \ast \omega').\end{align*}
We shall say that an entire function $f$ has \emph{exponential growth
  on the right half-plane} if there exist $C,d$ such that $|f(x+iy)|\leq
C\mathrm{e}^{dx}$  for all $x,y\in \R$ with $x>0$. 
\begin{Theorem} \label{thm:rep-characters}
  Assume that $(A,\Delta)$ has a faithful, positive, normalized
  invariant functional $\phi$.  There exists a unique family of characters
  $f_{z} \colon A\to \C$ such that
  \begin{enumerate}
  \item for each $a\in A$, the function $z\mapsto f_{z}(a)$ is entire
    and of exponential growth on the right half-plane;
  \item $f_{0} = \epsilon$ and $(f_{z} \otimes f_{z'}) \circ \tilde
    \Delta= f_{z+z'}$ for all $z,z' \in \C$.
  \item $\phi(ab)=\phi(b(f_{1} \ast a \ast f_{1}))$ for all $a,b\in A$.
  \end{enumerate}
  This family furthermore satisfies
  \begin{enumerate}\setcounter{enumi}{3}
  \item $S^{2}(a)=f_{-1} \ast a \ast f_{1}$ for all $a\in A$;
  \item $f_{z}(\lambda_{l}\rho_{n})=\delta_{l,n}$,  $f_{z} \circ S = f_{-z}$,
and    $f_{z} \circ \ast = \ast \circ f_{-\overline{z}}$.
\end{enumerate}
\end{Theorem}
\begin{proof}
  We first prove uniqueness.  Assume that $(f_{z})_{z}$ is a family of
  functionals satisfying (1)--(3).  Since $\phi$ is faithful, the map
  $\sigma\colon a \mapsto f_{1} \ast a \ast f_{1}$ is uniquely
  determined, and one easily sees that is is a homomorphism. Using
  (2), we find that $\epsilon \circ \sigma=f_{2}$ and $f_{2n}$ are
  uniquely determined and characters for each $n$. Using (1) and the
  fact that every entire function of exponential growth on the right
  half-plane is uniquely determined by its values at $\N \subseteq \C$
  \cite{}, we can conclude that each $f_{z}$ is uniquely determined
  and a character.

  Let us now prove existence.  By Corollary \ref{cor:rep-pw}, we can
  define for each $z\in \C$ a functional $f_{z} \colon A \to \C$ such
  that for every unitary irreducible locally finite corepresentation
  $(\mathcal{H},X)$,
    \begin{align*}
      f_{z}((\id \otimes \omega_{\xi,\eta})(\Gr{X}{k}{l}{m}{n})) &=
      \delta_{k,m}\delta_{l,n} \cdot
      \omega_{\xi,\eta}((\Gru{F_{X}}{k}{l})^{z}) \quad \text{for all }
      \xi \in \Gru{\mathcal{H}}{k}{l},\eta \in
      \Gru{\mathcal{H}}{m}{n},
    \end{align*}
    or, equivalently,
    \begin{align*}
      (f_{z} \otimes \id)(\Gr{X}{k}{l}{m}{n}) =
      \delta_{k,m}\delta_{l,n} \cdot (\Gru{F_{X}}{k}{l})^{z},
    \end{align*}
    where $F_{X}$ is a non-zero morphism from $(\mathcal{H},X)$ to
    $(\mathcal{H}, (S^{2} \otimes \id)(X))$ such that
    \begin{align*}
      \alpha_{X}:= \sum_{k} \Tr(\Gru{(F_{X}^{-1})}{k}{l}) = \sum_{n}
      \Tr(\Gru{F_{X}}{m}{n})
    \end{align*}
    for all $l,n$ (see Theorem \ref{thm:rep-orthogonality}). By
    construction, (1) holds. We show that $(f_{z})_{z}$ satisfies the
    assertions (2)--(5). The proof of uniqueness shows that each
    $f_{z}$ is a character. Throughout the following arguments, let 
    $(\mathcal{H},X)$ be a unitary irreducible corepresentation
    $(\mathcal{H},X)$ and let $F_{X}$ be as above.

    (2) This follows from the relations
    \begin{align*}
      (f_{0}  \otimes \id)(\Gr{X}{k}{l}{m}{n}) &=
      \delta_{k,m}\delta_{l,n} \id_{\Gru{\mathcal{H}}{k}{l}} =
      (\epsilon \otimes \id)(\Gr{X}{k}{l}{k}{l})
    \end{align*}
    and
    \begin{align*}
      ((f_{z}\otimes f_{z'})\circ \tilde \Delta \otimes
      \id)(\Gr{X}{k}{l}{m}{n}) &=  \delta_{k,m}\delta_{l,n}(f_{z} \otimes f_{z'} \otimes
      \id)\big((\Gr{X}{k}{l}{k}{l})_{13}
      (\Gr{X}{k}{l}{k}{l})_{23}\big) \\
      &=  \delta_{k,m}\delta_{l,n}(\Gru{F_{X}}{k}{l})^{z}  \cdot (\Gru{F_{X}}{k}{l})^{z'} \\
      &= (f_{z+z'} \otimes \id)(\Gr{X}{k}{l}{m}{n}).
    \end{align*}
    To conclude assertion (2), apply slice maps of the form $\id
    \otimes \omega_{\xi,\xi'}$.

   
    (3)  Write $\tilde \Delta^{(2)} = (\tilde
    \Delta \otimes \id)\circ \tilde \Delta = (\id \otimes \tilde
    \Delta) \circ \tilde \Delta$ and $\rho_{z,z'}:=(f_{z'} \otimes \id
    \otimes f_{z})\circ \tilde \Delta^{(2)}$. Then
    \begin{align*}
      (\rho_{z,z'} \otimes \id)(\Gr{X}{k}{l}{m}{n}) &= (f_{z'} \otimes
      \id \otimes f_{z} \otimes
      \id)((\Gr{X}{k}{l}{k}{l})_{14}(\Gr{X}{k}{l}{m}{n})_{24}(\Gr{X}{m}{n}{m}{n})_{34})
      \\
      &= (1 \otimes (\Gru{F_{X}}{k}{l})^{z'}) \Gr{X}{k}{l}{m}{n} (1
      \otimes (\Gru{F_{X}}{m}{n})^{z}).
    \end{align*}
    We take $z=z'=1$, use Theorem \ref{thm:rep-orthogonality}, where
    now $\alpha= \beta$ by our scaling of $F_{X}$, and obtain
    \begin{align*}
      (\phi \otimes \id \otimes
      \id)((\Gr{X}{k}{l}{m}{n})_{12}^{*}((\rho_{1,1} \otimes
      \id)(\Gr{X}{k}{l}{m}{n}))_{13}) &=\alpha^{-1}(\id \otimes
      \Gru{F_{X}}{k}{l}) (\id \otimes \Gru{(F_{X}^{-1})}{k}{l})
      \Sigma_{k,l,m,n} (\id \otimes
      \Gru{F_{X}}{m}{n}) \\
      &=\beta^{-1}(\Gru{F_{X}}{m}{n} \otimes \id) \Sigma_{k,l,m,n} \\
      &= (\phi \otimes \id \otimes
      \id)((\Gr{X}{k}{l}{m}{n})_{13}(\Gr{X}{k}{l}{m}{n})_{12}^{*}).
    \end{align*}
    To conclude assertion (3), applying slice maps of the form
    $\omega_{\xi,\xi'} \otimes \omega_{\eta,\eta'}$.

    (4) By Proposition \ref{prop:rep-f} and the calculation above,
    \begin{align*}
      (S^{2} \otimes \id)(\Gr{X}{k}{l}{m}{n}) &= (1
      \otimes\Gru{F_{X}}{k}{l})
      \Gr{X}{k}{l}{m}{n}(1 \otimes \Gru{F_{X}}{m}{n})^{-1} 
      =(\rho_{-1,1}  \otimes \id)(\Gr{X}{k}{l}{m}{n}).
    \end{align*}
     Assertion (4) follows again by applying slice maps.
    
     (5) The fact that $f_{z}$ is a character and
     that $f_{z}(\Gr{A}{k}{l}{m}{n})=0$ if $(k,l)\neq (m,n)$
     immediately implies the relation
     $f_{z}(\lambda_{l}\rho_{n})=\delta_{l,n}$  and the equality
     \begin{align*}
       (f_{-z} \otimes \id) (\Gr{X}{k}{l}{k}{l})=
       (\Gru{(F_{X})}{k}{l})^{-z} &= (f_{z} \otimes
       \id)(\Gr{X}{k}{l}{k}{l})^{-1} \\ &= (f_{z} \otimes
       \id)(\Gr{(X^{-1})}{k}{l}{k}{l}) = (f_{z} \circ S \otimes
       \id)(\Gr{X}{k}{l}{k}{l}).
     \end{align*}
Therefore, $f_{-z} = f_{z} \circ S$. Using the preceding calculation,
     the relation $(S \otimes \id)(\Gr{X}{k}{l}{k}{l}) =
     (\Gr{X}{k}{l}{k}{l})^{*}$ and positivity of $\Gru{F_{X}}{k}{l}$,
     we conclude
     \begin{align*}
       (\ast \circ f_{z} \circ \ast \otimes
       \id)(\Gr{X}{k}{l}{k}{l})
&=       (f_{z} \otimes
       \id)((\Gr{X}{k}{l}{k}{l})^{*})^{*} \\
& = (f_{-z} \otimes \id)(\Gr{X}{k}{l}{k}{l})^{*} 
 =
((\Gru{F_{X}}{k}{l})^{-z})^{*} 
=       (\Gru{F_{X}}{k}{l})^{-\overline{z}} = (f_{-\overline{z}}
\otimes \id)(\Gr{X}{k}{l}{k}{l}),
     \end{align*}
whence $\ast \circ f_{z} \circ \ast = f_{-\overline{z}}$.
\end{proof}

%  consider
%   the family
%   \begin{align*}
%     (\Gru{F}{m}{n})^{\top} \circ  \overline{\Sigma_{m,n,k,l}} &=
% \phi((\Gr{X}{k}{l}{m}{n})_{12}^{*}(\Gr{X}{k}{l}{m}{n})_{13})^{\top}
% \circ  \overline{\Sigma_{m,n,k,l}} \\
%  &=
%  \phi((\Gr{X}{k}{l}{m}{n})_{12}^{*\circ (\id \otimes
%   \top)}(\overline{\Sigma_{k,l,k,l}})_{23} (\Gr{X}{k}{l}{m}{n})^{\id \otimes
%   \top}_{12})  \\
% &=\phi((\Gr{\overline{X}}{l}{k}{n}{m})_{12}(\overline{\Sigma_{k,l,k,l}})_{23}
%  (\Gr{\overline{X}}{l}{k}{n}{m})_{12}).
% \end{align*} \fxnote{Treat $X^{\id \otimes \top}$}
% By Lemma \ref{lem:rep-average}, this family is a morphism from to
% $(\overline{\mathcal{H}}\otimes \overline{K},\overline{X}^{-*} \otimes
% \id_{\overline{K}})$  to
% $(\overline{\mathcal{H}}\otimes \overline{K},\overline{X} \otimes
% \id_{\overline{K}})$ and hence of the form
% $(\Gru{\overline{F_{X}}}{n}{m})^{-1} \otimes T$ with $T \in
% \mathcal{B}(\overline{K})$ not depending on $m,n$.

% Thus,
% \begin{align*}
%  (\id \otimes R)  \circ \Sigma_{k,l,m,n} =   \Gru{F}{m}{n} =
%  \Sigma_{k,l,m,n} \circ (\Gru{\overline{F_{X}}}{n}{m})^{-\top} \otimes T^{\top})
% \end{align*}

  
%   We may assume $(k,l,m,n)=(p,q,r,s)$ because otherwise both sides of
%   the equation that we want to prove vanish.

%   Applying Lemma \ref{lem:rep-average} to the corepresentation $X$ and
%   the family $\Gru{T}{p}{q}=
%   \delta_{p,k}\delta_{q,l} |\eta\rangle\langle\xi|$, we obtain an
%   endomorphism $\check{T}$ of $(\mathcal{H},X)$ which
%   necessarily has the form $\check{T}=\lambda(\xi,\eta) \id$ for some
%   $\lambda(\xi,\eta) \in \C$. Inserting the definition of
%   $\check{T}$, we find
%   \begin{align}\nonumber
%     \phi(b^{*}a) &= \phi\big((\id \otimes
%     \omega_{\eta',\eta})((\Gr{X}{k}{l}{m}{n})^{*}) \cdot (\id \otimes
%     \omega_{\xi,\xi'})(\Gr{X}{k}{l}{m}{n})\big) \\  &= (\phi \otimes
%     \id)\left(\langle\eta'|_{2} \Gr{(X^{-1})}{m}{n}{k}{l}(1 \otimes
%       |\eta\rangle\langle \xi|)
%       \Gr{X}{k}{l}{m}{n}|\xi'\rangle_{2}\right) 
%     = \langle \eta'|_{2} \Gru{\check{T}}{m}{n}|\xi'\rangle_{2} =
%     \lambda(\xi,\eta) \langle\eta'|\xi'\rangle. \label{eq:rep-orthogonal-1}
%   \end{align}
%   Next, we apply Lemma \ref{lem:rep-average} to the corepresentations
%   $\overline{X}$ and $\overline{X}^{-*}$ and the family
%   $\Gru{R}{p}{q}=\delta_{p,m}\delta_{q,n}|\overline{
%     \xi'}\rangle\langle\overline{\eta'}|$, and obtain a morphism
%   $\hat{R}$ from $\overline{X}^{-*}$ to $\overline{X}$ which
%   necessarily has the form $\hat{R}=\mu(\eta',\xi')\overline{F_{X}}$
%   for some $\mu(\eta',\xi') \in \C$. Using the relation
%   \begin{align*}
%     a &= (\id \otimes
%     \omega_{\overline{\xi},\overline{\xi'}})(\Gr{\overline{X}}{l}{k}{n}{m})^{*},
%     & b&= (\id \otimes
%     \omega_{\overline{\eta},\overline{\eta'}})(\Gr{\overline{X}}{k}{l}{m}{n})^{*}
%   \end{align*}
%   and the definition of $\hat{R}$, we obtain
%   \begin{align}
%     \phi(b^{*}a) &= (\phi \otimes \id)\left(\langle
%       \overline{\eta}|_{2} \Gr{\overline{X}}{k}{l}{m}{n}(1 \otimes
%       |\overline{\eta}'\rangle\langle \overline{\xi'}|)
%       \Gr{(\overline{X}^{*})}{m}{n}{k}{l} \right) \nonumber \\
%     &=\langle \overline{\eta}| \Gru{\hat{R}}{k}{l}
%     |\overline{\xi}\rangle = \langle
%     \overline{\eta}|\Gru{\overline{F_{X}}}{k}{l}\overline{\xi}\rangle
%     \mu(\eta',\xi'). \label{eq:rep-orthogonal-2}
%   \end{align}
%   We choose a basis $(\zeta_{i})_{i}$ for
%   $\bigoplus_{k}\Gru{\mathcal{H}}{k}{l}$ and calculate
%   \begin{align*}
%  \langle
%     \eta'|\xi'\rangle &=  (\phi \otimes
%     \id)(\langle\eta'|_{2}(\lambda_{l}\rho_{n} \otimes
%     \id_{\Gru{\mathcal{H}}{m}{n}})|\xi'\rangle_{2}) \\
%  &=
%     \sum_{k} (\phi \otimes \id)\left(\langle\eta'|_{2}
%       \Gr{(X^{-1})}{m}{n}{k}{l}
%       \Gr{X}{k}{l}{m}{n}|\xi'\rangle_{2}\right)
%     \\ &=    \sum_{i} \lambda(\zeta_{i},\zeta_{i}) \langle
%     \eta'|\xi'\rangle 
%     \\
%     &=\sum_{i} \langle
%     \overline{\zeta_{i}}| \Gru{\overline{F_{X}}}{}{l}\overline{\zeta_{i}}\rangle
%     \mu(\eta',\xi') 
% \\ &    = \mu(\eta',\xi') \cdot \sum_{k} \Tr(\Gru{F_{(X)}}{k}{l}),
%   \end{align*}
% where $\Gru{\overline{F_{X}}}{}{l}=\bigoplus_{k}
% \Gru{\overline{F_{X}}}{k}{l}$. Inserting this relation into
% \eqref{eq:rep-orthogonal-2}, we finally obtain the assertion.


%%% Local Variables: 
%%% mode: latex
%%% TeX-master: "dyn-suq-main"
%%% End: 
