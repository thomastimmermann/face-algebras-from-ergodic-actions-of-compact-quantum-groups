\section{Representation theory}

In this section, the representation theory of partial compact quantum groups is investigated. As the situation is quite similar to the case already studied by Hayashi \cite{Hay1}, we do not always provide fully written out proofs, but only draw attention to those parts of the theory which need modification.

In what follows, the homogeneous component $A(K) = \eGr{A}{k}{l}{m}{n}$ of a partial bialgebra will now be mainly written as $A(K) = \Gr{A}{k}{l}{m}{n}$. 

\subsection{Corepresentations of partial Hopf algebras}
% A \emph{locally finite corepresentation}
% of $(A,\Delta)$ consists of a row- and column-finite $I^{2}$-graded
% Hilbert space $\mathcal{H}=\bigoplus_{k,l} \Gru{\mathcal{H}}{k}{l}$
% together with a family $X=(\Gr{X}{k}{l}{m}{n})_{k,l,m,n}$ of elements $\Gr{X}{k}{l}{m}{n} \in \Gr{A}{k}{l}{m}{n}
% \otimes
% \mathcal{B}(\Gru{\mathcal{H}}{m}{n},\Gru{\mathcal{H}}{k}{l})$ such that
% \begin{enumerate}
% \item $(\tilde \Delta \otimes \id)(\Gr{X}{k}{l}{m}{n}) = \sum_{p,q}
%   \left(\Gr{X}{k}{l}{p}{q}\right)_{13}\left(\Gr{X}{p}{q}{m}{n}\right)_{23}$ and
% \item $X$ is invertible in the sense that there exist elements $\Gr{Z}{k}{l}{m}{n} \in \Gr{A}{n}{m}{l}{k}
%   \otimes
%   \mathcal{B}(\Gru{\mathcal{H}}{m}{n},\Gru{\mathcal{H}}{k}{l})$ such
%   that
%   \begin{align*}
%     \sum_{k} \Gr{Z}{m}{n'}{k}{l} \Gr{X}{k}{l}{m}{n} &= \delta_{n,n'}
%     \lambda_{l} \rho_{n} \otimes \id_{\Gru{\mathcal{H}}{m}{n}}, &
%     \sum_{n} \Gr{X}{k}{l}{m}{n} \Gr{Z}{m}{n}{k'}{l} &= \delta_{k,k'}
%     \lambda_{k} \rho_{m} \otimes \id_{\Gru{\mathcal{H}}{k}{l}}.
%   \end{align*}
% \end{enumerate}
% Here, the infinite sum in 1.\ makes sense in the multiplier algebra,
% and the sums in 2.\ are finite. The family
% $Z=(\Gr{Z}{k}{l}{m}{n})_{k,l,m,n}$ in 2.\ is uniquely
% determined by $X$ and will be denoted by $X^{-1}$.

Let $\mathscr{A}$ be an $I$-partial bialgebra. We denote $\Hom_\C(V,W)$ for the vector space of linear maps between two vector spaces.

\begin{Def} Let $\mathscr{A}$ be an $I$-partial bialgebra, and let $V=\bigoplus_{k,l} \Gru{V}{k}{l}$ be an $I^{2}$-graded vector space.
A \emph{corepresentation}  $X=(\Gr{X}{k}{l}{m}{n})_{k,l,m,n}$ of $\mathscr{A}$ on $V$ consists of a family of elements 
  $\Gr{X}{k}{l}{m}{n} \in \Gr{A}{k}{l}{m}{n} \otimes
  \Hom_\C(\Gru{V}{m}{n},\Gru{V}{k}{l})$
  satisfying
  \begin{align} \label{eq:rep-delta1}
    (\Delta_{pq} \otimes \id)(\Gr{X}{k}{l}{m}{n}) =
    \Big{(}\Gr{X}{k}{l}{p}{q}\Big{)}_{13}\Big{(}\Gr{X}{p}{q}{m}{n}\Big{)}_{23}
  \end{align}
and   \[(\epsilon \otimes
    \id)(\Gr{X}{k}{l}{m}{n})=\delta_{k,m}\delta_{l,n}\id_{\Gru{V}{k}{l}}\] 
for all possible indices.
\end{Def} % Drop s.f. and say we only work with s.f. coreps.?

We use here the standard leg numbering notation, e.g. $a_{12} = a\otimes 1$.

We will sometimes for convenience consider $\Gr{X}{k}{l}{m}{n}$ as a map in $\Hom_\C(\Gru{V}{m}{n},\Gr{A}{k}{l}{m}{n}\otimes \Gru{V}{k}{l})$. % Or better to keep leg numbering notation because confusing otherwise? Or use $1\otimes $ notation?

\begin{Def}\label{DefMorphism} A \emph{morphism} $T$ between two corepresentations
$(V,X)$ and $(W,Y)$ of a partial bialgebra $\mathscr{A}$ is a family of linear maps
\[\Gru{T}{k}{l} \in
\Hom_\C(\Gru{V}{k}{l},\Gru{W}{k}{l})\]
satisfying \[(1 \otimes \Gru{T}{k}{l})\Gr{X}{k}{l}{m}{n} = \Gr{Y}{k}{l}{m}{n}(1 \otimes \Gru{T}{m}{n}).\]
% We call $T$ \emph{bounded} when $\oplus_{k,l} \Gru{T}{k}{l}$ is bounded.
\end{Def}

\begin{Def} Let $(V,X)$ be a
corepresentation of a partial bialgebra $\mathscr{A}$. A family of subspaces
\[\Gru{W}{k}{l} \subseteq \Gru{V}{k}{l}\]
is called \emph{invariant} if \[(1\otimes \Gru{Q}{k}{l})\Gr{X}{k}{l}{m}{n}(1 \otimes \Gru{P}{m}{n}) =0,\]
where $\Gru{P}{m}{n}:\Gru{W}{m}{n}\to\Gru{V}{m}{n}$ is the inclusion map and $\Gru{Q}{k}{l}:\Gru{V}{k}{l}\to\Gru{V}{k}{l}/\Gru{W}{k}{l}$
denotes the quotient map.
\end{Def}

The following analogue of Schur's Lemma holds.

\begin{Lem} Let $T$ be a morphism
of corepresentations $(V,X)$ and
$(W,Y)$. Then $ \bigoplus_{k,l} \ker \Gru{T}{k}{l}$ and
$\bigoplus_{k,l} \img \Gru{T}{k}{l}$ are invariant subspaces of
$V$ and $W$, respectively.
\end{Lem} 

In particular, if $(V,X)$ and $(W,Y)$ are irreducible, then a morphism $T$ from $V$ to $W$ either has all $\Gru{T}{k}{l}=0$ or all
$\Gru{T}{k}{l}$ isomorphisms.

\begin{Def} Let $\mathscr{A}$ be a partial bialgebra. We denote by $\Corep(\mathscr{A})$ the category of corepresentations of $\mathscr{A}$ with morphisms as in Definition \ref{DefMorphism}.
\end{Def}

Clearly $\mathscr{A}$ is a $\C$-linear category. We will be interested in a full subcategory of it. 

\begin{Def} An $I^2$-graded vector space $V=\bigoplus_{k,l} \Gru{V}{k}{l}$ is called \emph{separately finitely supported (sfs)} %better name? 
if $\Gru{V}{k}{l} = \{0\}$ for almost all $k$ (resp. almost all $l$) when $l$ (resp. $k$) is fixed. 

If $\mathscr{A}$ is an $I$-partial bialgebra, a corepresentation $X$ will be called sfs if its carrier $I^2$-graded vector space is sfs. 
\end{Def}

We will denote by $\Corep_{\sfs}(\mathscr{A})$ the full category of $\Corep(\mathscr{A})$ consisting of sfs corepresentations.

The next lemma provides one with a tensor product structure on the category $\Corep_{\sfs}(\mathscr{A})$.

\begin{Lem} Let $X$ and $Y$ be two sfs copresentations on respective $I^2$-graded vector spaces $V$ and $W$. Put \[\Gru{(V\iitimes W)}{k}{m} = \oplus_l \left(\Gru{V}{k}{l} \otimes \Gru{W}{l}{m}\right),\] and consider their direct sum $V\iitimes W$ with its natural $I^2$-grading. Then $V\iitimes W$ is sfs, the sum \[ \Gr{(X\Circt Y)}{k}{p}{m}{q} := \sum_{l,n} \left(\Gr{X}{k}{l}{m}{n}\right)_{12}\left(\Gr{Y}{l}{p}{n}{q}\right)_{13}\] has only finitely many non-zero terms, and \[\Gr{(X\Circt Y)}{k}{p}{m}{q}\in \Gr{A}{k}{p}{m}{q}\otimes \Hom_\C(\Gru{(V\iitimes W)}{m}{q},\Gru{(V\iitimes W)}{k}{p})\] defines a corepresentation.
\end{Lem} 

\begin{proof} It is trivially checked that $V\iitimes W$ is sfs. Then for $k,m$ fixed, there are only finitely many $l,n$ for which $\Gr{X}{k}{l}{m}{n}$ is non-zero. This shows that the sum defining $\Gr{(X\Circt Y)}{k}{p}{m}{q}$ is in fact finite. Using the natural inclusions \begin{eqnarray*} &&\hspace{-2cm} \Gr{A}{k}{p}{m}{q}\otimes \Hom_\C(\Gru{V}{m}{n},\Gru{V}{k}{l})\otimes  \Hom_\C(\Gru{W}{n}{q},\Gru{W}{l}{p}) \\ && \subseteq \Gr{A}{k}{p}{m}{q} \otimes \Hom_\C(\Gru{V}{m}{n}\otimes \Gru{W}{n}{q},\Gru{V}{k}{l}\otimes \Gru{W}{l}{p})\\ &&\subseteq \Gr{A}{k}{p}{m}{q}\otimes \Hom_\C(\Gru{(V\iitimes W)}{m}{q},\Gru{(V\iitimes W)}{k}{p}),\end{eqnarray*} we see that $\Gr{(X\Circt Y)}{k}{p}{m}{q}$ is indeed an element of $\Gr{A}{k}{p}{m}{q}\otimes \Hom_\C(\Gru{(V\iitimes W)}{m}{q},\Gru{(V\iitimes W)}{k}{p})$. 

The fact that $\Gr{(X\Circt Y)}{k}{p}{m}{q}$ is a corepresentation then follows straightforwardly from the multiplicativity of $\Delta$ and the weak multiplicativity of $\epsilon$.
\end{proof} 

\begin{Cor} The category $\Corep_{\sfs}(\mathscr{A})$ with the above tensor product $\Circt$ forms a tensor category.
\end{Cor}
\begin{proof} Clear. We only remark that the unit object of the category is given by the graded vector space $\C_I$ with components $\delta_{kl}\C$ and associated corepresentation \[\Gr{X}{k}{l}{m}{n} = \delta_{m,n}\delta_{k,l} \UnitC{k}{m}.\] 
\end{proof} 

%  Antipode duality.% To continue.

The following lemma shows that when $\mathscr{A}$ is a partial \emph{Hopf} algebra, an sfs corepresentation is invertible.

% Local finite-dimensionality not needed?
\begin{Lem} \label{lemma:rep-corep} Let $\mathscr{A}$ be a partial Hopf algebra, and let $X$ be an sfs corepresentation of $\mathscr{A}$ on an $I^2$-graded vector space $V$.
  Then the elements % More logical to give the $Z$ the grading of the component in $A$ to which they belong
  \[\Gr{Z}{k}{l}{m}{n} =(S \otimes
  \id)(\Gr{X}{n}{m}{l}{k}) \in
    \Gr{A}{k}{l}{m}{n} \otimes
    \Hom_\C(\Gru{V}{l}{k},\Gru{V}{n}{m})\] are the unique elements such  that
  \begin{align}\label{eq:rep-delta2}
    \sum_{k} \Gr{Z}{l}{k}{n'}{m} \Gr{X}{k}{l}{m}{n} &= \delta_{n,n'}
    \UnitC{l}{n} \otimes \id_{\Gru{V}{m}{n}}, &
    \sum_{n} \Gr{X}{k}{l}{m}{n} \Gr{Z}{l}{k'}{n}{m} &= \delta_{k,k'}
    \UnitC{k}{m}\otimes \id_{\Gru{V}{k}{l}}.
  \end{align}
   In particular, they satisfy
  \begin{align*}
    (\Delta_{pq} \otimes \id)(\Gr{Z}{k}{l}{m}{n}) &=
    \left(\Gr{Z}{p}{q}{m}{n}\right)_{23}\left(\Gr{Z}{p}{q}{k}{l}\right)_{13}.
  \end{align*}
\end{Lem}
Note that the sums in \eqref{eq:rep-delta2} are finite because $\mathcal{H}$ is sfs. 
\begin{proof}
 Let  $\Gr{Z}{k}{l}{m}{n}=(S \otimes \id)(\Gr{X}{k}{l}{m}{n})$, and write the multiplication of $A$ as $M_A$. Then the antipode axiom implies
   \begin{align*}
     \sum_{k} \Gr{Z}{l}{k}{n'}{m} \Gr{X}{k}{l}{m}{n} &= \sum_{k}
     (S\otimes \id)(\Gr{X}{m}{n'}{k}{l}) \Gr{X}{k}{l}{m}{n}
     \\
     &=\sum_{k} (M_A \otimes \id)(S\otimes \id \otimes \id) (
     (\Gr{X}{m}{n
       '}{k}{l})_{13}(\Gr{X}{k}{l}{m}{n})_{23}) \\
     &= (M_A\otimes \id)(S\otimes \id \otimes \id)\sum_{k}((\Delta_{kl}
     \otimes \id)(\Gr{X}{m}{n'}{m}{n}) \\
     &= \delta_{n,n'} \UnitC{l}{n}\otimes (\epsilon\otimes \id)(\Gr{X}{m}{n}{m}{n}) \\
     &= \delta_{n,n'}\UnitC{l}{n} \otimes 
     \id_{\Gru{V}{m}{n}}.
   \end{align*} The second identity follows
   similarly.

 Uniqueness follows immediately from the identities  \eqref{eq:rep-delta2}.
\end{proof}

\begin{Def} Let $\mathscr{A}$ and $X$ be as in Lemma \ref{lemma:rep-corep}. The element $Z$, when it exists, will be denoted $X^{-1}$.
\end{Def} 

 \begin{Rem}\label{RemMorRep} % To see if this has to be modified.
 Using \eqref{eq:rep-delta2}, one finds that a collection of linear maps $\Gru{T}{k}{l} \in
\mathcal{L}(\Gru{V}{k}{l},\Gru{W}{k}{l})$ forms a morphism between two sfs corepresentations
$(V,X)$ and $(W,Y)$  if
and only if
  \begin{align} 
    \sum_{m} \Gr{(Y^{-1})}{n}{m}{l'}{k}(1 \otimes
    \Gru{T}{m}{n})\Gr{X}{m}{n}{k}{l} &= \delta_{l,l'}
    \UnitC{n}{l} \otimes \Gru{T}{k}{l}, \\
    \sum_{n} \Gr{Y}{k'}{l}{m}{n}(1 \otimes
    \Gru{T}{m}{n})\Gr{(X^{-1})}{l}{k}{n}{m} &= \delta_{k,k'}
    \UnitC{m}{k} \otimes \Gru{T}{k}{l}.
 \end{align}
For example, if the first equation holds, then
\begin{align*}
  \Gr{Y}{m}{n}{k}{l} (1 \otimes \Gru{T}{k}{l}) &=
  \sum_{m}  \Gr{Y}{m}{n}{k}{l} \Gr{(Y^{-1})}{n}{m}{l}{k}(1 \otimes
  \Gru{T}{m}{n})\Gr{X}{m}{n}{k}{l} = (\UnitC{k}{m}\otimes \Gru{T}{m}{n})\Gr{X}{m}{n}{k}{l}.
\end{align*}
Conversely, if $T$ is a morphism, then
\begin{align*}
  \sum_{m} \Gr{(Y^{-1})}{n}{m}{l'}{k}(1 \otimes
  \Gru{T}{m}{n})\Gr{X}{m}{n}{k}{l} &= \sum_{m}
  \Gr{(Y^{-1})}{n}{m}{l'}{k} \Gr{Y}{m}{n}{k}{l}(1 \otimes
  \Gru{T}{k}{l}) = \delta_{l',l} \UnitC{n}{l} \otimes
  \Gru{T}{k}{l}.
\end{align*}
\end{Rem}
%In the following we will write $X^{-1}:=Z$.

To actually have a tensor category with duality, we need something stronger than the sfs condition.

\begin{Def} An $I^2$-graded vector space $V$ is called \emph{seperately finite dimensional (sfd)} if $\oplus_l \Gru{V}{k}{l}$ (resp. $\oplus_l \Gru{V}{k}{l}$) is finite dimensional for $k$ (resp. $l$) fixed. Correspondingly, we talk of an sfd corepresentation of a partial bialgebra $\mathscr{A}$, and we then denote by $\Corep_{\sfd}(\mathscr{A})$ the full subcategory of $\Corep_{\sfs}(\mathscr{A})$ consisting of sfd representations. 
\end{Def} 

One easily sees that $\Corep_{\sfd}(\mathscr{A})$ is closed under $\Circt$. 

\begin{Lem} Let $\mathscr{A}$ be a partial Hopf algebra. Then $\Corep_{\sfd}(\mathscr{A})$ is a tensor category with left duality. 
\end{Lem} 

\begin{proof} Let $X$ be an sfd corepresentation on a bigraded vector space $V$. Put \[\Gru{(V^*)}{k}{l} = (\Gru{V}{k}{l})^*,\] and let $V^*$ denote their direct sum bigraded vector space. Using the natural contravariant identification \[\Hom_\C(\Gru{V}{l}{k},\Gru{V}{n}{m})\cong \Hom_\C(\Gru{(V^*)}{m}{n},\Gru{(V^*)}{k}{l}),\] we see (by means of Lemma \ref{lemma:rep-corep}) that $X^{-1}$ gets transformed into a corepresentation $X^d$ on $V^*$. 

We claim that $X^d$ is the left dual of $X$. To see this, consider the evaluation maps \[\Gru{T}{k}{m}: \Gru{(V^*\iitimes V)}{k}{m} \supseteq (\Gru{V}{l}{k})^*\otimes \Gru{V}{l}{m}\rightarrow \delta_{k,m} \C = \Gru{\C_I}{k}{m}.\] Then from Lemma \ref{lemma:rep-corep}, we obtain \begin{eqnarray*} (1\otimes \Gru{T}{k}{p})\Big{(} \sum_{l,n} \big{(} \Gr{(X^d)}{k}{l}{m}{n}\big{)}_{12} \big{(}\Gr{X}{l}{p}{n}{q}\big{)}_{13}\Big{)} &=& \delta_{p,k}(\id\otimes \Gru{T}{m}{m}) \left( \sum_{l,n}  \Gr{(X^{-1})}{k}{l}{m}{n}\Gr{X}{l}{k}{n}{q}\right)_{13} \\ &=& \delta_{p,k}\UnitC{k}{q}\otimes \Gru{T}{m}{q}.\end{eqnarray*} Hence the $\Gru{T}{k}{l}$ define an intertwiner between $V^*\iitimes V$ and $\C_I$. Similarly, the maps  \[\Gru{R}{k}{k}: \Gru{\C_I}{k}{k} = \C\rightarrow \Gru{(V\iitimes V^*)}{k}{k},\quad 1\mapsto \sum_{l,i} \Gru{v_i}{k}{l}\otimes \Gru{\omega_i}{l}{k},\] where the $\{\Gru{v_i}{k}{l}\mid i\}$ and $\{\Gru{\omega_i}{l}{k} \mid i\}$ form a dual basis of $\Gru{V}{k}{l}$, can be shown to form an intertwiner. It is then easy to check that $T$ and $R$ make $X^d$ into the left dual of $X$. % More info?
\end{proof}

Let us now enhance our partial Hopf algebras to partial compact quantum groups. One then considers corepresentations on sfd bigraded \emph{Hilbert spaces} such that the inverse of the corepresentation coincides with its adjoint. More precisely, we have the following definition. We denote $B(\Hsp,\mathcal{G})$ for the linear space of bounded morphisms between Hilbert spaces.

%\paragraph{Unitarity of corepresentations}

\begin{Def} Let $\mathscr{A}$ define a partial compact quantum group. We call an sfd corepresentation $(\mathcal{H},X)$ on an sfd $I^2$-graded Hilbert space $\mathcal{H}$
\emph{unitary} if \[\Gr{(X^{-1})}{k}{l}{m}{n}=(\Gr{X}{l}{k}{n}{m})^{*}\quad \textrm{in }\Gr{A}{k}{l}{m}{n}\otimes B(\Gru{\Hsp}{l}{k},\Gru{\Hsp}{n}{m}).\]  
\end{Def} 
\begin{Rem} \begin{enumerate} \item In the Hilbert space setting, it is more natural to let $\Hsp$ be the \emph{closed} (instead of the purely algebraic) direct sum of all (finite-dimensional) $\Gru{\Hsp}{k}{l}$. This does not change the notion of corepresentation, which had a local definition.
\item Concerning morphisms, we will say a collection of $\Gru{T}{k}{l}$ defines a \emph{bounded} intertwiner or morphism if the total operator $T= \oplus \Gru{T}{k}{l}$ is bounded. We will denote by $\Corep_{\sfd,u}(\mathscr{A})$ the category of unitary sfd corepresentations with arbitrary morphisms, and $\Corep_{\sfd,u}^{\infty}(\mathscr{A})$ for the category with bounded morphisms.
% Give a more prominent place, or check later if it is actually worthwhile to make this distinction.
\end{enumerate}
\end{Rem}

Our aim now is to show that every irreducible sfd corepresentation is
equivalent to a unitary one. We show this by embedding the corepresentation into a restriction of the
regular corepresentation.

\begin{Exa} \label{exa:rep-regular}
  Let $\mathscr{A}$ define a partial compact quantum group with normalized positive invariant functional $\phi$. 

  Let $\Gru{\mathcal{H}}{m}{n} \subseteq \bigoplus_{k,l}
  \Gr{A}{k}{l}{m}{n}$ be finite dimensional subspaces satisfying
\begin{align*}
  \Delta^{\op}_{pq}(\Gru{\mathcal{H}}{m}{n}) \subseteq
 \Gr{A}{p}{q}{m}{n} \otimes \Gru{\mathcal{H}}{p}{q}.
\end{align*}
for all indices.
Equip each $\Gru{\mathcal{H}}{k}{l}$ with the scalar product $\langle
a|b\rangle:=\phi(a^{*}b)$. By Lemma \ref{LemFaith}, these are finite-dimensional Hilbert spaces. Take the Hilbert space direct sum
$\mathcal{H}:=\bigoplus_{k,l} \Gru{\mathcal{H}}{k}{l}$.
Define \[ \Gr{V}{k}{l}{m}{n} \in \Hom_\C(\Gru{\mathcal{H}}{m}{n}, \Gr{A}{k}{l}{m}{n} \otimes \Gru{\mathcal{H}}{k}{l}) \cong \Gr{A}{k}{l}{m}{n} \otimes 
\mathcal{B}(\Gru{\mathcal{H}}{m}{n},\Gru{\mathcal{H}}{k}{l})\]  by the
equation
\begin{align*}
  \Gr{V}{k}{l}{m}{n}a &= \Delta^{\co}_{kl}(a).
\end{align*}
%where $\Gr{V}{k}{l}{m}{n}|a\rangle_{2}$  denotes the application of
%the second leg of
%$\Gr{V}{k}{l}{m}{n}$  to $a \in \Gru{\mathcal{H}}{m}{n}$. 

\begin{Lem} The couple $(\mathcal{H},V)$ defines a unitary corepresentation. 
\end{Lem} 
\begin{proof} It is clear that $V$ defines a corepresentation. It then suffices to prove that \begin{equation}\label{EqUnit} \sum_{k} (\Gr{V}{k}{l}{m}{n'})^* \Gr{V}{k}{l}{m}{n} = \delta_{n,n'}\UnitC{l}{n}\otimes \id_{\Gru{\Hsp}{m}{n}}.\end{equation} Take $a\in \Gru{\Hsp}{m}{n}$ and $b\in \Gru{\Hsp}{m}{n'}$. Then writing \[\Lambda(a): \C\rightarrow \Gru{\Hsp}{m}{n}, \quad 1\mapsto a,\] and similarly for $b$, we compute \begin{eqnarray*}(1\otimes \Lambda(b)^*)\left(\sum_{k} (\Gr{V}{k}{l}{m}{n'})^* \Gr{V}{k}{l}{m}{n}\right) (1\otimes \Lambda(a)) &=& \sum_k (\id\otimes \phi)(\Delta_{kl}^{\op}(b)^*\Delta_{kl}^{\op}(a))\\ &=&  \sum_k (\id\otimes \phi)(\Delta_{lk}^{\op}(b^*)\Delta_{kl}^{\op}(a)) \\ &=& (\id\otimes \phi)(\Delta_{ll}^{\op}(b^*a))\\ &=&  (\phi\otimes \id)(\Delta_{ll}(b^*a)) \\ &=& \phi(b^*a)\UnitC{l}{n} \\&=& \delta_{n',n} \UnitC{l}{n} \otimes \Lambda(b)^*\Lambda(a).\end{eqnarray*} 
This proves \eqref{EqUnit}.
\end{proof} 

We will call $(\mathcal{H},V)$  the 
\emph{sfd restriction of the regular corepresentation}
determined by the family $(\Gru{\mathcal{H}}{k}{l})_{k,l}$. 
\fxnote{proof this}
\end{Exa}

In the following, we will use the notation \[\omega_{\xi,\eta}:B(\Hsp,\mathcal{G})\rightarrow \C,\quad x\mapsto \langle \xi,x\eta\rangle,\quad \xi\in \mathcal{G},\eta\in \Hsp.\]

\begin{Lem} \label{lem:rep-morphism-regular}
  Let $\mathscr{A}$ define a partial compact quantum group. Let $(\mathcal{H},X)$ be an sfd corepresentation on a Hilbert space, and let $\xi \in
  \Gru{\mathcal{H}}{k}{l}$. Then the family of finite-dimensional
  subspaces
  \begin{align*}
   \Gru{\mathcal{K}}{m}{n}  &=  \{ (\id \otimes
   \omega_{\xi,\eta})(\Gr{X}{k}{l}{m}{n}) : \eta \in
   \Gru{\mathcal{H}}{m} {n}\} \subseteq \Gr{A}{k}{l}{m}{n}
  \end{align*}% Should explain notation
  defines an sfd restriction $(\mathcal{K},V)$ of the regular corepresentation, and the family of maps
  \begin{align*}
    \Gru{T_{(\xi)}}{m}{n} \colon \Gru{\mathcal{H}}{m}{n} \to
    \Gru{\mathcal{K}}{m}{n}, \ \eta \mapsto (\id \otimes
    \omega_{\xi,\eta})(\Gr{X}{k}{l}{m}{n}),
  \end{align*}
  is a morphism from $(\mathcal{H},X)$ to $(\mathcal{K},V)$ in $\Corep_{\sfd,u}(\mathscr{A})$. % Should $\xi$ then also be in a subscript for $\mathcal{K}$?
\end{Lem}
Note that the family $(\Gru{\mathcal{K}}{m}{n})_{m,n}$ is sfd because $(\Gru{\mathcal{H}}{m}{n})_{m,n}$
is. 
\begin{proof} Both assertions  follow from the fact
  that for all $\eta \in \Gru{\mathcal{H}}{p}{q}$,
\begin{eqnarray*}
\Delta^{\op}_{pq}(\Gru{T_{(\xi)}}{m}{n}(\eta)) &=&  \Delta_{pq}^{\op}\big((\id \otimes
  \omega_{\xi,\eta})(\Gr{X}{k}{l}{m}{n})\big)\\
 &=&(\id \otimes \id \otimes \omega_{\xi,\eta})\big(
  (\Gr{X}{k}{l}{p}{q})_{23}(\Gr{X}{p}{q}{m}{n})_{13})\big) \\ &=&(1
  \otimes \Gru{T_{(\xi)}}{p}{q})\Gr{X}{p}{q}{m}{n} \eta.
\end{eqnarray*}
\end{proof}
\begin{Prop}  \label{prop:rep-unitarisable}
  Let $\mathscr{A}$ define a partial compact quantum group. Then every irreducible sfd corepresentation on a Hilbert space
  is equivalent to a unitary sfd corepresentation.
\end{Prop}
\begin{proof}
  Let $(\mathcal{H},X)$ be an irreducible sfd
  corepresentation. Then for some $k,l$ and $\xi \in
  \Gru{\mathcal{H}}{k}{l}$,  the operator  $T_{(\xi)}$  defined in
  Lemma \ref{lem:rep-morphism-regular} has to be non-zero and hence,
  by Schur's Lemma, injective. Thus, it forms an equivalence between
  $(\mathcal{H},X)$ and a sub-corepresentation of an sfd
  restriction of the regular corepresentation, which is unitary by
  Example \ref{exa:rep-regular}.
\end{proof}

Our next goal is to obtain the analogue of Schur's orthogonality
relations for matrix coefficients of corepresentations.

\begin{Def} Let $\mathscr{A}$ define a partial compact quantum group. The space of \emph{matrix coefficients} $\mathcal{C}(X)$ of an sfd corepresentation $(\mathcal{H},X)$ is the sum of
the subspaces
\begin{align*}
  \Gr{\mathcal{C}(X)}{k}{l}{m}{n} &= \span \left\{ (\id \otimes
  \omega_{\xi,\eta})(\Gr{X}{k}{l}{m}{n}) \mid \xi \in
  \Gru{\mathcal{H}}{k}{l}, \eta \in \Gru{\mathcal{H}}{m}{n} \right\}
\subseteq \Gr{A}{k}{l}{m}{n}.
\end{align*}
\end{Def}

\begin{Lem} Every sfd unitary corepresentation $(X,\Hsp)$ of $\mathscr{A}$ decomposes into a direct sum of irreducible sfd unitary corepresentations.
\end{Lem} 
\begin{proof} From the unarity assumption, it follows immediately that an invariant subspace of $\Hsp$ also has an invariant orthogonal complement. Hence irreducibility and indecomposability of unitary corepresentations coincide. More generally, one deduces that the bounded self-interwiners of $\Hsp$ form a (von Neumann) $^*$-algebra.

% Trivial rep should maybe be introduced in a more conspicuous place
Let us now first show that the trivial representation decomposes into irreducibles. Let $I$ be the object set of $\mathscr{A}$, and say $k\sim l$ if $\UnitC{k}{l}\neq 0$. Then $\sim$ is an equivalence relation: as \[\Delta_{ll}(\UnitC{k}{m}) = \UnitC{k}{l}\otimes \UnitC{l}{m},\] the relation $\sim$ is transitive. As $S(\UnitC{k}{l}) = \UnitC{l}{k}$, we have that $\sim$ is symmetric. And as $\varepsilon(\UnitC{k}{k})=1$, we also have that $\sim$ is reflexive. 

Let then $I = \sqcup_{\alpha\in \mathscr{I}} I_{\alpha}$ be a labeled partition associated to $\sim$. Define $\C_{I_{\alpha}}\subseteq \C_I$ as the linear span of the homogeneous components with index in $\alpha$. It is clear then that the $\C_{I_{\alpha}}$ are invariant and irreducible.

Consider now a general corepresentation $(X,\Hsp)$. Let $\Grd{\Hsp}{\alpha}{\beta}$ be the closed linear span of the homogeneous components with index in $\alpha\times \beta$. As we can identify \[\Grd{\Hsp}{\alpha}{\beta} \cong \C_{I_{\alpha}}\,\Circt\, \Hsp\,\Circt\, \C_{I_{\beta}},\] we see that $\Grd{\Hsp}{\alpha}{\beta}$ is an invariant subspace of $\Hsp$. Hence we may as well suppose that $\Hsp = \Grd{\Hsp}{\alpha}{\beta}$. 

But let then $T$ be a bounded self-intertwiner of $\Hsp$. Then from the two equations in Remark \ref{RemMorRep}, we see that $T\rightarrow \Gru{T}{k}{l}$ is injective for any choice of $k\in \alpha,l\in \beta$. It follows that the algebra of self-intertwiners of $\Hsp$ is finite-dimensional. We then immediately conclude that $\Hsp$ is a finite direct sum of irreducible invariant subspaces.
\end{proof} 

\begin{Prop} \label{prop:rep-weak-pw}
  Assume that $\mathscr{A}$ defines a partial compact quantum group. Then the total algebra $A$ is the sum of the matrix coefficients of
 irreducible unitary sfd corepresentations.
\end{Prop}
\begin{proof}
  Let $a \in \Gr{A}{k}{l}{m}{n}$. Then
  $\Delta^{\co}_{pq}(a) \in
  \Gr{A}{p}{q}{m}{n} \otimes \Gr{A}{k}{l}{p}{q}$, and the subspace
    \begin{align*}
    \Gru{\mathcal{H}}{p}{q} &:= \span \{ (\omega \otimes \id)(
    \Delta^{\co}_{pq}(a)) : \omega \in \Hom_{\C}(\Gr{A}{p}{q}{m}{n},\C) \} \subseteq
    \Gr{A}{k}{l}{p}{q}
  \end{align*}
  has finite dimension. Since $a$ is fixed, the
  $(\Gru{\mathcal{H}}{p}{q})_{p,q}$ are in fact sfd. Using
  co-associativity, one checks that this family defines an sfd restriction $(\mathcal{H},V)$ of the regular
  corepresentation. Evidently, $a \in \mathcal{C}(V)$. Decomposing
  $(\mathcal{H},V)$, we find that
  $a$ is contained in the sum of matrix coefficients of unitary
  irreducible corepresentations.
\end{proof}
The key to the orthogonality relations is the following averaging procedure.
\begin{Lem} \label{lem:rep-average}
  Let $\mathscr{A}$ define a partial compact quantum group, and let  $\phi$ be any invariant functional for $\mathscr{A}$. Let
  $(\mathcal{H},X)$ and $(\mathcal{K},Y)$ be sfd corepresentations of $\mathscr{A}$ and let $T$ be
  a family of operators $\Gru{T}{k}{l} \in
  \mathcal{B}(\Gru{\mathcal{H}}{k}{l},\Gru{\mathcal{K}}{k}{l})$.
  
  Then for any fixed $n$, the family of linear maps
  \begin{align*}
    \Gru{\check T_n}{k}{l} &:= \sum_{m} (\phi \otimes
    \id)(\Gr{(Y^{-1})}{n}{m}{l}{k}(1\otimes
    \Gru{T}{m}{n})\Gr{X}{m}{n}{k}{l})
   \end{align*} 
   
  define a morphism $\check T_n$ from $(\mathcal{H},X)$ to $(\mathcal{K},Y)$ in $\Corep_{\sfd,u}(\mathscr{A})$.

     Similarly, for fixed $m$, the      
    
  \begin{align*} \Gru{\hat T_m}{k}{l} &:= \sum_{n} (\phi \otimes
    \id)(\Gr{Y}{k}{l}{m}{n}(1\otimes
    \Gru{T}{m}{n})\Gr{(X^{-1})}{l}{k}{n}{m})
  \end{align*}
  define a morphism from $(\mathcal{H},X)$ to $(\mathcal{K},Y)$. % Slightly changed the averaging so that I do not need a restriction of finite support on $T$.
\end{Lem} 
\begin{proof}
 Using Remark \ref{RemMorRep}, the assertion concerning the $\check{T}$ follows
  from the calculation
  \begin{align*}
    &\sum_{m} \Gr{(Y^{-1})}{n}{m}{l'}{k}(1 \otimes
    \Gru{\check{T}_q}{m}{n})\Gr{X}{m}{n}{k}{l} = \\
    &=\sum_{m,p} (\phi \otimes \id \otimes
    \id)\left(\left(\Gr{(Y^{-1})}{n}{m}{l'}{k}\right)_{23}
      \left(\Gr{(Y^{-1})}{q}{p}{n}{m}\right)_{13}(1 \otimes 1 \otimes
      \Gru{T}{p}{q})\left(\Gr{X}{p}{q}{m}{n}\right)_{13}\left(\Gr{X}{m}{n}{k}{l}\right)_{23}\right)
    \\
    % &= \sum_{m,p,q}(\phi \otimes \id \otimes \id)\left((1 \otimes
    %   \lambda_{n} \otimes 1)(\tilde \Delta \otimes
    %   \id)\left(\Gr{(Y^{-1})}{k}{l'}{p}{q}\right)(\rho_{m} \otimes 1
    %   \otimes \Gru{T}{p}{q})(\tilde \Delta \otimes
    %   \id)\left(\Gr{(X}{p}{q}{k}{l}\right)(1 \otimes\lambda_{n}
    %   \otimes 1)\right) \\
    &= \sum_{m,p} (\left((\phi \otimes \id)
    \circ \Delta_{mn}\right) \otimes \id)\left(\Gr{(Y^{-1})}{q}{p}{l'}{k}(1
      \otimes \Gru{T}{p}{q})\Gr{X}{p}{q}{k}{l}\right) 
    \\
    &= \delta_{l',l}\UnitC{n}{l}\otimes \left(\sum_{p,q} (\phi \otimes \id)\left(\Gr{(Y^{-1})}{q}{p}{l}{k}(1 
      \otimes \Gru{T}{p}{q})\Gr{X}{p}{q}{k}{l}\right)\right)\\
  &= \delta_{l,l'}\UnitC{n}{l} \otimes \Gru{\check{T}_q}{k}{l},
  \end{align*}
  where we used the relation $\phi(\Gr{A}{r}{s}{l'}{l})=0$ for $l'\neq l$
  for the last equality. 
  
  A similar calculation proves the assertion
  concerning the $\hat{T}$. 
\end{proof}

% Choose a representative family of unitary irreducible locally finite
% corepresentations
% $(\Grd{\mathcal{H}}{(\alpha)}{},{_{(\alpha)}X})_{\alpha}$ and a basis
% $(\Gr{\zeta}{k}{l}{(\alpha)}{i})_{i}$ for each
% $\Gr{\mathcal{H}}{k}{l}{(\alpha)}{}$, and let
% \begin{align*}
%   (\Gr{(u_{\alpha})}{k}{l}{m}{n}){i,j} &:= (\id \otimes
%   \Gr{\omega}{k}{l}{(\alpha)}{i,j})( )
% \end{align*}

The first part of the orthogonality relations concerns matrix
coefficients of inequivalent irreducible corepresentations.
\begin{Prop} \label{prop:rep-orthogonality-1}
  Let $(\mathcal{H},X)$ and $(\mathcal{K},Y)$ be inequivalent unitary
  irreducible sfd corepresentations, and let
  $\phi$ be an invariant functional for $(A,\Delta)$.  Then
  \[\phi(S(b)a) =\phi(b^{*}a) = \phi(bS(a))=\phi(ba^{*})=0\] for all
  $a\in \mathcal{C}(X), b \in \mathcal{C}(Y)$.
\end{Prop}
\begin{proof}
  Let $a=(\id \otimes \omega_{\xi,\xi'})(\Gr{X}{k}{l}{m}{n})$ and
  $b=(\id \otimes \omega_{\eta,\eta'})(\Gr{Y}{p}{q}{r}{s})$, where
  $\xi \in \Gru{\mathcal{H}}{k}{l}, \xi' \in \Gru{\mathcal{H}}{m}{n}$
  and $\eta \in \Gru{\mathcal{K}}{p}{q}, \eta' \in
  \Gru{\mathcal{K}}{r}{s}$. 
  
  If $(p,q,r,s) \neq (m,n,k,l)$, then clearly $\phi(S(b)a) = 0 = \phi(bS(a))$.
  
   If $(p,q,r,s) = (m,n,k,l)$,  then Lemma \ref{lem:rep-average}, applied to the 
  family $\Gru{T}{p}{q}= \delta_{p,k}\delta_{q,l}
  |\eta'\rangle\langle\xi|$, yields  morphisms $\check{T}_l,\hat{T}_k$
  from $(\mathcal{H},X)$ to $(\mathcal{K},Y)$ which necessarily are
  $0$. Inserting the definition of $\check{T}_l$, we find
  \begin{align*}
    \phi(S(b)a) &= \phi\big((S \otimes
    \omega_{\eta,\eta'})(\Gr{Y}{m}{n}{k}{l}) \cdot (\id \otimes
    \omega_{\xi,\xi'})(\Gr{X}{k}{l}{m}{n})\big) \\ &= (\phi \otimes
    \id)\left((1\otimes \langle\eta|) \Gr{(Y^{-1})}{l}{k}{n}{m}(1 \otimes
      |\eta'\rangle\langle \xi|)
      \Gr{X}{k}{l}{m}{n}(1\otimes |\xi'\rangle)\right) 
   \\ &= (1\otimes \langle \eta|) \Gru{\check{T}_l}{m}{n}(1\otimes |\xi'\rangle) = 0.
  \end{align*}% Resort notation on using leg notation, physics bra-ket, etc.
  
  A similar calculation involving $\hat{T}$ shows that
  $\phi(bS(a))=0$.  
  
  Using the relation $X^{*}=X^{-1}=(S\otimes
  \id)(X)$ and $Y^{*}=(S\otimes \id)(Y)$, we conclude
  $\phi(b^{*}a)=\phi(ba^{*})=0$.
\end{proof}

The second part of the orthogonality relations concerns inner products
as above but with $a,b\in \mathcal{C}(X)$ for some irreducible
corepresentation  $X$. It involves the conjugate corepresentation,
which is defined as follows.

Given Hilbert spaces $\Hsp,\mathcal{K}$, we denote by $\overline{\Hsp},\overline{\mathcal{K}}$
the conjugate Hilbert spaces, by $T \mapsto \overline{T}$ the
canonical conjugate-linear isomorphism $\mathcal{B}(\Hsp,\mathcal{K}) \to
\mathcal{B}(\overline{\Hsp},\overline{\mathcal{K}})$, and by $T \mapsto
T^{\top}:=\overline{T}^{*}$ the linear anti-isomorphism % Or contravariant isomorphism?
$\mathcal{B}(\Hsp,\mathcal{K}) \to \mathcal{B}(\overline{\mathcal{K}},\overline{\Hsp})$.

\begin{Lem} \label{lemma:rep-functors}Let $\Corep_{\sfd,\Hilb}(\mathscr{A})$ denote the category of (not necessarily unitary) corepresentations of $\mathscr{A}$ on sfd bigraded Hilbert spaces. 
   Then on $\Corep_{\sfd,\Hilb}(\mathscr{A})$ there exist
  \begin{enumerate}
  \item a covariant antilinear functor $(\mathcal{H},X) \mapsto
    (\overline{\mathcal{H}},\overline{X})$ and $T \mapsto
    \overline{T}$, where
    \begin{align*}
      \Gru{\overline{\mathcal{H}}}{k}{l} &= \overline{\Gru{\mathcal{H}}{l}{k}},
      & \Gr{\overline{X}}{k}{l}{m}{n} &= (\Gr{X}{l}{k}{n}{m})^{(*
        \otimes \overline{(\ \cdot \ ) })}
      =((\Gr{X}{l}{k}{n}{m})^{*})^{\id \otimes \top}, &
      \Gru{\overline{T}}{k}{l} &= \overline{\Gru{T}{l}{k}};
    \end{align*}
  \item a contravariant linear functor $(\mathcal{H},X) \mapsto
    (\overline{\mathcal{H}},X^{S\otimes \top})$ and
    $T\mapsto T^{\top}$, where 
    \begin{align*}\Gru{\overline{\mathcal{H}}}{k}{l} &= \overline{\Gru{\mathcal{H}}{l}{k}},&
     & \Gr{(X^{S\otimes \top})}{k}{l}{m}{n} = (S \otimes (\ \cdot \
      )^{\top})(\Gr{X}{n}{m}{l}{k}), & \Gru{(T^{\top})}{k}{l}
      &=(\Gru{T}{l}{k})^{\top};
    \end{align*}
  \item a covariant linear functor  $(\mathcal{H},X) \mapsto (\mathcal{H},X^{S^{2}\otimes \id})$
    and $T\mapsto T$, where the grading is unchanged and \[\Gr{(X^{S^{2}\otimes \id})}{k}{l}{m}{n}=(S^{2} \otimes
  \id)(\Gr{X}{k}{l}{m}{n}).\]
  \end{enumerate}
  If $(\mathcal{H},X)$ is unitary, then $\overline{X}=X^{S\otimes \top}$.\end{Lem}
\begin{proof}
The first assertion follows immediately from the fact that $\Delta_{rs}(a^*)=\Delta_{sr}(a)^*$ and the $^*$-compatibility of $\epsilon$. The second assertion follows from the fact that $\Delta_{pq}\circ S  = (S\otimes S)\circ \Delta_{qp}^{\op}$ and $\epsilon\circ S = \epsilon$. The final functor is just the square of the second functor. 

The fact that $\overline{X}=X^{S\otimes \top}$ for $X$ unitary is just by definition. % yes?
 % Some of these properties should be mentioned before.
  
  
  %\begin{align*}
  %  (\Delta_{pq} \otimes \id)( \Gr{\overline{X}}{k}{l}{m}{n}) &=
    %( \Gr{\overline{X}}{k}{l}{p}{q})_{13}
    %( \Gr{\overline{X}}{p}{q}{m}{n})_{23}, \\
    %(\epsilon \otimes \id)(\Gr{\overline{X}}{k}{l}{m}{n}) &=
   % \overline{(\epsilon \otimes \id)(\Gr{X}{l}{k}{n}{m})}
   % = \delta_{l,n}\delta_{k,m}
   % \overline{\id_{\Gru{\mathcal{H}}{l}{k}}}
   % = \delta_{l,n}\delta_{k,m} \id_{\Gru{\overline{\mathcal{H}}}{k}{l}}.
 % \end{align*}
  
\end{proof}
We call $(\overline{\mathcal{H}},\overline{X})$ the \emph{conjugate} of 
$(\mathcal{H},X)$.   

\begin{Prop} \label{prop:rep-f}
  Let $\mathscr{A}$ define a partial compact quantum group, and let $\phi$ be a positive normalized
  invariant functional. Let $(\mathcal{H},X)$ be a unitary
  irreducible sfd corepresentation. 
  \begin{enumerate}
  \item The conjugate $\overline{\mathcal{H}}$ with the family
    $\Gr{(\overline{X}^{-*})}{k}{l}{m}{n}:=\big(\Gr{(\overline{X}^{-1})}{l}{k}{n}{m}\big)^{*}$
    form an sfd corepresentation, and there
    exists an invertible, positive morphism $\overline{F_{X}}$ from
    $(\overline{\mathcal{H}},\overline{X})$ to
    $(\overline{\mathcal{H}},\overline{X}^{-*})$. % Better notation?
  \item The family
    $\Gru{F_{X}}{k}{l}:=\overline{\Gru{\overline{F_{X}}}{l}{k}}$ is an
    invertible, positive operator implementing a morphism from $(\mathcal{H},X)$ to
    $(\mathcal{H},X^{S^{2} \otimes \id})$.
  \end{enumerate}
\end{Prop}
\begin{proof}
(1) By Proposition \ref{prop:rep-unitarisable},
$(\overline{\mathcal{H}},\overline{X})$ is equivalent to a unitary
corepresentation, that is, there exists a family of invertible operators
$\Gru{T}{k}{l} \in \mathcal{B}(\Gru{\overline{\mathcal{H}}}{k}{l})$
such that the family 
\begin{align*}
\Gr{Z}{k}{l}{m}{n}:= (1 \otimes
\Gru{T}{k}{l})\Gr{\overline{X}}{k}{l}{m}{n}(1 \otimes
\Gru{T}{m}{n})^{-1} 
\end{align*}
 is a unitary corepresentation. The relation
 $\Gr{(Z^{-1})}{n}{m}{l}{k}=\big(\Gr{Z}{m}{n}{k}{l})^{*}$ then implies
 \begin{align*}
   \Gr{Z}{k}{l}{m}{n} = (1 \otimes
   (\Gru{T}{k}{l})^{-1})^{*}\big(\Gr{(\overline{X}^{-1})}{l}{k}{n}{m}\big)^{*}(1
   \otimes \Gru{T}{m}{n})^{*}
 \end{align*}
 and hence the family 
 $\Gr{(\overline{X}^{-*})}{k}{l}{m}{n}:=\big(\Gr{(\overline{X}^{-1})}{l}{k}{n}{m}\big)^{*}$
 is an irreducible sfd corepresentation. The maps
 $\Gru{\overline{F}_{X}}{k}{l}:=(\Gru{T}{k}{l})^{*}\Gru{T}{k}{l} \in
  \mathcal{B}(\Gru{\overline{\mathcal{H}}}{k}{l})$
then form an isomorphism from $(\overline{\mathcal{H}},\overline{X})$ to
$(\overline{\mathcal{H}},\overline{X}^{-*})$.   

(2) The morphism $T$  from $(\overline{\mathcal{H}},\overline{X})$  to
$(\overline{\mathcal{H}},Z)$ yields morphisms $\overline{T}$ from
$(\mathcal{H},X)$ to $(\mathcal{H},\overline{Z})$ and $T^{\top}$ from $(\mathcal{H},Z^{S\otimes\top})$ to
$(\mathcal{H},\overline{X}^{S \otimes \top})$. Since $X$ and $Z$ are
unitary, $\overline{Z}=Z^{S\otimes \top}$ and  $\overline{X}^{S \otimes
  \top} = X^{S^{2} \otimes \id}$. Thus $T^{\top}\overline{T} =
\overline{T^{*}T}$ is a morphism from $(\mathcal{H},X)$ to
$(\mathcal{H},X^{S^{2}\otimes \id})$.
\end{proof}


\begin{Theorem} \label{thm:rep-orthogonality} Let $\mathscr{A}$ define a partial
compact quantum group. 
  Let $\phi$ be a positive normalized invariant functional. Let $(\mathcal{H},X)$ be a unitary irreducible sfd corepresentation of $\mathscr{A}$, and let $F_{X}$ be a
  non-zero morphism from $(\mathcal{H},X)$ to $(\mathcal{H},X^{S^{2}
  \otimes \id})$.
  \begin{enumerate}
  \item The numbers $\alpha:=\sum_{k} \Tr(\Gru{(F_{X}^{-1})}{k}{l})$
    and $\beta:=\sum_{n} \Tr(\Gru{F_{X}}{m}{n})$ do not depend on $l$
    or $n$.
  \item  For all $k,l,m,n$,
    \begin{align*}
      (\phi \otimes \id)((\Gr{X}{k}{l}{m}{n})^{*}\Gr{X}{k}{l}{m}{n})
      &=\alpha^{-1}{\Tr(\Gru{(F^{-1}_{X})}{k}{l})} \cdot
      \id_{\Gru{\mathcal{H}}{m}{n}}, \\
      (\phi \otimes \id)(\Gr{X}{k}{l}{m}{n}(\Gr{X}{k}{l}{m}{n})^{*})
      &=\beta^{-1}{\Tr( \Gru{(F_{X})}{m}{n})} \cdot
      \id_{\Gru{\mathcal{H}}{k}{l}}.
    \end{align*}
  \item Denote by $\Sigma_{klmn}$ the flip map $\Gru{\mathcal{H}}{k}{l}
    \otimes \Gru{\mathcal{H}}{m}{n} \to \Gru{\mathcal{H}}{m}{n}
    \otimes \Gru{\mathcal{H}}{k}{l}$. Then
 \begin{align*}
   (\phi \otimes \id \otimes
   \id)((\Gr{X}{k}{l}{m}{n})_{12}^{*}(\Gr{X}{k}{l}{m}{n})_{13}) &=
   \alpha^{-1}
   (\id_{\Gru{\mathcal{H}}{m}{n}} \otimes \Gru{(F_{X}^{-1})}{k}{l})
   \circ \Sigma_{klmn}, \\
   (\phi \otimes \id \otimes
   \id)((\Gr{X}{k}{l}{m}{n})_{13}(\Gr{X}{k}{l}{m}{n})_{12}^{*}) &= \beta^{-1} (\Gru{F_{X}}{m}{n}
   \otimes \id_{\Gru{\mathcal{H}}{k}{l}}) \circ \Sigma_{klmn}.
 \end{align*}
\end{enumerate}
  \end{Theorem}
\begin{proof}
  We prove the assertions and equations involving $\alpha$ in (1), (2)
  and (3)  simultaneously; the assertions involving $\beta$  follow similarly.

  %As above, we denote by $\Sigma_{p,q,r,s}$ the flip
  %$\Gru{\mathcal{H}}{p}{q} \otimes \Gru{\mathcal{H}}{r}{s} \to
  %\Gru{\mathcal{H}}{r}{s} \otimes \Gru{\mathcal{H}}{p}{q}$.  
  Consider
  the following endomorphism $F_{m,n,k,l}$ of $\Gru{\mathcal{H}}{m}{n}\otimes \Gru{\mathcal{H}}{k}{l}$, 
  \begin{align*}
    F_{m,n,k,l}
    &:=(\phi \otimes \id \otimes \id)((\Gr{X}{k}{l}{m}{n})_{12}^{*}(\Gr{X}{k}{l}{m}{n})_{13})
    \circ \Sigma_{mnkl} \\ &= (\phi \otimes \id \otimes
    \id)\left((\Gr{(X^{-1})}{l}{k}{n}{m})_{12}
      \Sigma_{klkl,23}(\Gr{X}{k}{l}{m}{n})_{12}\right).
  \end{align*}
  By applying Lemma \ref{lem:rep-average} with respect to the flip map $\Sigma_{klkl}$, we see that the family $(F_{m,n,k,l})_{m,n}$ is
  an endomorphism of $(\mathcal{H} \otimes \Gru{\mathcal{H}}{k}{l}, X_{12})$ and hence
  \begin{align}
    F_{m,n,k,l} &= \id_{\Gru{\mathcal{H}}{m}{n}} \otimes \Gru{R}{k}{l} \label{eq:rep-orthogonal-1}
  \end{align}
  with some $\Gru{R}{k}{l} \in \mathcal{B}(\Gru{\mathcal{H}}{k}{l})$ not
  depending on $m,n$. % In using irreducibility, do we not miss any subtlety in allowing non-bounded morphisms?
  On the other hand, 
  \begin{align*}
    F_{m,n,k,l} &= (\phi \otimes \id \otimes \id)((S \otimes
    \id)(\Gr{X}{m}{n}{k}{l})_{12}(\Gr{X}{k}{l}{m}{n})_{13})
    \circ \Sigma_{mnkl} \\
    &= (\phi\circ S^{-1} \otimes \id \otimes \id)\left(((S \otimes
      \id)(\Gr{X}{k}{l}{m}{n}))_{13}
      ((S^{2} \otimes \id)(\Gr{X}{m}{n}{k}{l}))_{12}\right)     \circ \Sigma_{mnkl}\\
    &= (\phi\circ S^{-1} \otimes \id \otimes \id)\left(((S \otimes
      \id)(\Gr{X}{k}{l}{m}{n}))_{13} (\Sigma_{mnmn})_{23} ((S^{2}
      \otimes \id)(\Gr{X}{m}{n}{k}{l}))_{13}\right).
  \end{align*}
  Since $\phi\circ S^{-1}$ is an invariant functional for
  $\mathscr{A}$, we can again apply Lemma \ref{lem:rep-average} and
  find that the family $(F_{m,n,k,l})_{k,l}$ is a morphism \[(F_{m,n,k,l})_{k,l}:
  (\Gru{\mathcal{H}}{m}{n} \otimes \mathcal{H}, (X^{S^{2} \otimes
    \id})_{13})\rightarrow (\Gru{\mathcal{H}}{m}{n} \otimes \mathcal{H},
  X_{13}).\] Therefore,
  \begin{align}
    F_{m,n,k,l} &= \Gru{T}{m}{n} \otimes (\Gru{F_{X}}{k}{l})^{-1} \label{eq:rep-orthogonal-2}
  \end{align}
  with some $\Gru{T}{m}{n} \in \mathcal{B}(\Gru{\mathcal{H}}{m}{n})$
  not depending on $k,l$. Combining \eqref{eq:rep-orthogonal-1} and
  \eqref{eq:rep-orthogonal-2}, we conclude that, for some $\lambda\in \C$, \[F_{m,n,k,l} = \lambda
  (\id_{\Gru{\mathcal{H}}{m}{n}} \otimes (\Gru{F_{X}}{k}{l})^{-1})\]
  
  Choose a basis
  $(\zeta_{i})_{i}$ for $\Gru{\mathcal{H}}{k}{l}$. Then
  \begin{align*}
    \lambda \cdot \id_{\Gru{\mathcal{H}}{m}{n}} \cdot
    \Tr((\Gru{F_{X}}{k}{l})^{-1}) &= \sum_{i} (\id \otimes
    \omega_{\zeta_{i},\zeta_{i}})(F_{m,n,k,l}) = (\phi \otimes
    \id)((\Gr{X}{k}{l}{m}{n})^{*} \Gr{X}{k}{l}{m}{n}).
  \end{align*}
  Taking $n=l$ and summing over $k$, the relations $\sum_{k}
  (\Gr{X}{k}{l}{m}{n})^{*} \Gr{X}{k}{l}{m}{n} = \UnitC{l}{n}
  \otimes \id_{\Gru{\mathcal{H}}{m}{n}}$ and
  $\phi(\UnitC{l}{l})=1$ give
\begin{align*}
\lambda \cdot  \sum_{k} \Tr((\Gru{F_{X}}{k}{l})^{-1}) = 1.
\end{align*}
Now all assertions in (1)--(3) concerning $\alpha$ follow.
 % the second formula, we use the first formula for the
 %    opposite of $(A,\Delta)$. For this opposite, $\phi$ still is a
 %    faithful, positive, normalized invariant functional and
 %    $(\mathcal{H},X)$ still is a unitary irreducible locally finite
 %    corepresentation, but the antipode $S$ gets replaced by $S^{-1}$
 %    and therefore $F_{X}$ gets replaced by $F_{X}^{-1}$.
\end{proof}
\begin{Cor}\label{CorOrth}
  Assume that $\mathscr{A}$ defines a partial compact quantum group, and let $\phi$ be a normalized positive  invariant functional. Let $(\mathcal{H},X)$ be a unitary
  irreducible sfd corepresentation of $\mathscr{A}$, let
  $F_{X}$ be a non-zero morphism from $(\mathcal{H},X)$ to
  $(\mathcal{H},(S^{2} \otimes \id)(X))$, and let $a=(\id \otimes
  \omega_{\xi,\xi'})(\Gr{X}{k}{l}{m}{n})$ and $b=(\id \otimes
  \omega_{\eta,\eta'})(\Gr{X}{k}{l}{m}{n})$, where $\xi,\eta \in
  \Gru{\mathcal{H}}{k}{l}$ and $\xi',\eta' \in \Gru{\mathcal{H}}{m}{n}$.
  Then
\begin{align*}
  \phi(b^{*}a) &= \frac{\langle \eta'|\xi'\rangle \langle
    \xi|F^{-1}_{X}\eta\rangle}{\sum_{m}
    \Tr(\Gru{(F^{-1}_{X})}{m}{n})}, & \phi(ab^{*}) = \frac{\langle
    \eta'|F_{X}\xi'\rangle \langle \xi|\eta\rangle}{\sum_{n}
    \Tr(\Gru{F_{X}}{m}{n})}.
\end{align*}
\end{Cor}
\begin{proof}
  By Theorem \ref{thm:rep-orthogonality}, 
  \begin{align*}
    \phi(b^{*}a) &= (\phi \otimes \omega_{\eta',\eta} \otimes
    \omega_{\xi,\xi'})((\Gr{X}{k}{l}{m}{n})_{12}^{*}(\Gr{X}{k}{l}{m}{n})_{13})
    \\
    &= \frac{1}{\sum_{k} \Tr(\Gru{(F_{X}^{-1})}{k}{l})} 
    (\omega_{\eta',\eta} \otimes
    \omega_{\xi,\xi'})(   (
    \id_{\Gru{\mathcal{H}}{m}{n}} \otimes \Gru{(F_{X}^{-1})}{k}{l})
    \circ \Sigma_{k,l,m,n}). 
  \end{align*}
  The formula for $\phi(ab^{*})$ follows similarly or by considering
  the co-opposite of $\mathscr{A}$.
\end{proof}
\begin{Cor} \label{cor:rep-pw}
  Let $\mathscr{A}$ define a partial compact quantum group. Let
  $(\mathcal{H}_{\alpha},X_{\alpha})_{\alpha}$ be a representative
  family of all irreducible sfd corepresentations of
  $\mathscr{A}$. Then the map
  \begin{align*}
    \bigoplus_{\alpha} \bigoplus_{k,l,m,n}
    (\overline{\Gru{\mathcal{H}_{\alpha}}{k}{l}} \otimes
    \Gru{\mathcal{H}_{\alpha}}{m}{n}) \to A
  \end{align*}
  that sends $\overline{\xi} \otimes \eta \in
  \overline{\Gru{\mathcal{H}_{\alpha}}{k}{l}} \otimes
  \Gru{\mathcal{H}_{\alpha}}{m}{n}$ to $ (\id \otimes
  \omega_{\xi,\eta})(\Gr{(X_{\alpha})}{k}{l}{m}{n})$,
  is a linear isomorphism.
\end{Cor}
\begin{proof} This follows from Proposition \ref{prop:rep-weak-pw}, Proposition \ref{prop:rep-orthogonality-1} and Corollary \ref{CorOrth}.
\end{proof}

Suppose now $a\in \Gr{A}{k}{l}{m}{n}$ for some partial bialgebra $\mathscr{A}$. Then for $\omega \in \Hom_{\C}(A,\C)$, we can define
\begin{align*}
  \omega \aste{p,q} a
&:= (\id \otimes \omega) (\Delta_{pq}(a)), & a \aste{r,s}
\omega&:=(\omega \otimes \id)(\Delta_{rs}(a)).\end{align*} Clearly we can define
\begin{align*} \omega \aste{p,q} a \aste{r,s}
\omega'&:= (\omega \aste{p,q} a)\aste{r,s} \omega' = \omega \aste{p,q}(a \aste{r,s} \omega').\end{align*}
When $\omega$ has support on the $A(K)$ with $K_u=K_d$, we can write, for $a\in \Gr{A}{k}{l}{m}{n}$, \[\omega\ast a := \sum_{p,q} \omega\aste{p,q}a = \omega\aste{m,n}a,\quad  a\ast \omega = \sum_{r,s} a\aste{r,s}\omega = a\aste{k,l}\omega.\] 

We shall say that an entire function $f$ has \emph{exponential growth
  on the right half-plane} if there exist $C,d>0$ such that $|f(x+iy)|\leq
C\mathrm{e}^{dx}$  for all $x,y\in \R$ with $x>0$. 

\begin{Theorem} \label{thm:rep-characters}
  Assume that $\mathscr{A}$ defines a partial compact quantum group with positive normalized
  invariant functional $\phi$.  There exists a unique family of linear functionals
  $f_{z} \colon A\to \C$ such that
\begin{enumerate}[label={(\arabic*)}]
  \item $f_z$ vanishes on $A(K)$ when $K_u\neq K_d$.
  \item for each $a\in A$, the function $z\mapsto f_{z}(a)$ is entire
    and of exponential growth on the right half-plane.
  \item $f_{0} = \epsilon$ and $(f_{z} \otimes f_{z'}) \circ 
    \Delta= f_{z+z'}$ for all $z,z' \in \C$.
  \item $\phi(ab)=\phi(b(f_{1} \ast a \ast f_{1}))$ for all $a,b\in A$.
  \end{enumerate}
  This family furthermore satisfies
  \begin{enumerate}[label={(\arabic*)}]\setcounter{enumi}{4}
  \item $f_z(ab) = f_z(a)f_z(b)$ for $a\in A(K)$ and $b\in A(L)$ with $K_r = L_l$. 
  \item $S^{2}(a)=f_{-1} \ast a \ast f_{1}$ for all $a\in A$.
  \item $f_{z}(\UnitC{l}{n})=\delta_{l,n}$,  $f_{z} \circ S = f_{-z}$,
and    $f_{z}(a^*) = \overline{f_{-\overline{z}}(a)}$ for all $a\in A$.
\end{enumerate}
\end{Theorem}


Note that condition (3) is meaningful by condition (1).

\begin{proof}
  We first prove uniqueness.  Assume that $(f_{z})_{z}$ is a family of
  functionals satisfying (1)--(4).  Since $\phi$ is faithful, the map
  $\sigma\colon a \mapsto f_{1} \ast a \ast f_{1}$ is uniquely
  determined by $\phi$, and one easily sees that it is a homomorphism. Using
  (3), we find that $\epsilon \circ \sigma^n=f_{2n}$, which uniquely determines these functionals. Using (2) and the
  fact that every entire function of exponential growth on the right
  half-plane is uniquely determined by its values at $\N \subseteq \C$, we can conclude that the family $f_{z}$ is uniquely determined. Moreover, since the property (5) holds for $z = 2n$, we also conclude by the same argument as above that it holds for all $z\in \C$.

  Let us now prove existence.  By Corollary \ref{cor:rep-pw}, we can
  define for each $z\in \C$ a functional $f_{z} \colon A \to \C$ such
  that for every unitary irreducible sfd corepresentation
  $(\mathcal{H},X)$,
    \begin{align*}
      f_{z}((\id \otimes \omega_{\xi,\eta})(\Gr{X}{k}{l}{m}{n})) &=
      \delta_{k,m}\delta_{l,n} \cdot
      \omega_{\xi,\eta}((\Gru{F_{X}}{k}{l})^{z}) \quad \text{for all }
      \xi \in \Gru{\mathcal{H}}{k}{l},\eta \in
      \Gru{\mathcal{H}}{m}{n},
    \end{align*}
    or, equivalently,
    \begin{align*}
      (f_{z} \otimes \id)(\Gr{X}{k}{l}{m}{n}) =
      \delta_{k,m}\delta_{l,n} \cdot (\Gru{F_{X}}{k}{l})^{z},
    \end{align*}
    where $F_{X}$ is a non-zero positive operator implementing a morphism from $(\mathcal{H},X)$ to
    $(\mathcal{H}, (S^{2} \otimes \id)(X))$, scaled such that
    \begin{align*}
      \alpha_{X}:= \sum_{k} \Tr(\Gru{(F_{X}^{-1})}{k}{l}) = \sum_{n}
      \Tr(\Gru{F_{X}}{m}{n})
    \end{align*}
    for all $l,n$ (see Proposition \ref{prop:rep-f} and Theorem \ref{thm:rep-orthogonality}). By
    construction, (1) and (2) hold. We show that the $(f_{z})_{z}$ satisfy the
    assertions (3)--(7). 
    %We have already argued that (5) is satisfied
    %$f_{z}$ is a character. 
    Throughout the following arguments, let 
    $(\mathcal{H},X)$ be a unitary irreducible corepresentation
    $(\mathcal{H},X)$ and let $F_{X}$ be as above.

    We first prove property (3). This follows from the relations
    \begin{align*}
      (f_{0}  \otimes \id)(\Gr{X}{k}{l}{m}{n}) &=
      \delta_{k,m}\delta_{l,n} \id_{\Gru{\mathcal{H}}{k}{l}} =
      (\epsilon \otimes \id)(\Gr{X}{k}{l}{m}{n})
    \end{align*}
    and
    \begin{align*}
      (((f_{z}\otimes f_{z'})\circ \Delta) \otimes
      \id)(\Gr{X}{k}{l}{m}{n}) &=  \delta_{k,m}\delta_{l,n}(f_{z} \otimes f_{z'} \otimes
      \id)\big((\Gr{X}{k}{l}{k}{l})_{13}
      (\Gr{X}{k}{l}{k}{l})_{23}\big) \\
      &=  \delta_{k,m}\delta_{l,n}(\Gru{F_{X}}{k}{l})^{z}  \cdot (\Gru{F_{X}}{k}{l})^{z'} \\
      &= (f_{z+z'} \otimes \id)(\Gr{X}{k}{l}{m}{n}).
    \end{align*}
    Applying slice maps of the form $\id
    \otimes \omega_{\xi,\xi'}$ and invoking Theorem \ref{thm:rep-orthogonality}, this proves (3).

% Again? Check if this has already been used before   
    To prove (4), write again $ \Delta^{(2)} = (
    \Delta \otimes \id)\circ  \Delta = (\id \otimes 
    \Delta) \circ \Delta$, and put \[\theta_{z,z'}:=(f_{z'} \otimes \id
    \otimes f_{z})\circ  \Delta^{(2)}.\] Then
    \begin{align*}
      (\theta_{z,z'} \otimes \id)(\Gr{X}{k}{l}{m}{n}) &= (f_{z'} \otimes
      \id \otimes f_{z} \otimes
      \id)((\Gr{X}{k}{l}{k}{l})_{14}(\Gr{X}{k}{l}{m}{n})_{24}(\Gr{X}{m}{n}{m}{n})_{34})
      \\
      &= (1 \otimes (\Gru{F_{X}}{k}{l})^{z'}) \Gr{X}{k}{l}{m}{n} (1
      \otimes (\Gru{F_{X}}{m}{n})^{z}).
    \end{align*}
    We take $z=z'=1$, use Theorem \ref{thm:rep-orthogonality}, where
    now $\alpha= \beta$ by our scaling of $F_{X}$, and obtain
    \begin{eqnarray*}
     && \hspace{-2cm} (\phi \otimes \id \otimes
      \id)((\Gr{X}{k}{l}{m}{n})_{12}^{*}((\theta_{1,1} \otimes
      \id)(\Gr{X}{k}{l}{m}{n}))_{13})\\ && =\alpha^{-1}(\id \otimes
      \Gru{F_{X}}{k}{l}) (\id \otimes \Gru{(F_{X}^{-1})}{k}{l})
      \Sigma_{k,l,m,n} (\id \otimes
      \Gru{F_{X}}{m}{n}) \\
      &&=\beta^{-1}(\Gru{F_{X}}{m}{n} \otimes \id) \Sigma_{k,l,m,n} \\
      &&= (\phi \otimes \id \otimes
      \id)((\Gr{X}{k}{l}{m}{n})_{13}(\Gr{X}{k}{l}{m}{n})_{12}^{*}).
    \end{eqnarray*}
    To conclude the proof of assertion (4), apply again slice maps of the form
    $\omega_{\xi,\xi'} \otimes \omega_{\eta,\eta'}$.

We have then already argued that the property (5) automatically holds. To show the property (6), note that by Proposition \ref{prop:rep-f} and the calculation above,
    \begin{align*}
      (S^{2} \otimes \id)(\Gr{X}{k}{l}{m}{n}) &= (1
      \otimes\Gru{F_{X}}{k}{l})
      \Gr{X}{k}{l}{m}{n}(1 \otimes \Gru{F_{X}}{m}{n})^{-1} 
      =(\theta_{-1,1}  \otimes \id)(\Gr{X}{k}{l}{m}{n}).
    \end{align*}
     Assertion (6) follows again by applying slice maps.
    
     Finally, (1), (2) and (4)
     immediately imply the relation
     $f_{z}(\UnitC{k}{m})=\delta_{k,m}$. The concrete construction of $f_z$ combined with property (3), the identity \eqref{eq:rep-delta2} and the partial character property (5) gives the equality
     \begin{align*}
       (f_{-z} \otimes \id) (\Gr{X}{k}{l}{k}{l})=
       (\Gru{(F_{X})}{k}{l})^{-z} &=\left( (f_{z} \otimes
       \id)(\Gr{X}{k}{l}{k}{l})\right)^{-1} \\ &= (f_{z} \otimes
       \id)(\Gr{(X^{-1})}{l}{k}{l}{k}) = ((f_{z} \circ S) \otimes
       \id)(\Gr{X}{k}{l}{k}{l}).
     \end{align*}
Therefore, $f_{-z} = f_{z} \circ S$. Let us now write $\bar{f}_z(a) = \overline{f_z(a^*)}$. Using the preceding calculation,
     the relation $(S \otimes \id)(\Gr{X}{k}{l}{k}{l}) =
     (\Gr{X}{k}{l}{k}{l})^{*}$ and positivity of $\Gru{F_{X}}{k}{l}$,
     we conclude
     \begin{align*}
       (\bar{f}_z \otimes
       \id)(\Gr{X}{k}{l}{k}{l})
&=       \left((f_{z} \otimes
       \id)((\Gr{X}{k}{l}{k}{l})^{*})\right)^{*} \\
& = \left((f_{-z} \otimes \id)(\Gr{X}{k}{l}{k}{l})\right)^{*} 
 =
((\Gru{F_{X}}{k}{l})^{-z})^{*} 
=       (\Gru{F_{X}}{k}{l})^{-\overline{z}} = (f_{-\overline{z}}
\otimes \id)(\Gr{X}{k}{l}{k}{l}),
     \end{align*}
whence $\bar{f}_z = f_{-\overline{z}}$.
\end{proof}
\begin{Cor} \label{cor:rep-characters}
  Let $\mathscr{G}$ be an partial compact quantum group with
  underlying total algebra $A$ and define $\theta_{z,z'}
  \colon A \to A$ by $a \mapsto f_{z} \ast a \ast f_{z'}$ for each
  $z,z' \in \C$, where the functionals $f_{z}$ are as in Theorem
  \ref{thm:rep-characters}. Then for all $z,z',w,w'\in \C$, the
  following conditions hold:
  \begin{enumerate}
  \item $\theta_{z,z'}$ is an algebra automorphism and preserves
    each subspace $A(K)$; in particular,
    $\theta_{z,z'}(\lambda_{k}\rho_{m}) = \lambda_{k}\rho_{m}$ for all
    $k,m\in I$;
  \item $\theta_{z,z'} \circ * = * \circ
    \theta_{-\overline{z},-\overline{z'}}$; in particular,
    $\theta_{it,is}$ is a $*$-automorphism for each $t,s\in \R$;
  \item $\theta_{z,z'}\circ \theta_{w,w'} = \theta_{z+w,z'+w'}$;
  \item $ (\theta_{w,z'} \otimes \theta_{z,-w}) \circ \Delta = \Delta
    \circ \theta_{z,z'}$, $\epsilon \circ \theta_{z,z'} = f_{z+z'}$,
    $\theta_{z,z'} \circ S = S \circ \theta_{-z',-z}$ and
    $\phi \circ \theta_{z,z'} = \phi$;
  \item for every linear map $\omega \colon A \to \C$ and every $a\in
    A$, the map $(z,z') \mapsto \omega(\theta_{z,z'}(a))$ is entire.
  \end{enumerate}
\end{Cor}
\begin{proof}
  All of this follows easily from Theorem \ref{thm:rep-characters}.
\end{proof}
Using the two-parameter group $\theta$, we define the \emph{modular
  automorphism group} $\sigma$, the \emph{scaling group} $\tau$   and
the \emph{unitary antipode} of a partial compact quantum group $A$ by
\begin{align*}
  \sigma_{z} &:=\theta_{iz,iz}, & \tau_{z} &:=\theta_{iz,-iz}, & R&:=S
  \circ \tau_{i/2}.
\end{align*}
Using Corollary \ref{cor:rep-characters}, one verifies that
$\sigma,\tau,R$ share all the main relations known for locally compact
quantum groups and measured quantum groupoids, for example, $\sigma$
and $\tau$ are complex one-parameter groups of algebra automorphisms
of $A$, the map $R$ is a $*$-anti-automorphism,  $\tau_{t}$ and
$\sigma_{t}$ are $*$-automorphisms for all $t\in \R$, the family  $\tau$ commutes with
$\sigma$ and with $R$ in the obvious sense, 
  \begin{gather}
    \begin{aligned} \label{eq:modular}
      \phi\circ \sigma_{z} &= \phi \circ \tau_{z} = \phi \circ R =
      \phi, & \phi(ab) &= \phi(b\sigma_{-i}(a)),
    \end{aligned}
\\ \label{eq:scaling-modular-delta}
    \begin{aligned} 
    \Delta \circ \tau_{z} &= (\tau_{z} \otimes \tau_{z}) \circ \Delta,
    & (\tau_{z} \otimes \sigma_{z}) \circ \Delta &= \Delta \circ
    \sigma_{z} = (\sigma_{-z} \otimes \tau_{z}) \circ \Delta,      
  \end{aligned} \\
  \begin{aligned} \label{eq:unitary-antipode}
    R^{2} &= \id_{A}, & \Delta \circ R &= (R \otimes R) \circ
    \Delta^{\op}.
  \end{aligned}
  \end{gather}




%  consider
%   the family
%   \begin{align*}
%     (\Gru{F}{m}{n})^{\top} \circ  \overline{\Sigma_{m,n,k,l}} &=
% \phi((\Gr{X}{k}{l}{m}{n})_{12}^{*}(\Gr{X}{k}{l}{m}{n})_{13})^{\top}
% \circ  \overline{\Sigma_{m,n,k,l}} \\
%  &=
%  \phi((\Gr{X}{k}{l}{m}{n})_{12}^{*\circ (\id \otimes
%   \top)}(\overline{\Sigma_{k,l,k,l}})_{23} (\Gr{X}{k}{l}{m}{n})^{\id \otimes
%   \top}_{12})  \\
% &=\phi((\Gr{\overline{X}}{l}{k}{n}{m})_{12}(\overline{\Sigma_{k,l,k,l}})_{23}
%  (\Gr{\overline{X}}{l}{k}{n}{m})_{12}).
% \end{align*} \fxnote{Treat $X^{\id \otimes \top}$}
% By Lemma \ref{lem:rep-average}, this family is a morphism from to
% $(\overline{\mathcal{H}}\otimes \overline{K},\overline{X}^{-*} \otimes
% \id_{\overline{K}})$  to
% $(\overline{\mathcal{H}}\otimes \overline{K},\overline{X} \otimes
% \id_{\overline{K}})$ and hence of the form
% $(\Gru{\overline{F_{X}}}{n}{m})^{-1} \otimes T$ with $T \in
% \mathcal{B}(\overline{K})$ not depending on $m,n$.

% Thus,
% \begin{align*}
%  (\id \otimes R)  \circ \Sigma_{k,l,m,n} =   \Gru{F}{m}{n} =
%  \Sigma_{k,l,m,n} \circ (\Gru{\overline{F_{X}}}{n}{m})^{-\top} \otimes T^{\top})
% \end{align*}

  
%   We may assume $(k,l,m,n)=(p,q,r,s)$ because otherwise both sides of
%   the equation that we want to prove vanish.

%   Applying Lemma \ref{lem:rep-average} to the corepresentation $X$ and
%   the family $\Gru{T}{p}{q}=
%   \delta_{p,k}\delta_{q,l} |\eta\rangle\langle\xi|$, we obtain an
%   endomorphism $\check{T}$ of $(\mathcal{H},X)$ which
%   necessarily has the form $\check{T}=\lambda(\xi,\eta) \id$ for some
%   $\lambda(\xi,\eta) \in \C$. Inserting the definition of
%   $\check{T}$, we find
%   \begin{align}\nonumber
%     \phi(b^{*}a) &= \phi\big((\id \otimes
%     \omega_{\eta',\eta})((\Gr{X}{k}{l}{m}{n})^{*}) \cdot (\id \otimes
%     \omega_{\xi,\xi'})(\Gr{X}{k}{l}{m}{n})\big) \\  &= (\phi \otimes
%     \id)\left(\langle\eta'|_{2} \Gr{(X^{-1})}{m}{n}{k}{l}(1 \otimes
%       |\eta\rangle\langle \xi|)
%       \Gr{X}{k}{l}{m}{n}|\xi'\rangle_{2}\right) 
%     = \langle \eta'|_{2} \Gru{\check{T}}{m}{n}|\xi'\rangle_{2} =
%     \lambda(\xi,\eta) \langle\eta'|\xi'\rangle. \label{eq:rep-orthogonal-1}
%   \end{align}
%   Next, we apply Lemma \ref{lem:rep-average} to the corepresentations
%   $\overline{X}$ and $\overline{X}^{-*}$ and the family
%   $\Gru{R}{p}{q}=\delta_{p,m}\delta_{q,n}|\overline{
%     \xi'}\rangle\langle\overline{\eta'}|$, and obtain a morphism
%   $\hat{R}$ from $\overline{X}^{-*}$ to $\overline{X}$ which
%   necessarily has the form $\hat{R}=\mu(\eta',\xi')\overline{F_{X}}$
%   for some $\mu(\eta',\xi') \in \C$. Using the relation
%   \begin{align*}
%     a &= (\id \otimes
%     \omega_{\overline{\xi},\overline{\xi'}})(\Gr{\overline{X}}{l}{k}{n}{m})^{*},
%     & b&= (\id \otimes
%     \omega_{\overline{\eta},\overline{\eta'}})(\Gr{\overline{X}}{k}{l}{m}{n})^{*}
%   \end{align*}
%   and the definition of $\hat{R}$, we obtain
%   \begin{align}
%     \phi(b^{*}a) &= (\phi \otimes \id)\left(\langle
%       \overline{\eta}|_{2} \Gr{\overline{X}}{k}{l}{m}{n}(1 \otimes
%       |\overline{\eta}'\rangle\langle \overline{\xi'}|)
%       \Gr{(\overline{X}^{*})}{m}{n}{k}{l} \right) \nonumber \\
%     &=\langle \overline{\eta}| \Gru{\hat{R}}{k}{l}
%     |\overline{\xi}\rangle = \langle
%     \overline{\eta}|\Gru{\overline{F_{X}}}{k}{l}\overline{\xi}\rangle
%     \mu(\eta',\xi'). \label{eq:rep-orthogonal-2}
%   \end{align}
%   We choose a basis $(\zeta_{i})_{i}$ for
%   $\bigoplus_{k}\Gru{\mathcal{H}}{k}{l}$ and calculate
%   \begin{align*}
%  \langle
%     \eta'|\xi'\rangle &=  (\phi \otimes
%     \id)(\langle\eta'|_{2}(\lambda_{l}\rho_{n} \otimes
%     \id_{\Gru{\mathcal{H}}{m}{n}})|\xi'\rangle_{2}) \\
%  &=
%     \sum_{k} (\phi \otimes \id)\left(\langle\eta'|_{2}
%       \Gr{(X^{-1})}{m}{n}{k}{l}
%       \Gr{X}{k}{l}{m}{n}|\xi'\rangle_{2}\right)
%     \\ &=    \sum_{i} \lambda(\zeta_{i},\zeta_{i}) \langle
%     \eta'|\xi'\rangle 
%     \\
%     &=\sum_{i} \langle
%     \overline{\zeta_{i}}| \Gru{\overline{F_{X}}}{}{l}\overline{\zeta_{i}}\rangle
%     \mu(\eta',\xi') 
% \\ &    = \mu(\eta',\xi') \cdot \sum_{k} \Tr(\Gru{F_{(X)}}{k}{l}),
%   \end{align*}
% where $\Gru{\overline{F_{X}}}{}{l}=\bigoplus_{k}
% \Gru{\overline{F_{X}}}{k}{l}$. Inserting this relation into
% \eqref{eq:rep-orthogonal-2}, we finally obtain the assertion.


%%% Local Variables: 
%%% mode: latex
%%% TeX-master: "dyn-suq-main"
%%% End: 
