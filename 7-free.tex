% In definition of faithful $I^2$-algebra, we can indeed just suppose that the embedding $\Ff(I^2)\rightarrow M(A)$ is faithful.
% Correct use of $\cdot$ versus operator product?

% Podles sphere can be defined as an algebra in $\Rep(SU_q(2))$. Hence, it should make sense as an algebra under the forgetful functor to SU_q(2)-dynamical. By duality for coideals, the same should hold for passage to SU_q(1,1) in fact...

% Also: direct construction should help in making non-compact construction

% Thank Piotr, Makoto and ...(?) for discussions. Link with graph C*-algebras?



\section{Partial compact quantum groups from reciprocal random walks}

\subsection{Reciprocal random walks}

We recall some notions introduced in \cite{DCY1}. We slightly change the terminology for the sake of convenience.

\begin{Def} Let $t\in \R_0$. A \emph{$t$-reciprocal random walk} consists of a quadruple $(\Gamma,w,\sgn,i)$ with \begin{itemize}
\item[$\bullet$] $\Gamma=(\Gamma^{(0)},\Gamma^{(1)},s,t)$ a graph with \emph{source} and \emph{target} maps \[s,t:\Gamma^{(1)}\rightarrow \Gamma^{(0)},\]
\item[$\bullet$] $w$ a function (the \emph{weight} function) $w:\Gamma^{(1)}\rightarrow \R_0^+$,
\item[$\bullet$] $\sgn$ a function (the \emph{sign} function) $\sgn:\Gamma^{(1)}\rightarrow \{\pm 1\}$,
\item[$\bullet$] $i$ an involution \[i:\Gamma^{(1)} \rightarrow \Gamma^{(1)},\quad e\mapsto \overline{e}\] with $s(\bar{e}) = t(e)$ for all edges $e$,
\end{itemize}
such that the following conditions are satisfied:
\begin{enumerate}[label=(\arabic*)]
\item (weight reciprocality) $w(e)w(\bar{e}) = 1$ for all edges $e$,
\item (sign reciprocality) $\sgn(e)\sgn(\bar{e}) = \sgn(t)$ for all edges $e$,
\item (random walk property) $p(e) = \frac{1}{|t|}w(e)$ satisfies $\sum_{s(e)=v} p(e) = 1$ for all $v\in \Gamma^{(0)}$.
\end{enumerate}
\end{Def}

%Isomorphisms between $t$-random walks will simply be isomorphisms of weighted graphs (and hence do not remember any structure concerning involutions or signedness). 

Note that, by \cite[Proposition 3.1]{DCY1}, there is a uniform bound on the number of edges leaving from any given vertex $v$, i.e.~ $\Gamma$ has a finite degree.

For examples of $t$-reciprocal random walks, we refer to \cite{DCY1}. One particular example which will be needed for our construction of dynamical quantum $SU(2)$ is the following.

\begin{Exa}\label{ExaGraphPod} Take $0<|q|<1$ and $x\in \R$. Write $2_q = q+q^{-1}$. Then we have the reciprocal $-2_q$-random walk \[\Gamma_x =(\Gamma_x,w,\sgn,i)\] with \[ \Gamma^{(0)} = \Z,\quad \Gamma^{(1)} = \{(k,l)\mid |k-l|= 1\}\subseteq \Z\times \Z\] with projection on the first (resp. second) leg as source (resp. target) map, with weight function \[w(k,k\pm 1) = \frac{|q|^{x+k\pm 1}+|q|^{-(x+k\pm 1)}}{|q|^{x+k}+|q|^{-(x+k)}},\] sign function \[\sgn(k,k+1) = 1,\quad \sgn(k,k-1) = -\sgn(q),\] and involution $\overline{(k,k+1)} = (k+1,k)$. 

By translation we can shift the value of $x$ by an integer. By a point reflection and changing the direction of the arrows, we can change $x$ into $-x$. It follows that by some (unoriented) graph isomorphism, we can always arrange to have $x\in \lbrack 0,\frac{1}{2}\rbrack$.
\end{Exa} 

\subsection{Temperley-Lieb categories}

Let now $0<|q|\leq 1$, and let $SU_q(2)$ be Woronowicz's twisted $SU(2)$ group \cite{Wor1}. Then $SU_q(2)$ is a compact quantum group whose category of finite-dimensional unitary representations $\Rep(SU_q(2))$ is generated by the spin $1/2$-representation $\pi_{1/2}$ on $\C^2$. It has the same fusion rules as $SU(2)$, and conversely any compact quantum group with the fusion rules of $SU(2)$ has its representation category equivalent to $\Rep(SU_q(2))$ as a tensor C$^*$-category. Abstractly, these tensor C$^*$-categories are referred to as the \emph{Temperley-Lieb C$^*$-categories}.

Let now $\Gamma = (\Gamma,w,\sgn,i)$ be a $-2_q$-reciprocal random walk. Define $\Hsp^{\Gamma}$ as the $\Gamma^{(0)}$-bigraded Hilbert space $l^2(\Gamma^{(1)})$, where the $\Gamma^{(0)}$-bigrading is given by \[\delta_e \in \Gru{\Hsp^{\Gamma}}{s(e)}{t(e)}\] for the obvious Dirac functions. Note that, because $\Gamma$ has finite degree, $\Hsp^{\Gamma}$ is \emph{row- and column finite dimensional} (rcfd), i.e.~ $\oplus_{v\in \Gamma^{(0)}} \Gru{\Hsp^{\Gamma}}{v}{w}$ (resp.~ $\oplus_{w\in \Gamma^{(0)}} \Gru{\Hsp^{\Gamma}}{v}{w}$) is finite dimensional for all $w$ (resp.~ all $v$). 

Consider now $R_{\Gamma}$ as the (bounded) map \[R_{\Gamma}:l^2(\Gamma^{(0)})\rightarrow \Hsp^{\Gamma}\underset{\Gamma^{(0)}}{\otimes} \Hsp^{\Gamma}\] given by \begin{eqnarray*} R_{\Gamma} \delta_v &=& \sum_{e,s(e) = v} \sgn(e)\sqrt{w(e)}\delta_e \otimes \delta_{\bar{e}}.\end{eqnarray*} Then $R_{\Gamma}^*R_{\Gamma} = |q|+|q|^{-1}$ and \[(R_{\Gamma}^*\otimes \id)(\id\otimes R_{\Gamma}) = -\sgn(q)\id.\]


%Then we can define a couple $\Hsp(\Gamma) = (\Hsp,R)$ where $\Hsp$ is $l^2(\Gamma^{(1)})$ with the $\Gamma^{(0)}$-bigrading $\delta_e \in \Hsp(s(e),t(e))$ for the obvious Dirac functions, and where  

% Dropped Temperley-Lieb terminology

Hence, by the universal property of $\Rep(SU_q(2))$ (\cite[Theorem 1.4]{DCY1}, based on \cite{Tur1,EtO1,Yam1,Pin2,Pin3}), we have a strongly monoidal $^*$-functor
\begin{equation}\label{EqForget} F_{\Gamma}: \Rep(SU_q(2)) \rightarrow {}^{\Gamma^{(0)}}\Hilb_{\rcfd}^{\Gamma^{(0)}}\end{equation} into the tensor C$^*$-category of rcfd Hilbert spaces such that $F_{\Gamma}(\pi_{1/2}) = \Hsp_{\Gamma}$ and $F_{\Gamma}(\mathscr{R}) = R_{\Gamma}$, with \[(\pi_{1/2},\mathscr{R},-\sgn(q)\mathscr{R})\] a solution for the conjugate equations for $\pi_{1/2}$. Up to equivalence, $F_{\Gamma}$ only depends upon the isomorphism class of $(\Gamma,w)$, and is independent of the chosen involution or sign structure. Conversely, any strong monoidal $^*$-functor from $\Rep(SU_q(2))$ into $\Hilb_{\rcfd}^{I\times I}$ for some set $I$ arises in this way \cite{DCY2}.

\subsection{Universal orthogonal partial compact quantum groups}

% Concrete reference to be added.

It follows from the previous subsection and the Tannaka-Krein-Woronowicz duality of \cite[Theorem 4.14]{DCT1} that for each reciprocal random walk on a graph $\Gamma$, one obtains a $\Gamma^{(0)}$-partial compact quantum group $\mathscr{G}$, and conversely every partial compact quantum group $\mathscr{G}$ with the fusion rules of $SU(2)$ arises in this way. Our first aim is to give a direct representation of the associated algebras $A(\Gamma) = P(\mathscr{G})$ by generators and relations. We will write $\Gamma_{vw}\subseteq \Gamma^{(1)}$ for the set of edges with source $v$ and target $w$.


\begin{Theorem}\label{TheoGenRel} Let $0<|q|\leq 1$, and let $\Gamma = (\Gamma,w,\sgn,i)$ be a $-2_q$-reciprocal random walk. Let $A(\Gamma)$ be the total $^*$-algebra associated to the $\Gamma^{(0)}$-partial compact quantum group constructed from the fiber functor $F_{\Gamma}$ as in \eqref{EqForget}. Then $A(\Gamma)$ is the universal $^*$-algebra generated by a copy of the $^*$-algebra of finite support functions on $\Gamma^{(0)}\times \Gamma^{(0)}$ (with the Dirac functions written as $\UnitC{v}{w}$) and elements $(u_{e,f})_{e,f\in \Gamma^{(1)}}$ where $u_{e,f}\in \Gr{A(\Gamma)}{s(e)}{t(e)}{s(f)}{t(f)}$ and 
\begin{eqnarray} 
\label{EqUni1}\sum_{v\in \Gamma^{(0)}}\sum_{g\in \Gamma_{vw}} u_{g,e}^*u_{g,f} = \delta_{e,f}\mathbf{1}\Grru{w}{t(e)}, \qquad \forall w\in \Gamma^{(0)}, e,f\in \Gamma^{(1)},\\
\label{EqUni2}\sum_{w\in \Gamma^{(0)}} \sum_{g\in \Gamma_{vw}} u_{e,g}u_{f,g}^* = \delta_{e,f} \mathbf{1}\Grru{s(e)}{v}\qquad \forall v\in \Gamma{(0)}, e,f\in \Gamma^{(1)},\\ 
\label{EqInt}u_{e,f}^* \;=\; \sgn(e)\sgn(f)\sqrt{\frac{w(f)}{w(e)}} u_{\bar{e},\bar{f}},\qquad \forall e,f\in \Gamma^{(1)}.
\end{eqnarray}

If moreover $v,w\in \Gamma^{(0)}$ and $e,f\in \Gamma^{(1)}$, we have \[\Delta_{vw}(u_{e,f}) = \underset{t(g) = w}{\sum_{s(g) = v}} u_{e,g}\otimes u_{g,f},\]
\[\varepsilon(u_{e,f}) = \delta_{e,f}\] and \[S(u_{e,f}) = u_{f,e}^*.\] 
\end{Theorem} 

Note that the sums in \eqref{EqUni1} and \eqref{EqUni2} are in fact finite, as $\Gamma$ has finite degree. 

\begin{proof} Let $(\Hsp,V)$ be the generating unitary corepresentation of $A(\Gamma)$ on $\Hsp = l^2(\Gamma^{(1)})$. Then $V$ decomposes into parts \[ \Gr{V}{k}{l}{m}{n} = \sum_{e,f} v_{e,f} \otimes e_{e,f} \in \Gr{A}{k}{l}{m}{n}\otimes B(\Gru{\Hsp}{m}{n},\Gru{\Hsp}{k}{l}),\] where the $e_{e,f}$ are elementary matrix coefficients and with the sum over all $e$ with $s(e)=k,t(e)=l$ and all $f$ with $s(f) = m, t(f)=n$. By construction $V$ defines a unitary corepresentation of $A(\Gamma)$, hence the relations \eqref{EqUni1} and \eqref{EqUni2} are satisfied for the $v_{e,f}$. Now as $R_{\Gamma}$ is an intertwiner between the trivial representation on $\C^{\Gamma^{(0)}} = \oplus_{v\in \Gamma^{(0)}} \C$ and $V\Circtv{\Gamma^{(0)}} V$, we have for all $v\in \Gamma^{(0)}$ that \begin{equation}\label{EqMorR}\underset{t(f)=s(g),t(e)=s(f)}{\sum_{e,f,g,h\in \Gamma{(1)}}} v_{e,f}v_{g,h}\otimes \left((e_{e,f}\otimes e_{g,h})\circ R_{\Gamma} \delta_v\right) = \sum_w \UnitC{w}{v}\otimes R_{\Gamma}\delta_v,\end{equation} hence
\[\underset{t(e)=s(g),s(k)=v}{\sum_{e,g,k}} \sgn(k)\sqrt{w(k)}\left( v_{e,k}v_{g,\bar{k}} \otimes \delta_e\otimes \delta_{g}\right) =\sum_{k,s(k)=v,w}\sgn(k)\sqrt{w(k)} \left(\UnitC{w}{v} \otimes \delta_k\otimes \delta_{\bar{k}}\right).\] Hence if $t(e) = s(g)=z$, we have \[\sum_{k,s(k)=v} \sgn(k)\sqrt{w(k)} v_{e,k}v_{g,\bar{k}} =\sum_w \delta_{s(e),v} \delta_{e,\bar{g}} \sgn(e)\sqrt{w(e)}\UnitC{w}{v}.\] Multiplying to the left with $v_{e,l}^*$ and summing over all $e$ with $t(e) = z$, we see from \eqref{EqUni1} that also relation \eqref{EqInt} is satisfied. Hence the $v_{e,f}$ satisfy the universal relations in the statement of the theorem. The formulas for comultiplication, counit and antipode then follow immediately from the fact that $V$ is a unitary corepresentation.
% In fact, I have a small problem with the indices of source and target in the above derivation when summing...

Let us now a priori denote by $B(\Gamma)$ the $^*$-algebra determined by the relations \eqref{EqUni1},\eqref{EqUni2} and \eqref{EqInt} above, and write $\mathscr{B}(\Gamma)$ for the associated $\Gamma^{(0)}\times \Gamma^{(0)}$-partial $^*$-algebra. Write \[\Delta(u_{e,f}) = \sum_{g\in \Gamma^{(1)}} u_{e,g}\otimes u_{g,f},\] which makes sense in $M(B(\Gamma)\otimes B(\Gamma))$ as the degree of $\Gamma$ is finite. Then we compute \begin{eqnarray*} \sum_{v\in \Gamma^{(0)}}\sum_{g\in \Gamma_{vw}}\Delta(u_{g,e})^*\Delta(u_{g,f}) &=& \sum_{v\in \Gamma^{(0)}}\sum_{g\in \Gamma_{vw}} \sum_{h,k\in \Gamma^{(1)}} u_{g,h}^*u_{g,k}\otimes u_{h,e}^*u_{k,f}\\ &=& \sum_{h,k\in \Gamma^{(1)}} \delta_{h,k} \UnitC{w}{t(h)}\otimes u_{h,e}^*u_{k,f}\\ &=&  \sum_{z\in \Gamma^{(0)}}\underset{t(h)=z}{\sum_{h\in \Gamma^{(1)}}} \UnitC{w}{z}\otimes u_{h,e}^*u_{h,f} \\ &=& \delta_{e,f} \sum_{z\in \Gamma^{(0)}} \UnitC{w}{z}\otimes \UnitC{z}{t(e)}\\ &=& \delta_{e,f} \Delta(1).\end{eqnarray*}  Similarly, the analogue of \eqref{EqUni2} holds for $\Delta(u_{e,f})$. As also \eqref{EqInt} holds for $\Delta(u_{e,f})$ by its very form, it follows that we can define a $^*$-algebra homomorphism \[\Delta:B(\Gamma)\rightarrow M(B(\Gamma)\otimes B(\Gamma))\] sending $u_{e,f}$ to $\Delta(u_{e,f})$ and $\UnitC{v}{w}$ to $\sum_{z\in \Gamma^{(0)}} \UnitC{v}{z}\otimes \UnitC{z}{w}$. Cutting down, we obtain maps \[\Delta_{vw}:\Gr{B(\Gamma)}{r}{s}{t}{z}\rightarrow \Gr{B(\Gamma)}{r}{s}{v}{w}\otimes \Gr{B(\Gamma)}{v}{w}{t}{z}\] which then satisfy the properties \ref{Propa},\ref{Propd} and \ref{Prope} of Definition \ref{DefPartBiAlg}. Moreover, the $\Delta_{vw}$ are coassociative as they are coassociative on generators.

% Remark this in beginning that one can construct hom from counit
Let now $e_{\Grru{v}{w},\Grru{v'}{w'}}$ be the matrix units for $l^2(\Gamma^{(0)}\otimes \Gamma^{(0)})$. Then one verifies again directly from the defining relations of $B(\Gamma)$ that one can define a $^*$-homorphism \[\widetilde{\varepsilon}: B(\Gamma)\rightarrow B(l^2(\Gamma^{(0)})),\quad \left\{\begin{array}{lll} \UnitC{v}{w}&\mapsto &\delta_{v,w}\, e_{\Grru{v}{v},\Grru{v}{v}}\\ u_{e,f}&\mapsto& \delta_{e,f}\, e_{\Grru{s(e)}{s(f)},\Grru{t(e)}{t(f)}}\end{array}\right.\] We can hence define a map $\varepsilon: B(\Gamma)\rightarrow \C$ such that \[\widetilde{\varepsilon}(x) = \varepsilon(x) e_{\Grru{k}{m},\Grru{l}{n}},\qquad  \forall x\in \Gr{B(\Gamma)}{k}{l}{m}{n}.\] Clearly it satisfies the conditions \ref{Propb} and \ref{Propc} of a partial $^*$-bialgebra. As $\varepsilon$ satisfies the counit condition on generators, it follows by multiplicativity that it satisfies the counit condition on the whole of $B(\Gamma)$, i.e. $B(\Gamma)$ is a partial $^*$-bialgebra.

It is clear now that the $u_{e,f}$ define a corepresentation $U$ of $B(\Gamma)$. Moreover, from \eqref{EqUni1} and \eqref{EqInt} we can deduce that $R_{\Gamma}: \C_{\Gamma^{(0)}}\rightarrow \Hsp_{\Gamma}\underset{\Gamma^{(0)}}{\otimes}\Hsp_{\Gamma}$ is a morphism from $\C_{\Gamma^{(0)}}$ to $U\Circtv{\Gamma{(0)}} U$ in $\Corep_{\sfd,u}(\mathscr{B}(\Gamma))$, cf. \eqref{EqMorR}. From the universal property of $\Rep(SU_q(2))$, it then follows that we have a (unique) monoidal functor \[G_{\Gamma}: \Rep(SU_q(2)) \rightarrow \Corep_{\sfd,u}(\mathscr{B}(\Gamma))\] such that $G_{\Gamma}(\pi_{1/2}) = U$. This functor is faithful as $\Rep(SU_q(2))$ has no non-trivial ideals. On the other hand, as we have a $\Delta$-preserving $^*$-homomorphism $\mathscr{B}(\Gamma)\rightarrow \mathscr{A}(\Gamma)$ by the universal property of $\mathscr{B}(\Gamma)$, we have a monoidal functor $H_{\Gamma}:  \Corep_{\sfd,u}(\mathscr{B}(\Gamma))\rightarrow \Rep(SU_q(2))$ which is inverse to $G_{\Gamma}$. Then since the commutation relations of $\mathscr{A}(\Gamma)$ are completely determined by the morphism spaces of $\Rep(SU_q(2))$, it follows that we have a $^*$-homomorphism $\mathscr{A}(\Gamma)\rightarrow \mathscr{B}(\Gamma)$ sending $v_{e,f}$ to $u_{e,f}$. This proves the theorem. % Needs some finetuning after TK is spelled out.
\end{proof}



%%% Local Variables: 
%%% mode: latex
%%% TeX-master: "dyn-suq-main"
%%% End: 
