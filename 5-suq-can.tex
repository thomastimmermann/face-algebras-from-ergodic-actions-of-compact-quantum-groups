\section{The C$^*$-algebras of the canonical partical compact quantum group $\mathscr{G}(\Rep(SU_{\sigma q}(2))$}\label{SecUni}

In this final section, we look at the canonical partial compact quantum group associated to the C$^*$-tensor category $\Rep(SU_q(2))$, and show that it is coamenable.

\subsection{Definition}

Denote now $\tau_{\pm}(y) = y\pm y^{-1}$. The reciprocal random walk associated to the trivial coaction of $SU_q(2)$ is given by the graph with vertices $q^{\N_0}$, edge set $\{(y,z)\mid y/z \in \{q^{\pm 1}\}\}$, weights $w(y,z) = \frac{\tau_-(z)}{\tau_-(y)}$ and signs $\sgn(y,qy) = +, \sgn(y,y/q) = -\sigma$.

Following the definition of ..., we see that the associated partial compact quantum group $\mathscr{G}_{\can}$ has its associated partial Hopf $^*$-algebra $A =A_{\can}$ generated by two copies of the finite support functions on $q^{\N_0}$ as well as elements $u_{e,f} \in \Gr{A}{s(e)}{t(e)}{s(f)}{t(f)}$ with relations \begin{align*} &u_{e,f}^* = \sgn(e)\sgn(f) \sqrt{\frac{w(f)}{w(e)}}u_{\bar{e},\bar{f}},\\ &\sum_{t(g) = w} u_{g,e}^* u_{g,f} = \delta_{e,f} \UnitC{w}{t(e)},\\  &\sum_{s(g) = v} u_{e,g}u_{f,g}^* = \delta_{e,f} \UnitC{s(e)}{v}.\end{align*} 

As the above reciprocal random walk is not homogeneous, we can not find as nice a subalgebra inside $M(A)$ as before. However, there is still a suitable analogue available. Namely, for $\epsilon,\nu \in \{-,+\}$, define inside $M(A)$ the elements \[u_{\epsilon,\nu} =\underset{q^{\epsilon}y,q^{\nu}z \in q^{\N_0}}{ \sum_{y,z\in q^{\N_0}}} u_{(y,q^{\epsilon}y),(z,q^{\nu}z)}.\] Then we have \begin{align*}
& u_{\epsilon,\nu}^* \left(\tau_-(q^{\epsilon}\lambda)/\tau_-(\lambda)\right)^{1/2} = \sgn(\epsilon)\sgn(\nu) u_{-\epsilon,-\nu}\left(\tau_-(q^{\nu}\rho)/\tau_-(\rho)\right)^{1/2} \\
&\sum_{\mu} u_{\mu,\epsilon}^* u_{\mu,\nu} = \delta_{\epsilon,\nu}(1- \delta_{\epsilon,+}\delta_{q}(\rho)),\\ &\sum_{\mu} u_{\epsilon,\mu}u_{\nu,\mu}^* = \delta_{\epsilon,\nu}(1-\delta_{\epsilon,-}\delta_{q}(\lambda)), \\& f(\lambda,\rho) u_{\epsilon,\nu} = u_{\epsilon,\nu}f(q^{-\epsilon}\lambda,q^{-\nu}\rho).\end{align*}

Let us denote as before $u_{-,-}=\alpha$ etc. Then the defining relations become \begin{align*}  &\alpha \alpha^* +\beta\beta^* = 1-\delta_q(\lambda) && \alpha^*\alpha + \gamma^* \gamma = 1\\ &\gamma \gamma^* + \delta\delta^* = 1 && \beta^* \beta +\delta^*\delta =1-\delta_q(\rho)\\ & \alpha \gamma^* = -\beta \delta^*, && \alpha^* \beta = -\gamma^* \delta,\end{align*} and writing $w_{\pm}(y) = \tau_-(q^{\pm 1}y)/\tau_-(y)$, \begin{align*} & \alpha^* =  \frac{w_-^{1/2}(q\rho)}{w_-^{1/2}(q\lambda)}\delta,&& \beta^* =-\sigma w_-^{-1/2}(q\lambda)\gamma w_+^{1/2}(\rho),\\ &\gamma^* = -\sigma w_+^{-1/2}(\rho)\beta w_-^{1/2}(q\lambda) && \delta^* = \alpha \frac{w_-^{1/2}(q\lambda)}{w_-^{1/2}(q\rho)}.\end{align*} 

Commutation relations between the generators and $\lambda,\rho$ are as before.

\subsection{Representation theory}

As before, we can construct a central element inside $M(A)$.

\begin{Lem} The element $\Omega = \tau_+(\lambda/q\rho)+\tau_-(\lambda)\tau_-(q\rho)\gamma^*\gamma$ is central in $M(A)$.
\end{Lem}
\begin{proof} Elementary
\end{proof}

Also the following lemma is as straightforward as before.

\begin{Lem}  \begin{align*} &\gamma^*\gamma = \frac{\Omega - \tau_+(\lambda/q\rho)}{\tau_-(\lambda)\tau_-(q\rho)} \\ &\gamma \gamma^* = \frac{\Omega- \tau_+(q\lambda/\rho)}{\tau_-(q\lambda)\tau_-(\rho)} \\ &\delta \delta^* = \frac{\tau_+(q\lambda\rho)-\Omega}{\tau_-(q\lambda)\tau_-(\rho)} \\ & \delta^*\delta \tau_-(\rho/q)\tau_-(\lambda) = \tau_+(\lambda\rho/q)-\Omega.\end{align*}
\end{Lem} 

\begin{Lem} For each $m\in \N_0$, there exists exactly one irreducible representation $\pi_m$ of $A$ with $\pi_m(\Omega) = \tau(q^m) = q^m+q^{-m}$, and this exhausts all irreducible representations. Moreover, a concrete realization of $\pi_m$ is on the Hilbert space $\mathscr{H}_m = l^2(W_{m}$ with $W_m$ the `well' \[W_m = \{(q^k,q^l)\mid k,l\in \N_0, |k-l\pm1|\leq m \leq k+l-1\}\] and where the representation is determined by \begin{align*} &\gamma \delta_{y,z} = -\sigma_y \left(\frac{\tau_+(q^m)-\tau_+(y/qz)}{\tau_-(y)\tau_-(qz)}\right)^{1/2} \delta_{y/q,qz},\\ &\delta \delta_{y,z} = \left(\frac{\tau_+(yz/q)-\tau(q^m)}{\tau_-(y)\tau_-(z/q)}\right)^{1/2} \delta_{y/q,z/q},\\ & \lambda \delta_{y,z} = y\delta_{y,z},\quad \rho \delta_{y,z} = z \delta_{y,z}.\end{align*} 
\end{Lem} 

\subsection{Coamenability}

In this section, we prove that $\mathscr{G}_{\can}$ is coamenable. For this, we consider the auxiliary partial compact quantum group $\widetilde{\mathscr{G}}_{\can}$, obtained by enlarging our previous graph `at infinity' with the vertex $0$ and two edges $e_{\pm}$ from 0 to itself with weights $q^{\mp 1}$ and signs $\sgn(e_+)=+,\sgn(e_-)=-\sigma$. The resulting algebra $\widetilde{A}$ will then split into four components \[\widetilde{A} = A_{\can} \oplus A_{\can,\bullet} \oplus A_{\bullet,\can} \oplus A_{\bullet,\bullet},\] where for example $A_{\can,\bullet} = \delta_{q^{\N_0}}(\lambda)\delta_0(\rho) \widetilde{A}$. 

Note that the Haar functional on $\widetilde{A}$ restricts to a faithful functional on each of the components, so it makes sense to talk of the regular representation. 

\begin{Lem} Let $\pi_{\reg}^{a,b}$ be the regular representation of $A_{a,b}$ for $a,b\in \{\can,\bullet\}$. Then $\pi_{\reg}^{\can,\can} \preceq \pi_{\reg}^{\can,\bullet}\otimes \pi_{\reg}^{\bullet,\can}$. 
\end{Lem} 

Hence, we want to study the representation theory of $A_{\can,\bullet}$ and its companion $A_{\bullet,\can}$. Note that inside $M(A_{\can,\bullet})$ we can define `generators' $\alpha,\beta,\gamma,\delta$ just as before, by putting all occurences of $\rho$ in the defining relations equal to $0$. 

\begin{Lem} Up to equivalence, there is only irreducible $^*$-representation $\pi_l$ of $A_{\can,\bullet}$. A concrete realization of it is given on $l^2(W_{l})$ with \[W_l= \{(q^k,q^l)\mid k\in \N_0,l\in \Z, \{k-l,k+l\}\subseteq 2\N+1\}\] by the formulas \begin{align*} &\alpha \delta_{y,z}  = \left(\frac{qz-1/y}{\tau_-(y)}\right)^{1/2} \delta_{qy,qz},\\ & \beta \delta_{y,z}= \sigma_y \left(\frac{y-z/q}{\tau_-(y)}\right)^{1/2} \delta_{qy,z/q}\\ &\gamma \delta_{y,z} = -\sigma_y \left(\frac{y-qz}{\tau_-(y)}\right)^{1/2}\delta_{y/q,qz} \\ &\delta \delta_{y,z} = \left(\frac{z/q-1/y}{\tau_-(y)}\right)^{1/2} \delta_{y/q,z/q},\\ 
& \lambda \delta_{y,z} =y\delta_{y,z}.\end{align*}
\end{Lem}

It follows in particular that the regular representation of $A_{\can,\bullet}$ is just a multiple of the above representation.

Using the unitary antipode, we can find a $^*$-isomorphism $\theta:A_{\bullet,\can}\rightarrow A_{\can,\bullet}$ by \begin{align*} &\alpha \mapsto q^{-1/2}w_-^{1/2}(\lambda)\alpha,\\ &\beta \mapsto -\sigma \beta^*\\ &\rho\mapsto \lambda\end{align*} 

As a result, we obtain the next corollary. 

\begin{Cor} Up to equivalence, there is only irreducible $^*$-representation $\pi_r$ of $A_{\bullet,\can}$. A concrete realization of it is given on $l^2(W_{l})$ with \[W_r= \{(q^k,q^l)\mid k\in \Z,l\in \N_0, \{l-k,k+l\}\subseteq 2\N+1\}\] by the formulas \begin{align*} &\alpha \delta_{y,z}  = \left(\frac{y-1/qz}{\tau_-(qz)}\right)^{1/2} \delta_{qy,qz},\\ & \beta \delta_{y,z}= -\sigma_z \left(\frac{z/q-y}{\tau_-(z/q)}\right)^{1/2} \delta_{qy,z/q}\\ &\gamma \delta_{y,z} = \sigma_z \left(\frac{qz-y}{\tau_-(qz)}\right)^{1/2}\delta_{y/q,qz} \\ &\delta \delta_{y,z} = \left(\frac{y-q/z}{\tau_-(z/q)}\right)^{1/2} \delta_{y/q,z/q},\\ 
& \rho \delta_{y,z} =z\delta_{y,z}.\end{align*}
\end{Cor}

\begin{Theorem} The partial compact quantum group $\mathscr{G}_{\can}$ is coamenable.
\end{Theorem}
\begin{proof} It suffices to check that the trivial representation of $\mathscr{G}_{\can}$ is weakly contained in $\pi = \pi_l\otimes \pi_r$. Now \begin{align*} &\pi(\Omega)\delta_{y,z,y',z'} \\ &= -\sigma_y\sigma_{z'} \left[(1-yz/q)(1-z/qy)(1-y'/qz')(1-y'z'/q)\right]^{1/2} \delta_{y,z/q^2,y'/q^2,z'} \\ &+ \left[\tau(y/qz') -\frac{qz'}{y}(1-yz/q)(1-y'/qz') -\frac{y}{qz'}(1-qz/y)(1-qy'z')\right]\delta_{y,z,y',z'} \\ & - \sigma_y\sigma_z'\left[(1-qyz)(1-qz/y)(1-qy'/z')(1-qy'z')\right]^{1/2}\delta_{y,q^2z,q^2y',z'}.\end{align*} 

It follows immediately that $\delta_{y,1,1,y}$ is an eigenvector of $\pi(Omega)$ at eigenvalue $q+q^{-1}$, hence the representation spanned by $\delta_{y,1,1,y}$ is nothing but the trivial representation itself. 
\end{proof} 







%%% Local Variables: 
%%% mode: latex
%%% TeX-master: "dyn-suq-main"
%%% End: 