\section{The C$^*$-algebras of the canonical partical compact quantum group $\mathscr{G}(\Rep(SU_{\sigma q}(2))$}\label{SecUni}

In this final section, we look at the canonical partial compact quantum group associated to the Temperley-Lieb C$^*$-tensor category $\Rep(SU_{\sigma q}(2))$, and show that it is coamenable.

\subsection{Definition}

Denote $\tau_{\pm}(y) = y\pm y^{-1}$. Fix also $q>0$ and $\sigma\in \{\pm\}$. Consider the following $-\sigma(q+q^{-1})$-reciprocal random walk $\Gamma_{\can}$ \cite[Definition 5.1]{DCT1}: the underlying graph has vertices $q^{\N_0}$ and edge set $\{(y,z)\mid y/z \in \{q^{\pm 1}\}\}$, and we give them the weights $w(y,z) = \frac{\tau_-(z)}{\tau_-(y)}$ and signs $\sgn(y,qy) = +, \sgn(y,y/q) = -\sigma$. The involution is given by $\overline{(y,z)} = (z,y)$. This is the reciprocal random walk associated to $\Rep(SU_{\sigma q}(2))$ considered as a module category over itself, see the discussion following \cite[Definition 5.1]{DCT1}.

By that same discussion, we have an associated non-trivial partial compact quantum group $\mathscr{G}_{\can}$ with the same representation tensor C$^*$-category as $SU_{\sigma q}(2)$. By \cite[Theorem 5.2]{DCT1}, its associated partial Hopf $^*$-algebra $A =A_{\can}$ is generated by two copies of the finite support functions on $q^{\N_0}$ as well as elements $u_{e,f} \in \Gr{A}{s(e)}{t(e)}{s(f)}{t(f)}$ for each couple of edges $e,f$, with relations \begin{align*} &u_{e,f}^* = \frac{\sgn(f)\sqrt{w(f)}}{\sgn(e)\sqrt{w(e)}}u_{\bar{e},\bar{f}},\end{align*} \begin{align*}
 &\sum_{t(g) = w} u_{g,e}^* u_{g,f} = \delta_{e,f} \UnitC{w}{t(e)}, &&\sum_{s(g) = v} u_{e,g}u_{f,g}^* = \delta_{e,f} \UnitC{s(e)}{v}.\end{align*} 

As the graph underlying the above reciprocal random walk is not homogeneous, we can not find as nice a subalgebra inside $M(A)$ as in the previous section. However, there is still a suitable analogue available. Namely, for $\epsilon,\nu \in \{-,+\}$, define inside $M(A)$ the elements \[u_{\epsilon,\nu} =\underset{q^{\epsilon}y,q^{\nu}z \in q^{\N_0}}{ \sum_{y,z\in q^{\N_0}}} u_{(y,q^{\epsilon}y),(z,q^{\nu}z)}.\] Then we have the grading relation  $f(\lambda,\rho) u_{\epsilon,\nu} = u_{\epsilon,\nu}f(q^{-\epsilon}\lambda,q^{-\nu}\rho)$, where the value of a function is defined to be zero if its input is outside its domain of definition, together with \begin{align*}
& u_{\epsilon,\nu}^* \left(\tau_-(q^{\epsilon}\lambda)/\tau_-(\lambda)\right)^{1/2} = \sgn(\epsilon)\sgn(\nu) u_{-\epsilon,-\nu}\left(\tau_-(q^{\nu}\rho)/\tau_-(\rho)\right)^{1/2},\end{align*} 
\begin{align*} 
&\sum_{\mu} u_{\mu,\epsilon}^* u_{\mu,\nu} = \delta_{\epsilon,\nu}(1- \delta_{\epsilon,+}\delta_{q}(\rho)), && \sum_{\mu} u_{\epsilon,\mu}u_{\nu,\mu}^* = \delta_{\epsilon,\nu}(1-\delta_{\epsilon,-}\delta_{q}(\lambda)).\end{align*}

Let us denote as before $u_{-,-}=\alpha$ etc. Then the defining relations become 

\begin{equation*} \left\{\begin{array}{llllllll} \alpha \alpha^* +\beta\beta^* &=& 1-\delta_q(\lambda) && \gamma \gamma^* + \delta\delta^* &=& 1,\\
\alpha^*\alpha + \gamma^* \gamma &=& 1, && \beta^* \beta +\delta^*\delta &=&1-\delta_q(\rho),\\ \\  \alpha \gamma^* = -\beta \delta^*, &&&& \alpha^* \beta = -\gamma^* \delta,\end{array}\right.\end{equation*} 
and, writing $w_{\pm}(y) = \tau_-(q^{\pm 1}y)/\tau_-(y)$, 
\begin{align*} &\delta^* = \alpha \frac{w_-^{1/2}(q\lambda)}{w_-^{1/2}(q\rho)}, &&\gamma^* = -\sigma w_+^{-1/2}(\rho)\beta w_-^{1/2}(q\lambda),\\ 
&\beta^* = -\sigma w_-^{-1/2}(q\lambda)\gamma w_+^{1/2}(\rho),&&\alpha^* =  \frac{w_-^{1/2}(q\rho)}{w_-^{1/2}(q\lambda)}\delta.\end{align*}

Commutation relations between the generators and $\lambda,\rho$ are as before.

\subsection{Representation theory}

As before, we can construct a central element inside $M(A)$.

\begin{Lem} The element $\Omega = \tau_+(\lambda/q\rho)+\tau_-(\lambda)\tau_-(q\rho)\gamma^*\gamma$ is a central and selfadjoint element in $M(A)$.
\end{Lem}
\begin{proof} Elementary.
\end{proof}

Also the following lemma is as straightforward as before.

\begin{Lem}\label{LemOm}  \begin{align*}  & \alpha^*\alpha = \frac{\tau_+(q\lambda\rho)-\Omega}{\tau_-(\lambda)\tau_-(q\rho)}, && \gamma^*\gamma = \frac{\Omega - \tau_+(\lambda/q\rho)}{\tau_-(\lambda)\tau_-(q\rho)} 
\end{align*}
%\\ &\gamma \gamma^* = \frac{\Omega- \tau_+(q\lambda/\rho)}{\tau_-(q\lambda)\tau_-(\rho)} \\ &\delta \delta^* = \frac{\tau_+(q\lambda\rho)-\Omega}{\tau_-(q\lambda)\tau_-(\rho)} \\ & \delta^*\delta \tau_-(\rho/q)\tau_-(\lambda) = \tau_+(\lambda\rho/q)-\Omega.
\end{Lem} 

\begin{Lem} For each $m\in \N_0$, there exists exactly one irreducible representation $\pi_m$ of $A$ with $\pi_m(\Omega) = \tau_+(q^m) = q^m+q^{-m}$, and this exhausts all irreducible representations. Moreover, a concrete realization of $\pi_m$ is on the Hilbert space $\mathscr{H}_m = l^2(W_{m})$ with $W_m$ the `well' \[W_m = \{(q^k,q^l)\mid k,l\in \N_0, |k-l\pm1|\leq m \leq k+l-1\}\] and where the representation is determined by \begin{align*} & \lambda e_{y,z} = ye_{y,z},&& \rho e_{y,z} = z e_{y,z},\\
&\alpha e_{y,z} = \left(\frac{\tau_+(qyz)-\tau_+(q^m)}{\tau_-(y)\tau_-(qz)}\right)^{1/2} e_{qy,qz}, &&\beta e_{y,z} = \sigma_y \left(\frac{\tau_+(q^m)-\tau_+(qy/z)}{\tau_-(y)\tau_-(z/q)}\right)^{1/2}e_{qy,z/q}, \\
&\gamma e_{y,z} = -\sigma_y \left(\frac{\tau_+(q^m)-\tau_+(y/qz)}{\tau_-(y)\tau_-(qz)}\right)^{1/2} e_{y/q,qz}, && \delta e_{y,z} = \left(\frac{\tau_+(yz/q)-\tau_+(q^m)}{\tau_-(y)\tau_-(z/q)}\right)^{1/2} e_{y/q,z/q},\end{align*}
where $e_{y,z}$ is interpreted as the zero vector when $(y,z)\notin W_m$. 
\end{Lem} 

\begin{proof}% Elegant proof of first assertion? 
It can be proven in a straightforward (but slightly tedious) way that the set of elements \[\{\delta_{y}(\lambda)\delta_{z}(\rho)\alpha^m\beta^n\gamma^k\delta^l\mid k,l,m,n\in \N,y,z\in q^{\N_0}\}\] spans $A$. It follows that if $(\Hsp,\pi)$ is any non-degenerate irreducible representation of $A$, then the $\Hsp^y_z$ are at most one-dimensional for each $y,z\in q^{\N_0}$. 

Fix $(\Hsp,\pi)$, and write again $H = \oplus_{y,z\in q^{\N_0}} \Hsp^y_z$ for the algebraic direct sum of all $\Hsp^y_z$. It follows as before that the generators $\alpha,\beta,\gamma,\delta$ act as adjointable endomorphisms on $H$, and that $\Omega$ then acts as a scalar $c$. Write $T_{\pi} =\{(y,z)\mid \Hsp^{y,z}\neq \{0\}, y,z\in q^{\N_0}\}$. We can hence pick unit vectors $e_{y,z}\in \Hsp^y_z$ on which $A$ acts as in the statement of the lemma with $W_m$ replaced by $T_{\pi}$ and $q^m$ replaced by $c$.

Choose now $(y,z)\in T_{\pi}$ with $(y/q,qz)\notin T_{\pi}$. Then $\gamma^*\gamma e_{y,z} = 0$, hence $c=\tau_+(y/qz)$. So $c= \tau_+(q^m)$ for some $m\in \N$.

Let us first show that the kernel of $\alpha$ is zero. Suppose not, then there exists $(y,z)\in T_{\pi}$ with $\alpha e_{y,z}=0$. This implies $y= q^k$ and $z=q^l$ with $k,l\in \N_0$ and $k+l+1 = m$. In particular, $m\geq 2$. It follows from the well-definedness of the square root in the formula for $\delta$ that then necessarily $y=q$ or $z=q$, and in particular $e_{y,z}$ also in the kernel of $\delta$. It follows from irreducibility and the commutation relations between $\{\alpha,\delta\}$ and $\{\gamma,\beta\}$ that $\pi(\alpha) = \pi(\delta)=0$. But it is easy to check that dividing out by these relations forces also $\pi(\beta)=\pi(\gamma)=0$, giving a contradiction. 

As the kernel of $\alpha$ is zero, it follows that $(y,z)\in T_{\pi}$ implies $(qy,qz)\in T_{\pi}$. Denote now by $S_{\pi}$ the set of all $s\in \Z$ with $\{(y,q^sy)\mid y\in q^{\N_0}\}\cap T_{\pi}\neq \emptyset$. Then for each $s\in S_{\pi}$, we can find a unique element $(y_s,q^sy_s) \in T_{\pi}$ such that all $(y,z)\in T_{\pi}$ with $z/y= q^s$ are of the form $(q^ny_s,q^{n+s}y_s)$ for $n\in \N$, with all such elements appearing inside $T_{\pi}$. Looking at the actions of $\gamma$ and $\beta$, we see that if $s\in S_{\pi}$, then necessarily $-m+1\leq s\leq m-1$, enforcing $m\geq 1$. On the other hand, if $(q^k,q^l)\in T_{\pi}$, we see from the action of $\delta$ that necessarily $k+l-1\geq m$. 

It follows that $T_{\pi} \subseteq W_m$. To conclude the proof, it suffices to show that the proposed action on $l^2(W_m)$ is well-defined and irreducible. Well-definedness is immediate. Irreducibility follows also straightforwardly.
\end{proof}


\subsection{Coamenability}

In this section, we prove that $\mathscr{G}_{\can}$ is coamenable. For this, we consider the auxiliary partial compact quantum group $\widetilde{\mathscr{G}}_{\can}$, obtained by enlarging our previous graph $\Gamma_{\can}$ `at infinity' with the vertex $0$ and two edges $e_{\pm}$ from 0 to itself with weights $q^{\mp 1}$ and signs $\sgn(e_+)=+,\sgn(e_-)=-\sigma$. The resulting algebra $\widetilde{A}$ will then split into a direct sum of four $^*$-algebras \[\widetilde{A} = A_{\can,\can} \oplus A_{\can,\bullet} \oplus A_{\bullet,\can} \oplus A_{\bullet,\bullet},\] where for example $A_{\can,\bullet} = \delta_{q^{\N_0}}(\lambda)\delta_0(\rho) \widetilde{A}$. From the discussion in \cite[Section 4.2]{DCT1}, it follows that $A_{\can,\can}$ is the $^*$-algebra $A_{\can}$ from before, whereas $A_{\bullet,\bullet}$ will be the $^*$-algebra associated to the compact quantum group $SU_q(2)$. The same discussion also provides us with homomorphisms \[\Delta_{a,b}^c: A_{ab}\rightarrow M(A_{ac}\otimes A_{cb}),\qquad a,b,c \in \{\bullet,\can\},\] which allows us to take tensor representations of $A_{ac}$ and $A_{cb}$ to a representation of $A_{ab}$.

Note that the Haar functional on $\widetilde{A}$ restricts to a faithful functional on each of the components, so it makes sense to talk of the regular representation. Moreover, the restriction of the Haar functional to $A_{\can,\can}$ gives precisely the Haar functional on $A_{\can}$.

\begin{Lem} Let $\pi_{\reg}^{a,b}$ be the regular representation of $A_{a,b}$ for $a,b\in \{\can,\bullet\}$. Then $\pi_{\reg}^{\can,\can} \preceq \pi_{\reg}^{\can,\bullet}\boxtimes \pi_{\reg}^{\bullet,\can}$. 
\end{Lem} 

\begin{proof} Let $\phi$ be the Haar functional on $\widetilde{A}$, and denote by $\phi_{km}$ its restriction to $\gr{\widetilde{A}}{k}{k}{m}{m}$. Then by definition of invariance, we find by invariance of $\phi$ (and the fact that $\UnitC{k}{\bullet}\neq 0$ for all $k$) that for all $a\in A_{\can,\can}$, \[(\phi\otimes \phi)\Delta_{\bullet,\bullet}(a) = \phi(a).\] This implies the lemma.
\end{proof}


Hence, we want to study the representation theory of $A_{\can,\bullet}$ and its companion $A_{\bullet,\can}$. Note that inside $M(A_{\can,\bullet})$ we can define `generators' $\alpha,\beta,\gamma,\delta$ just as before, by putting all occurences of $\rho$ in the defining relations equal to $0$. 

\begin{Lem} Up to equivalence, there is only one irreducible $^*$-representation $\pi_L$ of $A_{\can,\bullet}$. A concrete realization of it is given on $l^2(W_{L})$ with \[W_L= \{(q^k,q^l)\mid k\in \N_0,l\in \Z, \{k-l,k+l\}\subseteq 2\N+1\}\] by the formulas \begin{align*}& \lambda e_{y,z} =ye_{y,z},\\
 &\alpha e_{y,z}  = \left(\frac{qz-1/y}{\tau_-(y)}\right)^{1/2} e_{qy,qz}, && \beta e_{y,z}= \sigma_y \left(\frac{y-z/q}{\tau_-(y)}\right)^{1/2} e_{qy,z/q}\\ &\gamma e_{y,z} = -\sigma_y \left(\frac{y-qz}{\tau_-(y)}\right)^{1/2}e_{y/q,qz} &&\delta e_{y,z} = \left(\frac{z/q-1/y}{\tau_-(y)}\right)^{1/2} e_{y/q,z/q},\\ 
\end{align*}
\end{Lem}

\begin{proof} Consider the element $\Omega_{\rho} = \lambda/q+q^{-1}\tau_-(\lambda)\gamma^*\gamma$, which is a formal limit $\lim_{\rho\rightarrow 0} \rho\Omega$ with $\Omega$ in $A_{\can}$ as before. Then it follows that $\Omega_{\rho}$ $q$-commutes with the generators $\alpha,\beta,\gamma,\delta$. 

Let $\pi$ be an irreducible representation of $A_{\can,\bullet}$. It is easy to see that, by irreducibility and $q$-commutation rules, the spectrum of $\Omega_{\rho}$ is purely discrete. Moreover, by the precise formula of $\Omega_{\rho}$, it follows that the spectrum is strictly positive. It then follows as before, using the limit analogues of the identities in Lemma \ref{LemOm}, that if $S_{\pi}$ is the spectrum of $\Omega_{\rho}$, each $\Hsp^y_z$ with $y\in q^{\N_0}$ and $z\in \Spec(\Omega_{\rho})$ is at most one-dimensional, and that, denoting $T_{\pi}$ for the set of $(y,z)\in q^{\N_0}\times S_{\pi}$ with $\Hsp^y_z\neq \{0\}$, we can find unit vectors $e_{y,z}$ for $(y,z)\in T_{\pi}$ such that the representation acts as in the statement of the lemma.  

Now choose $(y,z)\in T_{\pi}$ with $(y/q,qz)\notin T_{\pi}$. Then we deduce from the action of $\gamma$ that $y=qz$, and in particular $S_{\pi} \subseteq q^{\Z}$. But it is also clear that $e_{y,y/q}$ is not in the kernel of $\delta$ for $y\neq q$. It follows that $(q,1) \in T_{\pi}$. Moreover, as $\gamma e_{q,1} = \delta e_{q,1}$, it follows that $T_{\pi}\subseteq W_L$. As before, it then suffices to check that the given action on $l^2(W_L)$ is well-defined and irreducible.

\end{proof}

It follows in particular that the regular representation of $A_{\can,\bullet}$ is just a multiple of the above representation.

Using the unitary antipode of $\widetilde{A}$, we can find a $^*$-isomorphism $\theta:A_{\bullet,\can}\rightarrow A_{\can,\bullet}$ by \begin{align*} &\alpha \mapsto q^{-1/2}w_-^{1/2}(\lambda)\alpha,&&\beta \mapsto -\sigma \beta^* &&\rho\mapsto \lambda\end{align*} 

As a result, we obtain the next corollary. 

\begin{Cor} Up to equivalence, there is only irreducible $^*$-representation $\pi_R$ of $A_{\bullet,\can}$. A concrete realization of it is given on $l^2(W_R)$ with \[W_R= \{(q^k,q^l)\mid k\in \Z,l\in \N_0, \{l-k,k+l\}\subseteq 2\N+1\}\] by the formulas \begin{align*} & \rho e_{y,z} =z e_{y,z} \\ &\alpha e_{y,z}  = \left(\frac{y-1/qz}{\tau_-(qz)}\right)^{1/2} e_{qy,qz},&& \beta e_{y,z}= -\sigma_z \left(\frac{z/q-y}{\tau_-(z/q)}\right)^{1/2} e_{qy,z/q},\\ &\gamma e_{y,z} = \sigma_z \left(\frac{qz-y}{\tau_-(qz)}\right)^{1/2}e_{y/q,qz},  &&\delta e_{y,z} = \left(\frac{y-q/z}{\tau_-(z/q)}\right)^{1/2} e_{y/q,z/q}.\\ 
\end{align*}
\end{Cor}

\begin{Theorem} The partial compact quantum group $\mathscr{G}_{\can}$ is coamenable.
\end{Theorem}
\begin{proof} It suffices to check that the trivial representation of $\mathscr{G}_{\can}$ is weakly contained in $\pi = \pi_L\boxtimes \pi_R$. Now the latter is a (non-degenerate) representation on $l^2(W_L)\otimes l^2(W_L)$, and a small calculation yields \begin{align*} &\hspace{-0.4cm}\pi(\Omega)e_{y,z,y',z'} \\ &= -\sigma_y\sigma_{z'} \left[(1-yz/q)(1-z/qy)(1-y'/qz')(1-y'z'/q)\right]^{1/2} e_{y,z/q^2,y'/q^2,z'} \\ &+ \left[\tau_+(y/qz') -\frac{qz'}{y}(1-yz/q)(1-y'/qz') -\frac{y}{qz'}(1-qz/y)(1-qy'z')\right] e_{y,z,y',z'} \\ & - \sigma_y\sigma_z'\left[(1-qyz)(1-qz/y)(1-qy'/z')(1-qy'z')\right]^{1/2} e_{y,q^2z,q^2y',z'}.\end{align*} 

It follows immediately that $e_{q,1,1,q}$ is an eigenvector of $\pi(\Omega)$ at eigenvalue $q+q^{-1}$, hence the representation spanned by $e_{q,1,1,q}$ is nothing but the trivial representation itself. 
\end{proof} 







%%% Local Variables: 
%%% mode: latex
%%% TeX-master: "dyn-suq-main"
%%% End: 