\section{Tannaka-Krein duality for partial compact quantum groupoids}
% Rough version. 


In this section, we prove a Tannaka-Krein result for partial compact quantum groupoids. In the previous section, we saw that a partial Hopf algebra $(\mathscr{A},\Delta)$ gives rise to a tensor category with duals $(\CatC,\otimes)$, its category of row and column-finite corepresentations. Its unit is in general not an irreducible object. When $\mathscr{A}$ defnes a compact quantum groupoid, we moreover have that endomorphism spaces are $^*$-algebras, and upon imposing a uniform boundedness condition (as in ...), they are finite von Neumann algebras of type $I$ with discrete center, i.e. a von Neumann algebraic direct product of matrix algebras. General morphism spaces are then Hilbert W$^*$-bimodules between the endomorphism algebras of their source and target. Let us formalize this notion. % See how corep category for 

% but the endomorphism ring of the unit is a direct sum of copies of the ground field $\C$. %We have to properly define homomorphisms so that they only live on finitely many components! But then we have problems for the unit and 

\begin{Def} A tensor W$^*$-category $(\CatC,\otimes)$ with duals is said to have a \emph{decomposable unit} if $\End(\mathbf{1}_{\mathcal{C}}) = l^{\infty}(\mathcal{I})$ for some set $\mathcal{I}$. In this case, we call $(\CatC,\otimes)$ a \emph{multi}-fusion W$^*$-category. % Ok to use this name?
\end{Def}

As we have remarked before, tensor W$^*$-categories with decomposable unit can also be described as a collection of tensor W$^*$-categories with duality and irreducible units, and a collection of bimodule W$^*$-categories between them. 

\begin{Exa} Let $I$ be a set. The category $\Hilb_f^{I\times I}$ of row and column-finite $I\times I$-graded Hilbert spaces with tensor product $\itimes$ forms a tensor W$^*$-category with duals and decomposable unit. 
\end{Exa}

\begin{Exa} Weak Morita equivalence. 
\end{Exa}

Now if $\mathscr{A}$ defines a partial compact quantum group over the base set $I$, then its $\CatC = \Corep_{u,f}(\mathscr{A})$ comes equipped with a faithful forgetful $^*$-functor $\Forget: \CatC \rightarrow \Hilb_f^{I\times I}$. This $\Forget$ is equipped with a natural unitary strong monoidal structure.  Such pairs $((\CatC,\otimes),\Forget)$ also arise in other situations.

\begin{Exa}  Any ergodic action of a compact quantum group $\G$ on a unital C$^*$-algebra provides a tensor C$^*$-functor of $\CatC$ into $\Hilb_{I^2}$ for some set $I$.

\end{Exa}

Our aim is to show that this establishes a one-to-one correspondence between partial compact quantum groups over the base space $I$ and multifusion W$^*$-categories endowed with a unitary strong monoidal $^*$-functor into $\Hilb_f^{I\times I}$. We will need some preparation. As the situation is quite analogous to the one in ..., we will not verify all steps.

In the next part, we will fix once and for all a multifusion W$^*$-category $(\CatC,\otimes)$ equipped with a faithful forgetful unitary monoidal $^*$-functor $\Forget$ into $\Hilb_f^{I\times  I}$. % In fact, as the construction will only depend on the image of $\CatC$ under $\Forget$, we will (as in [Woronowicz]) identify $\mathcal{C}$ as a subcategory of $\Hilb_f^{I\times I}$. We then write $\Mor_{\CatC}$ for the morphism spaces of $\CatC$. 
We will in general view the tensor product of $\CatC$ as being strict.
We  choose a maximal family of mutually inequivalent irreducible objects $\{u_a\}_{a\in J}$ in $\CatC$. We let $J_0\subseteq J$ be the set of indices $a$ such that $u_a$ is a subrepresentation of $\mathbf{1}_{\CatC}$. Whenever convenient, we will replace $u_a$ by its associated index symbol $a$. 
We will also fix once and for all orthonormal bases $f_{c,j}^{a,b}$ for $\Mor(u_c,u_a\otimes u_b)$, where $j$ runs over an index set $J^{a,b}_{c}$.

\begin{Def} For $k,l,m,n\in I$, define vector spaces \[\Gr{A}{k}{l}{m}{n}(a) = B(\Forget(u_a)_{kl},\Forget(u_a)_{mn}).\] Write \[\Gr{A}{k}{l}{m}{n} = \underset{a\in J}{\oplus} \Gr{A}{k}{l}{m}{n}(a),\qquad A = \underset{k,l,m,n}{\oplus} \Gr{A}{k}{l}{m}{n}.\] The $a$-spectral subspace $A(a)$ of $A$ is defined as \[A(a) = \osum{k,l,m,n} \Gr{A}{k}{l}{m}{n}(a).\] For any element $x\in A$, its component in the $a$-spectral subspace is written $x_a$.
\end{Def} 

Our goal is to turn $A$ into to the total weak Hopf $^*$-algebra of a generalized compact Hopf face algebra with base set $I$. 

%Let $I$ be a (countable) set. We will write $\Hilb_{I^2}$ for the monoidal tensor C$^*$-category of $I$-bigraded Hilbert spaces $\Hsp = \osumc{r,s}\Hsp_{rs}$, where the direct sum on the right is understood as the completion of the ordinary algebraic one. The tensor product $\itimes$ in $\Hilb_{I^2}$ is defined by $(\Hsp\itimes \mathscr{G})_{rs} = \oplusc_t \left(\Hsp_{rt}\otimes\mathscr{G}_{ts}\right)$. The unit of $\Hilb_{I^2}$ is $l^2(I)$ with the obvious $I^2$-grading. We will view this monoidal category as being strict.

%We will be interested in strong tensor C$^*$-functors $F$ from $\CatC$ to $\Hilb_{I^2}$. As shown in [DCY], 

Let us denote the unitary compatibility morphisms of $\Forget$ by $\phi_{X,Y}:F(X)\itimes F(Y)\rightarrow F(X\otimes Y)$, where we recall that they are assumed to satisfy the coherence conditions  \[\phi_{X,Y\otimes Z}(\id_X\otimes \phi_{Y,Z}) = \phi_{X\otimes Y,Z}(\phi_{X,Y}\otimes \id_Z),\qquad \phi_{o,a} = \phi_{a,o} = \id_a.\] It will be convenient to extend $\phi_{X,Y}$ to a coisometry $F(X)\otimes F(Y)\rightarrow F(X\otimes Y)$, defining it to be zero on the orthogonal complement of $F(X)\itimes F(Y)$. Note however that then $\phi_{X,o}$ becomes the coisometry $F(X)\otimes l^2(I)\rightarrow F(X)$ sending $F(X)_{rs} \otimes \C\delta_t$ canonically onto $\delta_{s,t}F(X)_{rs}$, and similarly for $\phi_{o,X}$. Whenever $X,Y$ are clear, we will abbreviate $\phi_{X,Y}$ as $\phi$. We will use the notation \[F^{a,b}_{c,j} = \phi^* F(f^{a,b}_{c,j}) \in B(F(u_c),F(u_a)\otimes F(u_b)).\]

%As $\CatC$ is rigid, each $F(X)$ will be column-finite in the sense that for each $X$ in $\CatC$ and each fixed $s$ in $I$, the direct sum $\osum{r} F(X)_{rs}$ will be finite-dimensional. Similarly, each $F(X)$ will be row-finite. See ...

%If $X$ is an object in $\CatC$, we have a canonical unitary $F(X) \cong \oplus_{a\in I}\left( F(u_a)\otimes \Mor(u_a,X)\right)$, where the $\Mor(u_a,X)$ are Hilbert spaces by the inner product $\langle S,T\rangle = S^*T$. For any $\xi \in F(X)$, we will write $\xi =\sum_a \xi_{a} \otimes \xi^a$ under this isomorphism, so that $\sum_a F(\xi^a)\xi_a = \xi$. If $\xi \in F(u_a)$, we will also write $\xi = \xi_a$ for emphasis.

%Similarly, when $\xi \in F(X)\itimes F(Y)$, we will write $\xi_{(I)}\otimes \xi_{(J)}$ for the image of $\xi$ under the natural isomorphism $F(X)\itimes F(Y)\underset{\phi_{X,Y}}{\cong} F(X\otimes Y) \cong \oplus_{a\in I}\left(F(u_a)\otimes \Mor(u_a,X\otimes Y)\right)$.

%Define vector spaces \[\Gr{A}{k}{l}{m}{n}(a) = B(F(u_a)_{kl}, F(u_a)_{mn}).\] Write $\Gr{A}{k}{l}{m}{n} = \oplus_{a\in J} \Gr{A}{k}{l}{m}{n}(a)$ and $A = \oplus_{k,l,m,n} \Gr{A}{k}{l}{m}{n}$. The $a$-spectral subspace $A(a)$ of $A$ is defined as \[A(a) = \osum{k,l,m,n} \Gr{A}{k}{l}{m}{n}(a).\] For any element $x\in A$, its component in the $a$-spectral subspace is written $x_a$.

%Our goal is to turn $A$ into a generalized compact Hopf face algebra.

%If $\xi$ is a vector in $F(u_a)_{mn}$, we will denote it as $\xi_a$ to emphasize its dependence on $a$. On the other hand, a typical element $\xi$ in $\overline{F(u_a)_{mn}}$ will be denoted $\xi^a$. By $\xi\rightarrow \bar{\xi}$ we will denote the canonical real linear map $F(u_a)_{rs} \rightarrow \overline{F(u_a)_{rs}}$, so that $\overline{\xi_a} = \bar{\xi}^a$. An element in $\Gr{A}{k}{l}{m}{n}$ will then be written symbolically as $\xi^a\otimes \eta_a$, where we implicitly assume a summation over several elementary tensors and over $a$. In general, if abstract indices occur both in covariant and contravariant positions, the summation will be implicitly understood, unless explicitly indicated otherwise.

%If $\xi_a \in F(u_a)_{rs}$ and $\eta_b\in F(u_b)_{st}$, we will denote by $(\xi_a\otimes \eta_b)_c$ the $c$-component of the vector $\phi_{a,b}(\xi_a\otimes \eta_b)_c$ inside $F(u_a\otimes u_b)_{rt}$. Similarly, if $\xi^a \in \overline{F(u_a)_{rs}}$ and $\eta_b\in \overline{F(u_b)_{st}}$, we denote by $(\xi^a\otimes \eta^b)^c$ the vector $\overline{(\bar{\xi}_a\otimes \bar{\eta}_b)_c}$.

We first turn $A$ into an algebra. The multiplication $x\cdot y$ of $x\in \Gr{A}{k}{l}{m}{n}(a)$ and $y\in \Gr{A}{p}{q}{r}{s}(b)$ is the element in $\Gr{A}{k}{q}{m}{s}$ defined by the formula \[(x\cdot y)_c = \sum_{j\in J^{a,b}_c} \left(F^{a,b}_{c,j}\right)^*(x\otimes y)\left( F^{a,b}_{c,j}\right).\] Note that the product is independent of the specific choice of orthogonal bases $f^{a,b}_{c,j}$. We will continue to use the $\cdot$-notation to distinguish this product from the ordinary multiplication of operators.

\begin{Lem} With the above product, $A$ becomes a faithful strong $I^2$-algebra.
\end{Lem}

\begin{proof} Let $x\in \Gr{A}{k}{l}{m}{n}(a)$, $y\in \Gr{A}{p}{q}{r}{s}(b)$ and $z\in \Gr{A}{q}{t}{s}{v}$. From the fact that $\phi$ is a natural transformation, we find that \[((x\cdot y)\cdot z)_d = \sum_{e\in J}\sum_{k\in J_d^{e,c}}\sum_{j\in J_e^{a,b}} \left(\phi^*(\phi^*\otimes \id)F(f_{d,e,j,k}^{1,a,b,c})\right)^*(x\otimes y\otimes z)\left((\phi^*\otimes \id)\phi^* F(f_{d,e,j,k}^{1,a,b,c})\right)\] where $f_{d,e,j,k}^{1,a,b,c}=(f_{e,j}^{a,b}\otimes \id)f_{d,k}^{e,c}$. On the other hand, \[(x\cdot (y\cdot z))_d = \sum_{e\in J}\sum_{k\in J_d^{a,e}}\sum_{j\in J_e^{b,c}} \left(\phi(\id\otimes \phi)F(f_{d,e,j,k}^{2,a,b,c})\right)^*(x\otimes y\otimes z)\left((\id\otimes \phi)\phi F(f_{d,e,j,k}^{2,a,b,c})\right)\] where $f_{d,e,j,k}^{2,a,b,c} = (\id\otimes f_{e,j}^{b,c})f_{d,k}^{a,e}$. As $\phi(\phi\otimes \id)$ by $\phi(\id\otimes \phi)$ by assumption, and as the orthonormal bases $\{f_{d,e,j,k}^{1,a,b,c}\mid e,j,k\}$ or $\{f^{2,a,b,c}_{d,e,j,k}\mid e,j,k\}$ can clearly be replaced by any other orthonormal basis of $\Mor(u_d,u_a\otimes u_b\otimes u_c)$, it follows that $(x\cdot y)\cdot z = x\cdot (y\cdot z)$.

Define $1_{rs} \in B(F(u_o)_{rr},F(u_o)_{ss}) = \Gr{A}{r}{r}{s}{s}(o)$ as the map sending $\delta_r$ to $\delta_s$. By the compatibility assumption for $\phi_{a,o}$ and $\phi_{o,a}$, the map $\Ff(I^2)\rightarrow A$ mapping $\delta_{(r,s)}$ to $1_{rs}$ is an algebra homomorphism. Thus $A$ becomes a faithful strong $I^2$-algebra.
\end{proof}

In the following, we will again write $\lambda_r = \sum_s 1_{rs}$ and $\rho_s = \sum_r 1_{rs}$ inside $M(A)$, using the notation as at the end of the proof of the previous lemma.

We turn to the coproduct. Let $\{e_{a,i}\mid i\in B_a\}$ denote an orthonormal basis of $F(u_a)$ over an index set $B_a$ which is adapted to the bigrading (in the sense that each $e_{a,i}$ is inside exactly one component). Write $B_{a,rs}\subseteq B_a$ for the set of indices for which $e_{a,i}\in F(u_a)_{rs}$. Define elements \[P^{kl}_{mn}(a)\in \Gr{A}{k}{l}{m}{n}(a) \otimes \Gr{A}{m}{n}{k}{l}(a)\] by \[P^{kl}_{mn}(a) = \sum_{i\in B_{a,kl},j\in B_{a,mn}}  e_{a,j}e_{a,i}^*\otimes e_{a,i}e_{a,j}^*.\] As each $F(u_a)_{kl}$ is finite-dimensional, the above sums are finite.

%\begin{Lem} For any $x,y\in A$, one has $P^{kl}_{mn}(a)(x\otimes y) = (y\otimes x)P^{kl}_{mn}(a)$ under the ordinary operator product.
%\end{Lem}

%Here the product is considered to be zero if the range and source of a composition don't match.

%\begin{proof}Immediate.
%\end{proof}

Define now maps \[\Delta_{rs}: \Gr{A}{k}{l}{m}{n}(a) \rightarrow \Gr{A}{k}{l}{r}{s}(a)\otimes \Gr{A}{r}{s}{m}{n}(a)\] by the application \[x \mapsto P_{rs}^{mn}(a) (x\otimes 1) = (1\otimes x)P_{rs}^{kl}(a).\] They obviously extend to linear maps $\Delta_{rs}$ from $A$ to $A\itimes A$.

\begin{Lem} For each $x\in A$, the element $\Delta(x) = \sum_{rs}\Delta_{rs}(x)$ gives a well-defined multiplier of $A\itimes A$. The resulting map $\Delta: A\rightarrow M(A\itimes A)$ is an $I^2$-coproduct.
\end{Lem}

\begin{proof} As the grading on each $F(u_a)$ is column-finite, it follows at once that for each fixed $p,q$ and $x\in A$, the element $\Delta_{rs}(x)(1\itimes \lambda_p\rho_q)$ is zero except for finitely many $r$ and $s$. Similarly, $(1\itimes \lambda_p\rho_q)\Delta_{rs}(x)$ is zero except for finitely many $r,s$ because of row-finiteness of $F(u_a)$. Hence $\Delta(x)$ is well-defined as a multiplier for each $x\in A$. Once we show that $\Delta$ is multiplicative, it will be immediate that $\Delta$ is coassociative, since each $\Delta_{rs}$ is coassociative. Moreover, also the fact that $\Delta$ then is an $I^2$-morphism is clear from the definition.

To obtain the multiplicativity of $\Delta$, or rather of the coextension $\widetilde{\Delta}$, choose $x\in \Grd{A}{m}{n}(a)$ and $y\in \Grd{A}{n}{q}(b)$. Then \begin{eqnarray*}\wDelta(x)\cdot\wDelta(y) &=& \sum_{rstv}\left(P^{mn}_{rs}(a)(x\otimes 1)\right)\cdot \left(P^{nq}_{tv}(b)(y\otimes 1)\right)\\ &=& \sum_{cd}\sum_{ij}\sum_{klpt} \left(F_{c,i}^{a,b}\otimes F_{d,j}^{a,b}\right)^*\left(e_{a,k}e_{a,l}^*x\otimes e_{b,p}e_{b,t}^*y\otimes e_{a,l}e_{a,k}^*\otimes e_{b,t}e_{b,p}^*\right)\left(F_{c,i}^{a,b}\otimes F_{d,j}^{a,b}\right),\end{eqnarray*} where we may take the sum over all $k,l\in I_a$, $p,t\in I_b$ (and where the composition of operators with mismatching target and source is considered to be zero). Note that the infinite sums are convergent inside $M(A\otimes A)$ by the argument in the first paragraph.

Plugging in the identity operator $\sum_{cd}\sum_{rs} \left(e_{c,r}e_{c,r}^*\otimes e_{d,s}e_{d,s}^*\right)$ at the front, we obtain that the expression becomes \[ \sum_{cd}\sum_{rs}\sum_{ij}\sum_{klpt} X^{c,r,i}_{k,p}Y^{d,s,j}_{l,t}(e_{c,r}\otimes e_{d,s})\left(e_{a,l}^*x\otimes e_{b,t}^*y\otimes e_{a,k}^*\otimes e_{b,p}^*\right)\left(F_{c,i}^{a,b}\otimes F_{d,j}^{a,b}\right)^*\] where $X^{c,r,i}_{k,p} = e_{c,r}^*(F_{c,i}^{a,b})^*(e_{a,k}\otimes e_{b,p})$ and $Y^{d,s,j}_{l,t} = e_{d,t}^*(F_{d,j}^{a,b})^*(e_{a,l}\otimes e_{b,t})$. Resumming over the $k,l,p,t$, we obtain \[\sum_{cd}\sum_{rs}\sum_{ij} \left(e_{c,r}e_{d,s}^* (F_{d,j}^{a,b})^*(x\otimes y)F_{c,i}^{a,b}\right)\otimes \left(e_{d,s}e_{c,r}^* (F_{c,i}^{a,b})^*F_{d,j}^{a,b}\right).\] As $\phi$ is a coisometry and the $f_{c,i}^{a,b}$ are orthonormal, this expression simplifies to \[\sum_{c}\sum_{rs}\sum_{i}e_{c,r}e_{c,s}^* \left((F_{c,i}^{a,b})^*(x\otimes y)F_{c,i}^{a,b}\right)\otimes e_{c,s}e_{c,r}^*,\] which is precisely $\widetilde{\Delta}(x\cdot y)$.
\end{proof}

%Note that the extension of $\Delta$ to a map $\widetilde{\Delta}:A\rightarrow M(A\otimes A)$ is given by the same formula as before.

\begin{Prop} The couple $(A,\Delta)$ is a generalized face algebra over $I$.
\end{Prop}
\begin{proof} Let $\varepsilon$ assign to any $x\in \Gr{A}{k}{l}{m}{n}(a)$ the number $\Tr(x) = \sum_{i\in B_a} (e_{a,i}^*xe_{a,i})$ (where we keep the convention that mismatching operators compose to zero). We claim that $\varepsilon$ is a counit, satisfying the conditions in the definition of a generalized face algebra. The fact that $\varepsilon$ is a counit is immediate from the definition of $\Delta$. It is also computed directly that for $x\in \Gr{A}{k}{l}{m}{n}$ and $y\in \Gr{A}{l}{r}{n}{s}$, we have $\varepsilon(x\cdot y) = \varepsilon(x)\varepsilon(y)$, since the $\{\phi^*F(f_{c,i}^{a,b})e_{c,j}\mid c,i,j\}$ form an orthonormal basis of $F(u_a)\itimes F(u_b)$. From this formula, the second identity for the counit will hold true once we show that \[\varepsilon(\lambda_k\rho_mx\lambda_l\rho_n) = \varepsilon(\lambda_k\rho_mx_{(1)})\varepsilon(x_{(2)}\lambda_l\rho_n).\] But both left and right hand side are zero unless $k=m$, $n=l$ and $x\in \Gr{A}{k}{l}{m}{n}$, in which case both sides equal $\varepsilon(x)$.
\end{proof}

Our next job is to define a suitable antipode for $(A,\Delta)$. Here the rigidity of $\CatC$ will come into play, so we first fix our conventions. Let $a \mapsto \bar{a}$ be the involution induced by the rigidity on the index set $J$. We assume that $\overline{u_a} = u_{\bar{a}}$. For each $u_a$, we will fix duality morphisms $R_a: u_0\rightarrow u_{\bar{a}}\otimes u_a$ and $\bar{R}_a: u_0\rightarrow u_a\otimes u_{\bar{a}}$. By means of $F$ and $\phi$, they induce $I^2$-grading preserving maps $\mathscr{R}_a: l^2(I) \rightarrow F(u_{\bar{a}})\itimes F(u_a)$ and $\bar{\mathscr{R}}_a:l^2(I) \rightarrow F(u_a)\itimes F(u_{\bar{a}})$. These in turn provide an invertible anti-linear map $I_a:F(u_a)_{kl}\rightarrow F(u_{\bar{a}})_{lk}$ and $J_a: F(u_{\bar{a}})_{lk}\rightarrow F(u_{a})_{kl}$ such that $\langle I_{a}\xi_a,\eta_{\bar{a}}\rangle = \sum_r\delta_r^* \bar{\mathscr{R}}_a^*(\xi_a\otimes \eta_{\bar{a}})$ and $\langle J_a\eta_{\bar{a}},\xi_a\rangle = \sum_s \delta_s^* \mathscr{R}_a^* (\eta_{\bar{a}}\otimes \xi_a)$. The snake identities for $R_a$ and $\bar{R}_a$ guarantee that $J_a$ is the inverse of $I_a$.

We define \[S: \Gr{A}{k}{l}{m}{n}(a)\rightarrow \Gr{A}{n}{m}{l}{k}(\bar{a})\] by \[x \mapsto I_ax^*J_a.\]

% Anti-multiplicativity not proven, should be from general theory
\begin{Lem} By means of the map $S$, the couple $(A,\Delta)$ becomes a generalized Hopf face algebra.
\end{Lem}

\begin{proof} It is clear that $S$ is invertible. We also have $S(\lambda_k\rho_l) = \lambda_l\rho_k$ as $I_o\delta_k = \delta_k$.

Let us check that $S$ satisfies the condition $S(x_{(1)})\cdot x_{(2)} = \sum_p \varepsilon(x\cdot \lambda_p)\rho_p$ in the multiplier algebra for $x\in \Gr{A}{k}{l}{m}{n}(a)$. By definition, we have \[S(x_{(1)})\cdot x_{(2)} = \sum_{c}\sum_i\sum_{p,q\in B_a} \left(F_{c,i}^{\bar{a},a}\right)^*\left(I_ae_{a,q}e_{a,p}^*J_a\otimes xe_{a,q}e_{a,p}^*\right)\left(F_{c,i}^{\bar{a},a}\right).\]
%\\ &=& \sum_{c}\sum_i\sum_{rs}\sum_{p\in B_{a,rs},q\in B_{a,kl}}\sum_{vw} X_{q,v} Y_{p,w} \left(F_{c,i}^{\bar{a},a}\right)^*(e_{\bar{a},v}e_{\bar{a},w}^*\otimes x e_{a,q}e_{a,p}^*)\left(F_{c,i}^{\bar{a},a}\right)\end{eqnarray*} where $X_{q,v} = (e_{a,q}^*\otimes e_{\bar{a},v}^*)\bar{\mathscr{R}}_a\delta_k$ and $Y_{p,w} = (e_{\bar{a},w}^*\otimes e_{a,p}^*)\mathscr{R}_a\delta_s$.

Let $C:\C\rightarrow \C$ be complex conjugation. Then we can write $I_ae_{a,q}e_{a,p}^*J_a = (I_ae_{a,q}C)(Ce_{a,p}^*J_a)$. We now calculate, by definition of $J_a$ and $F_{c,i}^{\bar{a},a}$, that \[\sum_{p\in B_a} (Ce_{a,p}^*J_a\otimes e_{a,p}^*)\left(F_{c,i}^{\bar{a},a}\right) =  (R_a^*f_{c,i}^{\bar{a},a})\sum_{s\in I} \delta_s^*,\] since $\phi$ is a coisometry. Plugging this into our expression for $S(x_{(1)})\cdot x_{(2)}$, we obtain \[\sum_s\sum_{q\in B_a}\left(\sum_c\sum_i \left(f_{c,i}^{\bar{a},a})^*R_a\right) F_{c,i}^{\bar{a},a} \right)^* (I_ae_{a,q}C\otimes xe_{a,q}\delta_s^*).\] As $\phi$ is a coisometry and $\phi^*\phi \mathscr{R}_a = \mathscr{R}_a$, we can write $\left(f_{c,i}^{\bar{a},a})^*R_a\right) F_{c,i}^{\bar{a},a} = F_{c,i}^{\bar{a},a}(F_{c,i}^{\bar{a},a})^*\mathscr{R}_a$. As the $f_{c,i}^{\bar{a},a}$ form an orthonormal basis, we thus get \[S(x_{(1)})\cdot x_{(2)} = \sum_s\sum_{q\in B_a} \mathscr{R}_a^*(I_ae_{a,q}C\otimes xe_{a,q}\delta_s^*).\] Now the composition $I_ae_{a,q}C$ is the creation operator for the vector $I_ae_{a,q}$. Hence using again the definition of $J_a$, and using that $x\in \Gr{A}{k}{l}{m}{n}$, we get \begin{eqnarray*} S(x_{(1)})\cdot x_{(2)} &=& \sum_{s}\sum_q \delta_n\delta_s^* e_{a,q}^*xe_{a,q} \\ &=& \sum_s\Tr(x)\delta_n\delta_s^* \\ &=& \sum_p \varepsilon(x\lambda_p)\rho_p,\end{eqnarray*} since $\Tr(x) = \delta_{k,m}\delta_{n,l}\varepsilon(x)$.

The identity $x_{(1)}\cdot S(x_{(2)}) = \sum_p\varepsilon(\rho_px)\lambda_p$ is proven in a similar way.
\end{proof}

In the next step, we determine an invariant functional for $(A,\Delta)$.

\begin{Def} We define $\varphi: \Gr{A}{k}{l}{m}{n} \rightarrow \C$ as the projection onto the component $\Gr{A}{k}{l}{m}{n}(o) \cong \delta_{kl}\delta_{mn} \C$.
\end{Def}

\begin{Lem} The functional $\varphi$ is an invariant normalized functional.
\end{Lem}

\begin{proof} The fact that $\varphi$ is normalized is immediate, so let us check that it is invariant. Let $x\in \Gr{A}{k}{l}{m}{n}(a)$. Then \begin{eqnarray*} (\id\otimes \varphi)\widetilde{\Delta}(x) &=& \sum_{i,j}  \varphi(e_{a,j}e_{a,i}^*)e_{a,i}e_{a,j}^*x \\ &=& \delta_{a,o} \sum_{r,s}\delta_r\delta_s^* x \\ &=& \varphi(x)\sum_r\delta_r\delta_k^*  \\ &=& \sum_p\varphi(\lambda_px)\lambda_p.\end{eqnarray*}

The proof of right invariance follows similarly.
\end{proof}

Finally, we introduce the $^*$-structure and show that $(A,\Delta)$ is a generalized compact Hopf face algebra. To distinguish the new $^*$-operation from the ordinary operator algebraic one, we will denote it by $\dagger$.

\begin{Def} We define the anti-linear map $\dagger: \Gr{A}{k}{l}{m}{n} \rightarrow \Gr{A}{m}{n}{k}{l}$ by the formula \[ x^{\dagger} = S(x^*) \]
\end{Def}

\begin{Lem} The map $x\mapsto x^{\dagger}$ is an anti-multiplicative anti-linear involution on $A$.
\end{Lem}

\begin{proof} It is clear that $x\mapsto x^{\dagger}$ is anti-linear. It is also immediate from the definition of the product that $(x\cdot y)^* = x^*\cdot y^*$. Together with the anti-multiplicativity of $S$, this proves the anti-multiplicativity of $\dagger$.

Let us proof that $\dagger$ is an involution. It is sufficient to prove that $I_{\bar{a}}I_a= \lambda\id$ and $J_aJ_{\bar{a}} = \lambda^{-1}\id$ for some scalar $\lambda$. But this follows from the fact that $(\bar{R}_a,R_a)$ and $(R_{\bar{a}},\bar{R}_{\bar{a}})$ are both solutions to the conjugate equations for $\bar{a}$.
\end{proof}

The last property which needs to be proven is the positivity of $\varphi$. For this, recall that $R_a^*R_a$ and $\bar{R}_a^*\bar{R}_a$ are scalars as $u_a$ is irreducible. One can then rescale $R_a$ and $\bar{R}_a$ such that the scalar in both expressions is the same. This scalar is then a uniquely determined number $\dim_q(a)$, called the \emph{quantum dimension} of $a$. It follows that $\frac{1}{\dim_q(a)}F(R_aR_a^*)$ is the projection of $F(u_{\bar{a}}\otimes u_a)$ onto the copy of $F(u_o)$ inside, and a similar statement holds for $\bar{R}_a$.

\begin{Prop} For any $x\in A$, the scalar $\varphi(x^{\dagger}\cdot x)$ is positive.
\end{Prop}

\begin{proof} It is straightforward to see that the blocks $\Gr{A}{k}{l}{m}{n}$ are mutually orthogonal, and that moreover the spectral subspaces inside are mutually orthogonal. Let then $\xi,\zeta \in F(u_a)_{kl}$ and $\eta,\mu\in F(u_a)_{mn}$. We have, using the remark above, \begin{eqnarray*} \varphi(y^{\dag}\cdot x) &=& \varphi(\sum_c\sum_i \left(F_{c,i}^{\bar{a},a}\right)^*(I_ayJ_a\otimes x)\left(F_{c,i}^{\bar{a},a}\right))\\ &=& \delta_n^* \sum_i \left(F_{o,i}^{\bar{a},a}\right)^*(I_ayJ_a\otimes x)\left(F_{o,i}^{\bar{a},a}\right)\delta_l\\ &=& \frac{1}{\dim_q(u_a)} \delta_n^* \mathscr{R}_a^*(I_ayJ_a\otimes x)\mathscr{R}_a\delta_l \\ &=& \frac{1}{\dim_q(u_a)}\sum_{p,q}\delta_n^*\mathscr{R}_a^*(I_ayJ_ae_{\bar{a},p}e_{\bar{a},p}^*\otimes xe_{a,q}e_{a,q}^*)\mathscr{R}_a\delta_l.\end{eqnarray*} By the defining properties of $I_a$ and $J_a$, this expression becomes $\dim_q(u_a)^{-1}\sum_{p} \langle e_{\bar{a},p},J_a^*x^*yJ_ae_{\bar{a},p}\rangle$, thus clearly $\varphi$ will be positive on $A$.
\end{proof}

\section*{Corepresentations of generalized compact Hopf face algebras}

Let $(A,\Delta)$ be a generalized compact Hopf face algebra over an index set $I$. A \emph{locally finite-dimensional unitary corepresentation} of $(A,\Delta)$ consists of a row and column-finite $I^2$-graded Hilbert space $\Hsp = \osumc{k,l\in I}\Hsp_{kl}$ together with elements $\Gr{U}{k}{l}{m}{n}\in \Gr{A}{k}{l}{m}{n}\otimes B(\Gru{\Hsp}{m}{n},\Gru{\Hsp}{k}{l})$ such that \[\sum_{k} \left(\Gr{U}{k}{l}{m}{n}\right)^*\Gr{U}{k}{l}{m}{n} = \lambda_l\rho_n \otimes \id_{\Gru{\Hsp}{m}{n}}\] and \[\sum_{n} \Gr{U}{k}{l}{m}{n}\left(\Gr{U}{k}{l}{m}{n}\right)^* = \lambda_k\rho_m \otimes \id_{\Gru{\Hsp}{k}{l}},\] and \[(\wDelta\otimes \id)(\Gr{U}{k}{l}{m}{n}) = \sum_{p,q} \left(\Gr{U}{k}{l}{p}{q}\right)_{13}\left(\Gr{U}{p}{q}{m}{n}\right)_{23}.\] Note that in the first two identities, the sums are finite, while in the finite identity the possibly infinite sum is meaningful inside the multiplier algebra sense.

By a morphism between two locally finite-dimensional unitary corepresentations $(\mathscr{H},U)$ and $(\mathscr{G},V)$ is meant a grading-preserving bounded map $T = \osumc{k,l} \Gru{T}{k}{l}:\mathscr{H}\rightarrow \mathscr{G}$ such that $(1\otimes \Gru{T}{k}{l})\Gr{U}{k}{l}{m}{n} = \Gr{V}{k}{l}{m}{n}(1\otimes \Gru{T}{m}{n})$. The collection of all locally finite-dimensional unitary corepresentations clearly forms a semi-simple C$^*$-category $\Corep(A)$. We will say that $(A,\Delta)$ is \emph{of finite type} if the morphisms in $\Corep(A)$ are finite-dimensional.

% Annoying change of order by conventions followed... Necessary to repair?
% of finite type follows if unit object is irreducible. This is easy to characterize (biconnectedness?)
One can define a tensor product $\Circt$ between locally finite-dimensional corepresentations by means of the $\itimes$-product of bigraded Hilbert spaces and the operation
\[\Gr{(U\Circt V)}{k}{l}{m}{n} = \left(\Gr{U}{k}{r}{m}{s}\right)_{12}\left(\Gr{V}{r}{l}{s}{n}\right)_{13}.\] In this way, the category $\Corep(A)$ becomes a monoidal category. The unit object consists of the $I^2$-graded Hilbert space $l^2(I)$ together with the elements $\Gr{U}{k}{l}{m}{n}= \delta_{kl}\delta_{mn}\lambda_k\rho_m\otimes 1$.

Assume now that $\CatC$ is a semi-simple tensor C$^*$-category with irreducible unit, and $F:\CatC\rightarrow \Hilb$ a strong tensor C$^*$-functor. Let $(A,\Delta)$ be the associated generalized compact Hopf face algebra. Let us show that $\CatC\cong \Corep(A)$ by means of an equivalence functor $G$.

For $X$ an object of $\CatC$, we build a locally finite-dimensional unitary corepresentation $U$ on $F(X)$. Consider the canonical isomorphism $F(X) \cong \oplus_{a\in J} X_a \otimes \Mor(X_a,X)$. Let \[\Gr{U}{k}{l}{m}{n}(a) \in  \Gr{A}{k}{l}{m}{n}(a)\otimes B(F(u_a)_{mn},F(u_a)_{kl})= B(F(u_a)_{kl},F(u_a)_{mn})\otimes B(F(u_a)_{mn},F(u_a)_{kl})\] be determined as the element implementing the non-degenerate pairing $B(F(u_a)_{kl},F(u_a)_{mn})\otimes B(F(u_a)_{nm},F(u_a)_{lk})\rightarrow \C$ sending $S\otimes T$ to $\Tr(ST)$. Using notation as before, this means that \[\Gr{U}{k}{l}{m}{n}(a) = \sum_{p\in B_{a,mn},q\in B_{a,kl}} e_pe_q^*\otimes e_qe_p^*.\]



\section*{Monoidal equivalence of generalized compact Hopf face algebras}

Let $(A,\Delta)$ be a generalized Hopf face algebra over a set $I$. Assume that $I = I_1\sqcup I_2$, and let $\Lambda_j = \sum_{i\in I_j}\lambda_i$, resp. $\Rho_j = \sum_{i\in I_j} \rho_j$. If the $\Lambda_j$ and $\Rho_j$ are central in $M(A)$, then we can write $A = \osum{i,j} A(ij)$ where $A(ij) = \Lambda_i\Rho_jA$ are subalgebras. Moreover, the comultiplication $\wDelta$ splits into comultiplications \[\wDelta_{ij}^k:A(ij)\rightarrow M(A(ik)\otimes A(kj))\textrm{ s.t. } \wDelta = \wDelta_{ij}^1 +\wDelta_{ij}^2 \textrm{ on }A(ij).\] A similar decomposition holds for $\Delta$.

It is immediate to see that the $(A(ii),\Delta_{ii}^i)$ are two generalized Hopf face algebras over the respective $I_i$.

\begin{Def} We say $(A,\Delta)$ is a \emph{co-linking generalized (compact) Hopf face algebra} between $(A(11),\Delta_{11}^1)$ and $(A(22),\Delta_{22}^2)$ if $\lambda_i\Rho_2\neq 0$ for any $i\in I_1$.
\end{Def}

Upon applying the antipode, we see that then $\rho_j\Lambda_1\neq 0$ for any $j\in I_2$ as well.

\begin{Def} Two generalized (compact) Hopf face algebras are called \emph{comonoidally Morita equivalent} if they are isomorphic to the components $(A_{ii},\Delta_{ii}^i)$ of some co-linking generalized (compact) Hopf face algebra.\end{Def}

As an example, consider two sets $I_i$, and two tensor functors $(F_i,\phi_i)$ of a semi-simple rigid C$^*$-category $\CatC$ with irreducible unit into $\Hilb_{I_i^2}$. Then with $I= I_1\sqcup I_2$, we can form a new C$^*$-functor $F=F_1\oplus F_2$ of $\CatC$ into $\Hilb_{I^2}$ by putting $F(X) = F_1(X)\oplus F_2(X)$ with the obvious $I^2$-grading (and the obvious direct sum operation on morphisms). It becomes monoidal by means of the unitaries \[F(X\otimes Y) = F_1(X\otimes Y)\oplus F_2(X\otimes Y) \underset{\phi_1\oplus \phi_2}{\cong} (F_1(X)\underset{I_1}{\otimes} F_1(Y)) \oplus (F_2(X)\otimes F_2(Y)) \cong F(X)\itimes F(Y)\] (where the last map is unitary since $(F(X)\itimes F(Y))_{ij}=0$ for example for $i\in I_1$ and $j\in I_2$).

If we then consider the generalized compact Hopf face algebra $(A,\Delta)$ associated to $F$, we have immediately from the construction that the $\Lambda_i$ and $\Rho_i$ associated to the decomposition $I = I_1\sqcup I_2$ are indeed central elements in $M(A)$. Moreover, the parts $(A_{ii}^i,\Delta_{ii}^i)$ are seen to arise from applying the Tannaka-Krein construction to the respective functors $F_1$ and $F_2$. The fact that $(A,\Delta)$ is co-linking is immediate from the fact that \emph{none} of the $\lambda_i\rho_j$ are zero in this particular case (since $\Gr{A}{k}{k}{m}{m}(o) = B(F(u_o)_{kk},F(u_o)_{mm}) \cong \C$).

We will exploit the above extra structure in the following section to say something about the algebra $A$ appearing in ... This is the component $\tilde{A}(1,1)$ of the above algebra. The following lemma will be needed.

\begin{Lem} Assume $(A,\Delta)$ is a co-linking generalized Hopf face algebra. Then any of the maps $\wDelta_{ij}^k$ is injective.\end{Lem}

\begin{proof} Take a non-zero $x\in A_n(ij)$ where $n\in I_j$. Then for any $l\in I$ with $\rho_n\lambda_l\neq 0$, we know that $\wDelta(x)(1\otimes \rho_n\lambda_l)\neq 0$. Hence $\wDelta_{ij}^k(x)(1\otimes \rho_n\lambda_l)\neq 0$ for $l\in I_k$, and hence $\wDelta_{ij}^k(x)\neq 0$. Now if $j=k$, the condition $\rho_n\lambda_l\neq 0$ is satisfied by taking $l=n$ (since $\varepsilon(\lambda_n\rho_n)=1$). If $j\neq k$, it is satisfied for at least one $l$ by the co-linking assumption.
\end{proof}


%%% Local Variables: 
%%% mode: latex
%%% TeX-master: "dyn-suq-main"
%%% End: 

