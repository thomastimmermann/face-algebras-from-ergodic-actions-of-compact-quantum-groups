\section{Compact Hopf face algebras on the level of operator algebras}


Let $\mathscr{G}$ be a partial compact quantum group. We now construct
completions of the underlying $*$-algebra $P(\mathscr{G})$ in the form
of a universal $C^{*}$-algebra $\CuG$, a reduced $C^{*}$-algebra
$\CrG$ and a von Neumann algebra $\LGinf$. The existence of the first
one follows from the Peter-Weyl theorem \ref{}, and the second and
third one arise from a GNS-representation of $P(\mathscr{G})$ on the
Hilbert space $\LGtwo$ associated to the invariant functional of
$\mathscr{G}$.  We then lift the comultiplication, the invariant
functional, the unitary antipode and the scaling group to level of
operator algebras and show that $\LGinf$ becomes a measured quantum
groupoid in the sense of Lesieur \cite{} and Enock \cite{}.

Let us start with the construction of $\CuG$. Denote by $A$ the
underlying total $*$-algebra of the partial $*$-algebra
$P(\mathscr{G})$ and define a map $|\cdot |_{u} \colon A \to [0,\infty]$ by
\begin{align*}
  |a|_{u}&:= \sup \{ \|\pi(a)\| : \pi \text{ is a $*$-homomorphism
    from } A \text{ into some $C^{*}$-algebra } B\}.
\end{align*}
\begin{Lem}
  $|a|_{u}<\infty$ for each $a \in A$. 
\end{Lem}
\begin{proof}
  By Corollary \ref{cor:rep-pw}, we can write each $a\in A$ in the
  form $a=(\id \otimes \omega_{\xi,\eta})(X(K))$, where $X$ is a
  unitary sfd corepresentation of $P(\mathscr{G})$ on some sfd
  $I^{2}$-graded Hilbert space $\mathcal{H}$ and
  $K=\pmat{k}{l}{m}{n}\in M_{2}(I)$, $\xi\in \Gru{\mathcal{H}}{m}{n}$,
  $\eta\in \Gru{\mathcal{H}}{k}{l}$.  Since $X$ is unitary,
  \begin{align*}
    \sum_{p}X\pmat{p}{l}{m}{n}^{*} X\pmat{p}{l}{m}{n}  = \lambda_{l}\rho_{n}
    \otimes \id_{\Gru{\mathcal{H}}{m}{n}}
  \end{align*}
  by \ref{}, where the sum is finite because $X$ is sfd. As
  $\pi(\lambda_{k}\rho_{m})\in \C$ is a projection, we can conclude
  \begin{align*}
    \|\pi(a)\| &\leq \| \xi| \|\eta\| \|(\pi \otimes \id)(X(K))\| \\
    &=\|\xi\|\|\eta\| \|(\pi \otimes \id)(X(K)^{*}X(K))\| \leq
    \|\xi\|\|\eta\| \|\pi(\lambda_{l}\rho_{n}) \otimes
    \id_{\Gru{\mathcal{H}}{m} {n}}\| = \|\xi\|\|\eta\|.  
  \end{align*}
\end{proof}
Clearly, $|\cdot|_{u}$ defines a $C^{*}$-semi-norm on
$P(\mathscr{G})$, and the separated completion of $P(\mathscr{G})$
with respect to this norm is a $C^{*}$-algebra. We denote this
$C^{*}$-algebra by $\CuG$. By construction, every $*$-homorphism $\pi$
of $A$ into some $C^{*}$-algebra $C$ factorises through $\CuG$. 
We shall see that the canonical $*$-homomorphism $\pi_{u} \colon A \to
\CuG$ is injective.
\begin{Rem}
  The universal property of $\CuG$ implies that the modular
  automorphism group $\sigma$ and the scaling group $\tau$ for real
  parameters and the unitary antipode $R$ of $\mathscr{G}$ introduced
  after Corollary \ref{cor:rep-characters} lift to one-parameter
  groups $\tau^{u},\sigma^{u}$ and a $*$-anti-automorphism $R_{u}$ of
  $\CuG$, that is, $\tau^{u}_{t} \circ \pi_{u}= \pi_{u} \circ \tau_{t}$,
  $\sigma^{u}_{t} \circ \pi_{u} = \pi_{u} \circ \sigma_{t}$, $R_{u}
  \circ \pi_{u} = \pi_{u} \circ R$. Corollary \ref{cor:rep-characters}
  5.\ and A.1 in \cite{} [Takesaki:2]  imply that elements in
  $\pi_{u}(A)$ are analytic for $\tau^{u}$ and $\sigma^{u}$; in
  particular, $\tau^{u}$ and $\sigma^{u}$ are strongly continuous.  
\end{Rem}


We now turn to the construction of the reduced $C^{*}$-algebra $\CrG$
and the von Neumann algebra $\LGinf$. 
Denote by $\LGtwo$ the completion of $A$ with respect to the norm
associated to the inner product given by
\begin{align*}
  \langle a|b\rangle :=\phi(a^{*}b) \quad \text{for all } a,b\in A,
\end{align*}
and by $\Lambda \colon A \to \LGtwo$ the natural embedding.  This
product is definite because $\phi$ is faithful by \ref{LemFaith}, and
it extends to the space $\LGtwo$ which thus becomes a Hilbert space.
As such, $\LGtwo$ is the
orthogonal direct sum of the subspaces
$\Lambda(A(K)) \subseteq \LGtwo$, where $K\in M_{2}(I)$, because
$\phi(A(K)^{*}A(L)) = 0$ if $K\neq L$ by \ref{}.  In particular, there
exist  operators
$\lambda_{k},\lambda_{k}^{\op},\rho_{k},\rho_{k}^{\op}\in {\cal
  B}(\LGtwo)$ for each $k\in I$ such that
\begin{align*}
  \lambda_{k}\Lambda(a)&= \Lambda(\lambda_{k} a), &
  \lambda^{\op}_{k}\Lambda(a) &= \Lambda(a\lambda_{k}), &
  \rho_{k}\Lambda(a) &= \Lambda(\rho_{k}a), &
  \rho_{k}^{\op}\Lambda(a) &= \Lambda(a\rho_{k})
\end{align*}
for all $a\in A$, and faithful, normal $*$-homomorphisms
\begin{align} \label{eq:vn-lambda-rho}
  \lambda,\rho \colon l^{\infty}(I) \to
  {\cal B}(\LGtwo)
\end{align}
that send the delta function at
$k\in I$ to the operators $\lambda_{k}$ or $\rho_{k}$, respectively. 

 Define $\vnE,\overline{G} \in {\cal B}(\LGtwo \otimes
\LGtwo)$ by
\begin{align*}
  \vnE &:=\sum_{k} \rho_{k} \otimes \lambda_{k}, & 
  \overline{G} &:= \sum_{k} \rho_{k}^{\op} \otimes \rho_{k},
\end{align*}
where the sums converge with respect to the strong operator
topology.
\begin{Lem} \label{lemma:partial-isometry}
There exists a unique partial isometry $V$ on $\LGtwo \otimes \LGtwo$
such that
\begin{align*}
  V(\Lambda(a) \otimes \Lambda(b)) = \Lambda(a_{(1)}) \otimes \Lambda(a_{(2)}b)
\end{align*}
for all $a,b\in A$. Its range and domain projections are given by $VV^{*} = \vnE$
and $V^{*}V = \overline{G}$.
\end{Lem}
\begin{proof}
  Let $a,b \in A$. Since $\Delta$ is a $*$-homomorphism and $\phi$ is
invariant,
  \begin{align*}
    \langle \Lambda(a_{(1)}) \otimes
    \Lambda(a_{(2)}b)|\Lambda(a'_{(1)}) \otimes
    \Lambda(a'_{(2)}b')\rangle &=
    \phi(a_{(1)}^{*}a'_{(1)})\phi(b^{*}a_{(2)}^{*}a'_{(2)}b') \\
    &= \sum_{p}
    \phi(b^{*}\rho_{p}\phi(\rho_{p}a^{*}a'\rho_{p})\rho_{p}b') \\
    & =\sum_{p} \langle\Lambda(a\rho_{p}) \otimes \Lambda(\rho_{p}b) |
    \Lambda(a'\rho_{p}) \otimes \Lambda(b'\rho_{p})\rangle.
  \end{align*}
  Now, the assertion follows from Proposition \ref{prop:riti}.
\end{proof}

\begin{Prop} \label{prop:gns} Let $\mathscr{G}$ be a partial compact
  quantum group with underlying total $*$-algebra $A$ and associated
  Hilbert space $\LGtwo$. Then there exists a unique $*$-homomorphism
  $\pi_{r}\colon A \to {\cal B}(\LGtwo)$ such that
  $\pi_{r}(a)\Lambda(b)=\Lambda(ab)$ for all $a,b\in A$, and this
  $\pi_{r}$ is faithful.
\end{Prop}
\begin{proof} 
  Let $a,c \in A$. Then the formula $x \mapsto \langle
\Lambda(c) | x\Lambda(a)\rangle$ defines a bounded linear functional
  $\omega_{\Lambda(c),\Lambda(a)}$ on ${\cal B}(\LGtwo)$ and a
  straightforward computation shows that
  \begin{align} \label{eq:vn-slice}
    (\omega_{\Lambda(c),\Lambda(a)}\otimes \id)(V)\Lambda(b) =
    \Lambda(\varphi(c^*a_{(1)})a_{(2)}b)
  \end{align}
  for all $b\in A$. Therefore, left multiplication by
  $\varphi(c^*a_{(1)})a_{(2)}$ extends to a bounded linear operator on $\LGtwo$.
 Since $(A\otimes 1)\Delta(A) = (A\otimes
  A)\Delta(1)$ by Proposition \ref{prop:riti} and $\phi$ is
  normalized,  elements of the form $\phi(c^{*}a_{(1)})a_{(2)}$ span
  $A$. 
\end{proof}
\begin{Cor}
  Let $\mathscr{G}$ be a partial compact quantum group with underlying
  total algebra $A$. Then the
  canonical $*$-homomorphism $\pi_{u} \colon A \to\CuG$ is injective.
\end{Cor}
\begin{proof}
  The injective $*$-homomorphism $\pi_{r}$ factorises through
  $\pi_{u}$.
\end{proof}
Given a partial compact quantum group $\mathscr{G}$, we call
$(\LGtwo,\Lambda,\pi)$ the \emph{associated GNS-construction} and denote by
\begin{align}
  \CrG &\subseteq {\cal B}(\LGtwo) &&\text{and} & \LGinf &\subseteq {\cal B}(\LGtwo)
\end{align}
the $C^{*}$-algebra and the von Neumann algebra generated by $\pi_{r}(A)
\subseteq \LGtwo$, respectively, and identify $M(\CrG)$ with a
$C^{*}$-subalgebra of $\LGtwo$.  Since $\pi_{r}$ extends to a
$*$-homomorphism on $\CuG$, we get a sequence of $*$-homomorphisms
\begin{align*}
A \hookrightarrow \CuG \to 
  \CrG \hookrightarrow M(\CrG) \hookrightarrow
\LGinf \hookrightarrow {\cal B}(\LGtwo).
\end{align*}
Note that
 the $*$-homomorphisms $\lambda,\rho$ in
\eqref{eq:vn-lambda-rho} send $l^{\infty}(I)$ to $M(\CrG)$, and that
\begin{align*}
  \vnE \in M(\CrG \otimes \CrG) \subseteq \LGinf \otimes \LGinf
  \subseteq {\cal B}(\LGtwo \otimes \LGtwo),
\end{align*}
where $\otimes$ denotes the minimal tensor product
of $C^{*}$-algebras, the tensor product of von Neumann algebras, and
the tensor product of Hilbert spaces, respectively.

Consider the map 
\begin{align*}
  \vnDelta \colon \LGinf \to {\cal B}(\LGtwo \otimes \LGtwo), \ x
  \mapsto V(x \otimes 1)V^{*}.
\end{align*}
\begin{Lem} \label{lemma:vn-delta}
  \begin{enumerate}
  \item $\vnDelta(\pi_{r}(a)) (\Lambda(b) \otimes \Lambda(c)) =
    \Lambda(a_{(1)}b) \otimes \Lambda(a_{(2)}b)$ for all $a,b,c\in A$;
  \item $\vnDelta$ is a normal, faithful $*$-homomorphism;
  \item  $\vnDelta(\CrG) \subseteq \vnE M(\CrG \otimes
  \CrG)\vnE$ and $\vnDelta(\LGinf) \subseteq \vnE(\LGinf \otimes
  \LGinf)\vnE$.
  \end{enumerate}
\end{Lem}
\begin{proof}
  The equation in 1.{} is easily verified. The map $\vnDelta$ is
  normal by construction, a $*$-homo\-morphism by 1.{}, and faithful
  because $\vnDelta(x)=0$ implies $x\otimes 1=0$ on
  $V^{*}V(L^{2}(\mathscr{G}) \otimes L^{2}(\mathscr{G}))$ and hence
  $x=0$ on $\bigoplus_{k}
  \rho_{k}^{\op}L^{2}(\mathscr{G})=L^{2}(\mathscr{G})$. Finally, 3.\
  follows from the relation $\Delta(a)=E\Delta(a)E$, which holds for
  all $a\in A$.
\end{proof}


Next, we lift the invariant functional $\phi$ of $\mathscr{G}$ to
$\LGinf$ and define associated operator-valued weight
$T_{\lambda},T_{\rho}$ from $\LGinf$ to $l^{\infty}(I)$. Since $\phi$ is normalized, each
$\Lambda(\lambda_{k},\rho_{m})$ is a unit vector and the associated
vector functional
\begin{align*}
  \vnphic{k}{m}\colon \LGinf \to  \C, \quad x \mapsto \langle\Lambda(\lambda_{k}\rho_{m})|x\Lambda(\lambda_{k}\rho_{m})\rangle
\end{align*}
is a state.  Then the formulas
\begin{align} \label{eq:vn-weights}
  \vnphi(x) &:= \sum_{k,m} \vnphic{k}{m}(x), &
    T_{\lambda}(x)&:= \sum_{k,m}
\vnphic{k}{m}(x)\lambda_{k}, & 
T_{\rho}(x)&:=
    \sum_{k,m} \vnphic{k}{m}(x)\rho_{m},
\end{align}
where $x\in \LGinf_{+}$, define a a normal semi-finite weight $\vnphi$
on $\LGinf$ and normal semi-finite conditional expectations $T_{\lambda}$ and
$T_{\rho}$ from $\LGinf$ to $\lambda(l^{\infty}(I))$ and
$\rho(l^{\infty}(I))$, respectively. These maps are determined by
their restrictions to $\pi_{r}(A)$:
\begin{Lem} \label{lemma:vn-weights-unique} The normal weight $\vnphi$ and
  the normal conditional expectations $T_{\lambda},T_{\rho}$ satisfy $\pi_{r}(A) \subseteq
  \mathfrak{M}_{\vnphi} \cap \mathfrak{M}_{T} \cap \mathfrak{M}_{T'}$
  and
  \begin{align*}
    \vnphi(\pi_{r}(a))&= \phi(a), & T_{\lambda}(\pi_{r}(a))\Lambda(b) &=
    \sum_{k}\Lambda(\phi(\lambda_{k}a)\lambda_{k}b), &
    T_{\rho}(\pi_{r}(b))\Lambda(b) &= \sum_{m}\Lambda(\phi(\rho_{m}a)\rho_{m}b)
  \end{align*}
  for all $a,b\in A$, and are uniquely determined by these equations. 
\end{Lem}
\begin{proof}
  The equations follow immediately from the definition and the
  relation $\phi(a)=\sum_{k,m}
  \phi(\lambda_{k}\rho_{m}a\lambda_{k}\rho_{m})$, see \ref{}. To prove
  uniqueness, observe that the $p_{k,m}:=\pi_{r}(\lambda_{k}\rho_{m})$ are
  pairwise orthogonal projections in $\mathfrak{M}_{\vnphi}
  \cap \mathfrak{M}_{T_{\lambda}} \cap \mathfrak{M}_{T_{\rho}}$
  summing up to $1$, whence $\vnphi$, $T_{\lambda}$ and $T_{\rho}$ are
  the sums of the bounded linear maps that send an $x\in \LGinf_{+}$
  to $\vnphi(p_{k,m}xp_{l,n})$, $T_{\lambda}(p_{k,m}xp_{l,n})$, or
  $T_{\rho}(p_{k,m}xp_{l,n})$, respectively, which are determined by
  their restriction to $\pi_{r}(A)$.
\end{proof}

Invariance of $\phi$ implies invariance of $\vnphi$ as follows.
\begin{Prop} \label{prop:vn-invariance}
  Let $\mathscr{G}$ be a partial compact quantum group. Then for all
  $x\in \LGinf_{+}$, the
  normal, semi-finite weight $\vnphi$ on $\LGinf$ satisfies
  \begin{align*}
    (\id \otimes \vnphi)(\vnDelta(x)) &=  T_{\lambda}(x), &
    (\vnphi \otimes \id)(\vnDelta(x)) &= T_{\rho}(x).
  \end{align*}
\end{Prop}
\begin{proof}
  Let  $a \in A$. Then 
 \eqref{eq:integral} and the relation
  $\vnphic{k}{m}\circ \pi = \phic{k}{m}$ imply
  \begin{align*}
    (\id \otimes \vnphic{l}{m})(\vnDelta(\pi_{r}(a))) &= \sum_{k}
    \vnphic{k}{m}(\pi_{r}(a)) \lambda_{k}\rho_{l}.
  \end{align*}
  Since each $\vnphic{k}{m}$ is a vector state and $\pi_{r}(A)$ is
  weakly dense in $\LGinf$,  this equations
  remains true if we replace $\pi_{r}(a)$ by arbitrary $x\in
  \LGinf$. Summing over $l$ and $m$, we  obtain the first equation
  which we have to prove. The second one follows similarly.
\end{proof}
The next result gives a fairly complete description of the objects of
Tomita-Takesaki theory associated to $\vnphi$.
\begin{Lem} \label{lemma:vn-hilbert} The subspace
  $\Lambda(A) \subseteq \LGtwo$ is a Hilbert algebra with respect to
  the operations $\Lambda(a)\Lambda(b)=\Lambda(ab)$ and
  $\Lambda(a)^{*}= \Lambda(a^{*})$ for all $a,b\in A$, and a Tomita
  algebra with respect to the family of operators $\nabla_{z}$ given
  by $\nabla_{z}\Lambda(a)=\Lambda(\sigma_{z}(a))$ for all $a\in A$,
  $z\in \C$.  The associated left von Neumann
  algebra is $\LGinf$, the associated normal, semifinite, faithful
  weight is $\vnphi$, the modular operator  $\Delta_{\vnphi}$ is the
  closure of $\nabla_{-i}$,  the modular conjugation $J_{\vnphi}$ is
  given by $J_{\vnphi}\Lambda(a)=\Lambda(\sigma_{i/2}(a)^{*})$ for all
  $a\in A$, and the modular automorphism group $\sigma^{\vnphi}$
  satisfies $\sigma^{\vnphi}_{t} \circ \pi_{r} = \pi_{r} \circ
  \sigma_{t}$ for all $t\in \R$.
\end{Lem}
\begin{proof}
  We first show that $\Lambda(A)$ is a Hilbert algebra. Indeed, the
  map $\pi_{r}(a)\colon \Lambda(b) \to \Lambda(ab)$ is bounded for
  each $a \in A$ by Proposition \ref{prop:gns}, and the involution is
  pre-closed because  for all $a,b \in A$,
  \begin{align*}
    \langle \Lambda(a)|\Lambda(b^{*})\rangle = \phi(a^{*}b^{*}) =
    \phi(b^{*}\sigma(a^{*})) = \langle
    \Lambda(b)|\Lambda(\sigma(a^{*}))\rangle
  \end{align*}

To see that $\Lambda(A)$ and $(\nabla_{z})_{z}$ form a Tomita
  algebra, we have to verify that the map
  $z\mapsto
  \langle \Lambda(a)|\nabla_{z}\Lambda(b)\rangle =
  \phi(a^{*}\sigma_{z}(b))$ is entire for all $a,b\in A$ and that
  \begin{align*}
    \nabla_{z}\Lambda(a)^{*} &= \nabla_{\overline{z}}\Lambda(a)^{*}, &
    \langle \Lambda(a)|\Lambda(b)\rangle &= \langle
    \nabla_{-\overline{z}}\Lambda(a) |\Lambda(b)\rangle, & \langle
    \Lambda(a)^{*}|\Lambda(b)^{*}\rangle = \langle \Lambda(b)|\nabla_{-i}\Lambda(a)\rangle
  \end{align*}
  for all $a,b\in A$, $z\in \C$. But all of this follows immediately
  from Corollary \ref{cor:rep-characters}.


  The left von Neumann algebra of $\Lambda(A)$ is
  $\pi_{r}(A)''=\LGinf$ and the associated weight $\tilde\phi$
  satisfies $\tilde
  \phi(\pi_{r}(a^{*}a))=\langle\Lambda(a)|\Lambda(a)\rangle =
  \phi(a^{*}a)$ for all $a\in A$. By Lemma
  \ref{lemma:vn-weights-unique}, it coincides with $\vnphi$.  By
  \cite{} [Takesaki:2, Thm. VI.2.2 and its proof], the modular
  operator $\Delta_{\vnphi}$ is the closure of $\nabla_{-i}$ and the
  modular automorphism group is implemented by $(\nabla_{t})_{t}$. 
\end{proof}
 The general theory of Hilbert algebras \cite{} implies now:
\begin{Prop} \label{prop:hilbert-algebra} Let $\mathscr{G}$ be a
  partial compact quantum group. Then the extension $\vnphi$ of the
  invariant functional to $\LGinf$ is faithful. \qed
\end{Prop}
\begin{Rem}
  Without using the theory of Hilbert algebras, one could also check
  directly that the formula for $J_{\vnphi}$ defines an anti-linear
  isometry, that $J_{\vnphi}\pi_{r}(A)J_{\vnphi}$ commutes with $\pi_{r}(A)$ and hence
  with $\LGinf$, and that the family
  $(\Lambda(\lambda_{k}\rho_{m}))_{k,m}$ is cyclic for
  $J\pi_{r}(A)J$. Then this family is separating for $\LGinf$ and
  $\vnphi$ is faithful.
\end{Rem}

The scaling group $\tau$ and the unitary antipode $R$ of $\mathscr{G}$
can easily be lifted to $\CrG$ and $\LGinf$ using the following
result. Let us call a conjugate-linear map on a Hilbert space an
\emph{anti-symmetry} if it is isometric and its square is the
identity.
\begin{Lem} \label{lemma:vn-implementation}
  There exist a unique anti-symmetry $I$ and a strongly continuous
  one-parameter group $P=(P_{t})_{t}$ on
  on $\LGtwo$ such that for all $a\in A$, $t\in \R$,
  \begin{align*}
 I\Lambda(a) &=
    \Lambda(R(a)^{*}), & P_{t}\Lambda(a) &= \Lambda(\tau_{t}(a)).
  \end{align*}
\end{Lem}
\begin{proof}
  Corollary \ref{cor:rep-characters} implies that the formulas above
  define an anti-symmetry $I$ and unitaries $P_{t}$; for
  example, $\|I\Lambda(a)\|^{2})=\phi(R(a)R(a)^{*})=
  \phi(a^{*}a)=\|\Lambda(a)\|^{2}$, and $I^{2} = \id$
  because $*\circ R \circ * \circ R=  R^{2}=\id$. By A.1 in \cite{}
  [Takesaki:2] and Corollary  \ref{cor:rep-characters} 5., elements of
  $\Lambda(A)$ are analytic with respect to $P$; in
  particular,  $P$ is strongly continuous.
\end{proof}
\begin{Prop}
  Let $\mathscr{G}$ be an $I$-partial compact quantum group.  
  \begin{enumerate}
  \item  There exists a unique $*$-anti-automorphism $\vnR$ of
    $\LGinf$ such that $\vnR \circ \pi_{r} = \pi_{r} \circ R$.
    This $\vnR$ restricts to a $*$-anti-automorphism of
    $\CrG$.
  \item  There exists a unique strongly continuous
    one-parameter group $\vntau$ on $\LGinf$ such that $\vntau_{t}
    \circ \pi_{r} = \pi_{r} \circ \theta_{-it,it}$ for all $t\in \R$,
    and this $\vntau$ restricts to a strongly continuous one-parameter
    group on $\CrG$.
  \end{enumerate}
\end{Prop}
\begin{proof}
    Short calculations show that the maps $\vnR \colon x \mapsto
  Ix^{*}I$ and $\vntau_{t} \colon x \mapsto P_{t}xP_{t}^{*}$ have the
  desired properties.
\end{proof}
Note that the relations \eqref{eq:scaling-modular-delta} and
\eqref{eq:unitary-antipode} can be lifted to $\CrG$ and $\LGinf$ by
continuity.  The next result will allow us to relate $\vnR$ to the
unitary antipode of the measured quantum groupoid that we are going to
construct.
\begin{Lem} \label{lemma:vn-r-characterisation}
For all $a,b\in A$,
\begin{align*}
  \vnR(\id \otimes
  \omega_{J\Lambda(b),J\Lambda(b)})(\vnDelta(\pi(a^{*}a))) = (\id
  \otimes \omega_{J\Lambda(a),J\Lambda(a)})(\vnDelta(\pi(b^{*}b))).
\end{align*}
\end{Lem}
\begin{proof}
Let $c=a^{*}a$ and $d=b^{*}b$. 
A short calculation using \eqref{eq:modular} shows that  the right hand side is equal to
  \begin{align*}
    d_{(1)}\phi(\sigma_{i/2}(a)d_{(2)}\sigma_{i/2}(a)^{*})
    = d_{(1)}\phi(\sigma_{i/2}(c)d_{(2)}).
  \end{align*}
By Lemma \ref{lemma:strong-invariance} and
  \eqref{eq:scaling-modular-delta}, \eqref{eq:modular},  this equals 
  $S(\tau_{i/2}(c_{(1)}))
    \phi(\sigma_{i/2}(c_{(2)})d)$
which is the  left hand side.
\end{proof}

The operator-algebraic structures constructed so far fit into the
theory of measured quantum groupoids as follows.

Denote by $\nu$ the
normal, faithful, semifinite weight on $l^{\infty}(I)$ given by
\begin{align} \label{eq:vn-nu}
  \nu(f) &=\sum_{k} f(k) \quad \text{for all } f\in l^{\infty}(I)_{+}.
\end{align}
Then the relative tensor product of $\LGtwo$ with itself,
relative to the representations $\rho,\lambda$ of $l^{\infty}(I)$ and
the weight $\nu$, takes the simple form
\begin{align*}
\LGinf \otimesrl \LGinf \cong
  \bigoplus_{k} (\rho_{k}\LGtwo \otimes \lambda_{k}\LGtwo) =
  \vnE(\LGtwo \otimes \LGtwo),
\end{align*}
see \cite{},  the relative tensor product
of operators $S\in \rho(l^{\infty}(I))'$ and $T \in
\lambda(l^{\infty}(I))'$ gets identified with the compression
\begin{align*}
S \otimesrl T \equiv
  \vnE(S \otimes
  T) = (S \otimes T)\vnE \subseteq {\cal B}(\vnE(\LGtwo
  \otimes \LGtwo)),
\end{align*}
and the fiber product of  $  \LGinf$ with itself, relative to $\rho$
and $\lambda$,  gets identified with
\begin{align} \label{eq:vn-fiber}
  \begin{aligned}
    \LGinf \astrl \LGinf &= (\LGinf' \otimesrl \LGinf')' \\ &\equiv
    (\vnE(\LGinf' \otimes \LGinf'))' = \vnE(\LGinf \otimes
    \LGinf)\vnE.
  \end{aligned}
\end{align} 
By Lemma \ref{lemma:vn-delta} 3., we can co-restrict $\vnDelta$ to  a
normal, faithful $*$-homomorphism
\begin{align*}
  \tilde\Delta \colon \LGinf \to   \LGinf \astrl \LGinf.
\end{align*}
We now obtain a Hopf-von Neumann bimodule in the
sense of \cite{}.
\begin{Prop}
  Let $\mathscr{G}$ be an $I$-partial compact quantum group. Then
  \begin{enumerate}
  \item $\tilde\Delta(\lambda(x)) = \lambda(x) \otimesrl 1$ and
    $\tilde\Delta(\rho(x)) = 1 \otimesrl \rho(x)$ for all $x\in
    l^{\infty}(I)$, and
  \item $(\tilde\Delta \ast \id)\circ \tilde\Delta = (\id \ast \tilde\Delta)
    \circ \tilde\Delta$.
  \end{enumerate}
  In particular, $(l^{\infty}(I),\LGinf, \lambda,\rho,\tilde\Delta)$ is a
  Hopf-von Neumann bimodule.
\end{Prop}
\begin{proof}
Assertion 1.\ follows from \eqref{eq:delta-lambda-rho} and
Lemma \ref{lemma:vn-delta} 1.\ and ensures that  the $*$-homo\-morphisms
\begin{align*}
  \tilde\Delta \ast \id, \id \ast \tilde\Delta \colon  \LGinf \astrl \LGinf
  \to \LGinf \astrl \LGinf \astrl \LGinf
\end{align*}
are well-defined.  As in
\eqref{eq:vn-fiber}, we can identify
\begin{align*}
 \LGinf \astrl \LGinf \astrl \LGinf \cong \vnE^{(2)}(\LGinf
  \otimes \LGinf \otimes \LGinf)\vnE^{(2)},
\end{align*}
where $\vnE^{(2)}=(\vnE \otimes 1)(1 \otimes \vnE)$, and then the
$*$-homomorphisms become restrictions of the maps $\tilde\Delta \otimes
\id$ and $\id \otimes \tilde\Delta$, respectively. Now, 2.\ follows from
Lemma \ref{lemma:vn-delta} 1.\ and co-associativity of $\Delta$.
\end{proof}
This Hopf-von Neumann bimodule is a measured quantum groupoid in the
sense of \cite{}.
\begin{Theorem} \label{theorem:vn-measured} Let $\mathscr{G}$ be an
  $I$-partial compact quantum group. Then the Hopf-von Neumann
  bimodule $(l^{\infty}(I),\LGinf,\lambda,\rho,\tilde\Delta)$ and the
  weights $(T_{\lambda},T_{\rho}$ and $\nu$ defined in
  \eqref{eq:vn-weights} and \eqref{eq:vn-nu} form a measured quantum
  groupoid.  It is unimodular and its unitary antipode and scaling
  group coincide with $\vnR$ and $\vntau$, respectively.
\end{Theorem}
\begin{proof}
  First, observe that the  the modular
  automorphism groups of the weights $\nu \circ \lambda^{-1} \circ
  T_{\lambda}$ and $\nu \circ \rho^{-1} \circ T_{\rho}$ commute  because the two
  compositions coincide with $\vnphi$. Next, we need  to show that
  $T_{\lambda}$ is left-invariant in the sense that
  \begin{align*}
   (\id \underset{\nu}{_{\rho}\ast_{\lambda}} \vnphi)(\tilde\Delta(x)) = T_{\lambda}(x) 
  \end{align*}
  for all $x\in \LGinf_{+}$. But it is easy to see that the left hand
  side coincides with $(\id \ast \vnphi)(\vnDelta(x))$ so that the
  equation above follows from Proposition
  \ref{prop:vn-invariance}. Likewise $T_{\rho}$ is right-invariant in
  the appropriate sense. We thus obtain a measured quantum groupoid as
  claimed.  Denote by $\tilde R$ its unitary antipode and by
  $\tilde\tau$ its scaling group.

  Let us prove that $\tilde \tau_{t}=\vntau$ for all $t\in \R$. By
  \cite{} and \eqref{eq:scaling-modular-delta},
  \begin{align*}
    (\tilde \tau_{t} \astrl \sigma^{\vntau}_{t}) \circ \tilde \Delta
    &=\tilde \Delta \circ \sigma^{\vntau}_{t}, & (\vntau_{t} \otimes
    \sigma^{\vntau}_{t}) \circ \vnDelta &= \vnDelta \circ \vntau_{t}.
  \end{align*}
  The second equation implies that the first one remains true if we
  replace $\tilde\tau_{t}$ by $\vntau_{t}$.  Using Theorem A.7 in
  \cite{} [enock:action], we can conclude that $\tilde
  \tau_{t}=\vntau_{t}$.

 To prove that $\tilde R=\vnR$, we use the relations
  \begin{align*}
    \tilde R(\id \underset{\nu}{_{\rho} \ast_{\lambda}}
    \omega_{J\Lambda(b),J\Lambda(b)})(\vnDelta(\pi(a^{*}a))) &= (\id
    \underset{\nu}{_{\rho} \ast_{\lambda}}
    \omega_{J\Lambda(a),J\Lambda(a)})(\vnDelta(\pi(b^{*}b)))
  \end{align*}
from \cite{} and Lemma \ref{lemma:vn-r-characterisation}.
\end{proof}

%%% Local Variables: 
%%% mode: latex
%%% TeX-master: "dyn-suq-main"
%%% End: 
