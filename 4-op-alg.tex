\section{Compact Hopf face algebras on the level of operator algebras}



Let $\mathscr{G}$ be a partial compact quantum group, given by a
partial Hopf $*$-algebra $\mathscr{A}=P(\mathscr{G})$ with positive
normalized invariant functional $\phi$.  Since $\phi$ is faithful by
\ref{LemFaith},  the formula
\begin{align*}
  \langle a|b\rangle :=\phi(a^{*}b) \quad \text{for all } a,b\in A
\end{align*}
defines an inner product on $A$. We denote by $\LA$ the completion,
which is a Hilbert space, and by $\Lambda \colon A \to \LA$ the
natural embedding. Since $\phi(A(K)^{*}A(L)) = 0$ if $K\neq L$ by
\ref{}, this Hilbert space is the orthogonal direct sum of
the finite-dimensional subspaces  $\Lambda(A(K)) \subseteq \LA$, that is,
\begin{align*}
  \LA &= \bigoplus_{K} \Lambda(A(K)).
\end{align*}
 For each $k\in I$, we can therefore define operators
$\lambda_{k},\lambda_{k}^{\op},\rho_{k},\rho_{k}^{\op}\in {\cal
  B}(\LA)$ such that
\begin{align*}
  \lambda_{k}\Lambda(a)&= \Lambda(\lambda_{k} a), &
  \lambda^{\op}_{k}\Lambda(a) &= \Lambda(a\lambda_{k}), &
  \rho_{k}\Lambda(a) &= \Lambda(\rho_{k}a), &
  \rho_{k}^{\op}\Lambda(a) &= \Lambda(a\rho_{k})
\end{align*}
for all $a\in A$. Likewise, we define $*$-homomorphisms
$\lambda,\lambda^{\op},\rho,\rho^{\op} \colon l^{\infty}(I) \to {\cal
  B}(\LA)$.

We identify  the algebraic
tensor product $A\otimes A$ with a subspace of the Hilbert space
tensor product $\LA \otimes \LA$ using the map $\Lambda \otimes \Lambda$.
\begin{Lem} \label{lemma:partial-isometry}
The map $T_{1} \colon A \otimes A \to A \otimes A$ given by $a\otimes
b\mapsto \Delta(a)(1 \otimes b)$ extends to a partial isometry
  $V \colon \LA \otimes \LA \to \LA \otimes \LA$. Its range and domain projection  are given by
  \begin{align*}
    VV^{*} &= \sum_{k} \rho_{k} \otimes \lambda_{k}, &
    V^{*}V &=\sum_{k} \rho^{\op}_{k} \otimes \rho_{k}, 
  \end{align*}
 the sums converging in the strong operator topology.
\end{Lem}
\begin{proof}
  Let $a,b \in A$. Since $\Delta$ is a $*$-homomorphism and $\phi$ is
invariant,
  \begin{align*}
    \langle \Lambda(a_{(1)}) \otimes
    \Lambda(a_{(2)}b)|\Lambda(a'_{(1)}) \otimes
    \Lambda(a'_{(2)}b')\rangle &=
    \phi(a_{(1)}^{*}a'_{(1)})\phi(b^{*}a_{(2)}^{*}a'_{(2)}b') \\
    &= \sum_{p}
    \phi(b^{*}\rho_{p}\phi(\rho_{p}a^{*}a'\rho_{p})\rho_{p}b') \\
    & =\sum_{p} \langle\Lambda(a\rho_{p}) \otimes \Lambda(\rho_{p}b) |
    \Lambda(a'\rho_{p}) \otimes \Lambda(b'\rho_{p})\rangle.
  \end{align*}
  Now, the assertion follows from \ref{prop:riti}.
\end{proof}

\begin{Prop} \label{prop:gns} Let $\mathscr{G}$ be a partial compact quantum group with
  underlying total $*$-algebra $A$ and associated Hilbert
  space $\LA$. Then there exists a unique $*$-homomorphism $\pi\colon
  A \to {\cal B}(\LA)$ such that $\pi(a)\Lambda(b)=\Lambda(ab)$ for
  all $a,b\in A$.
\end{Prop}
\begin{proof} 
  Let $a,c \in A$. Then the formula $x \mapsto \langle
\Lambda(c) | x\Lambda(a)\rangle$ defines a bounded linear functional
  $\omega_{\Lambda(c),\Lambda(a)}$ on ${\cal B}(\LA)$ and a
  straightforward computation shows that
  \begin{align*}
    (\omega_{\Lambda(c),\Lambda(a)}\otimes \id)(V)\Lambda(b) =
    \Lambda(\varphi(c^*a_{(1)})a_{(2)}b)
  \end{align*}
  for all $b\in A$. Therefore, left multiplication by
  $\varphi(c^*a_{(1)})a_{(2)}$ extends to a bounded linear operator on $\LA$.
 Since $(A\otimes 1)\Delta(A) = (A\otimes
  A)\Delta(1)$ by Proposition \ref{prop:riti} and $\phi$ is
  normalized,  elements of the form $\phi(c^{*}a_{(1)})a_{(2)}$ span $A$.
\end{proof}

\begin{Lem} \label{lemma:hilbert-algebra}
  The subspace $\Lambda(A) \subseteq \LA$ is a Hilbert algebra with
  respect to the operations given by
  $\Lambda(a)\Lambda(b)=\pi(a)\Lambda(b)=\Lambda(ab)$ and
  $\Lambda(a)^{*}= \Lambda(a^{*})$ for all $a,b\in A$.
\end{Lem}
\begin{proof}
  The map $\pi(a)$ is bounded for each $a \in A$ by \ref{prop:gns},
  and the involution is pre-closed because
  \begin{align*}
    \langle \Lambda(a)|\Lambda(b^{*})\rangle = \phi(a^{*}b^{*}) =
    \phi(b^{*}\sigma(a^{*})) = \langle
    \Lambda(b)|\Lambda(\sigma(a^{*}))\rangle
  \end{align*}
  for all $a,b \in A$, where $\sigma$ denotes the modular automorphism
  of $\phi$; see \ref{},
\end{proof}
\begin{Prop}
  Let $\mathscr{G}$ be a partial compact quantum group 
with associated GNS-representation $\pi \colon A \to {\cal B}(\LA)$
and von Neumann algebra $M=\pi(A)''$. Then
  there exist a unique normal semi-finite faithful weight $\vnphi$ on
  $M$ and unique normal semi-finite faithful
  operator-valued weights $T,T'$ from $M$ to
  $\lambda(l^{\infty}(I))$ and $\rho(l^{\infty}(I))$, respectively,
  such that $\pi(A) \subseteq \mathfrak{M}_{\vnphi} \cap
  \mathfrak{M}_{T} \cap \mathfrak{M}_{T'}$ and
  \begin{align}\label{eq:vnphi}
    \vnphi(\pi(a))&= \phi(a), &
    T(\pi(a))\Lambda(b) &= \Lambda(_{\lambda}\phi(a)b), &
    T'(\pi(b))\Lambda(b) &= \Lambda(_{\rho}\phi(a)b)        
  \end{align}
  for all $a,b\in A$. Furthermore, $\nu \circ \alpha^{-1} \circ T = \vnphi =
  \nu \circ \beta^{-1} \circ T'$. 
\end{Prop}
\begin{proof}
  The existence of $\vnphi$ follows from Lemma
  \ref{lemma:hilbert-algebra} and the general theory of Hilbert
  algebras. By \cite{}, there exists a normal semi-finite faithful
  weight $\vnphi$ on the von-Neumann algebra $\pi(A)''$ such that
  $\pi(A) \subseteq \mathfrak{N}_{\tilde \phi}$ and
  $\vnphi(\pi(a^{*}a))=\|\Lambda(a)\|^{2}=\phi(a^{*}a)$ for all $a\in
  A$. The polarization identity now yields $\vnphi(\pi(a^{*}b)) =
  \phi(a^{*}b)$ for all $a,b\in A$. Since $A$ is idempotent, we can
  conclude $\pi(A) \subseteq \mathfrak{N}_{\vnphi}$ and $\vnphi\circ
  \pi=\phi$.

  We next prove uniqueness of $\vnphi$, $T$ and $T'$.  The elements
  $p_{k,m}:=\lambda_{k}\rho_{m}$ are pairwise orthogonal
  projections in $\mathfrak{M}_{\vnphi} \cap \mathfrak{M}_{T} \cap
  \mathfrak{M}_{T'}$ summing up to $1$, whence $\vnphi$, $T$ and $T'$
  are the sums of the bounded linear maps
  \begin{align*}
    \Gr{\vnphi}{k}{l}{m}{n}&\colon x\mapsto \vnphi(p_{k,m}xp_{l,n}),  &
    \Gr{T}{k}{l}{m}{n} &\colon x \mapsto T(p_{k,m}xp_{l,n}), &
    \Gr{T'{}}{k}{l}{m}{n} &\colon x\mapsto T'(p_{k,m}xp_{l,n}),
  \end{align*}
  respectively, which are determined by their restriction to
  $\pi(A)$. Note that in particular, the maps above vanish if
  $(k,m)\neq (l,n)$.
  
  We finally prove existence of $T$ and $T'$. For each $k,m\in I$, the vector
  $\xi_{k,m}:=\Lambda(\lambda_{k}\rho_{m})$ has norm one and the
  associated vector state $\omega_{\xi_{k,m}}$ satisfies
  \begin{align*}
    \omega_{\xi_{k,m}}(\pi(a))= \langle
    \Lambda(\lambda_{k}\rho_{m})|\Lambda(a\lambda_{k},\rho_{m})\rangle
    = \phi(\lambda_{k}\rho_{m}a\lambda_{k}\rho_{m} )=
    \phi(\lambda_{k}\rho_{m}a)
  \end{align*}
  for all $a\in A$. Therefore, the maps
  \begin{align} \label{eq:vnphi-explicit}
\tilde\phi&\colon x\mapsto\sum_{k,m} \omega_{\xi_{k,m}}(x), &
    T&\colon x\mapsto \sum_{k,m}
    \omega_{\xi_{k,m}}(x)\alpha(\delta_{k}), & T'&\colon x\mapsto
    \sum_{k,m} \omega_{\xi_{k,m}}(x) \beta(\delta_{m})
  \end{align}
  satisfy \eqref{eq:vnphi}. Clearly, these maps are normal,
  semi-finite and satisfy $\nu \circ \alpha^{-1}\circ T = \tilde\phi = \nu
  \circ \beta^{-1} \circ T'$.  By uniqueness, $\tilde
  \phi=\vnphi$. Since $\vnphi$ is faithful, so must be $T$ and $T'$.
\end{proof}
\begin{Theorem}
    $(l^{\infty}(I),M,\lambda,\rho,\vnDelta,T,T',\nu)$
  is a measured quantum groupoid in the sense of \cite{enock:action}.
\end{Theorem}
\begin{proof}
  We need to show that $T$ is left-invariant, $T'$ is right-invariant,
  and that $\nu$ is relatively invariant with respect to $T$ and $T'$
  in the sense that the modular automorphism groups of the weights
  $\nu \circ \alpha^{-1} \circ T$ and $\nu \circ \beta^{-1} \circ T'$
  commute.  We prove the first condition, and the second one follows from
  a similar argument. The third condition holds trivially because the
  two compositions coincide by Proposition \ref{prop:}. 

  Let $\xi_{k,m}=\Lambda(\lambda_{k}\rho_{m})$ and abbreviate
  $\vnphi_{k,m}=\omega_{\xi_{k,m}}$. Then the normal linear map
  \begin{align*}
    T_{m}\colon \pi(A)'' \to \alpha(l^{\infty}(I)), \ x\mapsto \sum_{k} \vnphi_{k,m}(x)\alpha(\delta_{k})
  \end{align*}
  is bounded and satisfies $T=\sum_{m} T_{m}$ by
  \eqref{eq:vnphi}. For all $a,b\in
  A$,
  \begin{align*}
    (\id {_{\rho}\ast_{\lambda}} \vnphi_{k,m})(\vnDelta(\pi(a)))\Lambda(b) &=
    (r^{\beta,\alpha}_{\xi_{k,m}})^{*}\vnDelta(\pi(a))r^{\beta,\alpha}_{\xi_{k,m}}
    \Lambda(b) \\
    &= (r^{\beta,\alpha}_{\xi_{k,m}})^{*}(\Lambda(a_{(1)}b)
    {_{\rho} \otimes_{\lambda}} \Lambda(a_{(2)}\lambda_{k}\rho_{m})) \\
    &=
    \Lambda(\phi(\lambda_{k}\rho_{m}a_{(2)}\lambda_{k}\rho_{m})\rho_{k}a_{(1)}b).
  \end{align*}
  Sum over $k \in I$ and using \eqref{eq:wdelta}, we find
  \begin{align*}
 \sum_{k} (\id
 {_{\rho}\ast_{\lambda}} \vnphi_{k,m})(\vnDelta(\pi(a))) &=
\pi(\phi(\rho_{m}a_{(2)}\rho_{m})a_{(1)})  =
\pi(_{\lambda}\phi(\rho_{m}a\rho_{m})) = T_{m}(\pi(a)).
  \end{align*}
  Finally, we insert the relation $\sum_{k} \vnphi_{k,m} = \nu \circ
  T_{m}$ and obtain
\begin{align*}
  (\id {_{\rho}\otimes_{\lambda}} T_{m})(\vnDelta(x)) = (\id {_{\rho}\otimes_{\lambda}}
  (\nu \circ T_{m}))(\vnDelta(x)) {_{\rho}\otimes_{\lambda}} 1 = T_{m}(x)  {_{\rho}\otimes_{\lambda}} 1
\end{align*}
for all $x\in \pi(A)$.  Since $T_{m}$ is normal and bounded, this
equation remains valid for all $x\in \pi(A)''$. Summing over $m$, we
obtain $(\id {_{\rho}\otimes_{\lambda}} T)(\vnDelta(x)) = T(x)  {_{\rho}\otimes_{\lambda}} 1$ for all
positive $x \in \pi(A)''$ which is the desired left-invariance of $T$.
\end{proof}


%%% Local Variables: 
%%% mode: latex
%%% TeX-master: "dyn-suq-main"
%%% End: 
