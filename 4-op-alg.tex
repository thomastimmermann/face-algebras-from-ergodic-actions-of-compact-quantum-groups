\section{Compact Hopf face algebras on the level of operator algebras}


Let $\mathscr{G}$ be a partial compact quantum group. Using the
invariant functional, we now construct completions of the underlying
$*$-algebra $P(\mathscr{G})$ in the form of a reduced $C^{*}$-algebra $\CrG$
and a von Neumann algebra $\LGinf$ which act on a Hilbert space
$\LGtwo$. Then, we lift the  comultiplication and the invariant
functional to the level of operator algebras and show that $\LGinf$ becomes a measured
quantum groupoid in the sense of Lesieur \cite{} and Enock \cite{}.


Denote by $\LGtwo$ the completion of $A$ with respect to the norm
associated to the inner product given by
\begin{align*}
  \langle a|b\rangle :=\phi(a^{*}b) \quad \text{for all } a,b\in A,
\end{align*}
and by $\Lambda \colon A \to \LGtwo$ the natural embedding.  This
product is definite because $\phi$ is faithful by \ref{LemFaith}, and
it extends to the space $\LGtwo$ which thus becomes a Hilbert space.
As such, $\LGtwo$ is the
orthogonal direct sum of the finite-dimensional subspaces
$\Lambda(A(K)) \subseteq \LGtwo$, where $K\in M_{2}(I)$, because
$\phi(A(K)^{*}A(L)) = 0$ if $K\neq L$ by \ref{}.  In particular, there
exist  operators
$\lambda_{k},\lambda_{k}^{\op},\rho_{k},\rho_{k}^{\op}\in {\cal
  B}(\LGtwo)$ for each $k\in I$ such that
\begin{align*}
  \lambda_{k}\Lambda(a)&= \Lambda(\lambda_{k} a), &
  \lambda^{\op}_{k}\Lambda(a) &= \Lambda(a\lambda_{k}), &
  \rho_{k}\Lambda(a) &= \Lambda(\rho_{k}a), &
  \rho_{k}^{\op}\Lambda(a) &= \Lambda(a\rho_{k})
\end{align*}
for all $a\in A$, and faithful, normal $*$-homomorphisms
\begin{align} \label{eq:vn-lambda-rho}
  \lambda,\rho \colon l^{\infty}(I) \to
  {\cal B}(\LGtwo)
\end{align}
that send the delta function at
$k\in I$ to the operators $\lambda_{k}$ or $\rho_{k}$, respectively. 

To construct the $C^{*}$-algebra $\CrG$ and $\LGinf$ and to lift the
comultiplication, we first lift the map $T_{1}$ considered in
Proposition \ref{prop:riti} to a partial isometry on $\LGtwo \otimes
\LGtwo$.  Define $\vnE,\overline{G} \in {\cal B}(\LGtwo \otimes
\LGtwo)$ by
\begin{align*}
  \vnE &:=\sum_{k} \rho_{k} \otimes \lambda_{k}, & 
  \overline{G} &:= \sum_{k} \rho_{k}^{\op} \otimes \rho_{k},
\end{align*}
where the sums converge with respect to the strong operator
topology.
\begin{Lem} \label{lemma:partial-isometry}
There exists a unique partial isometry $V$ on $\LGtwo \otimes \LGtwo$
such that
\begin{align*}
  V(\Lambda(a) \otimes \Lambda(b)) = \Lambda(a_{(1)}) \otimes \Lambda(a_{(2)}b)
\end{align*}
for all $a,b\in A$. Its range and domain projections are given by $VV^{*} = \vnE$
and $V^{*}V = \overline{G}$.
\end{Lem}
\begin{proof}
  Let $a,b \in A$. Since $\Delta$ is a $*$-homomorphism and $\phi$ is
invariant,
  \begin{align*}
    \langle \Lambda(a_{(1)}) \otimes
    \Lambda(a_{(2)}b)|\Lambda(a'_{(1)}) \otimes
    \Lambda(a'_{(2)}b')\rangle &=
    \phi(a_{(1)}^{*}a'_{(1)})\phi(b^{*}a_{(2)}^{*}a'_{(2)}b') \\
    &= \sum_{p}
    \phi(b^{*}\rho_{p}\phi(\rho_{p}a^{*}a'\rho_{p})\rho_{p}b') \\
    & =\sum_{p} \langle\Lambda(a\rho_{p}) \otimes \Lambda(\rho_{p}b) |
    \Lambda(a'\rho_{p}) \otimes \Lambda(b'\rho_{p})\rangle.
  \end{align*}
  Now, the assertion follows from \ref{prop:riti}.
\end{proof}

\begin{Prop} \label{prop:gns} Let $\mathscr{G}$ be a partial compact quantum group with
  underlying total $*$-algebra $A$ and associated Hilbert
  space $\LGtwo$. Then there exists a unique $*$-homomorphism $\pi\colon
  A \to {\cal B}(\LGtwo)$ such that $\pi(a)\Lambda(b)=\Lambda(ab)$ for
  all $a,b\in A$, and this $\pi$ is faithful.
\end{Prop}
\begin{proof} 
  Let $a,c \in A$. Then the formula $x \mapsto \langle
\Lambda(c) | x\Lambda(a)\rangle$ defines a bounded linear functional
  $\omega_{\Lambda(c),\Lambda(a)}$ on ${\cal B}(\LGtwo)$ and a
  straightforward computation shows that
  \begin{align*}
    (\omega_{\Lambda(c),\Lambda(a)}\otimes \id)(V)\Lambda(b) =
    \Lambda(\varphi(c^*a_{(1)})a_{(2)}b)
  \end{align*}
  for all $b\in A$. Therefore, left multiplication by
  $\varphi(c^*a_{(1)})a_{(2)}$ extends to a bounded linear operator on $\LGtwo$.
 Since $(A\otimes 1)\Delta(A) = (A\otimes
  A)\Delta(1)$ by Proposition \ref{prop:riti} and $\phi$ is
  normalized,  elements of the form $\phi(c^{*}a_{(1)})a_{(2)}$ span
  $A$. 
\end{proof}

Given a partial compact quantum group $\mathscr{G}$, we denote by
\begin{align}
  \CrG &\subseteq {\cal B}(\LGtwo) &&\text{and} & \LGinf &\subseteq {\cal B}(\LGtwo)
\end{align}
the $C^{*}$-algebra and the von Neumann algebra generated by $\pi(A)
\subseteq \LGtwo$, respectively, and identify $M(\CrG)$ with a
$C^{*}$-subalgebra of $\LGtwo$.  Thus, we get a sequence of inclusions
\begin{align*}
A \cong \pi(A) \subseteq
  \CrG \subseteq M(\CrG) \subseteq
\LGinf \subseteq {\cal B}(\LGtwo).
\end{align*}
Note that
 the $*$-homomorphisms $\lambda,\rho$ in
\eqref{eq:vn-lambda-rho} send $l^{\infty}(I)$ to $M(\CrG)$, and that
\begin{align*}
  \vnE \in M(\CrG \otimes \CrG) \subseteq \LGinf \otimes \LGinf
  \subseteq {\cal B}(\LGtwo \otimes \LGtwo),
\end{align*}
where $\otimes$ denotes the minimal tensor product
of $C^{*}$-algebras, the tensor product of von Neumann algebras, and
the tensor product of Hilbert spaces, respectively.

Consider the map 
\begin{align*}
  \vnDelta \colon \LGinf \to {\cal B}(\LGtwo \otimes \LGtwo), \ x
  \mapsto V(x \otimes 1)V^{*}.
\end{align*}
\begin{Lem} \label{lemma:vn-delta}
  \begin{enumerate}
  \item $\vnDelta(\pi(a)) (\Lambda(b) \otimes \Lambda(c)) =
    \Lambda(a_{(1)}b) \otimes \Lambda(a_{(2)}b)$ for all $a,b,c\in A$;
  \item $\vnDelta$ is a normal, faithful $*$-homomorphism;
  \item  $\vnDelta(\CrG) \subseteq \vnE M(\CrG \otimes
  \CrG)\vnE$ and $\vnDelta(\LGinf) \subseteq \vnE(\LGinf \otimes
  \LGinf)\vnE$.
  \end{enumerate}
\end{Lem}
\begin{proof}
  The equation in 1.{} is easily verified. The map $\vnDelta$ is
  normal by construction, a $*$-homo\-morphism by 1.{}, and faithful
  because $\vnDelta(x)=0$ implies $x\otimes 1=0$ on
  $V^{*}V(L^{2}(\mathscr{G}) \otimes L^{2}(\mathscr{G}))$ and hence
  $x=0$ on $\bigoplus_{k}
  \rho_{k}^{\op}L^{2}(\mathscr{G})=L^{2}(\mathscr{G})$. Finally, 3.\
  follows from the relation $\Delta(a)=E\Delta(a)E$, which holds for
  all $a\in A$.
\end{proof}

To connect to the theory of measured quantum groupoids, we need a few
preliminaries.

Denote by $\nu$ the
normal, faithful, semifinite weight on $l^{\infty}(I)$ given by
\begin{align*}
  \nu(f) &=\sum_{k} f(k) \quad \text{for all } f\in l^{\infty}(I)_{+}.
\end{align*}
Then the relative tensor product of $\LGtwo$ with itself,
relative to the representations $\rho,\lambda$ of $l^{\infty}(I)$ and
the weight $\nu$, takes the simple form
\begin{align*}
\LGinf \otimesrl \LGinf \cong
  \bigoplus_{k} (\rho_{k}\LGtwo \otimes \lambda_{k}\LGtwo) =
  \vnE(\LGtwo \otimes \LGtwo),
\end{align*}
see \cite{},  the relative tensor product
of operators $S\in \rho(l^{\infty}(I))'$ and $T \in
\lambda(l^{\infty}(I))'$ gets identified with the compression
\begin{align*}
S \otimesrl T \equiv
  \vnE(S \otimes
  T) = (S \otimes T)\vnE \subseteq {\cal B}(\vnE(\LGtwo
  \otimes \LGtwo)),
\end{align*}
and the fiber product of  $  \LGinf$ with itself, relative to $\rho$
and $\lambda$,  gets identified with
\begin{align} \label{eq:vn-fiber}
  \begin{aligned}
    \LGinf \astrl \LGinf &= (\LGinf' \otimesrl \LGinf')' \\ &\equiv
    (\vnE(\LGinf' \otimes \LGinf'))' = \vnE(\LGinf \otimes
    \LGinf)\vnE.
  \end{aligned}
\end{align} 
By Lemma \ref{lemma:vn-delta} 3., we can co-restrict $\vnDelta$ to  a
normal, faithful $*$-homomorphism
\begin{align*}
  \tilde\Delta \colon \LGinf \to   \LGinf \astrl \LGinf.
\end{align*}
We now obtain a Hopf-von Neumann bimodule in the
sense of \cite{} as follows.
\begin{Prop}
  Let $\mathscr{G}$ be an $I$-partial compact quantum group. Then
  \begin{enumerate}
  \item $\tilde\Delta(\lambda(x)) = \lambda(x) \otimesrl 1$ and
    $\tilde\Delta(\rho(x)) = 1 \otimesrl \rho(x)$ for all $x\in
    l^{\infty}(I)$, and
  \item $(\tilde\Delta \ast \id)\circ \tilde\Delta = (\id \ast \tilde\Delta)
    \circ \tilde\Delta$.
  \end{enumerate}
  In particular, $(l^{\infty}(I),\LGinf, \lambda,\rho,\tilde\Delta)$ is a
  Hopf-von Neumann bimodule.
\end{Prop}
\begin{proof}
Assertion 1.\ follows from \eqref{eq:delta-lambda-rho} and
Lemma \ref{lemma:vn-delta} 1.\ and ensures that  the $*$-homo\-morphisms
\begin{align*}
  \tilde\Delta \ast \id, \id \ast \tilde\Delta \colon  \LGinf \astrl \LGinf
  \to \LGinf \astrl \LGinf \astrl \LGinf
\end{align*}
are well-defined.  As in
\eqref{eq:vn-fiber}, we can identify
\begin{align*}
 \LGinf \astrl \LGinf \astrl \LGinf \cong \vnE^{(2)}(\LGinf
  \otimes \LGinf \otimes \LGinf)\vnE^{(2)},
\end{align*}
where $\vnE^{(2)}=(\vnE \otimes 1)(1 \otimes \vnE)$, and then the
$*$-homomorphisms become restrictions of the maps $\tilde\Delta \otimes
\id$ and $\id \otimes \tilde\Delta$, respectively. Now, 2.\ follows from
Lemma \ref{lemma:vn-delta} 1.\ and co-associativity of $\Delta$.
\end{proof}

To show that the Hopf-von Neumann bimodule above is a measured quantum
groupoid, we have to lift the invariant weight to $\LGinf$. This can
be done as follows.
\begin{Lem} \label{lemma:hilbert-algebra}
  The subspace $\Lambda(A) \subseteq \LGtwo$ is a Hilbert algebra with
  respect to the operations given by
  $\Lambda(a)\Lambda(b)=\pi(a)\Lambda(b)=\Lambda(ab)$ and
  $\Lambda(a)^{*}= \Lambda(a^{*})$ for all $a,b\in A$.
\end{Lem}
\begin{proof}
  The map $\pi(a)$ is bounded for each $a \in A$ by \ref{prop:gns},
  and the involution is pre-closed because
  \begin{align*}
    \langle \Lambda(a)|\Lambda(b^{*})\rangle = \phi(a^{*}b^{*}) =
    \phi(b^{*}\sigma(a^{*})) = \langle
    \Lambda(b)|\Lambda(\sigma(a^{*}))\rangle
  \end{align*}
  for all $a,b \in A$, where $\sigma$ denotes the modular automorphism
  of $\phi$; see \ref{},
\end{proof}
\begin{Prop}
  Let $\mathscr{G}$ be an $I$-partial compact quantum group. Then there exist a unique normal
  semi-finite faithful weight $\vnphi$ on $\LGinf$ and unique normal
  semi-finite faithful operator-valued weights $T,T'$ from $\LGinf$ to
  $\lambda(l^{\infty}(I))$ and $\rho(l^{\infty}(I))$, respectively,
  such that $\pi(A) \subseteq \mathfrak{M}_{\vnphi} \cap
  \mathfrak{M}_{T} \cap \mathfrak{M}_{T'}$ and
  \begin{align}\label{eq:vnphi}
    \vnphi(\pi(a))&= \phi(a), &
    T(\pi(a))\Lambda(b) &= \Lambda(_{\lambda}\phi(a)b), &
    T'(\pi(b))\Lambda(b) &= \Lambda(_{\rho}\phi(a)b)        
  \end{align}
  for all $a,b\in A$. Furthermore, $\nu \circ \alpha^{-1} \circ T = \vnphi =
  \nu \circ \beta^{-1} \circ T'$. 
\end{Prop}
\begin{proof}
  The existence of $\vnphi$ follows from Lemma
  \ref{lemma:hilbert-algebra} and the general theory of Hilbert
  algebras. By \cite{}, there exists a normal semi-finite faithful
  weight $\vnphi$ on the von-Neumann algebra $\pi(A)''=\LGinf$ such that
  $\pi(A) \subseteq \mathfrak{N}_{\tilde \phi}$ and
  $\vnphi(\pi(a^{*}a))=\|\Lambda(a)\|^{2}=\phi(a^{*}a)$ for all $a\in
  A$. The polarization identity now yields $\vnphi(\pi(a^{*}b)) =
  \phi(a^{*}b)$ for all $a,b\in A$. Since $A$ is idempotent, we can
  conclude $\pi(A) \subseteq \mathfrak{N}_{\vnphi}$ and $\vnphi\circ
  \pi=\phi$.

  We next prove uniqueness of $\vnphi$, $T$ and $T'$.  The elements
  $p_{k,m}:=\lambda_{k}\rho_{m}$ are pairwise orthogonal
  projections in $\mathfrak{M}_{\vnphi} \cap \mathfrak{M}_{T} \cap
  \mathfrak{M}_{T'}$ summing up to $1$, whence $\vnphi$, $T$ and $T'$
  are the sums of the bounded linear maps
  \begin{align*}
    \Gr{\vnphi}{k}{l}{m}{n}&\colon x\mapsto \vnphi(p_{k,m}xp_{l,n}),  &
    \Gr{T}{k}{l}{m}{n} &\colon x \mapsto T(p_{k,m}xp_{l,n}), &
    \Gr{T'{}}{k}{l}{m}{n} &\colon x\mapsto T'(p_{k,m}xp_{l,n}),
  \end{align*}
  respectively, which are determined by their restriction to
  $\pi(A)$. Note that in particular, the maps above vanish if
  $(k,m)\neq (l,n)$.
  
  We finally prove existence of $T$ and $T'$. For each $k,m\in I$, the vector
  $\xi_{k,m}:=\Lambda(\lambda_{k}\rho_{m})$ has norm one and the
  associated vector state $\omega_{\xi_{k,m}}$ satisfies
  \begin{align*}
    \omega_{\xi_{k,m}}(\pi(a))= \langle
    \Lambda(\lambda_{k}\rho_{m})|\Lambda(a\lambda_{k},\rho_{m})\rangle
    = \phi(\lambda_{k}\rho_{m}a\lambda_{k}\rho_{m} )=
    \phi(\lambda_{k}\rho_{m}a)
  \end{align*}
  for all $a\in A$. Therefore, the maps
  \begin{align} \label{eq:vnphi-explicit}
\tilde\phi&\colon x\mapsto\sum_{k,m} \omega_{\xi_{k,m}}(x), &
    T&\colon x\mapsto \sum_{k,m}
    \omega_{\xi_{k,m}}(x)\alpha(\delta_{k}), & T'&\colon x\mapsto
    \sum_{k,m} \omega_{\xi_{k,m}}(x) \beta(\delta_{m})
  \end{align}
  satisfy \eqref{eq:vnphi}. Clearly, these maps are normal,
  semi-finite and satisfy $\nu \circ \alpha^{-1}\circ T = \tilde\phi = \nu
  \circ \beta^{-1} \circ T'$.  By uniqueness, $\tilde
  \phi=\vnphi$. Since $\vnphi$ is faithful, so must be $T$ and $T'$.
\end{proof}
\begin{Theorem}
The tuple    $(l^{\infty}(I),\LGinf,\lambda,\rho,\tilde\Delta,T,T',\nu)$
  is a measured quantum groupoid in the sense of \cite{enock:action}.
\end{Theorem}
\begin{proof}
  We need to show that $T$ is left-invariant, $T'$ is right-invariant,
  and that $\nu$ is relatively invariant with respect to $T$ and $T'$
  in the sense that the modular automorphism groups of the weights
  $\nu \circ \alpha^{-1} \circ T$ and $\nu \circ \beta^{-1} \circ T'$
  commute.  We prove the first condition, and the second one follows from
  a similar argument. The third condition holds trivially because the
  two compositions coincide by Proposition \ref{prop:}. 

  Let $\xi_{k,m}=\Lambda(\lambda_{k}\rho_{m})$ and abbreviate
  $\vnphi_{k,m}=\omega_{\xi_{k,m}}$. Then the normal linear map
  \begin{align*}
    T_{m}\colon \pi(A)'' \to \alpha(l^{\infty}(I)), \ x\mapsto \sum_{k} \vnphi_{k,m}(x)\alpha(\delta_{k})
  \end{align*}
  is bounded and satisfies $T=\sum_{m} T_{m}$ by
  \eqref{eq:vnphi}. For all $a,b\in
  A$,
  \begin{align*}
    (\id {_{\rho}\ast_{\lambda}} \vnphi_{k,m})(\tilde\Delta(\pi(a)))\Lambda(b) &=
    (r^{\beta,\alpha}_{\xi_{k,m}})^{*}\tilde\Delta(\pi(a))r^{\beta,\alpha}_{\xi_{k,m}}
    \Lambda(b) \\
    &= (r^{\beta,\alpha}_{\xi_{k,m}})^{*}(\Lambda(a_{(1)}b)
    {_{\rho} \otimes_{\lambda}} \Lambda(a_{(2)}\lambda_{k}\rho_{m})) \\
    &=
    \Lambda(\phi(\lambda_{k}\rho_{m}a_{(2)}\lambda_{k}\rho_{m})\rho_{k}a_{(1)}b).
  \end{align*}
  Sum over $k \in I$ and using \eqref{eq:wdelta}, we find
  \begin{align*}
 \sum_{k} (\id
 {_{\rho}\ast_{\lambda}} \vnphi_{k,m})(\tilde\Delta(\pi(a))) &=
\pi(\phi(\rho_{m}a_{(2)}\rho_{m})a_{(1)})  =
\pi(_{\lambda}\phi(\rho_{m}a\rho_{m})) = T_{m}(\pi(a)).
  \end{align*}
  Finally, we insert the relation $\sum_{k} \vnphi_{k,m} = \nu \circ
  T_{m}$ and obtain
\begin{align*}
  (\id {_{\rho}\otimes_{\lambda}} T_{m})(\tilde\Delta(x)) = (\id {_{\rho}\otimes_{\lambda}}
  (\nu \circ T_{m}))(\tilde\Delta(x)) {_{\rho}\otimes_{\lambda}} 1 = T_{m}(x)  {_{\rho}\otimes_{\lambda}} 1
\end{align*}
for all $x\in \pi(A)$.  Since $T_{m}$ is normal and bounded, this
equation remains valid for all $x\in \pi(A)''$. Summing over $m$, we
obtain $(\id {_{\rho}\otimes_{\lambda}} T)(\tilde\Delta(x)) = T(x)  {_{\rho}\otimes_{\lambda}} 1$ for all
positive $x \in \pi(A)''$ which is the desired left-invariance of $T$.
\end{proof}


%%% Local Variables: 
%%% mode: latex
%%% TeX-master: "dyn-suq-main"
%%% End: 
