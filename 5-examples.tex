
\section{Examples}

\subsection{Classical examples}

\begin{Exa} Let $G$ be a discrete groupoid with object set $I = G^{(0)}$. Consider for $r,s\in I$ the vector space $\Gru{A}{r}{s}= \mathbb{C}\lbrack \Gru{G}{r}{s}\rbrack$ which has a basis $\{\lambda_g\}$ spanned by the morphisms $g$ from $r$ to $s$. Linearly extending the product of $G$ to the $\Gru{A}{r}{s}$ turns $\mathscr{A}$ into a partial algebra. It becomes a partial $^*$-algebra by putting $\lambda_g^* = \lambda_{g^{-1}}$. 

Extend the $I^2$-bigrading to an $I^4$-bigrading by putting $\Gr{A}{k}{l}{m}{n} = \delta_{kl}\delta_{mn}\Gru{A}{k}{l}$. Then together with the coproducts \[ \Delta: \Gru{A}{r}{s}\rightarrow \Gru{A}{r}{s}\otimes \Gru{A}{r}{s},\quad \lambda_g\mapsto \lambda_g\otimes \lambda_g,\] $(\mathscr{A},\Delta)$ defines a partial compact quantum group, the invariant functional $\phi$ being given by \[\phi(\lambda_g) = \delta_{rs}\delta_{g,\id_r},\quad r,s\in I,g\in \Mor(r,s).\]
\end{Exa}

\begin{Exa} Let $G$ be a proper locally compact groupoid with discrete object space $I=G^{(0)}$. Then each $\Gru{G}{r}{s} = \Mor(r,s)$ is a compact space. The $\Gru{G}{r}{r}$ are compact groups, hence come with probability Haar measures $\mu_{rr}$. The $\Gru{G}{r}{s}$ are $\Gru{G}{r}{r}$-$\Gru{G}{s}{s}$-bitorsors, and as such admit a bi-invariant probability measure $\mu_{rs}$. Let $A^r_s \subseteq C(\Gru{G}{r}{s})$ be the spaces of functions which transform as finite-dimensional representations of $\Gru{G}{r}{r}\times \Gru{G}{s}{s}$ under the natural actions. Then the $A^r_s$ form a partial coalgebra over $I$ by putting \[\Delta_{s}: A^r_t \rightarrow A^r_s\otimes A^s_t,\quad f\mapsto \left(\Delta_s(f):\Gru{G}{r}{s}\times \Gru{G}{s}{t}\mapsto \Gru{G}{r}{t},\quad (g,h)\mapsto f(gh)\right).\] We can extend the $I^2$-grading to an $I^4$-grading by putting \[\Gr{A}{k}{l}{m}{n} = \delta_{kl}\delta_{mn} A^l_n,\] and the resulting $\mathscr{A}$ becomes a partial $^*$-algebra by endowing each $A^r_s$ with the pointwise product and $^*$-algebra structure. The couple $(\mathscr{A},\Delta)$ then defines a partial compact quantum group with invariant integral \[\phi: A^r_s \mapsto \C,\quad f\mapsto \int_{\Gru{G}{r}{s}} f(g) \rd \Gru{\mu}{r}{s}(g).\]

\end{Exa}

\begin{Exa} Let $\mathscr{A}$ and $\mathscr{B}$ define two partial compact quantum groups $\mathscr{G}$ and $\mathscr{H}$ over respective sets $I$ and $J$. Then we can make a tensor product partial Hopf $^*$-algebra $\mathscr{A}\otimes \mathscr{B}$ over the index set $I\times J$ by putting \[\Gr{(A\otimes B)}{(k,k')}{(l,l')}{(m,m')}{(n,n')} = \Gr{A}{k}{l}{m}{n}\otimes \Gr{B}{k'}{l'}{m'}{n'}\] with the factorwise product and with coproducts \[\Delta_{(r,r'),(s,s')} = \sigma_{23}(\Delta_{rs}\otimes \Delta_{r',s'}),\] $\sigma$ being the switch map. It is easily seen that the tensor products of the positive invariant integrals for $\mathscr{A}$ and $\mathscr{B}$ produce a positive invariant integral on $\mathscr{A}\otimes \mathscr{B}$. Hence $\mathscr{A}\otimes \mathscr{B}$ defines a partial compact quantum group, which we will denote $\mathscr{G}\times \mathscr{H}$.
\end{Exa}

\subsection{Canonical partial compact quantum groups}

The following generalizes Hayashi's original construction.

\begin{Exa} 
Let $\CatCC$ be an $\mathscr{I}$-partial fusion C$^*$-category. Let $\mathcal{I}$ label a distinguished maximal set $\{u_k\}$ of mutually non-isomorphic irreducible objects of $\CatC$, with associated bigrading $\Gru{\mathcal{I}}{\alpha}{\beta}$ over $\mathscr{I}$. Define \[F_{kl}(X)  = \Hom(u_k,  X\otimes u_l),\qquad X\in \Gru{\CatC}{\alpha}{\beta}, k\in \Gru{\mathcal{I}}{\alpha}{\gamma},l\in \Gru{\mathcal{I}}{\beta}{\gamma}.\] Then each $F_{kl}(X)$ is a Hilbert space by the inner product $\langle f,g\rangle = f^*g$. Put $F_{kl}(X) = 0$ for $k,l$ outside their proper domains. Then clearly the application $(k,l)\mapsto F_{kl}(X)$ is rcf. Moreover, we have isometric compatibility morphisms \[F_{kl}(X)\otimes F_{lm}(Y)\rightarrow F_{km}(X\otimes Y),\quad f\otimes g \mapsto (\id\otimes g)f,\] while $F_{kl}(\Unit_{\alpha}) \cong \delta_{kl} \C$ for $k,l\in \Gru{\mathcal{I}}{\alpha}{\alpha}$. 

It is readily verified that $F$ defines a unital morphism from $\CatCC$ to $\{\Hilb_{\fin}\}_{\mathcal{I}\times \mathcal{I}}$ based over the partition \[\mathcal{I}_{\alpha} = \bigcup_{\beta} \Gru{\mathcal{I}}{\alpha}{\beta},\quad\alpha\in \mathscr{I}.\] From the Tannaka-Krein-Woronowicz reconstruction result, we obtain a partial compact quantum group $\mathscr{A}_{\CatCC}$ with object set $\mathcal{I}$, which we call the \emph{canonical partial compact quantum group} associated with $\CatCC$. 
\end{Exa} 

\begin{Exa} More generally, let $\CatCC$ be an $\mathscr{I}$-partial fusion C$^*$-category, and let $\CatDD$ be a \emph{semi-simple partial $\CatCC$-module C$^*$-category} based over a set $\mathscr{J}$. That is, $\CatDD$ consists of a collection of semi-simple C$^*$-categories $\CatD_{\alpha\beta}$ with $\alpha\in \mathscr{I},\beta\in \mathscr{J}$, together with tensor products $\otimes: \CatC_{\alpha\beta}\times \CatD_{\beta\gamma}\rightarrow \CatC{\alpha\gamma}$ satisfying the appropriate associativity and unit constraints. Then if $\mathcal{I}$ labels a distinguished maximal set $\{u_k\}$ of mutually non-isomorphic irreducible objects of $\CatD$, with associated bigrading $\Gru{\mathcal{I}}{\alpha}{\beta}$ over $\mathscr{I}\times \mathscr{J}$, we can again define \[F_{kl}(X)  = \Hom(u_k,  X\otimes u_l),\qquad X\in \Gru{\CatC}{\alpha}{\beta}, k\in \Gru{\mathcal{I}}{\alpha}{\gamma},l\in \Gru{\mathcal{I}}{\beta}{\gamma},\] and we obtain a unital morphism from $\CatCC$ to $\{\Hilb_{\fin}\}_{\mathcal{I}\times \mathcal{I}}$. The associated partial compact quantum group $\mathscr{A}_{\CatCC}$ will be called the \emph{canonical partial compact quantum group} associated with $(\CatCC,\CatDD)$. The previous construction coincides with the special case $\CatCC= \CatDD$.
\end{Exa}

\begin{Exa}\label{ExaErgo} As a particular instance, let $\G$ be a compact quantum group, and consider an ergodic action of $\G$ on a unital C$^*$-algebra $C(\mathbb{X})$. Then the collection of finitely generated $\G$-equivariant $C(\mathbb{X})$-Hilbert modules forms a module C$^*$-category over $\Rep_u(\G)$, cf.~ \cite{DCY1}. 
\end{Exa}

\subsection{Morita equivalence}

% Caenepeel Galois theory weak Hopf algebras

\begin{Def} Two partial compact quantum groups $\mathscr{G}$ and $\mathscr{H}$ are said to be \emph{Morita equivalent} if there exists an equivalence $\Rep_u(\mathscr{G}) \rightarrow \Rep_u(\mathscr{H})$ of partial fusion C$^*$-categories. 
\end{Def} 

In particular, if $\mathscr{G}$ and $\mathscr{H}$ are Morita equivalent they have the same hyperobject set, but they need not share the same object set.

Our goal is to give a concrete implementation of Morita equivalence, as has been done for compact quantum groups \cite{BDV1}. Note that we slightly changed their terminology of monoidal equivalence into Morita equivalence, as we feel the monoidality is intrinsic to the context. We introduce the following definition, in which indices are considered modulo 2. 

\begin{Def} A \emph{linking partial compact quantum group} consists of a partial compact quantum group $\mathscr{G}$ defined by a partial Hopf $^*$-algebra $\mathscr{A}$ over a set $I$ with a distinguished partition $I = I_1\sqcup I_2$ such that the units $\UnitC{i}{j} = \sum_{k\in I_i,l\in I_j} \UnitC{k}{l} \in M(A)$ are central, and such that for each $r\in I_i$, there exists $s\in I_{i+1}$ such that $\UnitC{r}{s}\neq 0$.
\end{Def}

If $\mathscr{A}$ defines a linking partial compact quantum group, we can split $A$ into four components $A^i_j = A\UnitC{i}{j}$. It is readily verified that the $A^i_i$ together with all $\Delta_{rs}$ with $r,s \in I_i$ define themselves partial compact quantum groups, which we call the \emph{corner} partial compact quantum groups of $\mathscr{A}$. 

\begin{Prop} Two partial compact quantum groups are Morita equivalent iff they arise as the corners of a linking partial compact quantum group.
\end{Prop}

\begin{proof} Suppose first that $\mathscr{G}_1$ and $\mathscr{G}_2$ are Morita equivalent partial compact quantum groups with associated partial Hopf $^*$-algebras $\mathscr{A}_1$ and $\mathscr{A}_2$ over respective sets $I_1$ and $I_2$. Then we may identify their corepresentation categories with the same abstract partial tensor C$^*$-category $\CatCC$ based over their common hyperobject set $\mathscr{I}$. Then $\CatCC$ comes endowed with two forgetful functors $F_i$ to $\{\Hilb_{\fin}\}_{I_i\times I_i}$ corresponding to the respective $\mathscr{A}_i$.

With $I = I_1\sqcup I_2$, we may then as well combine the $F_i$ into a global unital morphism $F:\CatCC \rightarrow \{\Hilb_{\fin}\}_{I\times I}$, with $F_{kl}(X)=F_i(X)$ if $k,l\in I_i$ and $F_{kl}(X)=0$ otherwise. Let $\mathscr{A}$ be the associated partial Hopf $^*$-algebra constructed from the Tannaka-Krein-Woronowicz reconstruction procedure. 

From the precise form of this reconstruction, it follows immediately that $\Gr{A}{k}{l}{m}{n} =0$ if either $k,l$ or $m,n$ do not lie in the same $I_i$. Hence the $\UnitC{i}{j} = \sum_{k\in I_i,l\in I_j} \UnitC{k}{l}$ are central. 

Moreover, fix $k\in I_i$ and any $l\in I_{i+1}$ with $k'=l'$. Then $\Nat(F_{ll},F_{kk})\neq \{0\}$. It follows that $\UnitC{k}{l}\neq 0$. Hence $\mathscr{A}$ is a linking compact quantum groupoid. It is clear that $\mathscr{A}_1$ and $\mathscr{A}_2$ are the corners of $\mathscr{A}$. 

Conversely, suppose that $\mathscr{A}_1$ and $\mathscr{A}_2$ arise from the corners of a linking partial compact quantum groupoid defined by $\mathscr{A}$ with invariant integral $\phi$. We will show that in fact the associated partial compact quantum groups $\mathscr{G}$ and $\mathscr{G}_1$ are Morita equivalent. Then by symmetry $\mathscr{G}$ and $\mathscr{G}_2$ are Morita equivalent, and hence also $\mathscr{G}_1$ and $\mathscr{G}_2$.

For $(V,\mathscr{X}) \in \Corep_u(\mathscr{A})$, let $F(V,\mathscr{X}) = (W,\mathscr{Y})$ be the pair obtained from $(V,\mathscr{X})$ by restricting all indices to those appearing to $I_1$. It is immediate that $(W,\mathscr{Y})$ is a unitary rcfd corepresentation of $\mathscr{A}_1$, and that the functor $F$ hence becomes a unital morphism in a trivial way. What remains to show is that $F$ is an equivalence of categories, i.e.~ that $F$ is faithful and essentially surjective. 

Let us first argue that any unitary rcfd corepresentation $(W,\mathscr{Y})$ of $\mathscr{A}_1$ arises as a summand of some $F(V,\mathscr{X})$. Indeed, we may assume that $W$ is an irreducible corepresentation, and arises as regular corepresentation inside $A$. Taking any non-zero element of this corepresentation, and considering a regular rcfd corepresentation of $\mathscr{A}$ containing this element, we obtain an rcfd corepresentation of $\mathscr{A}$ whose restriction contains $W$.

Let now $\mathscr{Y}$ define a unitary corepresentation of $\mathscr{A}_1$ on a partial Hilbert space $W= \{\Gru{W}{k}{l}\}$. For $m,n\in I_{2}$, define $G_{mn}(W) = \Gru{Z}{m}{n}$ by \[ \Gru{Z}{m}{n} = \{z \in \left(\prod_k\oplus_l\cap \prod_l\oplus_k\right) \Gr{A}{k}{l}{m}{n}\otimes \Gru{W}{k}{l}\mid (\Delta_{rs}\otimes \id)(z_{kl}) = (\Gr{Y}{k}{l}{r}{s})_{13}(z_{rs})_{23},\quad \forall r,s\in I_1\}.\] We aim to show that $G_{mn}(W)$ is rcfd. In fact, $G$ is clearly functorial and linear. By the previous paragraph, we may thus assume that $(W,\mathscr{Y})$ arises as the restriction of some corepresentation $(V,\mathscr{X})$ of $\mathscr{A}$. 

Let then $Z$ be obtained as above from the restriction $W$ of $V$. We claim that the maps \[\theta_{mn}:\Gru{V}{m}{n}\rightarrow \Gru{Z}{m}{n},\quad v \mapsto \sum_{k,l\in I_1} \Gr{X}{k}{l}{m}{n}(1\otimes v),\quad m,n\in I_2\]  are well-defined linear isomorphisms. In fact, well-definedness is immediate. To show that the map is an isomorphism, let us construct a concrete inverse map. Fix $m,n$, and pick an $s\in I_1$ with $\UnitC{s}{n}\neq 0$, which is possible by assumption. Define then \[\gamma_{mn}:\Gru{Z}{m}{n} \rightarrow \Gru{V}{m}{n},\quad z \mapsto (\phi\otimes \id)\sum_{k\in I_1} \Gr{(X^{-1})}{s}{k}{n}{m}z_{ks},\] where we note that the summation over $k$ is in fact finite as the variables $m,n,s$ are fixed. It follows immediately from the definition of $X^{-1}$ (and the fact that $\UnitC{s}{n}\neq 0$) that $\phi_{mn}\theta_{mn} = \id_{\Gru{V}{m}{n}}$. To show the reverse identity $\theta_{mn}\phi_{mn} = \id_{\Gru{Z}{m}{n}}$, it suffices to show that $z_{ks} = (\theta_{mn}\gamma_{mn}(z))_{ks}$. Let us first show that \begin{equation}\label{EqIdz} \UnitC{s}{n}\otimes \sum_k (\phi\otimes \id)(\Gr{(X^{-1})}{s}{k}{n}{m}z_{ks}) = \sum_k \Gr{(X^{-1})}{s}{k}{n}{m}z_{ks}.\end{equation} Apply $\Delta_{nn}\otimes \id$ to the right hand side and use the invertibility of $X$ to conclude that  \[(\Delta_{nn}\otimes \id)\left(\sum_k \Gr{(X^{-1})}{s}{k}{n}{m}z_{ks}\right) = \UnitC{s}{n}\otimes \Gr{(X^{-1})}{n}{n}{n}{m}z_{nn}.\] Apply the counit to the second leg to conclude that in fact $\sum_k \Gr{(X^{-1})}{s}{k}{n}{m}z_{ks} \in \UnitC{s}{n} \otimes \Gru{W}{m}{n}$, from which $\eqref{EqIdz}$ follows. Similarly, for $n\neq n'$ we find  \[(\Delta_{n'n}\otimes \id)\left(\sum_k \Gr{(X^{-1})}{s}{k}{n'}{m}z_{ks}\right) =0,\] and applying the counit gives \begin{equation}\label{EqIdz2} \sum_k \Gr{(X^{-1})}{s}{k}{n'}{m}z_{ks} =0.\end{equation} From \eqref{EqIdz} and \eqref{EqIdz2}, we then conclude 
\begin{eqnarray*} (\theta_{mn}\gamma_{mn}(z))_{ks} &=& \sum_{k'} \Gr{X}{k}{s}{m}{n}\Gr{(X^{-1})}{s}{k'}{n}{m}z_{k's} \\ &=& \sum_{k',n'} \Gr{X}{k}{s}{m}{n'}\Gr{(X^{-1})}{s}{k'}{n'}{m}z_{k's} \\ &=& z_{ks}.\end{eqnarray*} 

As a special case, obtained by considering the trivial linking quantum groupoid between $\mathscr{A}$ and itself, one has that $\Gru{W}{m}{n}\cong G_{mn}(W)$ for $m,n\in I_1$ and with $G_{mn}(W)$ defined by the same expression as above.

Let us now take again an arbitrary unitary rcfd corepresentation $(W,\mathscr{Y})$ of $\mathscr{A}_1$. By the above, $\Gru{V}{m}{n} := G_{mn}(W)$ is rcfd over $I$. It is easy to see that we can then build an rcfd corepresentation $\mathscr{X}$ of $\mathscr{A}$ on $V$ by putting \[\Gr{X}{k}{l}{m}{n}(1\otimes z) = (\Delta_{kl}^{\op}\otimes \id)(z),\qquad z\in \Gru{Z}{m}{n}, \quad k,l,m,n\in I.\] It is then easy to see that, when $V$ is a unitary rcfd corepresentation, $V$ is equivariantly isomorphic to $GF(V)$, while any unitary rcfd corepresentation $W$ of $\mathscr{A}$ is equivariantly isomorphic to $FG(W)$. This proves essential surjectivity and faithfulness.
\end{proof}

\begin{Exa} If $\mathscr{G}_1$ and $\mathscr{G}_2$ are Morita equivalent compact quantum groups, the total partial compact quantum group is the co-groupoid constructed in \cite{Bic1}. 
\end{Exa}

\begin{Exa}  Let $\G$ be a compact quantum group with ergodic action on a unital C$^*$-algebra $C(\mathbb{X})$. Consider the module C$^*$-category $\CatD$ of finitely generated $\G$-equivariant Hilbert $C(\mathbb{X})$-modules as in Example \ref{ExaErgo}. Then $\G$ is Morita equivalent with the canonical partial compact quantum group constructed from $(\Rep_u(\G),\CatD)$. The off-diagonal part of the associated linking partial compact quantum group was studied in \cite{DCY1}. We will make a detailed study of the case $\G = SU_q(2)$ in \cite{DCT2}, in particular for $\X$ a Podle\'{s} sphere. This will lead us to partial compact quantum group versions of the dynamical quantum $SU(2)$-group.
\end{Exa}


\subsection{Weak Morita equivalence}

\begin{Def} A \emph{linking} partial fusion C$^*$-category consists of a partial fusion C$^*$-category with a distinguished partition $\mathscr{I} =\mathscr{I}_1 \cup \mathscr{I}_2$ such that for each $\alpha\in \mathscr{I}_1$, there exists $\beta \in \mathscr{I}_{2}$ with $\CatC_{\alpha\beta}\neq \{0\}$.

The \emph{corners} of $\CatCC$ are the restrictions of $\CatCC$ to $\mathscr{I}_1$ and $\mathscr{I}_2$.
\end{Def}

The following notion is essentially the same as the one by M. M\"{u}ger \cite{Mug1}. 

\begin{Def} Two partial semi-simple tensor C$^*$-categories $\CatCC_1$ and $\CatCC_2$ with duality over respective sets $\mathscr{I}_1$ and $\mathscr{I}_2$ are called \emph{Morita equivalent} if there exists a linking partial fusion C$^*$-category $\CatCC$ over the set $\mathscr{I}=\mathscr{I}_1\sqcup \mathscr{I}_2$ whose corners are isomorphic to $\CatCC_1$ and $\CatCC_2$.

We say two partial compact quantum groups $\mathscr{G}_1$ and $\mathscr{G}_2$ are \emph{weakly Morita equivalent} if their corepresentation categories $\Corep_u(\mathscr{G}_i)$ are Morita equivalent. 
\end{Def} 

Although it is possible to prove directly that this is indeed an equivalence relation, we will take a roundabout way and first consider a more concrete notion based on the notion of \emph{co-Morita equivalence} of partial compact quantum groups.

\begin{Def}\label{DefCoLink} A \emph{co-linking partial compact quantum group} consists of a partial compact quantum group $\mathscr{G}$ defined by a Hopf $^*$-algebra $\mathscr{A}$ over an index set $I$, together with a distinguished partition $I = I_1\cup I_2$ such that each $\UnitC{k}{l}=0$ for $k\in I_i$ and $l\in I_{i+1}$, and such that for each $k\in I_i$, there exists $l\in I_{i+1}$ with $\Gr{A}{k}{l}{k}{l}\neq 0$.  
\end{Def} 

It is again easy to see that if we restrict all indices of a co-linking partial compact quantum group to one of the distinguished sets, we obtain a partial compact quantum group which we will call a corner. In fact, write $e_i = \sum_{k,l\in I_i} \UnitC{k}{l}$. Then we can decompose the total algebra $A$ into components $A_{ij} = e_{i}Ae_{j}$, and correspondingly write $A$ in matrix notation \[ A = \begin{pmatrix} A_{11} & A_{12}  \\ A_{21} & A_{22}\end{pmatrix},\] where multiplication is matrixwise and where comultiplication is entrywise. Note that we have $A_{12}A_{21} = A_{11}$, and similarly $A_{21}A_{12} = A_{22}$. Indeed, take $k\in I_1$, and pick $l\in I_2$ with $\Gr{A}{k}{l}{k}{l}\neq \{0\}$. Then in particular, we can find an $a\in \Gr{A}{k}{l}{k}{l}$ with $\epsilon(a)\neq 0$. Hence for any $m\in I_1$, we have $\UnitC{k}{m} = \UnitC{k}{m} a_{(1)}S(a_{(2)}) \in A_{12}A_{21}$. As this latter space contains all local units of $A_{11}$ and is a right $A_{11}$-module, it follows that it is in fact equal to $A_{11}$. We hence deduce that in fact $A_{11}$ and $A_{22}$ are Morita equivalent algebras, with the Morita equivalence implemented by $A$. % Cf. Abrams? 

\begin{Rem} For finite partial compact quantum groups, one can then easily show that the notion of co-linking partial compact quantum group is dual to the notion of linking partial compact quantum group.\end{Rem}

\begin{Def} We call two partial compact quantum groups \emph{co-Morita equivalent} if there exists a \emph{co-linking partial compact quantum group} having these partial compact quantum groups as its corners.
\end{Def}

\begin{Lem} Co-Morita equivalence is an equivalence relation. 
\end{Lem} 

\begin{proof} Symmetry is clear. Co-Morita equivalence of $\mathscr{A}$ with itself follows by considering as co-linking quantum groupoid the product of $\mathscr{A}$ with the partial compact quantum groupoid $M_2(\C)$, where $\Delta(e_{ij}) = e_{ij}\otimes e_{ij}$. 

Let us show the main elements to prove transitivity. Let us assume $\mathscr{G}_1$ and $\mathscr{G}_2$ as well as $\mathscr{G}_2$ and $\mathscr{G}_3$ are co-Morita equivalent. Let us write the global $^*$-algebras of the associated co-linking quantum groupoids as \[A_{\{1,2\}} = \begin{pmatrix} A_{11} & A_{12} \\ A_{21} & A_{22} \end{pmatrix}, \quad A_{\{2,3\}} = \begin{pmatrix} A_{22} & A_{23} \\ A_{32} & A_{33}\end{pmatrix}.\] Then we can define a new $^*$-algebra $A_{\{1,2,3\}}$ as \[ A_{\{1,2,3\}} = \begin{pmatrix} A_{11} & A_{12} &   A_{13} \\ A_{21} & A_{22} & A_{23} \\ A_{31} & A_{32} & A_{33} \end{pmatrix},\] where $A_{13} = A_{12}\underset{A_{22}}{\otimes } A_{23}$ and $A_{31} = A_{32}\underset{A_{22}}{\otimes} A_{21}$, and with multiplication and $^*$-structure defined in the obvous way. 

It is straightforward to verify that there exists a unique $^*$-homomorphism $\Delta: A_{\{1,2,3\}} \rightarrow M(A_{\{1,2,3\}}\otimes A_{\{1,2,3\}})$ whose restrictions to the $A_{ij}$ with $|i-j|\leq 1$ coincide with the already defined coproducts. We leave it to the reader to verify that $(A,\Delta)$ defines a regular weak multiplier Hopf $^*$-algebra satisfying the conditions of Proposition \ref{PropCharPBA}, and hence arises from a regular partial weak Hopf $^*$-algebra. 

Let now $\phi$ be the functional which is zero on the off-diagonal entries $A_{ij}$ and coincides with the invariant positive integrals on the $A_{ii}$. Then it is also easily checked that $\phi$ is invariant. To show that $\phi$ is positive, we invoke Remark \ref{RemPos}. Indeed, any irreducible corepresentation of $A_{\{1,2,3\}}$ has coefficients in a single $A_{ij}$. For those $i,j$ with $|i-j|\leq 1$, we know that the corepresentation is unitarizable by restricting to a corner $2\times 2$-block. If however the corepresentation $\mathscr{X}$ has coefficients living in (say) $A_{13}$, it follows from the identity $A_{12}A_{23}=A_{13}$ that the corepresentation is a direct summand of a product $\mathscr{Y}\Circt \mathscr{Z}$ of corepresentations with coefficients in respectively $A_{12}$ and $A_{23}$. This proves unitarizability of $\mathscr{X}$. It follows from Remark \ref{RemPos} that $\phi$ is positive, and hence $\mathscr{A}_{\{1,2,3\}}$ defines a partial compact quantum group.  

We claim that the subspace $\mathscr{A}_{\{1,3\}}$ (in the obvious notation) defines a co-linking compact quantum group between $\mathscr{G}_1$ and $\mathscr{G}_2$. In fact, it is clear that the $\mathscr{A}_{11}$ and $\mathscr{A}_{33}$ are corners of $\mathscr{A}_{\{1,3\}}$, and that $\UnitC{k}{l}=0$ for $k,l$ not both in $I_1$ and $I_{3}$. To finish the proof, it is sufficient to show now that for each $k\in I_1$, there exists $l\in I_{3}$ with $\Gr{A}{k}{l}{k}{l}\neq 0$, as the other case follows by symmetry using the antipode. But there exists $m\in I_2$ with $\Gr{A}{k}{m}{k}{m} \neq \{0\}$, and $l\in I_3$ with $\Gr{A}{m}{l}{m}{l}\neq\{0\}$. As in the discussion following Definition \ref{DefCoLink}, this implies that there exists $a\in \Gr{A}{k}{m}{k}{m}$ and $b\in \Gr{A}{m}{l}{m}{l}$ with $\epsilon(a)=\epsilon(b)=1$. Hence $\epsilon(ab)=1$, showing $\Gr{A}{k}{l}{k}{l}\neq \{0\}$.
\end{proof} 

\begin{Prop}\label{PropCoWeak} Assume that two partial compact quantum groups $\mathscr{G}_1$ and $\mathscr{G}_2$ are co-Morita equivalent. Then they are weakly Morita equivalent.
\end{Prop} 
\begin{proof} 
Consider the corepresentation category $\CatCC$ of a co-linking partial compact quantum group $\mathscr{A}$ over $I = I_1\cup I_2$. Let $\varphi:I\rightarrow \mathscr{I}$ define the corresponding partition along the hyperobject set. Then by the defining property of a co-linking partial compact quantum group, also $\mathscr{I} = \mathscr{I}_1\cup \mathscr{I}_2$ with $\mathscr{I}_i=\varphi(I_i)$ is a partition. Hence $\CatCC$ decomposes into parts $\CatCC_{ij}$ with $i,j\in \{1,2\}$ and $\CatC_{ii}\cong \Rep_u(\mathscr{G}_i)$. 

To show that $\mathscr{G}_1$ and $\mathscr{G}_2$ are weakly Morita equivalent, it thus suffices to show that $\{\CatCC_{ij}\}$ forms a linking partial fusion C$^*$-category. But fix $\alpha\in I_1$ and $k\in I_{\alpha}$. Then as $\mathscr{A}$ is co-linking, there exists $l \in I_2$ with $\Gr{A}{k}{l}{k}{l}\neq \{0\}$. Hence there exists a non-zero regular rcfd unitary corepresentation inside $\oplus_{m,n}\Gr{A}{k}{l}{m}{n}$. If then $l\in I_{\beta}$ with $\beta\in \mathscr{I}_2$, it follows that $\CatC_{\alpha\beta}\neq 0$. By symmetry, we also have that for each $\alpha \in \mathscr{I}_2$ there exists $\beta \in \mathscr{I}_1$ with $\CatC_{\alpha\beta}\neq \{0\}$. This proves that the $\{\CatCC_{ij}\}$ forms a linking partial fusion C$^*$-category.
\end{proof}

\begin{Prop}\label{PropCoLink} Let $\CatCC$ be a linking $\mathscr{I}$-partial fusion C$^*$-category. Then the associated canonical partial compact quantum group is a co-linking partial compact quantum group. 
\end{Prop} 

\begin{proof} Let $\mathscr{I}= \mathscr{I}_1\cup \mathscr{I}_2$ be the associated partition of $\mathscr{I}$. Let $\mathscr{A} = \mathscr{A}_{\CatCC}$ define the canonical partial compact quantum group with object set $I$ and hyperobject partition $\varphi:I\rightarrow \mathscr{I}$. Let $I=I_1\cup I_2$ with $I_i = \varphi^{-1}(\mathscr{I}_i)$ be the corresponding decomposition of $I$. By construction, $\UnitC{k}{l}=0$ if $k$ and $l$ are not both in $I_1$ or $I_2$. 

Fix now $k\in I_{\alpha}$ for some $\alpha \in I_i$. Pick $\beta\in I_{i+1}$ with $\CatC_{\alpha\beta}\neq\{0\}$, and let $(V,\mathscr{X})$ be a non-zero irreducible corepresentation inside $\CatC_{\alpha\beta}$. Then by irreducibility, we know that $\oplus_l \Gru{V}{k}{l} \neq \{0\}$, hence there exists $l\in I_{\beta}$ with $\Gru{V}{k}{l}\neq \{0\}$. As $(\epsilon\otimes \id)\Gr{X}{k}{l}{k}{l} = \id_{\Gru{V}{k}{l}}$, it follows that $\Gr{A}{k}{l}{k}{l} \neq 0$. This proves that $\mathscr{A}$ defines a co-linking partial compact quantum group.
\end{proof} 

\begin{Cor} The notion of Morita equivalence of partial fusion C$^*$-categories is an equivalence relation.
\end{Cor} 
\begin{proof}  It is clear that the relation is symmetric and reflexive. Let us show transitivity.

Let $\CatCC_1\underset{\textrm{Mor}}{\sim} \CatCC_2$ and $\CatCC_2\underset{\textrm{Mor}}{\sim} \CatCC_3$. Then $\mathscr{A}_{\CatCC_1}$ is co-Morita equivalent with  $\mathscr{A}_{\CatCC_2}$, which is in turn co-Morita equivalent with $\mathscr{A}_{\CatCC_3}$. This implies that $\mathscr{A}_{\CatCC_1}$ is co-Morita equivalent with $\mathscr{A}_{\CatCC_3}$, which proves $\CatCC_1\underset{\textrm{Mor}}{\sim} \CatCC_3$.
\end{proof} 

\begin{Theorem} Two partial compact quantum groups $\mathscr{G}_1$ and $\mathscr{G}_2$ are weakly Morita equivalent if and only if they are connected by a string of Morita and co-Morita equivalences. 
\end{Theorem}

\begin{proof} Clearly if two partial compact quantum groups are Morita equivalent, they are weakly Morita equivalent. By Proposition \ref{PropCoWeak}, the same goes for co-Morita equivalence. This proves one direction of the theorem. 

Conversely, assume $\mathscr{G}_1$ and $\mathscr{G}_2$ are weakly Morita equivalent. Then $\mathscr{G}_i$ is Morita equivalent with the canonical partial compact quantum group associated to its corepresentation category. But Proposition \ref{PropCoLink} shows that these canonical partial compact quantum groups are co-Morita equivalent. 
\end{proof} 

\begin{Rem} Note that it is essential that we allow the string of equivalences to pass through partial compact quantum groups, even if we start out with (genuine) compact quantum groups.\end{Rem}





%%% Local Variables: 
%%% mode: latex
%%% TeX-master: "dyn-suq-main"
%%% End: 

