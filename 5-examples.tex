
\section{Examples}

\subsection{Classical examples}

\begin{Exa} Let $G$ be a discrete groupoid with object set $I = G^{(0)}$. Consider for $r,s\in I$ the vector space $\Gru{A}{r}{s}= \mathbb{C}\lbrack \Gru{G}{r}{s}\rbrack$ which have a basis $\{\lambda_g\}$ spanned by the morphisms $g$ from $r$ to $s$. Linearly extending the product of $G$ to the $\Gru{A}{r}{s}$ turns $\mathscr{A}$ into a partial algebra. It becomes a partial $^*$-algebra by putting $\lambda_g^* = \lambda_{g^{-1}}$. 

Extend the $I^2$-bigrading to an $I^4$-bigrading by putting $\Gr{A}{k}{l}{m}{n} = \delta_{kl}\delta_{mn}\Gru{A}{k}{l}$. Then together with the coproducts \[ \Delta: \Gru{A}{r}{s}\rightarrow \Gru{A}{r}{s}\otimes \Gru{A}{r}{s},\quad \lambda_g\mapsto \lambda_g\otimes \lambda_g,\] $(\mathscr{A},\Delta)$ defines a partial compact quantum group, the invariant functional $\phi$ being given by \[\phi(\lambda_g) = \delta_{rs}\delta_{g,\id_r},\quad r,s\in I,g\in \Mor(r,s).\]
\end{Exa}

\begin{Exa} Let $G$ be a proper locally compact groupoid with discrete object space $I=G^{(0)}$. Then each $\Gru{G}{r}{s} = \Mor(r,s)$ is a compact space. The $\Gru{G}{r}{r}$ are compact groups, hence come with probability Haar measures $\mu_{rr}$. The $\Gru{G}{r}{s}$ are $\Gru{G}{r}{r}$-$\Gru{G}{s}{s}$-bitorsors, and as such admit a bi-invariant probability measure $\mu_{rs}$. Let $A^r_s \subseteq C(\Gru{G}{r}{s})$ be the spaces of functions which transform as finite-dimensional representations of $\Gru{G}{r}{r}\times \Gru{G}{s}{s}$ under the natural actions. Then the $A^r_s$ form a partial coalgebra over $I$ by putting \[\Delta_{s}: A^r_t \rightarrow A^r_s\otimes A^s_t,\quad f\mapsto \left(\Delta_s(f):\Gru{G}{r}{s}\times \Gru{G}{s}{t}\mapsto \Gru{G}{r}{t},\quad (g,h)\mapsto f(gh)\right).\] We can extend the $I^2$-grading to an $I^4$-grading by putting \[\Gr{A}{k}{l}{m}{n} = \delta_{kl}\delta_{mn} A^l_n,\] and the resulting $\mathscr{A}$ becomes a partial $^*$-algebra by endowing each $A^r_s$ with the pointwise product and $^*$-algebra structure. The couple $(\mathscr{A},\Delta)$ then defines a partial compact quantum group with invariant integral \[\phi: A^r_s \mapsto \C,\quad f\mapsto \int_{\Gru{G}{r}{s}} f(g) \rd \Gru{\mu}{r}{s}(g).\]

\end{Exa}

% Vacant double groupoids?

\begin{Exa} Let $\mathscr{A}$ and $\mathscr{B}$ define two partial compact quantum groups $\mathscr{G}$ and $\mathscr{H}$ over respective sets $I$ and $J$. Then we can make a tensor product partial Hopf $^*$-algebra $\mathscr{A}\otimes \mathscr{B}$ over the index set $I\times J$ by putting \[\Gr{(A\otimes B)}{(k,k')}{(l,l')}{(m,m')}{(n,n')} = \Gr{A}{k}{l}{m}{n}\otimes \Gr{B}{k'}{l'}{m'}{n'}\] with the factorwise product and with coproducts \[\Delta_{(r,r'),(s,s')} = \sigma_{23}(\Delta_{rs}\otimes \Delta_{r',s'}),\] $\sigma$ being the switch map. It is easily seen that the tensor products of the positive invariant integrals for $\mathscr{A}$ and $\mathscr{B}$ produce a positive invariant integral on $\mathscr{A}\otimes \mathscr{B}$. Hence $\mathscr{A}\otimes \mathscr{B}$ defines a partial compact quantum group, which we will denote $\mathscr{G}\times \mathscr{H}$.
\end{Exa}

\subsection{Canonical partial compact quantum groupoids}

The following generalizes Hayashi's original construction.

% Notion of faithfulness to remark upon.
\begin{Exa} 
Let $\CatCC$ be a semi-simple partial tensor C$^*$-category with duality based over a set $I_0$. Let $\mathcal{I}$ label a distinguished maximal set $\{u_k\}$ of mutually non-isomorphic irreducible objects of $\CatC$, with associated bigrading $\Gru{\mathcal{I}}{r}{s}$ over $I_0$. Define \[F_{kl}(X)  = \Hom(u_k,  X\otimes u_l),\qquad X\in \Gru{\CatC}{r}{s}, k\in \Gru{\mathcal{I}}{r}{t},l\in \Gru{\mathcal{I}}{s}{t}.\] Then each $F_{kl}(X)$ is a Hilbert space by the inner product $\langle f,g\rangle = f^*g$. Put $F_{kl}(X) = 0$ for $k,l$ outside their proper domains. Then clearly the application $(k,l)\mapsto F_{kl}(X)$ is rcf. Moreover, we have isometric compatibility morphisms \[F_{kl}(X)\otimes F_{lm}(Y)\rightarrow F_{km}(X\otimes Y),\quad f\otimes g \mapsto (\id\otimes g)f,\] while $F_{kl}(\Unit_r) \cong \delta_{kl} \C$ for $k,l\in \Gru{\mathcal{I}}{r}{r}$. 

It is readily verified that $F$ defines a unital morphism from $\CatCC$ to $\{\Hilb_{\fin}\}_{\mathcal{I}\times \mathcal{I}}$ based over the partition \[\mathcal{I}_r = \bigcup_{t} \Gru{\mathcal{I}}{r}{t},\quad r\in I_0.\] From the Tannaka-Krein-Woronowicz reconstruction result, we obtain a partial compact quantum group $\mathscr{A}_{\CatCC}$ with object set $\mathcal{I}$, which we call the \emph{canonical partial compact quantum group} associated with $\CatCC$. 
\end{Exa} 

\begin{Exa} More generally, let $\CatCC$ be a semi-simple partial tensor C$^*$-category with duality based over a set $I_0$, and let $\CatDD$ be a \emph{semi-simple partial $\CatCC$-module C$^*$-category} based over a set $I_1$. That is, $\CatDD$ consists of a collection of semi-simple C$^*$-categories $\CatD_{rs}$ with $r\in I_0,s\in I_1$, together with tensor products $\otimes: \CatC_{rs}\times \CatD_{st}\rightarrow \CatC{rt}$ satisfying the appropriate associativity and unit constraints. Then if $\mathcal{I}$ labels a distinguished maximal set $\{u_k\}$ of mutually non-isomorphic irreducible objects of $\CatD$, with associated bigrading $\Gru{\mathcal{I}}{r}{s}$ over $I_0\times I_1$, we can again define \[F_{kl}(X)  = \Hom(u_k,  X\otimes u_l),\qquad X\in \Gru{\CatC}{r}{s}, k\in \Gru{\mathcal{I}}{r}{t},l\in \Gru{\mathcal{I}}{s}{t},\] and we obtain a unital morphism from $\CatCC$ to $\{\Hilb_{\fin}\}_{\mathcal{I}\times \mathcal{I}}$. The associated partial compact quantum group $\mathscr{A}_{\CatCC}$ will be called the \emph{canonical partial compact quantum group} associated with $(\CatCC,\CatDD)$. The previous construction coincides with the special case $\CatCC= \CatDD$.
\end{Exa}

For example, let $\G$ be a compact quantum group, and consider an ergodic action of $\G$ on a unital C$^*$-algebra $C(\mathbb{X})$. Then the collection of finitely generated $\G$-equivariant $C(\mathbb{X})$-Hilbert modules forms a module C$^*$-category over $\Rep(\G)$, cf. []. 

\subsection{Morita equivalence}

% Caenepeel Galois theory weak Hopf algebras

\begin{Def} Two partial compact quantum groups $\mathscr{G}$ and $\mathscr{H}$ are said to be \emph{Morita equivalent} if there exists an equivalence $\Rep(\mathscr{G}) \rightarrow \Rep(\mathscr{H})$ of partial tensor C$^*$-categories. % notion of equivalence to be explained? $\varphi_0$ is bijective, and all $F_{rs}$ are faithful and essentially surjective.
\end{Def} 

In particular, if $\mathscr{G}$ and $\mathscr{H}$ are Morita equivalent they have the same hyperobject set, but they need not share the same object set.

Our goal is to give a concrete implementation of Morita equivalence, as has been done for compact quantum groups []. We introduce the following definition. In it, we treat $\{0,1\}$ as the two-element group under addition.

\begin{Def} A \emph{linking partial compact quantum group} consists of a partial compact quantum group $\mathscr{G}$ defined by a partial Hopf $^*$-algebra $\mathscr{A}$ over a set $I$ with a distinguished partition $I = I_1\sqcup I_2$ such that the units $\UnitC{i}{j} = \sum_{k\in I_i,l\in I_j} \UnitC{k}{l} \in M(A)$ are central, and such that for each $r\in I_i$, there exists $s\in I_{i+1}$ such that $\UnitC{r}{s}\neq 0$.
%moreover for each $a\in \Gr{A}{k}{l}{m}{n}$ with $k,l,m,n\in I_i$ we can find $r,s \in I_{i+1}$ such that $\Delta_{rs}(a) \neq0$. 
\end{Def}

If $\mathscr{A}$ defines a linking partial compact quantum group, we can split $A$ into four components $A^i_j = A\UnitC{i}{j}$. It is readily verified that the $A^i_i$ together with all $\Delta_{rs}$ with $r,s \in I_i$ define themselves partial compact quantum groups, which we call the \emph{corner} partial compact quantum groups of $\mathscr{A}$. 

\begin{Prop} Two partial compact quantum groups are Morita equivalent iff they arise as the corners of a linking partial compact quantum group.
\end{Prop}

% Some stylisation needed.
\begin{proof} Suppose first that $\mathscr{G}_1$ and $\mathscr{G}_2$ are partial compact quantum groups with associated partial Hopf $^*$-algebras $\mathscr{A}_1$ and $\mathscr{A}_2$ over respective sets $I_1$ and $I_2$. Then we may identify their corepresentation categories with the same abstract partial tensor C$^*$-category $\CatCC$ (based over $I_0$) which comes endowed with two forgetful functors $F_i$ to $\{\Hilb_{\fin}\}_{I_i\times I_i}$ corresponding respectively to the $\mathscr{A}_i$.

With $I = I_1\sqcup I_2$, we may then as well combine the $F_i$ into a global unital morphism $F:\CatCC \rightarrow \{\Hilb_{\fin}\}_{I\times I}$, with $F_{kl}(X)=F_i(X)$ if $k,l\in I_i$ and $F_{kl}(X)=0$ otherwise. Let $\mathscr{A}$ be the associated partial compact quantum group constructed from the Tannaka-Krein-Woronowicz reconstruction procedure. 

From the precise form of this reconstruction, it follows immediately that $\Gr{A}{k}{l}{m}{n} =0$ if either $k,l$ or $m,n$ do not lie in the same $I_i$. Hence the $\UnitC{i}{j} = \sum_{k\in I_i,l\in I_j} \UnitC{k}{l}$ are central. 

Moreover, fix $k\in I_i$ and any $l\in I_{i+1}$ with $k'=l'$. Then $\Nat(F_{ll},F_{kk})\neq \{0\}$. It follows that $\UnitC{k}{l}\neq 0$. Hence $\mathscr{A}$ is a linking compact quantum groupoid. It is clear that $\mathscr{A}_1$ and $\mathscr{A}_2$ are the corners of $\mathscr{A}$. 

Conversely, suppose that $\mathscr{A}_1$ and $\mathscr{A}_2$ arise from the corners of a linking partial compact quantum groupoid defined by $\mathscr{A}$ with invariant integral $\phi$. We will show that in fact the associated partial compact quantum groups $\mathscr{G}$ and $\mathscr{G}_1$ are Morita equivalent. Then by symmetry $\mathscr{G}$ and $\mathscr{G}_2$ are Morita equivalent, and hence also $\mathscr{G}_1$ and $\mathscr{G}_2$.

For $(V,X) \in \Corep(\mathscr{A})$, let $F(V,X) = (W,Y)$ be the pair obtained from $(V,X)$ by restricting all indices to those appearing to $I_1$. As the $\UnitC{i}{j}$ are central, it is easy to see that $(W,Y)$ is a unitary corepresentation of $\mathscr{A}_1$, and that the functor $F$ hence becomes a unital morphism in a trivial way. What remains to show is that $F$ is an equivalence of categories, i.e.~ that $F$ is faithful and essentially surjective. 

Let us first argue that any rcf corepresentation $W$ of $\mathscr{A}_1$ arises as a summand of some $F(V)$. Indeed, we may assume that $W$ is an irreducible corepresentation, and arises as regular corepresentation. Taking any non-zero element of this corepresentation, and considering an rcf regular corepresentation of $\mathscr{A}$ containing this element, we obtain an rcf corepresentation of $\mathscr{A}$ whose restriction contains $W$.

Let now $X$ define a unitary corepresentation of $\mathscr{A}_1$ on a partial Hilbert space $W= \{\Gru{W}{k}{l}\}$. For $m,n\in I_{2}$, define $G_{mn}(W) = \Gru{Z}{m}{n}$ by \[ \Gru{Z}{m}{n} = \{z \in \left(\prod_k\oplus_l\cap \prod_l\oplus_k\right) \Gr{A}{k}{l}{m}{n}\otimes \Gru{W}{k}{l}\mid (\Delta_{rs}\otimes \id)(z_{kl}) = (\Gr{X}{k}{l}{r}{s})_{13}(z_{rs})_{23},\quad \forall r,s\in I_1\}.\] We aim to show that $G_{mn}(W)$ is rcf. In fact, $G$ is clearly functorial and linear. By the previous paragraph, we may thus assume that $W$ arises as the restriction of some corepresentation $V$ of $\mathscr{A}$. 

Let then $Z$ be obtained as above from the restriction $W$ of $V$. We claim that the maps \[\theta_{mn}:\Gru{V}{m}{n}\rightarrow \Gru{Z}{m}{n},\quad v \mapsto \sum_{k,l\in I_1} \Gr{X}{k}{l}{m}{n}(1\otimes v),\quad m,n\in I_2\]  are well-defined linear isomorphisms. In fact, well-definedness is immediate. To show that the map is an isomorphism, let us construct a concrete inverse map. Fix $m,n$, and pick an $s\in I_1$ with $\UnitC{s}{n}\neq 0$, which is possible by assumption. Define then \[\gamma_{mn}:\Gru{Z}{m}{n} \rightarrow \Gru{V}{m}{n},\quad z \mapsto (\phi\otimes \id)\sum_{k\in I_1} \Gr{(X^{-1})}{s}{k}{n}{m}z_{ks},\] where we note that the summation over $k$ is in fact finite as the variables $m,n,s$ are fixed. It follows immediately from the definition of $X^{-1}$ (and the fact that $\UnitC{s}{n}\neq 0$) that $\phi_{mn}\theta_{mn} = \id_{\Gru{V}{m}{n}}$. To show the reverse identity $\theta_{mn}\phi_{mn} = \id_{\Gru{Z}{m}{n}}$, it suffices to show that $z_{ks} = (\theta_{mn}\gamma_{mn}(z))_{ks}$. Let us first show that \begin{equation}\label{EqIdz} \UnitC{s}{n}\otimes \sum_k (\phi\otimes \id)(\Gr{(X^{-1})}{s}{k}{n}{m}z_{ks}) = \sum_k \Gr{(X^{-1})}{s}{k}{n}{m}z_{ks}.\end{equation} Apply $\Delta_{nn}\otimes \id$ to the right hand side and use the invertibility of $X$ to conclude that  \[(\Delta_{nn}\otimes \id)\left(\sum_k \Gr{(X^{-1})}{s}{k}{n}{m}z_{ks}\right) = \UnitC{s}{n}\otimes \Gr{(X^{-1})}{n}{n}{n}{m}z_{nn}.\] Apply the counit to the second leg to conclude that in fact $\sum_k \Gr{(X^{-1})}{s}{k}{n}{m}z_{ks} \in \UnitC{s}{n} \otimes \Gru{W}{m}{n}$, from which $\eqref{EqIdz}$ follows. Similarly, for $n\neq n'$ we find  \[(\Delta_{n'n}\otimes \id)\left(\sum_k \Gr{(X^{-1})}{s}{k}{n'}{m}z_{ks}\right) =0,\] and applying the counit gives \begin{equation}\label{EqIdz2} \sum_k \Gr{(X^{-1})}{s}{k}{n'}{m}z_{ks} =0.\end{equation} From \eqref{EqIdz} and \eqref{EqIdz2}, we then conclude 
\begin{eqnarray*} (\theta_{mn}\gamma_{mn}(z))_{ks} &=& \sum_{k'} \Gr{X}{k}{s}{m}{n}\Gr{(X^{-1})}{s}{k'}{n}{m}z_{k's} \\ &=& \sum_{k',n'} \Gr{X}{k}{s}{m}{n'}\Gr{(X^{-1})}{s}{k'}{n'}{m}z_{k's} \\ &=& z_{ks}.\end{eqnarray*} 

In a similar way, one shows that $W\cong W'$ with, for $m,n\in I_1$, \[ \Gru{W'}{m}{n} = \{z \in \left(\prod_k\oplus_l\cap \prod_l\oplus_k\right) \Gr{A}{k}{l}{m}{n}\otimes \Gru{W}{k}{l}\mid (\Delta_{rs}\otimes \id)(z_{kl}) = (\Gr{X}{k}{l}{r}{s})_{13}(z_{rs})_{23},\quad \forall r,s\in I_1\}.\] 

Let us now take again an arbitrary unitary corepresentation $(W,Y)$ of $\mathscr{A}_1$. By the above, $\Gru{Z}{m}{n} = G_{mn}(W)$ is rcf. It is easy to see that we can then build an rcf corepresentation $U$ of $\mathscr{A}$ over $Z\oplus W'$ by putting \[\Gr{U}{k}{l}{m}{n}(1\otimes z) = (\Delta_{kl}^{\op}\otimes \id)(z),\qquad z\in \Gru{Z}{m}{n}, \quad m,n\in I_2,k,l\in I,\] and similarly for $W'$. It is then easy to see that, when $V$ is an rcf corepresentation, $V$ is equivariantly isomorphic to $Z\oplus W'$ for $W$ the restriction of $V$, while any $W$ is isomorphic to the restriction of $Z\oplus W'$.  This proves essential surjectivity and faithfulness.
\end{proof}

\begin{Exa} If $\mathscr{G}_1$ and $\mathscr{G}_2$ are Morita equivalent compact quantum groups, the total partial compact quantum group is the co-groupoid constructed by Bichon []. 
\end{Exa}

\begin{Exa}  Let $\G$ be a compact quantum group with ergodic action on a unital C$^*$-algebra $C(\mathbb{X})$. Consider the module C$^*$-category $\CatD$ of finitely generaetd $\G$-equivariant Hilbert $C(\mathbb{X})$-modules as before. Then $\G$ is Morita equivalent with the canonical partial compact quantum group constructed from $(\CatC,\CatD)$. The off-diagonal part of the associated linking partial compact quantum group was studied in ... We will make a detailed study of the case $\G = SU_q(2)$ in [], particularly for $\X$ a Podle\'{s} sphere, which will lead us to partial compact quantum group versions of the dynamical quantum $SU(2)$-group.
\end{Exa}


\subsection{Weak Morita equivalence}

% Concrete implementation, helps to define the concept in more general analytic context.


\begin{Def} A \emph{linking} partial tensor C$^*$-category with duality consists of a partial tensor C$^*$-category with duality $\CatCC$ with a distinguished partition $I_0 =I_1 \cup I_2$ such that for each $r\in I_1$, there exists $s \in I_{2}$ with $\CatC_{rs}\neq \{0\}$.

The \emph{corners} of $\CatCC$ are the restrictions of $\CatCC$ to $I_1$ and $I_2$.
\end{Def}

The following notion was introduced by M. M\"{u}ger []. % Sort out if definition is really the same + precise setting.

\begin{Def} Two partial semi-simple tensor C$^*$-categories $\CatCC_1$ and $\CatCC_2$ with duality over respective sets $I_1$ and $I_2$ are called \emph{Morita equivalent} if there exists a linking partial semi-simple tensor C$^*$-category $\CatCC$ with duality over the set $I=I_1\sqcup I_2$ whose corners are isomorphic to $\CatCC_1$ and $\CatCC_2$.

We say two partial compact quantum groups $\mathscr{G}_1$ and $\mathscr{G}_2$ are \emph{weakly Morita equivalent} if their corepresentation categories $\Corep_u(\mathscr{G}_i)$ are Morita equivalent. 
\end{Def} 

A particular example of weak Morita equivalence arises as follows.

\begin{Def} A \emph{co-linking partial compact quantum group} consists of a partial compact quantum group $\mathscr{G}$ defined by a Hopf $^*$-algebra $\mathscr{A}$ over an index set $I$, together with a distinguished partition $I = I_1\cup I_2$ such that each $\UnitC{k}{l}=0$ for $k,l$ in distinct sets, and such that for each $k\in I_1$, there exists $l\in I_2$ with $\Gr{A}{k}{l}{k}{l}\neq 0$.  
\end{Def} 

\begin{Rem} For finite compact quantum groupoids, one can easily show that the notion of co-linking partial compact quantum group is dual to the notion of linking partial compact quantum group.\end{Rem}

It is again easy to see that if we restrict all indices of a co-linking partial compact quantum group to one of the distinguished sets, we obtain a partial compact quantum group which we will again call a corner.

\begin{Def} We call two partial compact quantum groups \emph{co-Morita equivalent} if there exists a \emph{co-linking partial compact quantum group} having these partial compact quantum groups as its corners.
\end{Def}

\begin{Lem} Co-Morita equivalence is an equivalence relation. % To stylise
\end{Lem} 
\begin{proof} Symmetry is clear. Co-Morita equivalence of $\mathscr{A}$ with itself follows by considering as co-linking quantum groupoid the product of $\mathscr{A}$ with the partial compact quantum groupoid $M_2(\C)$, where $\Delta(e_{ij}) = e_{ij}\otimes e_{ij}$. 

Let us show transitivity. Suppose $\mathscr{G}$ defines a co-linking quantum groupoid between $\mathscr{G}_1$ and $\mathscr{G}_2$, and $\mathscr{H}$ a co-linking quantum groupoid between $\mathscr{G}_2$ and $\mathscr{G}_3$. We can write the total algebra of the partial Hopf $^*$-algebra $\mathscr{A}$ of $\mathscr{G}$ in the form $\begin{pmatrix} A_{11} & A_{12} \\ A_{21} & A_{22}\end{pmatrix}$, and the total algebra of the partial Hopf $^*$-algebra $\mathscr{B}$ of $\mathscr{H}$ as $\begin{pmatrix} A_{22} & A_{23} \\ A_{32} & A_{33}\end{pmatrix}$, where the $A_{ii}$ are the total algebras associated to the $\mathscr{G}_i$. 

Let us show that $A_{12}A_{21} = A_{11}$. 

Similarly,  $A_{21}A_{12} = A_{22}$, and the same statements hold for $\mathscr{B}$. 

Define now $A_{13} = A_{12}\underset{A_{22}}{\otimes} A_{23}$ and $A_{31} = A_{32}\underset{A_{22}}{\otimes} A_{21}$. Then we can make a partial $^*$-algebra $\mathscr{C}$ whose total algebra is $\begin{pmatrix} A_{11} & A_{13}\\ A_{31}& A_{33}\end{pmatrix}$, where for example the product of $A_{13}$ and $A_{31}$ is given by $(x\otimes y)(z\otimes w) =x(yz)w$, while $(x\otimes y)^* = y^*\otimes x^*$. We can make $A_{13}$ into a coalgebra by putting $\Delta(x\otimes y) = (x_{(1)}\otimes y_{(1)})\otimes (x_{(2)}\otimes y_{(2)})$, and similarly for $A_{31}$. Together with the coproducts on $A_{11}$ and $A_{33}$, this then forms a regular Hopf $^*$-algebra. It is easily checked that with $\phi$ the invariant integral on $A_{11}$ and $A_{33}$ and zero $A_{13}$ and $A_{31}$, this is a regular Hopf $^*$-algebra with invariant integral. It is obviously defines a co-linking quantum groupoid between $\mathscr{G}_{11}$ and $\mathscr{G}_{33}$. 
\end{proof} 

%This is indeed generalisation of Morita equivalence. 

\begin{Prop} Assume that two partial compact quantum groups $\mathscr{G}_1$ and $\mathscr{G}_2$ are co-Morita equivalent. Then they are weakly Morita equivalent.
\end{Prop} 
\begin{proof} % Stylise
Consider the corepresentation category $\CatCC$ of a co-linking partial compact quantum group $\mathscr{A}$ over $I = I_1\cup I_2$. Fix $r\in I_0$, and choose a corresponding partition $I_0 = J_1\cup J_2$ with $I_i \twoheadrightarrow J_i$ the associated partition. We want to show that $\Corep(\mathscr{A})_{rs} \neq \{0\}$ for $r\in J_1$ and $s\in J_2$. But  the collection of all $\Gr{A}{k}{l}{m}{n}$ with $k'=m'=r$ and $l=n=s'$ form a partial co-algebra under the $\Delta_{pq}$ with $p'=r$ and $q'=s$.  By assumption, this collection is not trivial. Hence any irreducible regular corepresentation in this class forms a non-trivial element of $\Corep(\mathscr{A})_{rs}$.
\end{proof}

%Notion of co-Morita equivalence. Weak Morita equivalence = Morita equivalence + co-Morita equivalence. 

\begin{Prop} Let $\CatCC$ be a linking partial tensor C$^*$-category with duals. Then the associated canonical partial compact quantum group is a co-linking partial compact quantum group. 
\end{Prop} 

\begin{proof}

\end{proof} 

\begin{Theorem} Two partial compact quantum groups $\mathscr{G}_1$ and $\mathscr{G}_2$ are weakly Morita equivalent if and only if they are connected by a string of Morita and co-Morita equivalences. 
\end{Theorem}

\begin{proof} The direction back has already been shown. Conversely, assume $\mathscr{G}_1$ and $\mathscr{G}_2$ are weakly Morita equivalent. Then $\mathscr{G}_i$ is weakly equivalent with the canonical partial compact quantum group associated to its corepresentation categories. But we have shown that then these canonical partial compact quantum groups are co-Morita equivalent. 
\end{proof} 

\begin{Rem} Note that it is essential that we allow the string of equivalences to pass through partial compact quantum groups, even if we start out with (genuine) compact quantum groups.\end{Rem}

% Gives concrete implementation of weak Morita equivalence.





%%% Local Variables: 
%%% mode: latex
%%% TeX-master: "dyn-suq-main"
%%% End: 

