% Remarks:
% Podles sphere can be defined as an algebra in $\Rep(SU_q(2))$. Hence, it should make sense as an algebra under the forgetful functor to SU_q(2)-dynamical. By duality for coideals, the same should hold for passage to SU_q(1,1) in fact...
%
% link with article `Racah - Wigner quantum 6j Symbols, Ocneanu Cells for AN diagrams and quantum groupoids' by Coquereaux?
% Ref to Schauenburg that face algebras are weak Hopf algebras
% Thanks to Makoto, Piotr (?), Leonid
% Cf. Enocks quantum groupoids of compact type
% Donin, J.(IL-BILN); Mudrov, A.(IL-BILN), Quantum groupoids and dynamical categories. 
% Leonid, cf. talk www.fields.utoronto.ca/programs/scientific/13-14/.../Vainerman.pdf
% Stokman vertex irf babelon cocycle twist
% Level 2 Hecke algebras Brundan
% Concerning locally unital algebras ask J. Vercruysse on recent work 
% Say construction groupoids from forgetful functors on temperley-Lieb already in Enock-Ostrik remark
% Say partial compact quantum groups are proper locally compact quantum groupoids with a discrete base set. Can this be made precise? Is e.g. every measured quantum groupoid which has commutative discrete base and which is `proper' (finiteness integrals on components) of this type? Also compare with properness Timmermann.
% Horizontally categorified algebra is algebroid, but that terminology would be too overused if applied to quantum groupoids
% Mention generalization of Neshveyev
% Include ref to Joyal-Street
% Uniformize enumerates and itemizes
% Reference to Larson paper on cosemisimplicity
% Implement sufficiently many references to Hayashi
% Notion of multiplier fusion category (interpolates between fusion and multifusion category)
% benefit partial approach: notion of partial coalgebra possble
% Change rcf to rcfd at appropriate places
% Van Daele and Khang reference?
% More direct references to Gaby's paper on comodules? 
% look up G. D. Abrams, Morita equivalence for rings with local units for section on co-Morita 
% Caenepeel paper on Galois theory weak Hopf algebras.

\documentclass[11pt]{article}

\usepackage{hyperref}
\usepackage{fixme}
\usepackage{mathrsfs}

\usepackage[a4paper]{geometry}
\usepackage{amssymb, amsthm, amsfonts, amsxtra, amsmath}
\usepackage{latexsym}
\usepackage{mathabx}
\usepackage{enumitem}
\usepackage[all]{xy}
\usepackage{graphics}
\usepackage{pdfpages}
\usepackage{epic}
\usepackage{fouridx}
\usepackage{parskip} % paragraphs have no indents and vertical spacings inbetween
\makeatletter % need this to avoid the conflict between amsthm and parskip
\def\thm@space@setup{%
  \thm@preskip=\parskip \thm@postskip=0pt
}
\makeatother

\DeclareMathOperator{\adj}{\mathrm{adj}}
\DeclareMathOperator{\can}{\mathrm{can}}
\DeclareMathOperator{\Char}{\mathrm{Char}}
\DeclareMathOperator{\dyn}{\mathrm{dyn}}
\DeclareMathOperator{\ext}{\mathrm{e}}
\DeclareMathOperator{\End}{\mathrm{End}}
\DeclareMathOperator{\fin}{\mathrm{f}}
\DeclareMathOperator{\hol}{\mathrm{hol}}
\DeclareMathOperator{\id}{id}
\DeclareMathOperator{\img}{img}
\DeclareMathOperator{\Ind}{\mathrm{Ind}}
\DeclareMathOperator{\Hom}{Hom}
\DeclareMathOperator{\Ker}{\mathrm{Ker}}
\DeclareMathOperator{\Nat}{\mathrm{Nat}}
\DeclareMathOperator{\op}{\mathrm{op}}
\DeclareMathOperator{\Pol}{\mathrm{P}}
\DeclareMathOperator{\Par}{\mathrm{Par}}
\DeclareMathOperator{\Ran}{\mathrm{Ran}}
\DeclareMathOperator{\rcf}{\mathrm{rcf}}
\DeclareMathOperator{\rd}{\mathrm{d}}
\DeclareMathOperator{\reg}{\mathrm{reg}}
\DeclareMathOperator{\sgn}{\mathrm{sgn}}
\DeclareMathOperator{\Span}{\mathrm{span}}
\DeclareMathOperator{\Spec}{\mathrm{Spec}}
\DeclareMathOperator{\tr}{\mathrm{tr}}
\DeclareMathOperator{\Zz}{\mathrm{Z}}


\newcommand{\dual}[1]{#1^{\vee}}
\newcommand{\predual}[1]{{^{\vee}\!#1}}
\newcommand{\co}{\mathrm{co}}
\newcommand{\Corep}{\mathrm{Corep}}
\newcommand{\Corepf}{\mathrm{Corep}^{f}}
\newcommand{\sff}{\textrm{s.f.~}}
\newcommand{\sfs}{\mathrm{sfs}}
\newcommand{\sfd}{\mathrm{sfd}}

\newcommand{\Circt}{{\mathop{\ooalign{$\ovoid$\cr\hidewidth\raise-.05ex\hbox{$\scriptstyle\mathsf T\mkern3.5mu$}\cr}}}} % Woronowicz style tensor product, USUAL SIZE
\newcommand{\Circtv}[1]{\underset{#1}{\mathop{\ooalign{$\ovoid$\cr\hidewidth\raise-.05ex\hbox{$\scriptstyle\mathsf T\mkern3.5mu$}\cr}}}} % Woronowicz style tensor product, USUAL SIZE
\newcommand{\smCirct}{\mathop{\ooalign{$\scriptstyle\ovoid$\cr\hidewidth\raise-.05ex\hbox{$\scriptscriptstyle\mathsf T\mkern2.8mu$}\cr}}}  % Woronowicz style tensor product, SCRIPT SIZE

\newcommand{\nc}{\R}
\newcommand{\g}{\mathfrak{g}}
\newcommand{\h}{\mathfrak{h}}

\newcommand{\kk}{\mathfrak{k}}
\newcommand{\ttt}{\mathfrak{t}}
\newcommand{\p}{\mathfrak{p}}
\newcommand{\n}{\mathfrak{n}}
\newcommand{\llll}{\mathfrak{l}}
\newcommand{\uu}{\mathfrak{u}}
\newcommand{\bb}{\mathfrak{b}}
\newcommand{\q}{\mathfrak{q}}
\newcommand{\su}{\mathfrak{su}}
\newcommand{\ssl}{\mathfrak{sl}}
\newcommand{\SSL}{\mathrm{SL}}
\newcommand{\so}{\mathfrak{so}}
\newcommand{\spp}{\mathfrak{sp}}
\newcommand{\G}{\mathbb{G}}
\newcommand{\e}{\mathfrak{e}}
\newcommand{\s}{\mathfrak{s}}
\newcommand{\C}{\mathbb{C}}
\newcommand{\R}{\mathbb{R}}
\newcommand{\Z}{\mathbb{Z}}
\newcommand{\N}{\mathbb{N}}
\newcommand{\X}{\mathbb{X}}
\newcommand{\Y}{\mathbb{Y}}
\newcommand{\Ss}{\mathbb{S}}
\newcommand{\ZZ}{\mathscr{Z}}
\newcommand{\ad}{\mathrm{ad}}
\newcommand{\Hsp}{\mathcal{H}}
\newcommand{\qn}[2]{\lbrack #1 \rbrack_{#2}}
\newcommand{\fqn}[2]{\lbrack #1 \rbrack_{#2}!}
\newcommand{\bqn}[3]{\left\lbrack \begin{array}{c} \!#1\! \\ \!#2\! \end{array}\right\rbrack_{#3}}
\newcommand{\Tr}{\mathrm{Tr}}
\newcommand{\RR}{\mathcal{R}}
\newcommand{\res}{\mathrm{res}}
\newcommand{\cop}{\mathrm{cop}}
\newcommand{\opp}{\mathrm{op}}
\newcommand{\coop}{\mathrm{coop}}
\newcommand{\Rm}{\mathcal{R}}
\newcommand{\wt}{\mathrm{wt}}
\newcommand{\Ad}{\mathrm{Ad}}
\newcommand{\CatC}{\mathcal{C}}
\newcommand{\CatD}{\mathcal{D}}
\newcommand{\CatCC}{\mathscr{C}}
\newcommand{\CatDD}{\mathscr{D}}
\newcommand{\Corr}{\mathrm{Corr}}

\newcommand{\Vectf}{\mathrm{Vect}^{f}}
\newcommand{\Vecti}{\mathrm{Vect}^{I^{2}}}
\newcommand{\Vectif}{\mathrm{Vect}_{f}^{I^{2}}}
\newcommand{\Vectrcf}{\mathrm{Vect}_{rcfd}^{I^{2}}}
\newcommand{\Hilb}{\mathrm{Hilb}}
\newcommand{\Hilbf}{\mathrm{Hilb}_{\mathrm{f}}}
\newcommand{\Hilbi}{\mathrm{Hilb}^{I^{2}}}
\newcommand{\Hilbif}{\mathrm{Hilb}^{I^{2}}_{\mathrm{f}}}
\newcommand{\Hilbrcf}{\mathrm{Hilb}^{I^2}_{\mathrm{rcfd}}}

\newcommand{\Star}[2]{{}_{#1}\!*_{#2}}
\newcommand{\vot}{\bar{\otimes}}
\newcommand{\A}{\mathcal{B}}
\newcommand{\Aa}{\mathscr{B}}
\newcommand{\Mor}{\mathrm{Mor}}
\newcommand{\alg}{\mathrm{alg}}
\newcommand{\Gg}{\mathscr{G}}
\newcommand{\ev}{\mathrm{ev}}
\newcommand{\coev}{\mathrm{coev}}
\newcommand{\Rtimes}{\underset{\R}{\times}}
\newcommand{\Rb}{\R^{\bullet}}
\newcommand{\vtimes}{\bar{\otimes}}
\newcommand{\Rr}{\mathscr{R}}
\newcommand{\Tt}{\mathscr{T}}
\newcommand{\Fun}{\mathrm{Fun}}
\newcommand{\Ff}{\Fun_{\fin}}
%\newcommand{\fin}{\mathrm{fin}}
%\newcommand{\iitimes}{\underset{I}{\otimes}}
\newcommand{\itimes}{\underset{I}{\otimes}}
\newcommand{\osum}[1]{\underset{#1}{\sum}^{\oplus}}
\newcommand{\osumc}[1]{\underset{#1}{\sum}^{\bar{\oplus}}}
\newcommand{\oplusc}{\bar{\oplus}}
\newcommand{\wDelta}{\widetilde{\Delta}}
\newcommand{\f}{\mathrm{fin}}
%\newcommand{\Hilb}{\mathrm{Hilb}}
\newcommand{\Rho}{\mathrm{P}}
\newcommand{\Rep}{\mathrm{Rep}}
\newcommand{\DA}{\mathcal{A}}
%\newcommand{\Circt}{\mathop{\ooalign{$\ovoid$\cr\hidewidth\raise-.05ex\hbox{$\scriptstyle\mathsf T\mkern3.5mu$}\cr}}} % Woronowicz style tensor product, USUAL SIZE
\newcommand{\even}{\mathrm{even}}
\newcommand{\odd}{\mathrm{odd}}
\newcommand{\fd}{\mathrm{fd}}
\newcommand{\Forget}{F}
\newcommand{\Vect}{\mathrm{Vect}}


\newcommand{\GrHA}[3]{#1{\begin{pmatrix} #2,  #3\end{pmatrix}}}% Horizontal grading ordinary style, with argument
\newcommand{\Grs}[3]{#1{\begin{pmatrix} #2,  #3\end{pmatrix}}}

\newcommand{\GrDA}[3]{{}_{\;#2}#1_{#3}} % Horizontal grading bottom style, with argument
%\newcommand{\Grd}[3]{\;{}_{\;#2}#1_{#3}}

\newcommand{\GrVA}[3]{#1{\tiny {\begin{pmatrix} #2\\#3\end{pmatrix}}}} % Vertical grading ordinary style, with argument
\newcommand{\Grt}[3]{#1{\tiny {\begin{pmatrix} #2\\#3\end{pmatrix}}}} 

\newcommand{\GrRA}[3]{#1^{#2}_{#3}} % Vertical grading right style, with argument

\newcommand{\Unit}{\mathbf{1}}
\newcommand{\UnitC}[2]{\Grt{\mathbf{1}}{#1}{#2}} 
\newcommand{\Grru}[2]{{\tiny \begin{pmatrix} #1 \\ #2\end{pmatrix}}}

\newcommand{\eGr}[5]{#1{{\tiny \begin{pmatrix} #2 \quad #3 \\ #4 \quad #5\end{pmatrix}}}}

\newcommand{\pma}[4]{\begin{pmatrix} #1 \quad #2 \\ #3 \quad #4\end{pmatrix}}
\newcommand{\pmat}[4]{{\tiny \begin{pmatrix} #1 \quad #2 \\ #3 \quad #4\end{pmatrix}}}

\newcommand{\UT}[2]{#1{\tiny #2 }}
\newcommand{\Gr}[5]{\fourIdx{#2}{#4}{#3}{#5}{#1}}%TODO: better typesetting
\newcommand{\Grl}[3]{\Gr{#1}{#2}{}{#3}{}}%TODO: better typesetting
\newcommand{\Gru}[3]{\Gr{#1}{}{}{#2}{#3}}
\newcommand{\Grd}[3]{\Gr{#1}{#2}{#3}{}{}}
% \newcommand{\Gr}[5]{\;{}^{\;#2}_{#4}#1_{#5}^{#3}}%TODO: better typesetting
% %\newcommand{\Gr}[5]{\UT{#1}{\begin{pmatrix} #2\quad #3 \\ #4 \quad #5\end{pmatrix}}}
% %\newcommand{\Gr}[5]{\UT{#1}{\begin{pmatrix} \, #2\;\\ #3 \qquad #4 \\ \,#5\;\end{pmatrix}}}
% \newcommand{\Grl}[3]{\;{}^{\;#2}_{#3}#1}%TODO: better typesetting
% \newcommand{\Gru}[3]{{}^{\;#2}#1^{#3}}
% \newcommand{\Grd}[3]{{}_{\;#2}#1_{#3}}
\newcommand{\gr}[5]{\;{}^{\;#2}_{#4}#1_{#5}^{#3}}%TODO: better typesetting
\newcommand{\eGrr}[3]{#1_{{\tiny \left(#2, #3\right)}}}
\newcommand{\eGrt}[4]{#1{{\tiny \begin{pmatrix} #2 \\ #3 \\ #4 \end{pmatrix}}}}
\newcommand{\Grr}[4]{\begin{pmatrix}#1 \quad #2\\#3&#4\end{pmatrix}}

\newcommand{\Grss}[3]{\UT{#1}{\begin{pmatrix} #2 \; #3\end{pmatrix}}}
\newcommand{\Grb}[7]{\UT{#1}{\begin{pmatrix} #2\quad #3 \\ #4 \quad #5\\ #6 \quad #7\end{pmatrix}}}
\newcommand{\un}[2]{e{{\tiny \begin{pmatrix}#1\\ #2\end{pmatrix}}}}
\newcommand{\unn}[3]{e{{\tiny \begin{pmatrix}#1\\ #2\\#3\end{pmatrix}}}}

\newcommand{\wmult}{\cdot}
\newcommand{\bmult}{*}
\newcommand{\wmate}{\rightarrow}% Change this to source/target notation l(eft) r(ight)
\newcommand{\bmate}{\downarrow}% Change this to source/target notation u(p) d(own)

\newcommand{\aste}[1]{\underset{#1}{\ast}}

\newcommand{\Vv}{\mathcal{V}}

\newcommand{\dT}{\dot T}

\newtheorem{Theorem}{Theorem}[section]
\newtheorem{Lem}[Theorem]{Lemma}
\newtheorem{Prop}[Theorem]{Proposition}
\newtheorem{Cor}[Theorem]{Corollary}

\theoremstyle{definition}
\newtheorem{Def}[Theorem]{Definition}
\newtheorem{Rem}[Theorem]{Remark}
\newtheorem{Exa}[Theorem]{Example}
\newtheorem{Not}[Theorem]{Notation}
\newtheorem{Que}[Theorem]{Question}
\newtheorem{Con}[Theorem]{Conjecture}

%%%%%%%%%%%%%%%%%%%
% Further notation for Section 1
\newcommand{\phic}[2]{\Grt{\phi}{#1}{#2}}

%%%%%%%%%%%%%%%%%%%
% Notation for Section 4
\newcommand{\LGtwo}{L^{2}(\mathscr{G})}
\newcommand{\LGinf}{L^{\infty}(\mathscr{G})}
\newcommand{\CrG}{C^{r}_{0}(\mathscr{G})}
\newcommand{\CuG}{C^{u}_{0}(\mathscr{G})}
\newcommand{\vnDelta}{\overline{\Delta}}
\newcommand{\vnE}{\overline{E}}
\newcommand{\astrl}{\underset{l^{\infty}(I)}{_{\rho}\ast_{\lambda}}}
\newcommand{\otimesrl}{\underset{\nu{}}{_{\rho}\otimes_{\lambda}}}
\newcommand{\vnphi}{\overline{\phi}}
\newcommand{\vnphic}[2]{\Grt{\vnphi}{#1}{#2}}
\newcommand{\vnR}{\overline{R}}
\newcommand{\vntau}{\overline{\tau}}

\date{}


\numberwithin{equation}{section}

\begin{document}
\title{Partial compact quantum groups}

\author{Kenny De Commer\thanks{Department of Mathematics, Vrije Universiteit Brussel, VUB, B-1050 Brussels, Belgium, email: {\tt kenny.de.commer@vub.ac.be}}
\and Thomas Timmermann\thanks{University of M\"{u}nster}}

\maketitle

% Terminology `compact' might have to be adapted if we work with infinitely many objects.
\begin{abstract}
\noindent Compact quantum groups of face type, as introduced by Hayashi, form a class of quantum groupoids with a classical, finite set of objects. Using the notions of weak multiplier bialgebra and weak multiplier Hopf algebra (resp.~ due to B{\"o}hm--G\'{o}mez-Torrecillas--L\'{o}pez-Centella and Van Daele--Wang), %Correct hyphenation?
 we generalize Hayashi's definition to allow for an infinite set of objects, and call the resulting objects partial compact quantum groups. We prove a Tannaka-Krein-Woronowicz reconstruction result for such partial compact quantum groups using the notion of partial tensor C$^*$-category. We also study their connection to the Lesieur-Enock theory of measured quantum groupoids.
 % As an application, we show how any quantum homogeneous space of an ordinary compact quantum group leads to a partial compact quantum group. 
%In a follow-up article, this construction is applied to the non-standard Podle\'{s} spheres. We obtain in this way partial compact quantum groups which are operator algebraic versions of the dynamical quantum $SU(2)$-group as studied by Etingof-Varchenko and Koelink-Rosengren.% References ok?
\end{abstract}


%\emph{Keywords}:

%AMS 2010 \emph{Mathematics subject classification}:


%17B37: Quantum groups, quantized enveloping algebras
%20G42: quantized function algebras
%46L65: Functional analysis, deformations, quantizations
%81R50: Quantum groups and related algebraic methods
%16T05: Hopf algebras and their applications
%16T10: Bialgebras
%16T15: Coalgebras and comodules; corings
%46L08  $C^*$-modules
%58B32: Geometry of quantum groups


\tableofcontents

\section{Introduction}

In the following, we follow the physicist's convention that inner products on Hilbert spaces are anti-linear in their \emph{first} argument. Whenever convenient, we will also identify $\mathscr{H}$ with $B(\C,\mathscr{H})$, so that we can write $\langle \xi,\eta\rangle = \xi^*\eta$.

\section{Partial compact quantum groups}

We generalize Hayashi's definition of a compact quantum group of face type \cite{Hay1} to the case where the commutative base algebra is no longer finite-dimensional. We will present two approaches, based on \emph{partial bialgebras} and \emph{weak multiplier bialgebras} \cite{Boh1,VDW1}. The first approach is piecewise and concrete, but requires some bookkeeping. The second approach is global but more abstract. As we will see from the general theory and the concrete examples, both approaches have their intrinsic value.

\subsection{Partial algebras}

Let $I$ be a set. We consider $I^2=I\times I$ as the pair groupoid with $\wmult$ denoting composition. That is, an element $K=(k,l)\in I^2$ has source $K_l = k$ and target $K_r=l$, and if $K=(k,l)$ and $L=(l,m)$ we write $K\wmult L = (k,m)$. 

\begin{Def} A \emph{partial algebra} $\mathscr{A}=(\mathscr{A},M)$ (over $\C$) is a set $I$ (the \emph{object} set) together with 
\begin{itemize}
\item[$\bullet$] for each $K=(k,l)\in I^2$ a vector space $A(K) = \Grs{A}{k}{l}=\!\!\GrDA{A}{k}{l}$ (possibly the zero vector space),
\item[$\bullet$] for each $K,L$ with $K_r = L_l$ a multiplication map \[M(K,L):A(K) \otimes A(L)\rightarrow A(K\cdot L),\qquad a\otimes b \mapsto ab\]  and 
\item[$\bullet$] elements $\Unit(k) = \Unit_k \in \Grs{A}{k}{k}$ (the units), % or the local units?
\end{itemize}
such that the obvious associativity and unit conditions are satisfied. 

By \emph{$I$-partial algebra} will be meant a partial algebra with object set $I$.
\end{Def}

% Also known as $\C$-algebroid

\begin{Rem}
\begin{enumerate}\item It will be important to allow the local units $\Unit_k$ to be zero.
\item A partial algebra is by definition the same as a small $\C$-linear category. However, we do not emphasize this viewpoint, as the natural notion of morphism for partial algebras will be \emph{contravariant} on objects, see Definition \ref{DefMor}.% Is this correct, or are local units = 0 not allowed? 
\end{enumerate}
\end{Rem}

Let $\mathscr{A}$ be an $I$-partial algebra. We define $A(K\wmult L)$ to be $\{0\}$ when $K\wmult L$ is ill-defined, i.e. $K_r\neq L_l$. We then let $\Grs{M}{K}{L}$ be the zero map.

\begin{Def} The \emph{total algebra} $A$ of an $I$-partial algebra $\mathscr{A}$ is the vector space \[A = \oplus_{K\in I^2} A(K)\] endowed with the unique multiplication whose restriction to $A(K)\otimes A(L)$ concides with $M(K,L)$. 
\end{Def} 
% Should a reference to the notion of locally unital algebra be given? I had a reference to Quillen, but no precise article, and I don't seem to find the link anymore...
Clearly $A$ is an associative algebra. If $A_0$ is infinite-dimensional it will not possess a unit, but it is a \emph{locally unital algebra} as there exist mutually orthogonal idempotents $\mathbf{1}_k$ with $A = \osum{k,l} \mathbf{1}_kA\mathbf{1}_l$. An element $a\in A$ can be interpreted as a function assigning to each element $(k,l)\in I^2$ an element $a_{kl}\in A(k,l)$, namely the $(k,l)$-th component of $a$. This identifies $A$ with finite support $I$-indexed matrices whose $(k,l)$-th entry lies in $A(k,l)$, equipped with the natural matrix multiplication. 

\begin{Rem}\label{RemGrad} When $\mathscr{A}$ is an $I$-partial algebra with total algebra $A$, then $A\otimes A$ can be naturally identified with the total algebra of an $I\times I$-partial algebra $\mathscr{A}\otimes \mathscr{A}$, where \[(A\otimes A)((k,k'),(l,l')) = A(k,l)\otimes A(k',l')\] with the obvious tensor product multiplications and the $\Unit_{k,k'} = \Unit_k\otimes \Unit_{k'}$ as units. 
\end{Rem}

Working with non-unital algebras necessitates the use of their \emph{multiplier algebra}. Let us first recall some general notions concerning non-unital algebras from \cite{Dau1,VDae1}.

\begin{Def} Let $A$ be an algebra over $\C$, not necessarily with unit. We call $A$ \emph{non-degenerate} if $A$ is faithfully represented on itself by left and right multiplication. It is called \emph{idempotent} if $A^2 = A$. 
\end{Def}

\begin{Def} Let $A$ be an algebra. A \emph{multiplier} $m$ for $A$ consists of a couple of maps \begin{eqnarray*} L_m:A\rightarrow A,\quad a\mapsto ma\\ R_m:A\rightarrow A,\quad a\mapsto am\end{eqnarray*} such that $(am)b = a(mb)$ for all $a,b\in A$. 

The set of all multipliers forms an algebra under composition for the $L$-maps and anti-composition for the $R$-maps. It is called the \emph{multiplier algebra} of $A$, and is denoted $M(A)$.
\end{Def}

One has a natural homomorphism $A\rightarrow M(A)$. When $A$ is non-degenerate,  this homomorphism is injective, and we can then identify $A$ as a subalgebra of the (unital) algebra $M(A)$. We then also have inclusions \[A\otimes A\subseteq M(A)\otimes M(A)\subseteq M(A\otimes A).\]

\begin{Exa}\label{ExaMult} \begin{enumerate}
\item Let $I$ be a set, and $\Fun_{\fin}(I)$ the algebra of all finite support functions on $I$. Then $M(\Fun_{\fin}(I)) = \Fun(I)$, the algebra of all functions on $I$. 
\item Let $A$ be the total algebra of an $I$-partial algebra $\mathscr{A}$. As $A$ has local units, it is non-degenerate and idempotent. Then one can identify $M(A)$ with \[M(A) = \left(\prod_l \oplus_k A(k,l)\right) \bigcap \left(\prod_k\oplus_l A(k,l)\right) \subseteq \prod_{k,l} A(k,l),\] i.e. with the space of functions \[m:I^2\rightarrow A,\quad m_{kl}\in A(k,l)\] which have finite support in either one of the variables when the other variable has been fixed. The multiplication is given by the formula \[(mn)_{kl} = \sum_p m_{kp}n_{pl}.\]
\item Let $m_i$ be any collection of multipliers of $A$, and assume that for each $a\in A$, $m_ia =0$ for almost all $i$, and similarly $am_i=0$ for almost all $i$. Then one can define a multiplier $\sum_i m_i$ in the obvious way by termwise multiplication. One says that the sum $\sum_i m_i$ converges in the \emph{strict} topology. 
\end{enumerate}
\end{Exa}

% Better location?
\begin{Rem} We will call any general assignment $(k,l)\rightarrow m_{kl}$ into a set with a distinguished zero element \emph{row-and column-finite} (rcf) if the assignment has finite support in either one of the variables when the other variable has been fixed. % More succinct terminology would be banded, but that seems to imply a uniform band
\end{Rem} 

% Added part on morphisms anyway because it will appear in the generalisation to partial tensor categories
Let us comment on the notion of morphism for partial algebras. We first introduce the piecewise definition.

\begin{Def}\label{DefMor} Let $\mathscr{A}$ and $\mathscr{B}$ be respectively $I$ and $J$-partial algebras. Let \[\phi: I \ni k \mapsto J_k \subseteq J\] with the $J_k$ disjoint. A \emph{homomorphism} (based on $\phi$) from $\mathscr{A}$ to $\mathscr{B}$ consists of linear maps \[\GrDA{f}{r}{s}: A(k\;l)\rightarrow B(r\;s),\quad a\mapsto \GrDA{f(a)}{r}{s}\] for all $r\in J_k, s\in J_l$, satisfying 
\begin{itemize}
\item[$\bullet$] (Unitality) $\GrDA{f(\Unit_{k})}{r}{s} = \delta_{rs}\Unit_r$ for all $r,s\in J_k$.
\item[$\bullet$] (Local finiteness) For each $k,l\in I$, $a\in A(k\;l)$ and $r\in J_k$ (resp.~ $s\in J_l$), the assigment $(r,s)\rightarrow \GrDA{f(a)}{r}{s}$ on $J_k\times J_l$ is rcf. 
\item[$\bullet$] (Multiplicativity) For all $k,l,m\in I$, all $r\in J_k$ and all $t\in J_m$, and all $a\in A(k\;l)$ and $b\in A(l\;m)$, one has \[\GrDA{f(ab)}{r}{t} = \sum_{s\in J_l} \GrDA{f(a)}{r}{s}\GrDA{f(b)}{s}{t}.\]
\end{itemize} 
The homomorphism is called \emph{unital} if $J=\bigcup \{J_k\mid k\in I\}$. % Can't call this a partition since $J_k$ can be empty 
\end{Def}
\begin{Rem}
\begin{enumerate}
\item
Note that the multiplicativity condition makes sense because of the local finiteness condition.
\item
If $J = \bigcup\{J_k\}$, we can interpret $\phi$ as a map \[J\rightarrow I,\quad r\mapsto k \iff r\in J_k.\] In the more general case, we obtain a function $J\rightarrow I^*$, where $I^*$ is $I$ with an extra point `at infinity' added.
\end{enumerate}
\end{Rem}

The following lemma provides the global viewpoint concerning homomorphisms. 

\begin{Lem} Let $\mathscr{A}$ and $\mathscr{B}$ be respective $I$- and $J$-partial algebras, and fix an assigment $\phi: k\mapsto J_k$. Then there is a one-to-one correspondence between homomorphisms $\mathscr{A}\rightarrow \mathscr{B}$ based on $\phi$ and homomorphisms $f:A\rightarrow M(B)$ with $f(\Unit_k) = \sum_{r\in J_k} \Unit_r$. 
\end{Lem} 
\begin{proof}
Straightforward, using the characterisation of the multiplier algebra provided in Remark \ref{ExaMult}.2.
\end{proof}

\begin{Rem}% Necessary?
Note that the assigment $k\mapsto J_k$ has to be kept as part of the information also in the global description, since local units are allowed to be zero. 
\end{Rem}

% Make reference to this lemma in bialgebra section.

\subsection{Partial coalgebras}

%The notion of partial algebra dualizes. For this we consider again $I^2$ as the pair groupoid, but now with elements considered as column vectors, and with $\bmult$ denoting the (vertical) composition. So $K=\Grt{}{k}{l}$ has source $K_u = k$ and target $K_d = l$, and if $K=\Grt{}{k}{l}$ and $L=\Grt{}{l}{m}$ then $K\bmult L = \Grt{}{k}{m}$. 

%There are two natural notions of morphism between partial algebras, functors and co-functors. %Different name? How natural are these?
% We will need the notion of \emph{relator} between partial algebras. When $R\subseteq I\times J$ is a relation, we write, for $k\in I$, $R_k= \{l'\in J\mid (k',l')\in R\}$. We write $D_R = \{k\in I\mid \#R_k\neq0\}$.  
% A \emph{functor} between an $I$-partial algebra $\mathscr{A}$ and $J$-partial algebra $\mathscr{B}$ consists of a map $F:I\rightarrow J$ and linear maps $F_{k,l}:A(k,l)\rightarrow B(F(k),F(l))$ such that the $F_{k,l}$ respect the algebra and unit maps in the obvious ways. 

% \begin{Def} A \emph{relator} between an $I$-partial algebra $\mathscr{A}$ and $J$-partial algebra $\mathscr{B}$ consists of a relation $F\subseteq I\times J$ and maps $F_{k',l'}:A(k,l)\rightarrow B(k',l')$ for $k'\in R_{k},l'\in R_{l}$ such that the following conditions hold.
% \begin{itemize}
% \item For all $k,l\in I$ and $a\in A(k,l)$, the $F_{k',l'}(a)$ are zero for almost all $k'\in R_{k}$ (resp. all $l'\in R_{l}$) when $l'$ (resp. $k'$) is fixed.  
% \item For all $a\in A(k,l)$ and $b\in A(l,m)$, and all $k'\in R_{k}, m'\in R_{m}$, one has \[F_{k',m'}(ab) = \sum_{l',l'\in R_{l}} F_{k',l'}(a)F_{l',m'}(b).\] 
% \item $F_{k',l'}(e_{k}) = \delta_{k',l'}e_{k'}$ for all $k'\in R_{k}$.  
% \end{itemize}
% \end{Def}

The notion of partial algebra dualizes. For this we consider again $I^2$ as the pair groupoid, but now with elements considered as column vectors, and with $\bmult$ denoting the (vertical) composition. So $K=\Grt{}{k}{l}$ has source $K_u = k$ and target $K_d = l$, and if $K=\Grt{}{k}{l}$ and $L=\Grt{}{l}{m}$ then $K\bmult L = \Grt{}{k}{m}$. %We write $K\bmate L$ if $K$ and $L$ are composable. 

% Then we have maps  \[\Grt{\Delta}{K}{L}: A(K\bmult L)\rightarrow A(K)\otimes A(L),\] which we interpret as zero maps when $\neg K\bmate L$. We also interpret $\Grt{\Delta}{K}{L}$ as the zero map on $A(M)$ if $M\neq K\bmult L$. The coassociativity condition can now be written  \[(\id\otimes \Grt{\Delta}{L}{M})\Grt{\Delta}{K}{L\bmult M} = ( \Grt{\Delta}{K}{L}\otimes \id)\Grt{\Delta}{K\bmult L}{M}.\]
% Should provide shortcut for notation where only the new indices appear at $\Delta$ (so disregard the source): write $\Delta(K over L)$ as $Delta_{rs}$ for $r=K_{ld}$ and $s=K_{rd}$. 
%>>>>>>> origin/master
\begin{Def} A \emph{partial coalgebra} $\mathscr{A}=(\mathscr{A},\Delta)$ (over $\C$) consists of a set $I$ (the object set) together with 
\begin{itemize}
\item[$\bullet$] for each $K=\Grru{k}{l}\in I^2$ a vector space $A(K) = \Grt{A}{k}{l}=\!\!\GrRA{A}{k}{l}$,
\item[$\bullet$] for each $K,L$ with $K_d = L_u$ a comultiplication map \[\Grt{\Delta}{K}{L}:A(K*L)\rightarrow A(K)\otimes A(L),\qquad a \mapsto a_{(1)K}\otimes a_{(2)L},\] and 
\item[$\bullet$] counit maps $\epsilon_k:\Grt{A}{k}{k}\rightarrow \C$,
\end{itemize} 
satisfying the obvious coassociativity and counitality conditions.

By \emph{$I$-partial coalgebra} will be meant a partial coalgebra with object set $I$.
\end{Def}

\begin{Not}\label{NotCom} As the index of $\epsilon_k$ is determined by the element to which it is applied, there is no harm in dropping the index $k$ and simply writing $\epsilon$.

Similarly, if $K = \Grt{}{k}{l}$ and $L = \Grt{}{l}{m}$, we abbreviate $\Delta_l = \Grt{\Delta}{K}{L}$, as the other indices are determined by the element to which $\Delta_l$ is applied.
\end{Not}

We also make again the convention that $A(K*L)=\{0\}$ and $\Grt{\Delta}{K}{L}$ the zero map when $K_d \neq L_u$. Similarly $\epsilon$ is seen as the zero functional on $A(K)$ when $K=\Grt{}{k}{l}$ with $k\neq l$. 

\subsection{Partial bialgebras}

% Formally, a finite morphism between an $I$-partial algebra $\mathscr{A}$ and $J$-partial algebra $\mathscr{B}$ equipped with an injection $\phi$ of a subset of $J$ into $I$ consists of maps $F_{K,L}:A(\phi(K),\phi(L))\rightarrow B(K,L)$ for $K,L$ in the domain of $\phi$, such that the map $(K,L)\mapsto F_{K,L}(a)$ is finite in rows and columns when the values of $K,L$ are restricted to one fiber of$ \phi$, and such that then $F_{K,M}(ab) = \sum_L F_{K,L}(a)F_{L,M}(b)$. 
% Should this more general notion be included?
%>>>>>>> origin/master

We can now superpose the notions of partial algebra and partial coalgebra. Let $I$ be a set, and let $M_2(I)$ be the set of 4-tuples of elements of $I$ arranged as 2$\times$2-matrices. We can endow $M_2(I)$ with two compositions, namely $\cdot$ (viewing $M_2(I)$ as a row vector of column vectors) and $*$ (viewing $M_2(I)$ as a column vector of row vectors). When $K\in M_2(I)$, we will write $K = \Grs{}{K_l}{K_r} = \Grt{}{K_u}{K_d} = \eGr{}{K_{lu}}{K_{ru}}{K_{ld}}{K_{rd}}$. One can view $M_2(I)$ as a double groupoid, and in fact as a \emph{vacant} double groupoid in the sense of \cite{AN1}. 

In the following, a vector $(r,s)$ will sometimes be written simply as $r,s$ (without parentheses) or $rs$ in an index. We also follow Notation \ref{NotCom}, but the reader should be aware that the index of $\Delta$ will now be a 1$\times$2 vector in $I^2$ as we will work with partial coalgebras over $I^2$.

\begin{Def}\label{DefPartBiAlg} A \emph{partial bialgebra} $\mathscr{A}=(\mathscr{A},M,\Delta)$ consists of a set $I$ (the \emph{object set}) and a collection of vector spaces $A(K)$ for $K\in M_2(I)$ such that 
\begin{itemize}
\item[$\bullet$] the $\Grs{A}{K_l}{K_r}$ form an $I^2$-partial algebra,
\item[$\bullet$] the $\Grt{A}{K_u}{K_d}$ form an $I^2$-partial coalgebra,
\end{itemize} 
and for which the following compatibility relations are satisfied.
\begin{enumerate}[label=(\alph*)]
\item\label{Propa} (Comultiplication of Units) For all $k,l,l',m\in I$, one has 
\[\Delta_{l,l'}(\UnitC{k}{m}) = \delta_{l,l'} \UnitC{k}{l}\otimes \UnitC{l}{m}.\]  
\item\label{Propb} (Counit of Multiplication) For all $K,L\in M_2(I)$ with $K_r = L_l$ and all $a\in A(K)$ and $b\in A(L)$, \[\epsilon(ab) = \epsilon(a)\epsilon(b).\]% Subtlety is that you already have to know composition to be able to apply this rule.
\item\label{Propc} (Non-degeneracy) For all $k\in I$, $\epsilon(\UnitC{k}{k})=1$. 
\item\label{Propd} (Finiteness) For each $K\in M_2(I)$ and each $a\in A(K)$, the assignment $(r,s)\rightarrow \Delta_{rs}(a)$ is rcf.
\item\label{Prope} (Comultiplication is multiplicative) For all $a\in A(K)$ and $b\in A(L)$ with $K_r= L_l$,  \[\Delta_{rs}(ab) = \sum_t \Delta_{rt}(a)\Delta_{ts}(b).\]
\end{enumerate}
\end{Def}

% To do: introduce notion of connectedness which should assure the trivial corepresentation is irreducible. This should correspond to the dual notion of pureness in [Bohm-Nill-Szlachanyi], i.e. to the left and right base algebras having $\C$ as intersection. This is the same as proving $\UnitC{k}{l}\neq 0$ in the presence of an antipode, since then the relation $k\sim l$ iff $\Unit{k}{l}\neq 0$ gives an equivalence relation, and fixing a non-trivial class $P$ and putting $c_k = \delta_{k\in P}, d_l = \delta_{l\in P}$, we obtain a non-trivial element $\sum_k c_k\lambda_k = \sum_l d_l \rho_l$ in the intersection.

\begin{Rem}\begin{enumerate}
\item By assumption \ref{Propd}, the sum on the right hand side in condition \ref{Prope} is in fact finite and hence well-defined. 
\item Note that the object set of the above $\mathscr{A}$ as a partial bialgebra is $I$, but the object set of its underlying partial algebra (or coalgebra) is $I^2$.
\item Properties \ref{Propa},\ref{Propd} and \ref{Prope} simply say that $\Delta$ is a homomorphism $\mathscr{A}\rightarrow \mathscr{A}\otimes \mathscr{A}$ of partial algebras based over the assignment $I^2\rightarrow \mathscr{P}(I^2\times I^2)$, the power set of $I^2\times I^2$, such that \[(I^2\times I^2)_{{\tiny \begin{pmatrix} k\\m \end{pmatrix}}} = \{\left(\begin{pmatrix} k \\ l \end{pmatrix},\begin{pmatrix} l \\ m \end{pmatrix}\right)\mid l\in I\}.\] 
\end{enumerate}
\end{Rem}

%There is a natural partition of the object set $I$ of a partial bialgebra into \emph{connected components}.

%\begin{Lem} Let $\mathscr{A}$ be a partial bialgebra over the object set $I$. Write $k\sim l$ if $\un{k}{l}\neq 0$. Then $\sim$ is an equivalence relation.
%\end{Lem}
%\begin{proof} Clearly $\un{k}{k}\neq 0$, by applying the counit. If $k\sim l$ and $l \sim m$, then $\Delta_{ll}(\un{k}{m}) = \un{k}{l}\otimes \un{l}{m}$ shows $k\sim m$. Finally, if $k\sim l$, then applying $(\id\otimes \epsilon)$ to $\Delta_{ll}(\un{k}{k}) = \un{k}{l}\otimes \un{l}{k}$ shows $\un{l}{k}\neq 0$. 

%\end{proof}

We relate the notion of partial bialgebra to the recently introduced notion of weak multiplier bialgebra \cite{Boh1}. Let us first introduce the following notation, using the notion introduced in Example \ref{ExaMult}.2.

\begin{Not}
If $\mathscr{A}$ is an $I$-partial bialgebra, we write \[\lambda_k = \sum_l \UnitC{k}{l},\qquad \rho_l = \sum_k\UnitC{k}{l} \qquad \in M(A).\]
\end{Not}

\begin{Rem} From Property \ref{Propc} of Definition \ref{DefPartBiAlg}, it follows that $\lambda_k\neq 0\neq \rho_k$ for any $k\in I$. 
\end{Rem} 

To show that the total algebra of a partial bialgebra becomes a weak multiplier bialgebra, we will need some easy lemmas. 

\begin{Lem} Let $\mathscr{A}$ be an $I$-partial bialgebra. Then for each $a\in A$, there exists a unique multiplier $\Delta(a) \in M(A\otimes A)$ such that \begin{align}\label{EqDel}
    \begin{aligned}
      \Delta_{rs}(a) &= (1\otimes \lambda_r)\Delta(a)(1\otimes
      \lambda_s) \\ &= (\rho_r\otimes 1)\Delta(a)(\rho_s\otimes 1)
    \end{aligned}
\end{align}  for all $r,s\in I$, all $K\in M_2(I)$ and all $a\in A(K)$. 

The resulting map \[\Delta:A\rightarrow M(A\otimes A),\quad a\mapsto \Delta(a)\] is a homomorphism.
\end{Lem} % Maybe write this up on the level of general homs and then refer to there. 
\begin{proof} For $a\in A$ homogeneous, we can define $\Delta(a) = \sum_{rs} \Delta_{rs}(a) \in M(A\otimes A)$, where the sum converges in the strict topology of $A\otimes A$ because of the property \ref{Propd} of Definition \ref{DefPartBiAlg}. This expression clearly satisfies the identities stated in the lemma. In turn, these identities uniquely define $\Delta(a)$ as a multiplier, as they determine the value of $\Delta(a)$ when cut down to the left and right with the local units of $\mathscr{A}\otimes \mathscr{A}$.

We can then extend $\Delta$ to $A$ by linearity. Since, for $a,b$ homogeneous, $\Delta_{rt}(a)\Delta_{t's}(b)=0$ unless $t=t'$, it follows from property \ref{Prope} of Definition \ref{DefPartBiAlg} that $\Delta$ is a homomorphism. 
\end{proof}

We will refer to $\Delta: A\rightarrow M(A\otimes A)$ as the \emph{total comultiplication} of $\mathscr{A}$. We will then also use the suggestive Sweedler notation for this map, \[\Delta(a) = a_{(1)}\otimes a_{(2)}.\] Note for example that \[\Delta(\UnitC{k}{m}) = \sum_{l}\UnitC{k}{l}\otimes \UnitC{l}{m} = \sum_l \lambda_k\rho_l\otimes \lambda_l\rho_m.\]

\begin{Lem} The element $E = \sum_{k,l,m} \UnitC{k}{l}\otimes \UnitC{l}{m}= \sum_l \rho_l\otimes \lambda_l$ is a well-defined idempotent in $A\otimes A$, and satisfies \[\Delta(A)(A\otimes A)=E(A\otimes A),\quad (A\otimes A)\Delta(A)= (A\otimes A)E.\]
\end{Lem} 
\begin{proof} Clearly the sum defining $E$ is strictly convergent, and makes $E$ into an idempotent. It is moreover immediate that $E\Delta(a)=\Delta(a) = \Delta(a)E$ for all $a\in A$. Since \[E(\UnitC{k}{l}\otimes \UnitC{m}{n}) = \Delta(\UnitC{k}{n})(\UnitC{k}{l}\otimes \UnitC{m}{n}) \] by the property \ref{Propa} of Definition \ref{DefPartBiAlg}, and analogously for multiplication with $E$ on the right, the lemma is proven. 
\end{proof} 

By \cite[Proposition A.3]{VDW2}, there is a unique homomorphism $\Delta:M(A)\rightarrow M(A\otimes A)$ extending $\Delta$ and such that $\Delta(1) = E$. Alternatively, if $m\in M(A)$, we can directly define $\Delta(m)$ as the strict limit of the series $\sum_{k,l,r,s} \Delta_{rs}(m_{kl})$. Similarly the maps $\id\otimes \Delta$ and $\Delta\otimes \id$ extend to maps from $M(A\otimes A)$ to $M(A\otimes A\otimes A)$. 

For example, note that
\begin{align} \label{eq:delta-lambda-rho} \Delta(\lambda_{k}) &=
  (\lambda_{k} \otimes 1)\Delta(1), & \Delta(\rho_{m}) &= (1 \otimes \rho_{m})\Delta(1).
\end{align}

The following proposition gathers the properties of $\Delta$, $\epsilon$ and $\Delta(1)$ which guarantee that $(A,\Delta)$ forms a weak multiplier bialgebra in the sense of \cite[Definition 2.1]{Boh1}.

\begin{Prop} Let $\mathscr{A}$ be a partial bialgebra with total algebra $A$, total comultiplication $\Delta$ and counit $\epsilon$. Then the following properties are satisfied.
\begin{enumerate}[label={(\arabic*)}]
\item Coassociativity: $(\Delta\otimes \id)\Delta = (\id\otimes \Delta)\Delta$ (as maps $M(A)\rightarrow M(A^{\otimes 3})$).
\item Counitality: $(\epsilon\otimes \id)(\Delta(a)(1\otimes b)) = ab = (\id\otimes \epsilon)((a\otimes 1)\Delta(b))$ for all $a,b\in A$.
\item Weak Comultiplicativity of Unit: \[(\Delta(1)\otimes 1)(1\otimes \Delta(1)) = (\Delta\otimes \id)\Delta(1) = (\id\otimes \Delta)\Delta(1) = (1\otimes \Delta(1))(\Delta(1)\otimes 1).\]
\item \label{WMC} Weak Multiplicativity of Counit: For all $a,b,c\in A$, one has \[(\epsilon\otimes \id)(\Delta(a)(b\otimes c)) = (\epsilon\otimes \id)((1\otimes a)\Delta(1)(b\otimes c))\] and 
\[(\epsilon\otimes \id)((a\otimes b)\Delta(c)) = (\epsilon\otimes \id)((a\otimes b)\Delta(1)(1\otimes c)).\]
\item Strong multiplier property: For all $a,b\in A$, one has \[\Delta(A)(1\otimes A)\cup (A\otimes 1)\Delta(A)\subseteq  A\otimes A.\] 
\end{enumerate}
\end{Prop}

\begin{proof} Most of these properties follow immediately from the definition of a partial bialgebra. For demonstrational purposes, let us check the first identity of property \ref{WMC}. Let us choose $a\in A(K)$, $b\in A(L)$ and $c\in A(M)$. Then \[\Delta(a)(b\otimes c) = \delta_{K_{ru},L_{lu}}\delta_{M_{lu},L_{ld}} \sum_r \Delta_{r,L_{ld}}(a)(b\otimes c).\]  Applying $(\epsilon\otimes \id)$ to both sides, we obtain by Proposition \ref{Propb} of Definition \ref{DefPartBiAlg} and counitality of $\epsilon$ that \[(\epsilon \otimes \id)(\Delta(a)(b\otimes c)) = \delta_{K_{ru},L_{lu},L_{ld},M_{lu}} \epsilon(b) ac.\] On the other hand, \begin{eqnarray*} (1\otimes a)\Delta(1)(b\otimes c) &=& \sum_{r,s,t} \UnitC{r}{s} b \otimes a\UnitC{s}{t}c \\ &=& \delta_{L_{ld},K_{ru},M_{lu}} b \otimes ac.\end{eqnarray*} Applying $(\epsilon\otimes \id)$, we find \begin{eqnarray*} (\epsilon\otimes \id)( (1\otimes a)\Delta(1)(b\otimes c) ) &=&  \delta_{L_{ld},K_{ru},M_{lu}}\delta_{L_{lu},L_{ld}}\delta_{L_{ru},L_{rd}} \epsilon(b)ac \\ &=&  \delta_{L_{ld},L_{lu},K_{ru},M_{lu}} \epsilon(b)ac,\end{eqnarray*} which agrees with the expression above.
\end{proof} 

\begin{Rem} 
Since also the expressions $\Delta(a)(b\otimes 1)$ and $(1\otimes a)\Delta(b)$ are in $A\otimes A$ for all $a,b\in A$, we see that $(A,\Delta)$ is in fact a \emph{regular} weak multiplier bialgebra \cite[Definition 2.3]{Boh1}.
\end{Rem} 

% Need to give a converse of the theorem, show that any weak multiplier bialgebra with object algebra $F_f(I)$ is of the above form?

Recall from \cite[Section 3]{Boh1} that a regular weak multiplier
bialgebra admits four projections $A\rightarrow M(A)$, given
by \begin{align*} \bar{\Pi}^L(a) = (\epsilon\otimes \id)((a\otimes
  1)\Delta(1)),\quad & \bar{\Pi}^R(a) = (\id\otimes
  \epsilon)(\Delta(1)(1\otimes a)),\\ \Pi^L(a) = (\epsilon\otimes
  \id)(\Delta(1)(a\otimes 1)),\quad& \Pi^R(a) =
  (\id\otimes\epsilon)((1\otimes a)\Delta(1)),\end{align*} where the
right hand side expressions are interpreted as multipliers in the
obvious way. The relation  $\Delta(1)=\sum_{p} \rho_{p} \otimes \lambda_{p}$ and  condition (c) in Definition \ref{DefPartBiAlg} imply
\begin{align*}
  \bar \Pi^{L}(A) &=\mathrm{span}\{\lambda_{p}:p\in I\} =  \Pi^{L}(A), &
  \bar \Pi^{R}(A) &= \mathrm{span}\{\rho_{p}:p\in I\} =\Pi^{R}(A).
\end{align*}
The \emph{base algebra} of $(A,\Delta)$ is therefore just the algebra
$\Fun_{\fin}(I)$ of finite support functions on $I$, and the
comultiplication of $A$ is (left and right) \emph{full} (meaning
roughly that the legs of $\Delta(A)$ span $A$) by \cite[Theorem
3.13]{Boh1}.  

 The maps $\Pi^{L}$ and $\Pi^{R}$ can also
be written in the form
\begin{align} \label{eq:pi} 
    \Pi^L(a) & = \sum_{p}\epsilon(\lambda_{p}a)\lambda_p, & \Pi^R(a) & =    \sum_{p}\epsilon(a \rho_{p}) \rho_p
\end{align}
because $\epsilon(\lambda_{k}\rho_{m} a \lambda_{l}\rho_{n})=0$  if $(k,l)\neq(m,n)$. These relations and  \eqref{EqDel}, \eqref{eq:delta-lambda-rho} imply
\begin{align} \label{eq:pi-l-delta}
  (\Pi^{L} \otimes \id)(\Delta(a)) &= \sum_{p} \lambda_{p}\otimes \lambda_{p}a, &
  (\id \otimes \Pi^{L})(\Delta(a)) &= \sum_{p} \rho_{p}a \otimes \lambda_{p}, & \\ \label{eq:pi-r-delta}
  (\Pi^{R} \otimes \id)(\Delta(a)) &= \sum_{p} \rho_{p} \otimes a\lambda_{p}, &
  (\id \otimes \Pi^{R})(\Delta(a)) &= \sum_{p} a\rho_{p} \otimes \rho_{p}.
\end{align}

Let us now show a converse. If $(A,\Delta)$ is a regular weak multiplier bialgebra, let us write $A^L = \Pi^L(A) = \bar{\Pi}^L(A)\subseteq M(A)$ and $A^R = \Pi^R(A)= \bar{\Pi}^R(A)\subseteq M(A)$ for the base algebras, where the identities follow from \cite[Theorem 3.13]{Boh1}. Then if moreover $(A,\Delta)$ is left and right full, we have that $A^L$ is (canonically) anti-isomorphic to $A^R$ by the map \[\sigma: A^L \rightarrow A^R, \quad \bar{\Pi}^L(a) \rightarrow \Pi^R(a), \qquad a\in A,\] by \cite[Lemma 4.8]{Boh1}. We then simply refer to $A^L$ as \emph{the} base algebra. 

\begin{Rem}\label{RemNak} We could also have used the map $\bar{\sigma}(\Pi^L(a)) = \bar{\Pi}^R(a)$ to identify $A^L$ and $A^R$. As it turns out, $\bar{\sigma}^{-1}\sigma$ is the (unique) Nakayama automorphism for some functional $\varepsilon$ on $A^L$, cf. \cite[Proposition 4.9]{Boh1}. Hence if $A^L$ is commutative, it follows that $\sigma = \bar{\sigma}$.
\end{Rem} 

\begin{Prop} Let $(A,\Delta)$ be a left and right full regular weak multiplier bialgebra whose base algebra is isomorphic to $\Fun_f(I)$ for some set $I$, and such that moreover $A^LA^R \subseteq A$. Then $(A,\Delta)$ is the total weak multiplier bialgebra of a uniquely determined partial bialgebra over $I$.
\end{Prop} 

\begin{Rem} The condition $A^LA^R \subseteq A$ is of course essential, as we want $A$ to behave locally as a bialgebra, not a multiplier bialgebra. Indeed, in case $A^L= \C$, the condition simply says that $A$ is unital.
\end{Rem} 

\begin{proof} Let us write the standard generators (Dirac functions) of $A^L$ as $\lambda_k$ for $k\in I$, and write $\sigma(\lambda_k) = \rho_k\in A^R$. By assumption, $\UnitC{k}{l} = \lambda_k\rho_l\in A$. Further $A= AA^R = AA^L = A^LA=A^RA$, cf.~ the proof of \cite[Theorem 3.13]{Boh1}. Hence the $\UnitC{k}{l}$ make $A$ into a the total algebra of an $I\times I$-partial algebra, as $A^L$ and $A^R$ pointwise commute by \cite[Lemma 3.5]{Boh1}. 

Define \[\Delta_{rs}(a) = (\rho_r\otimes \lambda_r)\Delta(a)(\rho_s\otimes \lambda_s).\] From \cite[Lemma 3.3]{Boh1}, it follows that $\Delta_{rs}$ is a map from $\Gr{A}{k}{l}{m}{n}$ to $\Gr{A}{k}{l}{r}{s}\otimes \Gr{A}{r}{s}{m}{n}$. That same lemma, together with the coassociativity of $\Delta$, show that the $\Delta_{rs}$ form a coassociative family.  

Now by \cite[Lemma 3.9]{Boh1}, we have $(\rho_k\otimes 1)\Delta(a) = (1\otimes \lambda_k)\Delta(a)$ for all $a$. By that same lemma and Remark \ref{RemNak}, we have as well $\Delta(a)(\rho_k\otimes 1) = \Delta(a)(1\otimes \lambda_k)$. Hence we may as well write \begin{eqnarray*} \Delta_{rs}(a) &=& (\rho_r\otimes 1)\Delta(a)(\rho_s\otimes 1) \\ &=& (1\otimes \lambda_r)\Delta(a)(1\otimes \lambda_s)\end{eqnarray*}  It is now straightforward that the counit map of $(A,\Delta)$ also provides a counit for the $\Delta_{rs}$, hence the $\Gr{A}{k}{l}{m}{n}$ also form a coalgebra. 

As $\Delta(a)(1\otimes \lambda_s)$ and $\Delta(a)(\rho_r\otimes 1)$ are already in $A\otimes A$, it is also clear that $\Delta_{rs}(a)$ is rcf for each $a$. The multiplicativity of the $\Delta_{rs}$ is then immediate from the multiplicativity of $\Delta$.

To show that $\Delta_{ll'}(\UnitC{k}{m}) = \delta_l \UnitC{k}{l}\otimes \UnitC{l}{m}$, it suffices to show that $\Delta(1) = \sum_k \rho_k\otimes \lambda_k$. Now as $\Delta(1)(A\otimes A)  = \Delta(A)(A\otimes A)$, and as clearly $\Delta(a) = \sum_{r,s}\Delta_{rs}(a)$ in the strict topology for all $a\in A$, it follows that \[\Delta(1) = \left(\sum_k \rho_k\otimes \lambda_k\right)\Delta(1).\]  Similarly, $\Delta(1) = \Delta(1)\left(\sum_k\rho_k\otimes \lambda_k\right)$. On the other hand, by \cite[Lemma 4.10]{Boh1} it follows that we can then write \[\sum_{k\in I'} \rho_k\otimes \lambda_k\] for some subset $I'\subseteq I$. As by definition $\bar{\Pi}^L(A) = \Fun_{\fin}(I)$, we deduce that $I=I'$. 

For $a\in \Gr{A}{k}{l}{p}{q}$ and $b\in \Gr{A}{l}{m}{q}{r}$, we then have $\epsilon(ab) = \epsilon(a\Unit{l}{q}b) = \epsilon(a)\epsilon(b)$ by \cite[Proposition 2.6.(4)]{Boh1}. 

Finally, assume that $k$ was such that $\varepsilon(\UnitC{k}{k})=0$. Then by the partial multiplication law, $\varepsilon$ is zero on all $\Gr{A}{k}{l}{k}{l}$. Using the counit property, it follows that $\Gr{A}{k}{l}{m}{n}=0$ for all $l,m,n$. In particular, $\UnitC{k}{m}=0$ for all $m$. But this entails $\lambda_k=0$, a contradiction. Hence $\varepsilon(\UnitC{k}{k})$. From the partial multiplication law, it follows that $\varepsilon(\UnitC{k}{k})^2 = \varepsilon(\UnitC{k}{k})$, hence $\varepsilon(\UnitC{k}{k})=1$.

This concludes the proof that $(A,\Delta)$ determines a partial bialgebra $\mathscr{A}$. It is immediate that $(A,\Delta)$ is in fact the total multiplier bialgebra of $\mathscr{A}$. 
\end{proof} 

\subsection{Partial Hopf algebras}

We now formulate the notion of partial Hopf algebra, whose total form will correspond to a weak multiplier Hopf algebra \cite{Boh1,VDW2,VDW1}. We will mainly refer to \cite{Boh1} for uniformity.

 Let us denote $\circ$ for the inverse of $\wmult$, and $\bullet$ for the inverse of $\bmult$, so \[\begin{pmatrix} k & l \\ m & n \end{pmatrix}^{\circ} = \begin{pmatrix} l & k \\ n & m \end{pmatrix},\quad \begin{pmatrix} k & l \\ m & n \end{pmatrix}^{\bullet} = \begin{pmatrix} m & n \\ k & l \end{pmatrix},\quad \begin{pmatrix} k & l \\ m & n \end{pmatrix}^{\circ \bullet} = \begin{pmatrix} n & m \\ l & k \end{pmatrix}.\] The notation $\circ$ (resp. $\bullet$) will also be used for row vectors (resp. column vectors).

\begin{Def}\label{DefPartBiAlgAnt} An \emph{antipode} for an
  $I$-partial bialgebra $\mathscr{A}$ consists of linear
maps \[S:A(K)\rightarrow A(K^{\circ\bullet})\]
  such that the following property holds: for all $M,P\in M_2(I)$ and
  all $a\in A(M)$, \begin{align} \label{eq:antipode-pi-l}\underset{K\wmult
      L^{\circ\bullet}=P}{\sum_{K\bmult L = M}} a_{(1)K}S(a_{(2)L})&=
    \delta_{P_l,P_r}\epsilon(a)\mathbf{1}(P_l),
    \\ \label{eq:antipode-pi-r}
    \underset{K^{\circ\bullet}\wmult L=P}{\sum_{K\bmult L = M}}
    S(a_{(1)K})a_{(2)L}&=
    \delta_{P_l,P_r}\epsilon(a)\mathbf{1}(P_r).\end{align}

A partial bialgebra $\mathscr{A}$ is called a \emph{partial Hopf algebra} if it admits an antipode.
\end{Def} 

\begin{Rem} Note that condition \ref{Propd} of Definition \ref{DefPartBiAlg} again guarantees that the above sums are in fact finite.
\end{Rem}

If $S$ is an antipode for a partial bialgebra, we can extend $S$ to a
linear map \[S:A\rightarrow A\] on the total algebra $A$.  Conditions
\eqref{eq:antipode-pi-l} and \eqref{eq:antipode-pi-r} then take the
following simple form:
\begin{Lem} \label{lemma:antipode}
  A family of maps $S \colon A(K) \to A(K^{\circ\bullet})$ satisfies
  \eqref{eq:antipode-pi-l} and \eqref{eq:antipode-pi-r} if and only if
  the total map $S\colon A \to A$ satisfies 
  \begin{align} \label{eq:total-antipode}
 a_{(1)}S(a_{(2)}) &= \Pi^{L}(a), &
 S(a_{(1)})a_{(2)} &= \Pi^{R}(a)
  \end{align}
for all $a\in A$.
\end{Lem}

Note that these should be considered a priori as equalities of left (resp. right) multipliers on $A$.

\begin{proof}
For $M,P\in M_{2}(I)$ and $a\in A(M)$, the left and the right hand side  of \eqref{eq:antipode-pi-l} are the $P$-homogeneous components of $ a_{(1)}S(a_{(2)})$ and $\Pi^{L}(a)=\sum_{p} \epsilon(\lambda_{p}a)\lambda_{p}$, respectively.
\end{proof}

\begin{Lem}\label{LemAntiUnit} Let $\mathscr{A}$ be a partial Hopf algebra with antipode $S$. For all $k,l\in I$, $S(\UnitC{k}{l}) = \UnitC{l}{k}$.
\end{Lem}
\begin{proof} For example the first identity in Equation \eqref{eq:total-antipode} of Lemma \ref{lemma:antipode} applied to $\UnitC{k}{k}$ gives \[\sum_l S(\UnitC{l}{k}) = \sum_l \UnitC{k}{l}S(\UnitC{l}{k}) = \lambda_k,\] as $S(\UnitC{l}{k}) \in \Gr{A}{k}{k}{l}{l}$ and $\Pi^{L}(\UnitC{k}{k}) = \lambda_k$. This implies the lemma.
\end{proof} 
\begin{Rem} \label{remark:index-equivalence}
  Let $\mathscr{A}$ be an $I$-partial Hopf algebra. Then the relation
  on $I$ defined by
  \begin{align*}
    k \sim l \Leftrightarrow \UnitC{k}{l} \neq 0
  \end{align*}
is an equivalence relation. Indeed, it is reflexive and transitive by
assumptions (c) and (a) in Definition \ref{DefPartBiAlg}, and
symmetric by the preceding result. We call the set $I_0$ of equivalence classes the \emph{hyperobject} set of $\mathscr{A}$. % Reference to Szlachanyi. 
\end{Rem}
The existence of an antipode is closely related to partial invertibility of
the maps $T_{1},T_{2} \colon A \otimes A \to A\otimes A$ given by
\begin{align} \label{eq:wt-12}
  T_{1} (a\otimes b)&= \Delta(a)(1 \otimes b), &
  T_{2} (a\otimes b)&= (a \otimes 1)\Delta(b).
 \end{align}
The precise formulation involves the linear maps $E_{i},G_{i}
 \colon A\otimes A\to A\otimes A$ given by
\begin{align} \label{eq:e1g1}
  G_{1}(a\otimes b) &=
 \sum_{p} a\rho_{p} \otimes \rho_{p}b, &  E_{1}(a \otimes b) &=\Delta(1)(a\otimes b)=\sum_{p} \rho_{p}a\otimes \lambda_{p}b, \\ \label{eq:e2g2}
 G_{2}(a \otimes b) &= \sum_{p} a\lambda_{p} \otimes
    \lambda_{p}b, &
E_{2}(a\otimes b) &= (a\otimes b)\Delta(1)=\sum_{p} a\rho_{p} \otimes b\lambda_{p}.
\end{align}
\begin{Prop} \label{prop:riti}
  Let $\mathscr{A}$ be a partial Hopf algebra with total algebra $A$,
  total comultiplication $\Delta$ and antipode  $S$. Then the maps
  $R_{1},R_{2} \colon A \otimes A \to M(A \otimes A)$ given by
  \begin{align*}
    R_{1}(a \otimes b) &= a_{(1)}\otimes S(a_{(2)})b, &
    R_{2}(a\otimes b) &= aS(b_{(1)})\otimes b_{(2)}
  \end{align*}
  take values in $A\otimes A$ and satisfy for $i=1,2$ the relations
  \begin{align} \label{eq:riti}
    T_{i}R_{i}&=E_{i}, & R_{i}T_{i}&= G_{i}, & T_{i}R_{i}T_{i}&= T_{i}, & R_{i}T_{i}R_{i} &= R_{i}.
  \end{align}
\end{Prop}
\begin{proof}
  The map $R_{1}$ takes values in $A\otimes A$ because
  \begin{align*}
  a_{(1)} \otimes
  S(a_{(2)})\lambda_{k}\rho_{l} =  a_{(1)} \otimes S(\rho_{l}\lambda_{k}a_{(2)}) \in A
  \otimes A
  \end{align*}
  for all $a\in A$, and Lemma
  \ref{lemma:antipode},  Equation \eqref{eq:pi-l-delta} and Lemma \ref{LemAntiUnit} imply
  \begin{align*}
    T_{1}R_{1}(a \otimes b)&= a_{(1)} \otimes a_{(2)}S(a_{(3)})b =
    a_{(1)} \otimes \Pi^{L}(a_{(2)})b =
    \sum_{p} \rho_{p}a \otimes \lambda_{p}b, \\
    R_{1}T_{1}(a \otimes b) &= a_{(1)} \otimes S(a_{(2)})a_{(3)}b =
    a_{(1)} \otimes \Pi^{R}(a_{(2)})b = \sum_{p} a\rho_{p}\otimes
    \rho_{p}b.
  \end{align*}
 The relations $T_{1}R_{1}T_{1}=T_{1}$ and
$R_{1}T_{1}R_{1}=R_{1}$ follow easily from  \eqref{EqDel} and
\eqref{eq:delta-lambda-rho}. The assertions concerning $R_{2}$ and
$T_{2}$ follow similarly.
\end{proof}
\begin{Theorem}  \label{theorem:partial-hopf-algebra}
  Let $\mathscr{A}$ be a partial bialgebra with total algebra $A$,
  total comultiplication $\Delta$ and counit $\epsilon$. Then the
  following conditions are equivalent:
  \begin{enumerate}[label={(\arabic*)}]
  \item\label{tph1} $\mathscr{A}$ is a partial Hopf algebra.
  \item\label{tph2} There exist linear maps $R_{1},R_{2} \colon A\otimes A\to
    A\otimes A$ satisfying  \eqref{eq:riti}.
  \item\label{tph3} $(A,\Delta,\epsilon)$  is a regular weak multiplier Hopf algebra in the sense of \cite{VDW1}.
  \end{enumerate}
  If these conditions hold, then the total  antipode of $\mathscr{A}$ coincides with the antipode of $(A,\Delta,\epsilon)$.
\end{Theorem}
\begin{proof}
\ref{tph1} implies \ref{tph2} by Proposition \ref{prop:riti}. \ref{tph2} is equivalent to \ref{tph3} by Definition
1.14 in \cite{VDW1}. Indeed, the maps $G_{1},G_{2}$ defined in \eqref{eq:e1g1} and \eqref{eq:e2g2} satisfy
\begin{align*}
  G_{1}(a_{(1)} \otimes b) \otimes a_{(2)}c &= \sum_{p} a_{(1)} \otimes \rho_{p}b
  \otimes a_{(2)}\lambda_{p}c, \\
  ac_{(1)} \otimes G_{2}(b\otimes c_{(2)}) &=\sum_{p} a\rho_{p}c_{(1)} \otimes b\lambda_{p} \otimes c_{(2)}
\end{align*}
and therefore coincide with the maps introduced in Proposition 1.14 in
\cite{VDW1}.  Finally, assume \ref{tph3}. Then
 Lemma 6.14 and equation (6.14) in \cite{Boh1} imply that the antipode
$S$ of $(A,\Delta)$ satisfies $S(A(K))\subseteq A(K^{\circ\bullet})$ and relation \eqref{eq:total-antipode}.  Now, Lemma \ref{lemma:antipode} implies \ref{tph1}.
\end{proof}

From \cite[Proposition 3.5 and Proposition 3.7]{VDW1} or \cite[Theorem
6.12 and Corollary 6.16]{Boh1}, we can conclude that the antipode of a
partial Hopf algebra reverses the multiplication and
comultiplication. Denote by $\Delta^{\op}$ the composition of
$\Delta$ with the flip map.

\begin{Cor} \label{corollary:antipode} Let $\mathscr{A}$ be a partial
  Hopf algebra. Then the total antipode $S:A\rightarrow A$ is uniquely determined and satisfies
  $S(ab) = S(b)S(a)$ and $\Delta(S(a)) = (S\otimes S)\Delta^{\op}(a)$
  for all $a,b\in A$.
\end{Cor} 
\begin{proof} Uniqueness of the antipode follows from the identities \eqref{eq:total-antipode}, see also \cite[Remark 2.8.(ii)]{VDW1}. 
\end{proof} 

We will need the following relation between $\varepsilon$ and $S$ at some point.

\begin{Lem}\label{LemCoAnt} Let $(\mathscr{A},\Delta)$ be a partial Hopf algebra. Then $\varepsilon\circ S = \varepsilon$ on each $\Gr{A}{k}{l}{m}{n}$.
\end{Lem}

\begin{proof} Using the notation in Proposition \ref{prop:riti} and the discussion preceding it, we have that \[T_1: \sum_p(A\rho_p\otimes \rho_p A)\rightarrow \Delta(1)(A\otimes A)\] is a bijection with $R_1$ as inverse. As one easily verifies that $(\id\otimes \varepsilon)T_1 = \id\otimes \varepsilon$ by the partial multiplicativity and counit property of $\varepsilon$, it follows that also $(\id\otimes \varepsilon)R_1 = \id\otimes \varepsilon$ on $\Delta(1)(A\otimes A)$. Applying both sides to $a\otimes \UnitC{k}{k}$ with $a\in \Gr{A}{k}{l}{k}{l}$, we find \[(\id\otimes (\varepsilon\circ S))\Delta_{kl}(a) = a.\] Applying $\varepsilon$ to this identity, we find $\varepsilon\circ S = \varepsilon$ on each $\Gr{A}{k}{l}{k}{l}$, and hence on all $\Gr{A}{k}{l}{m}{n}$.
\end{proof} 


In practice, it is convenient to have an \emph{invertible} antipode around. Although the invertibility often comes for free in case extra structure is around, we will mostly just impose it to make life easier. The following definition follows the terminology of \cite{VDae1}. 

\begin{Def} Let $\mathscr{A}$ be a partial Hopf algebra. We call $\mathscr{A}$ a \emph{regular} partial Hopf algebra if the antipode of $\mathscr{A}$ is invertible.
\end{Def}

From the uniqueness of the antipode, it follows immediately that $S^{-1}$ is then an antipode for $(\mathscr{A},\Delta^{\op})$. Conversely, if both $(\mathscr{A},\Delta)$ and $(\mathscr{A},\Delta{\op})$ have antipodes, then $(\mathscr{A},\Delta)$ is a regular partial Hopf algebra. 

\subsection{Invariant integrals}


\begin{Def}
  Let $\mathscr{A}$ be an $I$-partial bialgebra.  We call a family of
  functionals
\begin{align} \label{eq:functionals}
  \phic{k}{m} \colon A\pmat{k}{k}{m}{m} \to \C
\end{align}
a \emph{left invariant} \emph{integral} if
 $\phic{k}{k}(\UnitC{k}{k})=1$ for all $k\in
I$ and
\begin{align}
  \label{eq:integral}
   (\id \otimes \phic{l}{m})(\Delta_{ll}(a)) 
&= \delta_{k,p} \phic{k}{m}(a)
  \UnitC{k}{l} 
\end{align}
 for all $k,l,m,p\in I$, $a \in A\pmat{k}{p}{m}{m}$. 
 
 We call them a \emph{right invariant}  \emph{integral} if instead one has \begin{align}
  (\phic{k}{l} \otimes
  \id)(\Delta_{ll}(a))&= \delta_{m,p} \phic{k}{m}(a) \UnitC{l}{m}\end{align}
 for all $k,l,m,p\in I$, $a \in A\pmat{k}{k}{m}{p}$. 
 
 A left integral which is at the same time a right invariant integral will simply be called an \emph{invariant integral}.
\end{Def}

As before, we can extend a (left or right) invariant integral to a functional $\phi$ on $A$ by linearity and by putting $\phi=0$ on $\Gr{A}{k}{l}{m}{n}$ if $k\neq l$ or $m\neq n$. The total form of the invariance conditions
\eqref{eq:integral}  reads as follows. 

\begin{Lem} \label{lemma:total-integral}
  A family of functionals  as in   \eqref{eq:functionals}
  is left invariant
  %satisfies the conditions in \eqref{eq:integral} 
  if and only if
for all $a,b\in A$,
  \begin{align*}
(\id\otimes \phi)((b\otimes 1)\Delta(a)) &= \sum_{k}\phi(\lambda_{k}a)b\lambda_k.
      \end{align*}
      It defines a right invariant functional if and only if 
   \begin{align*}   (\phi\otimes \id)(\Delta(a)(1\otimes b)) &= \sum_{n}
\phi(\rho_{n} a)\rho_n b.\end{align*}
\end{Lem}

\begin{proof}
  Straightforward.
\end{proof}

We have the following form of \emph{strong invariance}.

\begin{Lem} \label{lemma:strong-invariance}
  Let $\mathscr{A}$ be a partial Hopf algebra with left invariant integral $\phi$. Then
  for all $a\in A$,
  \begin{align*}
    S\left(( \id\otimes
    \phi)(\Delta(b)(1 \otimes a))\right) &= (\id \otimes \phi)((1 \otimes b)\Delta(a)).
  \end{align*}
  Similarly, if $\mathscr{A}$ is a partial Hopf algebra with right invariant integral $\phi$, then 
   \begin{align*} S\left((\phi \otimes
    \id)(\Delta(a)(1\otimes b))\right) &= (\phi \otimes \id)(\Delta(a)(b \otimes 1)).\end{align*}
\end{Lem}
\begin{proof}
 The counit property, the relations \eqref{EqDel} and
 \eqref{eq:total-antipode} and Lemma \ref{lemma:total-integral} imply
  \begin{align*}
    a_{(1)}\phi(ba_{(2)}) &= \sum_{n}
    a_{(1)}\phi(\epsilon(b_{(1)}\rho_{n})b_{(2)}\lambda_{n}a_{(2)}) \\
&= \sum_{n} \epsilon(b_{(1)}\rho_{n})\rho_{n}a_{(1)}\phi(b_{(2)}a_{(2)})
\\
&= S(b_{(1)})b_{(2)}a_{(1)}\phi(b_{(3)}a_{(2)}) =
S(b_{(1)})\phi(b_{(2)}a)
  \end{align*}
for all $a,b \in A$. The second equation 
follows similarly.
\end{proof}

%\begin{Lem}
  %\begin{enumerate}[label={(\arabic*)}]
  %\item\label{LI1} If $\phi$ is a left or right invariant integral, then
   % $\phi(\UnitC{k}{m})=1$ for all $k,m\in I$ with $\UnitC{k}{m}\neq 0$.
  %\item\label{LI2} If a left integral and a right invariant integral exist, then the two
  %  are equal and unique.
   % \end{enumerate}
%\end{Lem}
%\begin{proof}

%To prove \ref{LI3}, assume that $\phi$ is a left or right invariant integral. Then
%Corollary \ref{corollary:antipode} implies that $\phi\circ S$ is a
%right or left invariant integral, and \ref{LI2} implies the claim.
%\end{proof}

\begin{Lem} Assume that $\mathscr{A}$ is an regular  $I$-partial Hopf algebra which admits a left invariant integral $\phi$. Then the following hold.
\begin{enumerate}[label = {(\arabic*)}]
\item\label{LI1} $\phi(\UnitC{k}{m})=1$ for all $k,m\in I$ with $\UnitC{k}{m}\neq 0$.
\item\label{LI2} $\phi$ is uniquely determined.
\item\label{LI3} $\phi=\phi S$.
\item\label{LI4} $\phi$ is invariant.
\end{enumerate}
\end{Lem}

\begin{proof} 
To see \ref{LI1}, take $a=\UnitC{k}{k}$ in \eqref{eq:integral}. 

Now by Corollary \ref{corollary:antipode}, we have that $\phi S$ is right invariant. But assume that $\psi$ is any
 right invariant integral.     Then for all $k,l,m\in I$, $a\in A\pmat{k}{k}{m}{m}$,
    \begin{align*}
      \phic{k}{m}(a)  &= (\Grt{\psi}{k}{k} \otimes
      \phic{k}{m})(\Delta_{kk}(a)) = \Grt{\psi}{k}{m}(a)\Grt{\phi}{k}{m}(\UnitC{k}{m}) = \Grt{\psi}{k}{m}(a) .
    \end{align*}
  This proves \ref{LI2},\ref{LI3} and \ref{LI4}.  
    

%For $\ref{LI2}$, note that as the map $T_1$ of \eqref{eq:wt-12} has range $\Delta(1)(A\otimes A)$, any element of $A$ can be written as a linear combination of elements of the form $(\id\otimes \phi)(\Delta(b)(1\otimes a))$. But then we compute \begin{eqnarray*} (\id\otimes \phi S)\Delta((\id\otimes \phi)(\Delta(b)(1\otimes a))) &=& b_{(1)} \phi(S(\id\otimes \phi)(\Delta(b_{(2)})(1\otimes a)))) \\ &=& b_{(1)}(\phi\otimes \phi)((1\otimes b_{(2)})\Delta(a)) \\ &=& \sum_n  \phi(\rho_n a) (\id\otimes \phi)(\Delta(b)(1\otimes \rho_n))
%\\ &=& \sum_{n,k} \phi(\rho_na)\phi(\lambda_kb\rho_n)\lambda_k \\ &=& \sum_{n,k} \phi(\rho_na)\phi(\lambda_k\rho_n b)\lambda_k .
%\end{eqnarray*} This is precisely $\sum_k \phi(\lambda_k (\id \otimes \phi)(\Delta(b)(1\otimes a))) \lambda_k$, proving left invariance of $\phi\circ S$. Right invariance of $\phi S$ follows similarly.

%For \ref{LI3},
%assume that $\phi$ and $\psi$ are a  left and a
 %right invariant integral.     Then for all $k,l,m\in I$, $a\in A\pmat{k}{k}{m}{m}$,
  %  \begin{align*}
   %   \phic{k}{m}(a)  &= (\Grt{\psi}{k}{k} \otimes
    %  \phic{k}{m})(\Delta_{kk}(a)) = \Grt{\psi}{k}{m}(a)\Grt{\phi}{k}{m}(\UnitC{k}{m}) = \Grt{\psi}{k}{m}(a) .
    %\end{align*}
    
 %   \ref{LI4} follows from combining \ref{LI2} and \ref{LI3}.
 \end{proof} 


We will need the following lemma at some point. It is an almost verbatim transcription of the argument in \cite[Proposition 3.4]{VDae2}.

\begin{Lem}\label{LemFaith} Let $\mathscr{A}$ be a regular partial Hopf
  algebra with an invariant integral $\phi$. Then
  $\phi$ is faithful in the following sense: if $a\in A$ and
  $\phi(ab) =0$ (resp. $\phi(ba)=0$) for all $b\in A$, then
  $a=0$.
\end{Lem} 
% Ref to Timmermann for case of quantum groupoids? Can be explicitly used here? Ask Fons if he has already derived this result, and acknowledge? 
\begin{proof} Suppose  $\phi(ba)=0$ for all $b$. Then for all $d\in A$ and all functionals $\omega$ on $A$, the element $p = (\omega\otimes \id)((d\otimes 1)\Delta(a))$ satisfies \[(\id\otimes \phi)((1\otimes c)\Delta(p)) = 0.\] Continuing as in the proof of \cite[Proposition 3.4]{VDae2}, we obtain from the antipode trick that \[\sum_n \phi(cS(q)\rho_n)\epsilon(p\lambda_n)=0.\] Choosing now for $c$ and $q$ local units of the form $\lambda_k\rho_l$, the normalization condition on $\phi$ gives that $\epsilon(p\lambda_n)=0$ for all $n$, hence $\epsilon(p)=0$. This implies $\omega(da)=0$. As $\omega$ and $d$ were arbitrary, it follows that $a=0$.

The other case follows similarly, or by considering the opposite comultiplication.
\end{proof} 



\subsection{Partial compact quantum groups}

We now turn towards the structures which will allow us to build operator algebraic quantum groupoids out of our partial Hopf algebras.

\begin{Def} A \emph{partial $^*$-algebra} $\mathscr{A}$ is a partial
  algebra whose total algebra $A$ is equipped with an antilinear,
  antimultiplicative involution $*\colon A\rightarrow A$, $ a\mapsto
  a^*$,  such that the $\mathbf{1}_k$ are selfadjoint for all $k$ in
  the object set. 
\end{Def} 

One can of course give an alternative definition directly in terms of the partial algebra structure by requiring that we are given antilinear maps $A(k,l)\rightarrow A(l,k)$ satisfying the obvious antimultiplicativity and involution properties.

\begin{Def} A \emph{partial $^*$-bialgebra} $\mathscr{A}$ is a
 partial bialgebra whose underlying partial algebra has been
  endowed with a partial $^*$-algebra structure such that
$\Delta_{rs}(a)^* = \Delta_{sr}(a^*)$ for all $a \in \Gr{A}{k}{l}{m}{n}$.
A \emph{partial Hopf $^*$-algebra} is a partial bialgebra which is at the same time a partial $^*$-bialgebra and a partial Hopf algebra.
\end{Def} 
Thus, a partial bialgebra is a partial
$*$-bialgebra if and only if the underlying weak multiplier bialgebra
 is a weak multiplier $*$-bialgebra.

From Theorem \ref{theorem:partial-hopf-algebra} and \cite{Boh1},
\cite{VDW1}, we can deduce:
\begin{Cor} \label{cor:involutive}
  An $I$-partial $^*$-bialgebra $\mathscr{A}$ is an $I$-partial Hopf
  $^*$-algebra if and only if the weak multiplier $^*$-bialgebra
  $(A,\Delta)$ is a weak multiplier Hopf $^*$-algebra. In that case,
  the counit and antipode satisfy
  $\epsilon(a^{*})=\overline{\epsilon(a)}$ and $S(S(a)^{*})^{*}=a$ for
  all $a\in A$. In particular, the total antipode is bijective.
\end{Cor}
\begin{proof}
  The if and only if part follows immediately from  Theorem
  \ref{theorem:partial-hopf-algebra}, the relation for the counit  from
uniqueness of the counit  \cite[Theorem 2.8]{Boh1}, and the relation
for the antipode from \cite[Proposition 4.11]{VDW1}.
\end{proof}


We are finally ready to formulate our main definition.
\begin{Def} A \emph{partial compact quantum group} $\mathscr{G}$ is a
  partial Hopf $^*$-algebra $\mathscr{A} = P(\mathscr{G})$ with an invariant integral  $\phi$ that is positive in the sense  that $\phi(a^*a)\geq 0$ for all $a\in A$. We also say that $\mathscr{G}$ is the partial compact quantum group \emph{defined by} $\mathscr{A}$.
\end{Def} 

\begin{Rem} Following \cite{Hay1}, we could also have called our objects \emph{compact quantum groups of face type}, but we feel this gives the wrong impression when the base algebra is infinite dimensional (i.e.~ the object set is not compact). When referring to partial compact quantum groups, we feel that it is better reflected that only the \emph{parts} of this object are to be considered compact, not the total object. % Is this a clear enough motivation? 
\end{Rem} 

%%% Local Variables: 
%%% mode: latex
%%% TeX-master: "dyn-suq-main"
%%% End: 

\section{Partial tensor categories}

% Rough version. 

The notion of partial algebra has a nice categorification. % Level of generality could possibly be avoided, but seems nice to incorporate a categorified version which neatly mimicks the local behaviour.  
Recall first that the appropriate (vertical) categorification of a unital $\C$-algebra is a $\C$-linear additive tensor category. From now on, by `category' we will by default mean a $\C$-linear additive category. 

\begin{Def} A \emph{partial tensor category} $\CatCC$ over a set $I_0$ consists of 
\begin{enumerate}[label=(\alph*)]
\item a collection of (small) categories $\mathcal{C}_{ij}$ with $i,j\in I_0$, 
\item $\C$-bilinear functors \[\otimes: \CatC_{ij}\times \CatC_{jk}\rightarrow \CatC_{ik},\] 
\item natural isomorphisms \[ \alpha_{X,Y,Z}: (X\otimes Y)\otimes Z \rightarrow X\otimes (Y\otimes Z),\qquad X \in \CatC_{ij},Y\in \CatC_{jk},Z\in \CatC_{kl},\] 
\item non-zero objects $\Unit_{i} \in \CatC_{ii}$,
\item natural isomorphisms \[\lambda_X^{(i)}:\Unit_i\otimes X \rightarrow X,\qquad \rho_X^{(j)}:X\otimes \Unit_j\rightarrow X, \qquad X\in \CatC_{ij},\]
\end{enumerate}
satisfying the obvious associativity and unit constraints. 
\end{Def}
% This definition should be in a previous section.

\begin{Rem} In true analogy with the partial algebra case, we could let the $\Unit_i$ also be zero objects, but this generalisation will not be needed in the following. 
\end{Rem}

The corresponding total notion is as follows. 

\begin{Def} A \emph{tensor category with local units (indexed by $I_0$)} consists of
\begin{enumerate}[label=(\alph*)]
\item a (small) category $\CatC$, 
\item a $\C$-bilinear functor $\otimes: \CatC\times \CatC \rightarrow \CatC$ with compatible associativity constraint $\alpha$, 
\item\label{FinSup} a collection $\{\Unit_i\}_{i\in I_0}$ of objects such that 
\begin{itemize}\item[$\bullet$] $\Unit_i\otimes \Unit_j \cong 0$ for each $i\neq j$, and
\item[$\bullet$] for each object $X$,  $\Unit_i\otimes X \cong 0 \cong X\otimes \Unit_i$ for all but a finite set of $i$,
\end{itemize}
\item\label{UnCon} natural isomorphisms $\lambda_X:\oplus_i (\Unit_i\otimes X) \rightarrow X$ and $\rho_X:\oplus_i(X\otimes \Unit_i)\rightarrow X$ satisfying the obvious unit conditions. 
\end{enumerate} 
\end{Def}
% Notion of local finiteness for abelian categories: finite-dim hom spaces and each object finite length

% Distinction between monoidal category and tensor category sometimes: latter abelian with irreducible unit. 

Note that the condition \ref{UnCon} makes sense because of the local support condition in \ref{FinSup}. 

\begin{Rem} \begin{enumerate}
\item There is no problem in modifying Maclane's coherence theorem, and we will henceforth assume that our partial tensor categories and tensor categories with local units are strict, just to lighten notation. 
\item One can also see the global tensor category $\CatC$ as an inductive limit of (unital) tensor categories. 
\end{enumerate}
\end{Rem}

\begin{Not} If $(\CatC,\otimes,\{\Unit_i\})$ is a tensor category with local units, and $X\in \CatC$, we define \[X_{ij} = \Unit_i\otimes X \otimes \Unit_j,\] and we denote by \[\eta_{ij}:X_{ij} \rightarrow \oplus_{k,l} \left(\Unit_k \otimes X \otimes \Unit_l\right) \cong X\] the natural inclusion maps. % Necessary to introduce eta-maps?
\end{Not}

%The following lemma is trivial. 

\begin{Lem} Up to the appropriate notion of equivalence, there is a canonical one-to-one correspondence between partial tensor categories and tensor categories with local units. 
\end{Lem}

We will not expand upon the appropriate notion of equivalence, as it can easily be furnished by the reader. % OK?

\begin{proof} Let $(\CatC,\otimes,\{\Unit_i\}_{i\in I_0})$ be a tensor category with local units indexed by $I_0$. Then the $\CatC_{ij} = \{X \in \CatC\mid X_{ij} \underset{\eta_{ij}}{\cong} X\}$, seen as full subcategories of $\CatC$, form a partial tensor category upon restriction of $\otimes$.

Conversely, let $\CatCC$ be a partial tensor category. Then we let $\CatC$ be the category formed by formal finite direct sums $\oplus X_{ij}$ with $X_{ij}\in \CatC_{ij}$, and with \[\Mor(\oplus X_{ij},\oplus Y_{ij}) := \oplus_{ij} \Mor(X_{ij},Y_{ij}).\] The tensor product can be extended to $\CatC$ by putting $X_{ij} \otimes X_{kl} = 0$ when $k\neq j$. The associativity constraints can then be summed to an associativity constraint for $\CatC$. It is evident that the $\Unit_i$ provide local units for $\CatC$. 
\end{proof}

\begin{Rem} Another global viewpoint is to see the collection of $\CatC_{ij}$ as a 2-category with 0-cells indexed by the set $I_0$, the objects of the $C_{ij}$ as 1-cells, and  the morphisms of the $C_{ij}$ as 2-cells. As for partial algebras vs.~ linear categories, we will not emphasize this way of looking at our structures, as this viewpoint is not compatible with the notion of monoidal functor between partial tensor categories.
\end{Rem} 

Continuing the analogy with the algebra case, we define the enveloping \emph{multiplier tensor category} of a tensor category with local units. 
%This notion is important for the formulation of the global notion of morphism between tensor categories with local units. 

\begin{Def} Let $\CatCC$ be a partial tensor category over $I_0$ with total tensor category $\CatC$. The \emph{multiplier tensor category} $M(\CatC)$ of $\CatC$ is defined to be the category consisting of formal sums $\oplus X_{ij}$ which are rcf, and with \[\Mor(\oplus X_{ij},\oplus Y_{ij}) = \left(\prod_j\oplus_i  \Mor(X_{ij},Y_{ij}) \right) \cap \left(\prod_i\oplus_j \Mor(X_{ij},Y_{ij})\right),\] the composition of morphisms being entry-wise (`Hadamard product'). 
\end{Def}

\begin{Rem} Because of the rcf condition on objects, we could in fact have written simply $\Mor(\oplus X_{ij},\oplus Y_{ij}) = \prod_{ij} \Mor(X_{ij},Y_{ij})$. 
\end{Rem} 

The tensor product of $\CatC$ can be extended to $M(\CatC)$ by putting \[\left(\oplus X_{ij}\right)\otimes \left(\oplus Y_{ij}\right) = \oplus_{i,j,k} \left(X_{ij}\otimes Y_{jk}\right),\] and similarly for morphism spaces. This makes sense because of the rcf condition of the objects of $M(\CatC)$. The associativity constraints of the $\CatC_{ij}$ can be summed to an associativity constraint for $M(\CatC)$, while $\Unit := \oplus_{i\in I_0} \Unit_i$ becomes a unit for $M(\CatC)$, rendering $M(\CatC)$ into an ordinary tensor category (with unit object).

\begin{Rem} With some effort, a more intrinsic construction of the multiplier tensor category can be given in terms of couples of endofunctors, in the same vein as the construction of the multiplier algebra of a non-unital algebra.
\end{Rem} 
%Alternatively, one could define $M(\CatC)$ more intrinsically (as in the algebra case) as a collection of couples of functors $L_X,R_X:\CatC\rightarrow \CatC$ together with natural isomorphisms $L_X(Y\otimes Z)\rightarrow L_X(Y)\otimes Z$ and $R_X(Y\otimes Z)\rightarrow Y\otimes R_X(Z)$ satisfying appropriate coherence conditions. % Cf. Leinster, Sketch of proof 1.2.15 in higher operads book

\begin{Exa}\label{ExaVectBiGr} Let $I$ be a set. We can consider the partial tensor category $\CatCC = \{\Vect_{\fin}\}_{i,j\in I}$ where each $\CatC_{ij}$ is a copy of the category of finite-dimensional vector spaces $\Vect_{\fin}$, and with each $\otimes$ the ordinary tensor product. The total category $\CatC$ can then be identified with the category $\Vectif$ of finite-dimensional bi-graded vector spaces with the `balanced' tensor product over $I$. More precisely, the tensor product of $V$ and $W$ is $V\itimes W$ with components \[\Gru{V\itimes W}{k}{m} = \oplus_l \;\Gru{V}{k}{l}\otimes \Gru{W}{l}{m}\subseteq V\otimes W.\] The multiplier category $M(\Vect^{I^2}_{\fin})$ equals $\Vectrcf$, the category of bigraded vector spaces which are rcf.
\end{Exa}

We now formulate the appropriate notion of functor between partial tensor categories. Let us first give an auxiliary definition.

\begin{Def} Let $\CatCC$ be a partial tensor category over $I_0$. If $J_0\subseteq I_0$, we call $\CatDD = \{\CatC_{rs}\}_{r,s\in J_0}$ a \emph{restriction} of $\CatCC$. 
\end{Def} 

\begin{Def} Let $\CatCC$ and $\CatDD$ be partial tensor categories over respective sets $I_0$ and $J_0$, and let \[\phi_0:J_0\twoheadrightarrow I_0,\quad k\mapsto k'\] determine a partition $J_0 = \{J_r\mid r\in I_0\}$ with $k\in J_r \iff \phi_0(k)=r$. 

A \emph{unital morphism} from $\CatCC$ to $\CatDD$ (based on $\phi_0$) consists of $\C$-linear functors \[F_{kl}: \CatC_{k'l'}\rightarrow \CatD_{kl},\] natural monomorphisms \[\iota^{(klm)}_{X,Y}: \GrDA{F(X)}{k}{l} \otimes \GrDA{F(Y)}{l}{m} \hookrightarrow \GrDA{F(X\otimes Y)}{k}{m}, \quad X\in \CatC_{k'l'},Y\in \CatD_{l'm'},\] and isomorphisms \[\mu_{k}:  \Unit_k \cong \GrDA{F(\Unit_{k'})}{k}{k}\] with \begin{enumerate}[label=(\alph*)]
\item (Unitality)  $\GrDA{F(\Unit_{r})}{k}{l}= 0$ if $k\neq l$ in $J_r$,
\item (Local finiteness) For each $r,s\in I_0$ and $X\in \CatC_{rs}$, the application $(k,l)\mapsto \GrDA{F(X)}{k}{l}$ is rcf on $J_k\times J_l$. 
\item (Multiplicativity) For all $X\in \CatC_{k's}$ and $Y\in \CatC_{sm'}$, one has\[\oplus_{l\in J_s} \iota^{(klm)}_{X,Y}: \left(\oplus_{l\in J_s} \GrDA{F(X)}{k}{l} \otimes \GrDA{F(Y)}{l}{m}\right) \cong \GrDA{F(X\otimes Y)}{k}{m}.\]
\item (Coherence) %should this be called coherence? 
The $\iota^{(klm)}$ satisfy the 2-cocycle condition making \[\xymatrix{F(X)\otimes F(Y)\otimes F(Z) \ar[rr]^{\id\otimes \iota^{(lmn)}_{Y,Z}} \ar[d]_{\iota^{(klm)}_{X,Y}\otimes\id}&& F(X)\otimes F(Y\otimes Z)\ar[d]^{\iota^{(kln)}_{X,Y\otimes Z}}\\ F(X\otimes Y)\otimes F(Z) \ar[rr]_{\iota^{(kmn)}_{X\otimes Y,Z}}&& F(X\otimes Y \otimes Z)}\] commute for all $X\in \CatC_{k'l'},Y\in \CatC_{l'm'}, Z\in \CatC_{m'n'}$, and the $\mu_k$ satisfy the commutation relations \[\xymatrix{ \GrDA{F(X)}{k}{l}\otimes \Unit_l \ar[r]^{\!\!\!\!\id\otimes \mu_l} \ar@{=}[d] & \GrDA{F(X)}{k}{l} \otimes \GrDA{F(\Unit_{l'})}{l}{l} \ar[d]^{\iota^{(kll)}_{X, \Unit_{l'}}} \\ \GrDA{F(X)}{k}{l} & \ar@{=}[l] \GrDA{F(X\otimes \Unit_{l'})}{k}{l}} \qquad \xymatrix{  \Unit_k\otimes \GrDA{F(X)}{k}{l}\ar[r]^{\!\!\!\!\mu_k\otimes \id} \ar@{=}[d] & \GrDA{F(\Unit_{k'})}{k}{k} \otimes \GrDA{F(X)}{k}{l} \ar[d]^{\iota^{(kkl)}_{\Unit_{k'},X}} \\ \GrDA{F(X)}{k}{l} & \ar@{=}[l] \GrDA{ F(\Unit_{k'}\otimes X)}{k}{l}} \]
\end{enumerate}

A \emph{morphism} from $\CatCC$ to $\CatDD$ is a unital morphism from $\CatCC$ to a restriction of $\CatDD$. 
\end{Def}

The corresponding global notion (of unital morphism) is as follows.

\begin{Lem} Let $\CatCC$ and $\CatDD$ be partial tensor categories over respective sets $I_0$ and $J_0$. Fix an application \[\phi_0: J_0\twoheadrightarrow I_0\] inducing a partition $\{J_k\mid k\in I_0\}$. Then there is a one-to-one correspondence between unital morphisms $\CatCC\rightarrow \CatDD$ based on $\phi_0$ and functors $F:\CatC \rightarrow M(\CatD)$ with isomorphisms \[\iota_{X,Y}:F(X)\otimes F(Y)\cong F(X\otimes Y),\qquad \mu_r:\oplus_{k\in J_r} \Unit_k \cong F(\Unit_r)\] satisfying the natural coherence conditions. 
\end{Lem} 
%a partition $J_0 = \sqcup_{i\in I_0} J_{0,i}$ and a strong monoidal functor $F: \CatCC\rightarrow M(\CatDD)$ with accompanying coherence isomorphisms $\alpha_{X,Y}: F(X)\otimes F(Y) \rightarrow F(X\otimes Y)$ and $\eta_i: F(\Unit_i) \rightarrow \oplus_{j\in J_{0,i}} \Unit_j$ satisfying the obvious compatibility conditions. % Don't overuse obvious.   

The reader will have no problem in furnishing the definition of \emph{equivalence} of partial tensor categories. There is a closely related but weaker notion of equivalence corresponding to chopping up a partial tensor category into smaller pieces (or, vice versa, gluing certain blocks of a partial tensor category together). Let us formalize this in the following definition.

\begin{Def} Let $\CatCC$ and $\CatDD$ be partial tensor categories. We say $\CatDD$ is a \emph{partitioning} of $\CatCC$ (or $\CatCC$ a \emph{globalisation} of $\CatDD$) if there exists a unital morphism $\CatCC\rightarrow \CatDD$ inducing an equivalence of categories $\CatC\rightarrow \CatD$.
\end{Def}

The partial tensor categories that we will be interested in will be required to have some further structure. 

%We will need partial tensor categories with more properties or structure, such as \emph{semi-simplicity}, \emph{duality} and \emph{C$^*$-structure}.

\begin{Def} A partial tensor category $\CatCC$ is called \emph{semi-simple} if all $\CatC_{ij}$ are semi-simple. 

A partial tensor category is said to have \emph{indecomposable units} if all units $\Unit_i$ are indecomposable. 
\end{Def}

It is easy to see that any semi-simple tensor category can be partitioned into a semi-simple tensor category with indecomposable units.  Hence we will from now on only consider semi-simple partial tensor categories with indecomposable units.


%Let $\CatCC$ be an $I_0$ partial semi-simple tensor category over an index set $I_0$. Then we may decompose the units $\Unit_k$ as direct summands $\Unit_k = \oplus_{r\in J_k} \Unit_r$ for certain finite sets $J_k$, with $\Unit_r$ indecomposable. Since $\CatCC_{kk}$ is a tensor category with unit, we know that $\End(\Unit_k)$ is abelian, hence $\Unit_r\cong \Unit_{r'}$ if and only if $r=r'$ in $J_k$. It is then easy to see that the associated total tensor category $\CatC$ also has a system of local units over the index set $J_0=\sqcup\{J_k\mid k\in I_0\}$ with associated map $\phi_0:J_0 \twoheadrightarrow I_0$. The associated $J_0$-partial tensor category $\CatDD$ is then semi-simple with indecomposable units, and admits a morphism $\CatDD\rightarrow \CatCC$ based on $\phi_0$. It is easy to see from this that there is no loss in generalisation by
 

The following definition introduces the notion of duality for partial tensor categories.

\begin{Def} Let $\CatCC$ be a partial tensor category. 

An object $X\in \CatC_{ij}$ is said to admit a \emph{right dual} if there exists an object $Y=X^* \in \CatC_{ji}$ and morphisms $\ev_{X}: X\otimes Y \rightarrow \Unit_i$ and $\coev_X: \Unit_j\rightarrow Y\otimes X$ satisfying the obvious snake identities.

We say $\CatCC$ \emph{admits right duality} if each object of each $\CatC_{ij}$ has a right dual.
\end{Def}

Similarly, one defines left duality and (two-sided) duality. As for tensor categories with unit, if $X$ admits a (left or right) dual, it is unique up to isomorphism. 

\begin{Lem}\label{LemMorDua}
\begin{enumerate}
\item Let $\CatCC$ be a partial tensor category. If $X$ has right dual $X^*$, then $X$ is a left dual to $X^*$. 
\item Let $F$ be a morphism $\CatCC\rightarrow \CatDD$. If $X\in \CatC_{k'l'}$ has a right dual, $F_{lk}(X^*)$ is a right dual to $F_{kl}(X)$.
 \end{enumerate}
\end{Lem}
\begin{proof}
We can consider the restriction $\CatCC'$ of $\CatCC$ to any two-element set $J_0$ of $I_0$, and apply the  usual arguments to $\CatCC'$ and the global functor $F:\CatC'\rightarrow M(\CatD)$ (using that local units are self-dual and that duality behaves anti-multiplicatively w.r.t.~ tensor products). % OK?
\end{proof}

A final ingredient which will be needed is an analytic structure on our partial tensor categories.

\begin{Def} A \emph{partial semi-simple tensor C$^*$-category} is a partial tensor category $(\CatC_{ij},\otimes)$ such that all $\CatC_{ij}$ are semi-simple C$^*$-categories, such that all functors $\otimes$ are $^*$-functors (in the sense that $(f\otimes g)^* = f^*\otimes g^*$ for morphisms), and such that the associativity and unit constraints are unitary.
\end{Def} 
% Terminology not completely appropriate, i.e. we shouldn't assume semi-simplicity, but we just do this for convenience of language. 



\begin{Rem}
\begin{enumerate}
\item If $\CatCC$ is a partial tensor C$^*$-category, the total category $\CatC$ only has pre-C$^*$-algebras as endomorphism spaces, as the morphisms spaces need not be closed in the C$^*$-norm. On the other hand, $M(\CatC)$ only has $^*$-algebras as endomorphism spaces, since we did not restrict our direct products. However, since we demand the constraints to be unitary, there is no problem to restrict to morphisms in $M(\CatC)$ which have bounded norm in the multiplier tensor category. We will denote the resulting category by $M_b(\CatC)$. % To expand? Necessary to include?
\item The notion of right duality for a partial tensor C$^*$-category is the same as in the absence of a C$^*$-structure. However, because of the presence of the $^*$-structure, any right dual is automatically a two-sided dual, and the dual object of $X$ is then simply denoted $\overline{X}$.  
\end{enumerate}
\end{Rem}

\begin{Exa} Let $I$ be a set. Then we can consider the partial semi-simple tensor C$^*$-category $\CatCC = \{\Hilb_{\fin}\}_{I\times I}$ of finite-dimensional Hilbert spaces, with all $\otimes$ the ordinary tensor product. The associated global category is the category $\Hilbif$ of finite-dimensional bi-graded Hilbert spaces. Note that $\CatCC$ has duality, with the dual of a Hilbert space $\Hsp \in \CatC_{ij}$ being the ordinary dual Hilbert space $\Hsp^* \cong \overline{\Hsp}$, but considered in the category $\CatC_{ji}$. 
\end{Exa}

The notion of morphism for partial semi-simple tensor C$^*$-categories has to be adapted in the following way.

\begin{Def} Let $\CatCC$ and $\CatDD$ be partial semi-simple tensor C$^*$-categories over respective sets $I_0$ and $J_0$, and let $\phi_0:J_0\rightarrow I_0$. 
A \emph{morphism} from $\CatCC$ to $\CatDD$ (based on $\phi_0$) is a $\phi_0$-based morphism $(F,\iota,\mu)$ from $\CatCC$ to $\CatDD$ as partial tensor categories, with the added requirement that all $F_{kl}$ are $^*$-functors and the $\iota$- and $\mu$-maps isometric. 
\end{Def} 


% Formulate this already as one arrow in the TK-reconstruction process.
% Ref to Gaby

\section{Representation theory of partial compact quantum groups}

In this section, the representation theory of partial compact quantum groups is investigated. In what follows, the homogeneous component $A(K) = \eGr{A}{k}{l}{m}{n}$ of a partial bialgebra will now be mainly written as $A(K) = \Gr{A}{k}{l}{m}{n}$. 

%As the situation is quite similar to the case already studied by Hayashi \cite{Hay1}, we do not always provide fully written out proofs, but only draw attention to those parts of the theory which need modification.


\subsection{Corepresentations of partial Hopf algebras}


Let $\mathscr{A}$ be an $I$-partial bialgebra. We write
$\Hom_\C(V,W)$ for the vector space of linear maps between two vector
spaces $V$ and $W$.

Recall that $\Vectrcf$ denotes the category whose objects are $I^{2}$-graded
vector spaces $V=\bigoplus_{k,l\in I} \Gru{V}{k}{l}$  which are row-and column-finite dimensional. Morphisms are linear maps $T$ that preserve the grading and therefore
can be written $T=\prod_{k,l\in I} \Gru{T}{k}{l}$. 
%The subcategory ofis written $\Vectrcf$ (this is a slight redefinition of the term rcf!). %Parenthetical remark formulated differently? 


%Clearly, this
%category is abelian and $\C$-linear.  We call an $I^{2}$-graded vector
%space $V=\bigoplus_{k,l\in I} \Gru{V}{k}{l}$ \emph{row- and
  %column-finite} if $\oplus_l \Gru{V}{k}{l}$ (resp. $\oplus_l
%\Gru{V}{k}{l}$) is finite-dimensional for $k$ (resp. $l$) fixed.
%We henceforth abbreviate ``row- and column-finite'' by rcf.

\begin{Def} \label{definition:corep} Let $\mathscr{A}$ be an
  $I$-partial bialgebra and let $V=\bigoplus_{k,l} \Gru{V}{k}{l}$ % lower the indices?
   be
an rcfd $I^{2}$-graded vector space.  A \emph{corepresentation}
  $\mathscr{X}=(\Gr{X}{k}{l}{m}{n})_{k,l,m,n}$ of $\mathscr{A}$ on $V$
  is a family of elements
 \begin{align} \label{eq:rep-blocks}
   \Gr{X}{k}{l}{m}{n} \in \Gr{A}{k}{l}{m}{n} \otimes
  \Hom_\C(\Gru{V}{m}{n},\Gru{V}{k}{l})
 \end{align}
 satisfying 
 \begin{align}
   \label{eq:rep-comultiplication}
    (\Delta_{pq} \otimes
    \id)(\Gr{X}{k}{l}{m}{n}) &=
    \Big{(}\Gr{X}{k}{l}{p}{q}\Big{)}_{13}\Big{(}\Gr{X}{p}{q}{m}{n}\Big{)}_{23},
    \\ \label{eq:rep-counit}
(\epsilon \otimes
  \id)(\Gr{X}{k}{l}{m}{n})&=\delta_{k,m}\delta_{l,n}\id_{\Gru{V}{k}{l}}
 \end{align}
  for all possible indices. We also call $(V,\mathscr{X})$ an
  \emph{rcfd corepresentation}.
\end{Def}
Here, we use here the standard leg numbering notation, e.g.~ $a_{23}=1\otimes a$.
\begin{Exa} \label{example:rep-triv} Equip the vector space
  $\C^{(I)}=\bigoplus_{k\in I} \C$ with the diagonal
  $I^{2}$-grading. Then the family $\mathscr{U}$ given by
  \begin{align} \label{eq:rep-triv}
    \Gr{U}{k}{l}{m}{n} = \delta_{k,l}\delta_{m,n} \UnitC{k}{m} \in
    \Gr{A}{k}{l}{m}{n}
  \end{align}
is a corepresentation of $\mathscr{A}$ on $\C^{(I)}$. We call it the
\emph{trivial corepresentation}.
\end{Exa}

\begin{Exa} \label{example:rep-regular}
  Assume  given an rcfd family of subspaces
  \begin{align*}
    \Gru{V}{m}{n} \subseteq \bigoplus_{k,l} \Gr{A}{k}{l}{m}{n}
  \end{align*}
  satisfying
  \begin{align} \label{eq:rep-regular-inclusion}
    \Delta_{pq}(\Gru{V}{m}{n}) &\subseteq \Gru{V}{p}{q} \otimes
    \Gr{A}{p}{q}{m}{n}.
  \end{align}
Then the  elements $\Gr{X}{k}{l}{m}{n} \in \Gr{A}{k}{l}{m}{n} \otimes
  \Hom_{\C}(\Gru{V}{m}{n},\Gru{V}{k}{l})$ defined by 
  \begin{align*}
    \Gr{X}{k}{l}{m}{n}(1 \otimes b) &= \Delta^{\op}_{kl}(b) \in
    \Gr{A}{k}{l}{m}{n} \otimes \Gru{V}{k}{l} \quad
    \text{for all } b\in \Gru{V}{m}{n}
  \end{align*}
  form a corepresentation $\mathscr{X}$ of $\mathscr{A}$ on
  $V$. Indeed, 
  \begin{align*}
    (\Delta_{pq} \otimes \id)(\Gr{X}{k}{l}{m}{n})(1 \otimes 1 \otimes
    b) &=(\Delta_{pq}\otimes \id)(\Delta^{\op}_{kl}(b)) =
    \Big{(}\Gr{X}{k}{l}{p}{q}\Big{)}_{13}\Big{(}\Gr{X}{p}{q}{m}{n}\Big{)}_{23}(1
    \otimes 1 \otimes b), \\
    (\epsilon \otimes \id)(\Gr{X}{k}{l}{m}{n})b &= (\epsilon \otimes
    \id)(\Delta^{\op}_{kl}(b)) = \delta_{k,m}\delta_{l,n}b
  \end{align*}
  for all $b\in \Gru{V}{m}{n}$.  We call $\mathscr{X}$ the
  \emph{regular corepresentation on $V$}. 
\end{Exa}

We next consider the total form of a corepresentation.

Let $\mathscr{A}$ be a partial bialgebra with total algebra $A$, and
let $V$ be an rcfd $I^{2}$-graded vector space.
Denote by $\lambda^{V}_{k},\rho^{V}_{l} \in \Hom_{\C}(V)$ the
projections onto the summands $\Gru{V}{k}{} = \bigoplus_{q}
\Gru{V}{k}{q}$, write $\Gru{V}{}{l}=\bigoplus_{p}\Gru{V}{p}{l}$,
respectively, and identify $\Hom_{\C}(\Gru{V}{m}{n},\Gru{V}{k}{l})$ with
$\lambda^{V}_{k}\rho^{V}_{l}\Hom_{\C}(V)\lambda^{V}_{m}\rho^{V}_{n}$. Denote by $\Hom_{\C}^{0}(V) \subseteq \Hom_{\C}(V)$ the algebraic sum of all
these subspaces. Then we can define a homomorphism
\begin{align*}
  \Delta \otimes \id \colon M(A \otimes \Hom_{\C}^{0}(V)) \to M(A
  \otimes A \otimes \Hom_{\C}^{0}(V))
\end{align*}
similarly as we defined $ \Delta \colon A \to M(A\otimes A)$.
\begin{Lem} \label{lemma:rep-multiplier}
  Let  $\mathscr{A}$ be an $I$-partial bialgebra and $V$  an rcfd $I^{2}$-graded vector space.  If $\mathscr{X}$ is a
  corepresentation of  $\mathscr{A}$ on $V$, then the sum
  \begin{align}
    \label{eq:rep-multiplier}
  X:=\sum_{k,l,m,n} \Gr{X}{k}{l}{m}{n} \in  M(A
  \otimes \Hom_{\C}^{0}(V))
  \end{align}
 converges strictly and satisfies the following conditions:
  \begin{enumerate}[label=(\arabic*)] %\setcounter{enumi}{-1}
  \item\label{repma} $(\lambda_{k}\rho_{m} \otimes \id){X}(\lambda_{l}\rho_{n}
    \otimes \id) = (1 \otimes \lambda^{V}_{k}\rho^{V}_{l}){X}(1 \otimes
    \lambda^{V}_{m}\rho^{V}_{n}) = \Gr{X}{k}{l}{m}{n}$,
  \item\label{repmb} $(A \otimes 1){X}$, $ {X}(A \otimes 1)$ and $(1 \otimes
    \Hom^{0}_{\C}(V))X(1 \otimes \Hom^{0}_{\C}(V))$ lie in $A \otimes \Hom_{\C}^{0}(V)$,%Last inlcusion superfluous, necessary to include?
  \item\label{repmc} $(\Delta\otimes \id)(X)=X_{13}X_{23}$, 
  \item\label{repmd} the sum $(\epsilon \otimes \id)({X}) :=\sum (\epsilon \otimes
    \id)(\Gr{X}{k}{l}{m}{n})$ converges in $M(\Hom^{0}_{\C}(V))$ strictly
    to $\id_{V}$.
  \end{enumerate}
  Conversely, if $ X \in M(A \otimes \Hom_{\C}^{0}(V))$ satisfies
  \ref{repma}--\ref{repmd} with $\Gr{X}{k}{l}{m}{n}$ defined by \ref{repma}, then
  $\mathscr{X}=(\Gr{X}{k}{l}{m}{n})_{k,l,m,n}$ is a corepresentation
  of $\mathscr{A}$ on $V$.
\end{Lem}
\begin{proof}
 Straightforward.
\end{proof}

\begin{Def} If $\mathscr{X}$ and $X$ are as in Lemma \ref{lemma:rep-multiplier}, we will call $X$ the \emph{corepresentation multiplier} of $\mathscr{X}$. 
\end{Def}


If $\mathscr{A}$ is a partial Hopf algebra,  then every
corepresentation multiplier has a generalized inverse.
\begin{Lem} \label{lemma:rep-invertible}
  Let $(V,\mathscr{X})$ be an rcfd corepresentation of an $I$-partial Hopf
  algebra $\mathscr{A}$. Then with $\Gr{Z}{k}{l}{m}{n} = (S\otimes \id)(\Gr{X}{n}{m}{l}{k})$, we have $\Gr{Z}{k}{l}{m}{n}\in \Gr{A}{k}{l}{m}{n}\otimes \Hom_{\C}(\Gru{V}{l}{k},\Gru{V}{n}{m})$ and
  \begin{align*}
    \Gr{X}{k}{l}{m}{n}  \cdot \Gr{Z}{l}{k'}{n}{m'} &=0 \text{ if } m'\neq m &
      \sum_{n} \Gr{X}{k}{l}{m}{n} \cdot \Gr{Z}{l}{k'}{n}{m} &= \delta_{k,k'}\UnitC{k}{m} \otimes
      \id_{\Gru{V}{k}{l}}, \\
      \Gr{Z}{n}{m}{l}{k}\cdot \Gr{X}{m}{n'}{k}{l'} &= 0
      \text{ if } n\neq n' & 
      \sum_{m} \Gr{Z}{n}{m}{l}{k}\cdot \Gr{X}{m}{n}{k}{l'} &=
      \delta_{l,l'} \UnitC{n}{l} \otimes \id_{\Gru{V}{k}{l}}.
  \end{align*}
  In particular, the multiplier $Z:=     (S \otimes
  \id)(X) \in M(A \otimes \Hom_{\C}^{0}(V))$
  satisfies
  \begin{align} \label{eq:rep-generalized-inverse}
    XZ &= \sum_{k} \lambda_{k} \otimes \lambda^{V}_{k}, &
    ZX &= \sum_{l} \rho_{l} \otimes \rho^{V}_{l},
  \end{align}
  and is a generalized inverse of $X$ in the sense that $XZX=X$ and $ZXZ=Z$.
\end{Lem}
\begin{proof}
  The grading property of $\Gr{Z}{k}{l}{m}{n}$ follows from  $S(\Gr{A}{p}{q}{r}{s})\subseteq \Gr{A}{s}{r}{q}{p}$, and then the upper left hand identity is immediate.  To
  verify the upper right hand one, we use identities \eqref{eq:rep-comultiplication}, \eqref{eq:rep-counit} and \eqref{eq:antipode-pi-l}. Namely, with $M_{A}$ denoting the multiplication of $A$, we find
  \begin{align*}
      \sum_{n} \Gr{X}{k}{l}{m}{n} \cdot (S \otimes
      \id)(\Gr{X}{m}{n}{k'}{l}) &= \sum_{n} (M_{A}  (\id \otimes S)
      \otimes \id)((\Gr{X}{k}{l}{m}{n})_{13}(\Gr{X}{m}{n}{k'}{l})_{23})
 \\ &= \sum_{n} (M_{A} (\id \otimes S)  \Delta_{m,n} \otimes
      \id)(\Gr{X}{k}{l}{k'}{l}) \\
      &= \delta_{k,k'} \UnitC{k}{l} \otimes (\epsilon \otimes
      \id)(\Gr{X}{k}{l}{k'}{l})
      \\ &=
\delta_{k,k'}\UnitC{k}{m} \otimes
      \id_{\Gru{V}{k}{l}}. \end{align*} The other
equations follow similarly, and the assertions concerning $Z$ are
direct consequences.
\end{proof}
\begin{Def}
  Let $\mathscr{X}$ be an rcfd corepresentation of a  partial Hopf
  algebra.  We  denote the generalized inverse $(S \otimes \id)(X)$
  of $X$  by $X^{-1}$ and let
  \begin{align*}
   \Gr{(X^{-1})}{k}{l}{m}{n}=(S \otimes \id)(\Gr{X}{l}{k}{n}{m}) \in
   \Gr{A}{k}{l}{m}{n} \otimes \Hom_{\C}(\Gru{V}{l}{k},\Gru{V}{n}{m})
  \end{align*}% Prefer to have grading in this way...
\end{Def}
For completeness, we mention the following following converse to Lemma \ref{lemma:rep-invertible}
\begin{Lem}
  Let $\mathscr{A}$ be an $I$-partial bialgebra, $V$ an rcfd $I^{2}$-graded vector space and $X,Z \in M(A \otimes
  \Hom_{\C}^{0}(V))$. If conditions \ref{repma}--\ref{repmc} in Lemma
  \ref{lemma:rep-multiplier} and
  \eqref{eq:rep-generalized-inverse} hold, then the corresponding
  family $\mathscr{X}=(\Gr{X}{k}{l}{m}{n})_{k,l,m,n}$ is a
  corepresentation of $\mathscr{A}$ on $V$.
\end{Lem}
\begin{proof}
  We have to verify condition \ref{repmd} in Lemma
  \ref{lemma:rep-multiplier}.  If $(k,l) \neq (p,q)$, then
  $\epsilon(\Gr{A}{k}{l}{p}{q})=0$ and hence $(\epsilon
  \otimes \id)(\Gr{X}{k}{l}{p}{q}) =0$. The counit property and condition
  \ref{repmc} in Lemma \ref{lemma:rep-multiplier} imply 
\begin{align*}
  \Gr{X}{k}{l}{m}{n} &= ((\epsilon\otimes \id)  \Delta \otimes
  \id)(  \Gr{X}{k}{l}{m}{n}) 
\\ &  = \sum_{p,q} (\epsilon\otimes \id \otimes
  \id)\left((\Gr{X}{k}{l}{p}{q})_{13}(\Gr{X}{p}{q}{m}{n})_{23}\right)
  =  (1 \otimes \Gru{T}{k}{l})\Gr{X}{k}{l}{m}{n},
\end{align*}
where $\Gru{T}{k}{l}=(\epsilon \otimes \id)(\Gr{X}{k}{l}{k}{l}) \in
\Hom_{\C}(\Gru{V}{k}{l})$.  Therefore,  $T=\prod_{k,l} T_{k,l}$  satisfies $(1 \otimes T)X =
X$. Multiplying on the right by $Z$, we find
$T\lambda^{V}_{k}=\lambda^{V}_{k}$ for all $k$. Thus, $T=\id_{V}$.
\end{proof}

Morphisms of corepresentations are defined as follows.
\begin{Def}
  Let $\mathscr{A}$ be an $I$-partial bialgebra.  A \emph{morphism}
  $T$ between rcfd corepresentations
  $(V,\mathscr{X})$ and $(W,\mathscr{Y})$ of $\mathscr{A}$ is a family
  of linear maps
  \[\Gru{T}{k}{l} \in
  \Hom_\C(\Gru{V}{k}{l},\Gru{W}{k}{l})\] satisfying \[(1 \otimes
  \Gru{T}{k}{l})\Gr{X}{k}{l}{m}{n} = \Gr{Y}{k}{l}{m}{n}(1 \otimes
  \Gru{T}{m}{n})\]
\end{Def}
We denote the category of all corepresentations of $\mathscr{A}$ on rcfd $I^2$-graded vector spaces by
$\Corep_{\rcf}(\mathscr{A})$.
\begin{Rem} \label{remark:rep-total-morphism}
 Equivalently, a morphism between
    $(V,\mathscr{X})$ and $(W,\mathscr{Y})$ is just a morphism of
    $I^{2}$-graded vector spaces $T\colon V\to W$ satisfying
    $(1\otimes T) X= Y(1 \otimes T)$. If $\mathscr{A}$ is  a
    partial Hopf algebra, this condition is equivalent to each of the relations
    \begin{align*}
      Y^{-1}(1 \otimes T)X&=\sum_{m,n} \rho_{n} \otimes \Gru{T}{m}{n},
      &
    Y(1\otimes T)X^{-1} &=\sum_{k,l} \lambda_{k} \otimes \Gru{T}{k}{l}.
    \end{align*}
\end{Rem}


Given an rcfd $I^{2}$-graded vector space $V=\bigoplus_{k,l} \Gru{V}{k}{l}$
and a family of subspaces $\Gru{W}{k}{l} \subseteq \Gru{V}{k}{l}$, we
denote by $\iota_{W}\colon W\to V$ and $\pi_{W} \colon V \to
V/W=\bigoplus_{k,l} \Gru{V}{k}{l}/\Gru{W}{k}{l}$ the embedding and the
quotient map.
\begin{Def} Let $(V,\mathscr{X})$ be an rcfd
  corepresentation of a partial bialgebra $\mathscr{A}$.  We call a
  family of subspaces $\Gru{W}{k}{l} \subseteq \Gru{V}{k}{l}$
  \emph{invariant (w.r.t.\ $\mathscr{X}$)} if
 \begin{align} \label{eq:rep-invariant} (1\otimes
   \Gr{\pi}{k}{l}{}{W})\Gr{X}{k}{l}{m}{n}(1 \otimes
   \Gr{\iota}{m}{n}{}{W}) =0.
  \end{align}
We call $(V,\mathscr{X})$ 
 \emph{irreducible} if the only invariant families of subspaces are
 $(0)_{k,l}$ and $(\Gru{V}{k}{l})_{k,l}$.
\end{Def}

The next lemmas deal with restriction, factorisation and Schur's lemma. We skip their proofs which are straightforward.

%In the next lemma, we show that if a corepresentation has an invariant subspace, it restricts to it and factorizes to the quotient.
\begin{Lem}
  Let $(V,\mathscr{X})$ be an rcfd corepresentation
  of a partial bialgebra and let $\Gru{W}{k}{l}
  \subseteq \Gru{V}{k}{l}$ be an invariant family of subspaces. Then
  there exist unique rcfd corepresentations
  $(W,\iota_{W}^{*}\mathscr{X})$ and $(V/W,(\pi_{W})_{*}\mathscr{X})$ 
  such that $\iota_{W}$  and  $\pi_{W}$  are  morphisms  $(W,\iota_{W}^{*}\mathscr{X}) \to (V,\mathscr{X}) \to (V/W,(\pi_{W})_{*}\mathscr{X})$.
\end{Lem}
%\begin{proof}
 % Straightforward.
%\end{proof}
%The following analogue of Schur's Lemma holds.
\begin{Lem} Let $T$ be a morphism of rcfd
  corepresentations $(V,\mathscr{X})$ and $(W,\mathscr{Y})$ of a
  partial bialgebra. Then the families of subspaces $\ker
  \Gru{T}{k}{l} \subseteq \Gru{V}{k}{l}$ and $\img\Gru{T}{k}{l}
  \subseteq \Gru{W}{k}{l}$ are invariant.  In particular, if
  $(V,\mathscr{X})$ and $(W,\mathscr{Y})$ are irreducible, then either
  all $\Gru{T}{k}{l}$ are zero or all $\Gru{T}{k}{l}$ are
  isomorphisms.
\end{Lem} 
%\begin{proof}
  %Straightforward again.
%\end{proof}


Given corepresentations $\mathscr{X}$ and $\mathscr{Y}$ of
a partial bialgebra $\mathscr{A}$ on respective rcfd $I^{2}$-graded vector spaces $V$ and $W$,
we  obtain an $I^{2}$-graded vector space $V\oplus W$ by taking
component-wise direct sums, and use the canonical embedding 
\begin{align*}
  \Hom(\Gru{V}{m}{n},\Gru{V}{k}{l}) \oplus
  \Hom(\Gru{W}{m}{n},\Gru{W}{k}{l}) \hookrightarrow
  \Hom(\Gru{V}{m}{n} \oplus \Gru{W}{m}{n},\Gru{V}{k}{l} \oplus
  \Gru{W}{k}{l})
\end{align*}
to define the \emph{direct sum} $\mathscr{X} \oplus \mathscr{Y}$,
which is a corepresentation of $\mathscr{A}$ on $V\oplus W$. Then the
natural embeddings from $V$ and $W$ into $V\oplus W$ and the
projections onto $V$ and $W$ are evidently morphisms of
corepresentations.  More generally, given a family of rcfd corepresentations
$((V_{\alpha},\mathscr{X}_{\alpha}))_{\alpha}$ such that the sum
$\bigoplus_{\alpha} V_{\alpha}$ is rcfd again, one
can form the direct sum $\bigoplus_{\alpha} \mathscr{X}_{\alpha}$,
which is a corepresentation on $\bigoplus_{\alpha} V_{\alpha}$.
\begin{Prop}
  Let $\mathscr{A}$ be an $I$-partial bialgebra. Then $\Corep_{\rcf}(\mathscr{A})$
  is a $\C$-linear abelian category, and the forgetful functor
  $\Corep_{\rcf}(\mathscr{A}) \to \Vectrcf$ lifts kernels, cokernels and biproducts.
\end{Prop}
\begin{proof}
  The preceding considerations show that the forgetful functor lifts
  kernels, cokernels and biproducts. Moreover, in
  $\Corep_{\rcf}(\mathscr{A})$, every monic is a kernel
  and every epic is a cokernel because the same is true in $\Vecti$
  and because kernels and cokernels lift.
\end{proof}


\subsection{Tensor product and duality}

Recall from Example \ref{ExaVectBiGr} that the category $\Vectrcf$ is a tensor category. The tensor product of morphisms is the
restriction of the ordinary tensor product.  We will interpret this product as being strictly associative.  The unit for this product is the vector
space $\C^{(I)}=\bigoplus_{k\in I} \C$. 
%Indeed, for every
%$I^{2}$-graded vector space $V$, there exist obvious natural
%isomorphisms $\C^{(I)} \itimes V \cong V \cong V \itimes \C^{(I)}$.

%Note that $V\itimes W$ is  rcf if $V$ and $W$ are.

Given $V$ and $W$ in $\Vectrcf$, we identify $\Hom_\C(\Gru{V}{m}{n},\Gru{V}{k}{l})\otimes
   \Hom_\C(\Gru{W}{n}{q},\Gru{W}{l}{p})$ with a subspace of
\begin{align*}
   \Hom_\C(\Gru{V}{m}{n}\otimes
   \Gru{W}{n}{q},\Gru{V}{k}{l}\otimes \Gru{W}{l}{p})\subseteq
   \Hom_\C(\Gru{(}{m}{}V\itimes
     W\Gru{)}{}{q},\Gru{(}{k}{}V\itimes W\Gru{)}{}{p}).
\end{align*}


We can now construct a product of corepresentations as follows.
\begin{Lem} Let $\mathscr{X}$ and $\mathscr{Y}$ be copresentations of
  $\mathscr{A}$ on respective  rcfd $I^{2}$-graded vector spaces $V$ and
  $W$. Then the sum
  \begin{align} \label{eq:rep-product-blocks}
     \Gr{(X\Circt Y)}{k}{p}{m}{q} := \sum_{l,n}
    \left(\Gr{X}{k}{l}{m}{n}\right)_{12}\left(\Gr{Y}{l}{p}{n}{q}\right)_{13}
  \end{align}
  has only finitely many non-zero terms, and the elements
 \[\Gr{(X\Circt
    Y)}{k}{p}{m}{q}\in \Gr{A}{k}{p}{m}{q} \otimes
  \Hom_\C(\Gru{(}{m}{}V\itimes W\Gru{)}{}{q},\Gru{(}{k}{}V\itimes W\Gru{)}{}{p})
\]
define an rcfd corepresentation $\mathscr{X} \Circt \mathscr{Y}$ of
$\mathscr{A}$ on $V\itimes W$. 
\end{Lem} 
\begin{proof}
  The sum \eqref{eq:rep-product-blocks} is finite because $V$ and
  $W$ are  rcfd. Using the identification above, we
  see that
 \[
  \left(\Gr{X}{k}{l}{m}{n}\right)_{12}\left(\Gr{Y}{l}{p}{n}{q}\right)_{13}\in \Gr{A}{k}{p}{m}{q} \otimes \Hom_\C(\Gru{(}{m}{}V\itimes
    W\Gru{)}{}{q},\Gru{(}{k}{}V\itimes W\Gru{)}{}{p}).\] Now,   the fact that $\Gr{(X\Circt
    Y)}{k}{p}{m}{q}$ is a corepresentation follows easily
  from the multiplicativity of $\Delta$ and the weak multiplicativity
  of $\epsilon$.
\end{proof}
\begin{Rem} \label{remark:rep-tensor-multiplier}
  The `total' multiplier corepresentation associated to $\mathscr{X}\Circt
  \mathscr{Y}$   is  just $X_{12}Y_{13}$.
\end{Rem}

\begin{Prop} \label{prop:rep-tensor} Let $\mathscr{A}$ be an
  $I$-partial bialgebra. Then  $\Corep_{\rcf}(\mathscr{A})$ carries the
  structure of strict tensor category such that the product of rcfd corepresentations $(V,\mathscr{X})$ and
  $(W,\mathscr{Y})$ is the corepresentation $(V\itimes
  W,\mathscr{X}\Circt \mathscr{Y})$, the unit is the trivial
  corepresentation $(\C^{(I)},\mathscr{U})$, and the forgetful functor
  $\Corep_{\rcf}(\mathscr{A}) \to \Vecti$ is a strict tensor functor.
\end{Prop}
\begin{proof}
This is clear.
\end{proof}


% Change notation from right duals to left duals !
Given an rcfd corepresentation of a partial Hopf algebra, one can use the
antipode to define a contragredient corepresentation on a dual space.
Denote the dual of vector spaces $V$ and  linear maps $T$ by
$\dual{V}$ and $\dualop{T}$, respectively, and define the dual of an
$I^{2}$-graded vector space $V=\bigoplus_{k,l} \Gru{V}{k}{l}$ to be
the space
\begin{align*}
  \dual{V}=\bigoplus_{k,l} \Gru{(\dual{V})}{k}{l}, \quad \text{where }
\Gru{(\dual{V})}{k}{l} = \dual{(\Gru{V}{l}{k})}.
\end{align*}


%Recall that an object $X$ in a strict tensor category is called a
%\emph{right dual} of an object $Y$ and $Y$ is called a \emph{left
 % dual} of $X$, if there are morphisms $X \otimes Y \to 1$ and $1 \to
%Y \otimes X$, where $1$ denotes the tensor unit, such
%that the obvious compositions
 % \begin{gather*}
 %X \otimes 1 \to X\otimes Y\otimes X
 % \to 1 \otimes X  \quad\text{and} \quad
  %   1 \otimes Y \to Y \otimes X \otimes Y \to Y
   % \otimes 1 
  %\end{gather*}
 % are the identity of $X$ and $Y$,
 % respectively.  


\begin{Prop}
  Let $\mathscr{A}$ be an $I$-partial Hopf algebra with antipode $S$
  and  let $(V,\mathscr{X})$ be an rcfd
  corepresentation of $\mathscr{A}$. Then $\dual{V}$ and the family
  $\dualco{\mathscr{X}}$ given by
   \begin{align} \label{eq:rep-left-dual}
\Gr{\dualco{X}}{k}{l}{m}{n}   :=  (S \otimes \dualop{-})(\Gr{X}{n}{m}{l}{k}) 
   \end{align} 
   form an rcfd corepresentation of $\mathscr{A}$  which is a left dual of $(V,\mathscr{X})$. If the antipode
   $S$ of $\mathscr{A}$ is bijective, then $\dual{V}$ and the family
   $\dualcor{\mathscr{X}}$ given by 
   \begin{align} \label{eq:rep-right-dual}
 \Gr{\dualcor{X}}{k}{l}{m}{n} :=(S^{-1}
   \otimes \dualop{-})(\Gr{X}{n}{m}{l}{k})    
   \end{align}
 form an rcfd corepresentation
 of $\mathscr{A}$ which is a
   right dual of $(V,\mathscr{X})$.
  \end{Prop}
  \begin{proof}
    We only prove the assertion concerning
    $(\dual{V},\dualco{\mathscr{X}})$. To see that this is a corepresentation, note that the element
    \eqref{eq:rep-left-dual} belongs to $\Gr{A}{k}{l}{m}{n} \otimes
    \Hom_{\C}(\Gru{(\dual{V})}{m}{n},\Gru{(\dual{V})}{k}{l})$ and use
    the relations $\Delta \circ S = (S \otimes S)\Delta^{\op}$ and
    $\epsilon \circ S = \epsilon$ from Corollary
    \ref{corollary:antipode} and Lemma \ref{LemCoAnt}.  
    Let us show that $(\dual{V},\dualco{\mathscr{X}})$ is a left dual
    of $(V,\mathscr{X})$.

    Given a finite-dimensional vector space $W$, denote by $\ev_{W}
    \colon \dual{W} \otimes W \to \C$ the evaluation map and by $\coev_{W}
    \colon \C \to W \otimes \dual{W}$ the coevaluation map, given by
    $1\mapsto \sum_{i} w_{i} \otimes \dual{w_{i}}$ if $(w_{i})_{i}$
    and $(\dual{w_{i}})_{i}$ are dual bases of $W$ and
    $\dual{W}$. With respect to these maps, $\dual{W}$ is a left dual
    of $W$. If $F\colon W_{1}\to W_{2}$ is a linear map between
    finite-dimensional spaces, then
\begin{align} \label{eq:coev-vee} (\id_{W_{2}} \otimes F^{\tr}) \circ \coev_{W_{2}} &= (F \otimes \id_{W_{1}^{*}})\circ
  \coev_{W_{1}}, &
\ev_{W_{1}}(F^{\tr}
  \otimes \id_{W_{2}})&=  \ev_{W_{2}}(\id_{W_{2}^{*}} \otimes F).
\end{align}

Now, define morphisms $\coev \colon \C^{(I)} \to V\itimes \dual{V}$ and
$\ev \colon \dual{V} \itimes V \to \C^{(I)}$ by
\begin{align*}
  \Gru{\coev}{k}{l} &= \delta_{k,l} \sum_{p} \coev_{\footnotesize\Gru{V}{k}{p}} \colon
  \C \to 
    \Gru{(}{k}{}V\itimes \dual{V}\Gru{)}{}{l}, &
  \Gru{\ev}{k}{l} &= \delta_{k,l} \sum_{p} \ev_{\Gru{V}{p}{k}} \colon
    \Gru{(}{k}{}V\itimes \dual{V}\Gru{)}{}{l} \to \C.
\end{align*}
One easily checks that with respect to these maps, $\dual{V}$ is a
left dual of $V$ in $\Vectrcf$. 

We therefore only need to show that $\ev$ is a morphism from
$\dualco{\mathscr{X}}\Circt\mathscr{X}$ to $\mathscr{U}$ and that $\coev$ is
a morphism from $\mathscr{U}$ to
$\mathscr{X}\Circt\dualco{\mathscr{X}}$.  But \eqref{eq:coev-vee} and
Lemma \ref{lemma:rep-invertible} imply
  \begin{align*}
    (1\otimes \Gru{\ev}{k}{k})
 \sum_{l,n}  \big(
\Gr{\dualco{X}}{k}{l}{m}{n}\big)_{12}
\big(\Gr{X}{l}{k}{n}{q}\big)_{13} &=
    (1\otimes \Gru{\ev}{k}{k})
 \sum_{l,n} 
(S \otimes \dualop{-})(\Gr{X}{n}{m}{l}{k})_{12}
    (\Gr{X}{l}{k}{n}{q})_{13} \\ &=
(1\otimes \Gru{\ev}{m}{m})  \sum_{l,n}
      (S \otimes \id)(\Gr{X}{n}{m}{l}{k})_{13}(\Gr{X}{l}{k}{n}{q})_{13} \\
    &= \delta_{m,q}\UnitC{k}{q}\otimes \Gru{\ev}{m}{m} \\
    &= \Gr{U}{k}{k}{m}{q}(1 \otimes \Gru{\ev}{m}{m}).
  \end{align*}
A similar  calculation shows that also $\coev$ is a morphism, whence the claim follows.

%\begin{align*}
 %\sum_{l,n}  \big(
%\Gr{X}{k}{l}{m}{n}\big)_{12}
%\big(\Gr{(\dual{X})}{l}{p}{n}{m}\big)_{13} 
 %   (1\otimes \Gru{\coev}{m}{m})
%&= \delta_{k,p} \UnitC{k}{m} \otimes \Gru{\coev}{k}{k} = (1 \otimes
%\Gru{\coev}{k}{k}) \Gr{U}{k}{p}{m}{m},
%\end{align*}
%whence the claim follows.
\end{proof}
\begin{Cor} \label{cor:rep-tensor-duality}
  Let $\mathscr{A}$ be a partial Hopf algebra. Then
  $\Corep_{\rcf}(\mathscr{A})$ is a tensor category with left
  duals and, if the antipode of $\mathscr{A}$ is invertible, with right duals.
\end{Cor}

Let $\mathscr{A}$ be an $I$-partial Hopf algebra.  Then the tensor
unit in $\Corep_{\rcf}(\mathscr{A})$, which is the trivial corepresentation
$\mathscr{U}$ on $\C^{(I)}$, need not be irreducible. Instead, it decomposes
into irreducible corepresentations indexed by the hyperobject set $\mathscr{I}$ of equivalence
classes for the relation $\sim$ on $I$ given by  $k \sim l \iff
  \UnitC{k}{l}\neq 0$ (see Remark
\ref{remark:index-equivalence}).
\begin{Lem}
  Let $\mathscr{A}$ be an $I$-partial Hopf algebra and let
  $(I_{\alpha})_{\alpha\in \mathscr{I}}$ be a labelled partition of $I$ into
  equivalence classes for the relation $\sim$.  Then for each $\alpha\in \mathscr{I}$, the subspace
  $\C^{(I_{\alpha})} \subseteq \C^{(I)}$ is invariant, and the restriction
  $\mathscr{U_{\alpha}}$ of $\mathscr{U}$ to $\C^{(I_{\alpha})}$ is
  irreducible. In particular, $\mathscr{U}=\bigoplus_{\alpha\in\mathscr{I}}
  \mathscr{U_{\alpha}}$ is a decomposition into irreducible corepresentations.
\end{Lem}
\begin{proof}
Immediate from the fact that $\Gr{U}{k}{k}{m}{m} = 
  \UnitC{k}{m}$  is $1$  if $k\sim m$  and $0$ if $k\not\sim m$. 
\end{proof}

\begin{Def} We denote by $\Corep(\mathscr{A})$ the category of rcfd corepresentations $(V,\mathscr{X})$ for which there exists a finite subset of the hyperobject set $\mathscr{I}$ such that $\Gru{V}{k}{l}=0$ for the equivalence classes of $k,l$ outside this subset.
\end{Def}

It is easily seen that $\Corep(\mathscr{A})$ is then a tensor category with local units indexed by $\mathscr{I}$. We will use the same notation for the associated partial tensor category. 

%The decomposition of the tensor unit leads to a decomposition of the
%whole tensor category into full subcategories, where the tensor
%product acts like the multiplication in a partial algebra.


\subsection{Decomposition into irreducible corepresentations}



When there is an invariant integral around, one can average morphisms of vector spaces to obtain morphisms of corepresentations. 
\begin{Lem} \label{lem:rep-average}  Let $(V,\mathscr{X})$ and
  $(W,\mathscr{Y})$ be rcfd corepresentations of  a partial
  Hopf algebra $\mathscr{A}$ with an invariant integral $\phi$, and let
  $\Gru{T}{k}{l} \in \Hom_{\C}(\Gru{V}{k}{l},\Gru{W}{k}{l})$ for all $k,l\in I$. Then for each $m,n$ fixed, the families
  \begin{align*}
    \Gr{\check T}{m}{n}{k}{l} &:= (\phi \otimes
    \id)(\Gr{(Y^{-1})}{n}{m}{l}{k}(1\otimes
    \Gru{T}{m}{n})\Gr{X}{m}{n}{k}{l}), \\
    \Gr{\hat T}{m}{n}{k}{l} &:=(\phi \otimes
    \id)(\Gr{Y}{k}{l}{m}{n}(1\otimes
    \Gru{T}{m}{n})\Gr{(X^{-1})}{l}{k}{n}{m})
  \end{align*}
  % think it's ok to keep this notation as we did not apply the previous hat and check to operators...
form  morphisms $\Grd{\check{T}}{m}{n}$ and $\Grd{\hat{T}}{m}{n}$ from $(V,\mathscr{X})$ to $(W,\mathscr{Y})$. % Slightly changed the averaging so that I do not need a restriction of finite support on $T$.
\end{Lem} 
\begin{proof} Clearly, we may suppose that $T$ is supported only on the component at index $(m,n)$, and we may then drop the upper indices and simply write $\Gru{\check{T}}{k}{l}$ and $\Gru{\hat{T}}{k}{l}$. Then 
 in total form, $\check{T}=(\phi \otimes \id)(Y^{-1}(1 \otimes T)X)$
  and $\hat{T}=(\phi \otimes \id)(Y(1 \otimes T)X^{-1})$.  Now, Lemma
  \ref{lemma:rep-multiplier} and Lemma \ref{lemma:total-integral} 
  imply
  \begin{align*}
    Y^{-1}(1 \otimes \check{T})X &= (\phi \otimes \id \otimes
    \id)((Y^{-1})_{23}(Y^{-1})_{13}(1 \otimes 1
    \otimes T)X_{13}X_{23})  \\
    &= ((\phi \otimes\id)  \Delta  \otimes \id)(Y^{-1}(1 \otimes T)X) \\
    &= \sum_{l} \rho_{l} \otimes (\phi \otimes \id)((\rho_{l} \otimes
    1)Y^{-1}(1 \otimes T)X)  \\
    &= \sum_{k,l} \rho_{l} \otimes \Gru{\check T}{k}{l},
  \end{align*}
  whence $\check{T}$ is a morphism from $\mathscr{X}$ to $\mathscr{Y}$
  by Remark \ref{remark:rep-total-morphism}. The assertion for $\hat
  T$ follows similarly.
\end{proof}

% Choose a representative family of unitary irreducible locally finite
% corepresentations
% $(\Grd{\mathcal{H}}{(\alpha)}{},{_{(\alpha)}X})_{\alpha}$ and a basis
% $(\Gr{\zeta}{k}{l}{(\alpha)}{i})_{i}$ for each
% $\Gr{\mathcal{H}}{k}{l}{(\alpha)}{}$, and let
% \begin{align*}
%   (\Gr{(u_{\alpha})}{k}{l}{m}{n}){i,j} &:= (\id \otimes
%   \Gr{\omega}{k}{l}{(\alpha)}{i,j})( )
% \end{align*}

\begin{Lem}
  Let $\mathscr{A}$ be an $I$-partial Hopf algebra with an invariant integral $\phi$.
  Let $(V,\mathscr{X})$ be an rcfd corepresentation
  and $\Gru{W}{k}{l} \subseteq \Gru{V}{k}{l}$ an invariant family of
  subspaces. Then there exists an idempotent endomorphism $T$ of
  $(V,\mathscr{X})$ such that $\Gru{W}{k}{l}=\img\Gru{T}{k}{l}$ for
  all $k,l$.
\end{Lem}
\begin{proof}
By a direct sum decomposition, we may assume that $V$ is in a fixed component $\Corep(\mathscr{A})_{\alpha\beta}$. For all $k\in I_{\alpha},l\in I_{\beta}$, choose idempotent endomorphisms $\Gru{T}{k}{l}$ of $\Gru{V}{k}{l}$
  with image $\Gru{W}{k}{l}$. By Lemma \ref{lem:rep-average}, we obtain
  endomorphisms $\Grd{\check{T}}{m}{n}$ of $(V,\mathscr{X})$. We want to show
  that linear combinations of these provide the sought-after morphism. %$\img \Gru{\check{T}}{k}{l}= \Gru{W}{k}{l}$. 
  
    In
  total form, invariance of $W$ implies  \[(1 \otimes T)X(1
  \otimes T)=X(1\otimes T).\] Applying
 $(S \otimes \id)$, we get   \[(1 \otimes T)X^{-1}(1
  \otimes T)=X^{-1}(1\otimes T).\]
Now choose $n\in I_{\beta}$ and write $\Grd{\check{T}}{}{n} = \sum_m \Grd{\check{T}}{m}{n}$, which makes sense because of column-finiteness of $V$. We combine  Lemma
  \ref{lemma:rep-multiplier}, Lemma \ref{lemma:rep-invertible} and
  normalisation of $\phi$, and find
  \begin{align*}
    \Grd{\check{T}}{}{n} T &= (\phi \otimes \id)(X^{-1}(1 \otimes
    \rho_{n}^{V}T)X(1 \otimes T)) \\  &= 
     (\phi \otimes \id)(X^{-1}(1 \otimes
    \rho_{n}^{V})X(1 \otimes T)) \\
    &=
  \sum_l \phi(\UnitC{n}{l}) \rho^{V}_{l}T \\& =T,
  \end{align*}
 since we only have to sum over $l\in I_{\beta}$ and $n\in \mathscr{I}_{\beta}$ by assumption. 
 
 Now as $W$ is invariant and $T$ sends $V$ into $W$, we have that $\Gr{\check{T}}{}{n}{k}{l}$ sends $\Gru{V}{k}{l}$ into $\Gru{W}{k}{l}$. Hence it follows that $\img{\check{T}^{n}}=\img T$, and $\check{T}^{n}$ is the sought-after intertwiner.

  %  Similarly, $T\check{T}_{n} =T$. Therefore,
  
\end{proof}
% Make the notion of cosemisimplicity formal, show equivalence existence invariant functionals and cosemisimplicity.
\begin{Cor}  \label{cor:rep-cosemisimple}% Changed! See if later refs are still ok.
  Let $\mathscr{A}$ be a partial Hopf algebra with an invariant integral.  Then
  every rcfd corepresentation of $\mathscr{A}$ decomposes into a (possibly infinite) direct
  sum of irreducible rcfd corepresentations.
\end{Cor} 
\begin{proof} 
The preceding lemma shows that  every non-zero corepresentation is either
irreducible or the direct sum of two non-zero corepresentations, and we can apply Zorn's lemma.
\end{proof}

% Trivial rep should maybe be introduced in a more conspicuous place

% Semi-simplicity requires that each element is a finite direct sum! But possibly this is convention

We can now prove that the category $\Corep(\mathscr{A})$ of a partial Hopf algebra with invariant integral is semisimple, that is, any object is a finite direct sum of irreducible objects. If one allows a more relaxed definition of semisimplicity allowing infinite direct sums, this will be true also for the a potentially bigger category $\Corep_{\rcf}(\mathscr{A})$.

We will first state a lemma which will also be convenient at other occasions.

\begin{Lem}\label{LemInjMor}  Let $\mathscr{A}$ be a partial Hopf algebra with an invariant integral, and fix $\alpha,\beta$ in the hyperobject set.  Then if $T$ is a morphism in $\Corep(\mathscr{A})_{\alpha\beta}$ and $\sum_{k\in I_\alpha} \Gru{T}{k}{l}=0$ for some $l \in I_\beta$, then $T=0$.
\end{Lem} 

\begin{proof} This follows from the equations in Remark \ref{remark:rep-total-morphism}
\end{proof}

\begin{Prop}\label{prop:rep-cosemisimple} Let $\mathscr{A}$ be a partial Hopf algebra with an invariant integral.   Then the components of the partial tensor category $\Corep(\mathscr{A})$ are semisimple.
\end{Prop}
\begin{proof} 

%\texttt{Put that into the first or second subsection}
%Let us now first show that the trivial representation decomposes into irreducibles. Let $I$ be the object set of $\mathscr{A}$, and say $k\sim l$ if $\UnitC{k}{l}\neq 0$. Then $\sim$ is an equivalence relation: as \[\Delta_{ll}(\UnitC{k}{m}) = \UnitC{k}{l}\otimes \UnitC{l}{m},\] the relation $\sim$ is transitive. As $S(\UnitC{k}{l}) = \UnitC{l}{k}$, we have that $\sim$ is symmetric. And as $\varepsilon(\UnitC{k}{k})=1$, we also have that $\sim$ is reflexive. 

%Let then $I = \sqcup_{\alpha\in \mathscr{I}} I_{\alpha}$ be a labeled partition associated to $\sim$. Define $\C_{I_{\alpha}}\subseteq \C_I$ as the linear span of the homogeneous components with index in $\alpha$. It is clear then that the $\C_{I_{\alpha}}$ are invariant and irreducible.

%Consider now a general corepresentation $(X,\Hsp)$. Let $\Grd{\Hsp}{\alpha}{\beta}$ be the closed linear span of the homogeneous components with index in $\alpha\times \beta$. As we can identify \[\Grd{\Hsp}{\alpha}{\beta} \cong \C_{I_{\alpha}}\,\Circt\, \Hsp\,\Circt\, \C_{I_{\beta}},\] we see that $\Grd{\Hsp}{\alpha}{\beta}$ is an invariant subspace of $\Hsp$. Hence we may as well suppose that $\Hsp = \Grd{\Hsp}{\alpha}{\beta}$. 

Let $V$ be in any object of $\Corep(\mathscr{A})_{\alpha\beta}$ for $\alpha,\beta\in \mathscr{I}$.  From the previous lemma, we see that for $T$ a morphism in $\Corep(\mathscr{A})_{\alpha\beta}$, the map $T\mapsto \sum_{k\in I_\alpha} \Gru{T}{k}{l}$ is injective for any choice of $l\in I_\beta$. It follows by column-finiteness of $V$ that the algebra of self-intertwiners of $V$ is finite-dimensional. We then immediately conclude from the previous corollary that $V$ is a finite direct sum of irreducible invariant subspaces.
%But let then $T$ be a bounded self-intertwiner of $\Hsp$.
\end{proof} 


%Another corollary is the following. % Needed? Or can this better be proven directly in our case later on?

% Reference to be added!
%\begin{Cor} \label{cor:rep-irreducible-bidual}
 % Let $\mathscr{A}$ be a partial Hopf algebra with an invariant integral. Then
 % every irreducible rcf corepresentation $(V,\mathscr{X})$ of
 % $\mathscr{A}$ is equivalent to its right bidual
 % $(V,\dual{\dual{\mathscr{X}}{}\!})$.
%\end{Cor}
%\begin{proof} In any tensor category, a right dual of an object $X$ has $X$ as its left dual. The corollary then follows as in every semi-simple tensor
%  category, left and right duals are isomorphic \cite{}. 
%\end{proof}

\subsection{Matrix coefficients of irreducible corepresentations}

Our next goal is to obtain the analogue of Schur's orthogonality
relations for matrix coefficients of corepresentations.

Given finite-dimensional vector spaces $V$ and $W$, the dual space of
$\Hom_{\C}(V,W)$ is linearly spanned by functionals of the form
\begin{align*}
  \omega_{f,v} \colon \Hom_{\C}(V,W) \to \C, \quad T \mapsto  (f|Tv),
\end{align*}
where $v\in V$, $f\in \dual{W}$, and $(-|-)$ denotes the natural
pairing of $\dual{W}$ with $W$.
\begin{Def} Let $\mathscr{A}$ be a partial bialgebra. The space of
  \emph{matrix coefficients} $\mathcal{C}(\mathscr{X})$ of an rcfd
  corepresentation $(V,\mathscr{X})$ is the sum of the subspaces
\begin{align*}
  \Gr{\mathcal{C}(\mathscr{X})}{k}{l}{m}{n} &= \Span \left\{ (\id \otimes
    \omega_{f,v})(\Gr{X}{k}{l}{m}{n}) \mid v\in \Gru{V}{m}{n}, f \in
    \dual{(\Gru{V}{k}{l})} \right\} \subseteq \Gr{A}{k}{l}{m}{n}.
\end{align*}
\end{Def}
Let $(V,\mathscr{X})$ be  an rcfd corepresentation of a partial bialgebra
$\mathscr{A}$.  Condition \eqref{eq:rep-comultiplication} in Definition \ref{definition:corep}
implies
\begin{align} \label{eq:rep-matrix-delta}
  \Delta_{pq}(\Gr{\mathcal{C}(\mathscr{X})}{k}{l}{m}{n}) \subseteq
  \Gr{\mathcal{C}(\mathscr{X})}{k}{l}{p}{q} \otimes
  \Gr{\mathcal{C}(\mathscr{X})}{p}{q}{m}{n}.
\end{align}
Thus, the $\Gr{\mathcal{C}(\mathscr{X})}{k}{l}{m}{n}$ form a partial
coalgebra with respect to $\Delta$ and $\epsilon$.  Moreover, for each
$k,l$, the $I^{2}$-graded vector  space
\begin{align*}
  \Grd{\mathcal{C}(\mathscr{X})}{k}{l}:=\bigoplus_{m,n }
  \Gr{\mathcal{C}(\mathscr{X})}{k}{l}{m}{n}
\end{align*}
is rcfd, and the inclusion above shows that one can
form the regular corepresentation on this space.
\begin{Lem} \label{lemma:rep-regular-embedding}
  Let $(V,\mathscr{X})$ be an rcfd corepresentation
  of a partial bialgebra and let $f\in
  \dual{(\Gru{V}{k}{l})}$. Then the family of maps
  \begin{align*}
    \Gr{T}{}{}{m}{n(f)} \colon \Gru{V}{m}{n} \to
    \Gr{\mathcal{C}(\mathscr{X})}{k}{l}{m}{n}, \ w \mapsto (\id
    \otimes \omega_{f,w})(\Gr{X}{k}{l}{m}{n})=(\id \otimes
    f)(\Gr{X}{k}{l}{m}{n}(1 \otimes w)),
  \end{align*}
  is a morphism from $\mathscr{X}$ to the regular corepresentation on
  $\Grd{\mathcal{C}(\mathscr{X})}{k}{l}$.
\end{Lem}
\begin{proof}
  Denote by $\mathscr{Y}$ the regular corepresentation on
  $\bigoplus_{m,n } \Gr{\mathcal{C}(\mathscr{X})}{k}{l}{m}{n}$. Then
  \begin{align*}
    \label{eq:1}
 \Gr{Y}{p}{q}{m}{n}    (1\otimes \Gr{T}{}{}{m}{n(f)}(v)) &= 
(\Delta^{\op}_{pq} \otimes \omega_{f,v})( \Gr{X}{k}{l}{m}{n}) 
\\ & = (\id \otimes \id \otimes
f)((\Gr{X}{k}{l}{p}{q})_{23}(\Gr{X}{p}{q}{m}{n})_{13}(1 \otimes 1
 \otimes v)) \\ &=(1 \otimes \Gr{T}{}{}{p}{q(f)})\Gr{X}{p}{q}{m}{n}(1 \otimes v)
  \end{align*}
for all $v \in \Gru{V}{m}{n}$.
\end{proof}
\begin{Prop} \label{prop:rep-weak-pw} Let $\mathscr{A}$ be a partial
  Hopf algebra with an invariant integral. Then the total algebra $A$ is the sum
  of the matrix coefficients of irreducible rcfd corepresentations.
\end{Prop}
\begin{proof} 
  Let $a \in \Gr{A}{k}{l}{m}{n}$. Write
  \begin{align*}
    \Delta_{pq}(a)=\sum_{i} b_{pq}^{i} \otimes c^{i}_{pq}
  \end{align*}
 with linearly independent
  $(c_{pq}^{i})_{i}$. Then the family of subspaces
  \begin{align*}
    \Gru{V}{p}{q} = \mathrm{span}\{b_{pq}^{i} : i \}\subseteq \bigoplus_{k,l}
  \Gr{A}{k}{l}{p}{q}
  \end{align*}
is rcfd, and the relation
  \begin{align*}
 \sum_{i}
    \Delta_{rs}(b^{i}_{pq}) \otimes c^{i}_{pq} =
    (\Delta_{rs} \otimes \id)\Delta_{pq}(a) = (\id \otimes
    \Delta_{pq}) \Delta_{rs}(a) = \sum_{j} b^{j}_{rs} \otimes
    \Delta_{pq}(c^{j}_{rs})
  \end{align*}
  implies $\Delta_{rs}(\Gru{V}{p}{q}) \subseteq \Gru{V}{r}{s} \otimes
  \Gr{A}{r}{s}{p}{q}$.  We can therefore form the regular
  corepresentation $\mathscr{X}$ on $V$ as in Example \ref{example:rep-regular}, and
  \begin{align*}
    a = (\id \otimes \epsilon)(\Delta^{\op}_{kl}(a)) =
    (\id \otimes \epsilon)(\Gr{X}{k}{l}{m}{n}(1 \otimes a)) \in
    \Gr{\mathcal{C}(\mathscr{X})}{k}{l}{m}{n}.
  \end{align*}
  Decomposing $(V,\mathscr{X})$, we find that
  $a$ is contained in the sum of matrix coefficients of irreducible
rcfd  corepresentations.
\end{proof}


The first part of the orthogonality relations concerns matrix
coefficients of inequivalent irreducible corepresentations. 
\begin{Prop} \label{prop:rep-orthogonality-1} Let $\mathcal{A}$ be a
  partial Hopf algebra with an invariant integral $\phi$ and inequivalent
  irreducible rcfd corepresentations $(V,\mathscr{X})$ and
  $(W,\mathscr{Y})$.  Then  for all
  $a\in \mathcal{C}(X), b \in \mathcal{C}(Y)$,
  \[\phi(S(b)a) = \phi(bS(a))=0.\]
\end{Prop}
\begin{proof}
Since $\phi$ vanishes on $S(\Gr{A}{k}{l}{m}{n})\Gr{A}{p}{q}{r}{s}$ and
on $\Gr{A}{p}{q}{r}{s}S(\Gr{A}{k}{l}{m}{n})$ unless
$(p,q,r,s) = (m,n,k,l)$, it suffices to prove the assertion for  elements of the form
\begin{align*}
  a&=(\id \otimes \omega_{f,v})(\Gr{X}{k}{l}{m}{n})  && \text{and} &
  b&=(\id \otimes \omega_{g,w})(\Gr{Y}{m}{n}{k}{l})
\end{align*}
where $f\in \dual{(\Gru{V}{k}{l})}, v \in \Gru{V}{m}{n}$ and $g \in
\dual{(\Gru{W}{m}{n})}, w \in \Gru{W}{k}{l}$.  Lemma
\ref{lem:rep-average}, applied to the family
  \begin{align*}
    \Gru{T}{p}{q} \colon \Gru{V}{p}{q} \to \Gru{W}{p}{q}, \quad u
    \mapsto  \delta_{p,k}\delta_{q,l}  f(u)w,
  \end{align*}
  yields morphisms $\Grd{\check{T}}{k}{l},\Grd{\hat{T}}{k}{l}$ from $(V,\mathscr{X})$ to
  $(W,\mathscr{Y})$ which necessarily are $0$. Inserting the
  definition of $\Grd{\check{T}}{k}{l}$, we find
  \begin{align*}
    \phi(S(b)a) &= \phi\big((S \otimes
    \omega_{g,w})(\Gr{Y}{m}{n}{k}{l}) \cdot (\id \otimes
    \omega_{f,v})(\Gr{X}{k}{l}{m}{n})\big) \\ &= (\phi \otimes \omega_{g,v})\left(\Gr{(Y^{-1})}{l}{k}{n}{m}(1 \otimes
      \Gru{T}{k}{l} )     \Gr{X}{k}{l}{m}{n}\right) 
    = \omega_{g,v}( \Gr{\check{T}}{k}{l}{m}{n}) = 0.
  \end{align*}% Resort notation on using leg notation, physics bra-ket, etc.
  
  A similar calculation involving $\hat{T}$ shows that
  $\phi(bS(a))=0$.  
\end{proof}

From now on, we will assume that our partial Hopf algebra is regular (i.e.~ has bijective antipode). %Remark that this is in fact automatic in presence of invariant functional.

\begin{Theorem} \label{thm:rep-orthogonality} Let $\mathcal{A}$ be a
  regular partial Hopf algebra with an invariant integral $\phi$. Let $\alpha,\beta\in \mathscr{I}$, and let $(V,\mathscr{X})$
  be an irreducible rcfd corepresentation of $\mathscr{A}$ inside $\Corep(\mathscr{A})_{\alpha\beta}$. Suppose
  $F_{\mathscr{X}}=F$ is an isomorphism from $(V,\mathscr{X})$ to
  $(V,\hat{\hat{\mathscr{X}}})$ with inverse
  $G_{\mathscr{X}}= G$. Then the following hold.
  \begin{enumerate}[label=(\arabic*)]
  \item The numbers $d_G:=\sum_{k} \Tr (\Gru{G}{k}{l})$ and $d_F:=\sum_{n} \Tr (\Gru{F}{m}{n})$ do not depend on the choice of $l \in I_\beta$ or $m\in I_\alpha$.
    \item  For all $k,m \in I_\alpha$ and $l,n\in I_\beta$,
    \begin{align*}
      (\phi \otimes \id)(\Gr{(X^{-1})}{l}{k}{n}{m}\Gr{X}{k}{l}{m}{n})
      &=d_G^{-1}\Tr(\Gru{G}{k}{l})
      \id_{\Gru{V}{m}{n}}, \\
      (\phi \otimes \id)(\Gr{X}{k}{l}{m}{n}\Gr{(X^{-1})}{l}{k}{n}{m})
      &=d_F^{-1}\Tr(\Gru{F}{m}{n})
      \id_{\Gru{V}{k}{l}}.
    \end{align*}
  \item Denote by $\Sigma_{klmn}$ the flip map $\Gru{V}{k}{l}
    \otimes \Gru{V}{m}{n} \to \Gru{V}{m}{n}
    \otimes \Gru{V}{k}{l}$. Then
 \begin{align*}
   (\phi \otimes \id \otimes
   \id)((\Gr{(X^{-1})}{l}{k}{n}{m})_{12}(\Gr{X}{k}{l}{m}{n})_{13}) &=
   d_G^{-1}
   (\id_{\Gru{V}{m}{n}} \otimes \Gru{G}{k}{l})
   \circ \Sigma_{klmn}, \\
   (\phi \otimes \id \otimes
   \id)((\Gr{X}{k}{l}{m}{n})_{13}(\Gr{(X^{-1})}{l}{k}{n}{m})_{12}) &= d_F^{-1} (\Gru{F}{m}{n}
   \otimes \id_{\Gru{V}{k}{l}}) \circ \Sigma_{klmn}.
 \end{align*}
\end{enumerate}
  \end{Theorem}
\begin{proof}
  We prove the assertions and equations involving $d_G$ in (1), (2)
  and (3)  simultaneously; the assertions involving $d_F$  follow similarly.

  %As above, we denote by $\Sigma_{p,q,r,s}$ the flip
  %$\Gru{\mathcal{H}}{p}{q} \otimes \Gru{\mathcal{H}}{r}{s} \to
  %\Gru{\mathcal{H}}{r}{s} \otimes \Gru{\mathcal{H}}{p}{q}$.  
  Consider
  the following endomorphism $F_{m,n,k,l}$ of $\Gru{V}{m}{n}\otimes \Gru{V}{k}{l}$, 
  \begin{align*}
    F_{m,n,k,l}
    &:=(\phi \otimes \id \otimes \id)\left((\Gr{(X^{-1})}{l}{k}{n}{m})_{12}(\Gr{X}{k}{l}{m}{n})_{13}\right)
    \circ \Sigma_{mnkl} \\ &= (\phi \otimes \id \otimes
    \id)\left((\Gr{(X^{-1})}{m}{n}{k}{l})_{12}
      \Sigma_{klkl,23}(\Gr{X}{k}{l}{m}{n})_{12}\right).
  \end{align*}
  By applying Lemma \ref{lem:rep-average} with respect to the flip map $\Sigma_{klkl}$, we see that the family $(F_{m,n,k,l})_{m,n}$ is
  an endomorphism of $(V \otimes \Gru{V}{k}{l}, X\otimes \id)$ and hence
  \begin{align}
    F_{m,n,k,l} &= \id_{\Gru{V}{m}{n}} \otimes \Gru{R}{k}{l} \label{eq:rep-orthogonal-1}
  \end{align}
  with some $\Gru{R}{k}{l} \in \Hom_{\C}(\Gru{V}{k}{l})$ not
  depending on $m,n$. % In using irreducibility, do we not miss any subtlety in allowing non-bounded morphisms?
  On the other hand, since $\phi = \phi S$,
  \begin{align*}
    F_{m,n,k,l} &= (\phi \otimes \id \otimes \id)((S \otimes
    \id)(\Gr{X}{m}{n}{k}{l})_{12}(\Gr{X}{k}{l}{m}{n})_{13})
    \circ \Sigma_{mnkl} \\
    &= (\phi \otimes \id \otimes \id)\left(((S \otimes
      \id)(\Gr{X}{k}{l}{m}{n}))_{13}
      ((S^{2} \otimes \id)(\Gr{X}{m}{n}{k}{l}))_{12}\right)     \circ \Sigma_{mnkl}\\
    &= (\phi \otimes \id \otimes
    \id)\left((\Gr{(X^{-1})}{k}{l}{m}{n})_{13} (\Sigma_{mnmn})_{23}
      (\Gr{(\dual{\dual{X}{}\!})}{m}{n}{k}{l})_{13}\right).
  \end{align*}
  Hence we can again apply Lemma \ref{lem:rep-average} and
  find that the family $(F_{m,n,k,l})_{k,l}$ is a morphism \[(F_{m,n,k,l})_{k,l}:
  (\Gru{V}{m}{n} \otimes V, \hat{\hat{X}}_{13})\rightarrow (\Gru{V}{m}{n} \otimes V,
 X_{13}).\] Therefore,
  \begin{align}
    F_{m,n,k,l} &= \Gru{T}{m}{n} \otimes \Gr{G}{k}{l}{}{\mathscr{X}} \label{eq:rep-orthogonal-2}
  \end{align}
  with some $\Gru{T}{m}{n} \in \mathcal{\Hom_{\C}}(\Gru{V}{m}{n})$
  not depending on $k,l$. Combining \eqref{eq:rep-orthogonal-1} and
  \eqref{eq:rep-orthogonal-2}, we conclude that, for some $\lambda\in \C$, \[F_{m,n,k,l} = \lambda
  (\id_{\Gru{V}{m}{n}} \otimes \Gr{G}{k}{l}{}{\mathscr{X}})\]
  
  Choose dual  bases
  $(v_{i})_{i}$ for $\Gru{V}{k}{l}$ and $(f_{i})_{i}$ for  $\dual{(\Gru{V}{k}{l})}$. Then
  \begin{align*}
    \lambda   \Tr( \Gr{G}{k}{l}{}{\mathscr{X}}) \id_{\Gru{V}{m}{n}}
 &= \sum_{i} (\id \otimes
    \omega_{f_{i},v_{i}})(F_{m,n,k,l}) = (\phi \otimes
    \id)((\Gr{(X^{-1})}{l}{k}{n}{m}) \Gr{X}{k}{l}{m}{n}).
  \end{align*}
  Take now $n=l$.  By Lemma \ref{LemInjMor}, we can choose $m\in I_{\alpha}$ with $\Gru{V}{m}{n}\neq 0$.   Then summing the previous relation over $k$, the relations $\sum_{k}
  (\Gr{(X^{-1})}{l}{k}{n}{m}) \Gr{X}{k}{l}{m}{n} = \UnitC{l}{n}
  \otimes \id_{\Gru{V}{m}{n}}$ and
  $\phi(\UnitC{l}{l})=1$ give
\begin{align*}
\lambda \cdot  \sum_{k} \Tr(\Gr{G}{k}{l}{}{\mathscr{X}}) = 1.  %Problem if $\Gru{V}{m}{n}=0$!
\end{align*}
Now all assertions in (1)--(3) concerning $d_G$ follow.
 % the second formula, we use the first formula for the
 %    opposite of $(A,\Delta)$. For this opposite, $\phi$ still is a
 %    faithful, positive, normalized invariant functional and
 %    $(\mathcal{H},X)$ still is a unitary irreducible locally finite
 %    corepresentation, but the antipode $S$ gets replaced by $S^{-1}$
 %    and therefore $F_{X}$ gets replaced by $F_{X}^{-1}$.
\end{proof}

\begin{Rem} For semi-simple tensor categories with duals, it is known that any object is isomorphic to its left bidual, hence there always exists an $F_{\mathscr{X}}$ as in the previous Theorem. In fact, from the faithfulness of $\phi$ and Proposition \ref{prop:rep-orthogonality-1}, it follows that not all $F_{m,n,k,l}$ in the previous proof are zero. Hence $G_{\mathscr{X}}$ is a non-zero morphism and thus an isomorphism from the left bidual of $\mathscr{X}$ to $\mathscr{X}$.  
\end{Rem}

\begin{Cor}\label{CorOrth}
  Let $\mathscr{A}$ be a regular partial Hopf algebra with an invariant integral $\phi$, let
  $(V,\mathscr{X})$ be an irreducible rcfd corepresentation of
  $\mathscr{A}$, let $F_{\mathscr{X}}$ be an isomorphism from
  $(V,\mathscr{X})$ to $(V,\dualco{\dualco{\mathscr{X}}})$ and
  $G_{\mathscr{X}}=F^{-1}_{{\mathscr{X}}}$, and let $a=(\id \otimes
  \omega_{f,v})(\Gr{X}{k}{l}{m}{n})$ and $b=(\id \otimes
  \omega_{g,w})(\Gr{X}{m}{n}{k}{l})$, where 
  $f \in   \dual{(\Gru{V}{k}{l})}$, $v \in\Gru{V}{m}{n}$, $g \in
  \dual{(\Gru{V}{m}{n})}$, $w \in  \Gru{V}{k}{l}$.  Then
\begin{align*}
  \phi(S(b)a) &= \frac{(g|v)(f|G_{\mathscr{X}}w)}{\sum_{r}
    \Tr(\Gr{G}{r}{n}{}{\mathscr{X}})}, & \phi(aS(b)) = \frac{(g|F_{\mathscr{X}}v)(f|w)}{\sum_{s}
    \Tr(\Gr{F}{m}{s}{}{\mathscr{X}})}.
\end{align*}
\end{Cor}
\begin{proof}
Apply $\omega_{g,w} \otimes
    \omega_{f,v}$ to the formulas in  Theorem
    \ref{thm:rep-orthogonality}.(c).
\end{proof}
\begin{Cor} \label{cor:rep-pw}
  Let $\mathscr{A}$ be a partial Hopf algebra with an invariant integral and let
  $((V_{i},\mathscr{X}_{i}))_{i \in \mathcal{I}}$ be a maximal family of mutually non-isomorphic irreducible rcfd corepresentations of
  $\mathscr{A}$. Then the map
  \begin{align*}
    \bigoplus_{i \in \mathcal{I}} \bigoplus_{k,l,m,n}
    (\dual{(\Gr{V}{k}{l}{}{i})} \otimes
    \Gr{V}{m}{n}{}{i}) \to A
  \end{align*}
  that sends $f \otimes w \in
  \dual{(\Gr{V}{k}{l}{}{\alpha})} \otimes
  \Gr{V}{m}{n}{}{i}$ to $ (\id \otimes
  \omega_{f,w})(\Gr{(X_{i})}{k}{l}{m}{n})$,
  is a linear isomorphism. 
\end{Cor}
\begin{proof} This follows from Proposition \ref{prop:rep-weak-pw}, Proposition \ref{prop:rep-orthogonality-1} and Corollary \ref{CorOrth}.
\end{proof}
\begin{Cor} \label{cor:rep-pw-morphisms}
  Let $\mathscr{A}$ be a regular partial Hopf algebra with an invariant integral, let
  $((V_{i},\mathscr{X}_{i}))_{i\in \mathcal{I}}$ be a maximal
  family of mutually non-isomorphic irreducible rcfd corepresentations of $\mathscr{A}$,
  fix $i \in \mathcal{I}$ and $k,l\in I$, and denote by $\Gr{\mathscr{Y}}{k}{l}{}{i}$
  the regular corepresentation on
  $\Grd{\mathcal{C}(\mathscr{X}_i)}{k}{l}$. Then there exists a
  linear isomorphism
  \begin{align*}
    \dual{( \Gru{V}{k}{l})} \to
    \Mor((V_{i},\mathscr{X}_{i}),
    (\Grd{\mathcal{C}(\mathscr{X}_i)}{k}{l},\Gr{\mathscr{Y}}{k}{l}{}{i}))
  \end{align*}
  assigning to each $f\in     \dual{( \Gru{V}{k}{l})}$ the morphism
  $T_{(f)}$ of Lemma \ref{lemma:rep-regular-embedding}.
\end{Cor}



\subsection{Unitary corepresentations of partial compact quantum groups}


Let us now enhance our partial Hopf algebras to partial compact
quantum groups. We write $B(\Hsp,\mathcal{G})$ for the linear space of
bounded morphisms between Hilbert spaces $\Hsp$ and $\mathcal{G}$. 
%To be consistent in terminology, we mean by representation of a partial compact quantum group a corepresentation of its associated partial Hopf $^*$-algebra.


%One then considers corepresentations on 

%rcf bigraded
%\emph{Hilbert spaces} such that the inverse of the corepresentation
%coincides with its adjoint. More precisely, we have the following
%definition. 
%Unitary corepresentations act on $I^{2}$-graded Hilbert spaces $\Hsp =
%\bigoplus_{k,l} \Gru{\Hsp}{k}{l}$ which are row- and column-finite,
%where the sum is a direct sum of Hilbert spaces. Associated to each
%such $I^{2}$-graded Hilbert space is the $I^{2}$-graded vector space
%which is obtained by taking the algebraic direct sum of the
%components. 

\begin{Def} Let $\mathscr{A}$ define a partial compact quantum
  group. We call an rcfd corepresentation $\mathscr{X}$ of $\mathscr{A}$ on a collection of Hilbert spaces $\Gru{\Hsp}{k}{l}$ 
   \emph{unitary}
  if \[\Gr{(X^{-1})}{n}{m}{l}{k}=(\Gr{X}{l}{k}{n}{m})^{*}\quad
  \textrm{in }\Gr{A}{k}{l}{m}{n}\otimes
  B(\Gru{\Hsp}{l}{k},\Gru{\Hsp}{n}{m}).\]
\end{Def} 


\begin{Rem} %\begin{enumerate} \item 
The total object $\Hsp$ will then only be a pre-Hilbert space, but as the local components are finite-dimensional, this will not be an issue.
%In the Hilbert space setting, it is more natural to let the total object $\Hsp$ be the \emph{closed} (instead of the purely algebraic) direct sum of all (finite-dimensional) $\Gru{\Hsp}{k}{l}$. This does not change the notion of corepresentation, which had a local definition.
%\item Concerning morphisms, we will say a collection of $\Gru{T}{k}{l}$ defines a \emph{bounded} intertwiner or morphism if the total operator $T= \oplus \Gru{T}{k}{l}$ is bounded. We will denote by $\Corep_{u}(\mathscr{A})$ the category of unitary rcf corepresentations with arbitrary morphisms, and $\Corep_{u}^{\infty}(\mathscr{A})$ for the category with bounded morphisms.
% Give a more prominent place, or check later if it is actually worthwhile to make this distinction.
%\end{enumerate}
\end{Rem}

\begin{Exa}\label{example:rep-trivial-unitary}
  Regard $\C^{(I)}$ as a direct sum of the trivial Hilbert spaces $\C$. Then the
  trivial corepresentation $\mathscr{U}$ on $\C^{(I)}$ is unitary.
\end{Exa}

The tensor product of rcfd corepresentations lifts to a tensor product
of unitary corepresentations as follows.  We define the tensor product
of rcfd $I^{2}$-graded Hilbert spaces similarly as for rcfd
$I^{2}$-graded vector spaces and pretend it to be strict again.
\begin{Lem}\label{lemma:rep-unitary-tensor}
  Let $(\Hsp,\mathscr{X})$ and $(\mathcal{G},\mathscr{Y})$ be unitary
  rcfd representations associated to a partial compact quantum group. Then the
  tensor product $(\Hsp \itimes \mathcal{G},\mathscr{X} \Circt
  \mathscr{Y})$ is unitary again.
\end{Lem}
\begin{proof}
In total form,  $(X\Circt Y)^{-1} = Y_{13}^{-1}X_{12}^{-1}
  =Y_{13}^{*}X_{12}^{*} = (X \Circt Y)^{*}$ by Remark \ref{remark:rep-tensor-multiplier}.
\end{proof}

We hence obtain a tensor C$^*$-category $\Corep_{u,\rcf}(\mathscr{A})$ of unitary rcfd corepresentations. We denote again by $\Corep_u(\mathscr{A})$ the subcategory of all corepresentations with finite support on the hyperobject set. It is the total tensor C$^*$-category with local units of a semisimple partial tensor C$^*$-category.

Our aim now is to show that every (irreducible) rcfd corepresentation is
equivalent to a unitary one. We show this by embedding the
corepresentation into a restriction of the regular corepresentation.
\begin{Lem} \label{lemma:rep-regular-unitary}
  Let $\mathscr{A}$ define a partial compact quantum group with
positive invariant  integral $\phi$, and let $\Gru{V}{m}{n} \subseteq
\bigoplus_{k,l} \Gr{A}{k}{l}{m}{n}$ be subspaces such that
$\Delta_{pq}(\Gru{V}{m}{n}) \subseteq \Gru{V}{p}{q} \otimes
    \Gr{A}{p}{q}{m}{n}$ and $V=\bigoplus_{k,l} \Gru{V}{k}{l}$ is rcfd. Then each $\Gru{V}{k}{l}$ is a Hilbert space with
    respect to the inner product given by $\langle
    a|b\rangle:=\phi(a^{*}b)$, and the regular corepresentation
    $\mathscr{X}$ on $V$ is unitary.
\end{Lem}
\begin{proof} 
By Lemma \ref{lemma:rep-invertible},  it suffices to show that
  \begin{equation}\label{EqUnit} \sum_{k}
    (\Gr{X}{k}{l}{m}{n'})^* \Gr{X}{k}{l}{m}{n} =
    \delta_{n,n'}\UnitC{l}{n}\otimes
    \id_{\Gru{\Hsp}{m}{n}}.
  \end{equation} 
Let  $a\in \Gru{\Hsp}{m}{n}$, $b\in \Gru{\Hsp}{m}{n'}$ and define $\omega_{b,a} \colon
\Hom_{\C}(\Gru{\Hsp}{m}{n},\Gru{\Hsp}{m}{n}) \to \C$ by $T
\mapsto \langle b|Ta\rangle$. Then
\begin{eqnarray*}
\sum_{k }(\id \otimes \omega_{b,a})
((\Gr{X}{k}{l}{m}{n'})^* \Gr{X}{k}{l}{m}{n}))  &=& \sum_k
(\id\otimes \phi)(\Delta_{kl}^{\op}(b)^*\Delta_{kl}^{\op}(a))\\
  &=& \sum_k (\phi\otimes
  \id)(\Delta_{lk}(b^*)\Delta_{kl}(a)) \\ &=& (\phi\otimes
  \id)(\Delta_{ll}(b^*a)) \\ &=& \phi(b^*a)\UnitC{l}{n} \\&=&
  \delta_{n',n} \UnitC{l}{n} \otimes \langle b|a\rangle.
\end{eqnarray*} 
This proves \eqref{EqUnit}.
\end{proof} 

\begin{Prop} \label{prop:rep-unitarisable} Every  rcfd
  corepresentation of a partial compact quantum group $\mathscr{A}$ is
  isomorphic to a unitary rcfd corepresentation.
\end{Prop}
\begin{proof}
  By Proposition \ref{prop:rep-cosemisimple}, it suffices to prove the
  assertion for every rcfd corepresentation $(V,\mathscr{X})$ that is
  irreducible.  For some $k,l$ and $f \in
  \dual{(\Gru{V}{k}{l})}$, the operator $T_{(f)}$ defined in
  Lemma \ref{lemma:rep-regular-embedding} has to be non-zero and
  hence, by Schur's Lemma, injective. Thus, it forms an equivalence
  between $(V,\mathscr{X})$ and a restriction of the regular
  corepresentation on $\Grd{\mathcal{C}(\mathscr{X})}{k}{l}$, which is
  unitary by Lemma \ref{lemma:rep-regular-unitary}.
\end{proof}
%This result and Proposition \ref{prop:rep-cosemisimple}
%imply that the category $\Corep_{u}(\mathscr{A})$ is semisimple:
\begin{Cor} The partial C$^*$-tensor category $\Corep_u(\mathscr{A})$ is a partial fusion category.
\end{Cor}

\begin{Rem}If $\mathscr{A}$ defines a partial compact quantum group $\mathscr{G}$, we will also write $\Corep_u(\mathscr{A})= \Rep_u(\mathscr{G})$, and talk of (unitary) rcfd representations of $\mathscr{G}$.
 \end{Rem}  
   

%With the evident definition on morphisms, we obtain a tensor product
%on $\Corep_{u}(\mathscr{A})$.  This tensor
%category is \emph{rigid} in the sense that every object has a
%left and a right dual:
%\begin{Cor}
 % Let $\mathscr{A}$ define a partial compact quantum group. Then the
 % category $\Corep_{u}(\mathscr{A})$ is a rigid tensor category.
%\end{Cor}
%\begin{proof}
  %Since the antipode of $\mathscr{A}$ is invertible (Corollary
  %\ref{cor:involutive}), every unitary rcf corepresentation
  %$(\Hsp,\mathscr{X})$ has a left and a right dual in
  %$\Corep(\mathscr{A})$ (Corollary \ref{cor:rep-tensor-duality}),
  %and these are equivalent isomorphic to unitary rcf corepresentations
  %(Proposition \ref{prop:rep-unitarisable}).
%\end{proof}
%Note that in $\Corep(\mathscr{A})$, the left dual of a unitary rcfd
%corepresentation $(\Hsp,\mathscr{X})$ is given by the $I^{2}$-graded
%vector space
%$\dual{\Hsp}$  and the corepresentation multiplier
%\begin{align} \label{eq:rep-unitary-right-dual}
 % (S\otimes -^{\vee})(X) = (\id \otimes -^{\vee})(X^{-1}) )=
  %(-^{*}\otimes -^{*\vee})(X).
%\end{align}
%By Corollary \ref{cor:rep-irreducible-bidual},
%$(\mathcal{H},\mathscr{X})$ is isomorphic to the right bidual, which
%is given by the $\Hsp$ and $(S^{2} \otimes \id)(X)$. This isomorphism can be
%chosen to be positive, as the next proposition shows.

Let now $\mathscr{X}$ be a unitary corepresentation of $\mathscr{A}$. Then there exists an isomorphism from $\mathscr{X}$ to $\dualco{\dualco{\mathscr{X}}} = (S^2\otimes \id)\mathscr{X}$. The following proposition shows that it can be implemented by positive operators.

\begin{Prop} \label{prop:rep-unitary-bidual}
  Let $\mathscr{A}$ define a partial compact quantum group and let
  $(\Hsp,\mathscr{X})$ be an irreducible unitary rcfd corepresentation of
  $\mathscr{A}$.  Then there exists an isomorphism $F=F_{\mathscr{X}}$
  from $(\Hsp,\mathscr{X})$ to 
  $(\Hsp,(S^{2} \otimes \id)(\mathscr{X}))$ in $\Corep(\mathscr{A})$ such
  that each $\Gru{F}{k}{l}$ is positive.
\end{Prop}
\begin{proof}
 By Proposition \ref{prop:rep-unitarisable}, there exists an
  isomorphism $T \colon \dualco{\mathscr{X}} \to \mathscr{Y}$ for some
  unitary rcfd corepresentation $\mathscr{Y}$ on $\dual{\Hsp}$, so that in total form,
  $(1\otimes T)\dualco{X} = Y(1 \otimes T)$.
We  apply   $S \otimes -^{\tr}$ and $-^{*} \otimes -^{*\tr}$,
respectively to find 
\begin{align*}
 \dualco{\dualco{X}}(1 \otimes \dualop{T}) &= (1 \otimes
  \dualop{T})\dualco{Y}, & (1 \otimes T^{*\tr})X=\dualco{Y}(1\otimes T^{*\tr}).
\end{align*}
%Here, we identify the the dual of a Hilbert space with its conjugate
%Hilbert space to make sense of $T^{*\tr}$.  
Combining both equations, we
find $\dualco{\dualco{X}}(1 \otimes \dualop{T}T^{*\tr})=(1 \otimes
\dualop{T}T^{*\tr})X$. Thus, we can take
$F_{\mathscr{X}}:=\dualop{T}T^{*\tr}$.
\end{proof}

The Schur orthogonality relations in Corollary \ref{CorOrth} can be
rewritten using the involution instead of the antipode as follows.
Let $(\Hsp,\mathscr{X})$ be a unitary rcfd corepresentation of
$\mathscr{A}$. Since $(S\otimes \id)(X)=X^{-1}=X^{*}$, the space of
matrix coefficients $\mathcal{C}(\mathscr{X})$ satisfies
\begin{align} \label{eq:rep-unitary-matrix-coefficients}
  S(\Gr{\mathcal{C}(\mathscr{X})}{k}{l}{m}{n}) &=
  (\Gr{\mathcal{C}(\mathscr{X})}{m}{n}{k}{l})^{*} \subseteq \Gr{A}{n}{m}{l}{k}.
\end{align}
More precisely, let $v \in \Gru{\Hsp}{k}{l}$, $v' \in \Gru{\Hsp}{m}{n}$
and denote by $\omega_{v,v'}$ the functional
given by $T \mapsto \langle v|Tv'\rangle$. Then
\begin{align*}
  S((\id \otimes \omega_{v,v'})(\Gr{X}{k}{l}{m}{n})) &=
  (\id \otimes \omega_{v,v'}) (\Gr{(X^{-1})}{n}{m}{l}{k})) \\ & =
  (\id \otimes \omega_{v,v'})( (\Gr{X}{m}{n}{k}{l})^{*}) =
  (\id \otimes \omega_{v',v})(\Gr{X}{m}{n}{k}{l})^{*}.
\end{align*}
This equation and Corollary \ref{CorOrth} imply the following corollary.
\begin{Cor}\label{cor:rep-unitary-schur-orthogonality}
  Let $\mathscr{A}$ define a partial compact quantum group with
  invertible integral $\phi$, let $(\Hsp,\mathscr{X})$ be an irreducible
  unitary rcfd corepresentation of $\mathscr{A}$, let $F_{\mathscr{X}}$ be a positive
  isomorphism from $(\Hsp,\mathscr{X})$ to
  $(\Hsp,\dualco{\dualco{\mathscr{X}}})$ and
  $G_{\mathscr{X}}=F^{-1}_{{\mathscr{X}}}$, and let $a=(\id \otimes
  \omega_{v,v'})(\Gr{X}{k}{l}{m}{n})$ and $b=(\id \otimes
  \omega_{w,w'})(\Gr{X}{k}{l}{m}{n})$, where $v,w \in
  \Gru{\Hsp}{k}{l}$ and $v',w' \in \Gru{\Hsp}{m}{n}$.  Then
\begin{align*}
  \phi(b^{*}a) &= \frac{\langle w|v'\rangle\langle v|G_{\mathscr{X}}w'\rangle}{\sum_{r}
    \Tr(\Gr{G}{r}{n}{}{\mathscr{X}})}, & \phi(ab^{*}) = \frac{\langle
    w|F_{\mathscr{X}}v'\rangle \langle v|w'\rangle}{\sum_{s}
    \Tr(\Gr{F}{m}{s}{}{\mathscr{X}})}.
\end{align*}
\end{Cor}
As a consequence of Proposition \ref{prop:rep-weak-pw} and Proposition
\ref{prop:rep-unitarisable} or Lemma \ref{lemma:rep-regular-unitary},
the matrix coefficients of irreducible unitary rcfd corepresentations
span $\mathscr{A}$, and in the Corollary \ref{cor:rep-pw}, we may
assume the irreducible rcfd corepresentations
$(V_{i},\mathscr{X}_{i})$ to be unitary if $\mathscr{A}$
defines a partial compact quantum group.

\begin{Rem}\label{RemPos} In fact, Proposition \ref{prop:rep-unitary-bidual} and Corollary \ref{cor:rep-unitary-schur-orthogonality} show the following. Let $\mathscr{A}$ be a partial Hopf $^*$-algebra admitting an invariant integral $\phi$, which a priori we do not assume to be positive. Suppose however that each irreducible corepresentation of $\mathscr{A}$ is equivalent to a unitary corepresentation. Then $\phi$ is necessarily positive.
\end{Rem} 

\subsection{Analogues of Woronowicz's  characters}

% We already need positivity here to define $F^z$!

Let $\mathscr{A}$ be a partial bialgebra, and $a\in \Gr{A}{k}{l}{m}{n}$. Then for $\omega \in \Hom_{\C}(A,\C)$, we can define
\begin{align*}
  \omega \aste{p,q} a
&:= (\id \otimes \omega) (\Delta_{pq}(a)), & a \aste{r,s}
\omega&:=(\omega \otimes \id)(\Delta_{rs}(a)).\end{align*} Clearly we can define
\begin{align*} \omega \aste{p,q} a \aste{r,s}
\omega'&:= (\omega \aste{p,q} a)\aste{r,s} \omega' = \omega \aste{p,q}(a \aste{r,s} \omega').\end{align*}
When $\omega$ has support on the $A(K)$ with $K_u=K_d$, we can write, for $a\in \Gr{A}{k}{l}{m}{n}$, \[\omega\ast a := \sum_{p,q} \omega\aste{p,q}a = \omega\aste{m,n}a,\quad  a\ast \omega = \sum_{r,s} a\aste{r,s}\omega = a\aste{k,l}\omega.\] 

We shall say that an entire function $f$ has \emph{exponential growth
  on the right half-plane} if there exist $C,d>0$ such that $|f(x+iy)|\leq
C\mathrm{e}^{dx}$  for all $x,y\in \R$ with $x>0$. 

\begin{Theorem} \label{thm:rep-characters} Let $\mathscr{A}$ be a partial Hopf $^*$-algebra with positive invariant integral $\phi$.  Then there exists a unique
  family of linear functionals $f_{z} \colon A\to \C$ such that
\begin{enumerate}[label={(\arabic*)}]
  \item $f_z$ vanishes on $A(K)$ when $K_u\neq K_d$.
  \item for each $a\in A$, the function $z\mapsto f_{z}(a)$ is entire
    and of exponential growth on the right half-plane.
  \item $f_{0} = \epsilon$ and $(f_{z} \otimes f_{z'}) \circ 
    \Delta= f_{z+z'}$ for all $z,z' \in \C$.
  \item $\phi(ab)=\phi(b(f_{1} \ast a \ast f_{1}))$ for all $a,b\in A$.
  \end{enumerate}
  This family furthermore satisfies
  \begin{enumerate}[label={(\arabic*)}]\setcounter{enumi}{4}
  \item $f_z(ab) = f_z(a)f_z(b)$ for $a\in A(K)$ and $b\in A(L)$ with $K_r = L_l$. 
  \item $S^{2}(a)=f_{-1} \ast a \ast f_{1}$ for all $a\in A$.
  \item $f_{z}(\UnitC{l}{n})=\delta_{l,n}$ and $f_{z} \circ S = f_{-z}$ for all $a\in A$.
\end{enumerate}
\end{Theorem}


Note that conditions (3), (4) and (6) are meaningful by condition (1).

\begin{proof}
  We first prove uniqueness.  Assume that $(f_{z})_{z}$ is a family of
  functionals satisfying (1)--(4).  Since $\phi$ is faithful, the map
  $\sigma\colon a \mapsto f_{1} \ast a \ast f_{1}$ is uniquely
  determined by $\phi$, and one easily sees that it is a homomorphism. Using
  (3), we find that $\epsilon \circ \sigma^n=f_{2n}$, which uniquely determines these functionals. Using (2) and the
  fact that every entire function of exponential growth on the right
  half-plane is uniquely determined by its values at $\N \subseteq \C$, we can conclude that the family $f_{z}$ is uniquely determined. Moreover, since the property (5) holds for $z = 2n$, we also conclude by the same argument as above that it holds for all $z\in \C$.

  Let us now prove existence.  By Theorem \ref{thm:rep-orthogonality}, Corollary \ref{cor:rep-pw} and Proposition \ref{prop:rep-unitary-bidual}, we can
  define for each $z\in \C$ a functional $f_{z} \colon A \to \C$ such
  that for every 
  %$\alpha,\beta$ in the hyperobject set $\mathscr{I}$ and each 
  irreducible rcfd corepresentation
  $(V,\mathscr{X})$ in $\Corep_{u}(\mathscr{A})$,
    \begin{align*}
      f_{z}((\id \otimes \omega_{\xi,\eta})(\Gr{X}{k}{l}{m}{n})) &=
      \delta_{k,m}\delta_{l,n} \cdot
      \omega_{\xi,\eta}((\Gr{F}{k}{l}{}{\mathscr{X}})^{z}) \quad \text{for all }
      \xi \in \Gru{V}{k}{l},\eta \in
      \Gru{V}{m}{n},
    \end{align*}
    or, equivalently,
    \begin{align*}
      (f_{z} \otimes \id)(\Gr{X}{k}{l}{m}{n}) =
      \delta_{k,m}\delta_{l,n} \cdot (\Gru{(F_{\mathscr{X}})}{k}{l})^{z},
    \end{align*}
    where $F_{\mathscr{X}}$ is a non-zero positive operator implementing a morphism from $(V,\mathscr{X})$ to
    $(V, \dualco{\dualco{\mathscr{X}}})$, scaled such that
    \begin{align*}
      d_{\mathscr{X}}:= \sum_{r} \Tr(\Gru{(F_{\mathscr{X}}^{-1})}{r}{l}) = \sum_{s}
      \Tr(\Gru{(F_{\mathscr{X}})}{m}{s})
    \end{align*}
    for all $l$ in the right and all $m$ in the left hyperobject support of $\mathscr{X}$. By
    construction, (1) and (2) hold. We show that the $(f_{z})_{z}$ satisfy the
    assertions (3)--(7). 
    %We have already argued that (5) is satisfied
    %$f_{z}$ is a character. 
    Throughout the following arguments, let 
    $(V,\mathscr{X})$ be a unitary irreducible corepresentation
    $(V,\mathscr{X})$ and let $F=F_{\mathscr{X}}$ be as above.

    We first prove property (3). This follows from the relations
    \begin{align*}
      (f_{0}  \otimes \id)(\Gr{X}{k}{l}{m}{n}) &=
      \delta_{k,m}\delta_{l,n} \id_{\Gru{V}{k}{l}} =
      (\epsilon \otimes \id)(\Gr{X}{k}{l}{m}{n})
    \end{align*}
    and
    \begin{align*}
      (((f_{z}\otimes f_{z'})\circ \Delta) \otimes
      \id)(\Gr{X}{k}{l}{m}{n}) &=  \delta_{k,m}\delta_{l,n}(f_{z} \otimes f_{z'} \otimes
      \id)\big((\Gr{X}{k}{l}{k}{l})_{13}
      (\Gr{X}{k}{l}{k}{l})_{23}\big) \\
      &=  \delta_{k,m}\delta_{l,n}(\Gru{F}{k}{l})^{z}  \cdot (\Gru{F}{k}{l})^{z'} \\
      &= (f_{z+z'} \otimes \id)(\Gr{X}{k}{l}{m}{n}).
    \end{align*}
    Applying slice maps of the form $\id
    \otimes \omega_{\xi,\xi'}$ and invoking Theorem \ref{thm:rep-orthogonality}, this proves (3).

% Again? Check if this has already been used before   
    To prove (4), write again $ \Delta^{(2)} = (
    \Delta \otimes \id)\circ  \Delta = (\id \otimes 
    \Delta) \circ \Delta$, and put \[\theta_{z,z'}:=(f_{z'} \otimes \id
    \otimes f_{z})\circ  \Delta^{(2)}.\] Then
    \begin{align*}
      (\theta_{z,z'} \otimes \id)(\Gr{X}{k}{l}{m}{n}) &= (f_{z'} \otimes
      \id \otimes f_{z} \otimes
      \id)((\Gr{X}{k}{l}{k}{l})_{14}(\Gr{X}{k}{l}{m}{n})_{24}(\Gr{X}{m}{n}{m}{n})_{34})
      \\
      &= (1 \otimes (\Gru{F}{k}{l})^{z'}) \Gr{X}{k}{l}{m}{n} (1
      \otimes (\Gru{F}{m}{n})^{z}).
    \end{align*}
    We take $z=z'=1$, use Theorem \ref{thm:rep-orthogonality}, where
    now $d_F= d_G=d_{\mathscr{X}}$ by our scaling of $F$, and obtain
    \begin{eqnarray*}
     && \hspace{-2cm} (\phi \otimes \id \otimes
      \id)((\Gr{X}{k}{l}{m}{n})_{12}^{*}((\theta_{1,1} \otimes
      \id)(\Gr{X}{k}{l}{m}{n}))_{13})\\ && =d_{\mathscr{X}}^{-1}(\id \otimes
      \Gru{F}{k}{l}) (\id \otimes \Gru{(F^{-1})}{k}{l})
      \Sigma_{k,l,m,n} (\id \otimes
      \Gru{F}{m}{n}) \\
      &&=d_{\mathscr{X}}^{-1}(\Gru{F}{m}{n} \otimes \id) \Sigma_{k,l,m,n} \\
      &&= (\phi \otimes \id \otimes
      \id)((\Gr{X}{k}{l}{m}{n})_{13}(\Gr{X}{k}{l}{m}{n})_{12}^{*}).
    \end{eqnarray*}
    To conclude the proof of assertion (4), apply again slice maps of the form
    $\omega_{\xi,\xi'} \otimes \omega_{\eta,\eta'}$.

We have then already argued that the property (5) automatically holds. To show the property (6), note that by Proposition \ref{prop:rep-unitary-bidual} and the calculation above,
    \begin{align*}
      (S^{2} \otimes \id)(\Gr{X}{k}{l}{m}{n}) &= (1
      \otimes\Gru{F_{\mathscr{X}}}{k}{l})
      \Gr{X}{k}{l}{m}{n}(1 \otimes \Gru{F_{\mathscr{X}}}{m}{n})^{-1} 
      =(\theta_{-1,1}  \otimes \id)(\Gr{X}{k}{l}{m}{n}).
    \end{align*}
     Assertion (6) follows again by applying slice maps.
    
     Finally, (1), (2) and (4)
     immediately imply the relation
     $f_{z}(\UnitC{k}{m})=\delta_{k,m}$. As both $z \rightarrow f_{-z}$ and $z\rightarrow f_z\circ S$ satisfy the conditions (1)--(4) for $\mathscr{A}$ with the opposite product and coproduct (using the partial character property (5) and the invariance of $\phi$ with respect to $S$), we find $f_{-z} = f_{z} \circ S$
     
     %The concrete construction of $f_z$ combined with property (3), the identity \eqref{eq:rep-delta2} and the partial character property (5) gives the equality
     %\begin{align}
      % (f_{-z} \otimes \id) (\Gr{X}{k}{l}{k}{l})=
       %(\Gru{(F_{\mathscr{X}})}{k}{l})^{-z} &=\left( (f_{z} \otimes
       %\id)(\Gr{X}{k}{l}{k}{l})\right)^{-1} \\ &= (f_{z} \otimes
       %\id)(\Gr{(X^{-1})}{l}{k}{l}{k}) = ((f_{z} \circ S) \otimes
       %\id)(\Gr{X}{k}{l}{k}{l}).
     %\end{align}
%Therefore, .
\end{proof}
\begin{Cor} \label{cor:rep-characters} Let $\mathscr{A}$ be a 
  partial Hopf $^*$-algebra with positive invariant integral $\phi$ and define $\theta_{z,z'} \colon A
  \to A$ by $a \mapsto f_{z} \ast a \ast f_{z'}$ for each $z,z' \in
  \C$, where the functionals $f_{z}$ are as in Theorem
  \ref{thm:rep-characters}. Then for all $z,z',w,w'\in \C$, the
  following conditions hold.
  \begin{enumerate}[label={(\arabic*)}]
  \item $\theta_{z,z'}$ is an algebra automorphism and preserves
    each subspace $A(K)$. In particular,
    $\theta_{z,z'}(\lambda_{k}\rho_{m}) = \lambda_{k}\rho_{m}$ for all
    $k,m\in I$.
  \item $\theta_{z,z'}\circ \theta_{w,w'} = \theta_{z+w,z'+w'}$.
  \item $ (\theta_{w,z'} \otimes \theta_{z,-w}) \circ \Delta = \Delta
    \circ \theta_{z,z'}$, $\epsilon \circ \theta_{z,z'} = f_{z+z'}$,
    $\theta_{z,z'} \circ S = S \circ \theta_{-z',-z}$ and
    $\phi \circ \theta_{z,z'} = \phi$.
  \item For every linear map $\omega \colon A \to \C$ and every $a\in
    A$, the map $(z,z') \mapsto \omega(\theta_{z,z'}(a))$ is entire.
  \end{enumerate}
\end{Cor}
\begin{proof}
  All of this follows easily from Theorem \ref{thm:rep-characters}.
\end{proof}
Using the two-parameter group $\theta$, we define the \emph{modular
  automorphism group} $\sigma$, the \emph{scaling group} $\tau$   and
the \emph{unitary antipode} of a partial compact quantum group $A$ by
\begin{align} \label{eq:rep-groups}
  \sigma_{z} &:=\theta_{iz,iz}, & \tau_{z} &:=\theta_{iz,-iz}, & R&:=S
  \circ \tau_{i/2}.
\end{align}
Using Corollary \ref{cor:rep-characters}, one verifies that
$\sigma,\tau,R$ share all the main relations known for locally compact
quantum groups and measured quantum groupoids. For example, $\sigma$
and $\tau$ are complex one-parameter groups of algebra automorphisms
of $A$, the map $R$ is an anti-automorphism, the family  $\tau$ commutes with
$\sigma$ and with $R$ in the obvious sense, 
  \begin{gather}
    \begin{aligned} \label{eq:modular}
      \phi\circ \sigma_{z} &= \phi \circ \tau_{z} = \phi \circ R =
      \phi, & \phi(ab) &= \phi(b\sigma_{-i}(a)),
    \end{aligned}
\\ \label{eq:scaling-modular-delta}
    \begin{aligned} 
    \Delta \circ \tau_{z} &= (\tau_{z} \otimes \tau_{z}) \circ \Delta = (\sigma_{-z}\otimes \sigma_{z})\circ\Delta,
    & (\tau_{z} \otimes \sigma_{z}) \circ \Delta &= \Delta \circ
    \sigma_{z} = (\sigma_{-z} \otimes \tau_{z}) \circ \Delta,      
  \end{aligned} \\
  \begin{aligned} \label{eq:unitary-antipode}
    R^{2} &= \id_{A}, & \Delta \circ R &= (R \otimes R) \circ
    \Delta^{\op}.
  \end{aligned}
  \end{gather}

The relation with the $*$-structure is determined by the following proposition.

%Finally, we consider the functionals $f_{z}$, the automorphisms
%$\theta_{z,z'}$, $\tau_{z}$, $\sigma_{z}$ and the anti-isomorphism $R$
%introduced in Theorem \ref{thm:rep-characters}, Corollary
%\ref{cor:rep-characters} and \eqref{eq:rep-groups}.
\begin{Prop} \label{prop:rep-unitary-characters}  Let $\mathscr{A}$ be a 
  partial Hopf $^*$-algebra with positive invariant integral $\phi$.  Then $f_{z}(a^*) =
   \overline{f_{-\overline{z}}(a)}$ and $\theta_{z,w}(a^*) = \overline{\theta_{-\overline{z},-\overline{w}}(a)}$ for all $z,w\in \C$ and $a\in A$. In
  particular, $R$ is a $*$-anti-automorphism and $\theta_{it,is}$,
  $\tau_{t}$ and $\sigma_{t}$ are $*$-automorphisms for all $s,t\in
  \R$.
\end{Prop}
\begin{proof}
  We only have to prove the first equation.  Write $\bar{f}_z(a) =
  \overline{f_z(a^*)}$. Using the relations
  $ (\Gr{X}{k}{l}{k}{l})^{*}=(S \otimes \id)(\Gr{X}{k}{l}{k}{l})$,  $f_{z} \circ S=f_{-z}$
  (Theorem \ref{thm:rep-characters}) and
  positivity of $\Gr{F}{k}{l}{}{\mathscr{X}}$ (Proposition \ref{prop:rep-unitary-bidual}), we conclude
     \begin{align*}
       (\bar{f}_z \otimes
       \id)(\Gr{X}{k}{l}{k}{l})
&=       \left((f_{z} \otimes
       \id)((\Gr{X}{k}{l}{k}{l})^{*})\right)^{*} \\
& = \left((f_{-z} \otimes \id)(\Gr{X}{k}{l}{k}{l})\right)^{*} 
 =
((\Gr{F}{k}{l}{}{\mathscr{X}})^{-z})^{*} 
=       (\Gr{F}{k}{l}{}{\mathscr{X}})^{-\overline{z}} = (f_{-\overline{z}}
\otimes \id)(\Gr{X}{k}{l}{k}{l}),
     \end{align*}
whence $\bar{f}_z(a) = f_{-\overline{z}}(a)$ for all $a\in
\Gr{\mathcal{C}(\mathscr{X})}{k}{l}{k}{l}$. Since $f_{z}$ and
$f_{-\overline{z}}$ vanish on $\Gr{A}{k}{l}{m}{n}$ if $(k,l)\neq
(m,n)$ and the matrix coefficients of unitary 
corepresentations span $A$, we can conclude $\bar{f}_{z}=f_{-\overline{z}}$.
\end{proof}



%  consider
%   the family
%   \begin{align*}
%     (\Gru{F}{m}{n})^{\top} \circ  \overline{\Sigma_{m,n,k,l}} &=
% \phi((\Gr{X}{k}{l}{m}{n})_{12}^{*}(\Gr{X}{k}{l}{m}{n})_{13})^{\top}
% \circ  \overline{\Sigma_{m,n,k,l}} \\
%  &=
%  \phi((\Gr{X}{k}{l}{m}{n})_{12}^{*\circ (\id \otimes
%   \top)}(\overline{\Sigma_{k,l,k,l}})_{23} (\Gr{X}{k}{l}{m}{n})^{\id \otimes
%   \top}_{12})  \\
% &=\phi((\Gr{\overline{X}}{l}{k}{n}{m})_{12}(\overline{\Sigma_{k,l,k,l}})_{23}
%  (\Gr{\overline{X}}{l}{k}{n}{m})_{12}).
% \end{align*} \fxnote{Treat $X^{\id \otimes \top}$}
% By Lemma \ref{lem:rep-average}, this family is a morphism from to
% $(\overline{\mathcal{H}}\otimes \overline{K},\overline{X}^{-*} \otimes
% \id_{\overline{K}})$  to
% $(\overline{\mathcal{H}}\otimes \overline{K},\overline{X} \otimes
% \id_{\overline{K}})$ and hence of the form
% $(\Gru{\overline{F_{X}}}{n}{m})^{-1} \otimes T$ with $T \in
% \mathcal{B}(\overline{K})$ not depending on $m,n$.

% Thus,
% \begin{align*}
%  (\id \otimes R)  \circ \Sigma_{k,l,m,n} =   \Gru{F}{m}{n} =
%  \Sigma_{k,l,m,n} \circ (\Gru{\overline{F_{X}}}{n}{m})^{-\top} \otimes T^{\top})
% \end{align*}

  
%   We may assume $(k,l,m,n)=(p,q,r,s)$ because otherwise both sides of
%   the equation that we want to prove vanish.

%   Applying Lemma \ref{lem:rep-average} to the corepresentation $X$ and
%   the family $\Gru{T}{p}{q}=
%   \delta_{p,k}\delta_{q,l} |\eta\rangle\langle\xi|$, we obtain an
%   endomorphism $\check{T}$ of $(\mathcal{H},X)$ which
%   necessarily has the form $\check{T}=\lambda(\xi,\eta) \id$ for some
%   $\lambda(\xi,\eta) \in \C$. Inserting the definition of
%   $\check{T}$, we find
%   \begin{align}\nonumber
%     \phi(b^{*}a) &= \phi\big((\id \otimes
%     \omega_{\eta',\eta})((\Gr{X}{k}{l}{m}{n})^{*}) \cdot (\id \otimes
%     \omega_{\xi,\xi'})(\Gr{X}{k}{l}{m}{n})\big) \\  &= (\phi \otimes
%     \id)\left(\langle\eta'|_{2} \Gr{(X^{-1})}{m}{n}{k}{l}(1 \otimes
%       |\eta\rangle\langle \xi|)
%       \Gr{X}{k}{l}{m}{n}|\xi'\rangle_{2}\right) 
%     = \langle \eta'|_{2} \Gru{\check{T}}{m}{n}|\xi'\rangle_{2} =
%     \lambda(\xi,\eta) \langle\eta'|\xi'\rangle. \label{eq:rep-orthogonal-1}
%   \end{align}
%   Next, we apply Lemma \ref{lem:rep-average} to the corepresentations
%   $\overline{X}$ and $\overline{X}^{-*}$ and the family
%   $\Gru{R}{p}{q}=\delta_{p,m}\delta_{q,n}|\overline{
%     \xi'}\rangle\langle\overline{\eta'}|$, and obtain a morphism
%   $\hat{R}$ from $\overline{X}^{-*}$ to $\overline{X}$ which
%   necessarily has the form $\hat{R}=\mu(\eta',\xi')\overline{F_{X}}$
%   for some $\mu(\eta',\xi') \in \C$. Using the relation
%   \begin{align*}
%     a &= (\id \otimes
%     \omega_{\overline{\xi},\overline{\xi'}})(\Gr{\overline{X}}{l}{k}{n}{m})^{*},
%     & b&= (\id \otimes
%     \omega_{\overline{\eta},\overline{\eta'}})(\Gr{\overline{X}}{k}{l}{m}{n})^{*}
%   \end{align*}
%   and the definition of $\hat{R}$, we obtain
%   \begin{align}
%     \phi(b^{*}a) &= (\phi \otimes \id)\left(\langle
%       \overline{\eta}|_{2} \Gr{\overline{X}}{k}{l}{m}{n}(1 \otimes
%       |\overline{\eta}'\rangle\langle \overline{\xi'}|)
%       \Gr{(\overline{X}^{*})}{m}{n}{k}{l} \right) \nonumber \\
%     &=\langle \overline{\eta}| \Gru{\hat{R}}{k}{l}
%     |\overline{\xi}\rangle = \langle
%     \overline{\eta}|\Gru{\overline{F_{X}}}{k}{l}\overline{\xi}\rangle
%     \mu(\eta',\xi'). \label{eq:rep-orthogonal-2}
%   \end{align}
%   We choose a basis $(\zeta_{i})_{i}$ for
%   $\bigoplus_{k}\Gru{\mathcal{H}}{k}{l}$ and calculate
%   \begin{align*}
%  \langle
%     \eta'|\xi'\rangle &=  (\phi \otimes
%     \id)(\langle\eta'|_{2}(\lambda_{l}\rho_{n} \otimes
%     \id_{\Gru{\mathcal{H}}{m}{n}})|\xi'\rangle_{2}) \\
%  &=
%     \sum_{k} (\phi \otimes \id)\left(\langle\eta'|_{2}
%       \Gr{(X^{-1})}{m}{n}{k}{l}
%       \Gr{X}{k}{l}{m}{n}|\xi'\rangle_{2}\right)
%     \\ &=    \sum_{i} \lambda(\zeta_{i},\zeta_{i}) \langle
%     \eta'|\xi'\rangle 
%     \\
%     &=\sum_{i} \langle
%     \overline{\zeta_{i}}| \Gru{\overline{F_{X}}}{}{l}\overline{\zeta_{i}}\rangle
%     \mu(\eta',\xi') 
% \\ &    = \mu(\eta',\xi') \cdot \sum_{k} \Tr(\Gru{F_{(X)}}{k}{l}),
%   \end{align*}
% where $\Gru{\overline{F_{X}}}{}{l}=\bigoplus_{k}
% \Gru{\overline{F_{X}}}{k}{l}$. Inserting this relation into
% \eqref{eq:rep-orthogonal-2}, we finally obtain the assertion.


% We use here the standard leg numbering notation, e.g. $a_{12} =
% a\otimes 1$.


%%% Local Variables: 
%%% mode: latex
%%% TeX-master: "dyn-suq-main"
%%% End: 

\section{Tannaka-Krein-Woronowicz duality for partial compact quantum groups}

% Assume categories small?
% See `Galois reconstruction of finite quantum groups' of Bichon for additional refs
% Any faithfulness conditions necessary on forgetful functor? 

In the previous section, we showed how any partial compact quantum group gave rise to a partial semisimple tensor C$^*$-category with duality and indecomposable units, together with a morphism into a partial tensor C$^*$-category of finite dimensional Hilbert spaces. In this section, we reverse this construction, and show that the two structures are in duality with each other. As the proof does not differ too much from the usual Tannaka-Krein reconstruction process, we do not spell out all details, emphasizing only the novel points having to do with well-definedness of the constructions.  % Say that implicitly, we do two steps at once: create a partial discrete quantum group, and then dualize. But we don't formally introduce partial discrete quantum groups.

In the following, let us at first fix an indecomposable semi-simple partial tensor category $\CatCC$ over a base set $I_0$. We will again view the tensor product of $\CatC$ as being strict, for notational convenience. 

Assume that we also have another set $I$ and a partition $I = \{I_k\mid k\in I_0\}$ with associated function \[\varphi_0:I\rightarrow I_0, \quad k\mapsto k'.\] Let $\Forget: \CatCC\rightarrow \{\Vect_{\fin}\}_{I\times I}$ be a morphism based on $\varphi_0$ (cf.a~ Example \ref{ExaVectBiGr}).  We will again denote by $\Forget_{kl}:\CatCC_{k'l'}\rightarrow \Vect_{\fin}$ the components of $\Forget$ at index $(k,l)$, and by $\iota$ and $\mu$ resp.~ the associativity and unit constraints.  For $X\in \CatC_{k's}$ and $Y\in \CatC_{sm'}$, we write the projection maps associated to the identification $\Forget_{km}(X\otimes Y)\cong \oplus_{l\in I_s} \left(\Forget_{kl}(X)\otimes \Forget_{lm}(Y)\right)$ as \[\pi^{(klm)}_{X,Y}:\Forget_{km}(X\otimes Y) \rightarrow \Forget_{kl}(X)\otimes \Forget_{lm}(Y).\]

We choose a maximal family of mutually inequivalent irreducible objects $\{u_a\}_{a\in \mathcal{I}}$ in $\CatC$. We assume that the $u_a$ include the unit objects $\Unit_{k'}$ for $k'\in I_0$, and we may hence identify $I_0\subseteq \mathcal{I}$. For $a\in \mathcal{I}$, we will write $u_a \in \CatC_{s_a,t_a}$ with $s_a,t_a\in I_0$. For $s,t\in I_0$ fixed, we write $\mathcal{I}_{rs}$ for the set of all $a\in \mathcal{I}$ with $s_a=s$ and $t_a=t$. When $a,b,c\in \mathcal{I}$ with $a\in \mathcal{I}_{rs},b\in \mathcal{I}_{st}$ and $c\in \mathcal{I}_{rt}$, we write $c\leq a\cdot b$ if $\Mor(u_c,u_a\otimes u_b)\neq \{0\}$. Note that with $a,b$ fixed, there is only a finite set of $c$ with $c\leq a\cdot b$. We also use this notation for multiple products.

\begin{Def} For $a\in \mathcal{I}$ and $k,l,m,n\in I$, define vector spaces \[\Gr{A}{k}{l}{m}{n}(a) =  \delta_{k,m,s_a}\delta_{l,n,t_a} \Hom_{\C}(\Forget(u_a)_{mn},\Forget(u_a)_{kl})^*.\] Write \[\Gr{A}{k}{l}{m}{n} =\underset{a\in \mathcal{I}}{\oplus}\, \Gr{A}{k}{l}{m}{n}(a),\quad A(a) = \underset{k,l,m,n}{\oplus} \Gr{A}{k}{l}{m}{n}(a),\quad A = \underset{k,l,m,n}{\oplus} \Gr{A}{k}{l}{m}{n}.\] 
%For any element $x\in A$, its component in the $a$-spectral subspace is written $x_a$.
\end{Def} 

%Note that by construction, $\Gr{A}{k}{l}{m}{n}=0$ if $k'\neq m'$ or $l'\neq n'$. 

We first turn the $\Gr{A}{k}{l}{m}{n}$ into a partial coalgebra $\mathscr{A}$ over $I^2$.

\begin{Def} For $r,s\in I$, we define \[\Delta_{rs}: \Gr{A}{k}{l}{m}{n}\rightarrow \Gr{A}{k}{l}{r}{s}\otimes \Gr{A}{r}{s}{m}{n}\] as the direct sums of the duals of the composition maps \[\Hom_{\C}(\Forget(u_a)_{rs},\Forget(u_a)_{kl}) \otimes \Hom_{\C}(\Forget(u_a)_{mn},\Forget(u_a)_{rs})\rightarrow \Hom_{\C}(\Forget(u_a)_{mn},\Forget(u_a)_{kl}),\]\[x\otimes y \mapsto x\circ y.\]
\end{Def} 

\begin{Lem} The couple $(\mathscr{A},\Delta)$ is a partial coalgebra with counit map \[\epsilon:\Gr{A(a)}{k}{l}{k}{l}\rightarrow \C,\quad f\mapsto f(\id_{\Forget(u_a)_{kl}}).\] Moreover, for each fixed $f\in \Gr{A(a)}{k}{l}{m}{n}$, the matrix $\left(\Delta_{rs}(f)\right)_{rs}$ is row- and column-finite.
\end{Lem} 
\begin{proof} Coassociativity and counitality are immediate by duality, as for each $a$ fixed the $\Hom_{\C}(\Forget(u_a)_{mn},\Forget(u_a)_{kl})$ form a partial algebra with units $\id_{F(u_a)_{kl}}$. The row-and column-finiteness condition follows immediately from the local finiteness condition for the morphism $F$.
\end{proof}

In the next step, we define a partial algebra structure on $\mathscr{A} = \{\Gr{A}{k}{l}{m}{n}\mid k,l,m,n\}$. First note that we can identify \[\Nat(\Forget_{mn},\Forget_{kl}) \cong \underset{t_a=l'=n'}{\underset{s_a=k'=m'}{\prod_a}} \Hom_{\C}(\Forget(u_a)_{mn},\Forget(u_a)_{kl}),\] where $\Nat(\Forget_{mn},\Forget_{kl})$ denotes the space of natural transformations from $\Forget_{mn}$ to $\Forget_{kl}$ when $k'=m'$ and $l'=n'$. Similarly, we can identify \[\Nat(\Forget_{mn}\otimes \Forget_{pq},\Forget_{kl}\otimes \Forget_{rs}) \cong  \prod_{b,c} \Hom_{\C}(\Forget(u_b)_{mn}\otimes \Forget(u_c)_{pq} ,\Forget(u_b)_{kl}\otimes \Forget(u_c)_{rs}),\] with the product over the appropriate index set and where \[\Forget_{kl}\otimes \Forget_{rs}:\CatC_{k'l'}\times \CatC_{r's'}\rightarrow \Vect_{\fin},\quad (X,Y) \mapsto F_{kl}(X)\otimes F_{rs}(Y).\] As such, there is a natural pairing of these spaces with resp.~ $\Gr{A}{k}{l}{m}{n}$ and $\Gr{A}{k}{l}{m}{n}\otimes \Gr{A}{r}{s}{p}{q}$. 

% To expand. 

\begin{Def} For $k'=r', l'=s'$ and $m'=t'$, we define a product map \[M:\Gr{A}{k}{l}{r}{s} \otimes \Gr{A}{l}{m}{s}{t}\rightarrow  \Gr{A}{k}{m}{r}{t},\quad f\otimes g \mapsto f\cdot g\] by the formula \[(f\cdot g)(x) = (f\otimes g)( \hat{\Delta}^{l}_{s}(x)), \qquad  x \in \Nat(\Forget_{rt},\Forget_{km}),\] where $\hat{\Delta}^l_s(n)$ is the natural transformation\[\hat{\Delta}^l_s(x):  \Forget_{rs}\otimes \Forget_{st}\rightarrow \Forget_{kl}\otimes \Forget_{lm},\quad \hat{\Delta}^l_s(x)_{X,Y} = \pi^{(klm)}_{X,Y}x_{X\otimes Y} \iota^{(rst)}_{X,Y},\quad X\in \CatC_{k'l'},Y\in \CatC_{l'm'}.\]
\end{Def}

\begin{Rem} It has to be argued that $f\cdot g$ has finite support (over $\mathcal{I})$ as a functional on $\Nat(\Forget_{rt},\Forget_{km})$. In fact, if $f$ is supported at $b\in \mathcal{I}_{r's'}$ and $g$ at $c\in \mathcal{I}_{s't'}$, then $f\cdot g$ has support in the finite set of $a\in \mathcal{I}_{r't'}$ with $a\leq b\cdot c$, since if $x$ is a natural transformation with support outside this set, one has $x_{u_b\otimes u_c}=0$, and hence any of the $\left(\hat{\Delta}^l_s(x)\right)_{u_b,u_c} =0$.
\end{Rem}

\begin{Lem} The above product maps turn $(\mathscr{A},M)$ into an $I^2$-partial algebra.
\end{Lem}
\begin{proof} We can extend the map $(\hat{\Delta}^l_s\otimes \id)$ on $\Nat(\Forget_{rt},\Forget_{km})\otimes \Nat(\Forget_{tu},\Forget_{mn})$ to a map \[(\hat{\Delta}^l_s\otimes \id): \Nat(\Forget_{rt}\otimes \Forget_{tu},\Forget_{km}\otimes \Forget_{mn}) \rightarrow  \Nat(\Forget_{rs}\otimes \Forget_{st}\otimes \Forget_{tu},\Forget_{kl}\otimes \Forget_{lm}\otimes \Forget_{mn}),\] \[(\hat{\Delta}^l_s\otimes \id)(x)_{X,Y,Z} = \left(\pi^{(klm)}_{X,Y}\otimes \id_{\Forget_{mn}(Z)}\right) x_{X\otimes Y, Z} \left(\iota^{(rst)}_{X,Y} \otimes \id_{\Forget_{tu}(Z)}\right).\]
By finite support, we then have that \[((f\cdot g)\cdot h)(x) = (f\otimes g\otimes h)(\hat{\Delta}^l_s\otimes \id)\hat{\Delta}^m_t(x))\] for all $f\in \Gr{A}{k}{l}{r}{s},g\in \Gr{A}{l}{m}{s}{t},h\in \Gr{A}{m}{n}{t}{u}$ and $x\in  \Nat(\Forget_{ru},\Forget_{kn})$. Similarly, \[(f\cdot g)\cdot h)(x) = (f\otimes g\otimes h)((\id\otimes \hat{\Delta}^m_t)\hat{\Delta}^l_s(x).\] The associativity then follows from the 2-cocycle condition for the $\iota$- and $\pi$-maps. 

By a similar argument, one sees that the (non-zero) units are given by $\UnitC{k}{l}\in \Gr{A}{k}{k}{l}{l}(\Unit_{r})$  (for $r=k'=l'$) corresponding to $1$ in the canonical identifications  \[\Gr{A}{k}{k}{l}{l}(r) \cong \Hom_{\C}(\Forget_{ll}(\Unit_{r}),\Forget_{kk}(\Unit_{r}))^*\cong \Hom_{\C}(\C,\C)^*  \cong \C^* \cong \C.\] Indeed, to prove for example the right unit property, we use that (essentially) $\pi_{u_a,\Unit_{r}}^{(kll)} =(\id\otimes \mu_l)$ and $\iota_{u_a,\Unit_{r}}^{(kll)} = (\id\otimes \mu_l^{-1})$, while \[\UnitC{k}{l}(\mu_kx_{\Unit_{r}}\mu_l^{-1}) = x_{\Unit_{r}} \in \C,\quad x\in \Nat(F_{ll},F_{kk}).\] % to complete
\end{proof} 

\begin{Prop} The partial algebra and coalgebra structures on $\mathscr{A}$ define a partial bialgebra structure on $\mathscr{A}$. 
\end{Prop}
\begin{proof} Let us check the properties in Definition \ref{DefPartBiAlg}. Properties \ref{Propa} and \ref{Propc} are left to the reader. Property \ref{Propd} was proven above. Property \ref{Propb} follows from the fact that for $k'=l'=s'=m'$, \[\hat{\Delta}^{l}_s(\id_{F_{km}}) = \delta_{ls} \id_{F_{kl}}\otimes \id_{F_{lm}}.\] 
It remains to show the multiplicativity property \ref{Prope}. This is equivalent with proving that, for each $x\in \Nat(F_{uw},F_{km})$ and $y\in \Nat(F_{rt},F_{uw})$ (with all first or second indices in the same class of $I_0$), one has (pointwise) that (for $l'=s'$) \[ \hat{\Delta}^l_s(x\circ y) = \sum_{v,v'=l'} \hat{\Delta}^v_s(x)\circ \hat{\Delta}^l_v(y).\] This follows from the fact that $\sum_v \pi^{(uvw)}_{X,Y}\iota^{(uvw)}_{X,Y} \cong \id_{\Forget_{uw}(X\otimes Y)}$ (where we again note that the left hand side sum is in fact finite).
\end{proof} 

Let us show now that the resulting bialgebra $\mathscr{A}$ has an invariant integral.

\begin{Def} Define $\phi: \Gr{A}{k}{k}{m}{m} \rightarrow \C$ as the functional which is zero on $\Gr{A}{k}{k}{m}{m}(a)$ with $a\neq \Unit_{k'}$, and the canonical identification $\Gr{A}{k}{k}{m}{m}(k')\cong \C$ on the unit component (for $k'=m'$).
\end{Def}

\begin{Lem} The functional $\phi$ is an invariant integral.
\end{Lem}

\begin{proof} The normalisation condition $\phi(\UnitC{k}{k})=1$ is immediate by construction. Let us check left invariance - right invariance will follow similarly.

Let $\hat{\phi}^k_l$ be the natural transformation from $\Forget_{ll}$ to $\Forget_{kk}$ which has support on multiples of $\Unit_{k'}$, and with $(\hat{\phi}^k_l)_{\Unit_{k'}} = 1$.  Then for $f\in \Gr{A}{k}{k}{l}{l}$, we have $\phi(f) = f(\hat{\phi}^k_l)$. The left invariance of $\phi$ then follows from the easy verification that for $x\in \Nat(F_{ll},F_{kn})$, \[x\circ \hat{\phi}^l_m =\delta_{k,n} \UnitC{k}{l}(y)\hat{\phi}^k_m.\] % To complete? 
\end{proof}

So far, we have constructed from $\CatCC$ and $F$ a partial bialgebra $\mathscr{A}$ with invariant integral $\phi$. Let us further impose for the rest of this section that $\CatCC$ admits duality.

\begin{Prop}\label{PropAnti} The partial bialgebra $\mathscr{A}$ is a regular partial Hopf algebra.
\end{Prop} 

\begin{proof} By Lemma \ref{LemMorDua}, we have for each $X\in \CatC_{rs}$ a canonical isomorphism \[d^{(mn)}_X:F_{mn}(X) \cong F_{nm}({}^*X)^*.\] 

For a linear map $x$ between vector spaces, let us denote $x^{\tr}$ for the transpose map between their duals. Then for any $x\in \Nat(F_{mn},F_{kl})$, let us define $\hat{S}(x) \in \Nat(F_{lk},F_{nm})$ by \[\hat{S}(x)_X = \left(d^{(nm)}_X\right)^{-1} x_{X^*}^{\tr} d_X^{(lk)}.\] % Ugly formatting

Then the assigment $\hat{S}$ dualizes to maps $S:\Gr{A}{k}{l}{m}{n} \rightarrow \Gr{A}{n}{m}{l}{k}$ by $S(f)(x) = f(\hat{S}(x))$. We claim that $S$ is an antipode for $\mathscr{A}$. 

Let us check for example the formula \[\sum_r f_{(1){\tiny \begin{pmatrix}k&l\\n & r\end{pmatrix}}} S(f_{(2){\tiny \begin{pmatrix} n & r \\ m & l\end{pmatrix}}}) = \delta_{k,m}\epsilon(f)\UnitC{k}{n}\] for $f\in \Gr{A}{k}{l}{m}{l}$. The other antipode identity follows similarly.

By duality, this is equivalent to the pointwise identity of natural transformations \[\sum_r\hat{M}^n_r(\id\otimes \hat{S})\hat{\Delta}^l_r(x) = \delta_{k,m}\UnitC{k}{n}(x) \id_{F_{kl}},\quad x\in \Nat(F_{nn},F_{km})\] where $\hat{M}^n_r$ and $(\id\otimes \hat{S})$ are dual to respectively $\Delta_{nr}$ and $\id\otimes S$. 

Let us fix $X\in \mathcal{C}_{k'l'}$, and let $e_i^{(kl)}$ be a basis for $F_{kl}(X)$ with dual basis $\omega_i^{(kl)}$. Let $\chi_i^{(lk)} = d_X^{(kl)}(e_i^{(kl)})$, and let $f_i^{(lk)}$ be the basis of $F_{lk}(X^*)$ dual to the basis $\{\chi_i^{(lk)}\}$. Then it is easily verified that \[\pi^{(klk)}_{X,{}^*X}F_{kk}(\coev_{X})(1) = \sum_i e_i^{(kl)}\otimes f_i^{(lk)},\quad F_{ll}(\ev_X)\iota^{(lkl)}_{{}^*X,X} = \sum_i \chi_i^{(lk)}\otimes \omega_i^{(kl)},\] where we identified $\C\cong F_{kk}(\Unit_{k'})$. We then check that for $x\in \Nat(F_{mn},F_{kl})$ and $v\in F_{lk}(X)$, one has \[\hat{S}(x)_Xv = \sum_{i,j} \omega_i^{(lk)}(v)\chi_i^{(kl)}(x_{X^*}f_j^{(mn)})e_j^{(nm)}.\] 


Hence for $x\in \Nat(F_{nr},F_{kl})$, $y\in \Nat(F_{rn},F_{lm})$ and $v \in F_{ml}(X)$, we have \begin{eqnarray*} \left(\hat{M}^n_r(\id\otimes \hat{S})(x\otimes y)\right)_X(v) &=& \sum_i \omega_i^{(ml)}\left(v\right)  x_Xd_X^{(nr)-1}(y_{X^*}^{\tr}(\chi_i^{(lm)}))\\ &=& \sum_{i,j} \omega_i^{(ml)}\left(v\right)  \chi_i^{(lm)}\left(y_{X^*}(f_j^{(rn)}))\right)x_Xe_j^{(nr)} \\ &=& \sum_{i,j}(\omega_i^{(ml)}\otimes \chi_i^{(lm)})\left(v\otimes y_{X^*}(f_j^{(rn)})\right)x_Xe_j^{(nr)}.\end{eqnarray*} So for $ x\in \Nat(F_{nn},F_{km})$ and $\omega \in F(X)_{kl}^*$,  \begin{eqnarray*} \omega\left(\sum_r\left(\hat{M}^n_r(\id\otimes \hat{S})\hat{\Delta}^l_r(x)\right)_X(v) \right)&=& \sum_{r,i,j}
\omega_i^{(ml)}(v)(\omega\otimes \chi_i^{(lm)})(\pi_{X,{}^*X}^{(mlm)}x_{X\otimes {}^*X}\iota_{X,{}^*X}^{(nrn)}(e_j^{(nr)}\otimes f_j^{(rn)})) \\
&=&  \sum_{i}
\omega_i^{(ml)}(v)(\omega\otimes \chi_i^{(lm)})(\pi_{X,{}^*X}^{(mlm)}x_{X\otimes {}^*X}F_{nn}(\coev_X)1)\\ &=&  \delta_{km} \UnitC{k}{n}(x)  \sum_{i}
\omega_i^{(kl)}(v)(\omega\otimes \chi_i^{(lk)})(\pi_{X,{}^*X}^{(klk)}F_{kk}(\coev_X)1) \\ &=&  \delta_{km} \UnitC{k}{n}(x)  \sum_{i,j}
\omega_i^{(kl)}(v)(\omega\otimes \chi_i^{(lk)})(e_j^{(kl)}\otimes f_j^{(lk)}) \\ &=&  \delta_{km} \UnitC{k}{n}(x)  \omega(v).
\end{eqnarray*}

Similarly, one shows that $\mathscr{A}$ with the opposite multiplication has an antipode, using right duality. It follows that $\mathscr{A}$ is a regular partial Hopf algebra.  
\end{proof} 

Now let us updgrade $\CatCC$ to a semisimple partial C$^*$-category with duality, and $\Forget$ to a morphism from $\CatCC$ to $\{\Hilb_{\fin}\}_{I\times I}$. Then we can of course still form the partial Hopf algebra $\mathscr{A}$ as above. Let us show that it becomes a partial Hopf $^*$-algebra with positive invariant integral.

We first introduce a $^*$-structure on $\mathscr{A}$. 
%To distinguish it from the $^*$-operations present on the C$^*$-category, we will denote it by $\dagger$.

\begin{Def} We define $^*: \Gr{A}{k}{l}{m}{n}\rightarrow \Gr{A}{l}{k}{n}{m}$ by the formula \[f^*(x) = \overline{f(\hat{S}(x)^*)},\qquad x\in \Nat(F_{nm},F_{lk}).\]
\end{Def}

\begin{Lem} The operation $^*$ is an anti-linear, anti-multiplicative, comultiplicative involution.
\end{Lem}

\begin{proof} Anti-linearity is clear. Comultiplicativity follows from the fact that $(xy)^* = y^*x^*$ and $\hat{S}(xy) = \hat{S}(y)\hat{S}(x)$ for natural transformations. To see anti-multiplicativity of $^*$, note first that, since $S$ is antimultiplicative for $\mathscr{A}$, $\hat{S}$ is anti-comultiplicative on natural transformations. Now as $(\iota_{X,Y}^{(klm)})^* = \pi_{X,Y}^{(klm)}$ by assumption, we also have $\hat{\Delta}^l_s(x)^* = \hat{\Delta}^s_l(x^*)$, which proves anti-multiplicativity of $^*$ on $\mathscr{A}$.  Finally, involutivity follows from the involutivity of $x\mapsto \hat{S}(x)^*$, which is a consequence of the identity $d_{X^*} = d_X^{\tr}$. %More info.
\end{proof}

\begin{Prop} The couple $(\mathscr{A},\Delta)$ with the above $^*$-structure defines a partial compact quantum group.
\end{Prop}
\begin{proof} The only thing which is left to prove is that our invariant integral $\phi$ is a positive functional. Now it is easily seen from the definition of $\phi$ that the $\Gr{A}{k}{l}{m}{n}(a)$ are all mutually orthogonal. Hence it suffices to prove that the sesquilinear inner product \[\langle f,g\rangle = \phi(f^*g)\] on $\Gr{A}{k}{l}{m}{n}(a)$ is positive-definite. 

Let us write $\bar{f}(x) = \overline{f(x^*)}$. Let again $\hat{\phi}^k_m$ be the natural transformation from $F_{mm}$ to $F_{kk}$ which is the identity on $\Unit_{k'}$ and zero on other irreducible objects. Then by definition, \[\phi(f^*g) = (\bar{f}\otimes g)((\hat{S}\otimes \id)\hat{\Delta}^k_m(\hat{\phi}^l_n)).\] Recall now the notation of Proposition \ref{PropAnti}. Let us write $\omega_i^{(kl)} = \langle E_i^{(kl)},-\rangle$ and $\chi_i^{(kl)} = \langle F_i^{(kl)},-\rangle$. Assume further that $f(x) = \langle v',x_a v\rangle$ and $g(x) = \langle w',x_aw\rangle$ for $v,w\in F_{mn}(u_a)$ and $v',w'\in F_{kl}(u_a)$. Then using the expression for $\hat{S}$ as in Proposition \ref{PropAnti}, we find that \[\phi(f^*g) = \sum_{ij} \langle v,e_j^{(mn)}\rangle\langle F_i^{(lk)}\otimes w',\hat{\Delta}_m^k(\hat{\phi}^{l}_n)_{\bar{a}\otimes a} (f_j^{(nm)}\otimes w)\rangle \langle E_i^{(kl)}, v'\rangle .\] However, up to a positive non-zero scalar, which we may assume to be 1 by proper rescaling, we have \[\hat{\Delta}^k_m(\hat{\phi}^l_n)_{\bar{a}\otimes a} = \sum_{i,j} \mid F_j^{(lk)} \otimes E_j^{(kl)}\rangle \langle F_i^{(nm)}\otimes E_i^{(mn)}\mid.\] Hence \begin{eqnarray*} \phi(f^*g) &=&\left(\sum_{ij}\langle v,e_j^{(mn)}\rangle \langle F_i^{(nm)},f_j^{(nm)}\rangle\langle E_i^{(mn)},w\rangle \right)  \left(\sum_{ij}  \langle w',E_j^{(kl)}\rangle \langle F_i^{(lk)},F_j^{(lk)}\rangle\langle E_i^{(kl)},v'\rangle\right)\\ &=& \langle v,w\rangle  \left(\sum_{ij}  \langle w',E_j^{(kl)}\rangle \langle F_i^{(lk)},F_j^{(lk)}\rangle\langle E_i^{(kl)},v'\rangle\right).
\end{eqnarray*} It follows that $\phi(f^*f)\geq 0$ for all $f$.

\end{proof} 

\begin{Theorem} \label{TheoTKPCQG}

The assigment $\mathscr{A}\rightarrow (\Corep(\mathscr{A}),\Forget)$ is (up to isomorphism/equivalence) a one-to-one correspondence between partial compact quantum groups based over $\varphi_0:I\twoheadrightarrow I_0$ and semi-simple $I_0$-partial tensor C$^*$-categories $\CatCC$ with duality and with forgetful strongly monoidal $^*$-functor $\Forget$ to $(\Hilb_f)_{I\times I}$ based over $\varphi_0$. 
\end{Theorem} 

\begin{proof} Fix first $\mathscr{A}$, and let $\mathscr{B}$ be the partial Hopf $^*$-algebra constructed from $\Corep(\mathscr{A})$ with its natural forgetful functor. Then we have a map $\mathscr{B} \rightarrow \mathscr{A}$ by \[ \Gr{B}{k}{l}{m}{n}(a) = \Hom(\Gr{V}{}{(a)}{m}{n},\Gr{V}{}{(a)}{k}{l})^* \rightarrow \Gr{A}{k}{l}{m}{n}(a):  f \mapsto (\id\otimes f)X_a,\] where the $(V^{(a)},X_a)$ run over all irreducible unitary corepresentations of $\mathscr{A}$. It is easy to check from the definitions of $\mathscr{B}$ that this map is an isomorphism of Hopf $^*$-algebras. As the matrix coefficients of irreducible unitary corepresentations span $\mathscr{A}$, the map is surjective. By the Schur orthogonality relations, it is bijective.

Conversely, let $\CatCC$ be a semi-simple $I_0$-partial tensor C$^*$-category with duality and with forgetful strongly monoidal $^*$-functor $\Forget$ to $(\Hilb_f)_{I\times I}$ based over $\varphi_0$. Let $\mathscr{A}$ be the associated Hopf $^*$-algebra. For each irreducible $u_a \in \CatCC$, let $V^{(a)} = F(u_a)$, and \[\Gr{(X_a)}{k}{l}{m}{n} = \sum_i e_i^*\otimes e_i,\] where $e_i$ is a basis of $\Nat(F_{mn},F_{kl})$ and $e_i^*$ a dual basis. Then from the definition of $\mathscr{A}$ it easily follows that $X_a$ is a unitary corepresentation for $\mathscr{A}$. Since $\Nat(F_{mn},F_{kl})$ spans the total space of operators at any irreducibel object, it follows that $X_a$ is irreducible. As the matrix coefficients of the $X_a$ span $\mathscr{A}$, it follows that the $X_a$ form a maximal class of non-isomorphic unitary corepresentations of $\mathscr{A}$. Hence we can make a unique equivalence \[\CatCC\rightarrow \Corep(\mathscr{A}), \quad u \mapsto (F(u),X_u)\] such that $u_a\rightarrow X_a$. From the definitions of the coproduct and product in $\mathscr{A}$, it is readily verified that the natural morphisms $\iota^{(klm)}_{u,v}:F_{kl}(u)\otimes F{lm}(v)\rightarrow F_{km}(u\otimes v)$ turn it into a monoidal equivalence. 
\end{proof}

\section{Examples}

\subsection{Classical examples}

% Example of categorical quantum SU(2)? 

\subsection{Canonical partial compact quantum groupoids}

$\CatCC$ acting on itself. Characterisation of resulting partial compact quantum groups? 

\subsection{Morita equivalence}

% Caenepeel Galois theory weak Hopf algebras

\begin{Def} Two partial compact quantum groups $\mathscr{G}$ and $\mathscr{H}$ are said to be \emph{Morita equivalent} if there exists an equivalence $\Rep(\mathscr{G}) \rightarrow \Rep(\mathscr{H})$ of partial tensor C$^*$-categories. % notion of equivalence to be explained? $\varphi_0$ is bijective, and all $F_{rs}$ are faithful and essentially surjective.
\end{Def} 

In particular, if $\mathscr{G}$ and $\mathscr{H}$ are Morita equivalent they have the same hyperobject set, but they need not share the same object set.

\begin{Prop} The partial compact quantum groups $\mathscr{G}_1$ and $\mathscr{G}_2$ over respective object sets $I_1$ and $I_2$ and with the same hyperobject set $I_0$ are Morita equivalent if and only if there exists a partial compact quantum group $\mathscr{G}$ over the object set $I_1\sqcup I_2$ such that the associated partial compact quantum group $\mathscr{A}$ has $\Gr{A}{k}{l}{m}{n}=\{0\}$ for $k,l$ or $m,n$ not in the same set $I_j$, such that $\Gr{A}{k}{l}{m}{n}\neq 0$ for ..., and such that the $\Gr{A}{k}{l}{m}{n}$ with all indices in the same set $I_j$, together with all $\Delta_{rs}$ with $rs$ also in this same set, form an isomorphic copy of the partial Hopf $^*$-algebra associated to $\mathscr{G}_j$. % Formulate more succinctly? 
\end{Prop}
\begin{proof} If $\mathscr{G}_1$ and $\mathscr{G}_2$ are Morita equivalent, we may identify the partial tensor C$^*$-categories of $\mathscr{G}_1$ and $\mathscr{G}_2$ with the same abstract partial tensor C$^*$-category $\CatCC$. Then the $\mathscr{G}_j$ are obtained from respective forgetful functors $F_j$ into $(\Hilb_{\fin})_{I_j\times I_j}$. Form $F_1\oplus F_j$, which is a morphism from $\CatCC$ into $(\Hilb_{\fin})_{(I_1\cup I_2)\times (I_1\cup I_2)}$. It is easily seen that the associated compact quantum groupoid is of the above form.

Conversely, ...
\end{proof} 

For example, co-groupoid of Bichon. 

Let $(A,\Delta)$ be a generalized Hopf face algebra over a set $I$. Assume that $I = I_1\sqcup I_2$, and let $\Lambda_j = \sum_{i\in I_j}\lambda_i$, resp. $\Rho_j = \sum_{i\in I_j} \rho_j$. If the $\Lambda_j$ and $\Rho_j$ are central in $M(A)$, then we can write $A = \osum{i,j} A(ij)$ where $A(ij) = \Lambda_i\Rho_jA$ are subalgebras. Moreover, the comultiplication $\wDelta$ splits into comultiplications \[\wDelta_{ij}^k:A(ij)\rightarrow M(A(ik)\otimes A(kj))\textrm{ s.t. } \wDelta = \wDelta_{ij}^1 +\wDelta_{ij}^2 \textrm{ on }A(ij).\] A similar decomposition holds for $\Delta$.

It is immediate to see that the $(A(ii),\Delta_{ii}^i)$ are two generalized Hopf face algebras over the respective $I_i$.

\begin{Def} We say $(A,\Delta)$ is a \emph{co-linking generalized (compact) Hopf face algebra} between $(A(11),\Delta_{11}^1)$ and $(A(22),\Delta_{22}^2)$ if $\lambda_i\Rho_2\neq 0$ for any $i\in I_1$.
\end{Def}

Upon applying the antipode, we see that then $\rho_j\Lambda_1\neq 0$ for any $j\in I_2$ as well.

\begin{Def} Two generalized (compact) Hopf face algebras are called \emph{comonoidally Morita equivalent} if they are isomorphic to the components $(A_{ii},\Delta_{ii}^i)$ of some co-linking generalized (compact) Hopf face algebra.\end{Def}

As an example, consider two sets $I_i$, and two tensor functors $(F_i,\phi_i)$ of a semi-simple rigid C$^*$-category $\CatC$ with irreducible unit into $\Hilb_{I_i^2}$. Then with $I= I_1\sqcup I_2$, we can form a new C$^*$-functor $F=F_1\oplus F_2$ of $\CatC$ into $\Hilb_{I^2}$ by putting $F(X) = F_1(X)\oplus F_2(X)$ with the obvious $I^2$-grading (and the obvious direct sum operation on morphisms). It becomes monoidal by means of the unitaries \[F(X\otimes Y) = F_1(X\otimes Y)\oplus F_2(X\otimes Y) \underset{\phi_1\oplus \phi_2}{\cong} (F_1(X)\underset{I_1}{\otimes} F_1(Y)) \oplus (F_2(X)\otimes F_2(Y)) \cong F(X)\itimes F(Y)\] (where the last map is unitary since $(F(X)\itimes F(Y))_{ij}=0$ for example for $i\in I_1$ and $j\in I_2$).

If we then consider the generalized compact Hopf face algebra $(A,\Delta)$ associated to $F$, we have immediately from the construction that the $\Lambda_i$ and $\Rho_i$ associated to the decomposition $I = I_1\sqcup I_2$ are indeed central elements in $M(A)$. Moreover, the parts $(A_{ii}^i,\Delta_{ii}^i)$ are seen to arise from applying the Tannaka-Krein construction to the respective functors $F_1$ and $F_2$. The fact that $(A,\Delta)$ is co-linking is immediate from the fact that \emph{none} of the $\lambda_i\rho_j$ are zero in this particular case (since $\Gr{A}{k}{k}{m}{m}(o) = B(F(u_o)_{kk},F(u_o)_{mm}) \cong \C$).

We will exploit the above extra structure in the following section to say something about the algebra $A$ appearing in ... This is the component $\tilde{A}(1,1)$ of the above algebra. The following lemma will be needed.


\begin{Lem} Assume $(A,\Delta)$ is a co-linking generalized Hopf face algebra. Then any of the maps $\wDelta_{ij}^k$ is injective.\end{Lem}

\begin{proof} Take a non-zero $x\in A_n(ij)$ where $n\in I_j$. Then for any $l\in I$ with $\rho_n\lambda_l\neq 0$, we know that $\wDelta(x)(1\otimes \rho_n\lambda_l)\neq 0$. Hence $\wDelta_{ij}^k(x)(1\otimes \rho_n\lambda_l)\neq 0$ for $l\in I_k$, and hence $\wDelta_{ij}^k(x)\neq 0$. Now if $j=k$, the condition $\rho_n\lambda_l\neq 0$ is satisfied by taking $l=n$ (since $\varepsilon(\lambda_n\rho_n)=1$). If $j\neq k$, it is satisfied for at least one $l$ by the co-linking assumption.
\end{proof}


\subsection{Weak Morita equivalence}

% Concrete implementation, helps to define the concept in more general analytic context.
Cf. Muger. 

\begin{Def} Two partial semi-simple tensor C$^*$-categories $\CatCC_1$ and $\CatC_2$ with duality over respective sets $I_1$ and $I_2$ are called \emph{Morita equivalent} if there exists a partial semi-simple tensor C$^*$-category $\CatCC$ with duality over the set $I=I_1\sqcup I_2$ such that the $\CatC_{rs}$ with $r,s$ in the same set $I_j$ form a copy of $\CatCC_j$, and such that each object in $\CatCC_{j}$ is a subobject of $X\otimes Y$ for some $X\in \CatCC_{kl}$ and $Y\in \CatCC{lk}$ with $k$ and $l$ in disjoint sets and $k\in I_j$. % Better formulation.

We say two partial compact quantum groups $\mathscr{G}_1$ and $\mathscr{G}_2$ are weakly Morita equivalent if their corepresentation categories are Morita equivalent. % ???
\end{Def} 

This is indeed generalisation of Morita equivalence. 

Notion of co-Morita equivalence. Weak Morita equivalence = Morita equivalence + co-Morita equivalence. 

%Induction of forgetful functor from $\CatCC_1$ to $\CatCC$.?? Maybe by internal representation of a module category by an algebra object. 

% Concrete implementation of weak Morita equivalence... 

\subsection{Ergodic actions of compact quantum group}

Let us now give a rich source of examples coming from ergodic actions of compact quantum groups. We recall the main result from DC-Yamashita.

Link with Morita equivalence. 

Concrete example of dynamical quantum $SU(2)$ to be studied in particular in separate paper. 








%%% Local Variables: 
%%% mode: latex
%%% TeX-master: "dyn-suq-main"
%%% End: 



\section{Examples}

\subsection{Classical examples}

\begin{Exa} Let $G$ be a discrete groupoid with object set $I = G^{(0)}$. Consider for $r,s\in I$ the vector space $\Gru{A}{r}{s}= \mathbb{C}\lbrack \Gru{G}{r}{s}\rbrack$ which have a basis $\{\lambda_g\}$ spanned by the morphisms $g$ from $r$ to $s$. Linearly extending the product of $G$ to the $\Gru{A}{r}{s}$ turns $\mathscr{A}$ into a partial algebra. It becomes a partial $^*$-algebra by putting $\lambda_g^* = \lambda_{g^{-1}}$. 

Extend the $I^2$-bigrading to an $I^4$-bigrading by putting $\Gr{A}{k}{l}{m}{n} = \delta_{kl}\delta_{mn}\Gru{A}{k}{l}$. Then together with the coproducts \[ \Delta: \Gru{A}{r}{s}\rightarrow \Gru{A}{r}{s}\otimes \Gru{A}{r}{s},\quad \lambda_g\mapsto \lambda_g\otimes \lambda_g,\] $(\mathscr{A},\Delta)$ defines a partial compact quantum group, the invariant functional $\phi$ being given by \[\phi(\lambda_g) = \delta_{rs}\delta_{g,\id_r},\quad r,s\in I,g\in \Mor(r,s).\]
\end{Exa}

\begin{Exa} Let $G$ be a proper locally compact groupoid with discrete object space $I=G^{(0)}$. Then each $\Gru{G}{r}{s} = \Mor(r,s)$ is a compact space. The $\Gru{G}{r}{r}$ are compact groups, hence come with probability Haar measures $\mu_{rr}$. The $\Gru{G}{r}{s}$ are $\Gru{G}{r}{r}$-$\Gru{G}{s}{s}$-bitorsors, and as such admit a bi-invariant probability measure $\mu_{rs}$. Let $A^r_s \subseteq C(\Gru{G}{r}{s})$ be the spaces of functions which transform as finite-dimensional representations of $\Gru{G}{r}{r}\times \Gru{G}{s}{s}$ under the natural actions. Then the $A^r_s$ form a partial coalgebra over $I$ by putting \[\Delta_{s}: A^r_t \rightarrow A^r_s\otimes A^s_t,\quad f\mapsto \left(\Delta_s(f):\Gru{G}{r}{s}\times \Gru{G}{s}{t}\mapsto \Gru{G}{r}{t},\quad (g,h)\mapsto f(gh)\right).\] We can extend the $I^2$-grading to an $I^4$-grading by putting \[\Gr{A}{k}{l}{m}{n} = \delta_{kl}\delta_{mn} A^l_n,\] and the resulting $\mathscr{A}$ becomes a partial $^*$-algebra by endowing each $A^r_s$ with the pointwise product and $^*$-algebra structure. The couple $(\mathscr{A},\Delta)$ then defines a partial compact quantum group with invariant integral \[\phi: A^r_s \mapsto \C,\quad f\mapsto \int_{\Gru{G}{r}{s}} f(g) \rd \Gru{\mu}{r}{s}(g).\]

\end{Exa}

% Vacant double groupoids?

\begin{Exa} Let $\mathscr{A}$ and $\mathscr{B}$ define two partial compact quantum groups $\mathscr{G}$ and $\mathscr{H}$ over respective sets $I$ and $J$. Then we can make a tensor product partial Hopf $^*$-algebra $\mathscr{A}\otimes \mathscr{B}$ over the index set $I\times J$ by putting \[\Gr{(A\otimes B)}{(k,k')}{(l,l')}{(m,m')}{(n,n')} = \Gr{A}{k}{l}{m}{n}\otimes \Gr{B}{k'}{l'}{m'}{n'}\] with the factorwise product and with coproducts \[\Delta_{(r,r'),(s,s')} = \sigma_{23}(\Delta_{rs}\otimes \Delta_{r',s'}),\] $\sigma$ being the switch map. It is easily seen that the tensor products of the positive invariant integrals for $\mathscr{A}$ and $\mathscr{B}$ produce a positive invariant integral on $\mathscr{A}\otimes \mathscr{B}$. Hence $\mathscr{A}\otimes \mathscr{B}$ defines a partial compact quantum group, which we will denote $\mathscr{G}\times \mathscr{H}$.
\end{Exa}

\subsection{Canonical partial compact quantum groupoids}

The following generalizes Hayashi's original construction.

% Notion of faithfulness to remark upon.
\begin{Exa} 
Let $\CatCC$ be a semi-simple partial tensor C$^*$-category with duality based over a set $I_0$. Let $\mathcal{I}$ label a distinguished maximal set $\{u_k\}$ of mutually non-isomorphic irreducible objects of $\CatC$, with associated bigrading $\Gru{\mathcal{I}}{r}{s}$ over $I_0$. Define \[F_{kl}(X)  = \Hom(u_k,  X\otimes u_l),\qquad X\in \Gru{\CatC}{r}{s}, k\in \Gru{\mathcal{I}}{r}{t},l\in \Gru{\mathcal{I}}{s}{t}.\] Then each $F_{kl}(X)$ is a Hilbert space by the inner product $\langle f,g\rangle = f^*g$. Put $F_{kl}(X) = 0$ for $k,l$ outside their proper domains. Then clearly the application $(k,l)\mapsto F_{kl}(X)$ is rcf. Moreover, we have isometric compatibility morphisms \[F_{kl}(X)\otimes F_{lm}(Y)\rightarrow F_{km}(X\otimes Y),\quad f\otimes g \mapsto (\id\otimes g)f,\] while $F_{kl}(\Unit_r) \cong \delta_{kl} \C$ for $k,l\in \Gru{\mathcal{I}}{r}{r}$. 

It is readily verified that $F$ defines a unital morphism from $\CatCC$ to $\{\Hilb_{\fin}\}_{\mathcal{I}\times \mathcal{I}}$ based over the partition \[\mathcal{I}_r = \bigcup_{t} \Gru{\mathcal{I}}{r}{t},\quad r\in I_0.\] From the Tannaka-Krein-Woronowicz reconstruction result, we obtain a partial compact quantum group $\mathscr{A}_{\CatCC}$ with object set $\mathcal{I}$, which we call the \emph{canonical partial compact quantum group} associated with $\CatCC$. 
\end{Exa} 

\begin{Exa} More generally, let $\CatCC$ be a semi-simple partial tensor C$^*$-category with duality based over a set $I_0$, and let $\CatDD$ be a \emph{semi-simple partial $\CatCC$-module C$^*$-category} based over a set $I_1$. That is, $\CatDD$ consists of a collection of semi-simple C$^*$-categories $\CatD_{rs}$ with $r\in I_0,s\in I_1$, together with tensor products $\otimes: \CatC_{rs}\times \CatD_{st}\rightarrow \CatC{rt}$ satisfying the appropriate associativity and unit constraints. Then if $\mathcal{I}$ labels a distinguished maximal set $\{u_k\}$ of mutually non-isomorphic irreducible objects of $\CatD$, with associated bigrading $\Gru{\mathcal{I}}{r}{s}$ over $I_0\times I_1$, we can again define \[F_{kl}(X)  = \Hom(u_k,  X\otimes u_l),\qquad X\in \Gru{\CatC}{r}{s}, k\in \Gru{\mathcal{I}}{r}{t},l\in \Gru{\mathcal{I}}{s}{t},\] and we obtain a unital morphism from $\CatCC$ to $\{\Hilb_{\fin}\}_{\mathcal{I}\times \mathcal{I}}$. The associated partial compact quantum group $\mathscr{A}_{\CatCC}$ will be called the \emph{canonical partial compact quantum group} associated with $(\CatCC,\CatDD)$. The previous construction coincides with the special case $\CatCC= \CatDD$.
\end{Exa}

For example, let $\G$ be a compact quantum group, and consider an ergodic action of $\G$ on a unital C$^*$-algebra $C(\mathbb{X})$. Then the collection of finitely generated $\G$-equivariant $C(\mathbb{X})$-Hilbert modules forms a module C$^*$-category over $\Rep(\G)$, cf. []. 

\subsection{Morita equivalence}

% Caenepeel Galois theory weak Hopf algebras

\begin{Def} Two partial compact quantum groups $\mathscr{G}$ and $\mathscr{H}$ are said to be \emph{Morita equivalent} if there exists an equivalence $\Rep(\mathscr{G}) \rightarrow \Rep(\mathscr{H})$ of partial tensor C$^*$-categories. % notion of equivalence to be explained? $\varphi_0$ is bijective, and all $F_{rs}$ are faithful and essentially surjective.
\end{Def} 

In particular, if $\mathscr{G}$ and $\mathscr{H}$ are Morita equivalent they have the same hyperobject set, but they need not share the same object set.

Our goal is to give a concrete implementation of Morita equivalence, as has been done for compact quantum groups []. We introduce the following definition. In it, we treat $\{0,1\}$ as the two-element group under addition.

\begin{Def} A \emph{linking partial compact quantum group} consists of a partial compact quantum group $\mathscr{G}$ defined by a partial Hopf $^*$-algebra $\mathscr{A}$ over a set $I$ with a distinguished partition $I = I_1\sqcup I_2$ such that the units $\UnitC{i}{j} = \sum_{k\in I_i,l\in I_j} \UnitC{k}{l} \in M(A)$ are central, and such that for each $r\in I_i$, there exists $s\in I_{i+1}$ such that $\UnitC{r}{s}\neq 0$.
%moreover for each $a\in \Gr{A}{k}{l}{m}{n}$ with $k,l,m,n\in I_i$ we can find $r,s \in I_{i+1}$ such that $\Delta_{rs}(a) \neq0$. 
\end{Def}

If $\mathscr{A}$ defines a linking partial compact quantum group, we can split $A$ into four components $A^i_j = A\UnitC{i}{j}$. It is readily verified that the $A^i_i$ together with all $\Delta_{rs}$ with $r,s \in I_i$ define themselves partial compact quantum groups, which we call the \emph{corner} partial compact quantum groups of $\mathscr{A}$. 

\begin{Prop} Two partial compact quantum groups are Morita equivalent iff they arise as the corners of a linking partial compact quantum group.
\end{Prop}

% Some stylisation needed.
\begin{proof} Suppose first that $\mathscr{G}_1$ and $\mathscr{G}_2$ are partial compact quantum groups with associated partial Hopf $^*$-algebras $\mathscr{A}_1$ and $\mathscr{A}_2$ over respective sets $I_1$ and $I_2$. Then we may identify their corepresentation categories with the same abstract partial tensor C$^*$-category $\CatCC$ (based over $I_0$) which comes endowed with two forgetful functors $F_i$ to $\{\Hilb_{\fin}\}_{I_i\times I_i}$ corresponding respectively to the $\mathscr{A}_i$.

With $I = I_1\sqcup I_2$, we may then as well combine the $F_i$ into a global unital morphism $F:\CatCC \rightarrow \{\Hilb_{\fin}\}_{I\times I}$, with $F_{kl}(X)=F_i(X)$ if $k,l\in I_i$ and $F_{kl}(X)=0$ otherwise. Let $\mathscr{A}$ be the associated partial compact quantum group constructed from the Tannaka-Krein-Woronowicz reconstruction procedure. 

From the precise form of this reconstruction, it follows immediately that $\Gr{A}{k}{l}{m}{n} =0$ if either $k,l$ or $m,n$ do not lie in the same $I_i$. Hence the $\UnitC{i}{j} = \sum_{k\in I_i,l\in I_j} \UnitC{k}{l}$ are central. 

Moreover, fix $k\in I_i$ and any $l\in I_{i+1}$ with $k'=l'$. Then $\Nat(F_{ll},F_{kk})\neq \{0\}$. It follows that $\UnitC{k}{l}\neq 0$. Hence $\mathscr{A}$ is a linking compact quantum groupoid. It is clear that $\mathscr{A}_1$ and $\mathscr{A}_2$ are the corners of $\mathscr{A}$. 

Conversely, suppose that $\mathscr{A}_1$ and $\mathscr{A}_2$ arise from the corners of a linking partial compact quantum groupoid defined by $\mathscr{A}$ with invariant integral $\phi$. We will show that in fact the associated partial compact quantum groups $\mathscr{G}$ and $\mathscr{G}_1$ are Morita equivalent. Then by symmetry $\mathscr{G}$ and $\mathscr{G}_2$ are Morita equivalent, and hence also $\mathscr{G}_1$ and $\mathscr{G}_2$.

For $(V,X) \in \Corep(\mathscr{A})$, let $F(V,X) = (W,Y)$ be the pair obtained from $(V,X)$ by restricting all indices to those appearing to $I_1$. As the $\UnitC{i}{j}$ are central, it is easy to see that $(W,Y)$ is a unitary corepresentation of $\mathscr{A}_1$, and that the functor $F$ hence becomes a unital morphism in a trivial way. What remains to show is that $F$ is an equivalence of categories, i.e.~ that $F$ is faithful and essentially surjective. 

Let us first argue that any rcf corepresentation $W$ of $\mathscr{A}_1$ arises as a summand of some $F(V)$. Indeed, we may assume that $W$ is an irreducible corepresentation, and arises as regular corepresentation. Taking any non-zero element of this corepresentation, and considering an rcf regular corepresentation of $\mathscr{A}$ containing this element, we obtain an rcf corepresentation of $\mathscr{A}$ whose restriction contains $W$.

Let now $X$ define a unitary corepresentation of $\mathscr{A}_1$ on a partial Hilbert space $W= \{\Gru{W}{k}{l}\}$. For $m,n\in I_{2}$, define $G_{mn}(W) = \Gru{Z}{m}{n}$ by \[ \Gru{Z}{m}{n} = \{z \in \left(\prod_k\oplus_l\cap \prod_l\oplus_k\right) \Gr{A}{k}{l}{m}{n}\otimes \Gru{W}{k}{l}\mid (\Delta_{rs}\otimes \id)(z_{kl}) = (\Gr{X}{k}{l}{r}{s})_{13}(z_{rs})_{23},\quad \forall r,s\in I_1\}.\] We aim to show that $G_{mn}(W)$ is rcf. In fact, $G$ is clearly functorial and linear. By the previous paragraph, we may thus assume that $W$ arises as the restriction of some corepresentation $V$ of $\mathscr{A}$. 

Let then $Z$ be obtained as above from the restriction $W$ of $V$. We claim that the maps \[\theta_{mn}:\Gru{V}{m}{n}\rightarrow \Gru{Z}{m}{n},\quad v \mapsto \sum_{k,l\in I_1} \Gr{X}{k}{l}{m}{n}(1\otimes v),\quad m,n\in I_2\]  are well-defined linear isomorphisms. In fact, well-definedness is immediate. To show that the map is an isomorphism, let us construct a concrete inverse map. Fix $m,n$, and pick an $s\in I_1$ with $\UnitC{s}{n}\neq 0$, which is possible by assumption. Define then \[\gamma_{mn}:\Gru{Z}{m}{n} \rightarrow \Gru{V}{m}{n},\quad z \mapsto (\phi\otimes \id)\sum_{k\in I_1} \Gr{(X^{-1})}{s}{k}{n}{m}z_{ks},\] where we note that the summation over $k$ is in fact finite as the variables $m,n,s$ are fixed. It follows immediately from the definition of $X^{-1}$ (and the fact that $\UnitC{s}{n}\neq 0$) that $\phi_{mn}\theta_{mn} = \id_{\Gru{V}{m}{n}}$. To show the reverse identity $\theta_{mn}\phi_{mn} = \id_{\Gru{Z}{m}{n}}$, it suffices to show that $z_{ks} = (\theta_{mn}\gamma_{mn}(z))_{ks}$. Let us first show that \begin{equation}\label{EqIdz} \UnitC{s}{n}\otimes \sum_k (\phi\otimes \id)(\Gr{(X^{-1})}{s}{k}{n}{m}z_{ks}) = \sum_k \Gr{(X^{-1})}{s}{k}{n}{m}z_{ks}.\end{equation} Apply $\Delta_{nn}\otimes \id$ to the right hand side and use the invertibility of $X$ to conclude that  \[(\Delta_{nn}\otimes \id)\left(\sum_k \Gr{(X^{-1})}{s}{k}{n}{m}z_{ks}\right) = \UnitC{s}{n}\otimes \Gr{(X^{-1})}{n}{n}{n}{m}z_{nn}.\] Apply the counit to the second leg to conclude that in fact $\sum_k \Gr{(X^{-1})}{s}{k}{n}{m}z_{ks} \in \UnitC{s}{n} \otimes \Gru{W}{m}{n}$, from which $\eqref{EqIdz}$ follows. Similarly, for $n\neq n'$ we find  \[(\Delta_{n'n}\otimes \id)\left(\sum_k \Gr{(X^{-1})}{s}{k}{n'}{m}z_{ks}\right) =0,\] and applying the counit gives \begin{equation}\label{EqIdz2} \sum_k \Gr{(X^{-1})}{s}{k}{n'}{m}z_{ks} =0.\end{equation} From \eqref{EqIdz} and \eqref{EqIdz2}, we then conclude 
\begin{eqnarray*} (\theta_{mn}\gamma_{mn}(z))_{ks} &=& \sum_{k'} \Gr{X}{k}{s}{m}{n}\Gr{(X^{-1})}{s}{k'}{n}{m}z_{k's} \\ &=& \sum_{k',n'} \Gr{X}{k}{s}{m}{n'}\Gr{(X^{-1})}{s}{k'}{n'}{m}z_{k's} \\ &=& z_{ks}.\end{eqnarray*} 

In a similar way, one shows that $W\cong W'$ with, for $m,n\in I_1$, \[ \Gru{W'}{m}{n} = \{z \in \left(\prod_k\oplus_l\cap \prod_l\oplus_k\right) \Gr{A}{k}{l}{m}{n}\otimes \Gru{W}{k}{l}\mid (\Delta_{rs}\otimes \id)(z_{kl}) = (\Gr{X}{k}{l}{r}{s})_{13}(z_{rs})_{23},\quad \forall r,s\in I_1\}.\] 

Let us now take again an arbitrary unitary corepresentation $(W,Y)$ of $\mathscr{A}_1$. By the above, $\Gru{Z}{m}{n} = G_{mn}(W)$ is rcf. It is easy to see that we can then build an rcf corepresentation $U$ of $\mathscr{A}$ over $Z\oplus W'$ by putting \[\Gr{U}{k}{l}{m}{n}(1\otimes z) = (\Delta_{kl}^{\op}\otimes \id)(z),\qquad z\in \Gru{Z}{m}{n}, \quad m,n\in I_2,k,l\in I,\] and similarly for $W'$. It is then easy to see that, when $V$ is an rcf corepresentation, $V$ is equivariantly isomorphic to $Z\oplus W'$ for $W$ the restriction of $V$, while any $W$ is isomorphic to the restriction of $Z\oplus W'$.  This proves essential surjectivity and faithfulness.
\end{proof}

\begin{Exa} If $\mathscr{G}_1$ and $\mathscr{G}_2$ are Morita equivalent compact quantum groups, the total partial compact quantum group is the co-groupoid constructed by Bichon []. 
\end{Exa}

\begin{Exa}  Let $\G$ be a compact quantum group with ergodic action on a unital C$^*$-algebra $C(\mathbb{X})$. Consider the module C$^*$-category $\CatD$ of finitely generaetd $\G$-equivariant Hilbert $C(\mathbb{X})$-modules as before. Then $\G$ is Morita equivalent with the canonical partial compact quantum group constructed from $(\CatC,\CatD)$. The off-diagonal part of the associated linking partial compact quantum group was studied in ... We will make a detailed study of the case $\G = SU_q(2)$ in [], particularly for $\X$ a Podle\'{s} sphere, which will lead us to partial compact quantum group versions of the dynamical quantum $SU(2)$-group.
\end{Exa}


\subsection{Weak Morita equivalence}

% Concrete implementation, helps to define the concept in more general analytic context.


\begin{Def} A \emph{linking} partial tensor C$^*$-category with duality consists of a partial tensor C$^*$-category with duality $\CatCC$ with a distinguished partition $I_0 =I_1 \cup I_2$ such that for each $r\in I_1$, there exists $s \in I_{2}$ with $\CatC_{rs}\neq \{0\}$.

The \emph{corners} of $\CatCC$ are the restrictions of $\CatCC$ to $I_1$ and $I_2$.
\end{Def}

The following notion was introduced by M. M\"{u}ger []. % Sort out if definition is really the same + precise setting.

\begin{Def} Two partial semi-simple tensor C$^*$-categories $\CatCC_1$ and $\CatCC_2$ with duality over respective sets $I_1$ and $I_2$ are called \emph{Morita equivalent} if there exists a linking partial semi-simple tensor C$^*$-category $\CatCC$ with duality over the set $I=I_1\sqcup I_2$ whose corners are isomorphic to $\CatCC_1$ and $\CatCC_2$.

We say two partial compact quantum groups $\mathscr{G}_1$ and $\mathscr{G}_2$ are \emph{weakly Morita equivalent} if their corepresentation categories $\Corep_u(\mathscr{G}_i)$ are Morita equivalent. 
\end{Def} 

A particular example of weak Morita equivalence arises as follows.

\begin{Def} A \emph{co-linking partial compact quantum group} consists of a partial compact quantum group $\mathscr{G}$ defined by a Hopf $^*$-algebra $\mathscr{A}$ over an index set $I$, together with a distinguished partition $I = I_1\cup I_2$ such that each $\UnitC{k}{l}=0$ for $k,l$ in distinct sets, and such that for each $k\in I_1$, there exists $l\in I_2$ with $\Gr{A}{k}{l}{k}{l}\neq 0$.  
\end{Def} 

\begin{Rem} For finite compact quantum groupoids, one can easily show that the notion of co-linking partial compact quantum group is dual to the notion of linking partial compact quantum group.\end{Rem}

It is again easy to see that if we restrict all indices of a co-linking partial compact quantum group to one of the distinguished sets, we obtain a partial compact quantum group which we will again call a corner.

\begin{Def} We call two partial compact quantum groups \emph{co-Morita equivalent} if there exists a \emph{co-linking partial compact quantum group} having these partial compact quantum groups as its corners.
\end{Def}

\begin{Lem} Co-Morita equivalence is an equivalence relation. % To stylise
\end{Lem} 
\begin{proof} Symmetry is clear. Co-Morita equivalence of $\mathscr{A}$ with itself follows by considering as co-linking quantum groupoid the product of $\mathscr{A}$ with the partial compact quantum groupoid $M_2(\C)$, where $\Delta(e_{ij}) = e_{ij}\otimes e_{ij}$. 

Let us show transitivity. Suppose $\mathscr{G}$ defines a co-linking quantum groupoid between $\mathscr{G}_1$ and $\mathscr{G}_2$, and $\mathscr{H}$ a co-linking quantum groupoid between $\mathscr{G}_2$ and $\mathscr{G}_3$. We can write the total algebra of the partial Hopf $^*$-algebra $\mathscr{A}$ of $\mathscr{G}$ in the form $\begin{pmatrix} A_{11} & A_{12} \\ A_{21} & A_{22}\end{pmatrix}$, and the total algebra of the partial Hopf $^*$-algebra $\mathscr{B}$ of $\mathscr{H}$ as $\begin{pmatrix} A_{22} & A_{23} \\ A_{32} & A_{33}\end{pmatrix}$, where the $A_{ii}$ are the total algebras associated to the $\mathscr{G}_i$. 

Let us show that $A_{12}A_{21} = A_{11}$. 

Similarly,  $A_{21}A_{12} = A_{22}$, and the same statements hold for $\mathscr{B}$. 

Define now $A_{13} = A_{12}\underset{A_{22}}{\otimes} A_{23}$ and $A_{31} = A_{32}\underset{A_{22}}{\otimes} A_{21}$. Then we can make a partial $^*$-algebra $\mathscr{C}$ whose total algebra is $\begin{pmatrix} A_{11} & A_{13}\\ A_{31}& A_{33}\end{pmatrix}$, where for example the product of $A_{13}$ and $A_{31}$ is given by $(x\otimes y)(z\otimes w) =x(yz)w$, while $(x\otimes y)^* = y^*\otimes x^*$. We can make $A_{13}$ into a coalgebra by putting $\Delta(x\otimes y) = (x_{(1)}\otimes y_{(1)})\otimes (x_{(2)}\otimes y_{(2)})$, and similarly for $A_{31}$. Together with the coproducts on $A_{11}$ and $A_{33}$, this then forms a regular Hopf $^*$-algebra. It is easily checked that with $\phi$ the invariant integral on $A_{11}$ and $A_{33}$ and zero $A_{13}$ and $A_{31}$, this is a regular Hopf $^*$-algebra with invariant integral. It is obviously defines a co-linking quantum groupoid between $\mathscr{G}_{11}$ and $\mathscr{G}_{33}$. 
\end{proof} 

%This is indeed generalisation of Morita equivalence. 

\begin{Prop} Assume that two partial compact quantum groups $\mathscr{G}_1$ and $\mathscr{G}_2$ are co-Morita equivalent. Then they are weakly Morita equivalent.
\end{Prop} 
\begin{proof} % Stylise
Consider the corepresentation category $\CatCC$ of a co-linking partial compact quantum group $\mathscr{A}$ over $I = I_1\cup I_2$. Fix $r\in I_0$, and choose a corresponding partition $I_0 = J_1\cup J_2$ with $I_i \twoheadrightarrow J_i$ the associated partition. We want to show that $\Corep(\mathscr{A})_{rs} \neq \{0\}$ for $r\in J_1$ and $s\in J_2$. But  the collection of all $\Gr{A}{k}{l}{m}{n}$ with $k'=m'=r$ and $l=n=s'$ form a partial co-algebra under the $\Delta_{pq}$ with $p'=r$ and $q'=s$.  By assumption, this collection is not trivial. Hence any irreducible regular corepresentation in this class forms a non-trivial element of $\Corep(\mathscr{A})_{rs}$.
\end{proof}

%Notion of co-Morita equivalence. Weak Morita equivalence = Morita equivalence + co-Morita equivalence. 

\begin{Prop} Let $\CatCC$ be a linking partial tensor C$^*$-category with duals. Then the associated canonical partial compact quantum group is a co-linking partial compact quantum group. 
\end{Prop} 

\begin{proof}

\end{proof} 

\begin{Theorem} Two partial compact quantum groups $\mathscr{G}_1$ and $\mathscr{G}_2$ are weakly Morita equivalent if and only if they are connected by a string of Morita and co-Morita equivalences. 
\end{Theorem}

\begin{proof} The direction back has already been shown. Conversely, assume $\mathscr{G}_1$ and $\mathscr{G}_2$ are weakly Morita equivalent. Then $\mathscr{G}_i$ is weakly equivalent with the canonical partial compact quantum group associated to its corepresentation categories. But we have shown that then these canonical partial compact quantum groups are co-Morita equivalent. 
\end{proof} 

\begin{Rem} Note that it is essential that we allow the string of equivalences to pass through partial compact quantum groups, even if we start out with (genuine) compact quantum groups.\end{Rem}

% Gives concrete implementation of weak Morita equivalence.





%%% Local Variables: 
%%% mode: latex
%%% TeX-master: "dyn-suq-main"
%%% End: 


\section{Partial compact quantum groups on the level of operator algebras}


Let $\mathscr{G}$ be a partial compact quantum group. We construct
completions of the underlying $*$-algebra $P(\mathscr{G})$ in the form
of a universal $C^{*}$-algebra $\CuG$, a reduced $C^{*}$-algebra
$\CrG$ and a von Neumann algebra $\LGinf$. The existence of the first
one follows from the analogue of the Peter-Weyl theorem, Proposition \ref{prop:rep-weak-pw}, and the second and
third one arise from a GNS-representation of $P(\mathscr{G})$ on the
Hilbert space $\LGtwo$ associated to the invariant integral of
$\mathscr{G}$.  We then lift the comultiplication, the invariant
functional, the unitary antipode and the scaling group to level of
operator algebras and show that $\LGinf$ becomes a measured quantum
groupoid in the sense of Lesieur \cite{Les1} and Enock \cite{Eno2}.

Let us start with the construction of $\CuG$. Denote by $A$ the
underlying total $*$-algebra of the partial $*$-algebra
$P(\mathscr{G})$ and define a map $|\cdot |_{u} \colon A \to [0,\infty]$ by
\begin{align*}
  |a|_{u}&:= \sup \{ \|\pi(a)\| : \pi \text{ is a $*$-homomorphism
    from } A \text{ into some $C^{*}$-algebra } B\}.
\end{align*}
\begin{Lem}
  $|a|_{u}<\infty$ for each $a \in A$. 
\end{Lem}
\begin{proof}
  By Corollary \ref{cor:rep-pw}, we can write each $a\in A$ in the
  form $a=(\id \otimes \omega_{\xi,\eta})(X(K))$, where $X$ is a
  unitary sfd corepresentation of $P(\mathscr{G})$ on some sfd
  $I^{2}$-graded Hilbert space $\mathcal{H}$ and
  $K=\pmat{k}{l}{m}{n}\in M_{2}(I)$, $\xi\in \Gru{\mathcal{H}}{m}{n}$,
  $\eta\in \Gru{\mathcal{H}}{k}{l}$.  Since $X$ is unitary,
  \begin{align*}
    \sum_{p}X\pmat{p}{l}{m}{n}^{*} X\pmat{p}{l}{m}{n}  = \lambda_{l}\rho_{n}
    \otimes \id_{\Gru{\mathcal{H}}{m}{n}}
  \end{align*}
  by \ref{}, where the sum is finite because $X$ is sfd. As
  $\pi(\lambda_{k}\rho_{m})\in \C$ is a projection, we can conclude
  \begin{align*}
    \|\pi(a)\| &\leq \| \xi| \|\eta\| \|(\pi \otimes \id)(X(K))\| \\
    &=\|\xi\|\|\eta\| \|(\pi \otimes \id)(X(K)^{*}X(K))\| \leq
    \|\xi\|\|\eta\| \|\pi(\lambda_{l}\rho_{n}) \otimes
    \id_{\Gru{\mathcal{H}}{m} {n}}\| = \|\xi\|\|\eta\|.  
  \end{align*}
\end{proof}
Clearly, $|\cdot|_{u}$ defines a $C^{*}$-semi-norm on
$P(\mathscr{G})$, and the separated completion of $P(\mathscr{G})$
with respect to this norm is a $C^{*}$-algebra. We denote this
$C^{*}$-algebra by $\CuG$. By construction, every $*$-homorphism $\pi$
of $A$ into some $C^{*}$-algebra $C$ factorises through $\CuG$. 
We shall see that the canonical $*$-homomorphism $\pi_{u} \colon A \to
\CuG$ is injective.
\begin{Rem}
  The universal property of $\CuG$ implies that the modular
  automorphism group $\sigma$ and the scaling group $\tau$ for real
  parameters and the unitary antipode $R$ of $\mathscr{G}$ introduced
  after Corollary \ref{cor:rep-characters} lift to one-parameter
  groups $\tau^{u},\sigma^{u}$ and a $*$-anti-automorphism $R_{u}$ of
  $\CuG$, that is, $\tau^{u}_{t} \circ \pi_{u}= \pi_{u} \circ \tau_{t}$,
  $\sigma^{u}_{t} \circ \pi_{u} = \pi_{u} \circ \sigma_{t}$, $R_{u}
  \circ \pi_{u} = \pi_{u} \circ R$. Corollary \ref{cor:rep-characters}
  5.\ and A.1 in \cite{} [Takesaki:2]  imply that elements in
  $\pi_{u}(A)$ are analytic for $\tau^{u}$ and $\sigma^{u}$; in
  particular, $\tau^{u}$ and $\sigma^{u}$ are strongly continuous.  
\end{Rem}


We now turn to the construction of the reduced $C^{*}$-algebra $\CrG$
and the von Neumann algebra $\LGinf$. 
Denote by $\LGtwo$ the completion of $A$ with respect to the norm
associated to the inner product given by
\begin{align*}
  \langle a|b\rangle :=\phi(a^{*}b) \quad \text{for all } a,b\in A,
\end{align*}
and by $\Lambda \colon A \to \LGtwo$ the natural embedding.  This
product is definite because $\phi$ is faithful by \ref{LemFaith}, and
it extends to the space $\LGtwo$ which thus becomes a Hilbert space.
As such, $\LGtwo$ is the
orthogonal direct sum of the subspaces
$\Lambda(A(K)) \subseteq \LGtwo$, where $K\in M_{2}(I)$, because
$\phi(A(K)^{*}A(L)) = 0$ if $K\neq L$ by \ref{}.  In particular, there
exist  operators
$\lambda_{k},\lambda_{k}^{\op},\rho_{k},\rho_{k}^{\op}\in {\cal
  B}(\LGtwo)$ for each $k\in I$ such that
\begin{align*}
  \lambda_{k}\Lambda(a)&= \Lambda(\lambda_{k} a), &
  \lambda^{\op}_{k}\Lambda(a) &= \Lambda(a\lambda_{k}), &
  \rho_{k}\Lambda(a) &= \Lambda(\rho_{k}a), &
  \rho_{k}^{\op}\Lambda(a) &= \Lambda(a\rho_{k})
\end{align*}
for all $a\in A$, and faithful, normal $*$-homomorphisms
\begin{align} \label{eq:vn-lambda-rho}
  \lambda,\rho \colon l^{\infty}(I) \to
  {\cal B}(\LGtwo)
\end{align}
that send the delta function at
$k\in I$ to the operators $\lambda_{k}$ or $\rho_{k}$, respectively. 

 Define $\vnE,\overline{G} \in {\cal B}(\LGtwo \otimes
\LGtwo)$ by
\begin{align*}
  \vnE &:=\sum_{k} \rho_{k} \otimes \lambda_{k}, & 
  \overline{G} &:= \sum_{k} \rho_{k}^{\op} \otimes \rho_{k},
\end{align*}
where the sums converge with respect to the strong operator
topology.
\begin{Lem} \label{lemma:partial-isometry}
There exists a unique partial isometry $V$ on $\LGtwo \otimes \LGtwo$
such that
\begin{align*}
  V(\Lambda(a) \otimes \Lambda(b)) = \Lambda(a_{(1)}) \otimes \Lambda(a_{(2)}b)
\end{align*}
for all $a,b\in A$. Its range and domain projections are given by $VV^{*} = \vnE$
and $V^{*}V = \overline{G}$.
\end{Lem}
\begin{proof}
  Let $a,b \in A$. Since $\Delta$ is a $*$-homomorphism and $\phi$ is
invariant,
  \begin{align*}
    \langle \Lambda(a_{(1)}) \otimes
    \Lambda(a_{(2)}b)|\Lambda(a'_{(1)}) \otimes
    \Lambda(a'_{(2)}b')\rangle &=
    \phi(a_{(1)}^{*}a'_{(1)})\phi(b^{*}a_{(2)}^{*}a'_{(2)}b') \\
    &= \sum_{p}
    \phi(b^{*}\rho_{p}\phi(\rho_{p}a^{*}a'\rho_{p})\rho_{p}b') \\
    & =\sum_{p} \langle\Lambda(a\rho_{p}) \otimes \Lambda(\rho_{p}b) |
    \Lambda(a'\rho_{p}) \otimes \Lambda(b'\rho_{p})\rangle.
  \end{align*}
  Now, the assertion follows from Proposition \ref{prop:riti}.
\end{proof}

\begin{Prop} \label{prop:gns} Let $\mathscr{G}$ be a partial compact
  quantum group with underlying total $*$-algebra $A$ and associated
  Hilbert space $\LGtwo$. Then there exists a unique $*$-homomorphism
  $\pi_{r}\colon A \to {\cal B}(\LGtwo)$ such that
  $\pi_{r}(a)\Lambda(b)=\Lambda(ab)$ for all $a,b\in A$, and this
  $\pi_{r}$ is faithful.
\end{Prop}
\begin{proof} 
  Let $a,c \in A$. Then the formula $x \mapsto \langle
\Lambda(c) | x\Lambda(a)\rangle$ defines a bounded linear functional
  $\omega_{\Lambda(c),\Lambda(a)}$ on ${\cal B}(\LGtwo)$ and a
  straightforward computation shows that
  \begin{align} \label{eq:vn-slice}
    (\omega_{\Lambda(c),\Lambda(a)}\otimes \id)(V)\Lambda(b) =
    \Lambda(\varphi(c^*a_{(1)})a_{(2)}b)
  \end{align}
  for all $b\in A$. Therefore, left multiplication by
  $\varphi(c^*a_{(1)})a_{(2)}$ extends to a bounded linear operator on $\LGtwo$.
 Since $(A\otimes 1)\Delta(A) = (A\otimes
  A)\Delta(1)$ by Proposition \ref{prop:riti} and $\phi$ is
  normalized,  elements of the form $\phi(c^{*}a_{(1)})a_{(2)}$ span
  $A$. 
\end{proof}
\begin{Cor}
  Let $\mathscr{G}$ be a partial compact quantum group with underlying
  total algebra $A$. Then the
  canonical $*$-homomorphism $\pi_{u} \colon A \to\CuG$ is injective.
\end{Cor}
\begin{proof}
  The injective $*$-homomorphism $\pi_{r}$ factorises through
  $\pi_{u}$.
\end{proof}
Given a partial compact quantum group $\mathscr{G}$, we call
$(\LGtwo,\Lambda,\pi)$ the \emph{associated GNS-construction} and denote by
\begin{align}
  \CrG &\subseteq {\cal B}(\LGtwo) &&\text{and} & \LGinf &\subseteq {\cal B}(\LGtwo)
\end{align}
the $C^{*}$-algebra and the von Neumann algebra generated by $\pi_{r}(A)
\subseteq \LGtwo$, respectively, and identify $M(\CrG)$ with a
$C^{*}$-subalgebra of $\LGtwo$.  Since $\pi_{r}$ extends to a
$*$-homomorphism on $\CuG$, we get a sequence of $*$-homomorphisms
\begin{align*}
A \hookrightarrow \CuG \to 
  \CrG \hookrightarrow M(\CrG) \hookrightarrow
\LGinf \hookrightarrow {\cal B}(\LGtwo).
\end{align*}
Note that
 the $*$-homomorphisms $\lambda,\rho$ in
\eqref{eq:vn-lambda-rho} send $l^{\infty}(I)$ to $M(\CrG)$, and that
\begin{align*}
  \vnE \in M(\CrG \otimes \CrG) \subseteq \LGinf \otimes \LGinf
  \subseteq {\cal B}(\LGtwo \otimes \LGtwo),
\end{align*}
where $\otimes$ denotes the minimal tensor product
of $C^{*}$-algebras, the tensor product of von Neumann algebras, and
the tensor product of Hilbert spaces, respectively.

Consider the map 
\begin{align*}
  \vnDelta \colon \LGinf \to {\cal B}(\LGtwo \otimes \LGtwo), \ x
  \mapsto V(x \otimes 1)V^{*}.
\end{align*}
\begin{Lem} \label{lemma:vn-delta}
  \begin{enumerate}
  \item $\vnDelta(\pi_{r}(a)) (\Lambda(b) \otimes \Lambda(c)) =
    \Lambda(a_{(1)}b) \otimes \Lambda(a_{(2)}b)$ for all $a,b,c\in A$;
  \item $\vnDelta$ is a normal, faithful $*$-homomorphism;
  \item  $\vnDelta(\CrG) \subseteq \vnE M(\CrG \otimes
  \CrG)\vnE$ and $\vnDelta(\LGinf) \subseteq \vnE(\LGinf \otimes
  \LGinf)\vnE$.
  \end{enumerate}
\end{Lem}
\begin{proof}
  The equation in 1.{} is easily verified. The map $\vnDelta$ is
  normal by construction, a $*$-homo\-morphism by 1.{}, and faithful
  because $\vnDelta(x)=0$ implies $x\otimes 1=0$ on
  $V^{*}V(L^{2}(\mathscr{G}) \otimes L^{2}(\mathscr{G}))$ and hence
  $x=0$ on $\bigoplus_{k}
  \rho_{k}^{\op}L^{2}(\mathscr{G})=L^{2}(\mathscr{G})$. Finally, 3.\
  follows from the relation $\Delta(a)=E\Delta(a)E$, which holds for
  all $a\in A$.
\end{proof}


Next, we lift the invariant functional $\phi$ of $\mathscr{G}$ to
$\LGinf$ and define associated operator-valued weight
$T_{\lambda},T_{\rho}$ from $\LGinf$ to $l^{\infty}(I)$. Since $\phi$ is normalized, each
$\Lambda(\lambda_{k},\rho_{m})$ is a unit vector and the associated
vector functional
\begin{align*}
  \vnphic{k}{m}\colon \LGinf \to  \C, \quad x \mapsto \langle\Lambda(\lambda_{k}\rho_{m})|x\Lambda(\lambda_{k}\rho_{m})\rangle
\end{align*}
is a state.  Then the formulas
\begin{align} \label{eq:vn-weights}
  \vnphi(x) &:= \sum_{k,m} \vnphic{k}{m}(x), &
    T_{\lambda}(x)&:= \sum_{k,m}
\vnphic{k}{m}(x)\lambda_{k}, & 
T_{\rho}(x)&:=
    \sum_{k,m} \vnphic{k}{m}(x)\rho_{m},
\end{align}
where $x\in \LGinf_{+}$, define a a normal semi-finite weight $\vnphi$
on $\LGinf$ and normal semi-finite conditional expectations $T_{\lambda}$ and
$T_{\rho}$ from $\LGinf$ to $\lambda(l^{\infty}(I))$ and
$\rho(l^{\infty}(I))$, respectively. These maps are determined by
their restrictions to $\pi_{r}(A)$:
\begin{Lem} \label{lemma:vn-weights-unique} The normal weight $\vnphi$ and
  the normal conditional expectations $T_{\lambda},T_{\rho}$ satisfy $\pi_{r}(A) \subseteq
  \mathfrak{M}_{\vnphi} \cap \mathfrak{M}_{T} \cap \mathfrak{M}_{T'}$
  and
  \begin{align*}
    \vnphi(\pi_{r}(a))&= \phi(a), & T_{\lambda}(\pi_{r}(a))\Lambda(b) &=
    \sum_{k}\Lambda(\phi(\lambda_{k}a)\lambda_{k}b), &
    T_{\rho}(\pi_{r}(b))\Lambda(b) &= \sum_{m}\Lambda(\phi(\rho_{m}a)\rho_{m}b)
  \end{align*}
  for all $a,b\in A$, and are uniquely determined by these equations. 
\end{Lem}
\begin{proof}
  The equations follow immediately from the definition and the
  relation $\phi(a)=\sum_{k,m}
  \phi(\lambda_{k}\rho_{m}a\lambda_{k}\rho_{m})$, see \ref{}. To prove
  uniqueness, observe that the $p_{k,m}:=\pi_{r}(\lambda_{k}\rho_{m})$ are
  pairwise orthogonal projections in $\mathfrak{M}_{\vnphi}
  \cap \mathfrak{M}_{T_{\lambda}} \cap \mathfrak{M}_{T_{\rho}}$
  summing up to $1$, whence $\vnphi$, $T_{\lambda}$ and $T_{\rho}$ are
  the sums of the bounded linear maps that send an $x\in \LGinf_{+}$
  to $\vnphi(p_{k,m}xp_{l,n})$, $T_{\lambda}(p_{k,m}xp_{l,n})$, or
  $T_{\rho}(p_{k,m}xp_{l,n})$, respectively, which are determined by
  their restriction to $\pi_{r}(A)$.
\end{proof}

Invariance of $\phi$ implies invariance of $\vnphi$ as follows.
\begin{Prop} \label{prop:vn-invariance}
  Let $\mathscr{G}$ be a partial compact quantum group. Then for all
  $x\in \LGinf_{+}$, the
  normal, semi-finite weight $\vnphi$ on $\LGinf$ satisfies
  \begin{align*}
    (\id \otimes \vnphi)(\vnDelta(x)) &=  T_{\lambda}(x), &
    (\vnphi \otimes \id)(\vnDelta(x)) &= T_{\rho}(x).
  \end{align*}
\end{Prop}
\begin{proof}
  Let  $a \in A$. Then 
 \eqref{eq:integral} and the relation
  $\vnphic{k}{m}\circ \pi = \phic{k}{m}$ imply
  \begin{align*}
    (\id \otimes \vnphic{l}{m})(\vnDelta(\pi_{r}(a))) &= \sum_{k}
    \vnphic{k}{m}(\pi_{r}(a)) \lambda_{k}\rho_{l}.
  \end{align*}
  Since each $\vnphic{k}{m}$ is a vector state and $\pi_{r}(A)$ is
  weakly dense in $\LGinf$,  this equations
  remains true if we replace $\pi_{r}(a)$ by arbitrary $x\in
  \LGinf$. Summing over $l$ and $m$, we  obtain the first equation
  which we have to prove. The second one follows similarly.
\end{proof}
The next result gives a fairly complete description of the objects of
Tomita-Takesaki theory associated to $\vnphi$.
\begin{Lem} \label{lemma:vn-hilbert} The subspace
  $\Lambda(A) \subseteq \LGtwo$ is a Hilbert algebra with respect to
  the operations $\Lambda(a)\Lambda(b)=\Lambda(ab)$ and
  $\Lambda(a)^{*}= \Lambda(a^{*})$ for all $a,b\in A$, and a Tomita
  algebra with respect to the family of operators $\nabla_{z}$ given
  by $\nabla_{z}\Lambda(a)=\Lambda(\sigma_{z}(a))$ for all $a\in A$,
  $z\in \C$.  The associated left von Neumann
  algebra is $\LGinf$, the associated normal, semifinite, faithful
  weight is $\vnphi$, the modular operator  $\Delta_{\vnphi}$ is the
  closure of $\nabla_{-i}$,  the modular conjugation $J_{\vnphi}$ is
  given by $J_{\vnphi}\Lambda(a)=\Lambda(\sigma_{i/2}(a)^{*})$ for all
  $a\in A$, and the modular automorphism group $\sigma^{\vnphi}$
  satisfies $\sigma^{\vnphi}_{t} \circ \pi_{r} = \pi_{r} \circ
  \sigma_{t}$ for all $t\in \R$.
\end{Lem}
\begin{proof}
  We first show that $\Lambda(A)$ is a Hilbert algebra. Indeed, the
  map $\pi_{r}(a)\colon \Lambda(b) \to \Lambda(ab)$ is bounded for
  each $a \in A$ by Proposition \ref{prop:gns}, and the involution is
  pre-closed because  for all $a,b \in A$,
  \begin{align*}
    \langle \Lambda(a)|\Lambda(b^{*})\rangle = \phi(a^{*}b^{*}) =
    \phi(b^{*}\sigma(a^{*})) = \langle
    \Lambda(b)|\Lambda(\sigma(a^{*}))\rangle
  \end{align*}

To see that $\Lambda(A)$ and $(\nabla_{z})_{z}$ form a Tomita
  algebra, we have to verify that the map
  $z\mapsto
  \langle \Lambda(a)|\nabla_{z}\Lambda(b)\rangle =
  \phi(a^{*}\sigma_{z}(b))$ is entire for all $a,b\in A$ and that
  \begin{align*}
    \nabla_{z}\Lambda(a)^{*} &= \nabla_{\overline{z}}\Lambda(a)^{*}, &
    \langle \Lambda(a)|\Lambda(b)\rangle &= \langle
    \nabla_{-\overline{z}}\Lambda(a) |\Lambda(b)\rangle, & \langle
    \Lambda(a)^{*}|\Lambda(b)^{*}\rangle = \langle \Lambda(b)|\nabla_{-i}\Lambda(a)\rangle
  \end{align*}
  for all $a,b\in A$, $z\in \C$. But all of this follows immediately
  from Corollary \ref{cor:rep-characters}.


  The left von Neumann algebra of $\Lambda(A)$ is
  $\pi_{r}(A)''=\LGinf$ and the associated weight $\tilde\phi$
  satisfies $\tilde
  \phi(\pi_{r}(a^{*}a))=\langle\Lambda(a)|\Lambda(a)\rangle =
  \phi(a^{*}a)$ for all $a\in A$. By Lemma
  \ref{lemma:vn-weights-unique}, it coincides with $\vnphi$.  By
  \cite{} [Takesaki:2, Thm. VI.2.2 and its proof], the modular
  operator $\Delta_{\vnphi}$ is the closure of $\nabla_{-i}$ and the
  modular automorphism group is implemented by $(\nabla_{t})_{t}$. 
\end{proof}
 The general theory of Hilbert algebras \cite{} implies now:
\begin{Prop} \label{prop:hilbert-algebra} Let $\mathscr{G}$ be a
  partial compact quantum group. Then the extension $\vnphi$ of the
  invariant functional to $\LGinf$ is faithful. \qed
\end{Prop}
\begin{Rem}
  Without using the theory of Hilbert algebras, one could also check
  directly that the formula for $J_{\vnphi}$ defines an anti-linear
  isometry, that $J_{\vnphi}\pi_{r}(A)J_{\vnphi}$ commutes with $\pi_{r}(A)$ and hence
  with $\LGinf$, and that the family
  $(\Lambda(\lambda_{k}\rho_{m}))_{k,m}$ is cyclic for
  $J\pi_{r}(A)J$. Then this family is separating for $\LGinf$ and
  $\vnphi$ is faithful.
\end{Rem}

The scaling group $\tau$ and the unitary antipode $R$ of $\mathscr{G}$
can easily be lifted to $\CrG$ and $\LGinf$ using the following
result. Let us call a conjugate-linear map on a Hilbert space an
\emph{anti-symmetry} if it is isometric and its square is the
identity.
\begin{Lem} \label{lemma:vn-implementation}
  There exist a unique anti-symmetry $I$ and a strongly continuous
  one-parameter group $P=(P_{t})_{t}$ on
  on $\LGtwo$ such that for all $a\in A$, $t\in \R$,
  \begin{align*}
 I\Lambda(a) &=
    \Lambda(R(a)^{*}), & P_{t}\Lambda(a) &= \Lambda(\tau_{t}(a)).
  \end{align*}
\end{Lem}
\begin{proof}
  Corollary \ref{cor:rep-characters} implies that the formulas above
  define an anti-symmetry $I$ and unitaries $P_{t}$; for
  example, $\|I\Lambda(a)\|^{2})=\phi(R(a)R(a)^{*})=
  \phi(a^{*}a)=\|\Lambda(a)\|^{2}$, and $I^{2} = \id$
  because $*\circ R \circ * \circ R=  R^{2}=\id$. By A.1 in \cite{}
  [Takesaki:2] and Corollary  \ref{cor:rep-characters} 5., elements of
  $\Lambda(A)$ are analytic with respect to $P$; in
  particular,  $P$ is strongly continuous.
\end{proof}
\begin{Prop}
  Let $\mathscr{G}$ be an $I$-partial compact quantum group.  
  \begin{enumerate}
  \item  There exists a unique $*$-anti-automorphism $\vnR$ of
    $\LGinf$ such that $\vnR \circ \pi_{r} = \pi_{r} \circ R$.
    This $\vnR$ restricts to a $*$-anti-automorphism of
    $\CrG$.
  \item  There exists a unique strongly continuous
    one-parameter group $\vntau$ on $\LGinf$ such that $\vntau_{t}
    \circ \pi_{r} = \pi_{r} \circ \theta_{-it,it}$ for all $t\in \R$,
    and this $\vntau$ restricts to a strongly continuous one-parameter
    group on $\CrG$.
  \end{enumerate}
\end{Prop}
\begin{proof}
    Short calculations show that the maps $\vnR \colon x \mapsto
  Ix^{*}I$ and $\vntau_{t} \colon x \mapsto P_{t}xP_{t}^{*}$ have the
  desired properties.
\end{proof}
Note that the relations \eqref{eq:scaling-modular-delta} and
\eqref{eq:unitary-antipode} can be lifted to $\CrG$ and $\LGinf$ by
continuity.  The next result will allow us to relate $\vnR$ to the
unitary antipode of the measured quantum groupoid that we are going to
construct.
\begin{Lem} \label{lemma:vn-r-characterisation}
For all $a,b\in A$,
\begin{align*}
  \vnR(\id \otimes
  \omega_{J\Lambda(b),J\Lambda(b)})(\vnDelta(\pi(a^{*}a))) = (\id
  \otimes \omega_{J\Lambda(a),J\Lambda(a)})(\vnDelta(\pi(b^{*}b))).
\end{align*}
\end{Lem}
\begin{proof}
Let $c=a^{*}a$ and $d=b^{*}b$. 
A short calculation using \eqref{eq:modular} shows that  the right hand side is equal to
  \begin{align*}
    d_{(1)}\phi(\sigma_{i/2}(a)d_{(2)}\sigma_{i/2}(a)^{*})
    = d_{(1)}\phi(\sigma_{i/2}(c)d_{(2)}).
  \end{align*}
By Lemma \ref{lemma:strong-invariance} and
  \eqref{eq:scaling-modular-delta}, \eqref{eq:modular},  this equals 
  $S(\tau_{i/2}(c_{(1)}))
    \phi(\sigma_{i/2}(c_{(2)})d)$
which is the  left hand side.
\end{proof}

The operator-algebraic structures constructed so far fit into the
theory of measured quantum groupoids as follows.

Denote by $\nu$ the
normal, faithful, semifinite weight on $l^{\infty}(I)$ given by
\begin{align} \label{eq:vn-nu}
  \nu(f) &=\sum_{k} f(k) \quad \text{for all } f\in l^{\infty}(I)_{+}.
\end{align}
Then the relative tensor product of $\LGtwo$ with itself,
relative to the representations $\rho,\lambda$ of $l^{\infty}(I)$ and
the weight $\nu$, takes the simple form
\begin{align*}
\LGinf \otimesrl \LGinf \cong
  \bigoplus_{k} (\rho_{k}\LGtwo \otimes \lambda_{k}\LGtwo) =
  \vnE(\LGtwo \otimes \LGtwo),
\end{align*}
see \cite{},  the relative tensor product
of operators $S\in \rho(l^{\infty}(I))'$ and $T \in
\lambda(l^{\infty}(I))'$ gets identified with the compression
\begin{align*}
S \otimesrl T \equiv
  \vnE(S \otimes
  T) = (S \otimes T)\vnE \subseteq {\cal B}(\vnE(\LGtwo
  \otimes \LGtwo)),
\end{align*}
and the fiber product of  $  \LGinf$ with itself, relative to $\rho$
and $\lambda$,  gets identified with
\begin{align} \label{eq:vn-fiber}
  \begin{aligned}
    \LGinf \astrl \LGinf &= (\LGinf' \otimesrl \LGinf')' \\ &\equiv
    (\vnE(\LGinf' \otimes \LGinf'))' = \vnE(\LGinf \otimes
    \LGinf)\vnE.
  \end{aligned}
\end{align} 
By Lemma \ref{lemma:vn-delta} 3., we can co-restrict $\vnDelta$ to  a
normal, faithful $*$-homomorphism
\begin{align*}
  \tilde\Delta \colon \LGinf \to   \LGinf \astrl \LGinf.
\end{align*}
We now obtain a Hopf-von Neumann bimodule in the
sense of \cite{}.
\begin{Prop}
  Let $\mathscr{G}$ be an $I$-partial compact quantum group. Then
  \begin{enumerate}
  \item $\tilde\Delta(\lambda(x)) = \lambda(x) \otimesrl 1$ and
    $\tilde\Delta(\rho(x)) = 1 \otimesrl \rho(x)$ for all $x\in
    l^{\infty}(I)$, and
  \item $(\tilde\Delta \ast \id)\circ \tilde\Delta = (\id \ast \tilde\Delta)
    \circ \tilde\Delta$.
  \end{enumerate}
  In particular, $(l^{\infty}(I),\LGinf, \lambda,\rho,\tilde\Delta)$ is a
  Hopf-von Neumann bimodule.
\end{Prop}
\begin{proof}
Assertion 1.\ follows from \eqref{eq:delta-lambda-rho} and
Lemma \ref{lemma:vn-delta} 1.\ and ensures that  the $*$-homo\-morphisms
\begin{align*}
  \tilde\Delta \ast \id, \id \ast \tilde\Delta \colon  \LGinf \astrl \LGinf
  \to \LGinf \astrl \LGinf \astrl \LGinf
\end{align*}
are well-defined.  As in
\eqref{eq:vn-fiber}, we can identify
\begin{align*}
 \LGinf \astrl \LGinf \astrl \LGinf \cong \vnE^{(2)}(\LGinf
  \otimes \LGinf \otimes \LGinf)\vnE^{(2)},
\end{align*}
where $\vnE^{(2)}=(\vnE \otimes 1)(1 \otimes \vnE)$, and then the
$*$-homomorphisms become restrictions of the maps $\tilde\Delta \otimes
\id$ and $\id \otimes \tilde\Delta$, respectively. Now, 2.\ follows from
Lemma \ref{lemma:vn-delta} 1.\ and co-associativity of $\Delta$.
\end{proof}
This Hopf-von Neumann bimodule is a measured quantum groupoid in the
sense of \cite{}.
\begin{Theorem} \label{theorem:vn-measured} Let $\mathscr{G}$ be an
  $I$-partial compact quantum group. Then the Hopf-von Neumann
  bimodule $(l^{\infty}(I),\LGinf,\lambda,\rho,\tilde\Delta)$ and the
  weights $(T_{\lambda},T_{\rho}$ and $\nu$ defined in
  \eqref{eq:vn-weights} and \eqref{eq:vn-nu} form a measured quantum
  groupoid.  It is unimodular and its unitary antipode and scaling
  group coincide with $\vnR$ and $\vntau$, respectively.
\end{Theorem}
\begin{proof}
  First, observe that the  the modular
  automorphism groups of the weights $\nu \circ \lambda^{-1} \circ
  T_{\lambda}$ and $\nu \circ \rho^{-1} \circ T_{\rho}$ commute  because the two
  compositions coincide with $\vnphi$. Next, we need  to show that
  $T_{\lambda}$ is left-invariant in the sense that
  \begin{align*}
   (\id \underset{\nu}{_{\rho}\ast_{\lambda}} \vnphi)(\tilde\Delta(x)) = T_{\lambda}(x) 
  \end{align*}
  for all $x\in \LGinf_{+}$. But it is easy to see that the left hand
  side coincides with $(\id \ast \vnphi)(\vnDelta(x))$ so that the
  equation above follows from Proposition
  \ref{prop:vn-invariance}. Likewise $T_{\rho}$ is right-invariant in
  the appropriate sense. We thus obtain a measured quantum groupoid as
  claimed.  Denote by $\tilde R$ its unitary antipode and by
  $\tilde\tau$ its scaling group.

  Let us prove that $\tilde \tau_{t}=\vntau$ for all $t\in \R$. By
  \cite{} and \eqref{eq:scaling-modular-delta},
  \begin{align*}
    (\tilde \tau_{t} \astrl \sigma^{\vntau}_{t}) \circ \tilde \Delta
    &=\tilde \Delta \circ \sigma^{\vntau}_{t}, & (\vntau_{t} \otimes
    \sigma^{\vntau}_{t}) \circ \vnDelta &= \vnDelta \circ \vntau_{t}.
  \end{align*}
  The second equation implies that the first one remains true if we
  replace $\tilde\tau_{t}$ by $\vntau_{t}$.  Using Theorem A.7 in
  \cite{} [enock:action], we can conclude that $\tilde
  \tau_{t}=\vntau_{t}$.

 To prove that $\tilde R=\vnR$, we use the relations
  \begin{align*}
    \tilde R(\id \underset{\nu}{_{\rho} \ast_{\lambda}}
    \omega_{J\Lambda(b),J\Lambda(b)})(\vnDelta(\pi(a^{*}a))) &= (\id
    \underset{\nu}{_{\rho} \ast_{\lambda}}
    \omega_{J\Lambda(a),J\Lambda(a)})(\vnDelta(\pi(b^{*}b)))
  \end{align*}
from \cite{} and Lemma \ref{lemma:vn-r-characterisation}.
\end{proof}

%%% Local Variables: 
%%% mode: latex
%%% TeX-master: "dyn-suq-main"
%%% End: 







\bibliographystyle{abbrv}
\bibliography{references}

\end{document}