% Remarks:
% Podles sphere can be defined as an algebra in $\Rep(SU_q(2))$. Hence, it should make sense as an algebra under the forgetful functor to SU_q(2)-dynamical. By duality for coideals, the same should hold for passage to SU_q(1,1) in fact...
%
% link with article `Racah - Wigner quantum 6j Symbols, Ocneanu Cells for AN diagrams and quantum groupoids' by Coquereaux?
% Ref to Schauenburg that face algebras are weak Hopf algebras
% Thanks to Makoto, Piotr (?), Leonid
% Cf. Enocks quantum groupoids of compact type
% Donin, J.(IL-BILN); Mudrov, A.(IL-BILN), Quantum groupoids and dynamical categories. 
% Leonid, cf. talk www.fields.utoronto.ca/programs/scientific/13-14/.../Vainerman.pdf
% Stokman vertex irf babelon cocycle twist
% Level 2 Hecke algebras Brundan
% Concerning locally unital algebras ask J. Vercruysse on recent work 
% Say construction groupoids from forgetful functors on temperley-Lieb already in Enock-Ostrik remark
\documentclass[11pt]{article}

\usepackage{hyperref}
\usepackage{fixme}
\usepackage{mathrsfs}
\usepackage[a4paper]{geometry}
\usepackage{amssymb, amsthm, amsfonts, amsxtra, amsmath}
\usepackage{latexsym}
\usepackage{mathabx}
\usepackage{enumitem}
%\usepackage[all]{xy}
%\usepackage{graphics}
\usepackage{pdfpages}
\usepackage{epic}
\usepackage{parskip} % paragraphs have no indents and vertical spacings inbetween
\makeatletter % need this to avoid the conflict between amsthm and parskip
\def\thm@space@setup{%
  \thm@preskip=\parskip \thm@postskip=0pt
}
\makeatother

%\theoremstyle{change}

\newcommand{\co}{\mathrm{co}}
\newcommand{\Corep}{\mathrm{Corep}}
\newcommand{\sff}{\textrm{s.f.~}}
\newcommand{\sfs}{\mathrm{sfs}}
\newcommand{\sfd}{\mathrm{sfd}}
\DeclareMathOperator{\Hom}{Hom}
\DeclareMathOperator{\img}{img}

\DeclareMathOperator{\id}{id}
\DeclareMathOperator{\ext}{\mathrm{e}}
\DeclareMathOperator{\can}{\mathrm{can}}
\DeclareMathOperator{\op}{\mathrm{op}}
\DeclareMathOperator{\fin}{\mathrm{f}}
\DeclareMathOperator{\Pol}{\mathrm{P}}
\DeclareMathOperator{\End}{\mathrm{End}}
\DeclareMathOperator{\Par}{\mathrm{Par}}
\DeclareMathOperator{\reg}{\mathrm{reg}}
\DeclareMathOperator{\sgn}{\mathrm{sgn}}
\DeclareMathOperator{\Zz}{\mathrm{Z}}
\DeclareMathOperator{\Ran}{\mathrm{Ran}}
\DeclareMathOperator{\hol}{\mathrm{hol}}
\DeclareMathOperator{\Ind}{\mathrm{Ind}}
\DeclareMathOperator{\Ker}{\mathrm{Ker}}
\DeclareMathOperator{\Char}{\mathrm{Char}}
\DeclareMathOperator{\dyn}{\mathrm{dyn}}
\DeclareMathOperator{\Spec}{\mathrm{Spec}}
\DeclareMathOperator{\adj}{\mathrm{adj}}

\newcommand{\Circt}{\underset{I}{\mathop{\ooalign{$\ovoid$\cr\hidewidth\raise-.05ex\hbox{$\scriptstyle\mathsf T\mkern3.5mu$}\cr}}}} % Woronowicz style tensor product, USUAL SIZE
\newcommand{\Circtv}[1]{\underset{#1}{\mathop{\ooalign{$\ovoid$\cr\hidewidth\raise-.05ex\hbox{$\scriptstyle\mathsf T\mkern3.5mu$}\cr}}}} % Woronowicz style tensor product, USUAL SIZE
\newcommand{\smCirct}{\mathop{\ooalign{$\scriptstyle\ovoid$\cr\hidewidth\raise-.05ex\hbox{$\scriptscriptstyle\mathsf T\mkern2.8mu$}\cr}}}  % Woronowicz style tensor product, SCRIPT SIZE

\newcommand{\nc}{\R}
\newcommand{\g}{\mathfrak{g}}
\newcommand{\h}{\mathfrak{h}}

\newcommand{\kk}{\mathfrak{k}}
\newcommand{\ttt}{\mathfrak{t}}
\newcommand{\p}{\mathfrak{p}}
\newcommand{\n}{\mathfrak{n}}
\newcommand{\llll}{\mathfrak{l}}
\newcommand{\uu}{\mathfrak{u}}
\newcommand{\bb}{\mathfrak{b}}
\newcommand{\q}{\mathfrak{q}}
\newcommand{\su}{\mathfrak{su}}
\newcommand{\ssl}{\mathfrak{sl}}
\newcommand{\SSL}{\mathrm{SL}}
\newcommand{\so}{\mathfrak{so}}
\newcommand{\spp}{\mathfrak{sp}}
\newcommand{\G}{\mathbb{G}}
\newcommand{\e}{\mathfrak{e}}
\newcommand{\s}{\mathfrak{s}}
\newcommand{\C}{\mathbb{C}}
\newcommand{\R}{\mathbb{R}}
\newcommand{\Z}{\mathbb{Z}}
\newcommand{\N}{\mathbb{N}}
\newcommand{\X}{\mathbb{X}}
\newcommand{\Y}{\mathbb{Y}}
\newcommand{\Ss}{\mathbb{S}}
\newcommand{\ZZ}{\mathscr{Z}}
\newcommand{\ad}{\mathrm{ad}}
\newcommand{\Hsp}{\mathcal{H}}
\newcommand{\qn}[2]{\lbrack #1 \rbrack_{#2}}
\newcommand{\fqn}[2]{\lbrack #1 \rbrack_{#2}!}
\newcommand{\bqn}[3]{\left\lbrack \begin{array}{c} \!#1\! \\ \!#2\! \end{array}\right\rbrack_{#3}}
\newcommand{\Tr}{\mathrm{Tr}}
\newcommand{\RR}{\mathcal{R}}
\newcommand{\rd}{\mathrm{d}}
\newcommand{\res}{\mathrm{res}}
\newcommand{\cop}{\mathrm{cop}}
\newcommand{\opp}{\mathrm{op}}
\newcommand{\coop}{\mathrm{coop}}
\newcommand{\Rm}{\mathcal{R}}
\newcommand{\wt}{\mathrm{wt}}
\newcommand{\Ad}{\mathrm{Ad}}
\newcommand{\CatC}{\mathcal{C}}
\newcommand{\CatD}{\mathcal{D}}
\newcommand{\Corr}{\mathrm{Corr}}
\newcommand{\Hilb}{\mathrm{Hilb}}
\newcommand{\Star}[2]{{}_{#1}\!*_{#2}}
\newcommand{\vot}{\bar{\otimes}}
\newcommand{\A}{\mathcal{B}}
\newcommand{\Aa}{\mathscr{B}}
\newcommand{\Mor}{\mathrm{Mor}}
\newcommand{\alg}{\mathrm{alg}}
\newcommand{\Gg}{\mathscr{G}}
\newcommand{\ev}{\mathrm{ev}}
\newcommand{\Rtimes}{\underset{\R}{\times}}
\newcommand{\Rb}{\R^{\bullet}}
\newcommand{\vtimes}{\bar{\otimes}}
\newcommand{\Rr}{\mathscr{R}}
\newcommand{\Tt}{\mathscr{T}}
\newcommand{\Fun}{\mathrm{Fun}}
\newcommand{\Ff}{\Fun_{\fin}}
%\newcommand{\fin}{\mathrm{fin}}
\newcommand{\iitimes}{\underset{I}{\otimes}}
\newcommand{\itimes}{\underset{I^2}{\otimes}}
\newcommand{\osum}[1]{\underset{#1}{\sum}^{\oplus}}
\newcommand{\osumc}[1]{\underset{#1}{\sum}^{\bar{\oplus}}}
\newcommand{\oplusc}{\bar{\oplus}}
\newcommand{\wDelta}{\widetilde{\Delta}}
\newcommand{\f}{\mathrm{fin}}
%\newcommand{\Hilb}{\mathrm{Hilb}}
\newcommand{\Rho}{\mathrm{P}}
\newcommand{\Rep}{\mathrm{Rep}}
\newcommand{\DA}{\mathcal{A}}
%\newcommand{\Circt}{\mathop{\ooalign{$\ovoid$\cr\hidewidth\raise-.05ex\hbox{$\scriptstyle\mathsf T\mkern3.5mu$}\cr}}} % Woronowicz style tensor product, USUAL SIZE
\newcommand{\CoRep}{\mathrm{Corep}}
\newcommand{\even}{\mathrm{even}}
\newcommand{\odd}{\mathrm{odd}}
\newcommand{\fd}{\mathrm{fd}}

\newcommand{\GrHA}[3]{#1{\begin{pmatrix} #2,  #3\end{pmatrix}}}% Horizontal grading ordinary style, with argument
\newcommand{\Grs}[3]{#1{\begin{pmatrix} #2,  #3\end{pmatrix}}}

\newcommand{\GrDA}[3]{\;{}_{\;#2}#1_{#3}} % Horizontal grading bottom style, with argument
%\newcommand{\Grd}[3]{\;{}_{\;#2}#1_{#3}}

\newcommand{\GrVA}[3]{#1{\tiny {\begin{pmatrix} #2\\#3\end{pmatrix}}}} % Vertical grading ordinary style, with argument
\newcommand{\Grt}[3]{#1{\tiny {\begin{pmatrix} #2\\#3\end{pmatrix}}}} 

\newcommand{\GrRA}[3]{#1^{#2}_{#3}} % Vertical grading right style, with argument

\newcommand{\Unit}{\mathbf{1}}
\newcommand{\UnitC}[2]{\Grt{\mathbf{1}}{#1}{#2}} 
\newcommand{\Grru}[2]{{\tiny \begin{pmatrix} #1 \\ #2\end{pmatrix}}}

\newcommand{\eGr}[5]{#1{{\tiny \begin{pmatrix} #2 \quad #3 \\ #4 \quad #5\end{pmatrix}}}}

\newcommand{\pma}[4]{\begin{pmatrix} #1 \quad #2 \\ #3 \quad #4\end{pmatrix}}
\newcommand{\pmat}[4]{{\tiny \begin{pmatrix} #1 \quad #2 \\ #3 \quad #4\end{pmatrix}}}

\newcommand{\UT}[2]{#1{\tiny #2 }}
\newcommand{\Gr}[5]{\;{}^{\;#2}_{#4}#1_{#5}^{#3}}%TODO: better typesetting
%\newcommand{\Gr}[5]{\UT{#1}{\begin{pmatrix} #2\quad #3 \\ #4 \quad #5\end{pmatrix}}}
%\newcommand{\Gr}[5]{\UT{#1}{\begin{pmatrix} \, #2\;\\ #3 \qquad #4 \\ \,#5\;\end{pmatrix}}}
\newcommand{\Grl}[3]{\;{}^{\;#2}_{#3}#1}%TODO: better typesetting
\newcommand{\Gru}[3]{{}^{\;#2}#1^{#3}}
\newcommand{\Grd}[3]{{}_{\;#2}#1_{#3}}
\newcommand{\gr}[5]{\;{}^{\;#2}_{#4}#1_{#5}^{#3}}%TODO: better typesetting
\newcommand{\eGrr}[3]{#1_{{\tiny \left(#2, #3\right)}}}
\newcommand{\eGrt}[4]{#1{{\tiny \begin{pmatrix} #2 \\ #3 \\ #4 \end{pmatrix}}}}
\newcommand{\Grr}[4]{\begin{pmatrix}#1 \quad #2\\#3&#4\end{pmatrix}}

\newcommand{\Grss}[3]{\UT{#1}{\begin{pmatrix} #2 \; #3\end{pmatrix}}}
\newcommand{\Grb}[7]{\UT{#1}{\begin{pmatrix} #2\quad #3 \\ #4 \quad #5\\ #6 \quad #7\end{pmatrix}}}
\newcommand{\un}[2]{e{{\tiny \begin{pmatrix}#1\\ #2\end{pmatrix}}}}
\newcommand{\unn}[3]{e{{\tiny \begin{pmatrix}#1\\ #2\\#3\end{pmatrix}}}}

\newcommand{\wmult}{\cdot}
\newcommand{\bmult}{*}
\newcommand{\wmate}{\rightarrow}% Change this to source/target notation l(eft) r(ight)
\newcommand{\bmate}{\downarrow}% Change this to source/target notation u(p) d(own)

\newcommand{\aste}[1]{\underset{#1}{\ast}}

\newcommand{\Vv}{\mathcal{V}}

\newcommand{\dT}{\dot T}

\newtheorem{Theorem}{Theorem}[section]
\newtheorem{Lem}[Theorem]{Lemma}
\newtheorem{Prop}[Theorem]{Proposition}
\newtheorem{Cor}[Theorem]{Corollary}

\theoremstyle{definition}
\newtheorem{Def}[Theorem]{Definition}
\newtheorem{Rem}[Theorem]{Remark}
\newtheorem{Exa}[Theorem]{Example}
\newtheorem{Not}[Theorem]{Notation}
\newtheorem{Que}[Theorem]{Question}
\newtheorem{Con}[Theorem]{Conjecture}

%%%%%%%%%%%%%%%%%%%
% Further notation for Section 1
\newcommand{\phic}[2]{\Grt{\phi}{#1}{#2}}

%%%%%%%%%%%%%%%%%%%
% Notation for Section 4
\newcommand{\LGtwo}{L^{2}(\mathscr{G})}
\newcommand{\LGinf}{L^{\infty}(\mathscr{G})}
\newcommand{\CrG}{C^{r}_{0}(\mathscr{G})}
\newcommand{\CuG}{C^{u}_{0}(\mathscr{G})}
\newcommand{\vnDelta}{\overline{\Delta}}
\newcommand{\vnE}{\overline{E}}
\newcommand{\astrl}{\underset{l^{\infty}(I)}{_{\rho}\ast_{\lambda}}}
\newcommand{\otimesrl}{\underset{\nu{}}{_{\rho}\otimes_{\lambda}}}
\newcommand{\vnphi}{\overline{\phi}}
\newcommand{\vnphic}[2]{\Grt{\vnphi}{#1}{#2}}
\newcommand{\vnR}{\overline{R}}
\newcommand{\vntau}{\overline{\tau}}

\date{}


\numberwithin{equation}{section}

\begin{document}
\title{Partial compact quantum groups, compact quantum homogeneous spaces and the dynamical quantum $SU(2)$ group}

\author{Kenny De Commer\thanks{Department of Mathematics, Vrije Universiteit Brussel, VUB, B-1050 Brussels, Belgium, email: {\tt kenny.de.commer@vub.ac.be}}
\and Thomas Timmermann\thanks{University of M\"{u}nster}}

\maketitle

% Terminology `compact' might have to be adapted if we work with infinitely many objects.
\begin{abstract}
\noindent Compact quantum groups of face type, as introduced by T. Hayashi, form a class of quantum groupoids with a classical, finite set of objects. We generalize Hayashi's definition to allow for an infinite set of objects, and call the resulting objects partial compact quantum groups. We then show how any quantum homogeneous space of an ordinary compact quantum group leads to a partial compact quantum group. In particular, when this construction is applied to the non-standard Podle\'{s} spheres, we obtain partial compact quantum groups which are operator algebraic versions of the dynamical quantum $SU(2)$-group as studied by Etingof-Varchenko and Koelink-Rosengren.% References ok?
\end{abstract}


%\emph{Keywords}:

%AMS 2010 \emph{Mathematics subject classification}:


%17B37: Quantum groups, quantized enveloping algebras
%20G42: quantized function algebras
%46L65: Functional analysis, deformations, quantizations
%81R50: Quantum groups and related algebraic methods
%16T05: Hopf algebras and their applications
%16T10: Bialgebras
%16T15: Coalgebras and comodules; corings
%46L08  $C^*$-modules


\tableofcontents

\section{Partial compact quantum groups}

We generalize Hayashi's definition of a compact quantum group of face type \cite{Hay1} to the case where the commutative base algebra is no longer finite-dimensional. We will present two approaches, based on \emph{partial bialgebras} and \emph{weak multiplier bialgebras} \cite{Boh1}. The first approach is piecewise and concrete, but requires some bookkeeping. The second approach is global but more abstract. As we will see from the general theory and the concrete examples, both approaches have their intrinsic value.

%\begin{Not} If $I$ is a set, we write $\Fun_{\fin}(I)$ for the algebra of \emph{finitely supported} $\C$-valued functions on $I$\\

%We write $\Fun(I)$ for the algebra of \emph{all} $\C$-valued functions on $I$. %We will always suppose that $I$ is at most countable.
%\end{Not}

% Notation $I$ ok, or better $O$?

Let $I$ be a set. We consider $I^2=I\times I$ as the pair groupoid with $\wmult$ denoting composition. That is, an element $K=(k,l)\in I^2$ has source $K_l = k$ and target $K_r=l$, and if $K=(k,l)$ and $L=(l,m)$ we write $K\wmult L = (k,m)$. %For general $K,L\in I^2$, we write the property `$K$ and $L$ are composable' as $K\wmate L$. 

\begin{Def} A \emph{partial algebra} $\mathscr{A}=(\mathscr{A},M)$ (over $\C$) is a small $\C$-linear category, that is, a set $I$ (the object set) together with %Change partial by face? Partial algebra might have distinct meaning
\begin{itemize}
\item[$\bullet$] for each $K=(k,l)\in I^2$ a vector space $A(K) = \Grs{A}{k}{l}=\!\!\GrDA{A}{k}{l}$ (possibly the zero vector space),
\item[$\bullet$] for each $K,L$ with $K_r = L_l$ a multiplication map \[M(K,L):A(K) \otimes A(L)\rightarrow A(K\cdot L),\qquad a\otimes b \mapsto ab\]  and 
\item[$\bullet$] elements $\Unit(k) = \Unit_k \in \Grs{A}{k}{k}$ (the units), % or the local units?
\end{itemize}
such that the obvious associativity and unit conditions are satisfied. 

By \emph{$I$-partial algebra} will be meant a partial algebra with object set $I$.
\end{Def}

\begin{Rem}We allow the local units $\Unit_k$ to be zero. %When none of the units are zero, we will call the partial algebra \emph{connected}. 
\end{Rem}

Let $\mathscr{A}$ be an $I$-partial algebra. We define $A(K\wmult L)$ to be $\{0\}$ when $K\wmult L$ is ill-defined, i.e. $K_r\neq L_l$. We then let $\Grs{M}{K}{L}$ be the zero map.

\begin{Def} The \emph{total algebra} $A$ of an $I$-partial algebra $\mathscr{A}$ is the vector space \[A = \oplus_{K\in I^2} A(K)\] endowed with the unique multiplication whose restriction to $A(K)\otimes A(L)$ concides with $M(K,L)$.  
\end{Def} 
% Should a reference to the notion of locally unital algebra be given? I had a reference to Quillen, but no precise article, and I don't seem to find the link anymore...
Clearly $A$ is an associative algebra. If $I$ is infinite it will not possess a unit, but it is a \emph{locally unital algebra} as there exist mutually orthogonal idempotents $\mathbf{1}_k$ with $A = \osum{k,l} \mathbf{1}_kA\mathbf{1}_l$. An element $a\in A$ can be interpreted as a function assigning to each element $(k,l)\in I^2$ an element $a_{kl}\in A(k,l)$, namely the $(k,l)$-th component of $a$. This identifies $A$ with finite support $I$-indexed matrices whose $(k,l)$-th entry lies in $A(k,l)$, equipped with the natural matrix multiplication. 

\begin{Rem}\label{RemGrad} When $\mathscr{A}$ is an $I$-partial algebra with total algebra $A$, then $A\otimes A$ can be naturally identified with the total algebra of an $I\times I$-partial algebra $\mathscr{A}\otimes \mathscr{A}$, where \[(A\otimes A)((k,k'),(l,l')) = A(k,l)\otimes A(k',l')\] with the obvious tensor product multiplications and the $\Unit_{k,k'} = \Unit_k\otimes \Unit_{k'}$ as units. 
\end{Rem} 

%There are two natural notions of morphism between partial algebras, functors and co-functors. %Different name? How natural are these?
%We will need the notion of \emph{relator} between partial algebras. When $R\subseteq I\times J$ is a relation, we write, for $k\in I$, $R_k= \{l'\in J\mid (k',l')\in R\}$. We write $D_R = \{k\in I\mid \#R_k\neq0\}$.  
%A \emph{functor} between an $I$-partial algebra $\mathscr{A}$ and $J$-partial algebra $\mathscr{B}$ consists of a map $F:I\rightarrow J$ and linear maps $F_{k,l}:A(k,l)\rightarrow B(F(k),F(l))$ such that the $F_{k,l}$ respect the algebra and unit maps in the obvious ways. 

%\begin{Def} A \emph{relator} between an $I$-partial algebra $\mathscr{A}$ and $J$-partial algebra $\mathscr{B}$ consists of a relation $F\subseteq I\times J$ and maps $F_{k',l'}:A(k,l)\rightarrow B(k',l')$ for $k'\in R_{k},l'\in R_{l}$ such that the following conditions hold.
%\begin{itemize}
%\item For all $k,l\in I$ and $a\in A(k,l)$, the $F_{k',l'}(a)$ are zero for almost all $k'\in R_{k}$ (resp. all $l'\in R_{l}$) when $l'$ (resp. $k'$) is fixed.  
%\item For all $a\in A(k,l)$ and $b\in A(l,m)$, and all $k'\in R_{k}, m'\in R_{m}$, one has \[F_{k',m'}(ab) = \sum_{l',l'\in R_{l}} F_{k',l'}(a)F_{l',m'}(b).\] 
%\item $F_{k',l'}(e_{k}) = \delta_{k',l'}e_{k'}$ for all $k'\in R_{k}$.  
%\end{itemize}
%\end{Def}

The notion of partial algebra dualizes. For this we consider again $I^2$ as the pair groupoid, but now with elements considered as column vectors, and with $\bmult$ denoting the (vertical) composition. So $K=\Grt{}{k}{l}$ has source $K_u = k$ and target $K_d = l$, and if $K=\Grt{}{k}{l}$ and $L=\Grt{}{l}{m}$ then $K\bmult L = \Grt{}{k}{n}$. %We write $K\bmate L$ if $K$ and $L$ are composable. 

%Then we have maps  \[\Grt{\Delta}{K}{L}: A(K\bmult L)\rightarrow A(K)\otimes A(L),\] which we interpret as zero maps when $\neg K\bmate L$. We also interpret $\Grt{\Delta}{K}{L}$ as the zero map on $A(M)$ if $M\neq K\bmult L$. The coassociativity condition can now be written  \[(\id\otimes \Grt{\Delta}{L}{M})\Grt{\Delta}{K}{L\bmult M} = ( \Grt{\Delta}{K}{L}\otimes \id)\Grt{\Delta}{K\bmult L}{M}.\]
% Should provide shortcut for notation where only the new indices appear at $\Delta$ (so disregard the source): write $\Delta(K over L)$ as $Delta_{rs}$ for $r=K_{ld}$ and $s=K_{rd}$. 
\begin{Def} A \emph{partial coalgebra} $\mathscr{A}=(\mathscr{A},\Delta)$ (over $\C$) consists of a set $I$ (the object set) together with 
\begin{itemize}
\item[$\bullet$] for each $K=\Grru{k}{l}\in I^2$ a vector space $A(K) = \Grt{A}{k}{l}=\!\!\GrRA{A}{k}{l}$,
\item[$\bullet$] for each $K,L$ with $K_d = L_u$ a comultiplication map \[\Grt{\Delta}{K}{L}:A(K*L)\rightarrow A(K)\otimes A(L),\qquad a \mapsto a_{(1)K}\otimes a_{(2)L},\] and 
\item[$\bullet$] counit maps $\epsilon_k:\Grt{A}{k}{k}\rightarrow \C$,
\end{itemize} 
satisfying the obvious coassociativity and counitality conditions.

By \emph{$I$-partial coalgebra} will be meant a partial coalgebra with object set $I$.
\end{Def}

\begin{Not}\label{NotCom} As the index of $\epsilon_k$ is determined by the element to which it is applied, there is no harm in dropping the index $k$ and simply writing $\epsilon$.

Similarly, if $K = \Grt{}{k}{l}$ and $L = \Grt{}{l}{m}$, we abbreviate $\Delta_l = \Grt{\Delta}{K}{L}$, as the other indices are determined by the element to which $\Delta_l$ is applied.
\end{Not}

We also make again the convention that $A(K*L)=\{0\}$ and $\Grt{\Delta}{K}{L}$ the zero map when $K_d \neq L_u$. Similarly $\epsilon$ is seen as the zero functional on $A(K)$ when $K=\Grt{}{k}{l}$ with $k\neq l$. 

%The $\Grt{A}{k}{l}$ can be combined into a comultiplication \[\Delta: \Prod_{K} A(K)\rightarrow \Prod_{L,M} \left(A(L)\otimes A(M)\right)\] which is coassociative in the obvious way. In the following, we will write elements of direct products as infinite sums.

We can now superpose the notions of partial algebra and partial coalgebra. To formulate the condition that the coalgebra maps form a `morphism of partial algebras', we will need to impose a finiteness condition which is automatically satisfied when the cardinality of $I$ is finite.
% Formally, a finite morphism between an $I$-partial algebra $\mathscr{A}$ and $J$-partial algebra $\mathscr{B}$ equipped with an injection $\phi$ of a subset of $J$ into $I$ consists of maps $F_{K,L}:A(\phi(K),\phi(L))\rightarrow B(K,L)$ for $K,L$ in the domain of $\phi$, such that the map $(K,L)\mapsto F_{K,L}(a)$ is finite in rows and columns when the values of $K,L$ are restricted to one fiber of$ \phi$, and such that then $F_{K,M}(ab) = \sum_L F_{K,L}(a)F_{L,M}(b)$. 
% Should this more general notion be included?

Let $I$ be a set, and let $M_2(I)$ be the set of 4-tuples of elements of $I$ arranged as 2$\times$2-matrices. We can endow $M_2(I)$ with two compositions, namely $\cdot$ (viewing $M_2(I)$ as a row vector of column vectors) and $*$ (viewing $M_2(I)$ as a column vector of row vectors). When $K\in M_2(I)$, we will write $K = \Grs{}{K_l}{K_r} = \Grt{}{K_u}{K_d} = \eGr{}{K_{lu}}{K_{ru}}{K_{ld}}{K_{rd}}$. One can view $M_2(I)$ as a double groupoid, and in fact as a \emph{vacant} double groupoid in the sense of \cite{AN1}. %, and many of the constructions below work with $M_2(I)$ replaced by an arbitrary double groupoid, if its vacant!

In the following, a vector $(r,s)$ will sometimes be written simply as $r,s$ (without parentheses) or $rs$ in an index. We also follow Notation \ref{NotCom}, but the reader should be aware that the index of $\Delta$ will now be a 1$\times$2 vector in $I^2$ as we will work with partial coalgebras over $I^2$.% Put less awkardly.
%\end{document}
\begin{Def}\label{DefPartBiAlg} A \emph{partial bialgebra} $\mathscr{A}=(\mathscr{A},M,\Delta)$ consists of a set $I$ and a collection of vector spaces $A(K)$ for $K\in M_2(I)$ such that 
\begin{itemize}
\item[$\bullet$] the $\Grs{A}{K_l}{K_r}$ form an $I^2$-partial algebra,
\item[$\bullet$] the $\Grt{A}{K_u}{K_d}$ form an $I^2$-partial coalgebra,
\end{itemize} 
and for which the following compatibility relations are satisfied.
\begin{enumerate}[label=(\alph*)]
\item\label{Propa} (Comultiplication of Units) For all $k,l,m\in I$, one has 
% $K = \begin{pmatrix} k & k\\p & p \end{pmatrix}$ and $L =\begin{pmatrix} p &p \\ l & l\end{pmatrix}$, one has 
\[\Delta_{l,l}(\UnitC{k}{m}) = \UnitC{k}{l}\otimes \UnitC{l}{m}.\]  
\item\label{Propb} (Counit of Multiplication) For all $K,L\in M_2(I)$ with $K_r = L_l$ and all $a\in A(K)$ and $b\in A(L)$, \[\epsilon(ab) = \epsilon(a)\epsilon(b).\]% Subtlety is that you already have to know composition to be able to apply this rule.
\item\label{Propc} (Non-degeneracy) For all $k\in I$, $\epsilon(\UnitC{k}{k})=1$. 
\item\label{Propd} (Finiteness) For each $K\in M_2(I)$ and each $a\in A(K)$, the element $\Delta_{rs}(a)$ is zero except for a finite number of indices $r$ (resp. $s$) when $s$ (resp. $r$) is fixed.
%\[\Grt{\Delta}{K}{L}(a)(1\otimes b) \neq0 \quad \textrm{or} \quad\Grt{\Delta}{K}{L}(a)(b\otimes 1) \neq 0.\]
% We encode regularity this way: if asymmetrically we ask finiteness only for $r$ fixed, we are in the non-regular situation. (?)
\item\label{Prope} (Comultiplication is multiplicative) For all $a\in A(K)$ and $b\in A(L)$ with $K_r= L_l$,  \[\Delta_{rs}(ab) = \sum_t \Delta_{rt}(a)\Delta_{ts}(b).\]
\end{enumerate}
\end{Def}

% To do: introduce notion of connectedness which should assure the trivial corepresentation is irreducible. This should correspond to the dual notion of pureness in [Bohm-Nill-Szlachanyi], i.e. to the left and right base algebras having $\C$ as intersection. I suspect this is the same as proving $\UnitC{k}{l}\neq 0$ for all $k,l$ (maybe in the presence of antipode integrals), but could not prove it immediately. Update: indeed, under the presence of an antipode the relation $k\sim l$ iff $\Unit{k}{l}\neq 0$ gives an equivalence relation, and then fixing a non-trivial class $P$ and putting $c_k = \delta_{k\in P}, d_l = \delta_{l\in P}$, we obtain a non-trivial element $\sum_k c_k\lambda_k = \sum_l d_l \rho_l$ in the intersection.

\begin{Rem} By assumption \ref{Propd}, the sum on the right hand side in condition \ref{Prope} is well-defined. 
\end{Rem}

%\begin{Rem}By assumption \ref{Propd}, the sum on the right hand side in condition \ref{Prope} is well-defined. We can write it less pedantically as \[\Delta_{rs}(ab) = \sum_t \Delta_{rt}(a)\Delta_{ts}(b),\] where we abbreviate $\Delta_{rs} = \Delta_{(r,s)}$.
%\end{Rem}

We want to relate the notion of partial bialgebra to the recently introduced notion of weak multiplier bialgebra \cite{Boh1}. We first recall some notions concerning non-unital algebras \cite{Dau1,VDae1}.

\begin{Def} Let $A$ be an algebra over $\C$, not necessarily with unit. We call $A$ \emph{non-degenerate} if $A$ is faithfully represented on itself by left and right multiplication. It is called \emph{idempotent} if $A^2 = A$. 
\end{Def}

\begin{Def} Let $A$ be an algebra. A \emph{multiplier} $m$ for $A$ consists of a couple of maps \begin{eqnarray*} L_m:A\rightarrow A,\quad a\mapsto ma\\ R_m:A\rightarrow A,\quad a\mapsto am\end{eqnarray*} such that $(am)b = a(mb)$ for all $a,b\in A$. 

The set of all multipliers forms an algebra under composition for the $L$-maps and anti-composition for the $R$-maps. It is called the \emph{multiplier algebra} of $A$, and is denoted $M(A)$.
\end{Def}

One has a natural homomorphism $A\rightarrow M(A)$. When $A$ is non-degenerate,  this homomorphism is injective, and we can then identify $A$ as a subalgebra of the (unital) algebra $M(A)$. We then also have inclusions \[A\otimes A\subseteq M(A)\otimes M(A)\subseteq M(A\otimes A).\]

\begin{Exa}\label{ExaMult} \begin{enumerate}
\item Let $A$ be the total algebra of an $I$-partial algebra $\mathscr{A}$. As $A$ has local units, it is non-degenerate and idempotent. Then one can identify $M(A)$ with \[M(A) = \left(\prod_l \oplus_k A(k,l)\right) \bigcap \left(\prod_k\oplus_l A(k,l)\right) \subseteq \prod_{k,l} A(k,l),\] i.e. with the space of functions \[m:I^2\rightarrow A,\quad m_{kl}\in A(k,l)\] which have finite support in either one of the variables when the other variable has been fixed. The multiplication is given by the formula \[(mn)_{kl} = \sum_p m_{kp}n_{pl}.\]
\item Let $m_i$ be any collection of multipliers of $A$, and assume that for each $a\in A$, $m_ia =0$ for almost all $i$, and similarly $am_i=0$ for almost all $i$. Then one can define a multiplier $\sum_i m_i$ in the obvious way by termwise multiplication. One says that the sum $\sum_i m_i$ converges in the \emph{strict} topology. 
\end{enumerate}
\end{Exa}

Using the notion introduced in Example \ref{ExaMult}.2, we can introduce the following notation.

\begin{Not}
If $\mathscr{A}$ is an $I$-partial bialgebra, we write \[\lambda_k = \sum_l \UnitC{k}{l},\qquad \rho_l = \sum_k\UnitC{k}{l} \qquad \in M(A).\]
\end{Not}

\begin{Rem} From Property \ref{Propc} of Definition \ref{DefPartBiAlg}, it follows that $\lambda_k\neq 0\neq \rho_k$ for any $k\in I$.
\end{Rem} 

To show that the total algebra of a partial bialgebra becomes a weak multiplier bialgebra, we will need some easy lemmas. 
%In the following, we will use the grading for tensor products as in Remark \ref{RemGrad}, coupled with the multiplier interpretation as in Example \ref{ExaMult}.

\begin{Lem} Let $\mathscr{A}$ be an $I$-partial bialgebra. Then for each $a\in A$, there exists a unique multiplier $\Delta(a) \in M(A\otimes A)$ such that \begin{eqnarray}\label{EqDel} \Delta_{rs}(a) &=& (1\otimes \lambda_r)\Delta(a)(1\otimes \lambda_s) \\ &=& (\rho_r\otimes 1)\Delta(a)(\rho_s\otimes 1)\end{eqnarray}  for all $r,s\in I$, all $K\in M_2(I)$ and all $a\in A(K)$. 

The resulting map \[\Delta:A\rightarrow M(A\otimes A),\quad a\mapsto \Delta(a)\] is a homomorphism.
\end{Lem}
\begin{proof} For $a\in A$ homogeneous, we can define $\Delta(a) = \sum_{rs} \Delta_{rs}(a) \in M(A\otimes A)$, where the sum converges in the strict topology of $A\otimes A$ because of the property \ref{Propd} of Definition \ref{DefPartBiAlg}. This expression clearly satisfies the identities stated in the lemma. In turn, these identities uniquely define $\Delta(a)$ as a multiplier, as they determine the value of $\Delta(a)$ when cut down to the left and right with the local units of $\mathscr{A}\otimes \mathscr{A}$.

We can then extend $\Delta$ to $A$ by linearity. Since, for $a,b$ homogeneous, $\Delta_{rt}(a)\Delta_{t's}(b)=0$ unless $t=t'$, it follows from property \ref{Prope} of Definition \ref{DefPartBiAlg} that $\Delta$ is a homomorphism. 
\end{proof}

We will refer to $\Delta: A\rightarrow M(A\otimes A)$ as the \emph{total comultiplication} of $\mathscr{A}$. We will then also use the suggestive Sweedler notation for this map, \[\Delta(a) = a_{(1)}\otimes a_{(2)}.\]

\begin{Lem} The element $E = \sum_{k,l,m} \UnitC{k}{l}\otimes \UnitC{l}{m}$ is a well-defined idempotent in $A\otimes A$, and satisfies \[\Delta(A)(A\otimes A)=E(A\otimes A),\quad (A\otimes A)\Delta(A)= (A\otimes A)E.\]
\end{Lem} 
\begin{proof} Clearly the sum defining $E$ is strictly convergent, and makes $E$ into an idempotent. It is moreover immediate that $E\Delta(a)=\Delta(a) = \Delta(a)E$ for all $a\in A$. Since \[E(\UnitC{k}{l}\otimes \UnitC{m}{n}) = \Delta(\UnitC{k}{n})(\UnitC{k}{l}\otimes \UnitC{m}{n}) \] by the property \ref{Propa} of Definition \ref{DefPartBiAlg}, and analogously for multiplication with $E$ on the right, the lemma is proven. 
\end{proof} 

By \cite[Proposition A.3]{VDW2}, there is a unique homomorphism $\Delta:M(A)\rightarrow M(A\otimes A)$ extending $\Delta$ and such that $\Delta(1) = E$. Alternatively, if $m\in M(A)$, we can directly define $\Delta(m)$ as the strict limit of the series $\sum_{k,l,r,s} \Delta_{rs}(m_{kl})$. Similarly the maps $\id\otimes \Delta$ and $\Delta\otimes \id$ extend to maps from $M(A\otimes A)$ to $M(A\otimes A\otimes A)$. The following proposition gathers the properties of $\Delta$, $\epsilon$ and $\Delta(1)$ which guarantee that $(A,\Delta)$ forms a weak multiplier bialgebra in the sense of \cite[Definition 2.1]{Boh1}.

\begin{Prop} Let $\mathscr{A}$ be a partial bialgebra with total algebra $A$, total comultiplication $\Delta$ and counit $\epsilon$. Then the following properties are satisfied.
\begin{enumerate}[label={(\arabic*)}]
\item Coassociativity: $(\Delta\otimes \id)\Delta = (\id\otimes \Delta)\Delta$ (as maps $M(A)\rightarrow M(A^{\otimes 3})$).
\item Counitality: $(\epsilon\otimes \id)(\Delta(a)(1\otimes b)) = ab = (\id\otimes \epsilon)((a\otimes 1)\Delta(b))$ for all $a,b\in A$.
\item Weak Comultiplicativity of Unit: \[(\Delta(1)\otimes 1)(1\otimes \Delta(1)) = (\Delta\otimes \id)\Delta(1) = (\id\otimes \Delta)\Delta(1) = (1\otimes \Delta(1))(\Delta(1)\otimes 1).\]
\item \label{WMC} Weak Multiplicativity of Counit: For all $a,b,c\in A$, one has \[(\epsilon\otimes \id)(\Delta(a)(b\otimes c)) = (\epsilon\otimes \id)((1\otimes a)\Delta(1)(b\otimes c))\] and 
\[(\epsilon\otimes \id)((a\otimes b)\Delta(c)) = (\epsilon\otimes \id)((a\otimes b)\Delta(1)(1\otimes c)).\]
\item Strong multiplier property: For all $a,b\in A$, one has \[\Delta(A)(1\otimes A)\cup (A\otimes 1)\Delta(A)\subseteq  A\otimes A.\] 
\end{enumerate}
\end{Prop}

\begin{proof} Most of these properties follow immediately from the definition of a partial bialgebra. For demonstrational purposes, let us check the first identity of property \ref{WMC}. Let us choose $a\in A(K)$, $b\in A(L)$ and $c\in A(M)$. Then \[\Delta(a)(b\otimes c) = \delta_{K_{ru},L_{lu}}\delta_{M_{lu},L_{ld}} \sum_r \Delta_{r,L_{ld}}(a)(b\otimes c).\]  Applying $(\epsilon\otimes \id)$ to both sides, we obtain by Proposition \ref{Propb} of Definition \ref{DefPartBiAlg} and counitality of $\epsilon$ that \[(\epsilon \otimes \id)(\Delta(a)(b\otimes c)) = \delta_{K_{ru},L_{lu},L_{ld},M_{lu}} \epsilon(b) ac.\] On the other hand, \begin{eqnarray*} (1\otimes a)\Delta(1)(b\otimes c) &=& \sum_{r,s,t} \UnitC{r}{s} b \otimes a\UnitC{s}{t}c \\ &=& \delta_{L_{ld},K_{ru},M_{lu}} b \otimes ac.\end{eqnarray*} Applying $(\epsilon\otimes \id)$, we find \begin{eqnarray*} (\epsilon\otimes \id)( (1\otimes a)\Delta(1)(b\otimes c) ) &=&  \delta_{L_{ld},K_{ru},M_{lu}}\delta_{L_{lu},L_{ld}}\delta_{L_{ru},L_{rd}} \epsilon(b)ac \\ &=&  \delta_{L_{ld},L_{lu},K_{ru},M_{lu}} \epsilon(b)ac,\end{eqnarray*} which agrees with the expression above.
\end{proof} 

\begin{Rem} 
%\begin{enumerate}\item 
Since also the expressions $\Delta(a)(b\otimes 1)$ and $(1\otimes a)\Delta(b)$ are in $A\otimes A$ for all $a,b\in A$, we see that $(A,\Delta)$ is in fact a \emph{regular} weak multiplier bialgebra \cite[Definition 2.3]{Boh1}.
%\item\label{RemStrongReg} In fact, for any multiplier $m$, all expressions of the form $\Delta(m)(1\otimes b)$ for $b\in A$ are already in $A\otimes A$. This follows from the fact that this holds for $m=1$. This is of course a very strong property for a general weak multiplier Hopf algebra.
%\end{enumerate} 
\end{Rem} 

%\begin{proof} This follows from equalities of the form $\Delta(a)(1\otimes \lambda_s) = \sum_r \Delta_{rs}(a)$, where the sum on the right is finite dimensional.
%\end{proof} 



% Need to give a converse of the theorem, show that any weak multiplier bialgebra with object algebra $F_f(I)$ is of the above form?

%\begin{Def}[B\"{o}hm] A \emph{multiplier weak bialgebra} consists of a non-degenerate idempotent algebra $A$, equipped with a homomorphism \[\Delta: A\rightarrow M(A\otimes A),\] a linear map $\varepsilon: A\rightarrow \C$ and an idempotent element $E\in M(A\otimes A)$ such that the following conditions are satisfied.
%\begin{itemize}
%\item[$\bullet$] For all $a,b\in A$, one has both $\Delta(a)(1\otimes b)$ and $(a\otimes 1)\Delta(b)$ inside $A\otimes A$. 
%\item[$\bullet$] \emph{Coassociativity}: for all $a,b,c\in A$, one has \[(b\otimes 1\otimes 1)(\Delta\otimes \id)\left(\Delta(a)(1\otimes c)\right) = (\id\otimes \Delta)\left((b\otimes 1)\Delta(a)\right)(1\otimes 1\otimes c)\]
%\item[$\bullet$] \emph{Counitality}: For all$ a,b\in A$, one has \[(\varepsilon \otimes \id)(\Delta(a)(1\otimes b) = ab = (\id\otimes \varepsilon)((a\otimes 1)\Delta(b)).\]
%\item[$\bullet$] The space $\Delta(A)(A\otimes A)$ equals $E(A\otimes A)$, and similarly the space $(A\otimes A)\Delta(A)$ equals $(A\otimes A)E$.  % Nicer formulation?
%\item[$\bullet$] 
%\end{itemize}

%\end{Def}

%The \emph{multiplier algebra} $M(A)$ of $A$ is the set One easily shows that this definition is independent of the realization of $A$ as a total algebra, and that it coincides with the notion of multiplier algebra found for example in \cite[Appendix]{VDae1}.

%In the following, we always view $A$ as a subalgebra of $M(A)$ in the natural way. Clearly $m$ is a multiplier if and only if for each $k$, the function $l\mapsto \eGrr{m}{k}{l}$ is almost everywhere zero, and similarly for each $l$, the function $k\mapsto \eGrr{m}{k}{l}$ is almost everywhere zero, while $m\in A$ if the function $(k,l)\mapsto \eGrr{m}{k}{l}$ is almost everywhere zero. The algebra $M(A)$ has a unit, $\eGrr{1}{k}{l} = \delta_{kl} e_k$. For $m\in M(A)$, we use the formal notation $m = \sum_{k,l} m_{kl}$ which can be justified as a topological sum using the multiplier topology. Then we may write the unit for example as $\sum_p e_p $.



%In the following, we will also use the notation \[\lambda_k = \sum_l e\Grru{k}{l},\qquad \rho_l = \sum_k e\Grru{k}{l} \qquad \in M(A).\]

%\begin{Lem} Let $(A(K),\Delta\Grru{K}{L})$ be a partial bialgebra. Then \[\Delta: M(A)\rightarrow M(A\otimes A),\quad a\mapsto \sum_{K,L,M} \Delta\Grru{K}{L}(a(M))\] is a well-defined homomorphism for which \begin{equation}\label{EqStrongMult} \Delta(A)(1\otimes A)\cup \Delta(A)(A\otimes 1)\cup (A\otimes 1)\Delta(A)\cup (1\otimes A)\Delta(A)\subseteq A\otimes A.\end{equation}
%Moreover, $\Delta$ satisfies the coassociativity assumption \[(c\otimes 1\otimes 1)(\Delta\otimes \id)\Delta(a)(1\otimes b)  = \left((\id\otimes \Delta)(c\otimes 1)\Delta(a)\right)(1\otimes 1\otimes b),\qquad \forall a,b,c\in A.\] 
%\end{Lem} 
%\begin{proof} The algebra $A\otimes A$ is a locally unital algebra with units $e_K\otimes e_L$ for column vectors $K,L$. Let us first verify that for $a\in A(M)$, one has $\Delta\Grru{K}{L}(a)$ almost always zero if $K_l$ and $L_l$ (resp. $K_r$ and $L_r$) are fixed. Since this expression is zero when $K\cdot L \neq M$ or $\neg K\bmate L$, it follows from condition b) in Definition \ref{DefPartBiAlg} that this condition is indeed satisfied. Hence $\Delta(a(M)) \in M(A\otimes A)$, and moreover, one obtains then also immediately that the inclusion \eqref{EqStrongMult} holds.

%If now $a\in M(A)$, we know that $a(M)$ is almost always zero if $m_{lu}$ and $m_{ru}$ (resp. $m_{ru}$ and $m_{rd}$) are fixed. It follows as before that $\Delta(a)$ is a well-defined multiplier.

%It is now easy to check that $\Delta$ is a homomorphism by condition c) in Definition \ref{DefPartBiAlg}, and that is coassociative by the partial coassociativity of the $\Delta\Grru{K}{L}$.
%\end{proof} 

%We can write then for example \[\Delta(1) = \sum_p \rho_p\otimes \lambda_p \in M(A\otimes A).\]

%In the following, we will use the Sweedler notation for $\Delta$, so \[\Delta(a) = a_{(1)}\otimes a_{(2)},\qquad a\in A.\] We will also use the notation \[\Delta\Gru{K}{L}(a) = a_{(1)K}\otimes a_{(2)L}.\]



%\begin{Def} A partial $^*$-algebra is a partial algebra equipped with anti-linear maps \[A_{K}\rightarrow A_{K^{\circ}},\quad a\mapsto a^*\] for which the total algebra $A$ becomes a $^*$-algebra.
%A partial $^*$-bialgebra is a partial bialgebra for which $\Grt{\Delta}{K}{L}(a)^* = \Grt{\Delta}{K^{\circ}}{L^{\circ}}(a^*)$ for all $a\in A$.
%\end{Def} 

%It is easily verified that in this case $\overline{\varepsilon_{K}(a)} = \varepsilon_{K^{\circ}}(a)$, and that $\Delta:A\rightarrow M(A\otimes A)$ becomes a $^*$-algebra homomorphism.

% To complete
\begin{Rem}\label{RemFull}
Recall from \cite[Section 3]{Boh1} that a regular weak multiplier bialgebra admits four projections $A\rightarrow M(A)$, given by \begin{align*} \bar{\Pi}^L(a) = (\epsilon\otimes \id)((a\otimes 1)\Delta(1)),\quad &  \bar{\Pi}^R(a) = (\id\otimes \epsilon)(\Delta(1)(1\otimes a)),\\ \Pi^L(a)  = (\epsilon\otimes \id)(\Delta(1)(a\otimes 1)),\quad& \Pi^R(a) = (\id\otimes\epsilon)((1\otimes a)\Delta(1)),\end{align*} where the right hand side expressions are interpreted as multipliers in the obvious way. A trivial computation shows that if $a\in \eGr{A}{k}{l}{m}{n}$, then \[ \overline{\Pi}^L(a) =\epsilon(a)\lambda_n,\quad \overline{\Pi}^R(a) =\epsilon(a)\rho_k,\quad \Pi^L(a) =\epsilon(a)\lambda_m,\quad \Pi^R(a) = \epsilon(a) \rho_l.\] In particular, we see that the \emph{base algebra} of $(A,\Delta)$ is just the algebra $\Fun_{\fin}(I)$ of finite support functions on $I$. From \cite[Theorem 3.13]{Boh1}, we conclude that the comultiplication of $A$ is (left and right) \emph{full} (meaning roughly that the legs of $\Delta(A)$ span $A$).
\end{Rem}

We now formulate the notion of partial Hopf algebra, whose total form will correspond to a weak multiplier Hopf algebra \cite{Boh1,VDW2,VDW1}. We will mainly refer to \cite{Boh1} for uniformity.% OK?

 Let us denote $\circ$ for the inverse of $\wmult$, and $\bullet$ for the inverse of $\bmult$, so \[\begin{pmatrix} k & l \\ m & n \end{pmatrix}^{\circ} = \begin{pmatrix} l & k \\ n & m \end{pmatrix},\quad \begin{pmatrix} k & l \\ m & n \end{pmatrix}^{\bullet} = \begin{pmatrix} m & n \\ k & l \end{pmatrix},\quad \begin{pmatrix} k & l \\ m & n \end{pmatrix}^{\circ \bullet} = \begin{pmatrix} n & m \\ l & k \end{pmatrix}.\] The notation $\circ$ (resp. $\bullet$) will also be used for row vectors (resp. column vectors).

% Write this out more explicitly
\begin{Def}\label{DefPartBiAlgAnt} An \emph{antipode} for an $I$-partial bialgebra $\mathscr{A}$ consists of maps \[S:A(K)\rightarrow A(K^{\circ\bullet})\]
% $S: \eGr{A}{k}{l}{m}{n} \rightarrow \eGr{A}{n}{m}{l}{k}$ 
such that the following property holds: for all $M,P\in M_2(I)$ and all $a\in A(M)$, \begin{eqnarray*} \underset{K\wmult L^{\circ\bullet}=P}{\sum_{K\bmult L = M}} a_{(1)K}S(a_{(2)L})= \delta_{P_l,P_r}\epsilon(a)\mathbf{1}(P_l),\\\underset{K^{\circ\bullet}\wmult L=P}{\sum_{K\bmult L = M}} S(a_{(1)K})a_{(2)L}= \delta_{P_l,P_r}\epsilon(a)\mathbf{1}(P_r).\end{eqnarray*}

A partial multiplier bialgebra $\mathscr{A}$ is called a \emph{partial Hopf algebra} if it admits an antipode.
\end{Def} 

\begin{Rem} Note that condition \ref{Propd} of Definition \ref{DefPartBiAlg} again guarantees that the above sums are in fact finite.
\end{Rem}

If $S$ is an antipode for a partial bialgebra, we can extends $S$ to a linear map \[S:A\rightarrow A\] on the total algebra $A$. 

\begin{Lem}\label{LemAnti} Let $S$ be an antipode for a partial Hopf algebra $\mathscr{A}$. Then for all $a,b,c\in A$, one has \begin{eqnarray*} (a_{(1)}c\otimes ba_{(2)}S(a_{(3)})) &=& (1\otimes b)\Delta(1)(ac\otimes 1)\\ (S(a_{(1)})a_{(2)}c\otimes ba_{(3)})&=& (1\otimes ba)\Delta(1)(c\otimes 1).\end{eqnarray*}
\end{Lem} 

%Note that we can here make use of Remark \ref{RemStrongReg} to make sense of the above expressions as elements of $A\otimes A$. 

\begin{proof} Take $a \in \eGr{A}{k}{l}{m}{n}$. Then in the strict topology, we obtain from the first identity in Definition \ref{DefPartBiAlgAnt} that 
\begin{eqnarray*}
a_{(1)}\otimes a_{(2)}S(a_{(3)}) 
&=& 
\sum_{r,s,t,u} a_{(1)\pmat{k}{l}{r}{s}} \otimes a_{(2)\pmat{r}{s}{t}{u}}S(a_{(3)\pmat{t}{u}{m}{n}})\\ 
&=& 
\sum_{r,t} a_{(1)\pmat{k}{l}{r}{n}} \otimes \left(\sum_{u} a_{(2)\pmat{r}{n}{t}{u}}S(a_{(3)\pmat{t}{u}{m}{n}})\right)\\ 
&=& 
\sum_{r,t} a_{(1)\pmat{k}{l}{r}{n}} \otimes \left(\delta_{r,m}\epsilon(a_{(2)\pmat{m}{n}{m}{n}})\UnitC{r}{t}\right)\\
&=&
a\otimes \lambda_m\\
&=&
\Delta(1)(a\otimes 1).
\end{eqnarray*}
The second identity is proven similarly.
\end{proof} 

\begin{Prop} A collection of maps $S:A(K)\rightarrow A(K^{\circ\bullet})$ defines an antipode for a partial bialgebra $\mathscr{A}$ if and only if the total map $S:A\rightarrow A$ defines an antipode for the total weak multiplier bialgebra $(A,\Delta)$ in the sense of \cite[Section 6]{Boh1}. 
\end{Prop} 

\begin{proof} We verify that the total antipode $S:A\rightarrow A$ satisfies the conditions of an antipode as stated in \cite[Theorem 6.8.(2)]{Boh1}. In fact, the two first identities of their Theorem 6.8.(2) are precisely our identities in Lemma \ref{LemAnti}. The third and last identity of their Theorem 6.8.(2) says that formally one should have \[\sum_kS(\rho_k a)\lambda_k = S(a),\quad \forall a\in A,\] but this follows immediately from the way $S$ acts on homogeneous components.% Should I spell out the definition of an antipode as given by Bohm et al?
\end{proof}

From the remarks following \cite[Theorem 6.8.(2)]{Boh1}, we conclude the following.

\begin{Cor} If $\mathscr{A}$ is a partial Hopf algebra, its total algebra $A$ with total comultiplication $\Delta$ forms a weak multiplier Hopf algebra in the sense of \cite[Definition 1.14]{VDW1}.
\end{Cor}

From \cite[Theorem 6.12 and Corollary 6.16]{Boh1}, we obtain the following corollary. The notation $\Delta_{rs}^{\op}$ signifies the composition of $\Delta_{rs}$ with the switch map.

\begin{Cor} The map $S:A\rightarrow A$ is an antihomomorphism s.t. \[\Delta_{rs}(S(a)) = (S\otimes S)\Delta_{sr}^{\op}(a).\]
\end{Cor} 


We now turn towards the structures which will allow us to build operator algebraic quantum groupoids out of our partial Hopf algebras.

\begin{Def} A partial $^*$-algebra $\mathscr{A}$ is a partial algebra whose total algebra $A$ is equipped with an antilinear, antimultiplicative involution \[*:A\rightarrow A,\quad a\mapsto a^*\] such that the $\mathbf{1}_k$ are selfadjoint for all $k$ in the object set. 
\end{Def} 

One can of course give an alternative definition directly in terms of the partial algebra structure by requiring that we are given antilinear maps $A(k,l)\rightarrow A(l,k)$ satisfying the obvious antimultiplicativity and involution properties.

\begin{Def} An $I$-partial $^*$-bialgebra $\mathscr{A}$ is an $I$-partial bialgebra whose underlying partial algebra has been endowed with a partial $^*$-algebra structure for which \[\Delta_{rs}(a)^* = \Delta_{sr}(a^*),\qquad \forall K\in M_2(I), \forall a\in A(K), \forall r,s\in I.\]

A partial Hopf $^*$-algebra is a partial bialgebra which is at the same time a partial $^*$-bialgebra and a partial Hopf algebra.
\end{Def} 

\begin{Rem}
\begin{enumerate}
\item By uniqueness of the counit \cite[Theorem 2.8]{Boh1}, it follows automatically that $\epsilon$ satisfies $\epsilon(a^*) = \overline{\epsilon(a)}$ for all $a$.
\item From \cite[Proposition 4.11]{VDW1}, it follows automatically that $S(S(a)^*)^* = a$ for all $a$. In particular, $S$ is invertible. % To substantiate
\end{enumerate}
\end{Rem}

\begin{Def} Let $\mathscr{A}$ be a partial bi-algebra. An \emph{invariant functional} $\varphi$ for $\mathscr{A}$ consists of a functional $\varphi:A\rightarrow \C$ %necessary to assume?
such that for all $a\in \eGr{A}{k}{l}{m}{n}$ and all $b$, one has  \[(\id\otimes \varphi)((b\otimes 1)\Delta(a)) = \varphi(a)b\lambda_k,\] and \[(\varphi\otimes \id)(\Delta(a)(1\otimes b)) = \varphi(a)\rho_nb.\]

We call $\varphi$ \emph{normalized} if $\varphi(\UnitC{k}{k}) = 1$ for all $k$. 
\end{Def} % More motivation for this notion to be provided?

\begin{Rem} 
\begin{enumerate}
\item It follows from the defining property of $\varphi$ (and the fact that $\lambda_k\neq 0\neq \rho_k$ for any $k\in I$) that $\varphi$ is zero on the $A(K)$ with $K_l\neq K_r$. 
\item If $\varphi$ is normalized, it also follows immediately from the definining property that $\varphi(\UnitC{k}{l})=1$ whenever $\UnitC{k}{l}\neq 0$.
\end{enumerate}
\end{Rem} 

We are finally ready to give our main definition.

\begin{Def} A \emph{partial compact quantum group} $\mathscr{G}$ is a partial Hopf $^*$-algebra $\mathscr{A} = P(\mathscr{G})$ with a \emph{positive} normalized invariant functional $\varphi$, that is $\varphi(a^*a)\geq 0$ for all $a\in A$. We also say that $\mathscr{G}$ is the partial compact quantum group \emph{defined by} $\mathscr{A}$.
%It is called \emph{connected} if the underlying partial algebra is connected, i.e.~ when $\UnitC{k}{l}\neq0$ for all $k,l$.  % Necessary to introduce?
\end{Def} 

\begin{Rem} Following \cite{Hay1}, we could also have called our objects \emph{compact quantum groups of face type}, but we feel this gives the wrong impression when the base algebra is infinite dimensional (i.e. the object set is not compact). When referring to partial compact quantum groups, we feel that it is better reflected that only the \emph{parts} of this object are to be considered compact, not the total object. % Is this a clear enough motivation? 
\end{Rem} 

%As in \cite[Example 2.14]{Boh1}, one can take direct sums of partial compact quantum groups. The following lemma shows that any partial compact quantum group is a direct sum of connected ones. FALSE!

%\begin{Lem} Let $\mathscr{A}$ define a partial compact quantum group. Then $\mathscr{A}\cong \oplus_k \mathscr{A}_j$ where the $\mathscr{A}_j$ define \emph{connected} partial compact quantum groups.
%\end{Lem} 
%\begin{proof} Let $I$ be the object set of $\mathscr{A}$, and say $k\sim l$ if $\UnitC{k}{l}\neq 0$. Then $\sim$ is an equivalence relation: as \[\Delta_{ll}(\UnitC{k}{m}) = \UnitC{k}{l}\otimes \UnitC{l}{m},\] it is transitive. As $S(\UnitC{k}{l}) = \UnitC{l}{k}, it is symmetric. And as $\varepsilon(\UnitC{1}{k}{k})=1$, it is reflexive.

%Let then $I = \sqcup_{\alpha\in \mathscr{I}} I_{\alpha}$ be a labeled partition associated to $\sim$. 

%\end{proof} 
 

We will need the following lemma at some point, which is an almost verbatim transcription of the argument in \cite[Proposition 3.4]{VDae2}.

\begin{Lem}\label{LemFaith} Let $\mathscr{A}$ be a partial Hopf algebra, and let $\varphi$ be a normalized invariant functional for $\mathscr{A}$. Then $\varphi$ is faithful: if $a\in A$ and $\varphi(ab) =0$ (resp. $\varphi(ba)=0$) for all $b\in A$, then $a=0$.
\end{Lem} 
% Ref to Timmermann for case of quantum groupoids? Can be explicitly used here? Ask Fons if he has already derived this result, and acknowledge? 
\begin{proof} Suppose  $\varphi(ba)=0$ for all $b$, we arrive at the conclusion that for all $d\in A$ and all functionals $\omega$ on $A$, the element $p = (\omega\otimes \id)((d\otimes 1)\Delta(b))$ satisfies \[(\id\otimes \varphi)((1\otimes c)\Delta(p)) = 0.\] Continuing as in the proof of \cite[Proposition 3.4]{VDae2}, we obtain from the antipode trick that \[\sum_n \varphi(cS(q)\rho_n)\epsilon(p\lambda_n)=0.\] Choosing now for $c$ and $q$ local units of the form $\lambda_k\rho_l$, the normalization condition on $\varphi$ gives that $\epsilon(p\lambda_n)=0$ for all $n$, hence $\epsilon(p)=0$. This implies $\omega(da)=0$. As $\omega$ and $d$ were arbitrary, it follows that $a=0$.

The other case follows similarly, or by considering the opposite comultiplication.
\end{proof} 




%%% Local Variables: 
%%% mode: latex
%%% TeX-master: "dyn-suq-main"
%%% End: 

\section{Representation theory}

In this section, the representation theory of partial compact quantum groups is investigated. As the situation is quite similar to the case already studied by Hayashi \cite{Hay1}, we do not always provide fully written out proofs, but only draw attention to those parts of the theory which need modification.

In what follows, the homogeneous component $A(K) = \eGr{A}{k}{l}{m}{n}$ of a partial bialgebra will now be mainly written as $A(K) = \Gr{A}{k}{l}{m}{n}$. 

\subsection{Corepresentations of partial Hopf algebras}


Let $\mathscr{A}$ be an $I$-partial bialgebra. We denote
$\Hom_\C(V,W)$ for the vector space of linear maps between two vector
spaces.


We denote by $\Vecti$ the category whose objects are $I^{2}$-graded
vector spaces $V=\bigoplus_{k,l\in I} \Gru{V}{k}{l}$ and whose
morphisms are linear maps $T$ that preserve the grading and therefore
can be written $T=\bigoplus_{k,l\in I} \Gru{T}{k}{l}$. 

\begin{Def} \label{definition:corep} Let $\mathscr{A}$ be an $I$-partial bialgebra and let
  $V=\bigoplus_{k,l} \Gru{V}{k}{l}$ be an $I^{2}$-graded vector space.  A \emph{corepresentation}
  $\mathscr{X}=(\Gr{X}{k}{l}{m}{n})_{k,l,m,n}$  of $\mathscr{A}$ on $V$ is a family
 of elements
 \begin{align} \label{eq:corep-blocks}
   \Gr{X}{k}{l}{m}{n} \in \Gr{A}{k}{l}{m}{n} \otimes
  \Hom_\C(\Gru{V}{m}{n},\Gru{V}{k}{l})
 \end{align}
 satisfying the following conditions:
  \begin{enumerate}
  \item $\Gr{X}{k}{l}{m}{n}=0$ for almost all $l,n$ (resp.\ almost all
    $k,m$) when $k,m$ (resp.\ $l,n$) are fixed,
  \item $ (\Delta_{pq} \otimes
    \id)(\Gr{X}{k}{l}{m}{n}) =
    \Big{(}\Gr{X}{k}{l}{p}{q}\Big{)}_{13}\Big{(}\Gr{X}{p}{q}{m}{n}\Big{)}_{23}$,
  \item $(\epsilon \otimes
  \id)(\Gr{X}{k}{l}{m}{n})=\delta_{k,m}\delta_{l,n}\id_{\Gru{V}{k}{l}}$
  \end{enumerate}
  for all possible indices.
\end{Def}
Here, we use here the standard leg numbering notation, e.g $a_{23}=1\otimes a$.
\begin{Rem}
If 3.\ holds, then 1.\ is satisfied if and only if
 $V$ is \emph{separately finitely supported
    (sfs)} in the sense that $\Gru{V}{k}{l} = \{0\}$ for almost all $k$ (resp. almost all $l$)
  when $l$ (resp. $k$) is fixed.
\end{Rem}
\begin{Exa} \label{example:corep-triv} Equip the vector space
  $\C^{(I)}=\bigoplus_{k\in I} \C$ with the diagonal
  $I^{2}$-grading. Then the family $\mathscr{U}$ given by
  \begin{align} \label{eq:corep-triv}
    \Gr{U}{k}{l}{m}{n} = \delta_{k,l}\delta_{m,n} \UnitC{k}{m} \in
    \Gr{A}{k}{l}{m}{n}
  \end{align}
is a corepresentation of $\mathscr{A}$ on $\C^{(I)}$. We call it the
\emph{trivial corepresentation}.
\end{Exa}

Let $V$ be an $I^{2}$-graded vector space. Then corepresentations of
$\mathscr{A}$ on $V$ can be summed up to ``total''
corepresentation multipliers as follows.  We denote by $A$ the total
algebra of $\mathscr{A}$ and by $\lambda^{V}_{k},\rho^{V}_{l} \in
\Hom_{\C}(V)$ the projections onto the summands $\Gru{V}{k}{} =
\bigoplus_{q} \Gru{V}{k}{q}$ and
$\Gru{V}{}{l}=\bigoplus_{p}\Gru{V}{p}{l}$, respectively, identify
$\Hom_{\C}(\Gru{V}{m}{n},\Gru{V}{k}{l})$ with
$\lambda^{V}_{k}\rho^{V}_{l}\Hom_{\C}(V)\lambda^{V}_{m}\rho^{V}_{n}$,
denote by $\Hom_{\C}^{0}(V) \subseteq \Hom_{\C}(V)$ the sum of all
these subspaces, and define a homomorphism
\begin{align*}
  \Delta \otimes \id \colon M(A \otimes \Hom_{\C}^{0}(V)) \to M(A
  \otimes A \otimes \Hom_{\C}^{0}(V))
\end{align*}
similarly as we defined $ \Delta \colon A \to M(A\otimes A)$.
\begin{Lem} \label{lemma:corep-multiplier}
  Let $V$ be an $I^{2}$-graded vector space.  If $(V,X)$ is a
  corepresentation of $\mathscr{A}$, then the sum
  \begin{align}
    \label{eq:corep-multiplier}
  X:=\sum_{k,l,m,n} \Gr{X}{k}{l}{m}{n} \in  M(A
  \otimes \Hom_{\C}^{0}(V))
  \end{align}
 converges strictly and satisfies the following conditions:
  \begin{enumerate}\setcounter{enumi}{-1}
  \item $(\lambda_{k}\rho_{m} \otimes \id){X}(\lambda_{l}\rho_{n}
    \otimes \id) = (1 \otimes \lambda^{V}_{k}\rho^{V}_{l}){X}(1 \otimes
    \lambda^{V}_{m}\rho^{V}_{n}) = \Gr{X}{k}{l}{m}{n}$,
  \item $(A \otimes 1){X}$ and $ {X}(A \otimes 1)$ are contained in $A \otimes \Hom_{\C}^{0}(V)$,
  \item $(\Delta\otimes \id)(X)=X_{13}X_{23}$, 
  \item the sum $(\epsilon \otimes \id)({X}) :=\sum (\epsilon \otimes
    \id)(\Gr{X}{k}{l}{m}{n})$ converges in $M(\Hom_{0}(V))$ strictly
    to $\id_{V}$.
  \end{enumerate}
  Conversely, if $ X \in M(A \otimes \Hom_{\C}^{0}(V))$ satisfies
  0.--3.\ with $\Gr{X}{k}{l}{m}{n}$ defined by 0., then
  $\mathscr{X}=(\Gr{X}{k}{l}{m}{n})_{k,l,m,n}$ is a corepresentation
  of $\mathscr{A}$ on $V$.
\end{Lem}
\begin{proof}
 Straightforward.
\end{proof}
Morphisms of corepresentations are defined as follows.
\begin{Def}
  Let $\mathscr{A}$ be an $I$-partial bialgebra.  A \emph{morphism}
  $T$ between two corepresentations $(V,\mathscr{X})$ and $(W,\mathscr{Y})$ of
  $\mathscr{A}$ is a family of linear maps
  \[\Gru{T}{k}{l} \in
  \Hom_\C(\Gru{V}{k}{l},\Gru{W}{k}{l})\] satisfying \[(1 \otimes
  \Gru{T}{k}{l})\Gr{X}{k}{l}{m}{n} = \Gr{Y}{k}{l}{m}{n}(1 \otimes
  \Gru{T}{m}{n})\]
\end{Def}
We denote the category of all corepresentations of $\mathscr{A}$ by
$\Corep(\mathscr{A})$.
\begin{Rem}
  Equivalently, a morphism between corepresentations $(V,\mathscr{X})$
  and $(W,\mathscr{Y})$ is just a morphism of $I^{2}$-graded vector
  spaces $T\colon V\to W$ satisfying $(1\otimes T) X= Y(1 \otimes T)$.
\end{Rem}
Note that the categories $\Vecti$ and $\Corep(\mathscr{A})$ are
$\C$-linear and that the forgetful functor $\Corep(\mathscr{A})\to
\Vecti$ is faithful.

% We use here the standard leg numbering notation, e.g. $a_{12} =
% a\otimes 1$.

% We will sometimes for convenience consider $\Gr{X}{k}{l}{m}{n}$ as a map in $\Hom_\C(\Gru{V}{m}{n},\Gr{A}{k}{l}{m}{n}\otimes \Gru{V}{k}{l})$. % Or better to keep leg numbering notation because confusing otherwise? Or use $1\otimes $ notation?

% \begin{Def} Let $(V,X)$ be a
% corepresentation of a partial bialgebra $\mathscr{A}$. A family of subspaces
% \[\Gru{W}{k}{l} \subseteq \Gru{V}{k}{l}\]
% is called \emph{invariant} if \[(1\otimes \Gru{Q}{k}{l})\Gr{X}{k}{l}{m}{n}(1 \otimes \Gru{P}{m}{n}) =0,\]
% where $\Gru{P}{m}{n}:\Gru{W}{m}{n}\to\Gru{V}{m}{n}$ is the inclusion map and $\Gru{Q}{k}{l}:\Gru{V}{k}{l}\to\Gru{V}{k}{l}/\Gru{W}{k}{l}$
% denotes the quotient map.
% \end{Def}

% The following analogue of Schur's Lemma holds.

% \begin{Lem} Let $T$ be a morphism
% of corepresentations $(V,X)$ and
% $(W,Y)$. Then $ \bigoplus_{k,l} \ker \Gru{T}{k}{l}$ and
% $\bigoplus_{k,l} \img \Gru{T}{k}{l}$ are invariant subspaces of
% $V$ and $W$, respectively.
% \end{Lem} 

% In particular, if $(V,X)$ and $(W,Y)$ are irreducible, then a morphism $T$ from $V$ to $W$ either has all $\Gru{T}{k}{l}=0$ or all
% $\Gru{T}{k}{l}$ isomorphisms.

The category $\Vecti$ is a tensor category, where
the product of $I^{2}$-graded vector spaces $V$ and $W$ is the sum of
the subspaces
\begin{align*}
 \Gru{(V\itimes W)}{k}{l} =
  \bigoplus_{p} (\Gru{V}{k}{p}\otimes \Gru{V}{p}{l}) \subseteq
  V\otimes W,
\end{align*}
which we denote by $V \itimes W$, and the product of morphisms is the
restriction of the ordinary tensor product.  We pretend this product
to be strictly associative.  The unit for this product is the vector
space $\C^{(I)}=\bigoplus_{k\in I} \C$. Indeed, for every
$I^{2}$-graded vector space $V$, there exist obvious natural
isomorphisms $\C^{(I)} \itimes V \cong V \cong V \itimes \C^{(I)}$.

Given $V$ and $W$ as above, we identify $\Hom_\C(\Gru{V}{m}{n},\Gru{V}{k}{l})\otimes
   \Hom_\C(\Gru{W}{n}{q},\Gru{W}{l}{p})$ with a subspace of
\begin{align*}
   \Hom_\C(\Gru{V}{m}{n}\otimes
   \Gru{W}{n}{q},\Gru{V}{k}{l}\otimes \Gru{W}{l}{p})\subseteq
   \Hom_\C(\Gru{(V\itimes
     W)}{m}{q},\Gru{(V\itimes W)}{k}{p}).
\end{align*}


We can now construct a product of corepresentations as follows.
\begin{Lem} Let $\mathscr{X}$ and $\mathscr{Y}$ be copresentations of
  $\mathscr{A}$ on respective $I^{2}$-graded vector spaces $V$ and
  $W$. Then the sum
  \begin{align} \label{eq:corep-product-blocks}
     \Gr{(X\Circt Y)}{k}{p}{m}{q} := \sum_{l,n}
    \left(\Gr{X}{k}{l}{m}{n}\right)_{12}\left(\Gr{Y}{l}{p}{n}{q}\right)_{13}
  \end{align}
  has only finitely many non-zero terms and the elements
 \[\Gr{(X\Circt
    Y)}{k}{p}{m}{q}\in \Gr{A}{k}{p}{m}{q} \otimes
  \Hom_\C(\Gru{(V\itimes W)}{m}{q},\Gru{(V\itimes W)}{k}{p})
\]
define a corepresentation $\mathscr{X} \Circt \mathscr{Y}$ of
$\mathscr{A}$ on $V\itimes W$. 
\end{Lem} 
\begin{proof}
  The sum \eqref{eq:corep-product-blocks} is finite by condition 1.\
  in \ref{definition:corep}. Using the identification above, we see that
 $
  \left(\Gr{X}{k}{l}{m}{n}\right)_{12}\left(\Gr{Y}{l}{p}{n}{q}\right)_{13}
  $ lies in $ \Gr{A}{k}{p}{m}{q} \otimes \Hom_\C(\Gru{(V\itimes
    W)}{m}{q},\Gru{(V\itimes W)}{k}{p})$. Now,   the fact that $\Gr{(X\Circt
    Y)}{k}{p}{m}{q}$ is a corepresentation follows easily
  from the multiplicativity of $\Delta$ and the weak multiplicativity
  of $\epsilon$.
\end{proof}
\begin{Rem}
  The ``total'' multiplier associated to $\mathscr{X}\Circt
  \mathscr{Y}$   is  just $X_{12}Y_{13}$.
\end{Rem}

\begin{Prop} \label{prop:corep-monoidal} Let $\mathscr{A}$ be an
  $I$-partial bialgebra. The $\Corep(\mathscr{A})$ carries the
  structure of strict tensor category such that the product of
  corepresentations $(V,\mathscr{X})$ and $(W,\mathscr{Y})$ is the
  corepresentation $(V\itimes W,\mathscr{X}\Circt \mathscr{Y})$,  the
  unit is the trivial corepresentation $(\C^{(I)},\mathscr{U})$, and the forgetful functor
  $\Corep(\mathscr{A}) \to \Vecti$ is a strict tensor functor.
\end{Prop}
\begin{proof}
This is clear.
\end{proof}

%  Antipode duality.% To continue.

The following lemma shows that when $\mathscr{A}$ is a partial
\emph{Hopf} algebra, every corepresentation multiplier has a
generalized inverse.  
\begin{Lem}
  Let $\mathscr{X}$ be a corepresentation of an $I$-partial Hopf
  algebra $\mathscr{A}$ on an $I^{2}$-graded vector space $V$. Then
  \begin{align*}
    \Gr{X}{k}{l}{m}{n}  (S \otimes
      \id)(\Gr{X}{p}{q}{r}{s}) &=0 \text{ if } (l,m,n)\neq(s,p,q), &
      \sum_{n} \Gr{X}{k}{l}{m}{n} \cdot (S \otimes
      \id)(\Gr{X}{m}{n}{k'}{l}) &= \delta_{k,k'}\UnitC{k}{m} \otimes
      \id_{\Gru{V}{k}{l}}, \\
      (S \otimes \id)(\Gr{X}{k}{l}{m}{n}) \Gr{X}{p}{q}{r}{s} &= 0
      \text{ if } (k,m,n)\neq (r,p,q), & \sum_{m}
      (S \otimes \id)(\Gr{X}{k}{l}{m}{n}) \Gr{X}{m}{n}{k}{l'} &=
      \delta_{l,l'} \UnitC{n}{l} \otimes \id_{\Gru{V}{k}{l}}.
  \end{align*}
  In particular, the multiplier $Z:=     (S \otimes
  \id)(X) \in M(A \otimes \Hom_{\C}^{0}(V))$
  satisfies
  \begin{align} \label{eq:corep-generalized-inverse}
    XZ &= \sum_{k} \lambda_{k} \otimes \lambda^{V}_{k}, &
    ZX &= \sum_{n} \rho_{n} \otimes \rho^{V}_{n},
  \end{align}
  and is a generalized inverse of $X$ in the sense that $XZX=X$ and $ZXZ=Z$.
\end{Lem}
\begin{proof}
  The first equation follows from \eqref{eq:corep-blocks} and the
  relation  $S(\Gr{A}{p}{q}{r}{s})\subseteq \Gr{A}{s}{r}{q}{p}$. To
  verify the second one, we use relations 2.\ and 3.\ in Definition
  \ref{definition:corep} and \eqref{eq:antipode-pi-l} and find
  \begin{align*}
      \sum_{n} \Gr{X}{k}{l}{m}{n} \cdot (S \otimes
      \id)(\Gr{X}{m}{n}{k'}{l}) &= \sum_{n} (m_{A} \circ (\id \otimes S)
      \otimes \id)((\Gr{X}{k}{l}{m}{n})_{13}(\Gr{X}{m}{n}{k'}{l})_{23})
 \\ &= \sum_{n} (m_{A} \circ (\id \otimes S) \circ \Delta_{m,n} \otimes
      \id)(\Gr{X}{k}{l}{k'}{l}) \\
      &= \delta_{k,k'} \UnitC{k}{l} \otimes (\epsilon \otimes
      \id)(\Gr{X}{k}{l}{k'}{l})
      \\ &=
\delta_{k,k'}\UnitC{k}{m} \otimes
      \id_{\Gru{V}{k}{l}},  \end{align*}
where $m_{A}$ denotes the multiplication of $A$. The third and fourth
equation follow similarly, and the assertions concerning $Z$ are
direct consequences.
\end{proof}
For completeness, we mention the following following converse.
\begin{Lem}
  Let $\mathscr{A}$ be an $I$-partial bialgebra, let $V$ be an
  $I^{2}$-graded vector space and let $X \in M(A \otimes
  \Hom_{\C}^{0}(V))$. If $X$ satisfies conditions 0.--2.\ in Lemma
  \ref{lemma:corep-multiplier} and  \eqref{eq:corep-generalized-inverse} for
  some $Z \in M(A \otimes
  \Hom_{\C}^{0}(V))$, then the corresponding family
  $\mathscr{X}=(\Gr{X}{k}{l}{m}{n})_{k,l,m,n}$ is a corepresentation
  of $\mathscr{A}$ on $V$.
\end{Lem}
\begin{proof}
  We have to check condition 3.\ in Lemma
  \ref{lemma:corep-multiplier}.   Let $\Gr{T}{k}{l}{m}{n}:=(\epsilon
  \otimes \id)(\Gr{X}{k}{l}{m}{n}) \in
  \Hom(\Gru{V}{m}{n},\Gru{V}{k}{l})$.  If $(k,l) \neq (m,n)$, then
  $\epsilon(\Gr{A}{k}{l}{m}{n})=0$ and hence
  $ \Gr{T}{k}{l}{m}{n} =0$.  Now,  the counit property and condition
  2.\ in Lemma \ref{lemma:corep-multiplier} imply
\begin{align*}
  \Gr{X}{k}{l}{m}{n} &= ((\epsilon\otimes \id)\circ  \Delta \otimes
  \id)(  \Gr{X}{k}{l}{m}{n}) 
  = \sum_{p,q} (\epsilon\otimes \id \otimes
  \id)\left((\Gr{X}{k}{l}{p}{q})_{13}(\Gr{X}{p}{q}{m}{n})_{23}\right)
  =  (1 \otimes \Gr{T}{k}{l}{k}{l})\Gr{X}{k}{l}{m}{n}.
\end{align*}
The morphism of $V$ given by the family
$T=(\Gr{T}{k}{l}{k}{l})_{k,l}$ therefore satisfies $(1 \otimes T)X =
X$. Multiplying on the right by $Z$, we can conclude
$T\lambda^{V}_{k}=\lambda^{V}_{k}$ for all $k$. Thus, $T=\id_{V}$.
\end{proof}

To actually have a tensor category with duality, we need something stronger than the sfs condition.

\begin{Def} An $I^2$-graded vector space $V$ is called \emph{seperately finite dimensional (sfd)} if $\oplus_l \Gru{V}{k}{l}$ (resp. $\oplus_l \Gru{V}{k}{l}$) is finite dimensional for $k$ (resp. $l$) fixed. Correspondingly, we talk of an sfd corepresentation of a partial bialgebra $\mathscr{A}$, and we then denote by $\Corep_{\sfd}(\mathscr{A})$ the full subcategory of $\Corep_{\sfs}(\mathscr{A})$ consisting of sfd representations. 
\end{Def} 

One easily sees that $\Corep_{\sfd}(\mathscr{A})$ is closed under $\Circt$. 

\begin{Lem} Let $\mathscr{A}$ be a partial Hopf algebra. Then $\Corep_{\sfd}(\mathscr{A})$ is a tensor category with left duality. 
\end{Lem} 

\begin{proof} Let $X$ be an sfd corepresentation on a bigraded vector space $V$. Put \[\Gru{(V^*)}{k}{l} = (\Gru{V}{k}{l})^*,\] and let $V^*$ denote their direct sum bigraded vector space. Using the natural contravariant identification \[\Hom_\C(\Gru{V}{l}{k},\Gru{V}{n}{m})\cong \Hom_\C(\Gru{(V^*)}{m}{n},\Gru{(V^*)}{k}{l}),\] we see (by means of Lemma \ref{lemma:rep-corep}) that $X^{-1}$ gets transformed into a corepresentation $X^d$ on $V^*$. 

We claim that $X^d$ is the left dual of $X$. To see this, consider the evaluation maps \[\Gru{T}{k}{m}: \Gru{(V^*\itimes V)}{k}{m} \supseteq (\Gru{V}{l}{k})^*\otimes \Gru{V}{l}{m}\rightarrow \delta_{k,m} \C = \Gru{\C_I}{k}{m}.\] Then from Lemma \ref{lemma:rep-corep}, we obtain \begin{eqnarray*} (1\otimes \Gru{T}{k}{p})\Big{(} \sum_{l,n} \big{(} \Gr{(X^d)}{k}{l}{m}{n}\big{)}_{12} \big{(}\Gr{X}{l}{p}{n}{q}\big{)}_{13}\Big{)} &=& \delta_{p,k}(\id\otimes \Gru{T}{m}{m}) \left( \sum_{l,n}  \Gr{(X^{-1})}{k}{l}{m}{n}\Gr{X}{l}{k}{n}{q}\right)_{13} \\ &=& \delta_{p,k}\UnitC{k}{q}\otimes \Gru{T}{m}{q}.\end{eqnarray*} Hence the $\Gru{T}{k}{l}$ define an intertwiner between $V^*\itimes V$ and $\C_I$. Similarly, the maps  \[\Gru{R}{k}{k}: \Gru{\C_I}{k}{k} = \C\rightarrow \Gru{(V\itimes V^*)}{k}{k},\quad 1\mapsto \sum_{l,i} \Gru{v_i}{k}{l}\otimes \Gru{\omega_i}{l}{k},\] where the $\{\Gru{v_i}{k}{l}\mid i\}$ and $\{\Gru{\omega_i}{l}{k} \mid i\}$ form a dual basis of $\Gru{V}{k}{l}$, can be shown to form an intertwiner. It is then easy to check that $T$ and $R$ make $X^d$ into the left dual of $X$. % More info?
\end{proof}

Let us now enhance our partial Hopf algebras to partial compact quantum groups. One then considers corepresentations on sfd bigraded \emph{Hilbert spaces} such that the inverse of the corepresentation coincides with its adjoint. More precisely, we have the following definition. We denote $B(\Hsp,\mathcal{G})$ for the linear space of bounded morphisms between Hilbert spaces.

%\paragraph{Unitarity of corepresentations}

\begin{Def} Let $\mathscr{A}$ define a partial compact quantum group. We call an sfd corepresentation $(\mathcal{H},X)$ on an sfd $I^2$-graded Hilbert space $\mathcal{H}$
\emph{unitary} if \[\Gr{(X^{-1})}{k}{l}{m}{n}=(\Gr{X}{l}{k}{n}{m})^{*}\quad \textrm{in }\Gr{A}{k}{l}{m}{n}\otimes B(\Gru{\Hsp}{l}{k},\Gru{\Hsp}{n}{m}).\]  
\end{Def} 
\begin{Rem} \begin{enumerate} \item In the Hilbert space setting, it is more natural to let $\Hsp$ be the \emph{closed} (instead of the purely algebraic) direct sum of all (finite-dimensional) $\Gru{\Hsp}{k}{l}$. This does not change the notion of corepresentation, which had a local definition.
\item Concerning morphisms, we will say a collection of $\Gru{T}{k}{l}$ defines a \emph{bounded} intertwiner or morphism if the total operator $T= \oplus \Gru{T}{k}{l}$ is bounded. We will denote by $\Corep_{\sfd,u}(\mathscr{A})$ the category of unitary sfd corepresentations with arbitrary morphisms, and $\Corep_{\sfd,u}^{\infty}(\mathscr{A})$ for the category with bounded morphisms.
% Give a more prominent place, or check later if it is actually worthwhile to make this distinction.
\end{enumerate}
\end{Rem}

Our aim now is to show that every irreducible sfd corepresentation is
equivalent to a unitary one. We show this by embedding the corepresentation into a restriction of the
regular corepresentation.

\begin{Exa} \label{exa:rep-regular}
  Let $\mathscr{A}$ define a partial compact quantum group with normalized positive invariant functional $\phi$. 

  Let $\Gru{\mathcal{H}}{m}{n} \subseteq \bigoplus_{k,l}
  \Gr{A}{k}{l}{m}{n}$ be finite dimensional subspaces satisfying
\begin{align*}
  \Delta^{\op}_{pq}(\Gru{\mathcal{H}}{m}{n}) \subseteq
 \Gr{A}{p}{q}{m}{n} \otimes \Gru{\mathcal{H}}{p}{q}.
\end{align*}
for all indices.
Equip each $\Gru{\mathcal{H}}{k}{l}$ with the scalar product $\langle
a|b\rangle:=\phi(a^{*}b)$. By Lemma \ref{LemFaith}, these are finite-dimensional Hilbert spaces. Take the Hilbert space direct sum
$\mathcal{H}:=\bigoplus_{k,l} \Gru{\mathcal{H}}{k}{l}$.
Define \[ \Gr{V}{k}{l}{m}{n} \in \Hom_\C(\Gru{\mathcal{H}}{m}{n}, \Gr{A}{k}{l}{m}{n} \otimes \Gru{\mathcal{H}}{k}{l}) \cong \Gr{A}{k}{l}{m}{n} \otimes 
\mathcal{B}(\Gru{\mathcal{H}}{m}{n},\Gru{\mathcal{H}}{k}{l})\]  by the
equation
\begin{align*}
  \Gr{V}{k}{l}{m}{n}a &= \Delta^{\co}_{kl}(a).
\end{align*}
%where $\Gr{V}{k}{l}{m}{n}|a\rangle_{2}$  denotes the application of
%the second leg of
%$\Gr{V}{k}{l}{m}{n}$  to $a \in \Gru{\mathcal{H}}{m}{n}$. 

\begin{Lem} The couple $(\mathcal{H},V)$ defines a unitary corepresentation. 
\end{Lem} 
\begin{proof} It is clear that $V$ defines a corepresentation. It then suffices to prove that \begin{equation}\label{EqUnit} \sum_{k} (\Gr{V}{k}{l}{m}{n'})^* \Gr{V}{k}{l}{m}{n} = \delta_{n,n'}\UnitC{l}{n}\otimes \id_{\Gru{\Hsp}{m}{n}}.\end{equation} Take $a\in \Gru{\Hsp}{m}{n}$ and $b\in \Gru{\Hsp}{m}{n'}$. Then writing \[\Lambda(a): \C\rightarrow \Gru{\Hsp}{m}{n}, \quad 1\mapsto a,\] and similarly for $b$, we compute \begin{eqnarray*}(1\otimes \Lambda(b)^*)\left(\sum_{k} (\Gr{V}{k}{l}{m}{n'})^* \Gr{V}{k}{l}{m}{n}\right) (1\otimes \Lambda(a)) &=& \sum_k (\id\otimes \phi)(\Delta_{kl}^{\op}(b)^*\Delta_{kl}^{\op}(a))\\ &=&  \sum_k (\id\otimes \phi)(\Delta_{lk}^{\op}(b^*)\Delta_{kl}^{\op}(a)) \\ &=& (\id\otimes \phi)(\Delta_{ll}^{\op}(b^*a))\\ &=&  (\phi\otimes \id)(\Delta_{ll}(b^*a)) \\ &=& \phi(b^*a)\UnitC{l}{n} \\&=& \delta_{n',n} \UnitC{l}{n} \otimes \Lambda(b)^*\Lambda(a).\end{eqnarray*} 
This proves \eqref{EqUnit}.
\end{proof} 

We will call $(\mathcal{H},V)$  the 
\emph{sfd restriction of the regular corepresentation}
determined by the family $(\Gru{\mathcal{H}}{k}{l})_{k,l}$. 
\fxnote{proof this}
\end{Exa}

In the following, we will use the notation \[\omega_{\xi,\eta}:B(\Hsp,\mathcal{G})\rightarrow \C,\quad x\mapsto \langle \xi,x\eta\rangle,\quad \xi\in \mathcal{G},\eta\in \Hsp.\]

\begin{Lem} \label{lem:rep-morphism-regular}
  Let $\mathscr{A}$ define a partial compact quantum group. Let $(\mathcal{H},X)$ be an sfd corepresentation on a Hilbert space, and let $\xi \in
  \Gru{\mathcal{H}}{k}{l}$. Then the family of finite-dimensional
  subspaces
  \begin{align*}
   \Gru{\mathcal{K}}{m}{n}  &=  \{ (\id \otimes
   \omega_{\xi,\eta})(\Gr{X}{k}{l}{m}{n}) : \eta \in
   \Gru{\mathcal{H}}{m} {n}\} \subseteq \Gr{A}{k}{l}{m}{n}
  \end{align*}% Should explain notation
  defines an sfd restriction $(\mathcal{K},V)$ of the regular corepresentation, and the family of maps
  \begin{align*}
    \Gru{T_{(\xi)}}{m}{n} \colon \Gru{\mathcal{H}}{m}{n} \to
    \Gru{\mathcal{K}}{m}{n}, \ \eta \mapsto (\id \otimes
    \omega_{\xi,\eta})(\Gr{X}{k}{l}{m}{n}),
  \end{align*}
  is a morphism from $(\mathcal{H},X)$ to $(\mathcal{K},V)$ in $\Corep_{\sfd,u}(\mathscr{A})$. % Should $\xi$ then also be in a subscript for $\mathcal{K}$?
\end{Lem}
Note that the family $(\Gru{\mathcal{K}}{m}{n})_{m,n}$ is sfd because $(\Gru{\mathcal{H}}{m}{n})_{m,n}$
is. 
\begin{proof} Both assertions  follow from the fact
  that for all $\eta \in \Gru{\mathcal{H}}{p}{q}$,
\begin{eqnarray*}
\Delta^{\op}_{pq}(\Gru{T_{(\xi)}}{m}{n}(\eta)) &=&  \Delta_{pq}^{\op}\big((\id \otimes
  \omega_{\xi,\eta})(\Gr{X}{k}{l}{m}{n})\big)\\
 &=&(\id \otimes \id \otimes \omega_{\xi,\eta})\big(
  (\Gr{X}{k}{l}{p}{q})_{23}(\Gr{X}{p}{q}{m}{n})_{13})\big) \\ &=&(1
  \otimes \Gru{T_{(\xi)}}{p}{q})\Gr{X}{p}{q}{m}{n} \eta.
\end{eqnarray*}
\end{proof}
\begin{Prop}  \label{prop:rep-unitarisable}
  Let $\mathscr{A}$ define a partial compact quantum group. Then every irreducible sfd corepresentation on a Hilbert space
  is equivalent to a unitary sfd corepresentation.
\end{Prop}
\begin{proof}
  Let $(\mathcal{H},X)$ be an irreducible sfd
  corepresentation. Then for some $k,l$ and $\xi \in
  \Gru{\mathcal{H}}{k}{l}$,  the operator  $T_{(\xi)}$  defined in
  Lemma \ref{lem:rep-morphism-regular} has to be non-zero and hence,
  by Schur's Lemma, injective. Thus, it forms an equivalence between
  $(\mathcal{H},X)$ and a sub-corepresentation of an sfd
  restriction of the regular corepresentation, which is unitary by
  Example \ref{exa:rep-regular}.
\end{proof}

Our next goal is to obtain the analogue of Schur's orthogonality
relations for matrix coefficients of corepresentations.

\begin{Def} Let $\mathscr{A}$ define a partial compact quantum group. The space of \emph{matrix coefficients} $\mathcal{C}(X)$ of an sfd corepresentation $(\mathcal{H},X)$ is the sum of
the subspaces
\begin{align*}
  \Gr{\mathcal{C}(X)}{k}{l}{m}{n} &= \span \left\{ (\id \otimes
  \omega_{\xi,\eta})(\Gr{X}{k}{l}{m}{n}) \mid \xi \in
  \Gru{\mathcal{H}}{k}{l}, \eta \in \Gru{\mathcal{H}}{m}{n} \right\}
\subseteq \Gr{A}{k}{l}{m}{n}.
\end{align*}
\end{Def}

\begin{Lem} Every sfd unitary corepresentation $(X,\Hsp)$ of $\mathscr{A}$ decomposes into a direct sum of irreducible sfd unitary corepresentations.
\end{Lem} 
\begin{proof} From the unarity assumption, it follows immediately that an invariant subspace of $\Hsp$ also has an invariant orthogonal complement. Hence irreducibility and indecomposability of unitary corepresentations coincide. More generally, one deduces that the bounded self-interwiners of $\Hsp$ form a (von Neumann) $^*$-algebra.

% Trivial rep should maybe be introduced in a more conspicuous place
Let us now first show that the trivial representation decomposes into irreducibles. Let $I$ be the object set of $\mathscr{A}$, and say $k\sim l$ if $\UnitC{k}{l}\neq 0$. Then $\sim$ is an equivalence relation: as \[\Delta_{ll}(\UnitC{k}{m}) = \UnitC{k}{l}\otimes \UnitC{l}{m},\] the relation $\sim$ is transitive. As $S(\UnitC{k}{l}) = \UnitC{l}{k}$, we have that $\sim$ is symmetric. And as $\varepsilon(\UnitC{k}{k})=1$, we also have that $\sim$ is reflexive. 

Let then $I = \sqcup_{\alpha\in \mathscr{I}} I_{\alpha}$ be a labeled partition associated to $\sim$. Define $\C_{I_{\alpha}}\subseteq \C_I$ as the linear span of the homogeneous components with index in $\alpha$. It is clear then that the $\C_{I_{\alpha}}$ are invariant and irreducible.

Consider now a general corepresentation $(X,\Hsp)$. Let $\Grd{\Hsp}{\alpha}{\beta}$ be the closed linear span of the homogeneous components with index in $\alpha\times \beta$. As we can identify \[\Grd{\Hsp}{\alpha}{\beta} \cong \C_{I_{\alpha}}\,\Circt\, \Hsp\,\Circt\, \C_{I_{\beta}},\] we see that $\Grd{\Hsp}{\alpha}{\beta}$ is an invariant subspace of $\Hsp$. Hence we may as well suppose that $\Hsp = \Grd{\Hsp}{\alpha}{\beta}$. 

But let then $T$ be a bounded self-intertwiner of $\Hsp$. Then from the two equations in Remark \ref{RemMorRep}, we see that $T\rightarrow \Gru{T}{k}{l}$ is injective for any choice of $k\in \alpha,l\in \beta$. It follows that the algebra of self-intertwiners of $\Hsp$ is finite-dimensional. We then immediately conclude that $\Hsp$ is a finite direct sum of irreducible invariant subspaces.
\end{proof} 

\begin{Prop} \label{prop:rep-weak-pw}
  Assume that $\mathscr{A}$ defines a partial compact quantum group. Then the total algebra $A$ is the sum of the matrix coefficients of
 irreducible unitary sfd corepresentations.
\end{Prop}
\begin{proof}
  Let $a \in \Gr{A}{k}{l}{m}{n}$. Then
  $\Delta^{\co}_{pq}(a) \in
  \Gr{A}{p}{q}{m}{n} \otimes \Gr{A}{k}{l}{p}{q}$, and the subspace
    \begin{align*}
    \Gru{\mathcal{H}}{p}{q} &:= \span \{ (\omega \otimes \id)(
    \Delta^{\co}_{pq}(a)) : \omega \in \Hom_{\C}(\Gr{A}{p}{q}{m}{n},\C) \} \subseteq
    \Gr{A}{k}{l}{p}{q}
  \end{align*}
  has finite dimension. Since $a$ is fixed, the
  $(\Gru{\mathcal{H}}{p}{q})_{p,q}$ are in fact sfd. Using
  co-associativity, one checks that this family defines an sfd restriction $(\mathcal{H},V)$ of the regular
  corepresentation. Evidently, $a \in \mathcal{C}(V)$. Decomposing
  $(\mathcal{H},V)$, we find that
  $a$ is contained in the sum of matrix coefficients of unitary
  irreducible corepresentations.
\end{proof}
The key to the orthogonality relations is the following averaging procedure.
\begin{Lem} \label{lem:rep-average}
  Let $\mathscr{A}$ define a partial compact quantum group, and let  $\phi$ be any invariant functional for $\mathscr{A}$. Let
  $(\mathcal{H},X)$ and $(\mathcal{K},Y)$ be sfd corepresentations of $\mathscr{A}$ and let $T$ be
  a family of operators $\Gru{T}{k}{l} \in
  \mathcal{B}(\Gru{\mathcal{H}}{k}{l},\Gru{\mathcal{K}}{k}{l})$.
  
  Then for any fixed $n$, the family of linear maps
  \begin{align*}
    \Gru{\check T_n}{k}{l} &:= \sum_{m} (\phi \otimes
    \id)(\Gr{(Y^{-1})}{n}{m}{l}{k}(1\otimes
    \Gru{T}{m}{n})\Gr{X}{m}{n}{k}{l})
   \end{align*} 
   
  define a morphism $\check T_n$ from $(\mathcal{H},X)$ to $(\mathcal{K},Y)$ in $\Corep_{\sfd,u}(\mathscr{A})$.

     Similarly, for fixed $m$, the      
    
  \begin{align*} \Gru{\hat T_m}{k}{l} &:= \sum_{n} (\phi \otimes
    \id)(\Gr{Y}{k}{l}{m}{n}(1\otimes
    \Gru{T}{m}{n})\Gr{(X^{-1})}{l}{k}{n}{m})
  \end{align*}
  define a morphism from $(\mathcal{H},X)$ to $(\mathcal{K},Y)$. % Slightly changed the averaging so that I do not need a restriction of finite support on $T$.
\end{Lem} 
\begin{proof}
 Using Remark \ref{RemMorRep}, the assertion concerning the $\check{T}$ follows
  from the calculation
  \begin{align*}
    &\sum_{m} \Gr{(Y^{-1})}{n}{m}{l'}{k}(1 \otimes
    \Gru{\check{T}_q}{m}{n})\Gr{X}{m}{n}{k}{l} = \\
    &=\sum_{m,p} (\phi \otimes \id \otimes
    \id)\left(\left(\Gr{(Y^{-1})}{n}{m}{l'}{k}\right)_{23}
      \left(\Gr{(Y^{-1})}{q}{p}{n}{m}\right)_{13}(1 \otimes 1 \otimes
      \Gru{T}{p}{q})\left(\Gr{X}{p}{q}{m}{n}\right)_{13}\left(\Gr{X}{m}{n}{k}{l}\right)_{23}\right)
    \\
    % &= \sum_{m,p,q}(\phi \otimes \id \otimes \id)\left((1 \otimes
    %   \lambda_{n} \otimes 1)(\tilde \Delta \otimes
    %   \id)\left(\Gr{(Y^{-1})}{k}{l'}{p}{q}\right)(\rho_{m} \otimes 1
    %   \otimes \Gru{T}{p}{q})(\tilde \Delta \otimes
    %   \id)\left(\Gr{(X}{p}{q}{k}{l}\right)(1 \otimes\lambda_{n}
    %   \otimes 1)\right) \\
    &= \sum_{m,p} (\left((\phi \otimes \id)
    \circ \Delta_{mn}\right) \otimes \id)\left(\Gr{(Y^{-1})}{q}{p}{l'}{k}(1
      \otimes \Gru{T}{p}{q})\Gr{X}{p}{q}{k}{l}\right) 
    \\
    &= \delta_{l',l}\UnitC{n}{l}\otimes \left(\sum_{p,q} (\phi \otimes \id)\left(\Gr{(Y^{-1})}{q}{p}{l}{k}(1 
      \otimes \Gru{T}{p}{q})\Gr{X}{p}{q}{k}{l}\right)\right)\\
  &= \delta_{l,l'}\UnitC{n}{l} \otimes \Gru{\check{T}_q}{k}{l},
  \end{align*}
  where we used the relation $\phi(\Gr{A}{r}{s}{l'}{l})=0$ for $l'\neq l$
  for the last equality. 
  
  A similar calculation proves the assertion
  concerning the $\hat{T}$. 
\end{proof}

% Choose a representative family of unitary irreducible locally finite
% corepresentations
% $(\Grd{\mathcal{H}}{(\alpha)}{},{_{(\alpha)}X})_{\alpha}$ and a basis
% $(\Gr{\zeta}{k}{l}{(\alpha)}{i})_{i}$ for each
% $\Gr{\mathcal{H}}{k}{l}{(\alpha)}{}$, and let
% \begin{align*}
%   (\Gr{(u_{\alpha})}{k}{l}{m}{n}){i,j} &:= (\id \otimes
%   \Gr{\omega}{k}{l}{(\alpha)}{i,j})( )
% \end{align*}

The first part of the orthogonality relations concerns matrix
coefficients of inequivalent irreducible corepresentations.
\begin{Prop} \label{prop:rep-orthogonality-1}
  Let $(\mathcal{H},X)$ and $(\mathcal{K},Y)$ be inequivalent unitary
  irreducible sfd corepresentations, and let
  $\phi$ be an invariant functional for $(A,\Delta)$.  Then
  \[\phi(S(b)a) =\phi(b^{*}a) = \phi(bS(a))=\phi(ba^{*})=0\] for all
  $a\in \mathcal{C}(X), b \in \mathcal{C}(Y)$.
\end{Prop}
\begin{proof}
  Let $a=(\id \otimes \omega_{\xi,\xi'})(\Gr{X}{k}{l}{m}{n})$ and
  $b=(\id \otimes \omega_{\eta,\eta'})(\Gr{Y}{p}{q}{r}{s})$, where
  $\xi \in \Gru{\mathcal{H}}{k}{l}, \xi' \in \Gru{\mathcal{H}}{m}{n}$
  and $\eta \in \Gru{\mathcal{K}}{p}{q}, \eta' \in
  \Gru{\mathcal{K}}{r}{s}$. 
  
  If $(p,q,r,s) \neq (m,n,k,l)$, then clearly $\phi(S(b)a) = 0 = \phi(bS(a))$.
  
   If $(p,q,r,s) = (m,n,k,l)$,  then Lemma \ref{lem:rep-average}, applied to the 
  family $\Gru{T}{p}{q}= \delta_{p,k}\delta_{q,l}
  |\eta'\rangle\langle\xi|$, yields  morphisms $\check{T}_l,\hat{T}_k$
  from $(\mathcal{H},X)$ to $(\mathcal{K},Y)$ which necessarily are
  $0$. Inserting the definition of $\check{T}_l$, we find
  \begin{align*}
    \phi(S(b)a) &= \phi\big((S \otimes
    \omega_{\eta,\eta'})(\Gr{Y}{m}{n}{k}{l}) \cdot (\id \otimes
    \omega_{\xi,\xi'})(\Gr{X}{k}{l}{m}{n})\big) \\ &= (\phi \otimes
    \id)\left((1\otimes \langle\eta|) \Gr{(Y^{-1})}{l}{k}{n}{m}(1 \otimes
      |\eta'\rangle\langle \xi|)
      \Gr{X}{k}{l}{m}{n}(1\otimes |\xi'\rangle)\right) 
   \\ &= (1\otimes \langle \eta|) \Gru{\check{T}_l}{m}{n}(1\otimes |\xi'\rangle) = 0.
  \end{align*}% Resort notation on using leg notation, physics bra-ket, etc.
  
  A similar calculation involving $\hat{T}$ shows that
  $\phi(bS(a))=0$.  
  
  Using the relation $X^{*}=X^{-1}=(S\otimes
  \id)(X)$ and $Y^{*}=(S\otimes \id)(Y)$, we conclude
  $\phi(b^{*}a)=\phi(ba^{*})=0$.
\end{proof}

The second part of the orthogonality relations concerns inner products
as above but with $a,b\in \mathcal{C}(X)$ for some irreducible
corepresentation  $X$. It involves the conjugate corepresentation,
which is defined as follows.

Given Hilbert spaces $\Hsp,\mathcal{K}$, we denote by $\overline{\Hsp},\overline{\mathcal{K}}$
the conjugate Hilbert spaces, by $T \mapsto \overline{T}$ the
canonical conjugate-linear isomorphism $\mathcal{B}(\Hsp,\mathcal{K}) \to
\mathcal{B}(\overline{\Hsp},\overline{\mathcal{K}})$, and by $T \mapsto
T^{\top}:=\overline{T}^{*}$ the linear anti-isomorphism % Or contravariant isomorphism?
$\mathcal{B}(\Hsp,\mathcal{K}) \to \mathcal{B}(\overline{\mathcal{K}},\overline{\Hsp})$.

\begin{Lem} \label{lemma:rep-functors}Let $\Corep_{\sfd,\Hilb}(\mathscr{A})$ denote the category of (not necessarily unitary) corepresentations of $\mathscr{A}$ on sfd bigraded Hilbert spaces. 
   Then on $\Corep_{\sfd,\Hilb}(\mathscr{A})$ there exist
  \begin{enumerate}
  \item a covariant antilinear functor $(\mathcal{H},X) \mapsto
    (\overline{\mathcal{H}},\overline{X})$ and $T \mapsto
    \overline{T}$, where
    \begin{align*}
      \Gru{\overline{\mathcal{H}}}{k}{l} &= \overline{\Gru{\mathcal{H}}{l}{k}},
      & \Gr{\overline{X}}{k}{l}{m}{n} &= (\Gr{X}{l}{k}{n}{m})^{(*
        \otimes \overline{(\ \cdot \ ) })}
      =((\Gr{X}{l}{k}{n}{m})^{*})^{\id \otimes \top}, &
      \Gru{\overline{T}}{k}{l} &= \overline{\Gru{T}{l}{k}};
    \end{align*}
  \item a contravariant linear functor $(\mathcal{H},X) \mapsto
    (\overline{\mathcal{H}},X^{S\otimes \top})$ and
    $T\mapsto T^{\top}$, where 
    \begin{align*}\Gru{\overline{\mathcal{H}}}{k}{l} &= \overline{\Gru{\mathcal{H}}{l}{k}},&
     & \Gr{(X^{S\otimes \top})}{k}{l}{m}{n} = (S \otimes (\ \cdot \
      )^{\top})(\Gr{X}{n}{m}{l}{k}), & \Gru{(T^{\top})}{k}{l}
      &=(\Gru{T}{l}{k})^{\top};
    \end{align*}
  \item a covariant linear functor  $(\mathcal{H},X) \mapsto (\mathcal{H},X^{S^{2}\otimes \id})$
    and $T\mapsto T$, where the grading is unchanged and \[\Gr{(X^{S^{2}\otimes \id})}{k}{l}{m}{n}=(S^{2} \otimes
  \id)(\Gr{X}{k}{l}{m}{n}).\]
  \end{enumerate}
  If $(\mathcal{H},X)$ is unitary, then $\overline{X}=X^{S\otimes \top}$.\end{Lem}
\begin{proof}
The first assertion follows immediately from the fact that $\Delta_{rs}(a^*)=\Delta_{sr}(a)^*$ and the $^*$-compatibility of $\epsilon$. The second assertion follows from the fact that $\Delta_{pq}\circ S  = (S\otimes S)\circ \Delta_{qp}^{\op}$ and $\epsilon\circ S = \epsilon$. The final functor is just the square of the second functor. 

The fact that $\overline{X}=X^{S\otimes \top}$ for $X$ unitary is just by definition. % yes?
 % Some of these properties should be mentioned before.
  
  
  %\begin{align*}
  %  (\Delta_{pq} \otimes \id)( \Gr{\overline{X}}{k}{l}{m}{n}) &=
    %( \Gr{\overline{X}}{k}{l}{p}{q})_{13}
    %( \Gr{\overline{X}}{p}{q}{m}{n})_{23}, \\
    %(\epsilon \otimes \id)(\Gr{\overline{X}}{k}{l}{m}{n}) &=
   % \overline{(\epsilon \otimes \id)(\Gr{X}{l}{k}{n}{m})}
   % = \delta_{l,n}\delta_{k,m}
   % \overline{\id_{\Gru{\mathcal{H}}{l}{k}}}
   % = \delta_{l,n}\delta_{k,m} \id_{\Gru{\overline{\mathcal{H}}}{k}{l}}.
 % \end{align*}
  
\end{proof}
We call $(\overline{\mathcal{H}},\overline{X})$ the \emph{conjugate} of 
$(\mathcal{H},X)$.   

\begin{Prop} \label{prop:rep-f}
  Let $\mathscr{A}$ define a partial compact quantum group, and let $\phi$ be a positive normalized
  invariant functional. Let $(\mathcal{H},X)$ be a unitary
  irreducible sfd corepresentation. 
  \begin{enumerate}
  \item The conjugate $\overline{\mathcal{H}}$ with the family
    $\Gr{(\overline{X}^{-*})}{k}{l}{m}{n}:=\big(\Gr{(\overline{X}^{-1})}{l}{k}{n}{m}\big)^{*}$
    form an sfd corepresentation, and there
    exists an invertible, positive morphism $\overline{F_{X}}$ from
    $(\overline{\mathcal{H}},\overline{X})$ to
    $(\overline{\mathcal{H}},\overline{X}^{-*})$. % Better notation?
  \item The family
    $\Gru{F_{X}}{k}{l}:=\overline{\Gru{\overline{F_{X}}}{l}{k}}$ is an
    invertible, positive operator implementing a morphism from $(\mathcal{H},X)$ to
    $(\mathcal{H},X^{S^{2} \otimes \id})$.
  \end{enumerate}
\end{Prop}
\begin{proof}
(1) By Proposition \ref{prop:rep-unitarisable},
$(\overline{\mathcal{H}},\overline{X})$ is equivalent to a unitary
corepresentation, that is, there exists a family of invertible operators
$\Gru{T}{k}{l} \in \mathcal{B}(\Gru{\overline{\mathcal{H}}}{k}{l})$
such that the family 
\begin{align*}
\Gr{Z}{k}{l}{m}{n}:= (1 \otimes
\Gru{T}{k}{l})\Gr{\overline{X}}{k}{l}{m}{n}(1 \otimes
\Gru{T}{m}{n})^{-1} 
\end{align*}
 is a unitary corepresentation. The relation
 $\Gr{(Z^{-1})}{n}{m}{l}{k}=\big(\Gr{Z}{m}{n}{k}{l})^{*}$ then implies
 \begin{align*}
   \Gr{Z}{k}{l}{m}{n} = (1 \otimes
   (\Gru{T}{k}{l})^{-1})^{*}\big(\Gr{(\overline{X}^{-1})}{l}{k}{n}{m}\big)^{*}(1
   \otimes \Gru{T}{m}{n})^{*}
 \end{align*}
 and hence the family 
 $\Gr{(\overline{X}^{-*})}{k}{l}{m}{n}:=\big(\Gr{(\overline{X}^{-1})}{l}{k}{n}{m}\big)^{*}$
 is an irreducible sfd corepresentation. The maps
 $\Gru{\overline{F}_{X}}{k}{l}:=(\Gru{T}{k}{l})^{*}\Gru{T}{k}{l} \in
  \mathcal{B}(\Gru{\overline{\mathcal{H}}}{k}{l})$
then form an isomorphism from $(\overline{\mathcal{H}},\overline{X})$ to
$(\overline{\mathcal{H}},\overline{X}^{-*})$.   

(2) The morphism $T$  from $(\overline{\mathcal{H}},\overline{X})$  to
$(\overline{\mathcal{H}},Z)$ yields morphisms $\overline{T}$ from
$(\mathcal{H},X)$ to $(\mathcal{H},\overline{Z})$ and $T^{\top}$ from $(\mathcal{H},Z^{S\otimes\top})$ to
$(\mathcal{H},\overline{X}^{S \otimes \top})$. Since $X$ and $Z$ are
unitary, $\overline{Z}=Z^{S\otimes \top}$ and  $\overline{X}^{S \otimes
  \top} = X^{S^{2} \otimes \id}$. Thus $T^{\top}\overline{T} =
\overline{T^{*}T}$ is a morphism from $(\mathcal{H},X)$ to
$(\mathcal{H},X^{S^{2}\otimes \id})$.
\end{proof}


\begin{Theorem} \label{thm:rep-orthogonality} Let $\mathscr{A}$ define a partial
compact quantum group. 
  Let $\phi$ be a positive normalized invariant functional. Let $(\mathcal{H},X)$ be a unitary irreducible sfd corepresentation of $\mathscr{A}$, and let $F_{X}$ be a
  non-zero morphism from $(\mathcal{H},X)$ to $(\mathcal{H},X^{S^{2}
  \otimes \id})$.
  \begin{enumerate}
  \item The numbers $\alpha:=\sum_{k} \Tr(\Gru{(F_{X}^{-1})}{k}{l})$
    and $\beta:=\sum_{n} \Tr(\Gru{F_{X}}{m}{n})$ do not depend on $l$
    or $n$.
  \item  For all $k,l,m,n$,
    \begin{align*}
      (\phi \otimes \id)((\Gr{X}{k}{l}{m}{n})^{*}\Gr{X}{k}{l}{m}{n})
      &=\alpha^{-1}{\Tr(\Gru{(F^{-1}_{X})}{k}{l})} \cdot
      \id_{\Gru{\mathcal{H}}{m}{n}}, \\
      (\phi \otimes \id)(\Gr{X}{k}{l}{m}{n}(\Gr{X}{k}{l}{m}{n})^{*})
      &=\beta^{-1}{\Tr( \Gru{(F_{X})}{m}{n})} \cdot
      \id_{\Gru{\mathcal{H}}{k}{l}}.
    \end{align*}
  \item Denote by $\Sigma_{klmn}$ the flip map $\Gru{\mathcal{H}}{k}{l}
    \otimes \Gru{\mathcal{H}}{m}{n} \to \Gru{\mathcal{H}}{m}{n}
    \otimes \Gru{\mathcal{H}}{k}{l}$. Then
 \begin{align*}
   (\phi \otimes \id \otimes
   \id)((\Gr{X}{k}{l}{m}{n})_{12}^{*}(\Gr{X}{k}{l}{m}{n})_{13}) &=
   \alpha^{-1}
   (\id_{\Gru{\mathcal{H}}{m}{n}} \otimes \Gru{(F_{X}^{-1})}{k}{l})
   \circ \Sigma_{klmn}, \\
   (\phi \otimes \id \otimes
   \id)((\Gr{X}{k}{l}{m}{n})_{13}(\Gr{X}{k}{l}{m}{n})_{12}^{*}) &= \beta^{-1} (\Gru{F_{X}}{m}{n}
   \otimes \id_{\Gru{\mathcal{H}}{k}{l}}) \circ \Sigma_{klmn}.
 \end{align*}
\end{enumerate}
  \end{Theorem}
\begin{proof}
  We prove the assertions and equations involving $\alpha$ in (1), (2)
  and (3)  simultaneously; the assertions involving $\beta$  follow similarly.

  %As above, we denote by $\Sigma_{p,q,r,s}$ the flip
  %$\Gru{\mathcal{H}}{p}{q} \otimes \Gru{\mathcal{H}}{r}{s} \to
  %\Gru{\mathcal{H}}{r}{s} \otimes \Gru{\mathcal{H}}{p}{q}$.  
  Consider
  the following endomorphism $F_{m,n,k,l}$ of $\Gru{\mathcal{H}}{m}{n}\otimes \Gru{\mathcal{H}}{k}{l}$, 
  \begin{align*}
    F_{m,n,k,l}
    &:=(\phi \otimes \id \otimes \id)((\Gr{X}{k}{l}{m}{n})_{12}^{*}(\Gr{X}{k}{l}{m}{n})_{13})
    \circ \Sigma_{mnkl} \\ &= (\phi \otimes \id \otimes
    \id)\left((\Gr{(X^{-1})}{l}{k}{n}{m})_{12}
      \Sigma_{klkl,23}(\Gr{X}{k}{l}{m}{n})_{12}\right).
  \end{align*}
  By applying Lemma \ref{lem:rep-average} with respect to the flip map $\Sigma_{klkl}$, we see that the family $(F_{m,n,k,l})_{m,n}$ is
  an endomorphism of $(\mathcal{H} \otimes \Gru{\mathcal{H}}{k}{l}, X_{12})$ and hence
  \begin{align}
    F_{m,n,k,l} &= \id_{\Gru{\mathcal{H}}{m}{n}} \otimes \Gru{R}{k}{l} \label{eq:rep-orthogonal-1}
  \end{align}
  with some $\Gru{R}{k}{l} \in \mathcal{B}(\Gru{\mathcal{H}}{k}{l})$ not
  depending on $m,n$. % In using irreducibility, do we not miss any subtlety in allowing non-bounded morphisms?
  On the other hand, 
  \begin{align*}
    F_{m,n,k,l} &= (\phi \otimes \id \otimes \id)((S \otimes
    \id)(\Gr{X}{m}{n}{k}{l})_{12}(\Gr{X}{k}{l}{m}{n})_{13})
    \circ \Sigma_{mnkl} \\
    &= (\phi\circ S^{-1} \otimes \id \otimes \id)\left(((S \otimes
      \id)(\Gr{X}{k}{l}{m}{n}))_{13}
      ((S^{2} \otimes \id)(\Gr{X}{m}{n}{k}{l}))_{12}\right)     \circ \Sigma_{mnkl}\\
    &= (\phi\circ S^{-1} \otimes \id \otimes \id)\left(((S \otimes
      \id)(\Gr{X}{k}{l}{m}{n}))_{13} (\Sigma_{mnmn})_{23} ((S^{2}
      \otimes \id)(\Gr{X}{m}{n}{k}{l}))_{13}\right).
  \end{align*}
  Since $\phi\circ S^{-1}$ is an invariant functional for
  $\mathscr{A}$, we can again apply Lemma \ref{lem:rep-average} and
  find that the family $(F_{m,n,k,l})_{k,l}$ is a morphism \[(F_{m,n,k,l})_{k,l}:
  (\Gru{\mathcal{H}}{m}{n} \otimes \mathcal{H}, (X^{S^{2} \otimes
    \id})_{13})\rightarrow (\Gru{\mathcal{H}}{m}{n} \otimes \mathcal{H},
  X_{13}).\] Therefore,
  \begin{align}
    F_{m,n,k,l} &= \Gru{T}{m}{n} \otimes (\Gru{F_{X}}{k}{l})^{-1} \label{eq:rep-orthogonal-2}
  \end{align}
  with some $\Gru{T}{m}{n} \in \mathcal{B}(\Gru{\mathcal{H}}{m}{n})$
  not depending on $k,l$. Combining \eqref{eq:rep-orthogonal-1} and
  \eqref{eq:rep-orthogonal-2}, we conclude that, for some $\lambda\in \C$, \[F_{m,n,k,l} = \lambda
  (\id_{\Gru{\mathcal{H}}{m}{n}} \otimes (\Gru{F_{X}}{k}{l})^{-1})\]
  
  Choose a basis
  $(\zeta_{i})_{i}$ for $\Gru{\mathcal{H}}{k}{l}$. Then
  \begin{align*}
    \lambda \cdot \id_{\Gru{\mathcal{H}}{m}{n}} \cdot
    \Tr((\Gru{F_{X}}{k}{l})^{-1}) &= \sum_{i} (\id \otimes
    \omega_{\zeta_{i},\zeta_{i}})(F_{m,n,k,l}) = (\phi \otimes
    \id)((\Gr{X}{k}{l}{m}{n})^{*} \Gr{X}{k}{l}{m}{n}).
  \end{align*}
  Taking $n=l$ and summing over $k$, the relations $\sum_{k}
  (\Gr{X}{k}{l}{m}{n})^{*} \Gr{X}{k}{l}{m}{n} = \UnitC{l}{n}
  \otimes \id_{\Gru{\mathcal{H}}{m}{n}}$ and
  $\phi(\UnitC{l}{l})=1$ give
\begin{align*}
\lambda \cdot  \sum_{k} \Tr((\Gru{F_{X}}{k}{l})^{-1}) = 1.
\end{align*}
Now all assertions in (1)--(3) concerning $\alpha$ follow.
 % the second formula, we use the first formula for the
 %    opposite of $(A,\Delta)$. For this opposite, $\phi$ still is a
 %    faithful, positive, normalized invariant functional and
 %    $(\mathcal{H},X)$ still is a unitary irreducible locally finite
 %    corepresentation, but the antipode $S$ gets replaced by $S^{-1}$
 %    and therefore $F_{X}$ gets replaced by $F_{X}^{-1}$.
\end{proof}
\begin{Cor}\label{CorOrth}
  Assume that $\mathscr{A}$ defines a partial compact quantum group, and let $\phi$ be a normalized positive  invariant functional. Let $(\mathcal{H},X)$ be a unitary
  irreducible sfd corepresentation of $\mathscr{A}$, let
  $F_{X}$ be a non-zero morphism from $(\mathcal{H},X)$ to
  $(\mathcal{H},(S^{2} \otimes \id)(X))$, and let $a=(\id \otimes
  \omega_{\xi,\xi'})(\Gr{X}{k}{l}{m}{n})$ and $b=(\id \otimes
  \omega_{\eta,\eta'})(\Gr{X}{k}{l}{m}{n})$, where $\xi,\eta \in
  \Gru{\mathcal{H}}{k}{l}$ and $\xi',\eta' \in \Gru{\mathcal{H}}{m}{n}$.
  Then
\begin{align*}
  \phi(b^{*}a) &= \frac{\langle \eta'|\xi'\rangle \langle
    \xi|F^{-1}_{X}\eta\rangle}{\sum_{m}
    \Tr(\Gru{(F^{-1}_{X})}{m}{n})}, & \phi(ab^{*}) = \frac{\langle
    \eta'|F_{X}\xi'\rangle \langle \xi|\eta\rangle}{\sum_{n}
    \Tr(\Gru{F_{X}}{m}{n})}.
\end{align*}
\end{Cor}
\begin{proof}
  By Theorem \ref{thm:rep-orthogonality}, 
  \begin{align*}
    \phi(b^{*}a) &= (\phi \otimes \omega_{\eta',\eta} \otimes
    \omega_{\xi,\xi'})((\Gr{X}{k}{l}{m}{n})_{12}^{*}(\Gr{X}{k}{l}{m}{n})_{13})
    \\
    &= \frac{1}{\sum_{k} \Tr(\Gru{(F_{X}^{-1})}{k}{l})} 
    (\omega_{\eta',\eta} \otimes
    \omega_{\xi,\xi'})(   (
    \id_{\Gru{\mathcal{H}}{m}{n}} \otimes \Gru{(F_{X}^{-1})}{k}{l})
    \circ \Sigma_{k,l,m,n}). 
  \end{align*}
  The formula for $\phi(ab^{*})$ follows similarly or by considering
  the co-opposite of $\mathscr{A}$.
\end{proof}
\begin{Cor} \label{cor:rep-pw}
  Let $\mathscr{A}$ define a partial compact quantum group. Let
  $(\mathcal{H}_{\alpha},X_{\alpha})_{\alpha}$ be a representative
  family of all irreducible sfd corepresentations of
  $\mathscr{A}$. Then the map
  \begin{align*}
    \bigoplus_{\alpha} \bigoplus_{k,l,m,n}
    (\overline{\Gru{\mathcal{H}_{\alpha}}{k}{l}} \otimes
    \Gru{\mathcal{H}_{\alpha}}{m}{n}) \to A
  \end{align*}
  that sends $\overline{\xi} \otimes \eta \in
  \overline{\Gru{\mathcal{H}_{\alpha}}{k}{l}} \otimes
  \Gru{\mathcal{H}_{\alpha}}{m}{n}$ to $ (\id \otimes
  \omega_{\xi,\eta})(\Gr{(X_{\alpha})}{k}{l}{m}{n})$,
  is a linear isomorphism.
\end{Cor}
\begin{proof} This follows from Proposition \ref{prop:rep-weak-pw}, Proposition \ref{prop:rep-orthogonality-1} and Corollary \ref{CorOrth}.
\end{proof}

Suppose now $a\in \Gr{A}{k}{l}{m}{n}$ for some partial bialgebra $\mathscr{A}$. Then for $\omega \in \Hom_{\C}(A,\C)$, we can define
\begin{align*}
  \omega \aste{p,q} a
&:= (\id \otimes \omega) (\Delta_{pq}(a)), & a \aste{r,s}
\omega&:=(\omega \otimes \id)(\Delta_{rs}(a)).\end{align*} Clearly we can define
\begin{align*} \omega \aste{p,q} a \aste{r,s}
\omega'&:= (\omega \aste{p,q} a)\aste{r,s} \omega' = \omega \aste{p,q}(a \aste{r,s} \omega').\end{align*}
When $\omega$ has support on the $A(K)$ with $K_u=K_d$, we can write, for $a\in \Gr{A}{k}{l}{m}{n}$, \[\omega\ast a := \sum_{p,q} \omega\aste{p,q}a = \omega\aste{m,n}a,\quad  a\ast \omega = \sum_{r,s} a\aste{r,s}\omega = a\aste{k,l}\omega.\] 

We shall say that an entire function $f$ has \emph{exponential growth
  on the right half-plane} if there exist $C,d>0$ such that $|f(x+iy)|\leq
C\mathrm{e}^{dx}$  for all $x,y\in \R$ with $x>0$. 

\begin{Theorem} \label{thm:rep-characters}
  Assume that $\mathscr{A}$ defines a partial compact quantum group with positive normalized
  invariant functional $\phi$.  There exists a unique family of linear functionals
  $f_{z} \colon A\to \C$ such that
\begin{enumerate}[label={(\arabic*)}]
  \item $f_z$ vanishes on $A(K)$ when $K_u\neq K_d$.
  \item for each $a\in A$, the function $z\mapsto f_{z}(a)$ is entire
    and of exponential growth on the right half-plane.
  \item $f_{0} = \epsilon$ and $(f_{z} \otimes f_{z'}) \circ 
    \Delta= f_{z+z'}$ for all $z,z' \in \C$.
  \item $\phi(ab)=\phi(b(f_{1} \ast a \ast f_{1}))$ for all $a,b\in A$.
  \end{enumerate}
  This family furthermore satisfies
  \begin{enumerate}[label={(\arabic*)}]\setcounter{enumi}{4}
  \item $f_z(ab) = f_z(a)f_z(b)$ for $a\in A(K)$ and $b\in A(L)$ with $K_r = L_l$. 
  \item $S^{2}(a)=f_{-1} \ast a \ast f_{1}$ for all $a\in A$.
  \item $f_{z}(\UnitC{l}{n})=\delta_{l,n}$,  $f_{z} \circ S = f_{-z}$,
and    $f_{z}(a^*) = \overline{f_{-\overline{z}}(a)}$ for all $a\in A$.
\end{enumerate}
\end{Theorem}


Note that condition (3) is meaningful by condition (1).

\begin{proof}
  We first prove uniqueness.  Assume that $(f_{z})_{z}$ is a family of
  functionals satisfying (1)--(4).  Since $\phi$ is faithful, the map
  $\sigma\colon a \mapsto f_{1} \ast a \ast f_{1}$ is uniquely
  determined by $\phi$, and one easily sees that it is a homomorphism. Using
  (3), we find that $\epsilon \circ \sigma^n=f_{2n}$, which uniquely determines these functionals. Using (2) and the
  fact that every entire function of exponential growth on the right
  half-plane is uniquely determined by its values at $\N \subseteq \C$, we can conclude that the family $f_{z}$ is uniquely determined. Moreover, since the property (5) holds for $z = 2n$, we also conclude by the same argument as above that it holds for all $z\in \C$.

  Let us now prove existence.  By Corollary \ref{cor:rep-pw}, we can
  define for each $z\in \C$ a functional $f_{z} \colon A \to \C$ such
  that for every unitary irreducible sfd corepresentation
  $(\mathcal{H},X)$,
    \begin{align*}
      f_{z}((\id \otimes \omega_{\xi,\eta})(\Gr{X}{k}{l}{m}{n})) &=
      \delta_{k,m}\delta_{l,n} \cdot
      \omega_{\xi,\eta}((\Gru{F_{X}}{k}{l})^{z}) \quad \text{for all }
      \xi \in \Gru{\mathcal{H}}{k}{l},\eta \in
      \Gru{\mathcal{H}}{m}{n},
    \end{align*}
    or, equivalently,
    \begin{align*}
      (f_{z} \otimes \id)(\Gr{X}{k}{l}{m}{n}) =
      \delta_{k,m}\delta_{l,n} \cdot (\Gru{F_{X}}{k}{l})^{z},
    \end{align*}
    where $F_{X}$ is a non-zero positive operator implementing a morphism from $(\mathcal{H},X)$ to
    $(\mathcal{H}, (S^{2} \otimes \id)(X))$, scaled such that
    \begin{align*}
      \alpha_{X}:= \sum_{k} \Tr(\Gru{(F_{X}^{-1})}{k}{l}) = \sum_{n}
      \Tr(\Gru{F_{X}}{m}{n})
    \end{align*}
    for all $l,n$ (see Proposition \ref{prop:rep-f} and Theorem \ref{thm:rep-orthogonality}). By
    construction, (1) and (2) hold. We show that the $(f_{z})_{z}$ satisfy the
    assertions (3)--(7). 
    %We have already argued that (5) is satisfied
    %$f_{z}$ is a character. 
    Throughout the following arguments, let 
    $(\mathcal{H},X)$ be a unitary irreducible corepresentation
    $(\mathcal{H},X)$ and let $F_{X}$ be as above.

    We first prove property (3). This follows from the relations
    \begin{align*}
      (f_{0}  \otimes \id)(\Gr{X}{k}{l}{m}{n}) &=
      \delta_{k,m}\delta_{l,n} \id_{\Gru{\mathcal{H}}{k}{l}} =
      (\epsilon \otimes \id)(\Gr{X}{k}{l}{m}{n})
    \end{align*}
    and
    \begin{align*}
      (((f_{z}\otimes f_{z'})\circ \Delta) \otimes
      \id)(\Gr{X}{k}{l}{m}{n}) &=  \delta_{k,m}\delta_{l,n}(f_{z} \otimes f_{z'} \otimes
      \id)\big((\Gr{X}{k}{l}{k}{l})_{13}
      (\Gr{X}{k}{l}{k}{l})_{23}\big) \\
      &=  \delta_{k,m}\delta_{l,n}(\Gru{F_{X}}{k}{l})^{z}  \cdot (\Gru{F_{X}}{k}{l})^{z'} \\
      &= (f_{z+z'} \otimes \id)(\Gr{X}{k}{l}{m}{n}).
    \end{align*}
    Applying slice maps of the form $\id
    \otimes \omega_{\xi,\xi'}$ and invoking Theorem \ref{thm:rep-orthogonality}, this proves (3).

% Again? Check if this has already been used before   
    To prove (4), write again $ \Delta^{(2)} = (
    \Delta \otimes \id)\circ  \Delta = (\id \otimes 
    \Delta) \circ \Delta$, and put \[\theta_{z,z'}:=(f_{z'} \otimes \id
    \otimes f_{z})\circ  \Delta^{(2)}.\] Then
    \begin{align*}
      (\theta_{z,z'} \otimes \id)(\Gr{X}{k}{l}{m}{n}) &= (f_{z'} \otimes
      \id \otimes f_{z} \otimes
      \id)((\Gr{X}{k}{l}{k}{l})_{14}(\Gr{X}{k}{l}{m}{n})_{24}(\Gr{X}{m}{n}{m}{n})_{34})
      \\
      &= (1 \otimes (\Gru{F_{X}}{k}{l})^{z'}) \Gr{X}{k}{l}{m}{n} (1
      \otimes (\Gru{F_{X}}{m}{n})^{z}).
    \end{align*}
    We take $z=z'=1$, use Theorem \ref{thm:rep-orthogonality}, where
    now $\alpha= \beta$ by our scaling of $F_{X}$, and obtain
    \begin{eqnarray*}
     && \hspace{-2cm} (\phi \otimes \id \otimes
      \id)((\Gr{X}{k}{l}{m}{n})_{12}^{*}((\theta_{1,1} \otimes
      \id)(\Gr{X}{k}{l}{m}{n}))_{13})\\ && =\alpha^{-1}(\id \otimes
      \Gru{F_{X}}{k}{l}) (\id \otimes \Gru{(F_{X}^{-1})}{k}{l})
      \Sigma_{k,l,m,n} (\id \otimes
      \Gru{F_{X}}{m}{n}) \\
      &&=\beta^{-1}(\Gru{F_{X}}{m}{n} \otimes \id) \Sigma_{k,l,m,n} \\
      &&= (\phi \otimes \id \otimes
      \id)((\Gr{X}{k}{l}{m}{n})_{13}(\Gr{X}{k}{l}{m}{n})_{12}^{*}).
    \end{eqnarray*}
    To conclude the proof of assertion (4), apply again slice maps of the form
    $\omega_{\xi,\xi'} \otimes \omega_{\eta,\eta'}$.

We have then already argued that the property (5) automatically holds. To show the property (6), note that by Proposition \ref{prop:rep-f} and the calculation above,
    \begin{align*}
      (S^{2} \otimes \id)(\Gr{X}{k}{l}{m}{n}) &= (1
      \otimes\Gru{F_{X}}{k}{l})
      \Gr{X}{k}{l}{m}{n}(1 \otimes \Gru{F_{X}}{m}{n})^{-1} 
      =(\theta_{-1,1}  \otimes \id)(\Gr{X}{k}{l}{m}{n}).
    \end{align*}
     Assertion (6) follows again by applying slice maps.
    
     Finally, (1), (2) and (4)
     immediately imply the relation
     $f_{z}(\UnitC{k}{m})=\delta_{k,m}$. The concrete construction of $f_z$ combined with property (3), the identity \eqref{eq:rep-delta2} and the partial character property (5) gives the equality
     \begin{align*}
       (f_{-z} \otimes \id) (\Gr{X}{k}{l}{k}{l})=
       (\Gru{(F_{X})}{k}{l})^{-z} &=\left( (f_{z} \otimes
       \id)(\Gr{X}{k}{l}{k}{l})\right)^{-1} \\ &= (f_{z} \otimes
       \id)(\Gr{(X^{-1})}{l}{k}{l}{k}) = ((f_{z} \circ S) \otimes
       \id)(\Gr{X}{k}{l}{k}{l}).
     \end{align*}
Therefore, $f_{-z} = f_{z} \circ S$. Let us now write $\bar{f}_z(a) = \overline{f_z(a^*)}$. Using the preceding calculation,
     the relation $(S \otimes \id)(\Gr{X}{k}{l}{k}{l}) =
     (\Gr{X}{k}{l}{k}{l})^{*}$ and positivity of $\Gru{F_{X}}{k}{l}$,
     we conclude
     \begin{align*}
       (\bar{f}_z \otimes
       \id)(\Gr{X}{k}{l}{k}{l})
&=       \left((f_{z} \otimes
       \id)((\Gr{X}{k}{l}{k}{l})^{*})\right)^{*} \\
& = \left((f_{-z} \otimes \id)(\Gr{X}{k}{l}{k}{l})\right)^{*} 
 =
((\Gru{F_{X}}{k}{l})^{-z})^{*} 
=       (\Gru{F_{X}}{k}{l})^{-\overline{z}} = (f_{-\overline{z}}
\otimes \id)(\Gr{X}{k}{l}{k}{l}),
     \end{align*}
whence $\bar{f}_z = f_{-\overline{z}}$.
\end{proof}
\begin{Cor} \label{cor:rep-characters}
  Let $\mathscr{G}$ be an partial compact quantum group with
  underlying total algebra $A$ and define $\theta_{z,z'}
  \colon A \to A$ by $a \mapsto f_{z} \ast a \ast f_{z'}$ for each
  $z,z' \in \C$, where the functionals $f_{z}$ are as in Theorem
  \ref{thm:rep-characters}. Then for all $z,z',w,w'\in \C$, the
  following conditions hold:
  \begin{enumerate}
  \item $\theta_{z,z'}$ is an algebra automorphism and preserves
    each subspace $A(K)$; in particular,
    $\theta_{z,z'}(\lambda_{k}\rho_{m}) = \lambda_{k}\rho_{m}$ for all
    $k,m\in I$;
  \item $\theta_{z,z'} \circ * = * \circ
    \theta_{-\overline{z},-\overline{z'}}$; in particular,
    $\theta_{it,is}$ is a $*$-automorphism for each $t,s\in \R$;
  \item $\theta_{z,z'}\circ \theta_{w,w'} = \theta_{z+w,z'+w'}$;
  \item $ (\theta_{w,z'} \otimes \theta_{z,-w}) \circ \Delta = \Delta
    \circ \theta_{z,z'}$, $\epsilon \circ \theta_{z,z'} = f_{z+z'}$,
    $\theta_{z,z'} \circ S = S \circ \theta_{-z',-z}$ and
    $\phi \circ \theta_{z,z'} = \phi$;
  \item for every linear map $\omega \colon A \to \C$ and every $a\in
    A$, the map $(z,z') \mapsto \omega(\theta_{z,z'}(a))$ is entire.
  \end{enumerate}
\end{Cor}
\begin{proof}
  All of this follows easily from Theorem \ref{thm:rep-characters}.
\end{proof}
Using the two-parameter group $\theta$, we define the \emph{modular
  automorphism group} $\sigma$, the \emph{scaling group} $\tau$   and
the \emph{unitary antipode} of a partial compact quantum group $A$ by
\begin{align*}
  \sigma_{z} &:=\theta_{iz,iz}, & \tau_{z} &:=\theta_{iz,-iz}, & R&:=S
  \circ \tau_{i/2}.
\end{align*}
Using Corollary \ref{cor:rep-characters}, one verifies that
$\sigma,\tau,R$ share all the main relations known for locally compact
quantum groups and measured quantum groupoids, for example, $\sigma$
and $\tau$ are complex one-parameter groups of algebra automorphisms
of $A$, the map $R$ is a $*$-anti-automorphism,  $\tau_{t}$ and
$\sigma_{t}$ are $*$-automorphisms for all $t\in \R$, the family  $\tau$ commutes with
$\sigma$ and with $R$ in the obvious sense, 
  \begin{gather}
    \begin{aligned} \label{eq:modular}
      \phi\circ \sigma_{z} &= \phi \circ \tau_{z} = \phi \circ R =
      \phi, & \phi(ab) &= \phi(b\sigma_{-i}(a)),
    \end{aligned}
\\ \label{eq:scaling-modular-delta}
    \begin{aligned} 
    \Delta \circ \tau_{z} &= (\tau_{z} \otimes \tau_{z}) \circ \Delta,
    & (\tau_{z} \otimes \sigma_{z}) \circ \Delta &= \Delta \circ
    \sigma_{z} = (\sigma_{-z} \otimes \tau_{z}) \circ \Delta,      
  \end{aligned} \\
  \begin{aligned} \label{eq:unitary-antipode}
    R^{2} &= \id_{A}, & \Delta \circ R &= (R \otimes R) \circ
    \Delta^{\op}.
  \end{aligned}
  \end{gather}




%  consider
%   the family
%   \begin{align*}
%     (\Gru{F}{m}{n})^{\top} \circ  \overline{\Sigma_{m,n,k,l}} &=
% \phi((\Gr{X}{k}{l}{m}{n})_{12}^{*}(\Gr{X}{k}{l}{m}{n})_{13})^{\top}
% \circ  \overline{\Sigma_{m,n,k,l}} \\
%  &=
%  \phi((\Gr{X}{k}{l}{m}{n})_{12}^{*\circ (\id \otimes
%   \top)}(\overline{\Sigma_{k,l,k,l}})_{23} (\Gr{X}{k}{l}{m}{n})^{\id \otimes
%   \top}_{12})  \\
% &=\phi((\Gr{\overline{X}}{l}{k}{n}{m})_{12}(\overline{\Sigma_{k,l,k,l}})_{23}
%  (\Gr{\overline{X}}{l}{k}{n}{m})_{12}).
% \end{align*} \fxnote{Treat $X^{\id \otimes \top}$}
% By Lemma \ref{lem:rep-average}, this family is a morphism from to
% $(\overline{\mathcal{H}}\otimes \overline{K},\overline{X}^{-*} \otimes
% \id_{\overline{K}})$  to
% $(\overline{\mathcal{H}}\otimes \overline{K},\overline{X} \otimes
% \id_{\overline{K}})$ and hence of the form
% $(\Gru{\overline{F_{X}}}{n}{m})^{-1} \otimes T$ with $T \in
% \mathcal{B}(\overline{K})$ not depending on $m,n$.

% Thus,
% \begin{align*}
%  (\id \otimes R)  \circ \Sigma_{k,l,m,n} =   \Gru{F}{m}{n} =
%  \Sigma_{k,l,m,n} \circ (\Gru{\overline{F_{X}}}{n}{m})^{-\top} \otimes T^{\top})
% \end{align*}

  
%   We may assume $(k,l,m,n)=(p,q,r,s)$ because otherwise both sides of
%   the equation that we want to prove vanish.

%   Applying Lemma \ref{lem:rep-average} to the corepresentation $X$ and
%   the family $\Gru{T}{p}{q}=
%   \delta_{p,k}\delta_{q,l} |\eta\rangle\langle\xi|$, we obtain an
%   endomorphism $\check{T}$ of $(\mathcal{H},X)$ which
%   necessarily has the form $\check{T}=\lambda(\xi,\eta) \id$ for some
%   $\lambda(\xi,\eta) \in \C$. Inserting the definition of
%   $\check{T}$, we find
%   \begin{align}\nonumber
%     \phi(b^{*}a) &= \phi\big((\id \otimes
%     \omega_{\eta',\eta})((\Gr{X}{k}{l}{m}{n})^{*}) \cdot (\id \otimes
%     \omega_{\xi,\xi'})(\Gr{X}{k}{l}{m}{n})\big) \\  &= (\phi \otimes
%     \id)\left(\langle\eta'|_{2} \Gr{(X^{-1})}{m}{n}{k}{l}(1 \otimes
%       |\eta\rangle\langle \xi|)
%       \Gr{X}{k}{l}{m}{n}|\xi'\rangle_{2}\right) 
%     = \langle \eta'|_{2} \Gru{\check{T}}{m}{n}|\xi'\rangle_{2} =
%     \lambda(\xi,\eta) \langle\eta'|\xi'\rangle. \label{eq:rep-orthogonal-1}
%   \end{align}
%   Next, we apply Lemma \ref{lem:rep-average} to the corepresentations
%   $\overline{X}$ and $\overline{X}^{-*}$ and the family
%   $\Gru{R}{p}{q}=\delta_{p,m}\delta_{q,n}|\overline{
%     \xi'}\rangle\langle\overline{\eta'}|$, and obtain a morphism
%   $\hat{R}$ from $\overline{X}^{-*}$ to $\overline{X}$ which
%   necessarily has the form $\hat{R}=\mu(\eta',\xi')\overline{F_{X}}$
%   for some $\mu(\eta',\xi') \in \C$. Using the relation
%   \begin{align*}
%     a &= (\id \otimes
%     \omega_{\overline{\xi},\overline{\xi'}})(\Gr{\overline{X}}{l}{k}{n}{m})^{*},
%     & b&= (\id \otimes
%     \omega_{\overline{\eta},\overline{\eta'}})(\Gr{\overline{X}}{k}{l}{m}{n})^{*}
%   \end{align*}
%   and the definition of $\hat{R}$, we obtain
%   \begin{align}
%     \phi(b^{*}a) &= (\phi \otimes \id)\left(\langle
%       \overline{\eta}|_{2} \Gr{\overline{X}}{k}{l}{m}{n}(1 \otimes
%       |\overline{\eta}'\rangle\langle \overline{\xi'}|)
%       \Gr{(\overline{X}^{*})}{m}{n}{k}{l} \right) \nonumber \\
%     &=\langle \overline{\eta}| \Gru{\hat{R}}{k}{l}
%     |\overline{\xi}\rangle = \langle
%     \overline{\eta}|\Gru{\overline{F_{X}}}{k}{l}\overline{\xi}\rangle
%     \mu(\eta',\xi'). \label{eq:rep-orthogonal-2}
%   \end{align}
%   We choose a basis $(\zeta_{i})_{i}$ for
%   $\bigoplus_{k}\Gru{\mathcal{H}}{k}{l}$ and calculate
%   \begin{align*}
%  \langle
%     \eta'|\xi'\rangle &=  (\phi \otimes
%     \id)(\langle\eta'|_{2}(\lambda_{l}\rho_{n} \otimes
%     \id_{\Gru{\mathcal{H}}{m}{n}})|\xi'\rangle_{2}) \\
%  &=
%     \sum_{k} (\phi \otimes \id)\left(\langle\eta'|_{2}
%       \Gr{(X^{-1})}{m}{n}{k}{l}
%       \Gr{X}{k}{l}{m}{n}|\xi'\rangle_{2}\right)
%     \\ &=    \sum_{i} \lambda(\zeta_{i},\zeta_{i}) \langle
%     \eta'|\xi'\rangle 
%     \\
%     &=\sum_{i} \langle
%     \overline{\zeta_{i}}| \Gru{\overline{F_{X}}}{}{l}\overline{\zeta_{i}}\rangle
%     \mu(\eta',\xi') 
% \\ &    = \mu(\eta',\xi') \cdot \sum_{k} \Tr(\Gru{F_{(X)}}{k}{l}),
%   \end{align*}
% where $\Gru{\overline{F_{X}}}{}{l}=\bigoplus_{k}
% \Gru{\overline{F_{X}}}{k}{l}$. Inserting this relation into
% \eqref{eq:rep-orthogonal-2}, we finally obtain the assertion.


%%% Local Variables: 
%%% mode: latex
%%% TeX-master: "dyn-suq-main"
%%% End: 

\input{3-tannaka.tex}
\section{Compact Hopf face algebras on the level of operator algebras}

It then follows by symmetry that also the maps \[(W^{k,t}_{m,n,u,v})^*: \oplus_l \Gr{A}{k}{l}{m}{n} \otimes \Gr{A}{l}{t}{u}{v} \rightarrow \oplus_r \Gr{A}{k}{t}{m}{r}\otimes \Gr{A}{n}{r}{u}{v}\] defined by the formula \[a\otimes b\rightarrow \Delta(b)(a\otimes 1)\] are unitaries, with inverse map $a\otimes b\mapsto S^{-1}(b_{(1)})a\otimes b_{(2)}$.

\begin{Lem}\label{LemUni} Let $(A,\Delta)$ be a generalized compact face algebra. Then each $V^{k,l,s,t}_{m,v}$ is a unitary, and similarly for the $W^{k,t}_{m,n,u,v}$.
\end{Lem}

\begin{proof} It is immediately checked that $V^{k,l,s,t}_{m,v}$ is isometric.
\end{proof}

Let us write $\mathscr{L}^2(A,\varphi)$ for the completion of $A$ with respect to the inner product $\langle a,b\rangle = \varphi(a^*b)$. The canonical inclusion of $A$ into $\mathscr{L}^2(A)$ will be denoted $\Lambda$.

\begin{Lem} Assume $(A,\Delta)$ is a generalized compact face algebra. The representation of $A$ by left multiplication on itself extends to a representation by bounded operators on the completion $\mathscr{L}^2(A,\varphi)$.
\end{Lem}

\begin{proof} Denote $\omega_{\xi,\eta}(x) = \langle \xi,x\eta\rangle$ for $\xi,\eta$ vectors and $x$ a bounded operator. Then a straightforward computation shows that \[(\omega_{\Lambda(a),\Lambda(b)}\otimes \id)(V) = \varphi(a^*b_{(1)})b_{(2)}\] as a left multiplication operator. As $(A\otimes 1)\Delta(A) = (A\otimes A)\Delta(1)$ by Lemma \ref{LemUni} (applied to the opposite algebra), it follows by normalization of $\varphi$ that each element of $A$ can be represented in the form $(\omega_{\Lambda(a),\Lambda(b)}\otimes \id)(V)$, and hence extends to a bounded operator on $\mathscr{L}^2(A,\varphi)$.
\end{proof}

In the following, we will abbreviate $\mathscr{L}^2(A)$ by $L^2A$.

Let $(A,\Delta)$ be a generalized compact face algebra. Denote the von Neumann algebraic completion of $A\subseteq B(L^2A)$ by $M$. Denote $L^2A\iitimes L^2A = E(L^2A\otimes L^2A)$, where $E = \sum_p \rho_p\otimes \lambda_p$ is extended to a bounded operator (in fact, a self-adjoint projection). Finally, denote $M\itimes M = E(M\otimes M)E$. Then $M\itimes M$ is the von Neumann algebraic completion of $A\itimes A$.

Extend now the $V^{k,l,s,t}_{m,v}$ to unitaries \[V: \oplus_p L^2(A_p)\otimes L^2({}_pA) \rightarrow \oplus_p L^2({}_pA)\otimes L^2({}^pA)= E(L^2A\otimes L^2A).\] Then we can construct a map \[\Delta: M\rightarrow M\itimes M,\quad x\rightarrow V(x\otimes 1)V^*.\] By direct computation, we see that $\Delta$ extends the comultiplication map on $A$. It is then immediate to check that $\Delta$ is in fact coassociative (where one may as well consider $\Delta$ as a non-unital map from $M$ to $M\otimes M$).

We aim to show that $(M,\Delta)$ can be fitted into the theory of measured quantum groupoids.

%%% Local Variables: 
%%% mode: latex
%%% TeX-master: "dyn-suq-main"
%%% End: 

% In definition of faithful $I^2$-algebra, we can indeed just suppose that the embedding $\Ff(I^2)\rightarrow M(A)$ is faithful.
% Correct use of $\cdot$ versus operator product?

% Podles sphere can be defined as an algebra in $\Rep(SU_q(2))$. Hence, it should make sense as an algebra under the forgetful functor to SU_q(2)-dynamical. By duality for coideals, the same should hold for passage to SU_q(1,1) in fact...

% Also: direct construction should help in making non-compact construction

% Thank Piotr, Makoto and ...(?) for discussions. Link with graph C*-algebras?

\section{Partial compact quantum groups from reciprocal random walks}

We recall some notions introduced in \cite{DCY1}. We slightly change the terminology for the sake of convenience.

% Should we assume that the graph has a finite number of components? 

\begin{Def} Let $t\in \R_0$. A \emph{$t$-reciprocal random walk} consists of a quadruple $(\Gamma,w,\sgn,i)$ with \begin{itemize}
\item[$\bullet$] $\Gamma=(\Gamma^{(0)},\Gamma^{(1)},s,t)$ a graph with \emph{source} and \emph{target} maps \[s,t:\Gamma^{(1)}\rightarrow \Gamma^{(0)},\]
\item[$\bullet$] $w$ a function (the \emph{weight} function) $w:\Gamma^{(1)}\rightarrow \R_0^+$,
\item[$\bullet$] $\sgn$ a function (the \emph{sign} function) $\sgn:\Gamma^{(1)}\rightarrow \{\pm 1\}$,
\item[$\bullet$] $i$ an involution \[i:\Gamma^{(1)} \rightarrow \Gamma^{(1)},\quad e\mapsto \overline{e}\] with $s(\bar{e}) = t(e)$ for all edges $e$,
\end{itemize}
such that the following conditions are satisfied:
\begin{itemize}
\item[$\bullet$] $w(e)w(\bar{e}) = 1$ for all edges $e$,
\item[$\bullet$] $\sgn(e)\sgn(\bar{e}) = \sgn(t)$ for all edges $e$,
\item[$\bullet$] $p(e) = \frac{1}{|t|}w(e)$ defines a \emph{random walk}:   $\sum_{s(e)=v} p(e) = 1$ for all $v\in \Gamma^{(0)}$.
\end{itemize}
\end{Def}

%Isomorphisms between $t$-random walks will simply be isomorphisms of weighted graphs (and hence do not remember any structure concerning involutions or signedness). 

For some examples of $t$-reciprocal random walks, we refer to \cite{DCY1}. One particular example which will be needed for our construction of dynamical quantum $SU(2)$ is the following.

\begin{Exa}\label{ExaGraphPod} Take $0<|q|<1$ and $x\in \R$. Write $2_q = q+q^{-1}$. Then we have the reciprocal $-2_q$-random walk \[\Gamma_x =(\Gamma_x,w,\sgn,i)\] with \[ \Gamma^{(0)} = \Z,\quad \Gamma^{(1)} = \{(k,l)\mid |k-l|= 1\}\subseteq \Z\times \Z\] with projection on the first (resp. second) leg as source (resp. target) map, with weight function \[w(k,k\pm 1) = \frac{|q|^{x+k\pm 1}+|q|^{-(x+k\pm 1)}}{|q|^{x+k}+|q|^{-(x+k)}},\] sign function \[\sgn(k,k+1) = 1,\quad \sgn(k,k-1) = -\sgn(q),\] and involution $\overline{(k,k+1)} = (k+1,k)$. 

By translation we shift the value of $x$ by an integer, and by inversion we change $x$ into $-x$ and multiply the sign function with a fixed sign. It follows that by some graph isomorphism, we can always arrange to have $x\in \lbrack 0,\frac{1}{2}\rbrack$ at the cost of having a different sign function.
\end{Exa} 

Let now $0<|q|\leq 1$, and let $SU_q(2)$ be Woronowicz's twisted $SU(2)$ group \cite{Wor1}. Then $SU_q(2)$ is a compact quantum group, and its category of finite-dimensional unitary representations $\Rep(SU_q(2))$ is generated by the spin $1/2$-representation $\pi_{1/2}$ on $\C^2$.

Let $\Gamma = (\Gamma,w,\sgn,i)$ be a $-2_q$-reciprocal random walk. Define $\Hsp_{\Gamma}$ as the $\Gamma^{(0)}$-bigraded Hilbert space $l^2(\Gamma^{(1)})$, where the $\Gamma^{(0)}$-bigrading is given by \[\delta_e \in \Gru{\Hsp_{\Gamma}}{s(e)}{t(e)}\] for the obvious Dirac functions. Further define $R_{\Gamma}$ as the (bounded) map \[R_{\Gamma}:l^2(\Gamma^{(0)})\rightarrow \Hsp_{\Gamma}\underset{\Gamma^{(0)}}{\otimes} \Hsp_{\Gamma}\] given by \begin{eqnarray*} R_{\Gamma} \delta_v &=& \sum_{e,s(e) = v} \sgn(e)\sqrt{w(e)}\delta_e \otimes \delta_{\bar{e}}.\end{eqnarray*} Then $R_{\Gamma}^*R_{\Gamma} = |q|+|q|^{-1}$ and \[(R_{\Gamma}^*\otimes \id)(\id\otimes R_{\Gamma}) = -\sgn(q)\id.\]


%Then we can define a couple $\Hsp(\Gamma) = (\Hsp,R)$ where $\Hsp$ is $l^2(\Gamma^{(1)})$ with the $\Gamma^{(0)}$-bigrading $\delta_e \in \Hsp(s(e),t(e))$ for the obvious Dirac functions, and where  

% Dropped Temperley-Lieb terminology

Hence, by the universal property of $\Rep(SU_q(2))$ (\cite[Theorem 1.4]{DCY1}, based on \cite{Tur1,Eti1,Yam1,Pin2,Pin3}), we have a strongly monoidal $^*$-functor
\[F_{\Gamma}: \Rep(SU_q(2)) \rightarrow {}^{\Gamma^{(0)}}\Hilb_f^{\Gamma^{(0)}}\] such that $F_{\Gamma}(\pi_{1/2}) = \Hsp_{\Gamma}$ and $F_{\Gamma}(\mathscr{R}) = R_{\Gamma}$, with \[(\pi_{1/2},\mathscr{R},-\sgn(q)\mathscr{R})\] a solution for the conjugate equations for $\pi_{1/2}$. Up to equivalence, $F_{\Gamma}$ only depends upon the isomorphism class of $(\Gamma,w)$, and is independent of the chosen involution or sign structure.

% Concrete reference to be added.
It follows from our main theorem that for each reciprocal random walk on a graph $\Gamma$, one obtains a $\Gamma^{(0)}$-partial compact quantum group. Our aim is to give a direct representation of it by generators and relations.

\begin{Theorem}\label{TheoGenRel} Let $0<|q|\leq 1$, and let $\Gamma$ be a $-2_q$-reciprocal random walk. Let $A(\Gamma)$ be the total $^*$-algebra associated to the $\Gamma^{(0)}$-partial compact quantum group constructed from the fiber functor $F_{\Gamma}$. Then $A(\Gamma)$ is the universal $^*$-algebra generated by a copy of the $^*$-algebra of finite support functions on $\Gamma^{(0)}\times \Gamma^{(0)}$ (with the Dirac functions written as $\UnitC{v}{w}$) and elements $(u_{e,f})_{e,f\in \Gamma^{(1)}}$ where $u_{e,f}\in \Gr{A(\Gamma)}{s(e)}{t(e)}{s(f)}{t(f)}$ and 
\begin{eqnarray} 
\label{EqUni1}\sum_{v\in \Gamma^{(0)}}\sum_{g\in \Gamma_{vw}} u_{g,e}^*u_{g,f} = \delta_{e,f}\mathbf{1}\Grru{w}{t(e)}, \qquad \forall w\in \Gamma^{(0)}, e,f\in \Gamma^{(1)},\\
\label{EqUni2}\sum_{w\in \Gamma^{(0)}} \sum_{g\in \Gamma_{vw}} u_{e,g}u_{f,g}^* = \delta_{e,f} \mathbf{1}\Grru{s(e)}{v}\qquad \forall v\in \Gamma{(0)}, e,f\in \Gamma^{(1)},\\ 
\label{EqInt}u_{e,f}^* \;=\; \sgn(e)\sgn(f)\sqrt{\frac{w(f)}{w(e)}} u_{\bar{e},\bar{f}},\qquad \forall e,f\in \Gamma^{(1)}.
\end{eqnarray}

If moreover $v,w\in \Gamma^{(0)}$ and $e,f\in \Gamma^{(1)}$, we have \[\Delta_{vw}(u_{e,f}) = \underset{t(g) = w}{\sum_{s(g) = v}} u_{e,g}\otimes u_{g,f},\]
\[\varepsilon(u_{e,f}) = \delta_{e,f}\] and \[S(u_{e,f}) = u_{f,e}^*.\] 
\end{Theorem} 

\begin{proof} We follow the proof of \cite[Theorem 5.5]{BDV1}.

First of all, it is straightforward to verify that $\mathscr{A}(\Gamma)$ indeed has a unique coproduct, counit and antipode for which it becomes a partial Hopf $^*$-algebra, and such that on the generators the formulas above for them are satisfied.% More detail?

On the other hand, denote by $\widetilde{\mathscr{A}}(\Gamma)$ the partial Hopf $^*$-algebra constructed from $F_{\Gamma}$. Let $\widetilde{U}$ be the corepresentation corresponding to $F_{\Gamma}(\pi_{1/2})$. Then the matrix entries $\widetilde{u}_{e,f}$ of $\widetilde{U}$ with respect to  the canonical basis of $\Hsp_{\Gamma} = l^2(\Gamma^{(1)})$ satisfy the relations \eqref{EqUni1} and \eqref{EqUni2} because of unitarity of $\widetilde{U}$. On the other hand, the relation \eqref{EqInt} follows immediately from the fact that $R_{\Gamma}$ is an intertwiner. As $\pi_{1/2}$ is a generating representation for $\Rep(SU_q(2))$, it follows that we have a surjective $^*$-representation $\mathscr{A}(\Gamma)\rightarrow \widetilde{\mathscr{A}}(\Gamma)$.

Let now $P_{n,m}$ for $n\in \N,m\in \frac{1}{2}\N$ with $2m = n \mod 2$ and $2m\leq n$ be the (Jones-Wenzl) projection in $\Mor(\pi_{1/2}^{n},\pi_{1/2}^{n})$ which projects onto the spin $m$-representation $\pi_m$. As $U$ is a (real or anti-real) selfconjugate corepresentation of $\mathscr{A}$, it follows that $P_{n,m}$ is an endomorphism of $U^{(n)} := U^{\Circt n}$. Hence we obtain a well-defined linear map \[\theta: \widetilde{\mathscr{A}}(\Gamma)\rightarrow \mathscr{A}(\Gamma)\] such that, with $\widetilde{U}_m$ the corepresentation of $\widetilde{\mathscr{A}}(\Gamma)$ associated to $F(\pi_m)$,  \[ (\theta\otimes \id)\widetilde{U}_m = (\id\otimes P_{n,m})U^{(n)}.\]
As in \cite{BDV1}, we then infer that \[(\theta\otimes\id)(\widetilde{U}_m)_{12}(\theta\otimes\id)(\widetilde{U}_{m'})_{13} = (\theta\otimes\id)((\widetilde{U}_m)_{12}(\widetilde{U}_{m'})_{13})\] inside $\End(\Hsp_{\Gamma}^{\iitimes n})\otimes \mathscr{A}(\Gamma)$, so that $\theta$ is multiplicative and hence an inverse to the quotient map  $\mathscr{A}(\Gamma)\rightarrow \widetilde{\mathscr{A}}(\Gamma)$.
\end{proof}



%%% Local Variables: 
%%% mode: latex
%%% TeX-master: "dyn-suq-main"
%%% End: 

\section{Further structure on the dynamical $SU_{q}(2)$}


\subsection{Representation theory of the function algebra on the dynamical quantum $SU(2)$ group}


% Study spectrum fundamental character
% Study dual quantum groupoid
% Make connection with dynamical cocycle
% In case of qgroupoid constructed from identity functor for Rep(SU_q(2)): rep theory of associated Galois object should just be: a single representation (Galois object is type I factor, cutdown of $B(\mathscr{L}^2(SU_q(2)))$). Yes: in general, Galois object is Morita equivalent with algebra of original ergodic action, should also be stressed for Podles spheres

\begin{Lem} There are faithful $^*$-representations $\pi_{\pm}$ of $\Pol_{\ext}(\X)$ as operators $\mathscr{D}^{\pm}\rightarrow \mathscr{D}^{\pm}$, given by the following formulas (where we suppress the explicit notations $\pi_{\pm}$): \begin{align*} \alpha\cdot e_{n,y}^+ = \left(\frac{1+q^{2n-2y}}{1+q^{-2y-2}}\right)^{1/2}e_{n,y+1}^+,&& \beta\cdot e_{n,y}^+ = \left(\frac{q^{-2y}-q^{2n-2y+2}}{1+q^{-2y-2}}\right)^{1/2}e_{n+1,y+1}^+,\end{align*}
\begin{align*} \alpha\cdot e_{n,y}^- = \left(\frac{1-q^{2n}}{1+q^{-2y-2}}\right)^{1/2}e_{n-1,y+1}^-,&& \beta\cdot e_{n,y}^- = \left(\frac{q^{2n+2}+q^{-2y}}{1+q^{-2y-2}}\right)^{1/2}e_{n,y+1}^-,\end{align*} the functions in $C_c(\R)$ simply acting by $fe_{n,y}^{\pm}= f(y)e_{n,y}^{\pm}$.

Both representations are bounded when restricted to $\Pol(\X)$.
\end{Lem}



%%% Local Variables: 
%%% mode: latex
%%% TeX-master: "dyn-suq-main"
%%% End: 


\begin{thebibliography}{99}
\bibitem{AN1} N. Andruskiewitsch and S. Natale, Double categories and quantum groupoids, \emph{Publ. Mat. Urug.} \textbf{10}, 11--51 (2005).
\bibitem{BDV1} J. Bichon, A. De Rijdt and S. Vaes, Ergodic coactions with large multiplicity and monoidal equivalence of quantum groups, \emph{Comm. Math. Phys.} \textbf{262} (2006), 703--728.
\bibitem{Boh1}  G. B\"{o}hm, J. Gómez-Torrecillas and E. López-Centella, Weak multiplier bialgebras, \emph{Trans. Amer. Math. Soc.}, in press., arXiv:1306.1466. 
\bibitem{Dau1} J. Dauns, Multiplier rings and primitive ideals, \emph{Trans. Amer. Math. Soc} \textbf{145} (1969), 124--158.
\bibitem{DCY1} K. De Commer and M. Yamashita, Tannaka-Kre\u{\i}n duality for compact quantum homogeneous spaces II. Classification of quantum homogeneous spaces for quantum $SU(2)$, J. Reine Angew. Math., DOI: 10.1515/crelle-2013-0074 (2013).
\bibitem{Eti1} P. Etingof and V. Ostrik, Module categories over representations of $\SSL_q(2)$ and graphs, \emph{Math. Res. Lett.} \textbf{11} (1) (2004), 103--114.
\bibitem{Hay1} T. Hayashi, Compact Quantum Groups of Face Type, \emph{PRIMS} \textbf{32} (1996), 351--369.
\bibitem{KoR1}  E. Koelink and H. Rosengren, Harmonic Analysis on the $SU(2)$ Dynamical Quantum Group, \emph{Acta Applicandae Mathematica} \textbf{69} (2) (2001), 163--220.
\bibitem{Pin2} C. Pinzari, The representation category of the Woronowicz quantum group $S_{\mu}U(d)$ as a braided tensor C$^*$-category, \emph{Int. J. Math.} \textbf{18} (2) (2007), 113--136.
\bibitem{Pin3} C. Pinzari and J.E. Roberts, Ergodic actions of compact quantum groups from solutions of the conjugate equations, preprint (2008) {\tt arXiv:0808.3326 [math.OA]}.
\bibitem{Tur1} V. Turaev, Quantum invariants of knots and 3-manifolds, \emph{de Gruyter Studies in Mathematics} \textbf{18}, Walter de Gruyter \& Co., Berlin (1994).
\bibitem{VDae1} A. Van Daele, Multiplier Hopf algebras, \emph{Trans. Amer. Math. Soc.} \textbf{342} (1994), 917--932.
\bibitem{VDae2} A. Van Daele, An algebraic framework for group duality, \emph{Adv. in Math.} \textbf{140} (1998), 323--366.
\bibitem{VDW2} A. Van Daele and  S. Wang, Weak multiplier Hopf algebras. Preliminaries, motivation and basic examples, \emph{Banach Center Publ.} \textbf{98} (2012), 367--415.
\bibitem{VDW1} A. Van Daele and S. Wang: Weak multiplier Hopf algebras I. The main theory, \emph{Preprint University of Leuven and Southeast University of Nanjing} (2012), to appear in
\emph{Crelles Journal}, {\tt arXiv:math/1210.4395 [math.RA]}.
\bibitem{Yam1} S. Yamagami, A categorical and diagrammatical approach to Temperley--Lieb algebras, preprint (2004) {\tt arXiv:math/0405267 [math.QA]}.
\bibitem{Wor1} S.L. Woronowicz, Twisted $\mathrm{SU}(2)$ group. An example of a non-commutative differential calculus, \emph{Publ. Res. Inst. Math. Sci.} \textbf{23} (1) (1987), 117--181.
\end{thebibliography}

\end{document}

Scraps of section 2 for further ref.

\section{Structure theory for compact quantum groups of face type}


\begin{Def} A \emph{(right) comodule} for a generalized face bialgebra $(A,\Delta)$ over $I$ is an $I^2$-graded vector space $V = \osum{K\in I^2} V(K)$ together with linear maps \[\delta\Grru{K}{L}:V(L)\rightarrow V(K)\otimes A\Grru{K}{L}\] satisfying \[(\id\otimes \Delta\Grru{K}{L})\delta(K*L) = (\delta(K)\otimes \id)\delta(L)\] and \[(\id\otimes \varepsilon_K)\delta\Grru{K}{K} = \id_{V(K)}.\]

We say $(V,\delta)$ is \emph{of finite type} (or simply \emph{finite}) if the support of $M\mapsto V(M)$ is finite in one variable if the other variable is held fixed.% Call this separately finite support? 
\end{Def}

For example, each $\oplus_{L} A\Grru{K}{L}$ is a right comodule under $\Delta$. We will write \[\delta(M)(v) = v_{(0)M_u}\otimes v_{(1)M},\] to be interpreted as zero if $v\notin V(M_u)$.

We have a tensor product $\boxtimes$ on finite comodules by putting \[(V\boxtimes W)(M) = \underset{K\cdot L = M}{\oplus} (V(K)\otimes V(L))\] with comodule structure \[\delta(M)(v\otimes w) = \underset{K\cdot L = M}{\sum} v_{(0)K_u}\otimes w_{(0)L_u} \otimes v_{(1)K}w_{(1)L}.\] We then obtain a tenor category with unit the vector space $\mathbf{1} = \Fun_{\fin}(I)$ of finite support functions on $I$ with grading $\mathbf{1}(k,l) = \delta_{k,l} \C \delta_k$ (with $\delta_k$ the Dirac function at $k$) and comodule structure \[\delta(K,K)(\delta_{K_d}) = \delta_{K_u}\otimes e(K).\]

When $(A,\Delta)$ is a Hopf face algebra, this tensor category admits left % or right - sort out
 duals. Indeed, define $(V^*)(M) = V(M^{\circ})^*$ with coaction \[(\delta(M)(\omega))(v) = \omega(v_{(0)M_d^{\circ}})S(v_{(1)M^{\circ\bullet}}),\qquad \omega\in V(M_d^{\circ})^*,v\in V(M_u^{\circ}).\] Then the natural dualities between the $V(K)$ and $V^*(K^{\circ})$ lead to comodule maps \[\oplus_{K} V^*(K^{\circ})\otimes V(K) \rightarrow \mathbf{1}_K,\qquad \mathbf{1}_K\rightarrow \oplus_K V(K)\otimes V^*(K^{\circ}).\]

%\subsection{Invariant functionals}

Let us now turn to the notion of invariant functional.

\begin{Def} Let $I$ be a set. An \emph{invariant functional} for a Hopf face algebra $(A,\Delta)$ over $I$ is a functional $\varphi:A \rightarrow \C$ such that for all $K,L$ and $a\in A(K*L)$ we have \[(\id\otimes \varphi)\Delta\Grru{K}{L}(a) = \delta_{K_l,K_r}\varphi(a)e(K_l),\qquad (\varphi\otimes \id)\Delta\Grru{K}{L}(a) = \delta_{L_l,L_r} \varphi(a)e(L_r).\] We say that $\varphi$ is normalized if $\varphi(\lambda_k\rho_l)=1$ for all $k,l\in I$ with $\lambda_k\rho_l\neq 0$.
\end{Def}

%\begin{Lem} We have $\varphi(\Gr{A}{k}{l}{m}{n}) = \delta_{k,l}\delta_{m,n}\C$.\end{Lem}

% TODO: make connection with weak multiplier Hopf algebras

\begin{Lem} An invariant normalized functional $\varphi$ is faithful, i.e. $\varphi(ab)=0$ for all $b$ implies $b=0$, and $\varphi(ab)=0$ for all $a$ implies $b=0$.
\end{Lem}

\begin{proof} We follow ad verbatim the proof of Proposition 3.4 in [VDae, Algebraic framework]: if $\varphi(ba)=0$ for all $a$, we arrive at the conclusion that for all $d\in A$ and all functionals $\omega$ on $A$, the element $p = (\omega\otimes \id)((d\otimes 1)\wDelta(a))$ satisfies $(\id\otimes \varphi)((1\otimes c)\wDelta(p)) = 0$. Continuing as in that proof, we obtain from the antipode trick that $\sum_n \varphi(cS(q)\rho_n)\varepsilon(p\lambda_n)=0$. Choosing now for $c$ and $q$ local units of the form $\lambda_k\rho_l$, the normalization condition on $\varphi$ gives that $\varepsilon(p\lambda_n)=0$ for all $n$, hence $\varepsilon(p)=0$. This implies $\omega(da)=0$. As $\omega$ and $d$ were arbitrary, it follows that $a=0$.

The other case follows similarly, considering the opposite algebra.
\end{proof}

% This part is necessary to put into Enock framework, but might possibly be skipped if we only want an operator algebra implementation of dynamical quantum SU(2), as then one has enough with the multiplicative unitaries
Our next aim is to prove that a normalized invariant functional is modular, that is, there exists an automorphism $\sigma: A\rightarrow A$ such that for all $a,b\in A$, we have \[\varphi(ba) = \varphi(a\sigma(b)).\]

\begin{Lem} Let $\varphi$ be a normalized invariant functional. For all $a\in A$ and $k,m\in I$, we have \[\varphi(a\lambda_k) = \varphi(\lambda_ka),\qquad \varphi(a\rho_m) = \varphi(\rho_ma).\]
\end{Lem}

\begin{proof}

\end{proof}



\begin{Def} Define \[V: \osum{n} A_n\otimes {}_{n}A \rightarrow \osum{r} {}_rA\otimes {}^rA\] by the formula \[a\otimes b\rightarrow \wDelta(a)(1\otimes b).\]
\end{Def}

\begin{Lem}\label{LemUni} The map $V$ is an isomorphism.
\end{Lem}
\begin{proof} As $\wDelta(A)(A\otimes A)\subseteq E(A\otimes A)$ with $E = \sum_p \rho_p\otimes \lambda_p$, it is clear that $V$ has the proper range. Define \[\widetilde{V}:  \osum{r} {}_rA\otimes {}^rA\rightarrow\osum{n} A_n\otimes {}_{n}A\] by means of the formula \[a\otimes b \mapsto a_{(1)}\otimes S(a_{(2)})b = a_{(1)}\otimes S(S^{-1}(b)a_{(2)}).\]

By the defining property of $S$, we find that for all $a,b,c\in A$, we have \[(c\otimes 1)\cdot (\widetilde{V}V)(a\otimes b) = \sum_p ca_{(1)}\varepsilon(a_{(2)}\lambda_p)\otimes \rho_pb.\] By the previous lemma, this equals $\sum_p ca_{(1)}\varepsilon(a_{(2)}\rho_p)\otimes \rho_pb$. But as $a\otimes b = \sum_p a\rho_p\otimes \rho_pb$ by assumption, we obtain that \[(c\otimes 1)\cdot (\widetilde{V}V)(a\otimes b) = ca_{(1)}\varepsilon(a_{(2)})\otimes b = ca\otimes b,\] proving that $\widetilde{V}V(a\otimes b) = a\otimes b$.

The identity $V\widetilde{V} = \id$ is proven similarly.
\end{proof}

\begin{Cor} Define \[W: \osum{n} {}_n A \otimes {}^nA  \rightarrow \osum{r} \; {}^rA\otimes A^r\] by the formula \[a\otimes b \rightarrow S^{-1}(b_{(1)})a\otimes d_{(2)} = S^{-1}(S(a)b_{(1)})\otimes b_{(2)}.\] Then $W$ is invertible, its inverse being given as \[W^{-1}(a\otimes b) = \wDelta(b)(a\otimes 1).\]
\end{Cor}

\begin{proof} Apply the previous Lemma to $(A,\Delta^{\op})$.
\end{proof}



\subsection{Generalized compact Hopf face algebras}


A non-degenerate algebra $A$ is called a $^*$-algebra if it comes equipped with an anti-linear involutive anti-homomorphism $A\rightarrow A, a\mapsto a^*$. In this case, $M(A)$ becomes a $^*$-algebra in a natural way. For example, we always consider $\Fun_{\fin}(I)$ as a $^*$-algebra by the ordinary complex conjugation of functions, $f^*(k) = \overline{f(k)}$.




\begin{Def} A couple $(A,\Delta)$ consisting of a generalized Hopf face $^*$-algebra with an invertible antipode invariant normalized functional $\varphi$ is called a \emph{generalized compact face algebra}.
\end{Def}

One proves that a generalized Hopf face $^*$-algebra has $S(S(x)^*)^*=x$ for all $x$, so $S$ is automatically invertible. It then follows by symmetry that also the maps \[(W^{k,t}_{m,n,u,v})^*: \oplus_l \Gr{A}{k}{l}{m}{n} \otimes \Gr{A}{l}{t}{u}{v} \rightarrow \oplus_r \Gr{A}{k}{t}{m}{r}\otimes \Gr{A}{n}{r}{u}{v}\] defined by the formula \[a\otimes b\rightarrow \Delta(b)(a\otimes 1)\] are unitaries, with inverse map $a\otimes b\mapsto S^{-1}(b_{(1)})a\otimes b_{(2)}$.

\begin{Lem}\label{LemUni} Let $(A,\Delta)$ be a generalized compact face algebra. Then each $V^{k,l,s,t}_{m,v}$ is a unitary, and similarly for the $W^{k,t}_{m,n,u,v}$.
\end{Lem}

\begin{proof} It is immediately checked that $V^{k,l,s,t}_{m,v}$ is isometric.
\end{proof}

Let us write $\mathscr{L}^2(A,\varphi)$ for the completion of $A$ with respect to the inner product $\langle a,b\rangle = \varphi(a^*b)$. The canonical inclusion of $A$ into $\mathscr{L}^2(A)$ will be denoted $\Lambda$.

\begin{Lem} Assume $(A,\Delta)$ is a generalized compact face algebra. The representation of $A$ by left multiplication on itself extends to a representation by bounded operators on the completion $\mathscr{L}^2(A,\varphi)$.
\end{Lem}

\begin{proof} Denote $\omega_{\xi,\eta}(x) = \langle \xi,x\eta\rangle$ for $\xi,\eta$ vectors and $x$ a bounded operator. Then a straightforward computation shows that \[(\omega_{\Lambda(a),\Lambda(b)}\otimes \id)(V) = \varphi(a^*b_{(1)})b_{(2)}\] as a left multiplication operator. As $(A\otimes 1)\Delta(A) = (A\otimes A)\Delta(1)$ by Lemma \ref{LemUni} (applied to the opposite algebra), it follows by normalization of $\varphi$ that each element of $A$ can be represented in the form $(\omega_{\Lambda(a),\Lambda(b)}\otimes \id)(V)$, and hence extends to a bounded operator on $\mathscr{L}^2(A,\varphi)$.
\end{proof}

In the following, we will abbreviate $\mathscr{L}^2(A)$ by $L^2A$.

Let $(A,\Delta)$ be a generalized compact face algebra. Denote the von Neumann algebraic completion of $A\subseteq B(L^2A)$ by $M$. Denote $L^2A\iitimes L^2A = E(L^2A\otimes L^2A)$, where $E = \sum_p \rho_p\otimes \lambda_p$ is extended to a bounded operator (in fact, a self-adjoint projection). Finally, denote $M\itimes M = E(M\otimes M)E$. Then $M\itimes M$ is the von Neumann algebraic completion of $A\itimes A$.

Extend now the $V^{k,l,s,t}_{m,v}$ to unitaries \[V: \oplus_p L^2(A_p)\otimes L^2({}_pA) \rightarrow \oplus_p L^2({}_pA)\otimes L^2({}^pA)= E(L^2A\otimes L^2A).\] Then we can construct a map \[\Delta: M\rightarrow M\itimes M,\quad x\rightarrow V(x\otimes 1)V^*.\] By direct computation, we see that $\Delta$ extends the comultiplication map on $A$. It is then immediate to check that $\Delta$ is in fact coassociative (where one may as well consider $\Delta$ as a non-unital map from $M$ to $M\otimes M$).

We aim to show that $(M,\Delta)$ can be fitted into the theory of measured quantum groupoids.

%%%%%%%%%%%%%%%%%%%%%%%%%%%%%%%%%%%%

Remnants of section 6


\subsection{Representation theory of the intertwiner function algebra on the dynamical quantum $SU(2)$ group (to be modified)}

\begin{Lem} There are faithful $^*$-representations $\pi_{\pm}$ of $\Pol_{\ext}(\X)$ as operators $\mathscr{D}^{\pm}\rightarrow \mathscr{D}^{\pm}$, given by the following formulas (where we suppress the explicit notations $\pi_{\pm}$): \begin{align*} \alpha\cdot e_{n,y}^+ = \left(\frac{1+q^{2n-2y}}{1+q^{-2y-2}}\right)^{1/2}e_{n,y+1}^+,&& \beta\cdot e_{n,y}^+ = \left(\frac{q^{-2y}-q^{2n-2y+2}}{1+q^{-2y-2}}\right)^{1/2}e_{n+1,y+1}^+,\end{align*}
\begin{align*} \alpha\cdot e_{n,y}^- = \left(\frac{1-q^{2n}}{1+q^{-2y-2}}\right)^{1/2}e_{n-1,y+1}^-,&& \beta\cdot e_{n,y}^- = \left(\frac{q^{2n+2}+q^{-2y}}{1+q^{-2y-2}}\right)^{1/2}e_{n,y+1}^-,\end{align*} the functions in $C_c(\R)$ simply acting by $fe_{n,y}^{\pm}= f(y)e_{n,y}^{\pm}$.

Both representations are bounded when restricted to $\Pol(\X)$.
\end{Lem}

%Miyashita Ulbrich?


%\section*{Representation theory of the dynamical quantum $SU(2)$ group}

%We let $A$ be the universal $^*$-algebra generated by the elements of a unitary $U = \begin{pmatrix} a & b \\ c & d\end{pmatrix}$ and a copy of $l^{\infty}(\Z\times \Z)$ such that \begin{eqnarray*} d^* &=& \frac{Z(\lambda)}{Z(\rho)}a ,\\ c^* &=& -\sgn(q)  Z(\lambda)Z(\rho-1),\end{eqnarray*} where $Z(k) = \left(\frac{|q|^{x+k}+|q|^{-x-k}}{|q|^{x+k+1}+|q|^{-x-k-1}}\right)^{1/2}$. We look for irreducible representations of $A$ such that the restriction to $c_c(\Z\times \Z)$ is non-degenerate. 
%We write $\beta = qF(\rho)^{1/2}u^{-,+}$, where $F(k) = |q|^{-1}Z(k)^{-2}$.

%Let \[\Omega = |q|^{\lambda-\rho+1}+|q|^{\rho-\lambda-1}-(|q|^{x+\lambda+1}+|q|^{-x-\lambda-1})(|q|^{x+\rho}+|q|^{-x-\rho})b^*b\] which we consider as a formal element. Then $\Omega$ is central. Since in any non-degenerate representation $\pi$ of $A$ the Hilbert space $\Hsp$ decomposes into direct summands $\Hsp_{k,l}$, the action of $\Omega$ on the algebraic direct sum of all $\Hsp_{k,l}$ is meaningful. If then $\pi$ is irreducible, it is clear that $\pi(\Omega)$ must be a scalar.


\subsection{Representation theory of the function algebra on the dynamical quantum $SU(2)$ group}


% Study spectrum fundamental character
% Study dual quantum groupoid
% Make connection with dynamical cocycle
% In case of qgroupoid constructed from identity functor for Rep(SU_q(2)): rep theory of associated Galois object should just be: a single representation (Galois object is type I factor, cutdown of $B(\mathscr{L}^2(SU_q(2)))$). Yes: in general, Galois object is Morita equivalent with algebra of original ergodic action, should also be stressed for Podles spheres

\begin{Lem} There are faithful $^*$-representations $\pi_{\pm}$ of $\Pol_{\ext}(\X)$ as operators $\mathscr{D}^{\pm}\rightarrow \mathscr{D}^{\pm}$, given by the following formulas (where we suppress the explicit notations $\pi_{\pm}$): \begin{align*} \alpha\cdot e_{n,y}^+ = \left(\frac{1+q^{2n-2y}}{1+q^{-2y-2}}\right)^{1/2}e_{n,y+1}^+,&& \beta\cdot e_{n,y}^+ = \left(\frac{q^{-2y}-q^{2n-2y+2}}{1+q^{-2y-2}}\right)^{1/2}e_{n+1,y+1}^+,\end{align*}
\begin{align*} \alpha\cdot e_{n,y}^- = \left(\frac{1-q^{2n}}{1+q^{-2y-2}}\right)^{1/2}e_{n-1,y+1}^-,&& \beta\cdot e_{n,y}^- = \left(\frac{q^{2n+2}+q^{-2y}}{1+q^{-2y-2}}\right)^{1/2}e_{n,y+1}^-,\end{align*} the functions in $C_c(\R)$ simply acting by $fe_{n,y}^{\pm}= f(y)e_{n,y}^{\pm}$.

Both representations are bounded when restricted to $\Pol(\X)$.
\end{Lem}

