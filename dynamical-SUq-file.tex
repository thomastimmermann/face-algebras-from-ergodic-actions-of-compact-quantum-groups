% Remarks:
% Podles sphere can be defined as an algebra in $\Rep(SU_q(2))$. Hence, it should make sense as an algebra under the forgetful functor to SU_q(2)-dynamical. By duality for coideals, the same should hold for passage to SU_q(1,1) in fact...
%
% link with article `Racah - Wigner quantum 6j Symbols, Ocneanu Cells for AN diagrams and quantum groupoids' by Coquereaux?
% Ref to Schauenburg that face algebras are weak Hopf algebras
% Thanks to Makoto, Piotr (?), Leonid
% Cf. Enocks quantum groupoids of compact type
% Donin, J.(IL-BILN); Mudrov, A.(IL-BILN), Quantum groupoids and dynamical categories. 
% Leonid, cf. talk www.fields.utoronto.ca/programs/scientific/13-14/.../Vainerman.pdf
% Stokman vertex irf babelon cocycle twist
% Level 2 Hecke algebras Brundan
% Concerning locally unital algebras ask J. Vercruysse on recent work 
% Say construction groupoids from forgetful functors on temperley-Lieb already in Enock-Ostrik remark
% Say partial compact quantum groups are proper locally compact quantum groupoids with a discrete base set. Can this be made precise? Is e.g. every measured quantum groupoid which has commutative discrete base and which is `proper' (finiteness integrals on components) of this type? Also compare with properness Timmermann.
% Horizontally categorified algebra is algebroid, but that terminology would be too overused if applied to quantum groupoids
% Make comparison with Hayashi's $\mathfrak{G}(A_1;t)$ in his introduction to face algebras paper - we should be able to prove the last comment he makes saying that his face algebra is the canonical one coming from quantum $SU(2)$ at root of unity.
% Stokman's Vertex-IRF transformations, dynamical quantum groups and harmonic analysis
% ref to `Finite quantum groupoids and their applications.' for Temperley-Lieb at root of unity
% Connection with amenability $\lambda$-lattices?

\documentclass[11pt]{article}

\usepackage{hyperref}
\usepackage{fixme}
\usepackage{mathrsfs}
\usepackage[a4paper]{geometry}
\usepackage{amssymb, amsthm, amsfonts, amsxtra, amsmath}
\usepackage{latexsym}
\usepackage{mathabx}
\usepackage{enumitem}
%\usepackage[all]{xy}
%\usepackage{graphics}
\usepackage{pdfpages}
\usepackage{epic}
\usepackage{fouridx}
\usepackage{parskip} % paragraphs have no indents and vertical spacings inbetween
\makeatletter % need this to avoid the conflict between amsthm and parskip
\def\thm@space@setup{%
  \thm@preskip=\parskip \thm@postskip=0pt
}
\makeatother

%\theoremstyle{change}

\newcommand{\dual}[1]{#1^{\vee}}
\newcommand{\predual}[1]{{^{\vee}\!#1}}
\newcommand{\co}{\mathrm{co}}
\newcommand{\Corep}{\mathrm{Corep}}
\newcommand{\Corepf}{\mathrm{Corep}^{f}}
\newcommand{\sff}{\textrm{s.f.~}}
\newcommand{\sfs}{\mathrm{sfs}}
\newcommand{\sfd}{\mathrm{sfd}}
\DeclareMathOperator{\Hom}{Hom}
\DeclareMathOperator{\img}{img}

\DeclareMathOperator{\id}{id}
\DeclareMathOperator{\ext}{\mathrm{e}}
\DeclareMathOperator{\can}{\mathrm{can}}
\DeclareMathOperator{\ctau}{\tau}
\DeclareMathOperator{\op}{\mathrm{op}}
\DeclareMathOperator{\fin}{\mathrm{f}}
\DeclareMathOperator{\Pol}{\mathrm{P}}
\DeclareMathOperator{\End}{\mathrm{End}}
\DeclareMathOperator{\Par}{\mathrm{Par}}
\DeclareMathOperator{\reg}{\mathrm{reg}}
\DeclareMathOperator{\sgn}{\mathrm{sgn}}
\DeclareMathOperator{\Zz}{\mathrm{Z}}
\DeclareMathOperator{\Ran}{\mathrm{Ran}}
\DeclareMathOperator{\hol}{\mathrm{hol}}
\DeclareMathOperator{\Ind}{\mathrm{Ind}}
\DeclareMathOperator{\Ker}{\mathrm{Ker}}
\DeclareMathOperator{\Char}{\mathrm{Char}}
\DeclareMathOperator{\dyn}{\mathrm{dyn}}
\DeclareMathOperator{\Spec}{\mathrm{Spec}}
\DeclareMathOperator{\adj}{\mathrm{adj}}
\DeclareMathOperator{\rcfd}{\mathrm{rcfd}}
\DeclareMathOperator{\rcf}{\mathrm{rcfd}}
\DeclareMathOperator{\stau}{\tau_{\mathrm{s}}}
\DeclareMathOperator{\tA}{\tilde{A}}
\DeclareMathOperator{\weps}{\tilde{\epsilon}}

\newcommand{\Circt}{{\mathop{\ooalign{$\ovoid$\cr\hidewidth\raise-.05ex\hbox{$\scriptstyle\mathsf T\mkern3.5mu$}\cr}}}} % Woronowicz style tensor product, USUAL SIZE
\newcommand{\Circtv}[1]{\underset{#1}{\mathop{\ooalign{$\ovoid$\cr\hidewidth\raise-.05ex\hbox{$\scriptstyle\mathsf T\mkern3.5mu$}\cr}}}} % Woronowicz style tensor product, USUAL SIZE
\newcommand{\smCirct}{\mathop{\ooalign{$\scriptstyle\ovoid$\cr\hidewidth\raise-.05ex\hbox{$\scriptscriptstyle\mathsf T\mkern2.8mu$}\cr}}}  % Woronowicz style tensor product, SCRIPT SIZE

\newcommand{\nc}{\R}
\newcommand{\g}{\mathfrak{g}}
\newcommand{\h}{\mathfrak{h}}

\newcommand{\kk}{\mathfrak{k}}
\newcommand{\ttt}{\mathfrak{t}}
\newcommand{\p}{\mathfrak{p}}
\newcommand{\n}{\mathfrak{n}}
\newcommand{\llll}{\mathfrak{l}}
\newcommand{\uu}{\mathfrak{u}}
\newcommand{\bb}{\mathfrak{b}}
\newcommand{\q}{\mathfrak{q}}
\newcommand{\su}{\mathfrak{su}}
\newcommand{\ssl}{\mathfrak{sl}}
\newcommand{\SSL}{\mathrm{SL}}
\newcommand{\so}{\mathfrak{so}}
\newcommand{\spp}{\mathfrak{sp}}
\newcommand{\G}{\mathbb{G}}
\newcommand{\e}{\mathfrak{e}}
\newcommand{\s}{\mathfrak{s}}
\newcommand{\C}{\mathbb{C}}
\newcommand{\R}{\mathbb{R}}
\newcommand{\Z}{\mathbb{Z}}
\newcommand{\N}{\mathbb{N}}
\newcommand{\X}{\mathbb{X}}
\newcommand{\Y}{\mathbb{Y}}
\newcommand{\Ss}{\mathbb{S}}
\newcommand{\ZZ}{\mathscr{Z}}
\newcommand{\ad}{\mathrm{ad}}
\newcommand{\Hsp}{\mathcal{H}}
\newcommand{\qn}[2]{\lbrack #1 \rbrack_{#2}}
\newcommand{\fqn}[2]{\lbrack #1 \rbrack_{#2}!}
\newcommand{\bqn}[3]{\left\lbrack \begin{array}{c} \!#1\! \\ \!#2\! \end{array}\right\rbrack_{#3}}
\newcommand{\Tr}{\mathrm{Tr}}
\newcommand{\RR}{\mathcal{R}}
\newcommand{\rd}{\mathrm{d}}
\newcommand{\res}{\mathrm{res}}
\newcommand{\cop}{\mathrm{cop}}
\newcommand{\opp}{\mathrm{op}}
\newcommand{\coop}{\mathrm{coop}}
\newcommand{\Rm}{\mathcal{R}}
\newcommand{\wt}{\mathrm{wt}}
\newcommand{\Ad}{\mathrm{Ad}}
\newcommand{\CatC}{\mathcal{C}}
\newcommand{\CatD}{\mathcal{D}}
\newcommand{\CatCC}{\mathscr{C}}
\newcommand{\CatDD}{\mathscr{D}}
\newcommand{\Corr}{\mathrm{Corr}}

\newcommand{\Vectf}{\mathrm{Vect}^{f}}
\newcommand{\Vecti}{\mathrm{Vect}_{I^{2}}}
\newcommand{\Vectif}{\mathrm{Vect}^{f}_{I^{2}}}
\newcommand{\Hilb}{\mathrm{Hilb}}
\newcommand{\Hilbf}{\mathrm{Hilb}^{\mathrm{f}}}
\newcommand{\Hilbi}{\mathrm{Hilb}_{I^{2}}}
\newcommand{\Hilbif}{\mathrm{Hilb}_{I^{2}}^{\mathrm{f}}}

\newcommand{\Star}[2]{{}_{#1}\!*_{#2}}
\newcommand{\vot}{\bar{\otimes}}
\newcommand{\A}{\mathcal{B}}
\newcommand{\Aa}{\mathscr{B}}
\newcommand{\Mor}{\mathrm{Mor}}
\newcommand{\alg}{\mathrm{alg}}
\newcommand{\Gg}{\mathscr{G}}
\newcommand{\ev}{\mathrm{ev}}
\newcommand{\Rtimes}{\underset{\R}{\times}}
\newcommand{\Rb}{\R^{\bullet}}
\newcommand{\vtimes}{\bar{\otimes}}
\newcommand{\Rr}{\mathscr{R}}
\newcommand{\Tt}{\mathscr{T}}
\newcommand{\Fun}{\mathrm{Fun}}
\newcommand{\Ff}{\Fun_{\fin}}
%\newcommand{\fin}{\mathrm{fin}}
%\newcommand{\iitimes}{\underset{I}{\otimes}}
\newcommand{\itimes}{\underset{I}{\otimes}}
\newcommand{\osum}[1]{\underset{#1}{\sum}^{\oplus}}
\newcommand{\osumc}[1]{\underset{#1}{\sum}^{\bar{\oplus}}}
\newcommand{\oplusc}{\bar{\oplus}}
\newcommand{\wDelta}{\widetilde{\Delta}}
\newcommand{\f}{\mathrm{fin}}
%\newcommand{\Hilb}{\mathrm{Hilb}}
\newcommand{\Rho}{\mathrm{P}}
\newcommand{\Rep}{\mathrm{Rep}}
\newcommand{\DA}{\mathcal{A}}
%\newcommand{\Circt}{\mathop{\ooalign{$\ovoid$\cr\hidewidth\raise-.05ex\hbox{$\scriptstyle\mathsf T\mkern3.5mu$}\cr}}} % Woronowicz style tensor product, USUAL SIZE
\newcommand{\even}{\mathrm{even}}
\newcommand{\odd}{\mathrm{odd}}
\newcommand{\fd}{\mathrm{fd}}
\newcommand{\Forget}{F}

\newcommand{\GrHA}[3]{#1{\begin{pmatrix} #2,  #3\end{pmatrix}}}% Horizontal grading ordinary style, with argument
\newcommand{\Grs}[3]{#1{\begin{pmatrix} #2,  #3\end{pmatrix}}}

\newcommand{\GrDA}[3]{\;{}_{\;#2}#1_{#3}} % Horizontal grading bottom style, with argument
%\newcommand{\Grd}[3]{\;{}_{\;#2}#1_{#3}}

\newcommand{\GrVA}[3]{#1{\tiny {\begin{pmatrix} #2\\#3\end{pmatrix}}}} % Vertical grading ordinary style, with argument
\newcommand{\Grt}[3]{#1{\tiny {\begin{pmatrix} #2\\#3\end{pmatrix}}}} 

\newcommand{\GrRA}[3]{#1^{#2}_{#3}} % Vertical grading right style, with argument

\newcommand{\GrLA}[3]{{}^{#2}_{#3}#1} % Vertical grading left style, with argument

\newcommand{\Unit}{\mathbf{1}}
\newcommand{\UnitC}[2]{\Grt{\mathbf{1}}{#1}{#2}} 
\newcommand{\Grru}[2]{{\tiny \begin{pmatrix} #1 \\ #2\end{pmatrix}}}

\newcommand{\eGr}[5]{#1{{\tiny \begin{pmatrix} #2 \quad #3 \\ #4 \quad #5\end{pmatrix}}}}

\newcommand{\pma}[4]{\begin{pmatrix} #1 \quad #2 \\ #3 \quad #4\end{pmatrix}}
\newcommand{\pmat}[4]{{\tiny \begin{pmatrix} #1 \quad #2 \\ #3 \quad #4\end{pmatrix}}}

\newcommand{\UT}[2]{#1{\tiny #2 }}
\newcommand{\Gr}[5]{\fourIdx{#2}{#4}{#3}{#5}{#1}}%TODO: better typesetting
\newcommand{\Grl}[3]{\Gr{#1}{#2}{}{#3}{}}%TODO: better typesetting
\newcommand{\Gru}[3]{\Gr{#1}{}{}{#2}{#3}}
\newcommand{\Grd}[3]{\Gr{#1}{}{}{#2}{#3}}
% \newcommand{\Gr}[5]{\;{}^{\;#2}_{#4}#1_{#5}^{#3}}%TODO: better typesetting
% %\newcommand{\Gr}[5]{\UT{#1}{\begin{pmatrix} #2\quad #3 \\ #4 \quad #5\end{pmatrix}}}
% %\newcommand{\Gr}[5]{\UT{#1}{\begin{pmatrix} \, #2\;\\ #3 \qquad #4 \\ \,#5\;\end{pmatrix}}}
% \newcommand{\Grl}[3]{\;{}^{\;#2}_{#3}#1}%TODO: better typesetting
% \newcommand{\Gru}[3]{{}^{\;#2}#1^{#3}}
% \newcommand{\Grd}[3]{{}_{\;#2}#1_{#3}}
\newcommand{\gr}[5]{\;{}^{\;#2}_{#4}#1_{#5}^{#3}}%TODO: better typesetting
\newcommand{\eGrr}[3]{#1_{{\tiny \left(#2, #3\right)}}}
\newcommand{\eGrt}[4]{#1{{\tiny \begin{pmatrix} #2 \\ #3 \\ #4 \end{pmatrix}}}}
\newcommand{\Grr}[4]{\begin{pmatrix}#1 \quad #2\\#3&#4\end{pmatrix}}

\newcommand{\Grss}[3]{\UT{#1}{\begin{pmatrix} #2 \; #3\end{pmatrix}}}
\newcommand{\Grb}[7]{\UT{#1}{\begin{pmatrix} #2\quad #3 \\ #4 \quad #5\\ #6 \quad #7\end{pmatrix}}}
\newcommand{\un}[2]{e{{\tiny \begin{pmatrix}#1\\ #2\end{pmatrix}}}}
\newcommand{\unn}[3]{e{{\tiny \begin{pmatrix}#1\\ #2\\#3\end{pmatrix}}}}

\newcommand{\wmult}{\cdot}
\newcommand{\bmult}{*}
\newcommand{\wmate}{\rightarrow}% Change this to source/target notation l(eft) r(ight)
\newcommand{\bmate}{\downarrow}% Change this to source/target notation u(p) d(own)

\newcommand{\aste}[1]{\underset{#1}{\ast}}

\newcommand{\Vv}{\mathcal{V}}

\newcommand{\dT}{\dot T}

\newtheorem{Theorem}{Theorem}[section]
\newtheorem{Lem}[Theorem]{Lemma}
\newtheorem{Prop}[Theorem]{Proposition}
\newtheorem{Cor}[Theorem]{Corollary}

\theoremstyle{definition}
\newtheorem{Def}[Theorem]{Definition}
\newtheorem{Rem}[Theorem]{Remark}
\newtheorem{Exa}[Theorem]{Example}
\newtheorem{Not}[Theorem]{Notation}
\newtheorem{Que}[Theorem]{Question}
\newtheorem{Con}[Theorem]{Conjecture}

%%%%%%%%%%%%%%%%%%%
% Further notation for Section 1
\newcommand{\phic}[2]{\Grt{\phi}{#1}{#2}}

%%%%%%%%%%%%%%%%%%%
% Notation for Section 4
\newcommand{\LGtwo}{L^{2}(\mathscr{G})}
\newcommand{\LGinf}{L^{\infty}(\mathscr{G})}
\newcommand{\CrG}{C^{r}_{0}(\mathscr{G})}
\newcommand{\CuG}{C^{u}_{0}(\mathscr{G})}
\newcommand{\vnDelta}{\overline{\Delta}}
\newcommand{\vnE}{\overline{E}}
\newcommand{\astrl}{\underset{l^{\infty}(I)}{_{\rho}\ast_{\lambda}}}
\newcommand{\otimesrl}{\underset{\nu{}}{_{\rho}\otimes_{\lambda}}}
\newcommand{\vnphi}{\overline{\phi}}
\newcommand{\vnphic}[2]{\Grt{\vnphi}{#1}{#2}}
\newcommand{\vnR}{\overline{R}}
\newcommand{\vntau}{\overline{\tau}}

% q-special functions macros

\newcommand{\qbin}[2]{\left[ \begin{array}{c} #1 \\ #2 \end{array}\right]_{q^2}}
\newcommand{\qortc}[4]{\,\;_1\varphi_1\left(\begin{array}{c} #1  \\#2 \end{array}\mid #3,#4\right)}
\newcommand{\qortPsi}[4]{\Psi\left(\begin{array}{c} #1  \\#2 \end{array}\mid #3,#4\right)}
\newcommand{\qorta}[5]{\,\;_2\varphi_1\left(\begin{array}{cc} #1 & #2 \\ & \!\!\!\!\!\!\!\!\!\!\!#3 \end{array}\mid #4,#5\right)}
\newcommand{\qortd}[6]{\,\;_2\varphi_2\left(\begin{array}{cc} #1 & #2 \\ #3 & #4 \end{array}\mid #5,#6\right)}
\newcommand{\qortb}[7]{\,\;_3\varphi_2\left(\begin{array}{ccc} #1 & #2 & #3 \\ & \!\!\!\!\!\!\!\!#4 & \!\!\!\!\!\!\!\!#5\end{array}\mid #6,#7\right)}

\date{}


\numberwithin{equation}{section}

\begin{document}
\title{Dynamical quantum $SU(2)$ groups on the operator algebra level}

\author{Kenny De Commer\thanks{Department of Mathematics, Vrije Universiteit Brussel, VUB, B-1050 Brussels, Belgium, email: {\tt kenny.de.commer@vub.ac.be}}
\and Thomas Timmermann\thanks{University of M\"{u}nster}}

\maketitle

\begin{abstract}
\noindent In a previous paper, we introduced the notion of partial compact quantum group, based on Hayashi's notion of compact face algebra. In this paper, we classify all partial compact quantum groups which the fusion rules of $SU(2)$. In particular, we obtain operator algebraic versions of the dynamical quantum $SU(2)$-group (as studied by Etingof-Varchenko and Koelink-Rosengren), and we make a detailed study of the resulting C$^*$-algebras. 
%We then show how any quantum homogeneous space of an ordinary compact quantum group leads to a partial compact quantum group. In particular, when this construction is applied to the non-standard Podle\'{s} spheres, we obtain partial compact quantum groups which are operator algebraic versions of % References ok?
\end{abstract}


%\emph{Keywords}:

%AMS 2010 \emph{Mathematics subject classification}:


%17B37: Quantum groups, quantized enveloping algebras
%20G42: quantized function algebras
%46L65: Functional analysis, deformations, quantizations
%81R50: Quantum groups and related algebraic methods
%16T05: Hopf algebras and their applications
%16T10: Bialgebras
%16T15: Coalgebras and comodules; corings
%46L08  $C^*$-modules


\tableofcontents

\section{Partial compact quantum groups on the level of operator algebras}


Let $\mathscr{G}$ be a partial compact quantum group. We construct
completions of the underlying $*$-algebra $P(\mathscr{G})$ in the form
of a universal $C^{*}$-algebra $\CuG$, a reduced $C^{*}$-algebra
$\CrG$ and a von Neumann algebra $\LGinf$. The existence of the first
one follows from the analogue of the Peter-Weyl theorem, Proposition \ref{prop:rep-weak-pw}, and the second and
third one arise from a GNS-representation of $P(\mathscr{G})$ on the
Hilbert space $\LGtwo$ associated to the invariant integral of
$\mathscr{G}$.  We then lift the comultiplication, the invariant
functional, the unitary antipode and the scaling group to level of
operator algebras and show that $\LGinf$ becomes a measured quantum
groupoid in the sense of Lesieur \cite{Les1} and Enock \cite{Eno2}.

Let us start with the construction of $\CuG$. Denote by $A$ the
underlying total $*$-algebra of the partial $*$-algebra
$P(\mathscr{G})$ and define a map $|\cdot |_{u} \colon A \to [0,\infty]$ by
\begin{align*} 
  |a|_{u}&:= \sup \{ \|\pi(a)\| : \pi \text{ is a $*$-homomorphism from } A
  \text{ into some $C^{*}$-algebra } B\}.
\end{align*}
\begin{Lem}
  $|a|_{u}<\infty$ for each $a \in A$. 
\end{Lem}
\begin{proof}
  By Corollary \ref{cor:rep-pw}, we can write each $a\in A$ in the form
  $a=(\id \otimes \omega_{\xi,\eta})(\Gr{X}{k}{l}{m}{n})$, where $\mathscr{X}$ is a unitary
  sfd corepresentation of $P(\mathscr{G})$ on some sfd $I^{2}$-graded
  Hilbert space $\mathcal{H}$ and
  $\xi\in \Gru{\mathcal{H}}{k}{l}$, $\eta\in \Gru{\mathcal{H}}{m}{n}$.
  Since $X$ is unitary,
  \begin{align*}
    \sum_{p}(\Gr{X}{p}{l}{m}{n})^{*} \Gr{X}{p}{l}{m}{n}  = \lambda_{l}\rho_{n}
    \otimes \id_{\Gru{\mathcal{H}}{m}{n}}
  \end{align*}
  by \ref{}, where the sum is finite because $X$ is sfd. As
  $\pi(\lambda_{k}\rho_{m})\in \C$ is a projection, we can conclude
  \begin{align} \label{eq:corep-estimate}
    \sum_{p} \| (\pi \otimes \id)(\Gr{X}{p}{l}{m}{n})\|^{2} \leq 1
  \end{align}
  and deduce $\|\pi(a)\| \leq \| \xi| \|\eta\| \|(\pi \otimes \id)(\Gr{X}{k}{l}{m}{n})\| \leq
    \|\xi\|\|\eta\| $. 
\end{proof}
Clearly, $|\cdot|_{u}$ defines a $C^{*}$-semi-norm on
$P(\mathscr{G})$, and the separated completion of $P(\mathscr{G})$
with respect to this norm is a $C^{*}$-algebra. We denote this
$C^{*}$-algebra by $\CuG$. By construction, every $*$-homorphism $\pi$
of $A$ into some $C^{*}$-algebra $C$ factorises through $\CuG$. 

Later we shall see that the canonical $*$-homomorphism $\pi_{u} \colon A \to
\CuG$ is injective.


The universal property of $\CuG$ implies, for example, that the modular automorphism
group $\sigma$ and the scaling group $\tau$ for real parameters and
the unitary antipode $R$ of $\mathscr{G}$ introduced after Corollary
\ref{cor:rep-characters} lift to one-parameter groups
$\tau^{u},\sigma^{u}$ and a $*$-anti-automorphism $R_{u}$ of $\CuG$,
that is, $\tau^{u}_{t} \circ \pi_{u}= \pi_{u} \circ \tau_{t}$,
$\sigma^{u}_{t} \circ \pi_{u} = \pi_{u} \circ \sigma_{t}$, $R_{u}
\circ \pi_{u} = \pi_{u} \circ R$. Corollary \ref{cor:rep-characters}
5.\ and A.1 in \cite{} [Takesaki:2] imply that elements in
$\pi_{u}(A)$ are analytic for $\tau^{u}$ and $\sigma^{u}$; in
particular, $\tau^{u}$ and $\sigma^{u}$ are strongly continuous.

The comultiplication lifts as well.  The proof involves an argument
that will be used later again.
 \begin{Lem} \label{LemBoundDim} Let $\mathscr{X}$ be a unitary rcfd representation of $\mathscr{G}$ on $\Hsp$ with finite hyperobject support, and write $d^X_{kl} = \dim(\Gru{\Hsp}{k}{l})$. Then the matrix $D^X = (d^X_{kl})$ defines a bounded operator on $l^2(I)$. 
  \end{Lem}
  \begin{proof} We may assume that $\mathscr{X}$ is irreducible, say
    with left hyperobject support $\alpha$ and right hyperobject support
    $\beta$. Then we can find (a priori not necessarily bounded)
    morphisms \[R: \C^{(I)}\rightarrow \underset{k,l,m}{\oplus} \GrDA{H}{k}{m}\otimes
    \GrDA{H}{m}{l},\qquad \bar{R}: \C^{(I)} \rightarrow \underset{k,l,m}{\oplus}
    \GrDA{\bar{H}}{k}{m}\otimes \GrDA{\bar{H}}{m}{l},\] establishing a
    duality between $\underset{k,l}{\oplus} \GrDA{H}{k}{l}$ and
    $\underset{k,l}{\oplus}\GrDA{\bar{H}}{k}{l}$ inside the category of
    rcfd pre-Hilbert spaces. However, as $R^*R$ is a scalar multiple
    of the projection onto $\C^{(I_{\alpha})}$ by irreducibility, and
    similarly for $\bar{R}$, it follows that $R$ and $\bar{R}$ can be
    completed to bounded operators between the respective Hilbert
    space completions. It then follows from \cite[Lemma A.3.2]{DCY1}
    that 
    \begin{align} \label{eq:dim-estimate}
  \sup_r (\sum_s (d_{rs}^X+d_{sr}^X)) < \infty.    
    \end{align}
 By the Schur test,
    this implies that $D^X$ is bounded.
\end{proof} 


\begin{Lem}
  There exists a unique $*$-homomorphism $\Delta^{u}\colon C^{u}_{0}(\mathscr{G})
  \to M(C^{u}_{0}(\mathscr{G}) \otimes C^{u}_{0}(\mathscr{G}))$ such that   for all $a \in P(\mathscr{G})$,
  \begin{align*}
    (\rho_{p} \otimes \lambda_{p})\Delta^{u}(a)(\rho_{q}\otimes \lambda_{q}) = \Delta_{pq}(a)
  \end{align*}
\end{Lem}
\begin{proof}
  We need to show that for every $a \in P(\mathscr{G})$, the sum
  $\sum_{p,q} \Delta_{pq}(a)$ converges in $M(C^{u}_{0}(\mathscr{G})
  \otimes C^{u}_{0}(\mathscr{G}))$ strictly.  It suffices to prove
  that the matrix of operators 
\[\left((\pi \otimes
    \pi)(\Delta_{pq}(a))\right)_{k,l}\] is bounded
 for some faithful $^*$-representation $\pi$ of
  $C^{u}_{0}(\mathscr{G})$, and using the Schur test as above, it
  suffices to show that
  \begin{align} \label{eq:delta-estimate}
    \sup_{p} \sum_{q} (\|(\pi \otimes \pi)(\Delta_{pq}(a))\| + \|(\pi \otimes \pi)(\Delta_{qp}(a))\|) < \infty.
  \end{align}
  As in the proof above, we write
  $a=(\id \otimes \omega_{\xi,\eta})(\Gr{X}{k}{l}{m}{n})$, where $\mathscr{X}$ is a unitary
  sfd corepresentation of $P(\mathscr{G})$ on some sfd $I^{2}$-graded
  Hilbert space $\mathcal{H}$ and
  $\xi\in \Gru{\mathcal{H}}{k}{l}$, $\eta\in \Gru{\mathcal{H}}{m}{n}$. Then
  \begin{align*}
    \Delta_{pq}(a) &= (\id \otimes \id \otimes \omega_{\xi,\eta})((\Delta_{pq} \otimes
    \id)(\Gr{X}{k}{l}{m}{n})) =
    (\id \otimes \id \otimes \omega_{\xi,\eta})((\Gr{X}{k}{l}{p}{q})_{13}(\Gr{X}{p}{q}{m}{n})_{23})
  \end{align*}
and hence, using \eqref{eq:corep-estimate}
\begin{align*}
  \|(\pi \otimes \pi)(\Delta_{pq}(a))\| \leq \|\xi\|\|\eta\| \|(\pi \otimes \id)(\Gr{X}{k}{l}{p}{q})\|
  \|(\pi \otimes \id)(\Gr{X}{p}{q}{m}{n})\|  \leq \|\xi\|\|\eta\|.
\end{align*}
Now \eqref{eq:delta-estimate} follows from \eqref{eq:dim-estimate}.
\end{proof}



We now turn to the construction of the reduced $C^{*}$-algebra $\CrG$
and the von Neumann algebra $\LGinf$. 
Denote by $\LGtwo$ the completion of $A$ with respect to the norm
associated to the inner product given by
\begin{align*}
  \langle a|b\rangle :=\phi(a^{*}b) \quad \text{for all } a,b\in A,
\end{align*}
and by $\Lambda \colon A \to \LGtwo$ the natural embedding.  This
product is definite because $\phi$ is faithful by \ref{LemFaith}, and
it extends to the space $\LGtwo$ which thus becomes a Hilbert space.
As such, $\LGtwo$ is the
orthogonal direct sum of the subspaces
$\Lambda(A(K)) \subseteq \LGtwo$, where $K\in M_{2}(I)$, because
$\phi(A(K)^{*}A(L)) = 0$ if $K\neq L$ by \ref{}.  In particular, there
exist  operators
$\lambda_{k},\lambda_{k}^{\op},\rho_{k},\rho_{k}^{\op}\in {\cal
  B}(\LGtwo)$ for each $k\in I$ such that
\begin{align*}
  \lambda_{k}\Lambda(a)&= \Lambda(\lambda_{k} a), &
  \lambda^{\op}_{k}\Lambda(a) &= \Lambda(a\lambda_{k}), &
  \rho_{k}\Lambda(a) &= \Lambda(\rho_{k}a), &
  \rho_{k}^{\op}\Lambda(a) &= \Lambda(a\rho_{k})
\end{align*}
for all $a\in A$, and faithful, normal $*$-homomorphisms
\begin{align} \label{eq:vn-lambda-rho}
  \lambda,\rho \colon l^{\infty}(I) \to
  {\cal B}(\LGtwo)
\end{align}
that send the delta function at
$k\in I$ to the operators $\lambda_{k}$ or $\rho_{k}$, respectively. 

 Define $\vnE,\overline{G} \in {\cal B}(\LGtwo \otimes
\LGtwo)$ by
\begin{align*}
  \vnE &:=\sum_{k} \rho_{k} \otimes \lambda_{k}, & 
  \overline{G} &:= \sum_{k} \rho_{k}^{\op} \otimes \rho_{k},
\end{align*}
where the sums converge with respect to the strong operator
topology.
\begin{Lem} \label{lemma:partial-isometry}
There exists a unique partial isometry $V$ on $\LGtwo \otimes \LGtwo$
such that
\begin{align*}
  V(\Lambda(a) \otimes \Lambda(b)) = \Lambda(a_{(1)}) \otimes \Lambda(a_{(2)}b)
\end{align*}
for all $a,b\in A$. Its range and domain projections are given by $VV^{*} = \vnE$
and $V^{*}V = \overline{G}$.
\end{Lem}
\begin{proof}
  Let $a,b \in A$. Since $\Delta$ is a $*$-homomorphism and $\phi$ is
invariant,
  \begin{align*}
    \langle \Lambda(a_{(1)}) \otimes
    \Lambda(a_{(2)}b)|\Lambda(a'_{(1)}) \otimes
    \Lambda(a'_{(2)}b')\rangle &=
    \phi(a_{(1)}^{*}a'_{(1)})\phi(b^{*}a_{(2)}^{*}a'_{(2)}b') \\
    &= \sum_{p}
    \phi(b^{*}\rho_{p}\phi(\rho_{p}a^{*}a'\rho_{p})\rho_{p}b') \\
    & =\sum_{p} \langle\Lambda(a\rho_{p}) \otimes \Lambda(\rho_{p}b) |
    \Lambda(a'\rho_{p}) \otimes \Lambda(b'\rho_{p})\rangle.
  \end{align*}
  Now, the assertion follows from Proposition \ref{prop:riti}.
\end{proof}

\begin{Prop} \label{prop:gns} Let $\mathscr{G}$ be a partial compact
  quantum group with underlying total $*$-algebra $A$ and associated
  Hilbert space $\LGtwo$. Then there exists a unique $*$-homomorphism
  $\pi_{r}\colon A \to {\cal B}(\LGtwo)$ such that
  $\pi_{r}(a)\Lambda(b)=\Lambda(ab)$ for all $a,b\in A$, and this
  $\pi_{r}$ is faithful.
\end{Prop}
\begin{proof} 
  Let $a,c \in A$. Then the formula $x \mapsto \langle
\Lambda(c) | x\Lambda(a)\rangle$ defines a bounded linear functional
  $\omega_{\Lambda(c),\Lambda(a)}$ on ${\cal B}(\LGtwo)$ and a
  straightforward computation shows that
  \begin{align} \label{eq:vn-slice}
    (\omega_{\Lambda(c),\Lambda(a)}\otimes \id)(V)\Lambda(b) =
    \Lambda(\varphi(c^*a_{(1)})a_{(2)}b)
  \end{align}
  for all $b\in A$. Therefore, left multiplication by
  $\varphi(c^*a_{(1)})a_{(2)}$ extends to a bounded linear operator on $\LGtwo$.
 Since $(A\otimes 1)\Delta(A) = (A\otimes
  A)\Delta(1)$ by Proposition \ref{prop:riti} and $\phi$ is
  normalized,  elements of the form $\phi(c^{*}a_{(1)})a_{(2)}$ span
  $A$. 
\end{proof}
\begin{Cor}
  Let $\mathscr{G}$ be a partial compact quantum group with underlying
  total algebra $A$. Then the
  canonical $*$-homomorphism $\pi_{u} \colon A \to\CuG$ is injective.
\end{Cor}
\begin{proof}
  The injective $*$-homomorphism $\pi_{r}$ factorises through
  $\pi_{u}$.
\end{proof}
Given a partial compact quantum group $\mathscr{G}$, we call
$(\LGtwo,\Lambda,\pi)$ the \emph{associated GNS-construction} and denote by
\begin{align}
  \CrG &\subseteq {\cal B}(\LGtwo) &&\text{and} & \LGinf &\subseteq {\cal B}(\LGtwo)
\end{align}
the $C^{*}$-algebra and the von Neumann algebra generated by $\pi_{r}(A)
\subseteq \LGtwo$, respectively, and identify $M(\CrG)$ with a
$C^{*}$-subalgebra of $\LGtwo$.  Since $\pi_{r}$ extends to a
$*$-homomorphism on $\CuG$, we get a sequence of $*$-homomorphisms
\begin{align*}
A \hookrightarrow \CuG \to 
  \CrG \hookrightarrow M(\CrG) \hookrightarrow
\LGinf \hookrightarrow {\cal B}(\LGtwo).
\end{align*}
Note that
 the $*$-homomorphisms $\lambda,\rho$ in
\eqref{eq:vn-lambda-rho} send $l^{\infty}(I)$ to $M(\CrG)$, and that
\begin{align*}
  \vnE \in M(\CrG \otimes \CrG) \subseteq \LGinf \otimes \LGinf
  \subseteq {\cal B}(\LGtwo \otimes \LGtwo),
\end{align*}
where $\otimes$ denotes the minimal tensor product
of $C^{*}$-algebras, the tensor product of von Neumann algebras, and
the tensor product of Hilbert spaces, respectively.

Consider the map 
\begin{align*}
  \vnDelta \colon \LGinf \to {\cal B}(\LGtwo \otimes \LGtwo), \ x
  \mapsto V(x \otimes 1)V^{*}.
\end{align*}
\begin{Lem} \label{lemma:vn-delta}
  \begin{enumerate}
  \item $\vnDelta(\pi_{r}(a)) (\Lambda(b) \otimes \Lambda(c)) =
    \Lambda(a_{(1)}b) \otimes \Lambda(a_{(2)}b)$ for all $a,b,c\in A$;
  \item $\vnDelta$ is a normal, faithful $*$-homomorphism;
  \item  $\vnDelta(\CrG) \subseteq \vnE M(\CrG \otimes
  \CrG)\vnE$ and $\vnDelta(\LGinf) \subseteq \vnE(\LGinf \otimes
  \LGinf)\vnE$.
  \end{enumerate}
\end{Lem}
\begin{proof}
  The equation in 1.{} is easily verified. The map $\vnDelta$ is
  normal by construction, a $*$-homo\-morphism by 1.{}, and faithful
  because $\vnDelta(x)=0$ implies $x\otimes 1=0$ on
  $V^{*}V(L^{2}(\mathscr{G}) \otimes L^{2}(\mathscr{G}))$ and hence
  $x=0$ on $\bigoplus_{k}
  \rho_{k}^{\op}L^{2}(\mathscr{G})=L^{2}(\mathscr{G})$. Finally, 3.\
  follows from the relation $\Delta(a)=E\Delta(a)E$, which holds for
  all $a\in A$.
\end{proof}


Next, we lift the invariant functional $\phi$ of $\mathscr{G}$ to
$\LGinf$ and define associated operator-valued weight
$T_{\lambda},T_{\rho}$ from $\LGinf$ to $l^{\infty}(I)$. Since $\phi$ is normalized, each
$\Lambda(\lambda_{k},\rho_{m})$ is a unit vector and the associated
vector functional
\begin{align*}
  \vnphic{k}{m}\colon \LGinf \to  \C, \quad x \mapsto \langle\Lambda(\lambda_{k}\rho_{m})|x\Lambda(\lambda_{k}\rho_{m})\rangle
\end{align*}
is a state.  Then the formulas
\begin{align} \label{eq:vn-weights}
  \vnphi(x) &:= \sum_{k,m} \vnphic{k}{m}(x), &
    T_{\lambda}(x)&:= \sum_{k,m}
\vnphic{k}{m}(x)\lambda_{k}, & 
T_{\rho}(x)&:=
    \sum_{k,m} \vnphic{k}{m}(x)\rho_{m},
\end{align}
where $x\in \LGinf_{+}$, define a a normal semi-finite weight $\vnphi$
on $\LGinf$ and normal semi-finite conditional expectations $T_{\lambda}$ and
$T_{\rho}$ from $\LGinf$ to $\lambda(l^{\infty}(I))$ and
$\rho(l^{\infty}(I))$, respectively. These maps are determined by
their restrictions to $\pi_{r}(A)$:
\begin{Lem} \label{lemma:vn-weights-unique} The normal weight $\vnphi$ and
  the normal conditional expectations $T_{\lambda},T_{\rho}$ satisfy $\pi_{r}(A) \subseteq
  \mathfrak{M}_{\vnphi} \cap \mathfrak{M}_{T} \cap \mathfrak{M}_{T'}$
  and
  \begin{align*}
    \vnphi(\pi_{r}(a))&= \phi(a), & T_{\lambda}(\pi_{r}(a))\Lambda(b) &=
    \sum_{k}\Lambda(\phi(\lambda_{k}a)\lambda_{k}b), &
    T_{\rho}(\pi_{r}(b))\Lambda(b) &= \sum_{m}\Lambda(\phi(\rho_{m}a)\rho_{m}b)
  \end{align*}
  for all $a,b\in A$, and are uniquely determined by these equations. 
\end{Lem}
\begin{proof}
  The equations follow immediately from the definition and the
  relation $\phi(a)=\sum_{k,m}
  \phi(\lambda_{k}\rho_{m}a\lambda_{k}\rho_{m})$, see \ref{}. To prove
  uniqueness, observe that the $p_{k,m}:=\pi_{r}(\lambda_{k}\rho_{m})$ are
  pairwise orthogonal projections in $\mathfrak{M}_{\vnphi}
  \cap \mathfrak{M}_{T_{\lambda}} \cap \mathfrak{M}_{T_{\rho}}$
  summing up to $1$, whence $\vnphi$, $T_{\lambda}$ and $T_{\rho}$ are
  the sums of the bounded linear maps that send an $x\in \LGinf_{+}$
  to $\vnphi(p_{k,m}xp_{l,n})$, $T_{\lambda}(p_{k,m}xp_{l,n})$, or
  $T_{\rho}(p_{k,m}xp_{l,n})$, respectively, which are determined by
  their restriction to $\pi_{r}(A)$.
\end{proof}

Invariance of $\phi$ implies invariance of $\vnphi$ as follows.
\begin{Prop} \label{prop:vn-invariance}
  Let $\mathscr{G}$ be a partial compact quantum group. Then for all
  $x\in \LGinf_{+}$, the
  normal, semi-finite weight $\vnphi$ on $\LGinf$ satisfies
  \begin{align*}
    (\id \otimes \vnphi)(\vnDelta(x)) &=  T_{\lambda}(x), &
    (\vnphi \otimes \id)(\vnDelta(x)) &= T_{\rho}(x).
  \end{align*}
\end{Prop}
\begin{proof}
  Let  $a \in A$. Then 
 \eqref{eq:integral} and the relation
  $\vnphic{k}{m}\circ \pi = \phic{k}{m}$ imply
  \begin{align*}
    (\id \otimes \vnphic{l}{m})(\vnDelta(\pi_{r}(a))) &= \sum_{k}
    \vnphic{k}{m}(\pi_{r}(a)) \lambda_{k}\rho_{l}.
  \end{align*}
  Since each $\vnphic{k}{m}$ is a vector state and $\pi_{r}(A)$ is
  weakly dense in $\LGinf$,  this equations
  remains true if we replace $\pi_{r}(a)$ by arbitrary $x\in
  \LGinf$. Summing over $l$ and $m$, we  obtain the first equation
  which we have to prove. The second one follows similarly.
\end{proof}
The next result gives a fairly complete description of the objects of
Tomita-Takesaki theory associated to $\vnphi$.
\begin{Lem} \label{lemma:vn-hilbert} The subspace
  $\Lambda(A) \subseteq \LGtwo$ is a Hilbert algebra with respect to
  the operations $\Lambda(a)\Lambda(b)=\Lambda(ab)$ and
  $\Lambda(a)^{*}= \Lambda(a^{*})$ for all $a,b\in A$, and a Tomita
  algebra with respect to the family of operators $\nabla_{z}$ given
  by $\nabla_{z}\Lambda(a)=\Lambda(\sigma_{z}(a))$ for all $a\in A$,
  $z\in \C$.  The associated left von Neumann
  algebra is $\LGinf$, the associated normal, semifinite, faithful
  weight is $\vnphi$, the modular operator  $\Delta_{\vnphi}$ is the
  closure of $\nabla_{-i}$,  the modular conjugation $J_{\vnphi}$ is
  given by $J_{\vnphi}\Lambda(a)=\Lambda(\sigma_{i/2}(a)^{*})$ for all
  $a\in A$, and the modular automorphism group $\sigma^{\vnphi}$
  satisfies $\sigma^{\vnphi}_{t} \circ \pi_{r} = \pi_{r} \circ
  \sigma_{t}$ for all $t\in \R$.
\end{Lem}
\begin{proof}
  We first show that $\Lambda(A)$ is a Hilbert algebra. Indeed, the
  map $\pi_{r}(a)\colon \Lambda(b) \to \Lambda(ab)$ is bounded for
  each $a \in A$ by Proposition \ref{prop:gns}, and the involution is
  pre-closed because  for all $a,b \in A$,
  \begin{align*}
    \langle \Lambda(a)|\Lambda(b^{*})\rangle = \phi(a^{*}b^{*}) =
    \phi(b^{*}\sigma(a^{*})) = \langle
    \Lambda(b)|\Lambda(\sigma(a^{*}))\rangle
  \end{align*}

To see that $\Lambda(A)$ and $(\nabla_{z})_{z}$ form a Tomita
  algebra, we have to verify that the map
  $z\mapsto
  \langle \Lambda(a)|\nabla_{z}\Lambda(b)\rangle =
  \phi(a^{*}\sigma_{z}(b))$ is entire for all $a,b\in A$ and that
  \begin{align*}
    \nabla_{z}\Lambda(a)^{*} &= \nabla_{\overline{z}}\Lambda(a)^{*}, &
    \langle \Lambda(a)|\Lambda(b)\rangle &= \langle
    \nabla_{-\overline{z}}\Lambda(a) |\Lambda(b)\rangle, & \langle
    \Lambda(a)^{*}|\Lambda(b)^{*}\rangle = \langle \Lambda(b)|\nabla_{-i}\Lambda(a)\rangle
  \end{align*}
  for all $a,b\in A$, $z\in \C$. But all of this follows immediately
  from Corollary \ref{cor:rep-characters}.


  The left von Neumann algebra of $\Lambda(A)$ is
  $\pi_{r}(A)''=\LGinf$ and the associated weight $\tilde\phi$
  satisfies $\tilde
  \phi(\pi_{r}(a^{*}a))=\langle\Lambda(a)|\Lambda(a)\rangle =
  \phi(a^{*}a)$ for all $a\in A$. By Lemma
  \ref{lemma:vn-weights-unique}, it coincides with $\vnphi$.  By
  \cite{} [Takesaki:2, Thm. VI.2.2 and its proof], the modular
  operator $\Delta_{\vnphi}$ is the closure of $\nabla_{-i}$ and the
  modular automorphism group is implemented by $(\nabla_{t})_{t}$. 
\end{proof}
 The general theory of Hilbert algebras \cite{} implies now:
\begin{Prop} \label{prop:hilbert-algebra} Let $\mathscr{G}$ be a
  partial compact quantum group. Then the extension $\vnphi$ of the
  invariant functional to $\LGinf$ is faithful. \qed
\end{Prop}
\begin{Rem}
  Without using the theory of Hilbert algebras, one could also check
  directly that the formula for $J_{\vnphi}$ defines an anti-linear
  isometry, that $J_{\vnphi}\pi_{r}(A)J_{\vnphi}$ commutes with $\pi_{r}(A)$ and hence
  with $\LGinf$, and that the family
  $(\Lambda(\lambda_{k}\rho_{m}))_{k,m}$ is cyclic for
  $J\pi_{r}(A)J$. Then this family is separating for $\LGinf$ and
  $\vnphi$ is faithful.
\end{Rem}

The scaling group $\tau$ and the unitary antipode $R$ of $\mathscr{G}$
can easily be lifted to $\CrG$ and $\LGinf$ using the following
result. Let us call a conjugate-linear map on a Hilbert space an
\emph{anti-symmetry} if it is isometric and its square is the
identity.
\begin{Lem} \label{lemma:vn-implementation}
  There exist a unique anti-symmetry $I$ and a strongly continuous
  one-parameter group $P=(P_{t})_{t}$ on
  on $\LGtwo$ such that for all $a\in A$, $t\in \R$,
  \begin{align*}
 I\Lambda(a) &=
    \Lambda(R(a)^{*}), & P_{t}\Lambda(a) &= \Lambda(\tau_{t}(a)).
  \end{align*}
\end{Lem}
\begin{proof}
  Corollary \ref{cor:rep-characters} implies that the formulas above
  define an anti-symmetry $I$ and unitaries $P_{t}$; for
  example, $\|I\Lambda(a)\|^{2})=\phi(R(a)R(a)^{*})=
  \phi(a^{*}a)=\|\Lambda(a)\|^{2}$, and $I^{2} = \id$
  because $*\circ R \circ * \circ R=  R^{2}=\id$. By A.1 in \cite{}
  [Takesaki:2] and Corollary  \ref{cor:rep-characters} 5., elements of
  $\Lambda(A)$ are analytic with respect to $P$; in
  particular,  $P$ is strongly continuous.
\end{proof}
\begin{Prop}
  Let $\mathscr{G}$ be an $I$-partial compact quantum group.  
  \begin{enumerate}
  \item  There exists a unique $*$-anti-automorphism $\vnR$ of
    $\LGinf$ such that $\vnR \circ \pi_{r} = \pi_{r} \circ R$.
    This $\vnR$ restricts to a $*$-anti-automorphism of
    $\CrG$.
  \item  There exists a unique strongly continuous
    one-parameter group $\vntau$ on $\LGinf$ such that $\vntau_{t}
    \circ \pi_{r} = \pi_{r} \circ \theta_{-it,it}$ for all $t\in \R$,
    and this $\vntau$ restricts to a strongly continuous one-parameter
    group on $\CrG$.
  \end{enumerate}
\end{Prop}
\begin{proof}
    Short calculations show that the maps $\vnR \colon x \mapsto
  Ix^{*}I$ and $\vntau_{t} \colon x \mapsto P_{t}xP_{t}^{*}$ have the
  desired properties.
\end{proof}
Note that the relations \eqref{eq:scaling-modular-delta} and
\eqref{eq:unitary-antipode} can be lifted to $\CrG$ and $\LGinf$ by
continuity.  The next result will allow us to relate $\vnR$ to the
unitary antipode of the measured quantum groupoid that we are going to
construct.
\begin{Lem} \label{lemma:vn-r-characterisation}
For all $a,b\in A$,
\begin{align*}
  \vnR(\id \otimes
  \omega_{J\Lambda(b),J\Lambda(b)})(\vnDelta(\pi(a^{*}a))) = (\id
  \otimes \omega_{J\Lambda(a),J\Lambda(a)})(\vnDelta(\pi(b^{*}b))).
\end{align*}
\end{Lem}
\begin{proof}
Let $c=a^{*}a$ and $d=b^{*}b$. 
A short calculation using \eqref{eq:modular} shows that  the right hand side is equal to
  \begin{align*}
    d_{(1)}\phi(\sigma_{i/2}(a)d_{(2)}\sigma_{i/2}(a)^{*})
    = d_{(1)}\phi(\sigma_{i/2}(c)d_{(2)}).
  \end{align*}
By Lemma \ref{lemma:strong-invariance} and
  \eqref{eq:scaling-modular-delta}, \eqref{eq:modular},  this equals 
  $S(\tau_{i/2}(c_{(1)}))
    \phi(\sigma_{i/2}(c_{(2)})d)$
which is the  left hand side.
\end{proof}

The operator-algebraic structures constructed so far fit into the
theory of measured quantum groupoids as follows.

Denote by $\nu$ the
normal, faithful, semifinite weight on $l^{\infty}(I)$ given by
\begin{align} \label{eq:vn-nu}
  \nu(f) &=\sum_{k} f(k) \quad \text{for all } f\in l^{\infty}(I)_{+}.
\end{align}
Then the relative tensor product of $\LGtwo$ with itself,
relative to the representations $\rho,\lambda$ of $l^{\infty}(I)$ and
the weight $\nu$, takes the simple form
\begin{align*}
\LGinf \otimesrl \LGinf \cong
  \bigoplus_{k} (\rho_{k}\LGtwo \otimes \lambda_{k}\LGtwo) =
  \vnE(\LGtwo \otimes \LGtwo),
\end{align*}
see \cite{},  the relative tensor product
of operators $S\in \rho(l^{\infty}(I))'$ and $T \in
\lambda(l^{\infty}(I))'$ gets identified with the compression
\begin{align*}
S \otimesrl T \equiv
  \vnE(S \otimes
  T) = (S \otimes T)\vnE \subseteq {\cal B}(\vnE(\LGtwo
  \otimes \LGtwo)),
\end{align*}
and the fiber product of  $  \LGinf$ with itself, relative to $\rho$
and $\lambda$,  gets identified with
\begin{align} \label{eq:vn-fiber}
  \begin{aligned}
    \LGinf \astrl \LGinf &= (\LGinf' \otimesrl \LGinf')' \\ &\equiv
    (\vnE(\LGinf' \otimes \LGinf'))' = \vnE(\LGinf \otimes
    \LGinf)\vnE.
  \end{aligned}
\end{align} 
By Lemma \ref{lemma:vn-delta} 3., we can co-restrict $\vnDelta$ to  a
normal, faithful $*$-homomorphism
\begin{align*}
  \tilde\Delta \colon \LGinf \to   \LGinf \astrl \LGinf.
\end{align*}
We now obtain a Hopf-von Neumann bimodule in the
sense of \cite{}.
\begin{Prop}
  Let $\mathscr{G}$ be an $I$-partial compact quantum group. Then
  \begin{enumerate}
  \item $\tilde\Delta(\lambda(x)) = \lambda(x) \otimesrl 1$ and
    $\tilde\Delta(\rho(x)) = 1 \otimesrl \rho(x)$ for all $x\in
    l^{\infty}(I)$, and
  \item $(\tilde\Delta \ast \id)\circ \tilde\Delta = (\id \ast \tilde\Delta)
    \circ \tilde\Delta$.
  \end{enumerate}
  In particular, $(l^{\infty}(I),\LGinf, \lambda,\rho,\tilde\Delta)$ is a
  Hopf-von Neumann bimodule.
\end{Prop}
\begin{proof}
Assertion 1.\ follows from \eqref{eq:delta-lambda-rho} and
Lemma \ref{lemma:vn-delta} 1.\ and ensures that  the $*$-homo\-morphisms
\begin{align*}
  \tilde\Delta \ast \id, \id \ast \tilde\Delta \colon  \LGinf \astrl \LGinf
  \to \LGinf \astrl \LGinf \astrl \LGinf
\end{align*}
are well-defined.  As in
\eqref{eq:vn-fiber}, we can identify
\begin{align*}
 \LGinf \astrl \LGinf \astrl \LGinf \cong \vnE^{(2)}(\LGinf
  \otimes \LGinf \otimes \LGinf)\vnE^{(2)},
\end{align*}
where $\vnE^{(2)}=(\vnE \otimes 1)(1 \otimes \vnE)$, and then the
$*$-homomorphisms become restrictions of the maps $\tilde\Delta \otimes
\id$ and $\id \otimes \tilde\Delta$, respectively. Now, 2.\ follows from
Lemma \ref{lemma:vn-delta} 1.\ and co-associativity of $\Delta$.
\end{proof}
This Hopf-von Neumann bimodule is a measured quantum groupoid in the
sense of \cite{}.
\begin{Theorem} \label{theorem:vn-measured} Let $\mathscr{G}$ be an
  $I$-partial compact quantum group. Then the Hopf-von Neumann
  bimodule $(l^{\infty}(I),\LGinf,\lambda,\rho,\tilde\Delta)$ and the
  weights $(T_{\lambda},T_{\rho}$ and $\nu$ defined in
  \eqref{eq:vn-weights} and \eqref{eq:vn-nu} form a measured quantum
  groupoid.  It is unimodular and its unitary antipode and scaling
  group coincide with $\vnR$ and $\vntau$, respectively.
\end{Theorem}
\begin{proof}
  First, observe that the  the modular
  automorphism groups of the weights $\nu \circ \lambda^{-1} \circ
  T_{\lambda}$ and $\nu \circ \rho^{-1} \circ T_{\rho}$ commute  because the two
  compositions coincide with $\vnphi$. Next, we need  to show that
  $T_{\lambda}$ is left-invariant in the sense that
  \begin{align*}
   (\id \underset{\nu}{_{\rho}\ast_{\lambda}} \vnphi)(\tilde\Delta(x)) = T_{\lambda}(x) 
  \end{align*}
  for all $x\in \LGinf_{+}$. But it is easy to see that the left hand
  side coincides with $(\id \ast \vnphi)(\vnDelta(x))$ so that the
  equation above follows from Proposition
  \ref{prop:vn-invariance}. Likewise $T_{\rho}$ is right-invariant in
  the appropriate sense. We thus obtain a measured quantum groupoid as
  claimed.  Denote by $\tilde R$ its unitary antipode and by
  $\tilde\tau$ its scaling group.

  Let us prove that $\tilde \tau_{t}=\vntau$ for all $t\in \R$. By
  \cite{} and \eqref{eq:scaling-modular-delta},
  \begin{align*}
    (\tilde \tau_{t} \astrl \sigma^{\vntau}_{t}) \circ \tilde \Delta
    &=\tilde \Delta \circ \sigma^{\vntau}_{t}, & (\vntau_{t} \otimes
    \sigma^{\vntau}_{t}) \circ \vnDelta &= \vnDelta \circ \vntau_{t}.
  \end{align*}
  The second equation implies that the first one remains true if we
  replace $\tilde\tau_{t}$ by $\vntau_{t}$.  Using Theorem A.7 in
  \cite{} [enock:action], we can conclude that $\tilde
  \tau_{t}=\vntau_{t}$.

 To prove that $\tilde R=\vnR$, we use the relations
  \begin{align*}
    \tilde R(\id \underset{\nu}{_{\rho} \ast_{\lambda}}
    \omega_{J\Lambda(b),J\Lambda(b)})(\vnDelta(\pi(a^{*}a))) &= (\id
    \underset{\nu}{_{\rho} \ast_{\lambda}}
    \omega_{J\Lambda(a),J\Lambda(a)})(\vnDelta(\pi(b^{*}b)))
  \end{align*}
from \cite{} and Lemma \ref{lemma:vn-r-characterisation}.
\end{proof}

%%% Local Variables: 
%%% mode: latex
%%% TeX-master: "dynamical-SUq-file"
%%% End: 

\section{Coamenability of partial compact quantum groups}

Let $\mathscr{G}$ be a partial compact quantum group. 

%We will use throughout that if $\pi: P(\mathscr{G}) \rightarrow \End(V)$ is any $^*$-representation on a pre-Hilbert space (meaning $\langle v,\pi(a)w\rangle = \langle \pi(a^*)v,w\rangle$ for all $v,w\in V$), then $\pi$ extends to a $^*$-representation of $\CuG$ on the completion of $V$.

\begin{Def} A partial compact quantum group $\mathscr{G}$ will be called \emph{coamenable} if the natural projection map $\pi_{\red}:\CuG\twoheadrightarrow\CrG$ is an isomorphism.
\end{Def}

%Our aim is to give a characterisation of coamenability in terms of the representation theory of the fusion algebra, cf.~ \cite{Kye1} and references therein.

We will show that the usual characterisation of coamenability in terms of the counit representation holds. For this we need the following version of Fell's absorption principle. We will continue to use the notation introduced in the previous section.
% Definition \ref{DefTenProd} and the discussion preceding it.

\begin{Lem}\label{LemFell} Let $(\Hsp,\pi)$ be a non-degenerate $^*$-representation of $\CuG$. Then $\pi_{\red}\iboxtimes \pi$ factorizes over $\CrG$.
\end{Lem}

\begin{proof}  
  By Lemma \ref{lemma:partial-isometry}, the partial isometry
  $V_{\pi}$ restricts to a unitary \[U: \oplus_q
  (L^2(\mathscr{G})_q\otimes {}_q\Hsp ) \rightarrow \oplus_l ({}_l
  L^2(\mathscr{G})\otimes {}^l\Hsp),\]\[\Lambda(x)\otimes \xi \mapsto
  \Lambda(x_{(1)})\otimes \pi(x_{(2)})\xi,\quad x\in
  P(\mathscr{G}),\xi\in H,\] with inverse $\Lambda(x)\otimes \xi
  \mapsto \Lambda(x_{(1)})\otimes
  \pi(S(x_{(2)}))\xi$. %more info needed?
  Denote $\pi_{\red,q}$ for the restriction of $\pi_{\red}$ to
  $L^2(\mathscr{G})_q$. Then it is clear that $U$ intertwines
  $\oplus_q (\pi_{\red,q}\otimes 1)$ with $\pi_{\red}\boxtimes
  \pi$. As each $\pi_{\red,q}$ descends to $\CrG$, this proves the
  lemma.
\end{proof} 

Recall now from \cite{DCT1} the $^*$-representation 
\begin{align} \label{eq:weps}
  \weps: P(\mathscr{G}) \rightarrow \End(\C^{(I)}), \quad x\in
  \Gr{A}{k}{l}{m}{n}\mapsto \epsilon(x) e_{kl},
\end{align}
where $\epsilon$ denotes the counit of $P(\mathscr{G})$ and $\C^{(I)}$ the direct sum of $I$ copies of $\C$ with the natural pre-Hilbert space structure. We will denote its extension to a $^*$-representation $\CuG\rightarrow B(l^2(I))$ by the same symbol.

\begin{Lem}\label{LemUnit} Let $(\Hsp,\pi)$ be a non-degenerate $^*$-representation of $\CuG$. Then $\weps\iboxtimes \pi$ is unitarily equivalent to $\pi$.
\end{Lem} 
\begin{proof} The map \[U: \Hsp \rightarrow \oplus_k \left({}_kl^2(I)\otimes {}^k\Hsp\right),\quad \xi \mapsto \sum_k e_k\otimes \pi(\lambda_k)\xi\] is a unitary intertwiner.
\end{proof} 

\begin{Prop} The partial compact quantum group $\mathscr{G}$ is coamenable if and only if $\weps$ descends to a $^*$-representation of $\CrG$.
\end{Prop} 
\begin{proof} $\Rightarrow$ is clear. For $\Leftarrow$, note that $\Delta^u$ descends to a (non-unital) $^*$-homomorphism $\CrG\rightarrow M(\CrG\otimes \CuG)$ by Fell's absorption principle. Composing with $\weps\otimes \id$ and using Lemma \ref{LemUnit} with respect to a faithful representation of $\CuG$, we see that there exists a splitting $\CrG\rightarrow \CuG$ of $\pi_{\red}$, hence $\pi_{\red}$ is an isomorphism.
\end{proof}
 
We now pass to the character algebra of $\mathscr{G}$.
 
 \begin{Def} Let $\mathscr{X}$ be a unitary rcfd representation of a partial compact quantum group $\mathscr{G}$ on an $I$-bigraded Hilbert space $\Hsp$. The \emph{partial characters} of $\mathscr{X}$ are defined as \[\Gr{\chi}{}{X}{k}{l} = (\id\otimes \Tr)(\Gr{X}{k}{l}{k}{l}) = \sum_i(\Gr{X}{k}{l}{k}{l})_{ii} \in \Gr{P(\mathscr{G})}{k}{l}{k}{l},\] where $\Tr$ is the non-normalized trace on $B(\GrDA{\Hsp}{k}{l})$, and where $(\Gr{X}{k}{l}{k}{l})_{ij}$ denote matrix coefficients with respect to some fixed orthonormal basis. The \emph{total character} of $\mathscr{X}$ is the multiplier $\chi^X  =\underset{k,l}{\sum} \,\Gr{\chi}{}{X}{k}{l}\in M(P(\mathscr{G}))$.
 \end{Def}
 
 For example, if $\mathscr{U}$ is the trivial representation of $\mathscr{G}$, then $\Gr{\chi}{}{U}{k}{l} = \delta_{k,l}\UnitC{k}{k}$, and $\chi^U = \sum_k \UnitC{k}{k}$.
 
 We aim to show that each character $\chi^X$ is in fact a well-defined element in $M(\CuG)$. Recall from \cite{DCT1} that $\smCirct$ denotes the tensor product of two representations of $\mathscr{G}$. 
 
 \begin{Lem} \label{LemCharMult} We have $\Gr{\chi}{}{X\smCirct Y}{k}{m}= \underset{l}{\sum}\, \Gr{\chi}{}{X}{k}{l}\Gr{\chi}{}{Y}{l}{m}$.
 \end{Lem}
 
 \begin{proof} 
 This follows from the definition of $\smCirct$, namely $\Gr{(X\Circt Y)}{k}{l}{m}{n} = \sum_{r,s} \left(\Gr{X}{k}{r}{m}{s}\right)_{12} \left(\Gr{Y}{r}{l}{s}{n}\right)_{13}$.
 \end{proof}

For the next lemma, recall from \cite{DCT1} that a unitary rcfd representation of $\mathscr{G}$ is said to have \emph{finite hypersupport} if it is a \emph{finite} direct sum of irreducible unitary rcfd representations.  The left hyperobject support of $\mathscr{X}$ is then the set of all $k$ such that $\GrDA{\Hsp}{k}{l}\neq \{0\}$ for some $l$. The right hyperobject support is defined similarly. Note that the left (resp. right) hyperobject support consists of  hyperobject classes, where $k$ and $l$ define the same hyperobject class if $\UnitC{k}{l}\neq0$. 


 \begin{Lem} \label{LemBoundDim} Let $\mathscr{X}$ be an rcfd unitary
   representation of $\mathscr{G}$ on an $I$-bigraded Hilbert space $\Hsp$ with finite hyperobject support, and write $d^X_{kl} = \dim(\Gru{H}{k}{l})$. Then the matrix $D^X = (d^X_{kl})_{k,l}$ defines a bounded operator on $l^2(I)$. 
  \end{Lem}
  \begin{proof} We may assume that $\mathscr{X}$ is irreducible, say
    with left hyperobject support $\alpha$ and right hyperobject support
    $\beta$. Then we can find (a priori not necessarily bounded)
    morphisms \[R: \C^{(I)}\rightarrow \underset{k,l,m}{\oplus} \GrDA{H}{k}{m}\otimes
    \GrDA{H}{m}{l},\qquad \bar{R}: \C^{(I)} \rightarrow \underset{k,l,m}{\oplus}
    \GrDA{\bar{H}}{k}{m}\otimes \GrDA{\bar{H}}{m}{l},\] establishing a
    duality between $\underset{k,l}{\oplus} \GrDA{H}{k}{l}$ and
    $\underset{k,l}{\oplus}\GrDA{\bar{H}}{k}{l}$ inside the category of
    rcfd pre-Hilbert spaces. However, as $R^*R$ is a scalar multiple
    of the projection onto $\C^{(I_{\alpha})}$ by irreducibility, and
    similarly for $\bar{R}$, it follows that $R$ and $\bar{R}$ can be
    completed to bounded operators between the respective Hilbert
    space completions. It then follows from \cite[Lemma A.3.2]{DCY1}
    that 
    \begin{align} \label{eq:dim-estimate}
  \sup_k (\sum_l (d_{kl}^X+d_{lk}^X)) < \infty.    
    \end{align}
 By the Schur test,
    this implies that $D^X$ is bounded.
\end{proof} 


   
\begin{Lem} The sum $\chi^X  =\sum \Gr{\chi}{}{X}{k}{l}$ converges strictly to an element in $M(\CuG)$. 
\end{Lem} 
 
\begin{proof} It is sufficient to prove that the matrix of operators $\left(\pi\left(\Gr{\chi}{}{X}{k}{l}\right)\right)_{k,l}$ defines a bounded matrix of operators in some faithful $^*$-representation $\pi$ of $C^{u}_{0}(\mathscr{G})$. But as we have that each $\|\pi\left(\left(\Gr{X}{k}{l}{k}{l}\right)_{ij}\right)\|\leq 1$ by unitarity of $\mathscr{X}$, we have $\| \pi\left(\Gr{\chi}{}{X}{k}{l}\right)\| \leq d^X_{kl}$.  Hence boundedness follows from Lemma \ref{LemBoundDim}.
\end{proof} 
 
Denote now $\mathcal{C}$ for the C$^*$-algebra generated by all $\chi^X$ inside $M(\CrG)$, where $\mathscr{X}$ runs over all irreducible unitary representations. For $\alpha$ a hyperobject and $l\in \alpha$, write  $L^2(\mathcal{C})_l$ for the Hilbert space spanned by all $\Lambda(\chi^X_l)$, where $\chi^X_l = \sum_{k} \Gr{\chi}{}{X}{k}{l}$ and where the $X$ run over all irreducibles with right hyperobject support equal to $\alpha$. By the orthogonality relations, \cite[Corollary 2.23]{DCT1}, the $\Lambda(\chi^X_l)$ form an orthonormal basis for $L^2(\mathcal{C})_l$. Moreover, $\mathcal{C}$ preserves each $L^2(\mathcal{C})_l$, hence defines a $^*$-representation of $\mathcal{C}$ which we will denote by $\pi_l$. 

\begin{Lem}\label{LemEq} Assume $l$ and $l'$ are in the same hyperobject class. Then the representations of $\mathcal{C}$ on $L^2(\mathcal{C})_l$ and $L^2(\mathcal{C})_{l'}$ are equivalent.
\end{Lem} 
\begin{proof} The map $\Lambda(\chi^X_l)\rightarrow \Lambda(\chi^X_{l'})$ is a unitary intertwiner (since $\chi^X_l\neq 0$ if and only if $l$ in the right hyperobject support of $\mathscr{X}$ if and only if $l'$ in the right hyperobject support of $\mathscr{X}$).
\end{proof} 

Denote now $L^2(\mathcal{C})\subseteq L^2(\mathscr{G})$ for the direct sum of the $L^2(\mathcal{C})_l$ where the $l$ run over a fixed section of the hyperobject set, and write $\pi_{\mathcal{C}}= \underset{l}{\oplus} \pi_l$. 
 
\begin{Prop} The representation $\pi_{\mathcal{C}}$ of $\mathcal{C}$ on $L^2(\mathcal{C})$ is faithful.
\end{Prop} 

\begin{proof} Let $p$ be the projection of $L^2(\mathscr{G})$ onto $L^2(\mathcal{C})$. Then for $\chi = \chi^X$, $a\in \Gr{P(\mathscr{G})}{k}{l}{m}{n}$ and $b\in \Gr{P(\mathscr{G})}{r}{s}{p}{q}$, we compute that \[\langle \Lambda(b),\chi \Lambda(a)\rangle =  \langle \Lambda( \UnitC{r}{r}),\chi p\Lambda(a\sigma_{-i}(b^*))\rangle.\] By continuity, this holds for all elements $\chi \in \mathcal{C}$. As $p\Lambda(a\sigma_{-i}(b^*)) \in L^2(\mathcal{C})_r$, and as $\UnitC{r}{r} = \chi^U_r$ for $\mathscr{U}$ the trivial character, the lemma follows by Lemma \ref{LemEq}.
\end{proof}

Let now $\mathscr{C}$ be the partial fusion $^*$-algebra of $\Rep(\mathscr{G})$, that is, the locally unital $^*$-algebra generated as a vector space over $\C$ by equivalence classes of irreducible unitary representations of $\mathscr{G}$, with multiplication determined by the fusion rules and the $^*$-structure obtained by taking the dual. It follows from the above that $\mathcal{C}$ is a C$^*$-completion of $\mathscr{C}$ which is completely determined by $\Rep(\mathscr{G})$.

On $\mathscr{C}$, the $*$-representation $\weps \colon P(\mathscr{G})
\to l^{2}(\C^{(I)})$ defined in
\eqref{eq:weps} is given by
\begin{align*}
  \weps(\chi^{X}) = \sum_{k,l} \epsilon(\Gr{\chi}{}{X}{k}{l})e_{kl} =
  \sum_{k,l} (\epsilon \otimes \Tr)(\Gr{X}{k}{l}{k}{l}) e_{kl}
   = \sum_{k,l} d^{X}_{kl} e_{kl} =D^{X}
\end{align*}
because $(\epsilon \otimes
\id)(\Gr{X}{k}{l}{k}{l})=\id_{\Gru{H}{k}{l}}$  for   $\mathscr{X}$ and
$\mathcal{H}$ as in Lemma \ref{LemBoundDim}.

%. In fact, since the representation on $L^2(\mathcal{C})_l$ is equivalent to the one on $L^2(\mathcal{C}_{l'})$ when $l$ and $l'$ belong to the same hyperobject, we see that $\mathcal{C}$ is determined completely by  itself.  


\begin{Def} We say $\mathscr{G}$ is \emph{dimension} coamenable %Fusion coamenable?
if the representation \[\tilde{\epsilon}: \mathscr{C} \rightarrow B(l^2(I)), \quad \chi^X \mapsto D^X\] extends to a representation of $\mathcal{C}$.
\end{Def}

It is immediately clear that a coamenable partial compact quantum group is dimension coamenable. However, as we will see later, the converse is not true. This is in marked contrast with the case of quantum groups, where amenability of the fusion algebra (with respect to its \emph{scalar} dimension function) \emph{does} determine coamenability of the associated compact quantum group, cf. \cite[Theorem 4.5]{Kye1}.
 
 % Corollary: canonical partial quantum group constructed from a tensor C*-cat is always dimension coamenable, cf. Neshveyev-Tuset section 2.7

Recall now from \cite[Definition 1.23 and Section 4.1]{DCT1} the notion of \emph{canonical partial compact quantum group} associated to a \emph{partial fusion tensor C$^*$-category $\CatCC$}.

\begin{Prop} Let $\mathscr{G}_{\CatCC}$ be the canonical partial compact quantum group associated to an $\mathcal{I}$-partial fusion tensor C$^*$-category $\CatCC$. Then $\mathscr{G}_{\CatCC}$ is dimension coamenable.  
\end{Prop}% Ref to add to canonical partial compact quantum group
\begin{proof} 
  By construction, $\mathscr{G}_{\CatCC}$ is an $I$-partial compact quantum group, where $\Rep(\mathscr{G}_{\CatCC})$ can be canonically
  identified with $\CatCC$, and the set $I$ labels a maximal set
  $\{U_{k}\}$ of mutually non-isomorphic irreducible objects of
  $\CatCC$.  Under the Tannaka duality, an object $X$ of $\CatCC$ corresponds to a corepresentation $\mathscr{X}$ of $\mathscr{G}_{\CatCC}$ on an $I$-bigraded Hilbert space $\Hsp$ with $\GrDA{\Hsp}{k}{l} = \Hom(U_k,X\otimes U_l)$. Note then that the hyperobject class of $k\in I$ is nothing but the \emph{left} hyperobject support of $U_k$. % More info! Also, small mistake in first paper, order of indices in canonical TK should be reversed.
  
  Fix now $l\in I$. Then $L^2(\mathcal{C})_l$ has an orthonormal basis consisting of all $\Lambda(\chi_{l}^{U_k})$ where the right hyperobject support of $U_k$ is the left hyperobject support of $U_l$. Then, by
  Lemma \ref{LemCharMult}, 
  \begin{align*}
 \chi^{X}\Lambda(\chi^{U_{n}}_{l}) = \chi^{X \Circt U_{n}}_{l} = \sum_{m}
 \dim(\Hom(U_{m},X\Circt U_{n})) \Lambda(\chi^{U_{m}}_{l}) = \sum_{m}
 D^{X}_{mn} \Lambda(\chi^{U_{m}}_{l}).
  \end{align*}
  
  Choose now a section $\mathcal{I} \rightarrow I, \alpha \mapsto l(\alpha)$. Then we can define a map $\theta:I\rightarrow I$ by sending $m$ to $l(\alpha)$ for $\alpha$ the right hyperobject support of $U_m$. Then \[l^2(I)\rightarrow L^2(\mathcal{C}),\quad \delta_m\mapsto \Lambda(\chi_{\theta(m)}^{U_m})\] is a unitary intertwining the regular representation of the fusion algebra of $\CatCC$ on $L^2(\mathcal{C})$ with the
representation $\tilde \epsilon$. Since the former extends to
$\mathcal{C}$, so does the latter.
 \end{proof}
 
 In the next proposition, we consider the partial compact quantum groups associated to weighted reciprocal random walks, see \cite[Section 5]{DCT1}. 
 
 \begin{Prop} Consider the partial compact quantum group $\mathscr{G}(\Gamma)$ associated to a weighted reciprocal random walk $\Gamma = (\Gamma,i,w,\sgn)$. Then $\mathscr{G}(\Gamma)$ is dimension coamenable if and only if $\|\Gamma\| \leq 2$. 
 \end{Prop}% Ref to add to first article
 \begin{proof} As $\Rep(\mathscr{G}(\Gamma))$ is by construction a Temperley-Lieb category,  the character algebra $\mathscr{G}(\Gamma)$ is now the polynomial ring on one generator $\C\lbrack \chi_{1/2}\rbrack$, with $\chi_{1/2}$ the character of the generating object $u_{1/2}$ of $\Rep(\mathscr{G}(\Gamma))$. It has regular C$^*$-completion $C(\lbrack -2,2\rbrack)$, with $\chi_{1/2}$ corresponding to the identity function. 
 
Let now $\Gamma^{(0)}$ be the vertex set of $\Gamma$. Then $\Gamma^{(0)}$ is identified with the index set $I$ of $\mathscr{G}(\Gamma)$, and $D^{u_{1/2}}_{kl}$ is the number of arrows from $k$ to $l$. Hence we obtain that there exists a map $C(\lbrack -2,2\rbrack) \rightarrow B(l^2(\Gamma^{(0)}))$ with $\chi_{1/2}\mapsto D^{u_{1/2}}$ if and only if $\|D^{u_{1/2}}\| \leq  2$ if and only if $\|\Gamma\|\leq 2$.   
 \end{proof} 
 
 In particular, the partial compact quantum groups coming from
 dynamical quantum $SU(2)$ are dimension coamenable. However, as we
 will see in the next section, they are not coamenable.

% Coamenability in terms of invariant measure not considered yet.
%%% Local Variables: 
%%% mode: latex
%%% TeX-master: "dynamical-SUq-file"
%%% End: 


%\input{3-free.tex}

% Can also go beyond and define dynamical quantum $SU(n,1)$, starting from one-parameter space of complex projective spaces. Works for all hermitian symmetric spaces


%The relations for the adjoint further imply that \begin{eqnarray*}u_{kl}^{+-} &=&\frac{E(l,l-1)}{E(k,k+1)}(u_{k+1,l-1}^{-+})^*,\\  u_{kl}^{++}&=& \frac{E(l,l+1)}{E(k,k+1)}(u_{k+1,l+1}^{--})^* .\end{eqnarray*}


%\begin{eqnarray*} u_{++} &=& \frac{w_+(\rho)^{1/2}}{w_+(\lambda)^{1/2}}u_{--}^*,\\u_{+-}&=& (-1)^{s_q}  \frac{w_-^{1/2}(\rho)}{w_+^{1/2}(\lambda)}u_{-+}^*.  \\ 
%\end{eqnarray*}

%Let us write \[F(k) = |q|^{-1}w_+(k) =  |q|^{-1}\frac{|q|^{x+k+1}+|q|^{-x-k-1}}{|q|^{x+k}+|q|^{-x-k}},\] and further put\[\alpha = \frac{F^{1/2}(\rho-1)}{F^{1/2}(\lambda-1)}u_{--},\qquad \beta = \frac{1}{F^{1/2}(\lambda-1)}u_{-+}.\] Then the unitarity of $(u_{\epsilon,\nu})_{\epsilon,\nu}$ together with \eqref{EqAdju} and \eqref{EqGradu} are equivalent to the commutation relations \begin{equation}\label{EqqCom} \alpha \beta = qF(\rho-1)\beta\alpha \qquad \alpha\beta^* = qF(\lambda)\beta^*\alpha\end{equation} \begin{equation}\label{EqDet} \alpha\alpha^* +F(\lambda)\beta^*\beta = 1,\qquad \alpha^*\alpha+q^{-2}F(\rho-1)^{-1}\beta^*\beta = 1,\end{equation}\begin{equation*} F(\rho-1)^{-1}\alpha\alpha^* +\beta\beta^* = F(\lambda-1)^{-1},\qquad  F(\lambda)\alpha^*\alpha +q^{-2}\beta\beta^* = F(\rho),\end{equation*} \begin{equation}\label{EqGrad} f(\lambda)g(\rho)\alpha =
%\alpha f(\lambda+1)g(\rho+1),\qquad f(\lambda)g(\rho)\beta = \beta f(\lambda+1)g(\rho-1).\end{equation}%layout might be nicer

%These are precisely the commutation relations for the dynamical quantum $SU(2)$-group as in \cite[Definition 2.6]{KoR1}, except that the precise value of $F$ has been changed by a shift in the parameter domain by a complex constant. Clearly, by Theorem \ref{TheoGenRel} the (total) coproduct on $A_x$ also agrees with the one on the dynamical quantum $SU(2)$-group, namely \begin{eqnarray*} \Delta(\alpha) &=& \Delta(1) (\alpha\otimes \alpha - q^{-1}\beta\otimes \beta^*),\\ \Delta(\beta) &=& \Delta(1)(\beta\otimes \alpha^* +\alpha\otimes \beta)\end{eqnarray*} where $\Delta(1) = \sum_{k\in \Z} \rho_k\otimes \lambda_k$.

%Let us write $B_x = B_x(\Gamma)$ for the $*$-algebra generated by a copy of the $*$-algebra $c_b(\Z\times \Z)$ of bounded functions on $\Z\times \Z$ as well as elements $\alpha,\beta$ satisfying the relations \eqref{EqqCom}, \eqref{EqDet} and \eqref{EqGrad}.

% Lemma doesn't work because $\alpha$,$\beta$ not bounded.
%\begin{Lem} There is a one-to-one correspondence between non-degenerate $*$-representations of $A_x$ on Hilbert spaces and unital $^*$-representations of $B_x$ on Hilbert spaces for which the restriction to $c_b(\Z\times \Z)$ is strictly continuous.
%\end{Lem}



\section{The universal C$^*$-algebra of the dynamical quantum $SU(2)$}\label{SecUni}

We assume in the following that $0<q<1$ and $\sigma \in \{-1,+1\}$. We also choose $x>0$.
%We also choose $x\in \lbrack \sqrt{q},1\rbrack$.

Denote $\ctau(y) = y+y^{-1}$ and $w_{\pm}(y) = \frac{\ctau(q^{\pm}y)}{\ctau(y)}$. Denote $\sigma_+ = 1$ and $\sigma_- = -\sigma$. Let $\Lambda_x = xq^{\Z}$, and let $B$ be the $^*$-algebra of finite support functions on $\Lambda_x\times \Lambda_x$, whose Dirac functions we write as $\delta_{(y,z)}=\UnitC{y}{z}$. Let $A_x$ be the $^*$-algebra generated by a copy of $B$ and elements \[(u_{\epsilon,\nu})_{y,z}\] for $\epsilon,\nu\in \{-1,1\}=\{-,+\}$ and $y,z\in \Lambda_x$ with defining relations \begin{eqnarray*} \sum_{\mu\in \{\pm\}} (u_{\mu,\epsilon})_{q^{-\mu}w,y}^* (u_{\mu,\nu})_{q^{-\mu}w,z}&=& \delta_{y,z} \delta_{\epsilon,\nu} \UnitC{w}{q^{\epsilon}y},\\ \sum_{\mu\in \{\pm\}} (u_{\epsilon,\mu})_{y,w} (u_{\nu,\mu})_{z,w}^* &=& \delta_{\epsilon,\nu}\delta_{y,z} \UnitC{y}{w} \\ (u_{\epsilon,\nu})_{y,z}^* &=& \frac{\sigma_{\nu}w_{\nu}(z)^{1/2}}{\sigma_{\epsilon}w_{\epsilon}(y)^{1/2}} (u_{-\epsilon,-\nu})_{q^{\epsilon}y,q^{\nu}z}.\end{eqnarray*} The element $(u_{\epsilon,\nu})_{y,z}$ lives inside the component $\Gr{(A_x)}{y}{q^{\epsilon}y}{z}{q^{\nu}z}$.

Consider $M(A_x)$, the multiplier algebra of $A_x$. For a function $f$ on $\Lambda_x\times \Lambda_x$, write $f(\lambda,\rho) = \sum_{y,z} f(y,z)\UnitC{y}{z} \in M(A_x)$. Similarly, for a function $f$ on $\Lambda_x$ we write $f(\lambda) = \sum_{y,z} f(y)\UnitC{y}{z}$ and $f(\rho) = \sum_{y,z}f(z)\UnitC{y}{z}$. We then write for example $f(q\lambda,\rho)$ for the element corresponding to the function $(y,z)\mapsto f(qy,z)$.

We can further form in $M(A_x)$ the elements $u_{\epsilon,\nu} = \sum_{y,z} (u_{\epsilon,\nu})_{y,z}$. Then $u=(u_{\epsilon,\nu})$ is a unitary 2$\times$2 matrix. Moreover, \begin{equation}\label{EqAdju}u_{\epsilon,\nu}^* = u_{-\epsilon,-\nu}\frac{ \sigma_{\nu}w_{\nu}^{1/2}(\rho)}{\sigma_{\epsilon}w_{\epsilon}^{1/2}(\lambda)} ,\end{equation} where $w_{\pm}^{1/2}(y) = w_{\pm}(y)^{1/2}$.  We then have the following commutation relations between functions on $\Lambda_x\times \Lambda_x$ and the entries of $u$: \begin{equation}\label{EqGradu} f(\lambda,\rho)u_{\epsilon,\nu} = u_{\epsilon,\nu}f(q^{-\epsilon}\lambda,q^{-\nu}\rho).\end{equation}

In the following, we write $u_{--}=\alpha, u_{-+}= \beta, u_{+-}=\gamma,u_{++}=\delta$. We then have the following relations.


\[\left\{\begin{array}{lllllll} \alpha\alpha^* + \beta\beta^* &=& 1 &&  \gamma\gamma^* + \delta\delta^* &=& 1,\\ \alpha^*\alpha+ \gamma^*\gamma &=&1,&&\beta^*\beta+ \delta^*\delta &=& 1,\\ \\ \alpha \gamma^* = -\beta \delta^*, &&&& \alpha^*\beta = -\gamma^*\delta, \end{array}\right.\]

\[ \delta =  \frac{w_+^{1/2}(\rho)}{w_+^{1/2}(\lambda)}\alpha^*, \quad \gamma=  -\sigma \frac{w_{-}^{1/2}(\rho)}{w_+^{1/2}(\lambda)}\beta^*,\quad  \beta = -\sigma \frac{w_+^{1/2}(\lambda)}{w_{-}^{1/2}(\rho)}\gamma^*, \quad  \alpha = \delta^* \frac{w_+^{1/2}(\lambda)}{w_+^{1/2}(\rho)}.\]


We wish to classify the irreducible $^*$-representations of $A_x$. We will need an auxiliary notion.

%The parametrisation will hinge on the classification of what we call irreducible $$-adapted sets, which we will now discuss.

%\subsection{Irreducible $c$-sets}

\begin{Def}\label{DefAdapt} Let $c\in\R$. For $\epsilon \in \{\pm\}$, an element $y>0$ will be called \emph{$c_{\epsilon}$-adapted} if \begin{equation}\label{EqAd+} c \leq \ctau(q^{-\epsilon}y),\end{equation} and \emph{strictly} $c_{\epsilon}$-adapted if this holds strictly. It is called \emph{$c$-adapted} if it is both $c_+$- and $c_-$-adapted. 

A subset $Z$ of $\R^+_0$ is called a \emph{$c$-set} if the following conditions hold: \begin{itemize} 
\item[$\bullet$] $Z$ is not empty.
\item[$\bullet$] $Z$ consists of $c$-adapted points.
\item[$\bullet$] If $y\in Z$ is strictly $c_{\epsilon}$-adapted, then $q^{-2\epsilon}y$ is in $Z$.
\end{itemize}
%An $(x,c)$-set $Z$ is called \emph{even} (resp.~ \emph{odd}) if $Z\subseteq q^{2\Z}$ (resp.~ $Z\subseteq q^{2\Z+1}$).
A $c$-set is called \emph{irreducible} if it can not be written as the union of two disjoint $c$-sets.
\end{Def}

We will classify irreducible $c$-sets. For $y>0$, we write $T_y = yq^{1+2\Z}$, $T_y^+ = yq^{1+2\N}$ and $T_y^{-} = yq^{-1-2\N}$. 

%We use the convention $\N = \{0,1,2,\ldots\}$ and $\N_0=\{1,2,\ldots\}$.

\begin{Prop}\label{PropClass1D}

% Picture!
\begin{itemize} The following list exhausts all irreducible $c$-sets.
\item[$\bullet$] $c<2: T_y, y>0$
\item[$\bullet$] $c \geq 2$, $c=\ctau(z)$ with $0 < z\leq 1$:
\begin{itemize}
\item[$\bullet$] $T_y$ for $\frac{q}{z}<qy<\frac{z}{q}$.
\item[$\bullet$] $T_{z}^+$ and $T_{1/z}^-$.
\item[$\bullet$] $\{1\}$ if $z=q$.
\end{itemize}
\end{itemize}
\end{Prop} 
\begin{proof} Elementary.
\end{proof} 

\subsection{Representation theory of $A_x$}

Let us now return to the representation theory of the $^*$-algebra $A_x$.

Let $\pi$ be any (necessarily bounded) non-degenerate $^*$-representation of $A_x$ on a Hilbert space $\Hsp_{\pi}$. Then \[\Hsp_{\pi} = \oplus \Hsp^{y}_{z},\qquad \Hsp^{y}_{z} = \pi(\UnitC{y}{z})\Hsp.\] Write \[H_{\pi} =  \textrm{the (non-closed) linear span of all }\Hsp^{y}_{z}.\] Then $\pi(A_x)H_{\pi} = H_{\pi}$. It follows that one can extend $\pi$ to a map \[\pi: M(A_x) \rightarrow \End_{\adj}(H_{\pi}),\] where $\End_{\adj}(H_{\pi})$ denotes the $^*$-algebra of adjointable operators on the pre-Hilbert space $H_{\pi}$. In particular, the generators $\alpha,\beta,\gamma,\delta$ and their adjoints give rise to (contractive) maps $H_{\pi}\rightarrow H_{\pi}$. 

We have the following easy lemma.

\begin{Lem} There is a one-to-one correspondence between\begin{itemize}\item[$\bullet$] non-degenerate $^*$-representations $(\Hsp_{\pi},\pi)$ of $A_x$ on Hilbert spaces, and 
\item[$\bullet$] $\Lambda_x$-bigraded pre-Hilbert spaces $H_{\pi}$ with norm-complete components and equipped with adjointable maps $\alpha,\beta:H_{\pi}\rightarrow H_{\pi}$ satisfying the commutation relations as in \eqref{EqqCom}, \eqref{EqDet} and \eqref{EqGrad} and with $f(\lambda,\rho)\xi = f(y,z)\xi$ for $f$ a function on $\Lambda_x\times \Lambda_x$ and $\xi\in H^x_y$.
\end{itemize}
\end{Lem}

\begin{Def} The \emph{Casimir} of $A_x$ is defined to be the following element $\Omega\in M(A_x)$, \[\Omega = \ctau(q\lambda/\rho) - \ctau(\lambda)\ctau(\rho/q)\beta^*\beta.\]  
\end{Def}

\begin{Lem} The element $\Omega$ is a self-adjoint central element in $M(A_x)$.
\end{Lem}
\begin{proof}
This follows from a straightforward calculation, cf. \cite[Lemma 3.3]{KoR1}. By self-adjointness, it suffices to check that $\Omega$ commutes with $\delta^*$ and $\beta^*$. The first commutation follows from the skew commutativity relations and the relation $\ctau(q\lambda)\ctau(\rho)\gamma\gamma^* =\ctau(\lambda)\ctau(\rho/q)\beta^*\beta$. The second commutation follows by using the identity \[\beta\beta^* =  \frac{w_+(\rho/q)-w_+(\lambda/q)}{w_+(\rho/q)} + \frac{w_+(\lambda/q)}{w_+(\rho/q)}\beta^*\beta\] and some elementary computations with the $\tau$-function.
\end{proof}

\begin{Cor}\label{CorCas} If $\pi$ is an irreducible $^*$-representation of $A_x$, there exists $c\in \R$ such that $\pi(\Omega)\xi = c\xi$ for all $\xi \in V_{\pi}$. 
\end{Cor} 
\begin{proof} As $\pi(\Omega)$ is bounded when restricted to any $V^y_z$, this follows immediately from a spectral argument. 
\end{proof} 

The following lemma follows from a straightforward computation, using the relations \eqref{EqDet}.

\begin{Lem}\label{LemAmp} Inside $M(A_x)$, we have the following identities:
\begin{eqnarray*}
\alpha^*\alpha &=& \frac{\ctau(q\lambda\rho)+\Omega}{\ctau(\lambda)\ctau(q\rho)}\\
\alpha\alpha^* &=& \frac{\ctau(\lambda\rho/q)+\Omega}{\ctau(\lambda/q)\ctau(\rho)}\\ 
\beta^*\beta &=& \frac{\ctau(q\lambda/\rho)-\Omega}{\ctau(\lambda)\ctau(\rho/q)}\\
\beta\beta^* &=&  \frac{\ctau(\lambda/q\rho)-\Omega}{\ctau(\lambda/q)\ctau(\rho)}.
\end{eqnarray*}
\end{Lem}

Note that the right hand sides are well-defined because of centrality of $\Omega$.

\begin{Cor}\label{CorOneDim} If $\pi$ is an irreducible $^*$-representation of $A_x$ on a Hilbert space $\Hsp_{\pi}$, then $\Hsp^y_z$ is at most one-dimensional for each $y,z\in \Lambda_x$. Moreover, either all $\Hsp^y_z$ with $y/z\in q^{2\Z+1}$ are zero, or all $\Hsp^y_z$ with $y/z\in q^{2\Z}$ are zero. 
\end{Cor} 
\begin{proof} 
Using Corollary \ref{CorCas}, the first assertion follows immediately from \eqref{EqqCom}, the grading relations \eqref{EqGrad} and Lemma \ref{LemAmp}. The second assertion follows immediately from the grading relations \eqref{EqGrad}.
\end{proof}

\begin{Def} Let $(\Hsp_{\pi},\pi)$ be an irreducible $^*$-representation of $A_x$. We call $\pi$ even (resp. odd) if all $\Hsp^y_z$ with $y/z \in q^{2\Z+1}$ (resp.~ $q^{2\Z}$) are zero.
\end{Def} 

%We can hence identity $\Hsp$ as a quotient of $l^2(\Z\times \Z)$. Write the images of the standard basis vectors $e_{m,n}$ of $l^2(\Z\times \Z)$ as $f_{m,n}$. Let us write $F = \{(m,n)\mid f_{m,n}\neq 0\}$.

With the above preliminaries, we can now classify the irreducible $^*$-representations of $A_x$. We first extend the terminology of Definition \ref{DefAdapt}.
% Different order of introducing?


\begin{Def} Fix $c\in \R$. For $\epsilon,\nu\in \{-,+\}$, a couple $(y,z)\in \R_+^2$ is called \emph{$c_{\epsilon,\nu}$-adapted} if the following inequality holds: \begin{equation}\label{EqAd} \ctau(q^{-\epsilon}yz^{\epsilon\nu})+\epsilon\nu c\geq 0.\end{equation} A couple $(y,z)$ is called \emph{strictly} $c_{\epsilon,\nu}$-adapted if this is a strict equality. We call $(y,z)$ \emph{$c$-adapted} if it is $c_{\epsilon,\nu}$-adapted for all $\epsilon,\nu\in \{+,-\}$. 
\end{Def} 

\begin{Def} Fix $c\in \R$. We call a subset $T\subseteq \R_+^2$ a \emph{$c$-set} if the following conditions are satisfied: 
\begin{itemize} 
\item[$\bullet$] $T$ is not empty.
\item[$\bullet$] $T$ consists of $c$-adapted points.
\item[$\bullet$] If $(y,z)\in T$ is strictly $c_{\epsilon,\nu}$-adapted, then $(q^{-\epsilon}y,q^{-\nu}z)$ is in $T$.
\end{itemize}

We say that $T$ \emph{irreducible} if it is not the disjoint union of two $c$-sets.

%Writing $\Z^2_{\even} = \{(k,l)\mid k-l \textrm{ even}\}$ and $\Z^2_{\odd} = \Z^2\setminus \Z^2_{\even}$, we call an $(x,c)$-set even or odd according to whether it lies in $\Z^2_{\even}$ or $\Z^2_{\odd}$.
\end{Def}

\begin{Def} Fix $x>0$. For $\pi$ an irreducible representation of $A_x$, a couple $(y,z)\in \Lambda_x\times \Lambda_x$ is called $\pi$-compatible if $\Hsp^y_z\neq 0$. 

For $c\in \R$, a subset $T\subseteq \Lambda_x\times \Lambda_x$ is called \emph{$(x,c)$-compatible} if there exists an irreducible representation $\pi$ of $A_x$ with $\pi(\Omega) = c$ and $T=\{(y,z)\in \Lambda_x\times \Lambda_x\mid \Hsp^{y}_{z}\neq \{0\}\}$. In this case, we say that $\pi$ is \emph{$T$-adapted}.
\end{Def}

\begin{Prop}\label{PropClassRep} A set $T\subseteq \Lambda_x\times \Lambda_x$ is an irreducible $c$-set if and only if it is an $(x,c)$-compatible set. Moreover, for any $(x,c)$-compatible set $T$ there is exactly one irreducible $^*$-representation $\pi$ of $A_x$, up to unitary equivalence, which is $T$-compatible.
\end{Prop}

\begin{proof} Assume first that $T$ is $(x,c)$-compatible, and let $\pi$ be a $T$-compatible irreducible $^*$-representation of $A_x$. If $(y,z)\in T$, then it follows from Lemma \ref{LemAmp} that $(y,z)$ is $c$-adapted. Moreover, if $(y,z)\in T$ is strictly $c_{\epsilon,\nu}$-adapted, then we have that $\|u_{\epsilon,\nu}\xi\|\neq 0$ for a non-zero $\xi\in \Hsp^y_z$, hence also $\Hsp^{q^{-\epsilon}y}_{q^{-\nu}z}\neq \{0\}$. It follows that $T$ is a $c$-set. Now if $T=T_1\cup T_2$ a disjoint union of $c$-sets, it would follow that $\pi$ restricts to the direct sum of all $\Hsp^y_z$ with $(y,z)\in T_1$, contradicting irreducibility. It follows that $T$ is an irreducible $c$-set.

Conversely, let $T$ be an irreducible $c$-set inside $\Lambda_x\times \Lambda_x$. Put $\Hsp_{\pi} = l^2(T)$ with \begin{eqnarray*} \pi(\alpha) e_{y,z} &=&  \left(\frac{\ctau(qyz)+c}{\ctau(y)\ctau(qz)}\right)^{1/2}e_{qy,qz},\\ \pi(\beta) e_{y,z} &=& \sigma_y\left(\frac{\ctau(qy/z)-c}{\ctau(y)\ctau(z/q)}\right)^{1/2} e_{qy,z/q},\end{eqnarray*} where the right hand side is considered as the zero vector when the scalar factor on the right is zero, and where $\sigma_y = -\sigma$ if $y\in xq^{2\Z+1}$ and $1$ otherwise. Note that the roots on the right hand side are well-defined precisely because $T$ is a $c$-set. 

By direct computation, using the defining commutation relations \eqref{EqqCom} and \eqref{EqDet}, we see that $\pi$ defines a $^*$-representation of $A_x$ with $\pi(\Omega) =c$. Moreover, $\pi$ is irreducible since otherwise, by Corollary \ref{CorOneDim}, $T$ would split as a disjoint union of $(x,c)$-compatible sets. Hence $T$ is an $(x,c)$-compatible set.

Now the formulas for $\pi(\alpha)$ and $\pi(\beta)$ are uniquely determined up to a unimodular gauge factor. As any non-zero $\Hsp^{y}_z$ is cyclic for $\pi$, it follows that these gauge factors are determined by their value at one component. We then easily conclude that $\pi$ is in fact the unique $T$-compatible $^*$-representation, up to unitary equivalence.
\end{proof}

What remains is to classify irreducible $c$-sets for each $c\in \R$. 

\begin{Lem}\label{LemClass2D} A set $T\subseteq \R_+^2$ is an irreducible $c$-set if and only if there exists an irreducible $-c$-set $Z_+\subseteq \R_+$ and an irreducible $c$-set $Z_-\subseteq \R_+$ such that $(y,z)\in T$ if and only if $yz\in Z_+$ and $y/z\in Z_-$.
\end{Lem} 

\begin{proof} It is immediate that $(y,z)\in \R_+^2$ is (strictly) $c_{\epsilon,\nu}$-adapted if and only if $yz^{\epsilon\nu}$ is (strictly) $(-\epsilon\nu c)_{\epsilon}$-adapted. The conclusion of the lemma then follows immediately.
\end{proof}

Combining Proposition \ref{PropClassRep} with Proposition \ref{PropClass1D} and Lemma \ref{LemClass2D}, we thus obtain a concrete description of the spectrum of $A_x$. The following pictures illustrate the form of the spectrum of $A_x$ for the case $q>0$.

\begin{figure}[h]
  \centering
\input{specx0-even.epic}
  \caption{Case $x=0$, even}
\label{figy}
\end{figure}

% Probably in figure below the middle dot at $-t_0$ should be unfilled!
\begin{figure}[h]
  \centering
\input{specx0-odd.epic}
  \caption{Case $x=0$, odd}
\label{figy}
\end{figure}


\begin{figure}[h]
  \centering
\input{specx12-even.epic}
  \caption{Case $x=\frac{1}{2}$, even}
\label{figy}
\end{figure}


\begin{figure}[h]
  \centering
  \input{specx12-odd.epic}
  
  \caption{Case $x=\frac{1}{2}$, odd}
\label{figy}
\end{figure}

% Should be latexed?
%\includepdf[pages={1}]{ScanSpec.pdf}


% Make remark on regular representation cf. Koelink-Rosengren

% Include a concrete descripition of the universal envelope of $A_x$ from this?


%More generally, 

%As a concrete instance of the example of monoidal equivalence, let $\tilde{A}$ be the generalized compact Hopf face algebra obtained from the set $\tilde{I} =I_1\sqcup I_2$ with $I_1= \Z$ and $I_2= \{\bullet\}$ with the $B_{kl} =\emptyset$ and $E(k,l)$ for $k,l\neq \bullet$ as in section ..., with $B_{k,\bullet} = B_{\bullet,k}= \emptyset$, and $B_{\bullet,\bullet} = \{\pm\}$ with $E_{\bullet,\bullet} = \begin{pmatrix} 0 & |q|^{1/2} \\ -\sgn(q)|q|^{-1/2}&0\end{pmatrix}$ (with the basis ordered as $-,+$). Then this will be obtained from the direct sum of the functor from ... and the ordinary forgetful functor from $\Rep(SU_q(2))$ into $\Hilb$. It follows that the components $\tilde{A}(ij)$ can be described by the generators and relations as in ..., but with $F(\lambda)$ and $F(\rho)$ set equal to 1 whenever the corresponding index is $\bullet$.




% Study spectrum fundamental character
% Study dual quantum groupoid
% Make connection with dynamical cocycle
% In case of qgroupoid constructed from identity functor for Rep(SU_q(2)): rep theory of associated Galois object should just be: a single representation (Galois object is type I factor, cutdown of $B(\mathscr{L}^2(SU_q(2)))$). Yes: in general, Galois object is Morita equivalent with algebra of original ergodic action, should also be stressed for Podles spheres

\section{The reduced C$^*$-algebra of the dynamical quantum $SU(2)$ group}

We now study the reduced C$^*$-algebra $\mathcal{A}_x$ associated to $(\mathscr{A}_x,\Delta)$. It will be convenient here to use the linking partial compact quantum group between $\mathscr{A}(\Gamma_x)$ and $\Pol(SU_q(2))$.  Let us denote $\widetilde{\Gamma}_x = \Gamma_x\cup \{\bullet\}$, the disconnected sum of $\Gamma_x$ with the graph $\bullet$ consisting of a single vertex and two loops denoted $+$ and $-$ with respective weights $q$ and $q^{-1}$ and respective signs $+$ and $-\sigma$. Then the partial Hopf $^*$-algebra associated to the above-mentioned linking partial compact quantum group is $\mathscr{A}(\widetilde{\Gamma}_x)$. It splits into four components $\mathscr{A}(\Lambda,\Lambda')$, where $\Lambda,\Lambda' \in \{\Gamma_x,\Gamma\}$, and with the total algebras of these components generated by the $u_{e,f}$ with $e \in \Lambda,f\in \Lambda'$. We have $\mathscr{A}(\Gamma_x,\Gamma_x) = \mathscr{A}_x$ and $\mathscr{A}(\bullet,\bullet) = \Pol(SU_q(2))$. Let us investigate the remaining components.

In fact, the total algebra $A_{x,\bullet} = A(\Gamma_x,\bullet)$ was studied in \cite{DCY1}. Following the discussion in Section \cite{SecUni}, its generators can be written as \[(u_{\epsilon,\nu})_{q^{x+k}} =  u_{(k,k-\epsilon),\nu}, \qquad k\in \Z,\epsilon,\nu \in \{+,-\}\] the base algebra being reduced to a single copy of $\Fun_{\fin}(\Lambda_x)$ whose elements we write in the form $f(\lambda)$. The remaining disussion in Section \cite{SecUni} is then just as before, except that one now puts $w_+(\rho) = q, w_-(\rho) = q^{-1}$, i.e.~ the limit $\rho\rightarrow +\infty$ is taken. A similar discussion holds for $A_{\bullet,x} = A(\bullet,\Gamma_x)$, with the roles of $\rho$ and $\lambda$ interchanged. 

Now the total comultiplication of $A(\widetilde{\Gamma}_x)$ splits into parts \[\Delta_{\Lambda,\Lambda''}^{\Lambda'}: A(\Lambda,\Lambda'')\rightarrow M(A(\Lambda,\Lambda')\otimes A(\Lambda',\Lambda'')),\quad \Lambda,\Lambda',\Lambda'' \in \{\Gamma_x,\bullet\}.\] Consider the invariant positive integral $\phi$ on $A(\widetilde{\Gamma}_x)$. Then by invariance, we have \[\phi(a) = (\phi\otimes \phi)(\Delta_{\Gamma_x,\Gamma_x}^{\bullet}(a)),\qquad a \in A(\Gamma_x).\] It follows that the reduced C$^*$-algebra of $\mathcal{A}(\Gamma_x)$ can be understood as the completion of $\Delta_{\Gamma_x,\Gamma_x}^{\bullet}(A(\Gamma_x))$ in $\mathcal{A}_{x,\bullet}\otimes \mathcal{A}_{\bullet,x}$. 

Let us thus first study the reduced C$^*$-algebraic completion $\mathcal{A}_{x,\bullet}$. Note that $\phi$ is supported on $\sum_y \lambda_y A_{x,\bullet}\lambda_y$. Now any element of $A_{x,\bullet}$ can be written as linear combination of a polynomial in the $\alpha,\beta,\gamma,\delta$ times a finite support function on $\Lambda_x$. It easily follows then that $\lambda_y A_{x,\bullet}\lambda_y$ is generated by $\beta^*\beta\lambda_y,\alpha\beta^*\lambda_y$ and $\beta\alpha^*\lambda_y$. Let us fix $y$, and write \begin{eqnarray*} X_y &=& \ctau(y/q)\alpha \beta^*\lambda_y,\\ qZ_y &=& (y^{-1}-\ctau(y)\beta^*\beta)\lambda_y\\ Y_y &=& \ctau(y/q)\beta \alpha^*\lambda_y.\end{eqnarray*}  Then using that formally $Z_y = \lim_{\rho\rightarrow +\infty} \rho^{-1}\Omega \lambda_y$, we find that $X_yZ_y = q^2Z_yX_y$, $Y_yZ_y= q^{-2}Z_yY_y$, $Z_y^*=Z_y$ and \begin{eqnarray*} X_yY_y &=& 1+ q (y^{-1}-y)Z_y -  q^2 Z_y^2 \\ Y_yX_y &=& 1+ q^{-1}(y^{-1}-y)Z_y-q^{-2}Z_y^2,
\end{eqnarray*}
where we have treated $\lambda_y$ as the unit of $\lambda_y A_{x,\bullet}\lambda_y$.

We see that we obtain a copy of a (non-standard) Podle\'{s} sphere $\mathbb{S}_{q,y}^2$. Moreover, $\Delta_{x,\bullet}^{\bullet}$ restricts to the ordinary action $SU_q(2)$ on this Podle\'{s} sphere. The associated C$^*$-algebra $C(\mathbb{S}_{q,y}^2)$ is then faithfully represented on $l^2(J_{y})$, where $J_y = q^{1+2\mathbb{N}}/y\cup -q^{2\mathbb{N}+1}y$, by $Z_ye_z = z e_{z}$ and \begin{eqnarray*} X_ye_{z} &=& \sigma_{qy} \lbrack (1-yz/q)(1+z/qy)\rbrack^{1/2}e_{q^{-2}z}\\ Y_ye_{z} &=& \sigma_{qy} \lbrack (1-qyz)(1+qz/y)\rbrack^{1/2}e_{q^2z}.\end{eqnarray*} The invariant state on the Podle\'{s} sphere is implemented by the (normalized) trace class operator associated to $|Z|$. 

Let us extend the above Podle\'{s} sphere representation to $A_{x,\bullet}$.  Let $J =\{(z,y)\mid y \in \Lambda_x, qz \in J_{y}\}$. Then we define the following representation of $A_{x,\bullet}$ on $l^2(J)$: \begin{eqnarray*} \alpha e_{y,z} &=&  \left(\frac{y+z/q}{\ctau(y)}\right)^{1/2}e_{qy,z/q}\\ \beta e_{y,z} &=&  \sigma_y \left( \frac{1/y-qz}{\ctau(y)}\right)^{1/2}e_{qy,qz},  \\  \alpha^* e_{y,z} &=&  \left(\frac{y/q+z}{\ctau(y/q)}\right)^{1/2}e_{y/q,qz} \\ \beta^* e_{y,z} &=& \sigma_{qy}\left( \frac{q/y-z}{\ctau(y/q)}\right)^{1/2}e_{y/q,z/q}
 \\ \lambda_{y'} e_{y,z} &=& \delta_{y,y'}e_{y,z}. \end{eqnarray*} It is easily seen that this is well-defined, that we get our standard representation on the $e_{z} = e_{qy,qz}$ for the Podle\'{s} sphere $\lambda_y A_{x,\bullet}\lambda_y$, and that this defines a $^*$-representation of $A_{x,\bullet}$. It is also easy to see this representation as a limit of the representations obtained in ...

% Reference to Zinn-Justin Banica paper

It is now easy to find also a $^*$-representation of $A_{\bullet,x}$. Here the relations between the generators are obtained from $A_{x}$ by letting $\lambda\rightarrow +\infty$. In fact we have \[S^2(u_{\epsilon,\nu}) = \frac{w_{\epsilon}(\lambda)}{w_{\nu}(\rho)} u_{\epsilon,\nu},\] hence the unitary antipode $R$ satisfies \[ R(u_{\epsilon,\nu}) = u_{\nu,\epsilon}^* \frac{w_{\nu}^{1/2}(\lambda)}{w_{\epsilon}^{1/2}(\rho)}\] By composing with $*$ and the obvious conjugacy operator, it follows that we have a $^*$-algebra isomorphism \[u_{\epsilon,\nu} \mapsto  \frac{w_{\nu}^{1/2}(\lambda)}{w_{\epsilon}^{1/2}(\rho)}u_{\nu,\epsilon},\] which as well gives an identification $A_{\bullet,x}\rightarrow A_{x,\bullet}$. Concretely, this gives \[\alpha \mapsto q^{1/2}w_{-}^{1/2}(\lambda) \alpha,\qquad \beta \mapsto -\sigma\beta^*.\] Hence, for $A_{\bullet,x}$ we obtain the $^*$-representation on $l^2(J_{y})$ by \begin{eqnarray*} \alpha e_{y,z} &=&  \left(\frac{qy+z}{\ctau(qy)}\right)^{1/2}e_{qy,z/q}\\ \beta e_{y,z} &=&  - \sigma_{y} \left( \frac{q/y-z}{\ctau(y/q)}\right)^{1/2}e_{y/q,z/q}\\ \rho_{y'}e_{y,z} &=& \delta_{y',y}e_{y,z}.\end{eqnarray*}

Finally, let us consider then the $^*$-representation of $A_{x}$ on $l^2(J)\boxtimes l^2(J)$, which is just $l^2(J)\otimes l^2(J)$. Let us identify the basis of the latter by $e_{y,z,y',z'} = e_{y,z}\otimes e_{y',z'}$. Then \begin{eqnarray*} \alpha e_{y,z,y',z'} &=& \left(\frac{(y+z/q)(qy'+z')}{\tau(y)\tau(qy')}\right)^{1/2}e_{qy,z/q,qy',z'/q}  \\ && \qquad + \left(\frac{(1/y-qz)(1/qy' -z')}{\tau(y)\tau(qy')}\right)^{1/2}e_{qy,qz,qy',qz'}, \\
\beta e_{y,z,y',z'} &=&  \sigma_y \left(\frac{(1/qy-z)(y'+qz')}{\tau(y)\tau(y'/q)}\right)^{1/2}e_{qy,qz,y'/q,qz'}  \\ &&\qquad -\sigma_{y'}\left(\frac{(y+z/q)(q/y'-z')}{\tau(y)\tau(y'/q)}\right)^{1/2} e_{qy,z/q,y'/q,z'/q}
\end{eqnarray*}

We see that there is an obvious observable in the commutant, namely the quantity $z/z'$. 

Fix now $y,y'$, and write $e_{z,z'} = e_{y,z,y',z'}$. Then writing $\tilde{\Omega} =  \sigma_y\sigma_y'\tau(y)\tau(y'/q) \Omega$, we have \begin{eqnarray*}  \tilde{\Omega} e_{z,z'} &=& \lbrack (1/qy-z)(y'+qz')(y+qz)(q/y'-q^2z')\rbrack^{1/2} e_{q^2z,q^2z'} \\ && + \sigma_{yy'} \left(\tau(qy/y')\tau(y)\tau(y'/q) - (1/qy-z)(y'+qz') - (y+z/q)(q/y'-z')\right) e_{z,z'} \\ && + \lbrack (1/qy-z/q^2)(y'+z'/q)(y+z/q)(q/y'-z')\rbrack^{1/2} e_{q^{-2}z,q^{-2}z'}
\end{eqnarray*}



%%% Local Variables: 
%%% mode: latex
%%% TeX-master: "dyn-suq-main"
%%% End: 


\bibliographystyle{abbrv}
\bibliography{referencesSUq}
% \begin{thebibliography}{99}
% \bibitem{AN1} N. Andruskiewitsch and S. Natale, Double categories and quantum groupoids, \emph{Publ. Mat. Urug.} \textbf{10}, 11--51 (2005).
% \bibitem{BDV1} J. Bichon, A. De Rijdt and S. Vaes, Ergodic coactions with large multiplicity and monoidal equivalence of quantum groups, \emph{Comm. Math. Phys.} \textbf{262} (2006), 703--728.
% \bibitem{Boh1}  G. B\"{o}hm, J. Gómez-Torrecillas and E. López-Centella, Weak multiplier bialgebras, \emph{Trans. Amer. Math. Soc.}, in press., arXiv:1306.1466. 
% \bibitem{Dau1} J. Dauns, Multiplier rings and primitive ideals, \emph{Trans. Amer. Math. Soc} \textbf{145} (1969), 124--158.
% \bibitem{DCY1} K. De Commer and M. Yamashita, Tannaka-Kre\u{\i}n duality for compact quantum homogeneous spaces II. Classification of quantum homogeneous spaces for quantum $SU(2)$, J. Reine Angew. Math., DOI: 10.1515/crelle-2013-0074 (2013).
% \bibitem{Eti1} P. Etingof and V. Ostrik, Module categories over representations of $\SSL_q(2)$ and graphs, \emph{Math. Res. Lett.} \textbf{11} (1) (2004), 103--114.
% \bibitem{Hay1} T. Hayashi, Compact Quantum Groups of Face Type, \emph{PRIMS} \textbf{32} (1996), 351--369.
% \bibitem{KoR1}  E. Koelink and H. Rosengren, Harmonic Analysis on the $SU(2)$ Dynamical Quantum Group, \emph{Acta Applicandae Mathematica} \textbf{69} (2) (2001), 163--220.
% \bibitem{Pin2} C. Pinzari, The representation category of the Woronowicz quantum group $S_{\mu}U(d)$ as a braided tensor C$^*$-category, \emph{Int. J. Math.} \textbf{18} (2) (2007), 113--136.
% \bibitem{Pin3} C. Pinzari and J.E. Roberts, Ergodic actions of compact quantum groups from solutions of the conjugate equations, preprint (2008) {\tt arXiv:0808.3326 [math.OA]}.
% \bibitem{Tur1} V. Turaev, Quantum invariants of knots and 3-manifolds, \emph{de Gruyter Studies in Mathematics} \textbf{18}, Walter de Gruyter \& Co., Berlin (1994).
% \bibitem{VDae1} A. Van Daele, Multiplier Hopf algebras, \emph{Trans. Amer. Math. Soc.} \textbf{342} (1994), 917--932.
% \bibitem{VDae2} A. Van Daele, An algebraic framework for group duality, \emph{Adv. in Math.} \textbf{140} (1998), 323--366.
% \bibitem{VDW2} A. Van Daele and  S. Wang, Weak multiplier Hopf algebras. Preliminaries, motivation and basic examples, \emph{Banach Center Publ.} \textbf{98} (2012), 367--415.
% \bibitem{VDW1} A. Van Daele and S. Wang: Weak multiplier Hopf algebras I. The main theory, \emph{Preprint University of Leuven and Southeast University of Nanjing} (2012), to appear in
% \emph{Crelles Journal}, {\tt arXiv:math/1210.4395 [math.RA]}.
% \bibitem{Yam1} S. Yamagami, A categorical and diagrammatical approach to Temperley--Lieb algebras, preprint (2004) {\tt arXiv:math/0405267 [math.QA]}.
% \bibitem{Wor1} S.L. Woronowicz, Twisted $\mathrm{SU}(2)$ group. An example of a non-commutative differential calculus, \emph{Publ. Res. Inst. Math. Sci.} \textbf{23} (1) (1987), 117--181.
% \end{thebibliography}

\end{document}
