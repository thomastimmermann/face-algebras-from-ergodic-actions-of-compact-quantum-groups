\section{Tannaka-Krein-Woronowicz for partial compact quantum groups}

% Assume categories small?
% See `Galois reconstruction of finite quantum groups' of Bichon for additional refs

In the previous section, we showed how any partial compact quantum group gave rise to a partial semi-simple tensor C$^*$-category with duality and indecomposable units, together with a morphism into a partial tensor C$^*$-category of finite dimensional Hilbert spaces. In this section, we reverse this construction, and show that the two structures are in duality with each other. As the proof does not differ too much from the usual Tannaka-Krein reconstruction process, we do not spell out all details, emphasizing only the novel points having to do with well-definedness of the constructions. 

In the following, let us at first fix an indecomposable semi-simple partial tensor category $\CatCC$ over a base set $I_0$. We will again view the tensor product of $\CatC$ as being strict, for notational convenience. 

Assume that we also have another set $I$ and a partition $I = \{I_k\mid k\in I_0\}$ with associated function \[\varphi_0:I\rightarrow I_0, \quad k\mapsto k'.\] Let $\Forget: \CatCC\rightarrow \{\Vect_{\fin}\}_{I\times I}$ be a morphism based on $\varphi_0$ (cf.a~ Example \ref{ExaVectBiGr}).  We will again denote by $\Forget_{kl}:\CatCC_{k'l'}\rightarrow \Vect_{\fin}$ the components of $\Forget$ at index $(k,l)$, and by $\iota$ and $\mu$ resp.~ the associativity and unit constraints.  For $X\in \CatC_{k's}$ and $Y\in \CatC_{sm'}$, we write the projection maps associated to the identification $\Forget_{km}(X\otimes Y)\cong \oplus_{l\in I_s} \left(\Forget_{kl}(X)\otimes \Forget_{lm}(Y)\right)$ as \[\pi^{(klm)}_{X,Y}:\Forget_{km}(X\otimes Y) \rightarrow \Forget_{kl}(X)\otimes \Forget_{lm}(Y).\]

We choose a maximal family of mutually inequivalent irreducible objects $\{u_a\}_{a\in \mathcal{I}}$ in $\CatC$. We assume that the $u_a$ include the unit objects $\Unit_{k'}$ for $k'\in I_0$, and we may hence identify $I_0\subseteq \mathcal{I}$. For $a\in \mathcal{I}$, we will write $u_a \in \CatC_{s_a,t_a}$ with $s_a,t_a\in I_0$.

\begin{Def} For $a\in \mathcal{I}$, $k,l,m,n\in I$ with $k'=m'=a_s,l'=n'=a_t$, define vector spaces \[\Gr{A}{k}{l}{m}{n}(a) = \Hom_{\C}(\Forget(u_a)_{kl},\Forget(u_a)_{mn})^*.\] Write \[\Gr{A}{k}{l}{m}{n} = \underset{a\in \mathcal{I}}{\oplus} \Gr{A}{k}{l}{m}{n}(a),\qquad A = \underset{k,l,m,n}{\oplus} \Gr{A}{k}{l}{m}{n}.\] The $a$-spectral subspace $A(a)$ of $A$ is defined as \[A(a) = \osum{k,l,m,n} \Gr{A}{k}{l}{m}{n}(a).\] For any element $x\in A$, its component in the $a$-spectral subspace is written $x_a$.
\end{Def} 

Note that by construction, $\Gr{A}{k}{l}{m}{n}=0$ if $k'\neq m'$ or $l'\neq n'$. 

We first turn the $\Gr{A}{k}{l}{m}{n}$ into a partial coalgebra $\mathscr{A}$ over $I^2$.

\begin{Def} For $r,s\in I$, we define \[\Delta_{rs}: \Gr{A}{k}{l}{m}{n}\rightarrow \Gr{A}{k}{l}{r}{s}\otimes \Gr{A}{r}{s}{m}{n}\] as the direct sums of the duals of the composition maps \[\Hom_{\C}(\Forget(u_a)_{kl},\Forget(u_a)_{rs}) \otimes \Hom_{\C}(\Forget(u_a)_{rs},\Forget(u_a)_{mn})\rightarrow \Hom_{\C}(\Forget(u_a)_{kl},\Forget(u_a)_{mn}),\]\[f\otimes g \mapsto g\circ f.\]
\end{Def} 

\begin{Lem} The couple $(\mathscr{A},\Delta)$ is a partial coalgebra with counit map \[\epsilon:\Gr{A(a)}{k}{l}{k}{l}\rightarrow \C,\quad x\mapsto x(\id_{\Forget(u_a)_{kl}}).\] Moreover, for each fixed $x\in \Gr{A(a)}{k}{l}{m}{n}$, the matrix $\left(\Delta_{rs}(x)\right)_{rs}$ is row- and column-finite.
\end{Lem} 
\begin{proof} Coassociativity and counitality are immediate by duality, as for each $a$ fixed the $\Hom_{\C}(\Forget(u_a)_{kl},\Forget(u_a)_{mn})$ form a partial algebra with units $\id_{F(u_a)_{kl}}$. The row-and column-finiteness condition follows immediately from the local finiteness condition for the morphism $F$.
\end{proof}

In the next step, we define a partial algebra structure on $\mathscr{A}$. % Slight misuse of notation, $\mathscr{A}$ already the coalgebra.
First note that we can identify $\Nat(\Forget_{kl},\Forget_{mn}) \cong \prod_a \Hom_{\C}(\Forget(u_a)_{kl},\Forget(u_a)_{mn})$. Similarly, we can identify \[\Nat(\Forget_{kl}\otimes \Forget_{rs},\Forget_{mn}\otimes \Forget_{pq}) \cong  \prod_{b,c} \Hom_{\C}(\Forget(u_b)_{kl}\otimes \Forget(u_c)_{rs} ,\Forget(u_b)_{mn}\otimes \Forget(u_c)_{pq}),\] where \[\Forget_{kl}\otimes \Forget_{rs}:\CatC_{k'l'}\times \CatC_{r's'}\rightarrow \Vect_{\fin},\quad X\times Y \mapsto F_{kl}(X)\otimes F_{rs}(Y).\] As such, there is a natural paring of these spaces with resp.~ $\Gr{A}{k}{l}{m}{n}$ and $\Gr{A}{k}{l}{m}{n}\otimes \Gr{A}{r}{s}{p}{q}$. 

% To expand. 

\begin{Def} We define a product map $\Gr{A}{k}{l}{r}{s} \otimes \Gr{A}{l}{m}{s}{t}\rightarrow  \Gr{A}{k}{m}{r}{t}$ by the formula \[(x\cdot y)(n) = (x\otimes y)( \hat{\Delta}^{l}_{s}(n)), \qquad  n \in \Nat(\Forget_{km},\Forget_{rt}),\] where $\hat{\Delta}^l_s(n)$ is the natural transformation\[\hat{\Delta}^l_s(n):  \Forget_{kl}\otimes \Forget_{lm}\rightarrow \Forget_{rs}\otimes \Forget_{st},\quad \hat{\Delta}^l_s(n)(X,Y) = \pi^{(rst)}n_{X\otimes Y} \iota^{(klm)},\quad X\in \CatC_{k'l'},Y\in \CatC_{l'm'}.\]
\end{Def}

Note that we assumed here implicitly that $k'=r', l'=s'$ and $m'=t'$.

\begin{Lem} The above product maps turn the $\Gr{A}{k}{l}{m}{n}$ into a partial algebra $\mathscr{A}$.
\end{Lem}
\begin{proof} The associativity follows from the 2-cocycle condition for the $\iota$ and $\pi$-maps, coupled with the naturality of $n$. The units are given by $\un{k}{m}\in \Gr{A}{k}{k}{m}{m}(\Unit_{k'})$  (for $k'=m'$) corresponding to $1$ in the canonical identifications  \[\Gr{A}{k}{k}{m}{m}(\Unit_{k'}) \cong \End_{\C}(\Forget_{km}(1_{k'}))^* \cong \C^* \cong \C.\]
\end{proof} 

\begin{Prop} The partial algebra and coalgebra structures on $\mathscr{A}$ define a partial bialgebra structure on $\mathscr{A}$. 
\end{Prop}
\begin{proof} Let us check the properties in Definition \ref{DefPartBiAlg}. Properties \ref{Propa} and \ref{Propc} are immediate. Property \ref{Propd} was proven above. Property \ref{Propb} follows from the fact that \[\hat{\Delta}^{l}_s(\id_{F_{km}}) = \delta_{ls} \id_{F_{kl}}\otimes \id_{F_{lm}}.\] % To check more rigorously.
It remains to show the multiplicativity property \ref{Prope}. This follows from the fact that $\oplus_s \pi^{(rst})\iota^{(rst)} \cong \id_{F_{rt}}$. 
\end{proof} 

\begin{Theorem}\label{TheoTKBialg} The assigment $\mathscr{A}\rightarrow \Corep(\mathscr{A})$ provides a one-to-one correspondence between cosemi-simple partial bialgebras $\mathscr{A}$ based over $(I,I_0,\varphi_0)$ % New terminology
and semi-simple $I_0$-partial tensor categories $\CatCC$ with forgetful strongly monoidal functor $\Forget$ to $\Vect_f^{I\times I}$ based over $\varphi_0$. % To reformulate.
\end{Theorem}

Let us give some more info on the associated invariant integral.

\begin{Lem} Under the above correspondence, the invariant integral on $\mathscr{A}$ is given by ...

\end{Lem}

The following proposition upgrades Theorem \ref{TheoTKBialg} under the presence of duality.

\begin{Prop} The partial bialgebra $\mathscr{A}$ admits an antipode if and only if $\Corep(\mathscr{A})$ admits duality.
\end{Prop} 

Finally, we introduce some analytic structure. Let us say $\CatCC$ is a semisimple partial C$^*$-category if all $\CatCC$ are semisimple C$^*$-categories, and the associativity and unit constraints are unitary. For example, for a set $I$ we can form the semisimple partial C$^*$-category $\Hilb_{\fin}^{I\times I}$ where each block is a copy of $\Hilb_{\fin}$, the C$^*$-category of finite-dimensional Hilbert spaces, with tensor products as usual. Duality is defined as in the absence of a C$^*$-structure, but now we automatically have also right duals because of the presence of the $^*$-operation. A \emph{morphism} between semisimple partial tensor C$^*$-categories is then simply a collection of $^*$-functors for which the connecting maps $\iota^{rst}$ are isometric (and the unit constraints unitary). 

Let us now fix a semisimple $I_0$-partial tensor C$^*$-category $\CatCC$ admitting duality and with forgetful functor $\Forget$ into $\Hilb_{\fin}^{I\times I}$ based over $I_0$. Then we can of course form the cosemisimple partial Hopf algebra $(\mathscr{A},\Delta)$ as before. Our next lemma shows that $\mathscr{A}$ acquires a $^*$-structure. To distinguish it from the $^*$-operations present on the C$^*$-category, we will denote it by $\dagger$.

\begin{Def} We define $^\dag: \Gr{A}{k}{l}{m}{n}\rightarrow \Gr{A}{l}{k}{n}{m}$ by ...

\end{Def}

\begin{Theorem} \label{TheoTKPCQG} The assigment $\mathscr{A}\rightarrow \Corep(\mathscr{A})$ provides a one-to-one correspondence between partial compact quantum groups based over $(I,I_0,\varphi_0)$ % New terminology
and semi-simple $I_0$-partial tensor C$^*$-categories $\CatCC$ with forgetful strongly monoidal $^*$-functor $\Forget$ to $\Vect_f^{I\times I}$ based over $\varphi_0$. % To reformulate.
\end{Theorem} 

\begin{proof}

\end{proof}


Let us now present some examples.

\begin{Exa} Let $I$ be a set. The category $\Hilb_f^{I\times I}$ of row and column-finite $I\times I$-graded Hilbert spaces with tensor product $\itimes$ forms a tensor W$^*$-category with duals and decomposable unit. 
\end{Exa}

\begin{Exa} Morita equivalence.
\end{Exa}

Let $(A,\Delta)$ be a generalized Hopf face algebra over a set $I$. Assume that $I = I_1\sqcup I_2$, and let $\Lambda_j = \sum_{i\in I_j}\lambda_i$, resp. $\Rho_j = \sum_{i\in I_j} \rho_j$. If the $\Lambda_j$ and $\Rho_j$ are central in $M(A)$, then we can write $A = \osum{i,j} A(ij)$ where $A(ij) = \Lambda_i\Rho_jA$ are subalgebras. Moreover, the comultiplication $\wDelta$ splits into comultiplications \[\wDelta_{ij}^k:A(ij)\rightarrow M(A(ik)\otimes A(kj))\textrm{ s.t. } \wDelta = \wDelta_{ij}^1 +\wDelta_{ij}^2 \textrm{ on }A(ij).\] A similar decomposition holds for $\Delta$.

It is immediate to see that the $(A(ii),\Delta_{ii}^i)$ are two generalized Hopf face algebras over the respective $I_i$.

\begin{Def} We say $(A,\Delta)$ is a \emph{co-linking generalized (compact) Hopf face algebra} between $(A(11),\Delta_{11}^1)$ and $(A(22),\Delta_{22}^2)$ if $\lambda_i\Rho_2\neq 0$ for any $i\in I_1$.
\end{Def}

Upon applying the antipode, we see that then $\rho_j\Lambda_1\neq 0$ for any $j\in I_2$ as well.

\begin{Def} Two generalized (compact) Hopf face algebras are called \emph{comonoidally Morita equivalent} if they are isomorphic to the components $(A_{ii},\Delta_{ii}^i)$ of some co-linking generalized (compact) Hopf face algebra.\end{Def}

As an example, consider two sets $I_i$, and two tensor functors $(F_i,\phi_i)$ of a semi-simple rigid C$^*$-category $\CatC$ with irreducible unit into $\Hilb_{I_i^2}$. Then with $I= I_1\sqcup I_2$, we can form a new C$^*$-functor $F=F_1\oplus F_2$ of $\CatC$ into $\Hilb_{I^2}$ by putting $F(X) = F_1(X)\oplus F_2(X)$ with the obvious $I^2$-grading (and the obvious direct sum operation on morphisms). It becomes monoidal by means of the unitaries \[F(X\otimes Y) = F_1(X\otimes Y)\oplus F_2(X\otimes Y) \underset{\phi_1\oplus \phi_2}{\cong} (F_1(X)\underset{I_1}{\otimes} F_1(Y)) \oplus (F_2(X)\otimes F_2(Y)) \cong F(X)\itimes F(Y)\] (where the last map is unitary since $(F(X)\itimes F(Y))_{ij}=0$ for example for $i\in I_1$ and $j\in I_2$).

If we then consider the generalized compact Hopf face algebra $(A,\Delta)$ associated to $F$, we have immediately from the construction that the $\Lambda_i$ and $\Rho_i$ associated to the decomposition $I = I_1\sqcup I_2$ are indeed central elements in $M(A)$. Moreover, the parts $(A_{ii}^i,\Delta_{ii}^i)$ are seen to arise from applying the Tannaka-Krein construction to the respective functors $F_1$ and $F_2$. The fact that $(A,\Delta)$ is co-linking is immediate from the fact that \emph{none} of the $\lambda_i\rho_j$ are zero in this particular case (since $\Gr{A}{k}{k}{m}{m}(o) = B(F(u_o)_{kk},F(u_o)_{mm}) \cong \C$).

We will exploit the above extra structure in the following section to say something about the algebra $A$ appearing in ... This is the component $\tilde{A}(1,1)$ of the above algebra. The following lemma will be needed.


\begin{Lem} Assume $(A,\Delta)$ is a co-linking generalized Hopf face algebra. Then any of the maps $\wDelta_{ij}^k$ is injective.\end{Lem}

\begin{proof} Take a non-zero $x\in A_n(ij)$ where $n\in I_j$. Then for any $l\in I$ with $\rho_n\lambda_l\neq 0$, we know that $\wDelta(x)(1\otimes \rho_n\lambda_l)\neq 0$. Hence $\wDelta_{ij}^k(x)(1\otimes \rho_n\lambda_l)\neq 0$ for $l\in I_k$, and hence $\wDelta_{ij}^k(x)\neq 0$. Now if $j=k$, the condition $\rho_n\lambda_l\neq 0$ is satisfied by taking $l=n$ (since $\varepsilon(\lambda_n\rho_n)=1$). If $j\neq k$, it is satisfied for at least one $l$ by the co-linking assumption.
\end{proof}



\begin{Exa} Weak Morita equivalence. 
\end{Exa}
% Concrete implementation, helps to define the concept in more general analytic context.




%%% Local Variables: 
%%% mode: latex
%%% TeX-master: "dyn-suq-main"
%%% End: 

