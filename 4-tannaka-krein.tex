% Corep should read Corep_u$ further on

\section{Tannaka-Krein-Woronowicz duality for partial compact quantum groups}

% Assume categories small?
% See `Galois reconstruction of finite quantum groups' of Bichon for additional refs
% Any faithfulness conditions necessary on forgetful functor? 

In the previous section, we showed how any partial compact quantum group gave rise to a partial semisimple tensor C$^*$-category with duality and indecomposable units, together with a morphism into a partial tensor C$^*$-category of finite dimensional Hilbert spaces. In this section, we reverse this construction, and show that the two structures are in duality with each other. As the proof does not differ too much from the usual Tannaka-Krein reconstruction process, we do not spell out all details, emphasizing only the novel points having to do with well-definedness of the constructions.  % Say that implicitly, we do two steps at once: create a partial discrete quantum group, and then dualize. But we don't formally introduce partial discrete quantum groups.

In the following, let us at first fix an indecomposable semi-simple partial tensor category $\CatCC$ over a base set $I_0$. We will again view the tensor product of $\CatC$ as being strict, for notational convenience. 

Assume that we also have another set $I$ and a partition $I = \{I_k\mid k\in I_0\}$ with associated function \[\varphi_0:I\rightarrow I_0, \quad k\mapsto k'.\] Let $\Forget: \CatCC\rightarrow \{\Vect_{\fin}\}_{I\times I}$ be a morphism based on $\varphi_0$ (cf.a~ Example \ref{ExaVectBiGr}).  We will again denote by $\Forget_{kl}:\CatCC_{k'l'}\rightarrow \Vect_{\fin}$ the components of $\Forget$ at index $(k,l)$, and by $\iota$ and $\mu$ resp.~ the associativity and unit constraints.  For $X\in \CatC_{k's}$ and $Y\in \CatC_{sm'}$, we write the projection maps associated to the identification $\Forget_{km}(X\otimes Y)\cong \oplus_{l\in I_s} \left(\Forget_{kl}(X)\otimes \Forget_{lm}(Y)\right)$ as \[\pi^{(klm)}_{X,Y}:\Forget_{km}(X\otimes Y) \rightarrow \Forget_{kl}(X)\otimes \Forget_{lm}(Y).\]

We choose a maximal family of mutually inequivalent irreducible objects $\{u_a\}_{a\in \mathcal{I}}$ in $\CatC$. We assume that the $u_a$ include the unit objects $\Unit_{k'}$ for $k'\in I_0$, and we may hence identify $I_0\subseteq \mathcal{I}$. For $a\in \mathcal{I}$, we will write $u_a \in \CatC_{s_a,t_a}$ with $s_a,t_a\in I_0$. For $s,t\in I_0$ fixed, we write $\mathcal{I}_{rs}$ for the set of all $a\in \mathcal{I}$ with $s_a=s$ and $t_a=t$. When $a,b,c\in \mathcal{I}$ with $a\in \mathcal{I}_{rs},b\in \mathcal{I}_{st}$ and $c\in \mathcal{I}_{rt}$, we write $c\leq a\cdot b$ if $\Mor(u_c,u_a\otimes u_b)\neq \{0\}$. Note that with $a,b$ fixed, there is only a finite set of $c$ with $c\leq a\cdot b$. We also use this notation for multiple products.

\begin{Def} For $a\in \mathcal{I}$ and $k,l,m,n\in I$, define vector spaces \[\Gr{A}{k}{l}{m}{n}(a) =  \delta_{k,m,s_a}\delta_{l,n,t_a} \Hom_{\C}(\Forget(u_a)_{mn},\Forget(u_a)_{kl})^*.\] Write \[\Gr{A}{k}{l}{m}{n} =\underset{a\in \mathcal{I}}{\oplus}\, \Gr{A}{k}{l}{m}{n}(a),\quad A(a) = \underset{k,l,m,n}{\oplus} \Gr{A}{k}{l}{m}{n}(a),\quad A = \underset{k,l,m,n}{\oplus} \Gr{A}{k}{l}{m}{n}.\] 
%For any element $x\in A$, its component in the $a$-spectral subspace is written $x_a$.
\end{Def} 

%Note that by construction, $\Gr{A}{k}{l}{m}{n}=0$ if $k'\neq m'$ or $l'\neq n'$. 

We first turn the $\Gr{A}{k}{l}{m}{n}$ into a partial coalgebra $\mathscr{A}$ over $I^2$.

\begin{Def} For $r,s\in I$, we define \[\Delta_{rs}: \Gr{A}{k}{l}{m}{n}\rightarrow \Gr{A}{k}{l}{r}{s}\otimes \Gr{A}{r}{s}{m}{n}\] as the direct sums of the duals of the composition maps \[\Hom_{\C}(\Forget(u_a)_{rs},\Forget(u_a)_{kl}) \otimes \Hom_{\C}(\Forget(u_a)_{mn},\Forget(u_a)_{rs})\rightarrow \Hom_{\C}(\Forget(u_a)_{mn},\Forget(u_a)_{kl}),\]\[x\otimes y \mapsto x\circ y.\]
\end{Def} 

\begin{Lem} The couple $(\mathscr{A},\Delta)$ is a partial coalgebra with counit map \[\epsilon:\Gr{A(a)}{k}{l}{k}{l}\rightarrow \C,\quad f\mapsto f(\id_{\Forget(u_a)_{kl}}).\] Moreover, for each fixed $f\in \Gr{A(a)}{k}{l}{m}{n}$, the matrix $\left(\Delta_{rs}(f)\right)_{rs}$ is row- and column-finite.
\end{Lem} 
\begin{proof} Coassociativity and counitality are immediate by duality, as for each $a$ fixed the $\Hom_{\C}(\Forget(u_a)_{mn},\Forget(u_a)_{kl})$ form a partial algebra with units $\id_{F(u_a)_{kl}}$. The row-and column-finiteness condition follows immediately from the local finiteness condition for the morphism $F$.
\end{proof}

In the next step, we define a partial algebra structure on $\mathscr{A} = \{\Gr{A}{k}{l}{m}{n}\mid k,l,m,n\}$. First note that we can identify \[\Nat(\Forget_{mn},\Forget_{kl}) \cong \underset{t_a=l'=n'}{\underset{s_a=k'=m'}{\prod_a}} \Hom_{\C}(\Forget(u_a)_{mn},\Forget(u_a)_{kl}),\] where $\Nat(\Forget_{mn},\Forget_{kl})$ denotes the space of natural transformations from $\Forget_{mn}$ to $\Forget_{kl}$ when $k'=m'$ and $l'=n'$. Similarly, we can identify \[\Nat(\Forget_{mn}\otimes \Forget_{pq},\Forget_{kl}\otimes \Forget_{rs}) \cong  \prod_{b,c} \Hom_{\C}(\Forget(u_b)_{mn}\otimes \Forget(u_c)_{pq} ,\Forget(u_b)_{kl}\otimes \Forget(u_c)_{rs}),\] with the product over the appropriate index set and where \[\Forget_{kl}\otimes \Forget_{rs}:\CatC_{k'l'}\times \CatC_{r's'}\rightarrow \Vect_{\fin},\quad (X,Y) \mapsto F_{kl}(X)\otimes F_{rs}(Y).\] As such, there is a natural pairing of these spaces with resp.~ $\Gr{A}{k}{l}{m}{n}$ and $\Gr{A}{k}{l}{m}{n}\otimes \Gr{A}{r}{s}{p}{q}$. 

% To expand. 

\begin{Def} For $k'=r', l'=s'$ and $m'=t'$, we define a product map \[M:\Gr{A}{k}{l}{r}{s} \otimes \Gr{A}{l}{m}{s}{t}\rightarrow  \Gr{A}{k}{m}{r}{t},\quad f\otimes g \mapsto f\cdot g\] by the formula \[(f\cdot g)(x) = (f\otimes g)( \hat{\Delta}^{l}_{s}(x)), \qquad  x \in \Nat(\Forget_{rt},\Forget_{km}),\] where $\hat{\Delta}^l_s(n)$ is the natural transformation\[\hat{\Delta}^l_s(x):  \Forget_{rs}\otimes \Forget_{st}\rightarrow \Forget_{kl}\otimes \Forget_{lm},\quad \hat{\Delta}^l_s(x)_{X,Y} = \pi^{(klm)}_{X,Y}x_{X\otimes Y} \iota^{(rst)}_{X,Y},\quad X\in \CatC_{k'l'},Y\in \CatC_{l'm'}.\]
\end{Def}

\begin{Rem} It has to be argued that $f\cdot g$ has finite support (over $\mathcal{I})$ as a functional on $\Nat(\Forget_{rt},\Forget_{km})$. In fact, if $f$ is supported at $b\in \mathcal{I}_{r's'}$ and $g$ at $c\in \mathcal{I}_{s't'}$, then $f\cdot g$ has support in the finite set of $a\in \mathcal{I}_{r't'}$ with $a\leq b\cdot c$, since if $x$ is a natural transformation with support outside this set, one has $x_{u_b\otimes u_c}=0$, and hence any of the $\left(\hat{\Delta}^l_s(x)\right)_{u_b,u_c} =0$.
\end{Rem}

\begin{Lem} The above product maps turn $(\mathscr{A},M)$ into an $I^2$-partial algebra.
\end{Lem}
\begin{proof} We can extend the map $(\hat{\Delta}^l_s\otimes \id)$ on $\Nat(\Forget_{rt},\Forget_{km})\otimes \Nat(\Forget_{tu},\Forget_{mn})$ to a map \[(\hat{\Delta}^l_s\otimes \id): \Nat(\Forget_{rt}\otimes \Forget_{tu},\Forget_{km}\otimes \Forget_{mn}) \rightarrow  \Nat(\Forget_{rs}\otimes \Forget_{st}\otimes \Forget_{tu},\Forget_{kl}\otimes \Forget_{lm}\otimes \Forget_{mn}),\] \[(\hat{\Delta}^l_s\otimes \id)(x)_{X,Y,Z} = \left(\pi^{(klm)}_{X,Y}\otimes \id_{\Forget_{mn}(Z)}\right) x_{X\otimes Y, Z} \left(\iota^{(rst)}_{X,Y} \otimes \id_{\Forget_{tu}(Z)}\right).\]
By finite support, we then have that \[((f\cdot g)\cdot h)(x) = (f\otimes g\otimes h)(\hat{\Delta}^l_s\otimes \id)\hat{\Delta}^m_t(x))\] for all $f\in \Gr{A}{k}{l}{r}{s},g\in \Gr{A}{l}{m}{s}{t},h\in \Gr{A}{m}{n}{t}{u}$ and $x\in  \Nat(\Forget_{ru},\Forget_{kn})$. Similarly, \[(f\cdot g)\cdot h)(x) = (f\otimes g\otimes h)((\id\otimes \hat{\Delta}^m_t)\hat{\Delta}^l_s(x).\] The associativity then follows from the 2-cocycle condition for the $\iota$- and $\pi$-maps. 

By a similar argument, one sees that the (non-zero) units are given by $\UnitC{k}{l}\in \Gr{A}{k}{k}{l}{l}(\Unit_{r})$  (for $r=k'=l'$) corresponding to $1$ in the canonical identifications  \[\Gr{A}{k}{k}{l}{l}(r) \cong \Hom_{\C}(\Forget_{ll}(\Unit_{r}),\Forget_{kk}(\Unit_{r}))^*\cong \Hom_{\C}(\C,\C)^*  \cong \C^* \cong \C.\] Indeed, to prove for example the right unit property, we use that (essentially) $\pi_{u_a,\Unit_{r}}^{(kll)} =(\id\otimes \mu_l)$ and $\iota_{u_a,\Unit_{r}}^{(kll)} = (\id\otimes \mu_l^{-1})$, while \[\UnitC{k}{l}(\mu_kx_{\Unit_{r}}\mu_l^{-1}) = x_{\Unit_{r}} \in \C,\quad x\in \Nat(F_{ll},F_{kk}).\] % to complete
\end{proof} 

\begin{Prop} The partial algebra and coalgebra structures on $\mathscr{A}$ define a partial bialgebra structure on $\mathscr{A}$. 
\end{Prop}
\begin{proof} Let us check the properties in Definition \ref{DefPartBiAlg}. Properties \ref{Propa} and \ref{Propc} are left to the reader. Property \ref{Propd} was proven above. Property \ref{Propb} follows from the fact that for $k'=l'=s'=m'$, \[\hat{\Delta}^{l}_s(\id_{F_{km}}) = \delta_{ls} \id_{F_{kl}}\otimes \id_{F_{lm}}.\] 
It remains to show the multiplicativity property \ref{Prope}. This is equivalent with proving that, for each $x\in \Nat(F_{uw},F_{km})$ and $y\in \Nat(F_{rt},F_{uw})$ (with all first or second indices in the same class of $I_0$), one has (pointwise) that (for $l'=s'$) \[ \hat{\Delta}^l_s(x\circ y) = \sum_{v,v'=l'} \hat{\Delta}^v_s(x)\circ \hat{\Delta}^l_v(y).\] This follows from the fact that $\sum_v \pi^{(uvw)}_{X,Y}\iota^{(uvw)}_{X,Y} \cong \id_{\Forget_{uw}(X\otimes Y)}$ (where we again note that the left hand side sum is in fact finite).
\end{proof} 

Let us show now that the resulting bialgebra $\mathscr{A}$ has an invariant integral.

\begin{Def} Define $\phi: \Gr{A}{k}{k}{m}{m} \rightarrow \C$ as the functional which is zero on $\Gr{A}{k}{k}{m}{m}(a)$ with $a\neq \Unit_{k'}$, and the canonical identification $\Gr{A}{k}{k}{m}{m}(k')\cong \C$ on the unit component (for $k'=m'$).
\end{Def}

\begin{Lem} The functional $\phi$ is an invariant integral.
\end{Lem}

\begin{proof} The normalisation condition $\phi(\UnitC{k}{k})=1$ is immediate by construction. Let us check left invariance - right invariance will follow similarly.

Let $\hat{\phi}^k_l$ be the natural transformation from $\Forget_{ll}$ to $\Forget_{kk}$ which has support on multiples of $\Unit_{k'}$, and with $(\hat{\phi}^k_l)_{\Unit_{k'}} = 1$.  Then for $f\in \Gr{A}{k}{k}{l}{l}$, we have $\phi(f) = f(\hat{\phi}^k_l)$. The left invariance of $\phi$ then follows from the easy verification that for $x\in \Nat(F_{ll},F_{kn})$, \[x\circ \hat{\phi}^l_m =\delta_{k,n} \UnitC{k}{l}(y)\hat{\phi}^k_m.\] % To complete? 
\end{proof}

So far, we have constructed from $\CatCC$ and $F$ a partial bialgebra $\mathscr{A}$ with invariant integral $\phi$. Let us further impose for the rest of this section that $\CatCC$ admits duality.

\begin{Prop}\label{PropAnti} The partial bialgebra $\mathscr{A}$ is a regular partial Hopf algebra.
\end{Prop} 

\begin{proof} By Lemma \ref{LemMorDua}, we have for each $X\in \CatC_{rs}$ a canonical isomorphism \[d^{(mn)}_X:F_{mn}(X) \cong F_{nm}({}^*X)^*.\] 

For a linear map $x$ between vector spaces, let us denote $x^{\tr}$ for the transpose map between their duals. Then for any $x\in \Nat(F_{mn},F_{kl})$, let us define $\hat{S}(x) \in \Nat(F_{lk},F_{nm})$ by \[\hat{S}(x)_X = \left(d^{(nm)}_X\right)^{-1} x_{X^*}^{\tr} d_X^{(lk)}.\] % Ugly formatting

Then the assigment $\hat{S}$ dualizes to maps $S:\Gr{A}{k}{l}{m}{n} \rightarrow \Gr{A}{n}{m}{l}{k}$ by $S(f)(x) = f(\hat{S}(x))$. We claim that $S$ is an antipode for $\mathscr{A}$. 

Let us check for example the formula \[\sum_r f_{(1){\tiny \begin{pmatrix}k&l\\n & r\end{pmatrix}}} S(f_{(2){\tiny \begin{pmatrix} n & r \\ m & l\end{pmatrix}}}) = \delta_{k,m}\epsilon(f)\UnitC{k}{n}\] for $f\in \Gr{A}{k}{l}{m}{l}$. The other antipode identity follows similarly.

By duality, this is equivalent to the pointwise identity of natural transformations \[\sum_r\hat{M}^n_r(\id\otimes \hat{S})\hat{\Delta}^l_r(x) = \delta_{k,m}\UnitC{k}{n}(x) \id_{F_{kl}},\quad x\in \Nat(F_{nn},F_{km})\] where $\hat{M}^n_r$ and $(\id\otimes \hat{S})$ are dual to respectively $\Delta_{nr}$ and $\id\otimes S$. 

Let us fix $X\in \mathcal{C}_{k'l'}$, and let $e_i^{(kl)}$ be a basis for $F_{kl}(X)$ with dual basis $\omega_i^{(kl)}$. Let $\chi_i^{(lk)} = d_X^{(kl)}(e_i^{(kl)})$, and let $f_i^{(lk)}$ be the basis of $F_{lk}(X^*)$ dual to the basis $\{\chi_i^{(lk)}\}$. Then it is easily verified that \[\pi^{(klk)}_{X,{}^*X}F_{kk}(\coev_{X})(1) = \sum_i e_i^{(kl)}\otimes f_i^{(lk)},\quad F_{ll}(\ev_X)\iota^{(lkl)}_{{}^*X,X} = \sum_i \chi_i^{(lk)}\otimes \omega_i^{(kl)},\] where we identified $\C\cong F_{kk}(\Unit_{k'})$. We then check that for $x\in \Nat(F_{mn},F_{kl})$ and $v\in F_{lk}(X)$, one has \[\hat{S}(x)_Xv = \sum_{i,j} \omega_i^{(lk)}(v)\chi_i^{(kl)}(x_{X^*}f_j^{(mn)})e_j^{(nm)}.\] 


Hence for $x\in \Nat(F_{nr},F_{kl})$, $y\in \Nat(F_{rn},F_{lm})$ and $v \in F_{ml}(X)$, we have \begin{eqnarray*} \left(\hat{M}^n_r(\id\otimes \hat{S})(x\otimes y)\right)_X(v) &=& \sum_i \omega_i^{(ml)}\left(v\right)  x_Xd_X^{(nr)-1}(y_{X^*}^{\tr}(\chi_i^{(lm)}))\\ &=& \sum_{i,j} \omega_i^{(ml)}\left(v\right)  \chi_i^{(lm)}\left(y_{X^*}(f_j^{(rn)}))\right)x_Xe_j^{(nr)} \\ &=& \sum_{i,j}(\omega_i^{(ml)}\otimes \chi_i^{(lm)})\left(v\otimes y_{X^*}(f_j^{(rn)})\right)x_Xe_j^{(nr)}.\end{eqnarray*} So for $ x\in \Nat(F_{nn},F_{km})$ and $\omega \in F(X)_{kl}^*$,  \begin{eqnarray*} \omega\left(\sum_r\left(\hat{M}^n_r(\id\otimes \hat{S})\hat{\Delta}^l_r(x)\right)_X(v) \right)&=& \sum_{r,i,j}
\omega_i^{(ml)}(v)(\omega\otimes \chi_i^{(lm)})(\pi_{X,{}^*X}^{(mlm)}x_{X\otimes {}^*X}\iota_{X,{}^*X}^{(nrn)}(e_j^{(nr)}\otimes f_j^{(rn)})) \\
&=&  \sum_{i}
\omega_i^{(ml)}(v)(\omega\otimes \chi_i^{(lm)})(\pi_{X,{}^*X}^{(mlm)}x_{X\otimes {}^*X}F_{nn}(\coev_X)1)\\ &=&  \delta_{km} \UnitC{k}{n}(x)  \sum_{i}
\omega_i^{(kl)}(v)(\omega\otimes \chi_i^{(lk)})(\pi_{X,{}^*X}^{(klk)}F_{kk}(\coev_X)1) \\ &=&  \delta_{km} \UnitC{k}{n}(x)  \sum_{i,j}
\omega_i^{(kl)}(v)(\omega\otimes \chi_i^{(lk)})(e_j^{(kl)}\otimes f_j^{(lk)}) \\ &=&  \delta_{km} \UnitC{k}{n}(x)  \omega(v).
\end{eqnarray*}

Similarly, one shows that $\mathscr{A}$ with the opposite multiplication has an antipode, using right duality. It follows that $\mathscr{A}$ is a regular partial Hopf algebra.  
\end{proof} 

Now let us updgrade $\CatCC$ to a semisimple partial C$^*$-category with duality, and $\Forget$ to a morphism from $\CatCC$ to $\{\Hilb_{\fin}\}_{I\times I}$. Then we can of course still form the partial Hopf algebra $\mathscr{A}$ as above. Let us show that it becomes a partial Hopf $^*$-algebra with positive invariant integral.

We first introduce a $^*$-structure on $\mathscr{A}$. 
%To distinguish it from the $^*$-operations present on the C$^*$-category, we will denote it by $\dagger$.

\begin{Def} We define $^*: \Gr{A}{k}{l}{m}{n}\rightarrow \Gr{A}{l}{k}{n}{m}$ by the formula \[f^*(x) = \overline{f(\hat{S}(x)^*)},\qquad x\in \Nat(F_{nm},F_{lk}).\]
\end{Def}

\begin{Lem} The operation $^*$ is an anti-linear, anti-multiplicative, comultiplicative involution.
\end{Lem}

\begin{proof} Anti-linearity is clear. Comultiplicativity follows from the fact that $(xy)^* = y^*x^*$ and $\hat{S}(xy) = \hat{S}(y)\hat{S}(x)$ for natural transformations. To see anti-multiplicativity of $^*$, note first that, since $S$ is antimultiplicative for $\mathscr{A}$, $\hat{S}$ is anti-comultiplicative on natural transformations. Now as $(\iota_{X,Y}^{(klm)})^* = \pi_{X,Y}^{(klm)}$ by assumption, we also have $\hat{\Delta}^l_s(x)^* = \hat{\Delta}^s_l(x^*)$, which proves anti-multiplicativity of $^*$ on $\mathscr{A}$.  Finally, involutivity follows from the involutivity of $x\mapsto \hat{S}(x)^*$, which is a consequence of the identity $d_{X^*} = d_X^{\tr}$. %More info.
\end{proof}

\begin{Prop} The couple $(\mathscr{A},\Delta)$ with the above $^*$-structure defines a partial compact quantum group.
\end{Prop}
\begin{proof} The only thing which is left to prove is that our invariant integral $\phi$ is a positive functional. Now it is easily seen from the definition of $\phi$ that the $\Gr{A}{k}{l}{m}{n}(a)$ are all mutually orthogonal. Hence it suffices to prove that the sesquilinear inner product \[\langle f,g\rangle = \phi(f^*g)\] on $\Gr{A}{k}{l}{m}{n}(a)$ is positive-definite. 

Let us write $\bar{f}(x) = \overline{f(x^*)}$. Let again $\hat{\phi}^k_m$ be the natural transformation from $F_{mm}$ to $F_{kk}$ which is the identity on $\Unit_{k'}$ and zero on other irreducible objects. Then by definition, \[\phi(f^*g) = (\bar{f}\otimes g)((\hat{S}\otimes \id)\hat{\Delta}^k_m(\hat{\phi}^l_n)).\] Recall now the notation of Proposition \ref{PropAnti}. Let us write $\omega_i^{(kl)} = \langle E_i^{(kl)},-\rangle$ and $\chi_i^{(kl)} = \langle F_i^{(kl)},-\rangle$. Assume further that $f(x) = \langle v',x_a v\rangle$ and $g(x) = \langle w',x_aw\rangle$ for $v,w\in F_{mn}(u_a)$ and $v',w'\in F_{kl}(u_a)$. Then using the expression for $\hat{S}$ as in Proposition \ref{PropAnti}, we find that \[\phi(f^*g) = \sum_{ij} \langle v,e_j^{(mn)}\rangle\langle F_i^{(lk)}\otimes w',\hat{\Delta}_m^k(\hat{\phi}^{l}_n)_{\bar{a}\otimes a} (f_j^{(nm)}\otimes w)\rangle \langle E_i^{(kl)}, v'\rangle .\] However, up to a positive non-zero scalar, which we may assume to be 1 by proper rescaling, we have \[\hat{\Delta}^k_m(\hat{\phi}^l_n)_{\bar{a}\otimes a} = \sum_{i,j} \mid F_j^{(lk)} \otimes E_j^{(kl)}\rangle \langle F_i^{(nm)}\otimes E_i^{(mn)}\mid.\] Hence \begin{eqnarray*} \phi(f^*g) &=&\left(\sum_{ij}\langle v,e_j^{(mn)}\rangle \langle F_i^{(nm)},f_j^{(nm)}\rangle\langle E_i^{(mn)},w\rangle \right)  \left(\sum_{ij}  \langle w',E_j^{(kl)}\rangle \langle F_i^{(lk)},F_j^{(lk)}\rangle\langle E_i^{(kl)},v'\rangle\right)\\ &=& \langle v,w\rangle  \left(\sum_{ij}  \langle w',E_j^{(kl)}\rangle \langle F_i^{(lk)},F_j^{(lk)}\rangle\langle E_i^{(kl)},v'\rangle\right).
\end{eqnarray*} It follows that $\phi(f^*f)\geq 0$ for all $f$.

\end{proof} 

\begin{Theorem} \label{TheoTKPCQG}

The assigment $\mathscr{A}\rightarrow (\Corep(\mathscr{A}),\Forget)$ is (up to isomorphism/equivalence) a one-to-one correspondence between partial compact quantum groups based over $\varphi_0:I\twoheadrightarrow I_0$ and semi-simple $I_0$-partial tensor C$^*$-categories $\CatCC$ with duality and with forgetful strongly monoidal $^*$-functor $\Forget$ to $(\Hilb_f)_{I\times I}$ based over $\varphi_0$. 
\end{Theorem} 

\begin{proof} Fix first $\mathscr{A}$, and let $\mathscr{B}$ be the partial Hopf $^*$-algebra constructed from $\Corep(\mathscr{A})$ with its natural forgetful functor. Then we have a map $\mathscr{B} \rightarrow \mathscr{A}$ by \[ \Gr{B}{k}{l}{m}{n}(a) = \Hom(\Gr{V}{}{(a)}{m}{n},\Gr{V}{}{(a)}{k}{l})^* \rightarrow \Gr{A}{k}{l}{m}{n}(a):  f \mapsto (\id\otimes f)X_a,\] where the $(V^{(a)},X_a)$ run over all irreducible unitary corepresentations of $\mathscr{A}$. It is easy to check from the definitions of $\mathscr{B}$ that this map is an isomorphism of Hopf $^*$-algebras. As the matrix coefficients of irreducible unitary corepresentations span $\mathscr{A}$, the map is surjective. By the Schur orthogonality relations, it is bijective.

Conversely, let $\CatCC$ be a semi-simple $I_0$-partial tensor C$^*$-category with duality and with forgetful strongly monoidal $^*$-functor $\Forget$ to $(\Hilb_f)_{I\times I}$ based over $\varphi_0$. Let $\mathscr{A}$ be the associated Hopf $^*$-algebra. For each irreducible $u_a \in \CatCC$, let $V^{(a)} = F(u_a)$, and \[\Gr{(X_a)}{k}{l}{m}{n} = \sum_i e_i^*\otimes e_i,\] where $e_i$ is a basis of $\Nat(F_{mn},F_{kl})$ and $e_i^*$ a dual basis. Then from the definition of $\mathscr{A}$ it easily follows that $X_a$ is a unitary corepresentation for $\mathscr{A}$. Since $\Nat(F_{mn},F_{kl})$ spans the total space of operators at any irreducibel object, it follows that $X_a$ is irreducible. As the matrix coefficients of the $X_a$ span $\mathscr{A}$, it follows that the $X_a$ form a maximal class of non-isomorphic unitary corepresentations of $\mathscr{A}$. Hence we can make a unique equivalence \[\CatCC\rightarrow \Corep(\mathscr{A}), \quad u \mapsto (F(u),X_u)\] such that $u_a\rightarrow X_a$. From the definitions of the coproduct and product in $\mathscr{A}$, it is readily verified that the natural morphisms $\iota^{(klm)}_{u,v}:F_{kl}(u)\otimes F{lm}(v)\rightarrow F_{km}(u\otimes v)$ turn it into a monoidal equivalence. 
\end{proof}
