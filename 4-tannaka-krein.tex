\section{Tannaka-Krein-Woronowicz duality for partial compact quantum groups}

% Assume categories small?
% See `Galois reconstruction of finite quantum groups' of Bichon for additional refs

In the previous section, we showed how any partial compact quantum group gave rise to a partial semi-simple tensor C$^*$-category with duality and indecomposable units, together with a morphism into a partial tensor C$^*$-category of finite dimensional Hilbert spaces. In this section, we reverse this construction, and show that the two structures are in duality with each other. As the proof does not differ too much from the usual Tannaka-Krein reconstruction process, we do not spell out all details, emphasizing only the novel points having to do with well-definedness of the constructions.  % Say that implicitly, we do two steps at once: create a partial discrete quantum group, and then dualize. But we don't formally introduce partial discrete quantum groups.

In the following, let us at first fix an indecomposable semi-simple partial tensor category $\CatCC$ over a base set $I_0$. We will again view the tensor product of $\CatC$ as being strict, for notational convenience. 

Assume that we also have another set $I$ and a partition $I = \{I_k\mid k\in I_0\}$ with associated function \[\varphi_0:I\rightarrow I_0, \quad k\mapsto k'.\] Let $\Forget: \CatCC\rightarrow \{\Vect_{\fin}\}_{I\times I}$ be a morphism based on $\varphi_0$ (cf.a~ Example \ref{ExaVectBiGr}).  We will again denote by $\Forget_{kl}:\CatCC_{k'l'}\rightarrow \Vect_{\fin}$ the components of $\Forget$ at index $(k,l)$, and by $\iota$ and $\mu$ resp.~ the associativity and unit constraints.  For $X\in \CatC_{k's}$ and $Y\in \CatC_{sm'}$, we write the projection maps associated to the identification $\Forget_{km}(X\otimes Y)\cong \oplus_{l\in I_s} \left(\Forget_{kl}(X)\otimes \Forget_{lm}(Y)\right)$ as \[\pi^{(klm)}_{X,Y}:\Forget_{km}(X\otimes Y) \rightarrow \Forget_{kl}(X)\otimes \Forget_{lm}(Y).\]

We choose a maximal family of mutually inequivalent irreducible objects $\{u_a\}_{a\in \mathcal{I}}$ in $\CatC$. We assume that the $u_a$ include the unit objects $\Unit_{k'}$ for $k'\in I_0$, and we may hence identify $I_0\subseteq \mathcal{I}$. For $a\in \mathcal{I}$, we will write $u_a \in \CatC_{s_a,t_a}$ with $s_a,t_a\in I_0$. For $s,t\in I_0$ fixed, we write $\mathcal{I}_{rs}$ for the set of all $a\in \mathcal{I}$ with $s_a=s$ and $t_a=t$. When $a,b,c\in \mathcal{I}$ with $a\in \mathcal{I}_{rs},b\in \mathcal{I}_{st}$ and $c\in \mathcal{I}_{rt}$, we write $c\leq a\cdot b$ if $\Mor(u_c,u_a\otimes u_b)\neq \{0\}$. Note that with $a,b$ fixed, there is only a finite set of $c$ with $c\leq a\cdot b$. We also use this notation for multiple products.

\begin{Def} For $a\in \mathcal{I}$ and $k,l,m,n\in I$, define vector spaces \[\Gr{A}{k}{l}{m}{n}(a) =  \delta_{k,m,s_a}\delta_{l,n,t_a} \Hom_{\C}(\Forget(u_a)_{kl},\Forget(u_a)_{mn})^*.\] Write \[\Gr{A}{k}{l}{m}{n} =\underset{a\in \mathcal{I}}{\oplus}\, \Gr{A}{k}{l}{m}{n}(a),\quad A(a) = \underset{k,l,m,n}{\oplus} \Gr{A}{k}{l}{m}{n}(a),\quad A = \underset{k,l,m,n}{\oplus} \Gr{A}{k}{l}{m}{n}.\] 
%For any element $x\in A$, its component in the $a$-spectral subspace is written $x_a$.
\end{Def} 

%Note that by construction, $\Gr{A}{k}{l}{m}{n}=0$ if $k'\neq m'$ or $l'\neq n'$. 

We first turn the $\Gr{A}{k}{l}{m}{n}$ into a partial coalgebra $\mathscr{A}$ over $I^2$.

\begin{Def} For $r,s\in I$, we define \[\Delta_{rs}: \Gr{A}{k}{l}{m}{n}\rightarrow \Gr{A}{k}{l}{r}{s}\otimes \Gr{A}{r}{s}{m}{n}\] as the direct sums of the duals of the composition maps \[\Hom_{\C}(\Forget(u_a)_{kl},\Forget(u_a)_{rs}) \otimes \Hom_{\C}(\Forget(u_a)_{rs},\Forget(u_a)_{mn})\rightarrow \Hom_{\C}(\Forget(u_a)_{kl},\Forget(u_a)_{mn}),\]\[x\otimes y \mapsto y\circ x.\]
\end{Def} 

\begin{Lem} The couple $(\mathscr{A},\Delta)$ is a partial coalgebra with counit map \[\epsilon:\Gr{A(a)}{k}{l}{k}{l}\rightarrow \C,\quad f\mapsto f(\id_{\Forget(u_a)_{kl}}).\] Moreover, for each fixed $f\in \Gr{A(a)}{k}{l}{m}{n}$, the matrix $\left(\Delta_{rs}(f)\right)_{rs}$ is row- and column-finite.
\end{Lem} 
\begin{proof} Coassociativity and counitality are immediate by duality, as for each $a$ fixed the $\Hom_{\C}(\Forget(u_a)_{kl},\Forget(u_a)_{mn})$ form a partial algebra with units $\id_{F(u_a)_{kl}}$. The row-and column-finiteness condition follows immediately from the local finiteness condition for the morphism $F$.
\end{proof}

In the next step, we define a partial algebra structure on $\mathscr{A} = \{\Gr{A}{k}{l}{m}{n}\mid k,l,m,n\}$. First note that we can identify \[\Nat(\Forget_{kl},\Forget_{mn}) \cong \underset{t_a=l'=n'}{\underset{s_a=k'=m'}{\prod_a}} \Hom_{\C}(\Forget(u_a)_{kl},\Forget(u_a)_{mn}),\] where $\Nat(\Forget_{kl},\Forget_{mn})$ denotes the space of natural transformations from $\Forget_{kl}$ to $\Forget_{mn}$ when $k'=m'$ and $l'=n'$. Similarly, we can identify \[\Nat(\Forget_{kl}\otimes \Forget_{rs},\Forget_{mn}\otimes \Forget_{pq}) \cong  \prod_{b,c} \Hom_{\C}(\Forget(u_b)_{kl}\otimes \Forget(u_c)_{rs} ,\Forget(u_b)_{mn}\otimes \Forget(u_c)_{pq}),\] with the product over the appropriate index set and where \[\Forget_{kl}\otimes \Forget_{rs}:\CatC_{k'l'}\times \CatC_{r's'}\rightarrow \Vect_{\fin},\quad (X,Y) \mapsto F_{kl}(X)\otimes F_{rs}(Y).\] As such, there is a natural pairing of these spaces with resp.~ $\Gr{A}{k}{l}{m}{n}$ and $\Gr{A}{k}{l}{m}{n}\otimes \Gr{A}{r}{s}{p}{q}$. 

% To expand. 

\begin{Def} For $k'=r', l'=s'$ and $m'=t'$, we define a product map \[M:\Gr{A}{k}{l}{r}{s} \otimes \Gr{A}{l}{m}{s}{t}\rightarrow  \Gr{A}{k}{m}{r}{t},\quad f\otimes g \mapsto f\cdot g\] by the formula \[(f\cdot g)(n) = (f\otimes g)( \hat{\Delta}^{l}_{s}(n)), \qquad  n \in \Nat(\Forget_{km},\Forget_{rt}),\] where $\hat{\Delta}^l_s(n)$ is the natural transformation\[\hat{\Delta}^l_s(n):  \Forget_{kl}\otimes \Forget_{lm}\rightarrow \Forget_{rs}\otimes \Forget_{st},\quad \hat{\Delta}^l_s(n)_{X,Y} = \pi^{(rst)}_{X,Y}n_{X\otimes Y} \iota^{(klm)}_{X,Y},\quad X\in \CatC_{k'l'},Y\in \CatC_{l'm'}.\]
\end{Def}

\begin{Rem} It has to be argued that $f\cdot g$ has finite support (over $\mathcal{I})$ as a functional on $\Nat(\Forget_{km},\Forget_{rt})$. In fact, if $f$ is supported at $b\in \mathcal{I}_{rs}$ and $g$ at $c\in \mathcal{I}_{st}$, then $f\cdot g$ has support in the finite set of $a\in \mathcal{I}_{rt}$ with $a\leq b\cdot c$, since if $n$ is a natural transformation with support outside this set, one has $n_{u_b\otimes u_c}=0$, and hence any of the $\left(\hat{\Delta}^l_s(n)\right)_{u_b,u_c} =0$.
\end{Rem}

\begin{Lem} The above product maps turn $(\mathscr{A},M)$ into an $I^2$-partial algebra.
\end{Lem}
\begin{proof} We can extend the map $(\hat{\Delta}^l_s\otimes \id)$ on $\Nat(\Forget_{km},\Forget_{rt})\otimes \Nat(\Forget_{mn},\Forget_{tu})$ to a map \[(\hat{\Delta}^l_s\otimes \id): \Nat(\Forget_{km}\otimes \Forget_{mn},\Forget_{rt}\otimes \Forget_{tu}) \rightarrow  \Nat(\Forget_{kl}\otimes \Forget_{lm}\otimes \Forget_{mn},\Forget_{rs}\otimes \Forget_{st}\otimes \Forget_{tu}),\] \[(\hat{\Delta}^l_s\otimes \id)(x)_{X,Y,Z} = \left(\pi^{(rst)}_{X,Y}\otimes \id_{\Forget_{tu}(Z)}\right) x_{X\otimes Y, Z} \left(\iota^{(klm)}_{X,Y} \otimes \id_{\Forget_{mn}(Z)}\right).\]
By finite support, we then have that \[((f\cdot g)\cdot h)(x) = (f\otimes g\otimes h)(\hat{\Delta}^l_s\otimes \id)\hat{\Delta}^m_t(x))\] for all $f\in \Gr{A}{k}{l}{r}{s},g\in \Gr{A}{l}{m}{s}{t},h\in \Gr{A}{m}{n}{t}{u}$ and $x\in  \Nat(\Forget_{kn},\Forget_{ru})$. Similarly, \[(f\cdot g)\cdot h)(x) = (f\otimes g\otimes h)((\id\otimes \hat{\Delta}^m_t)\hat{\Delta}^l_s(x).\] The associativity then follows from the 2-cocycle condition for the $\iota$- and $\pi$-maps. 

By a similar argument, one sees that the (non-zero) units are given by $\UnitC{k}{r}\in \Gr{A}{k}{k}{l}{l}(\Unit_{r})$  (for $r=k'=l'$) corresponding to $1$ in the canonical identifications  \[\Gr{A}{k}{k}{l}{l}(r) \cong \Hom_{\C}(\Forget_{kk}(\Unit_{r}),\Forget_{ll}(\Unit_{r}))^*\cong \Hom_{\C}(\C,\C)^*  \cong \C^* \cong \C.\] Indeed, to prove for example the right unit property, we use that (essentially) $\pi_{u_a,\Unit_{r}}^{(kll)} =(\id\otimes \mu_l)$ and $\iota_{u_a,\Unit_{r}}^{(kll)} = (\id\otimes \mu_l^{-1})$, while \[\UnitC{k}{l}(\mu_ln_{\Unit_{r}}\mu_k^{-1}) = n_{\Unit_{r}} \in \C,\quad n\in \Nat(F_{kk},F_{ll}).\] % to complete
\end{proof} 

\begin{Prop} The partial algebra and coalgebra structures on $\mathscr{A}$ define a partial bialgebra structure on $\mathscr{A}$. 
\end{Prop}
\begin{proof} Let us check the properties in Definition \ref{DefPartBiAlg}. Properties \ref{Propa} and \ref{Propc} are left to the reader. Property \ref{Propd} was proven above. Property \ref{Propb} follows from the fact that for $k'=l'=s'=m'$, \[\hat{\Delta}^{l}_s(\id_{F_{km}}) = \delta_{ls} \id_{F_{kl}}\otimes \id_{F_{lm}}.\] 
It remains to show the multiplicativity property \ref{Prope}. This is equivalent with proving that, for each $x\in \Nat(F_{km},F_{uw})$ and $y\in \Nat(F_{uw},F_{rt})$ (with all first or second indices in the same class of $I_0$), one has (pointwise) that (for $l'=s'$) \[ \hat{\Delta}^l_s(y\circ x) = \sum_{v,v'=l'} \hat{\Delta}^v_s(y)\circ \hat{\Delta}^l_v(x).\] This follows from the fact that $\sum_v \pi^{(uvw)}_{X,Y}\iota^{(uvw)}_{X,Y} \cong \id_{\Forget_{uw}(X\otimes Y)}$ (where we again note that the left hand side sum is in fact finite).
\end{proof} 

Let us show now that the resulting bialgebra $\mathscr{A}$ has an invariant integral.

\begin{Def} Define $\phi: \Gr{A}{k}{k}{m}{m} \rightarrow \C$ as the functional which is zero on $\Gr{A}{k}{k}{m}{m}(a)$ with $a\neq \Unit_{k'}$, and the canonical identification $\Gr{A}{k}{k}{m}{m}(k')\cong \C$ on the unit component (for $k'=m'$).
\end{Def}

\begin{Lem} The functional $\phi$ is an invariant integral.
\end{Lem}

\begin{proof} The normalisation condition $\phi(\UnitC{k}{k})=1$ is immediate by construction. Let us check left invariance - right invariance will follow similarly.

Let $\hat{\phi}^k_l$ be the natural transformation from $\Forget_{kk}$ to $\Forget_{ll}$ which has support on multiples of $\Unit_{k'}$, and with $(\hat{\phi}^k_l)_{\Unit_{k'}} = 1$.  Then for $f\in \Gr{A}{k}{k}{l}{l}$, we have $\phi(f) = f(\hat{\phi}^k_l)$. The left invariance of $\phi$ then follows from the easy verification that for $y\in \Nat(F_{kn},F_{ll})$, \[\hat{\phi}^l_m\circ y =\delta_{k,n} \UnitC{k}{l}(y)\hat{\phi}^k_m.\] % To complete? 
\end{proof}

So far, we have constructed from $\CatCC$ and $F$ a partial bialgebra $\mathscr{A}$ with invariant integral $\phi$. Let us further impose for the rest of this section that $\CatCC$ admits right duality.

\begin{Prop} The partial bialgebra $\mathscr{A}$ is a partial Hopf algebra.
\end{Prop} 

\begin{proof} By Lemma \ref{LemMorDua}, we have for each $X\in \CatC_{rs}$ a canonical isomorphism \[d^{(mn)}_X:F_{mn}(X) \cong F_{nm}(X^*)^*,\] since left and right duals coincide for $\Vect_{\fin}$.

For a linear map $x$ between vector spaces, let us denote $x^{\tr}$ for the transpose map between their duals. Then for any $x\in \Nat(F_{kl},F_{mn})$, let us define $\hat{S}(x) \in \Nat(F_{nm},F_{lk})$ by \[\hat{S}(x)_X = \left(d^{(lk)}_X\right)^{-1} x_{X^*}^t d_X^{(nm)}.\] % Ugly formatting

Then the assigment $\hat{S}$ dualizes to maps $S:\Gr{A}{k}{l}{m}{n} \rightarrow \Gr{A}{n}{m}{l}{k}$ by $S(f)(x) = f(\hat{S}(x))$. We claim that $S$ is an antipode for $\mathscr{A}$. 

Let us check for example the formula \[\sum_r f_{(1){\tiny \begin{pmatrix}k&l\\n & r\end{pmatrix}}} S(f_{(2){\tiny \begin{pmatrix} n & r \\ m & l\end{pmatrix}}}) = \delta_{k,m}\epsilon(f)\UnitC{k}{n}\] for $f\in \Gr{A}{k}{l}{m}{l}$. By duality, this is equivalent to the pointwise identity of natural transformations \[\sum_r\hat{M}^n_r(\id\otimes \hat{S})\hat{\Delta}^l_r(x) = \delta_{k,m}\UnitC{k}{n}(x) \id_{F_{kl}},\quad x\in \Nat(F_{km},F_{nn})\] where $\hat{M}^p_r$ and $(\id\otimes \hat{S})$ are dual to respectively $\Delta_{pr}$ and $\id\otimes S$. 

Let us fix $X\in \mathcal{C}_{k'l'}$, and let $e_i^{(kl)}$ be a basis for $F_{kl}(X)$ with dual basis $\omega_i^{(kl)}$. Let $\chi_i^{(lk)} = d_X^{(kl)}(e_i)$, and let $f_i^{(lk)}$ be the basis of $F_{lk}(X^*)$ dual to the basis $\{\chi_i^{(lk)}\}$. Then it is easily verified that \[\pi^{(lkl)}_{X^*,X}F_{ll}(\coev_{X})(1) = \sum_i f_i^{(lk)}\otimes e_i^{(kl)},\quad F_{kk}(\ev_X)\iota^{klk}_{X,X^*} = \sum_i \omega_i^{(kl)}\otimes \chi_i^{(lk)},\] where we identified $\C\cong F_{kk}(\Unit_{k'})$. 

Then for $x\in \Nat(F_{kl},F_{nr})$, $y\in \Nat(F_{lm},F_{rn})$ and $v \in F_{kl}(X)$, we have \begin{eqnarray*} \left(\hat{M}^n_r(\id\otimes \hat{S})(x\otimes y)\right)(v) &=& \sum_i \omega_i^{(nr)}\left(x_X(v)\right)  d_X^{(ml)-1}(y_{X^*}^{\tr}(\chi_i^{(rn)}))\\ &=& \sum_{i,j} \omega_i^{(nr)}\left(x_X(v)\right)  \chi_i^{(rn)}\left(y_{X^*}(f_j^{(lm)}))\right)e_j^{(ml)} \\ &=& \sum_{i,j}(\omega_i^{(nr)}\otimes \chi_i^{(rn)})\left(x_X(v)\otimes y_{X^*}(f_j^{(lm)})\right)e_j^{(ml)}.\end{eqnarray*} Hence for $ x\in \Nat(F_{km},F_{nn})$, \begin{eqnarray*} \sum_r \left(\hat{M}^n_r(\id\otimes \hat{S})\hat{\Delta}^l_r(x)\right)_X(v) &=& \sum_{r,i,j}(\omega_i^{(nr)}\otimes \chi_i^{(rn)})\left( \pi_{X, X^*}^{(nrn)}x_{X\otimes X^*}\iota_{X, X^*}^{(klm)}(v\otimes f_j^{(lm)})\right)e_j^{(ml)} \\ &=& \sum_{r,j} F_{nn}(\ev_X)\left(\iota_{X, X^*}^{(nrn)}\pi_{X, X^*}^{(nrn)}x_{X\otimes X^*}\iota_{X, X^*}^{(klm)}(v\otimes f_j^{(lm)})\right)e_j^{(ml)}\\ &=&  \sum_{j} F_{nn}(\ev_X)x_{X\otimes X^*}\iota_{X, X^*}^{(klm)}(v\otimes f_j^{(lm)})e_j^{(ml)}\\
&=& \sum_{j}x_{\Unit_{k'}} F_{km}(\ev_X)\iota_{X, X^*}^{(klm)}(v\otimes f_j^{(lm)})e_j^{(ml)}
\\ &=& \delta_{km} \UnitC{k}{n}(x)  \sum_{i,j}  \omega_i^{(kl)}(v)\chi_i^{(lk)}(f_j^{(lk)})e_j^{(kl)}\\
&=&  \delta_{km} \UnitC{k}{n}(x)  v.
\end{eqnarray*}
\end{proof} 

Now let us updgrade $\CatCC$ to a semisimple partial C$^*$-category with duality, and $\Forget$ to a morphism from $\CatCC$ to $\{\Hilb_{\fin}\}_{I\times I}$. Then we can of course still form the partial Hopf algebra $\mathscr{A}$ as above. Let us show that it becomes a partial Hopf $^*$-algebra with positive invariant integral.

We first introduce a $^*$-structure on $\mathscr{A}$. 
%To distinguish it from the $^*$-operations present on the C$^*$-category, we will denote it by $\dagger$.

\begin{Def} We define $^*: \Gr{A}{k}{l}{m}{n}\rightarrow \Gr{A}{l}{k}{n}{m}$ by the formula \[f^*(x) = \overline{f(\hat{S}(x)^*)},\qquad x\in \Nat(F_{lk},F_{nm}).\]
\end{Def}

\begin{Lem} The operation $^*$ is an anti-linear, anti-multiplicative involution.
\end{Lem}

\begin{proof} Anti-linearity is clear. Anti-multiplicativity follows from ... Involutiveness follows from ...

\end{proof}

\begin{Prop} The couple $(\mathscr{A},\Delta)$ with the above $^*$-structure defines a partial compact quantum group.
\end{Prop}
\begin{proof} ...

\end{proof} 

\begin{Theorem} \label{TheoTKPCQG} The assigment $\mathscr{A}\rightarrow (\Corep(\mathscr{A}),\Forget)$ is essentially a one-to-one correspondence between partial compact quantum groups based over $\varphi_0:I\twoheadrightarrow I_0$ and semi-simple $I_0$-partial tensor C$^*$-categories $\CatCC$ with forgetful strongly monoidal $^*$-functor $\Forget$ to $(\Hilb_f)_{I\times I}$ based over $\varphi_0$.
\end{Theorem} 

\begin{proof}

\end{proof}

\section{Examples}

\subsection{Classical examples}

\subsection{Canonical partial compact quantum groupoids}

$\CatCC$ acting on itself. Characterisation of resulting partial compact quantum groups? 

\subsection{Morita equivalence}

% Caenepeel Galois theory weak Hopf algebras

\begin{Def} Two partial compact quantum groups $\mathscr{G}$ and $\mathscr{H}$ are said to be \emph{Morita equivalent} if there exists an equivalence $\Rep(\mathscr{G}) \rightarrow \Rep(\mathscr{H})$ of partial tensor C$^*$-categories. % notion of equivalence to be explained? $\varphi_0$ is bijective, and all $F_{rs}$ are faithful and essentially surjective.
\end{Def} 

In particular, if $\mathscr{G}$ and $\mathscr{H}$ are Morita equivalent they have the same hyperobject set, but they need not share the same object set.

\begin{Prop} The partial compact quantum groups $\mathscr{G}_1$ and $\mathscr{G}_2$ over respective object sets $I_1$ and $I_2$ and with the same hyperobject set $I_0$ are Morita equivalent if and only if there exists a partial compact quantum group $\mathscr{G}$ over the object set $I_1\sqcup I_2$ such that the associated partial compact quantum group $\mathscr{A}$ has $\Gr{A}{k}{l}{m}{n}=\{0\}$ for $k,l$ or $m,n$ not in the same set $I_j$, such that $\Gr{A}{k}{l}{m}{n}\neq 0$ for ..., and such that the $\Gr{A}{k}{l}{m}{n}$ with all indices in the same set $I_j$, together with all $\Delta_{rs}$ with $rs$ also in this same set, form an isomorphic copy of the partial Hopf $^*$-algebra associated to $\mathscr{G}_j$. % Formulate more succinctly? 
\end{Prop}
\begin{proof} If $\mathscr{G}_1$ and $\mathscr{G}_2$ are Morita equivalent, we may identify the partial tensor C$^*$-categories of $\mathscr{G}_1$ and $\mathscr{G}_2$ with the same abstract partial tensor C$^*$-category $\CatCC$. Then the $\mathscr{G}_j$ are obtained from respective forgetful functors $F_j$ into $(\Hilb_{\fin})_{I_j\times I_j}$. Form $F_1\oplus F_j$, which is a morphism from $\CatCC$ into $(\Hilb_{\fin})_{(I_1\cup I_2)\times (I_1\cup I_2)}$. It is easily seen that the associated compact quantum groupoid is of the above form.

Conversely, ...
\end{proof} 

For example, co-groupoid of Bichon. 

Let $(A,\Delta)$ be a generalized Hopf face algebra over a set $I$. Assume that $I = I_1\sqcup I_2$, and let $\Lambda_j = \sum_{i\in I_j}\lambda_i$, resp. $\Rho_j = \sum_{i\in I_j} \rho_j$. If the $\Lambda_j$ and $\Rho_j$ are central in $M(A)$, then we can write $A = \osum{i,j} A(ij)$ where $A(ij) = \Lambda_i\Rho_jA$ are subalgebras. Moreover, the comultiplication $\wDelta$ splits into comultiplications \[\wDelta_{ij}^k:A(ij)\rightarrow M(A(ik)\otimes A(kj))\textrm{ s.t. } \wDelta = \wDelta_{ij}^1 +\wDelta_{ij}^2 \textrm{ on }A(ij).\] A similar decomposition holds for $\Delta$.

It is immediate to see that the $(A(ii),\Delta_{ii}^i)$ are two generalized Hopf face algebras over the respective $I_i$.

\begin{Def} We say $(A,\Delta)$ is a \emph{co-linking generalized (compact) Hopf face algebra} between $(A(11),\Delta_{11}^1)$ and $(A(22),\Delta_{22}^2)$ if $\lambda_i\Rho_2\neq 0$ for any $i\in I_1$.
\end{Def}

Upon applying the antipode, we see that then $\rho_j\Lambda_1\neq 0$ for any $j\in I_2$ as well.

\begin{Def} Two generalized (compact) Hopf face algebras are called \emph{comonoidally Morita equivalent} if they are isomorphic to the components $(A_{ii},\Delta_{ii}^i)$ of some co-linking generalized (compact) Hopf face algebra.\end{Def}

As an example, consider two sets $I_i$, and two tensor functors $(F_i,\phi_i)$ of a semi-simple rigid C$^*$-category $\CatC$ with irreducible unit into $\Hilb_{I_i^2}$. Then with $I= I_1\sqcup I_2$, we can form a new C$^*$-functor $F=F_1\oplus F_2$ of $\CatC$ into $\Hilb_{I^2}$ by putting $F(X) = F_1(X)\oplus F_2(X)$ with the obvious $I^2$-grading (and the obvious direct sum operation on morphisms). It becomes monoidal by means of the unitaries \[F(X\otimes Y) = F_1(X\otimes Y)\oplus F_2(X\otimes Y) \underset{\phi_1\oplus \phi_2}{\cong} (F_1(X)\underset{I_1}{\otimes} F_1(Y)) \oplus (F_2(X)\otimes F_2(Y)) \cong F(X)\itimes F(Y)\] (where the last map is unitary since $(F(X)\itimes F(Y))_{ij}=0$ for example for $i\in I_1$ and $j\in I_2$).

If we then consider the generalized compact Hopf face algebra $(A,\Delta)$ associated to $F$, we have immediately from the construction that the $\Lambda_i$ and $\Rho_i$ associated to the decomposition $I = I_1\sqcup I_2$ are indeed central elements in $M(A)$. Moreover, the parts $(A_{ii}^i,\Delta_{ii}^i)$ are seen to arise from applying the Tannaka-Krein construction to the respective functors $F_1$ and $F_2$. The fact that $(A,\Delta)$ is co-linking is immediate from the fact that \emph{none} of the $\lambda_i\rho_j$ are zero in this particular case (since $\Gr{A}{k}{k}{m}{m}(o) = B(F(u_o)_{kk},F(u_o)_{mm}) \cong \C$).

We will exploit the above extra structure in the following section to say something about the algebra $A$ appearing in ... This is the component $\tilde{A}(1,1)$ of the above algebra. The following lemma will be needed.


\begin{Lem} Assume $(A,\Delta)$ is a co-linking generalized Hopf face algebra. Then any of the maps $\wDelta_{ij}^k$ is injective.\end{Lem}

\begin{proof} Take a non-zero $x\in A_n(ij)$ where $n\in I_j$. Then for any $l\in I$ with $\rho_n\lambda_l\neq 0$, we know that $\wDelta(x)(1\otimes \rho_n\lambda_l)\neq 0$. Hence $\wDelta_{ij}^k(x)(1\otimes \rho_n\lambda_l)\neq 0$ for $l\in I_k$, and hence $\wDelta_{ij}^k(x)\neq 0$. Now if $j=k$, the condition $\rho_n\lambda_l\neq 0$ is satisfied by taking $l=n$ (since $\varepsilon(\lambda_n\rho_n)=1$). If $j\neq k$, it is satisfied for at least one $l$ by the co-linking assumption.
\end{proof}


\subsection{Weak Morita equivalence}

% Concrete implementation, helps to define the concept in more general analytic context.
Cf. Muger. 

\begin{Def} Two partial semi-simple tensor C$^*$-categories $\CatCC_1$ and $\CatC_2$ with duality over respective sets $I_1$ and $I_2$ are called \emph{Morita equivalent} if there exists a partial semi-simple tensor C$^*$-category $\CatCC$ with duality over the set $I=I_1\sqcup I_2$ such that the $\CatC_{rs}$ with $r,s$ in the same set $I_j$ form a copy of $\CatCC_j$, and such that each object in $\CatCC_{j}$ is a subobject of $X\otimes Y$ for some $X\in \CatCC_{kl}$ and $Y\in \CatCC{lk}$ with $k$ and $l$ in disjoint sets and $k\in I_j$. % Better formulation.

We say two partial compact quantum groups $\mathscr{G}_1$ and $\mathscr{G}_2$ are weakly Morita equivalent if their corepresentation categories are Morita equivalent. % ???
\end{Def} 

This is indeed generalisation of Morita equivalence. 

Notion of co-Morita equivalence. Weak Morita equivalence = Morita equivalence + co-Morita equivalence. 

%Induction of forgetful functor from $\CatCC_1$ to $\CatCC$.?? Maybe by internal representation of a module category by an algebra object. 

% Concrete implementation of weak Morita equivalence... 

\subsection{Ergodic actions of compact quantum group}

Let us now give a rich source of examples coming from ergodic actions of compact quantum groups. We recall the main result from DC-Yamashita.

Link with Morita equivalence. 

Concrete example of dynamical quantum $SU(2)$ to be studied in particular in separate paper. 








%%% Local Variables: 
%%% mode: latex
%%% TeX-master: "dyn-suq-main"
%%% End: 

