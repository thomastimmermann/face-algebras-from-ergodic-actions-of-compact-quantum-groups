\section{Coamenability of partial compact quantum groups}

\begin{Def} A partial compact quantum group $\mathscr{G}$ will be called \emph{co-amenable} if the natural projection map $\pi_r:\CuG\rightarrow \CrG$ is an isomorphism.
\end{Def}

Our aim is to give a characterisation of coamenability in terms of the representation theory of the fusion algebra, cf. [Kyed].

We start of with the following analogue of the Fell absorption principle. For $(\Hsp,\pi)$ a non-degenerate $^*$-representation of $\CuG$, we will write $\GrLA{\Hsp}{k}{l} = \pi(\UnitC{k}{l})\Hsp$, which then gives a direct sum decomposition of $\Hsp$. We write $H$ for the linear span of all $\GrLA{\Hsp}{k}{l}$. When $(\mathcal{K},\pi')$ is another non-degenerate $^*$-representation, we will write $\pi \boxtimes \pi'$ for the representation of $\CuG$ on $\oplus_l ({}_l\Hsp\otimes {}^l\mathcal{K})$ via $\vnDelta$.

\begin{Lem} Let $(\Hsp,\pi)$ be a non-degenerate $^*$-representation of $\CuG$. Then $\pi_r\boxtimes \pi$ factorizes over $\CrG$.
\end{Lem}

\begin{proof}  We have a unitary \[U: \oplus_q (L^2(\mathscr{G})_q\otimes {}_q\Hsp ) \rightarrow \oplus_l ({}_l L^2(\mathscr{G})\otimes {}^l\Hsp),\]\[\Lambda(x)\otimes \xi \mapsto \Lambda(x_{(1)})\otimes \pi(x_{(2)})\xi,\quad x\in P(\mathscr{G}),\xi\in H,\] with inverse $\Lambda(x)\otimes \xi \mapsto \Lambda(x_{(1)})\otimes \pi(S(x_{(2)}))\xi$. It is clear that $U$ intertwines $\pi_r\boxtimes \pi$ with $\oplus_q (\pi_r\otimes 1)$.
\end{proof} 

Recall now (from previous paper) the $^*$-representation \[\weps: \CuG \rightarrow B(l^2(I)), \quad x\in \Gr{A}{k}{l}{m}{n}\mapsto \epsilon(x) e_{kl}.\]

\begin{Lem}\label{LemUnit} Let $(\Hsp,\pi)$ be a non-degenerate $^*$-representation of $\CuG$. Then $\weps\boxtimes \pi$ is unitarily equivalent to $\pi$.
\end{Lem} 
\begin{proof} The map \[U: \Hsp \rightarrow \oplus_k {}_kl^2(I)\otimes {}^k\Hsp,\quad \xi \mapsto \sum_k e_k\otimes \pi(\lambda_k)\xi\] is a unitary intertwiner.
\end{proof} 

\begin{Cor} The partial compact quantum group $\mathscr{G}$ is coamenable if and only if $\weps$ descends to a $^*$-representation of $\CrG$.
\end{Cor} 
\begin{proof} $\Rightarrow$ is clear. For $\Leftarrow$, note that $\vnDelta$ descends to a (non-unital) $^*$-homomorphism $\CrG\rightarrow M(\CrG\otimes \CuG)$ by Fell's absorption principle. Composing with $\weps\otimes \id$ and using Lemma \ref{LemUnit} with respect to a faithful representation of $\CuG$, we see that there exists a splitting $\CrG\rightarrow \CuG$ of $\pi_r$, hence $\pi_r$ is an isomorphism.
\end{proof}
 
 We now pass to the character algebra of $\mathscr{G}$.
 
 \begin{Def} Let $\mathscr{X}$ be a unitary rcfd representation of $\mathscr{G}$ on $\oplus_{k,l} \Gru{\Hsp}{k}{l}$. The \emph{partial characters} of $\mathscr{X}$ are defined as \[\Gr{\chi}{}{X}{k}{l} = (\id\otimes \Tr)(\Gr{X}{k}{l}{k}{l}) = \sum_i(\Gr{X}{k}{l}{k}{l})_{ii} \in \Gr{A}{k}{l}{k}{l}.\]
 The \emph{total character} of $\mathscr{X}$ is the multiplier $\chi^X  =\sum_{kl} \Gr{\chi}{}{X}{k}{l}\in M(A)$.
 \end{Def}
 
 For example, if $U$ is the trivial corepresentation, then $\Gr{\chi}{}{U}{k}{l} = \delta_{k,l}\UnitC{k}{k}$, and $\chi^U = \sum_k \UnitC{k}{k}$.
 
 We aim to show that $\chi$ is in fact a well-defined element in $M(\CuG)$. 
 
 \begin{Lem}\label{LemBoundDim} We have $\Gr{\chi}{}{X\otimes Y}{k}{m}= \sum_l \Gr{\chi}{}{X}{k}{l}\Gr{\chi}{}{Y}{l}{m}$.
 \end{Lem}
 
 \begin{Lem} Let $\mathscr{X}$ be a unitary rcfd representation of $\mathscr{G}$ on $\Hsp$ with finite hyperobject support, and write $\Gr{d}{}{X}{k}{l} = \dim(\Gru{\Hsp}{k}{l})$. Then the matrix $D^X = (\Gr{d}{}{X}{k}{l})_{kl}$ defines a bounded operator on $l^2(I)$. 
  \end{Lem}
  \begin{proof} This follows from [DCY1].
   \end{proof} 
   
\begin{Cor} The sum $\chi^X  =\sum_{kl} \Gr{\chi}{}{X}{k}{l}$ converges strictly to an element in $M(C_u(\mathscr{G}))$. 
\end{Cor} 
 
\begin{proof} It is sufficient to prove that the matrix of operators $(\Gr{\chi}{}{X}{k}{l})_{kl})$ defines a bounded matrix in any $^*$-representation of $C_u(\mathscr{G})$. But \[\| \Gr{\chi}{}{X}{k}{l}\| \leq \Gr{d}{}{X}{k}{l},\] hence this follows from Lemma \ref{LemBoundDim}.
\end{proof} 
 
Denote now $\mathcal{C}$ for the C$^*$-algebra generated by all $\chi^X$ inside $M(\CrG)$, where $\mathscr{X}$ runs over all irreducible unitary representations. Denote $L^2(\mathcal{C})$ for the Hilbert space spanned by all $\Lambda(\chi^X_l)$, where $\chi^X_l = \sum_{k} \Gr{\chi}{}{X}{k}{l}$. By the orthogonality relations, the $\Lambda(\chi^X_l)$ form an orthonormal basis when the $X$ vary over all irreducible corepresentations and the $l$ over over the right hyperobject support.
 
\begin{Lem} The natural representation of $\mathcal{C}$ on $L^2(\mathcal{C})$ is faithful.
\end{Lem} 

\begin{proof} Let $p$ be the projection of $L^2(\mathscr{G})$ onto $L^2(\mathcal{C})$. Then for $\chi = \chi^X$, $a\in \Gr{A}{k}{l}{m}{n}$ and $b\in \Gr{A}{r}{s}{p}{q}$, we compute that \[\langle \Lambda(b),\chi \Lambda(a)\rangle =  \langle \Lambda( \UnitC{r}{r}),\chi p\Lambda(a\sigma_{-i}(b^*))\rangle.\] By continuity, this holds for all elements $\chi \in \mathcal{C}$. As $\UnitC{r}{r} = \chi^U_r$ for $U$ the trivial character, the lemma follows.
\end{proof}

\begin{Lem} Assume $l$ and $l'$ are in the same hyperobject class. Then the representations of $\mathcal{C}$ on $L^2(\mathcal{C})_l$ and $L^2(\mathcal{C})_{l'}$ are equivalent.
\end{Lem} 
\begin{proof} The map $\Lambda(\chi^X_l)\rightarrow \Lambda(\chi^X_{l'})$ is a unitary intertwiner (since $\chi^X_l\neq 0$ if and only if $l$ in the right hyperobject support of $\mathscr{X}$ if and only if $l'$ in the right hyperobject support of $\mathscr{X}$).
\end{proof} 

Let now $\mathscr{C}$ be the partial fusion $^*$-algebra of $\Rep(\mathscr{G})$. It follows from the above that $\mathcal{C}$ is a C$^*$-completion of $\mathscr{C}$ which is completely determined by $\Rep(\mathscr{G})$.

%. In fact, since the representation on $L^2(\mathcal{C})_l$ is equivalent to the one on $L^2(\mathcal{C}_{l'})$ when $l$ and $l'$ belong to the same hyperobject, we see that $\mathcal{C}$ is determined completely by  itself.  

\begin{Theorem} A partial compact quantum group $\mathscr{G}$ is coamenable if and only if the representation \[\tilde{\epsilon}: \mathscr{C} \rightarrow B(l^2(I)), \quad \chi^X \mapsto D^X\] extends to a representation of $\mathcal{C}$.
\end{Theorem}

\begin{proof} If $\mathscr{G}$ is coamenable, then the existence of such an extension follows from the previous lemmas. 

Conversely, assume $\tilde{\epsilon}:\mathcal{C}\rightarrow B(l^2(I))$. We follow the strategy of [NT]. First of all, note that it is sufficient to show that for each $k$, the functional $\epsilon_k(a) = \langle e_k,\weps(a)e_k\rangle$ extends to a state on $\CrG$, since the associated GNS-representations imbed and span the representation $\weps$. 

By assumption, $\epsilon_k$ is well-defined on $\mathcal{C}\subseteq M(\CrG)$: it is the functional sending $\chi^X$ to $\Gr{d}{}{X}{k}{k}$. The same reasoning as before shows that this state can then be approximated by vector states of the form $\Lambda(\chi \UnitC{k}{k})$ with $\chi\in M(\CrG)$. It follows that we can find a state $\tilde{\nu}_k$ on $\UnitC{k}{k}\CrG\UnitC{k}{k}$ such that, with $\nu_k = \tilde{\nu}_k(\UnitC{k}{k}\cdot \UnitC{k}{k})$, $\nu_k(\chi) = \epsilon_k(\chi)$. 

Take now $X$ an irreducible corepresentation. Then $\|(\nu_k\otimes \id)(\Gr{X}{r}{s}{p}{q})\| = \delta_{k,r,s,p,q} \|(\nu_k\otimes \id)(\Gr{X}{k}{k}{k}{k})\| \leq \delta_{k,r,s,p,q}$, and furthermore $\Tr( (\nu_k\otimes \id)(\Gr{X}{k}{k}{k}{k})) = \nu_k(\Gr{\chi}{}{X}{k}{k}) = \epsilon_k(\chi^X) = \Gr{d}{}{X}{k}{k} = \dim \Gr{\Hsp}{}{X}{k}{k}$. Hence $ (\nu_k\otimes \id)(\Gr{X}{r}{s}{p}{q}) = \delta_{k,r,s,p,q} (\nu_k\otimes \id)(\Gr{X}{k}{k}{k}{k})$, and it follows that $\nu_k$ is an extension of $\epsilon_k$. 
\end{proof}  
 
 % Corollary: canonical partial quantum group constructed from a tensor C*-cat is always coamenable, cf. Neshveyev-Tuset section 2.7
 \begin{Cor} Let $\mathscr{G}_{\CatCC}$ be the canonical partial compact quantum group associated to a partial fusion tensor C$^*$-category $\CatCC$. Then $\mathscr{G}_{\CatCC}$ is coamenable.  
 \end{Cor}
 \begin{proof} The representation $D$ is now just the canonical `regular' representation of the fusion algebra of $\CatCC$ on its $L^2$-space. 
 \end{proof}
 