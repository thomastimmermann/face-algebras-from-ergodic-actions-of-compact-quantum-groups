\section{Coamenability of partial compact quantum groups}

Let $\mathscr{G}$ be a partial compact quantum group. 

%We will use throughout that if $\pi: P(\mathscr{G}) \rightarrow \End(V)$ is any $^*$-representation on a pre-Hilbert space (meaning $\langle v,\pi(a)w\rangle = \langle \pi(a^*)v,w\rangle$ for all $v,w\in V$), then $\pi$ extends to a $^*$-representation of $\CuG$ on the completion of $V$.

\begin{Def} A partial compact quantum group $\mathscr{G}$ will be called \emph{coamenable} if the natural projection map $\pi_{\red}:\CuG\twoheadrightarrow\CrG$ is an isomorphism.
\end{Def}

%Our aim is to give a characterisation of coamenability in terms of the representation theory of the fusion algebra, cf.~ \cite{Kye1} and references therein.

We will show that the usual characterisation of coamenability in terms of the counit representation holds. For this we need the following version of Fell's absorption principle. We will continue to use the notation introduced in the previous section.
% Definition \ref{DefTenProd} and the discussion preceding it.

\begin{Lem}\label{LemFell} Let $(\Hsp,\pi)$ be a non-degenerate $^*$-representation of $\CuG$. Then $\pi_{\red}\iboxtimes \pi$ factorizes over $\CrG$.
\end{Lem}

\begin{proof}  Similar calculations as in the proof of Lemma \ref{lemma:} We have a unitary \[U: \oplus_q (L^2(\mathscr{G})_q\otimes {}_q\Hsp ) \rightarrow \oplus_l ({}_l L^2(\mathscr{G})\otimes {}^l\Hsp),\]\[\Lambda(x)\otimes \xi \mapsto \Lambda(x_{(1)})\otimes \pi(x_{(2)})\xi,\quad x\in P(\mathscr{G}),\xi\in H,\] with inverse $\Lambda(x)\otimes \xi \mapsto \Lambda(x_{(1)})\otimes \pi(S(x_{(2)}))\xi$. %more info needed?
Denote $\pi_{\red,q}$ for the restriction of $\pi_{\red}$ to $L^2(\mathscr{G})_q$. Then it is clear that $U$ intertwines $\oplus_q (\pi_{\red,q}\otimes 1)$ with $\pi_{\red}\boxtimes \pi$. As each $\pi_{\red,q}$ descends to $\CrG$, this proves the lemma.
\end{proof} 

Recall now from \cite{DCT1} the $^*$-representation 
\begin{align} \label{eq:weps}
  \weps: P(\mathscr{G}) \rightarrow \End(\C^{(I)}), \quad x\in
  \Gr{A}{k}{l}{m}{n}\mapsto \epsilon(x) e_{kl},
\end{align}
where $\epsilon$ denotes the counit of $P(\mathscr{G})$ and $\C^{(I)}$ the direct sum of $I$ copies of $\C$ with the natural pre-Hilbert space structure. We will denote its extension to a $^*$-representation $\CuG\rightarrow B(l^2(I))$ by the same symbol.

\begin{Lem}\label{LemUnit} Let $(\Hsp,\pi)$ be a non-degenerate $^*$-representation of $\CuG$. Then $\weps\iboxtimes \pi$ is unitarily equivalent to $\pi$.
\end{Lem} 
\begin{proof} The map \[U: \Hsp \rightarrow \oplus_k \left({}_kl^2(I)\otimes {}^k\Hsp\right),\quad \xi \mapsto \sum_k e_k\otimes \pi(\lambda_k)\xi\] is a unitary intertwiner.
\end{proof} 

\begin{Prop} The partial compact quantum group $\mathscr{G}$ is coamenable if and only if $\weps$ descends to a $^*$-representation of $\CrG$.
\end{Prop} 
\begin{proof} $\Rightarrow$ is clear. For $\Leftarrow$, note that $\Delta^u$ descends to a (non-unital) $^*$-homomorphism $\CrG\rightarrow M(\CrG\otimes \CuG)$ by Fell's absorption principle. Composing with $\weps\otimes \id$ and using Lemma \ref{LemUnit} with respect to a faithful representation of $\CuG$, we see that there exists a splitting $\CrG\rightarrow \CuG$ of $\pi_{\red}$, hence $\pi_{\red}$ is an isomorphism.
\end{proof}
 
We now pass to the character algebra of $\mathscr{G}$.
 
 \begin{Def} Let $\mathscr{X}$ be a unitary rcfd representation of a partial compact quantum group $\mathscr{G}$ on an $I$-bigraded Hilbert space $\Hsp$. The \emph{partial characters} of $\mathscr{X}$ are defined as \[\Gr{\chi}{}{X}{k}{l} = (\id\otimes \Tr)(\Gr{X}{k}{l}{k}{l}) = \sum_i(\Gr{X}{k}{l}{k}{l})_{ii} \in \Gr{P(\mathscr{G})}{k}{l}{k}{l},\] where $\Tr$ is the non-normalized trace on $B(\GrDA{\Hsp}{k}{l})$, and where $(\Gr{X}{k}{l}{k}{l})_{ij}$ denote matrix coefficients with respect to some fixed orthonormal basis. The \emph{total character} of $\mathscr{X}$ is the multiplier $\chi^X  =\underset{k,l}{\sum} \,\Gr{\chi}{}{X}{k}{l}\in M(P(\mathscr{G}))$.
 \end{Def}
 
 For example, if $\mathscr{U}$ is the trivial representation of $\mathscr{G}$, then $\Gr{\chi}{}{U}{k}{l} = \delta_{k,l}\UnitC{k}{k}$, and $\chi^U = \sum_k \UnitC{k}{k}$.
 
 We aim to show that each character $\chi^X$ is in fact a well-defined element in $M(\CuG)$. Recall from \cite{DCT1} that $\smCirct$ denotes the tensor product of two representations of $\mathscr{G}$. 
 
 \begin{Lem} \label{LemCharMult} We have $\Gr{\chi}{}{X\smCirct Y}{k}{m}= \underset{l}{\sum}\, \Gr{\chi}{}{X}{k}{l}\Gr{\chi}{}{Y}{l}{m}$.
 \end{Lem}
 
 \begin{proof} 
 This follows from the definition of $\smCirct$, namely $\Gr{(X\Circt Y)}{k}{l}{m}{n} = \sum_{r,s} \left(\Gr{X}{k}{r}{m}{s}\right)_{12} \left(\Gr{Y}{r}{l}{s}{n}\right)_{13}$.
 \end{proof}

For the next lemma, recall from \cite{DCT1} that a unitary rcfd representation of $\mathscr{G}$ is said to have \emph{finite hypersupport} if it is a \emph{finite} direct sum of irreducible unitary rcfd representations.  The left hyperobject support of $\mathscr{X}$ is then the set of all $k$ such that $\GrDA{\Hsp}{k}{l}\neq \{0\}$ for some $l$. The right hyperobject support is defined similarly. Note that the left (resp. right) hyperobject support consists of  hyperobject classes, where $k$ and $l$ define the same hyperobject class if $\UnitC{k}{l}\neq0$. 


 \begin{Lem} \label{LemBoundDim} Let $\mathscr{X}$ be an rcfd unitary
   representation of $\mathscr{G}$ on an $I$-bigraded Hilbert space $\Hsp$ with finite hyperobject support, and write $d^X_{kl} = \dim(\Gru{H}{k}{l})$. Then the matrix $D^X = (d^X_{kl})_{k,l}$ defines a bounded operator on $l^2(I)$. 
  \end{Lem}
  \begin{proof} We may assume that $\mathscr{X}$ is irreducible, say
    with left hyperobject support $\alpha$ and right hyperobject support
    $\beta$. Then we can find (a priori not necessarily bounded)
    morphisms \[R: \C^{(I)}\rightarrow \underset{k,l,m}{\oplus} \GrDA{H}{k}{m}\otimes
    \GrDA{H}{m}{l},\qquad \bar{R}: \C^{(I)} \rightarrow \underset{k,l,m}{\oplus}
    \GrDA{\bar{H}}{k}{m}\otimes \GrDA{\bar{H}}{m}{l},\] establishing a
    duality between $\underset{k,l}{\oplus} \GrDA{H}{k}{l}$ and
    $\underset{k,l}{\oplus}\GrDA{\bar{H}}{k}{l}$ inside the category of
    rcfd pre-Hilbert spaces. However, as $R^*R$ is a scalar multiple
    of the projection onto $\C^{(I_{\alpha})}$ by irreducibility, and
    similarly for $\bar{R}$, it follows that $R$ and $\bar{R}$ can be
    completed to bounded operators between the respective Hilbert
    space completions. It then follows from \cite[Lemma A.3.2]{DCY1}
    that 
    \begin{align} \label{eq:dim-estimate}
  \sup_k (\sum_l (d_{kl}^X+d_{lk}^X)) < \infty.    
    \end{align}
 By the Schur test,
    this implies that $D^X$ is bounded.
\end{proof} 


   
\begin{Lem} The sum $\chi^X  =\sum \Gr{\chi}{}{X}{k}{l}$ converges strictly to an element in $M(\CuG)$. 
\end{Lem} 
 
\begin{proof} It is sufficient to prove that the matrix of operators $\left(\pi\left(\Gr{\chi}{}{X}{k}{l}\right)\right)_{k,l}$ defines a bounded matrix of operators in some faithful $^*$-representation $\pi$ of $C^{u}_{0}(\mathscr{G})$. But as we have that each $\|\pi\left(\left(\Gr{X}{k}{l}{k}{l}\right)_{ij}\right)\|\leq 1$ by unitarity of $\mathscr{X}$, we have $\| \pi\left(\Gr{\chi}{}{X}{k}{l}\right)\| \leq d^X_{kl}$.  Hence boundedness follows from Lemma \ref{LemBoundDim}.
\end{proof} 
 
Denote now $\mathcal{C}$ for the C$^*$-algebra generated by all $\chi^X$ inside $M(\CrG)$, where $\mathscr{X}$ runs over all irreducible unitary representations. For $\alpha$ a hyperobject and $l\in \alpha$, write  $L^2(\mathcal{C})_l$ for the Hilbert space spanned by all $\Lambda(\chi^X_l)$, where $\chi^X_l = \sum_{k} \Gr{\chi}{}{X}{k}{l}$ and where the $X$ run over all irreducibles with right hyperobject support equal to $\alpha$. By the orthogonality relations, \cite[Corollary 2.23]{DCT1}, the $\Lambda(\chi^X_l)$ form an orthonormal basis for $L^2(\mathcal{C})_l$. Moreover, $\mathcal{C}$ preserves each $L^2(\mathcal{C})_l$, hence defines a $^*$-representation of $\mathcal{C}$ which we will denote by $\pi_l$. 

\begin{Lem}\label{LemEq} Assume $l$ and $l'$ are in the same hyperobject class. Then the representations of $\mathcal{C}$ on $L^2(\mathcal{C})_l$ and $L^2(\mathcal{C})_{l'}$ are equivalent.
\end{Lem} 
\begin{proof} The map $\Lambda(\chi^X_l)\rightarrow \Lambda(\chi^X_{l'})$ is a unitary intertwiner (since $\chi^X_l\neq 0$ if and only if $l$ in the right hyperobject support of $\mathscr{X}$ if and only if $l'$ in the right hyperobject support of $\mathscr{X}$).
\end{proof} 

Denote now $L^2(\mathcal{C})\subseteq L^2(\mathscr{G})$ for the direct sum of the $L^2(\mathcal{C})_l$ where the $l$ run over a fixed section of the hyperobject set, and write $\pi_{\mathcal{C}}= \underset{l}{\oplus} \pi_l$. 
 
\begin{Prop} The representation $\pi_{\mathcal{C}}$ of $\mathcal{C}$ on $L^2(\mathcal{C})$ is faithful.
\end{Prop} 

\begin{proof} Let $p$ be the projection of $L^2(\mathscr{G})$ onto $L^2(\mathcal{C})$. Then for $\chi = \chi^X$, $a\in \Gr{P(\mathscr{G})}{k}{l}{m}{n}$ and $b\in \Gr{P(\mathscr{G})}{r}{s}{p}{q}$, we compute that \[\langle \Lambda(b),\chi \Lambda(a)\rangle =  \langle \Lambda( \UnitC{r}{r}),\chi p\Lambda(a\sigma_{-i}(b^*))\rangle.\] By continuity, this holds for all elements $\chi \in \mathcal{C}$. As $p\Lambda(a\sigma_{-i}(b^*)) \in L^2(\mathcal{C})_r$, and as $\UnitC{r}{r} = \chi^U_r$ for $\mathscr{U}$ the trivial character, the lemma follows by Lemma \ref{LemEq}.
\end{proof}

Let now $\mathscr{C}$ be the partial fusion $^*$-algebra of $\Rep(\mathscr{G})$, that is, the locally unital $^*$-algebra generated as a vector space over $\C$ by equivalence classes of irreducible unitary representations of $\mathscr{G}$, with multiplication determined by the fusion rules and the $^*$-structure obtained by taking the dual. It follows from the above that $\mathcal{C}$ is a C$^*$-completion of $\mathscr{C}$ which is completely determined by $\Rep(\mathscr{G})$.

On $\mathscr{C}$, the $*$-representation $\weps \colon P(\mathscr{G})
\to l^{2}(\C^{(I)})$ defined in
\eqref{eq:weps} is given by
\begin{align*}
  \weps(\chi^{X}) = \sum_{k,l} \epsilon(\Gr{\chi}{}{X}{k}{l})e_{kl} =
  \sum_{k,l} (\epsilon \otimes \Tr)(\Gr{X}{k}{l}{k}{l}) e_{kl}
   = \sum_{k,l} d^{X}_{kl} e_{kl} =D^{X}
\end{align*}
because $(\epsilon \otimes
\id)(\Gr{X}{k}{l}{k}{l})=\id_{\Gru{H}{k}{l}}$  for   $\mathscr{X}$ and
$\mathcal{H}$ as in Lemma \ref{LemBoundDim}.

%. In fact, since the representation on $L^2(\mathcal{C})_l$ is equivalent to the one on $L^2(\mathcal{C}_{l'})$ when $l$ and $l'$ belong to the same hyperobject, we see that $\mathcal{C}$ is determined completely by  itself.  


\begin{Def} We say $\mathscr{G}$ is \emph{dimension} coamenable %Fusion coamenable?
if the representation \[\tilde{\epsilon}: \mathscr{C} \rightarrow B(l^2(I)), \quad \chi^X \mapsto D^X\] extends to a representation of $\mathcal{C}$.
\end{Def}

It is immediately clear that a coamenable partial compact quantum group is dimension coamenable. However, as we will see later, the converse is not true. This is in marked contrast with the case of quantum groups, where amenability of the fusion algebra (with respect to its \emph{scalar} dimension function) \emph{does} determine coamenability of the associated compact quantum group, cf. \cite[Theorem 4.5]{Kye1}.
 
 % Corollary: canonical partial quantum group constructed from a tensor C*-cat is always dimension coamenable, cf. Neshveyev-Tuset section 2.7

Recall now from \cite[Definition 1.23 and Section 4.1]{DCT1} the notion of \emph{canonical partial compact quantum group} associated to a \emph{partial fusion tensor C$^*$-category $\CatCC$}.

\begin{Prop} Let $\mathscr{G}_{\CatCC}$ be the canonical partial compact quantum group associated to an $\mathcal{I}$-partial fusion tensor C$^*$-category $\CatCC$. Then $\mathscr{G}_{\CatCC}$ is dimension coamenable.  
\end{Prop}% Ref to add to canonical partial compact quantum group
\begin{proof} 
  By construction, $\mathscr{G}_{\CatCC}$ is an $I$-partial compact quantum group, where $\Rep(\mathscr{G}_{\CatCC})$ can be canonically
  identified with $\CatCC$, and the set $I$ labels a maximal set
  $\{U_{k}\}$ of mutually non-isomorphic irreducible objects of
  $\CatCC$.  Under the Tannaka duality, an object $X$ of $\CatCC$ corresponds to a corepresentation $\mathscr{X}$ of $\mathscr{G}_{\CatCC}$ on an $I$-bigraded Hilbert space $\Hsp$ with $\GrDA{\Hsp}{k}{l} = \Hom(U_k,X\otimes U_l)$. Note then that the hyperobject class of $k\in I$ is nothing but the \emph{left} hyperobject support of $U_k$. % More info! Also, small mistake in first paper, order of indices in canonical TK should be reversed.
  
  Fix now $l\in I$. Then $L^2(\mathcal{C})_l$ has an orthonormal basis consisting of all $\Lambda(\chi_{l}^{U_k})$ where the right hyperobject support of $U_k$ is the left hyperobject support of $U_l$. Then, by
  Lemma \ref{LemCharMult}, 
  \begin{align*}
 \chi^{X}\Lambda(\chi^{U_{n}}_{l}) = \chi^{X \Circt U_{n}}_{l} = \sum_{m}
 \dim(\Hom(U_{m},X\Circt U_{n})) \Lambda(\chi^{U_{m}}_{l}) = \sum_{m}
 D^{X}_{mn} \Lambda(\chi^{U_{m}}_{l}).
  \end{align*}
  
  Choose now a section $\mathcal{I} \rightarrow I, \alpha \mapsto l(\alpha)$. Then we can define a map $\theta:I\rightarrow I$ by sending $m$ to $l(\alpha)$ for $\alpha$ the right hyperobject support of $U_m$. Then \[l^2(I)\rightarrow L^2(\mathcal{C}),\quad \delta_m\mapsto \Lambda(\chi_{\theta(m)}^{U_m})\] is a unitary intertwining the regular representation of the fusion algebra of $\CatCC$ on $L^2(\mathcal{C})$ with the
representation $\tilde \epsilon$. Since the former extends to
$\mathcal{C}$, so does the latter.
 \end{proof}
 
 In the next proposition, we consider the partial compact quantum groups associated to weighted reciprocal random walks, see \cite[Section 5]{DCT1}. 
 
 \begin{Prop} Consider the partial compact quantum group $\mathscr{G}(\Gamma)$ associated to a weighted reciprocal random walk $\Gamma = (\Gamma,i,w,\sgn)$. Then $\mathscr{G}(\Gamma)$ is dimension coamenable if and only if $\|\Gamma\| \leq 2$. 
 \end{Prop}% Ref to add to first article
 \begin{proof} As $\Rep(\mathscr{G}(\Gamma))$ is by construction a Temperley-Lieb category,  the character algebra $\mathscr{G}(\Gamma)$ is now the polynomial ring on one generator $\C\lbrack \chi_{1/2}\rbrack$, with $\chi_{1/2}$ the character of the generating object $u_{1/2}$ of $\Rep(\mathscr{G}(\Gamma))$. It has regular C$^*$-completion $C(\lbrack -2,2\rbrack)$, with $\chi_{1/2}$ corresponding to the identity function. 
 
Let now $\Gamma^{(0)}$ be the vertex set of $\Gamma$. Then $\Gamma^{(0)}$ is identified with the index set $I$ of $\mathscr{G}(\Gamma)$, and $D^{u_{1/2}}_{kl}$ is the number of arrows from $k$ to $l$. Hence we obtain that there exists a map $C(\lbrack -2,2\rbrack) \rightarrow B(l^2(\Gamma^{(0)}))$ with $\chi_{1/2}\mapsto D^{u_{1/2}}$ if and only if $\|D^{u_{1/2}}\| \leq  2$ if and only if $\|\Gamma\|\leq 2$.   
 \end{proof} 
 
 In particular, the partial compact quantum groups coming from
 dynamical quantum $SU(2)$ are dimension coamenable. However, as we
 will see in the next section, they are not coamenable.

% Coamenability in terms of invariant measure not considered yet.
%%% Local Variables: 
%%% mode: latex
%%% TeX-master: "dynamical-SUq-file"
%%% End: 

